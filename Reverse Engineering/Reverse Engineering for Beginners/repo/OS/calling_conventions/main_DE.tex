\mysection{Methoden zur Argumentenübergabe (Aufrufkonventionen)}
\label{sec:callingconventions}

\subsection{cdecl}
\myindex{cdecl}
\label{cdecl}

Hierbei handelt es sich um die am weitesten verbreitete Methode um in \CCpp-Sprachen
Argumente an Funktionen zu übergeben.

Der \gls{caller} muss den Wert des \gls{stack pointer} (\ESP) auf den ursprünglichen Stand
bringen, nachdem \gls{callee}-Funktion beendet wurde.

\begin{lstlisting}[caption=cdecl,style=customasmx86]
push Argument3
push Argument2
push Argument1
call Funktion
add esp, 12 ; gibt ESP zurueck
\end{lstlisting}

\subsection{stdcall}
\label{sec:stdcall}
\myindex{stdcall}

\newcommand{\SIZEOFINT}{Die Größe einer Variablen vom Datentyp \Tint ist 4 in x86-Systemen und 8 in x64-Systemen}

Dies ist fast gleich zu der \emph{cdecl}-Aufrufkonvention, mit Ausnahme, dass die
\gls{callee} den Wert von \ESP auf den ursprünglichen Wert setzen muss. Dies geschieht
durch die \TT{RET x} anstatt \RET, wobei gilt \TT{x = Nummer des Arguments * sizeof(int)\footnote{\SIZEOFINT}}.

Der \gls{caller} passt den \gls{stack pointer} nicht an, es sind also keine \TT{add esp, x}-Anweisungen vorhanden.

\begin{lstlisting}[caption=stdcall,style=customasmx86]
push Argument3
push Argument2
push Argument1
call Funktion

Funktion:
;... tue etwas ...
ret 12
\end{lstlisting}

Diese Methode ist in Win32-Standard-Bibliotheken allgegenwärtig, fehlt jedoch in Win64 (siehe unten).

\par Beispielsweise kann die Funktion von \myref{src:passing_arguments_ex} genommen werden und durch
Hinzufügen des \TT{\_\_stdcall}-Modifizierers leicht verändert werden:

\begin{lstlisting}
int __stdcall f2 (int a, int b, int c)
{
	return a*b+c;
};
\end{lstlisting}

Das Kompilat ist fast das gleiche wie bei \myref{src:passing_arguments_ex_MSVC_cdecl},
jedoch wird \TT{RET 12} anstatt \TT{RET} genutzt. Der \ac{SP} wird im \gls{caller} nicht
aktualisiert.

Als Konsequenz daraus kann die Anzahl der Funktionsargumente einfach von der \TT{RETN n}-Anweisung
abgeleitet werden, indem $n$ durch 4 geteilt wird

%TODO Compilation of german version fails with style=customasm
%\lstinputlisting[caption=MSVC 2010,style=customasm]{OS/calling_conventions/stdcall_ex.asm}
\lstinputlisting[caption=MSVC 2010]{OS/calling_conventions/stdcall_ex.asm}

\subsubsection{Funktionen mit einer variablen Anzahl von Argumenten}

\printf-ähnliche Funktionen sind vielleicht die einzigen Funktionen in \CCpp mit einer Variablen
Anzahl von Argumenten, aber mit ihnen kann ein wichtiger Unterschied zwischen \emph{cdecl} und
\emph{stdcall} veranschaulicht werden.
Beginnen wir mit der Vorstellung, dass der Compiler die Anzahl der Argumente für jeden Aufruf
von \printf kennt.

Die aufgerufene und bereits kompilierte \printf-Funktion befindet sich in der Datei MSVCRT.DLL
(wenn über Windows geredet wird) und hat aber keinerlei Informationen darüber wie viele Argumente
übergeben wurden; dies kann jedoch über den Formatstring herausgefunden werden.

%TODO: The following sentence is much too long.
Wenn \printf eine \emph{stdcall}-Funktion wäre und den \gls{stack pointer} durch Zählen der Zahl
der Argumente im Formatstring auf den ursprünglichen Wert setzen würde, kann eine gefährliche
Situation entstehen, da ein Schreibfehler des Programmierers zu einem plötzliche Programmabsturz
führen könnte.
Aus diesem Grund ist für diese Art von Funktionen \emph{stdcall} ungeeignet und \emph{cdecl} ist
zu bevorzugen.

\subsection{fastcall}
\label{fastcall}
\myindex{fastcall}

Dies ist der allgemeine Name für eine Methode, in der einige Argumente mittels Registern und
der Rest über den Stack übergeben werden. Für ältere CPUs ist \emph{fastcall} schneller als
\emph{cdecl} und \emph{stdcall} wegen der geringeren Stack-Nutzung.
Auf neueren \ac{CPU}s wird dieser Ansatz vermutlich keine signifikante Geschwindigkeitserhöhung
nach sich ziehen.

\emph{fastcall} ist nicht standardisiert, so das verschiedene Compiler eine unterschiedliche
Umsetzung machen können.
Dies kann zu Problemen führen, wenn zwei DLLs genutzt werden, von denen eine die andere nutzt
und durch das Nutzen verschiedener Compiler unterschiedliche \emph{fastcall}-Aufrufkonventionen
genutzt werden.

Sowohl MSVC als auch GCC übergeben das erste und zweite Argument über \ECX und \EDX und den Rest
der Arguments mittels des Stacks.

Der \gls{stack pointer} muss vom \gls{callee} auf den ursprünglichen Wert gesetzt werden,
wie in \emph{stdcall} auch.

\begin{lstlisting}[caption=fastcall]
push Argument3
mov edx, Argument2
mov ecx, Argument1
call Funktion

Funktion:
.. tue etwas ..
ret 4
\end{lstlisting}

Beispielsweise kann die Funktion von \myref{src:passing_arguments_ex} genommen werden und durch
Hinzufügen des \TT{\_\_fastcall}-Modifizierers leicht verändert werden:

\begin{lstlisting}
int __fastcall f3 (int a, int b, int c)
{
	return a*b+c;
};
\end{lstlisting}

Nachfolgend das Kompilat:

%TODO Compilation of german version fails with style=customasm
%\lstinputlisting[caption=\Optimizing MSVC 2010 /Ob0,style=customasm]{OS/calling_conventions/fastcall_ex.asm}
\lstinputlisting[caption=\Optimizing MSVC 2010 /Ob0]{OS/calling_conventions/fastcall_ex.asm}

Es ist erkennbar, dass der \gls{callee} den \ac{SP} mit der \TT{RETN}-Anweisung
und einem Operanden auf den ursprünglichen Wert setzt.

Dies bedeutet, dass die Anzahl der Argumente ebenfalls einfach abgeleitet werden kann.

\subsubsection{GCC regparm}

\newcommand{\URLREGPARMM}{\url{http://www.ohse.de/uwe/articles/gcc-attributes.html\#func-regparm}}

Dies ist in gewisser Weise die Weiterentwicklung von \emph{fastcall}\footnote{\URLREGPARMM}.
Mit der \TT{-mregparm}-Option ist es möglich festzulegen, wieviele Argumente per Register
übergeben werden (maximal 3).
Aus diesem Grund werden die Register \EAX, \EDX und \ECX genutzt.

Natürlich werden, wenn die Anzahl der Argumente kleiner als drei ist, nicht alle drei
Register genutzt.

Der \gls{caller} setzt den \gls{stack pointer} auf den initialen Zustand.

Als Beispiel siehe (\myref{regparm}).

\subsubsection{Watcom/OpenWatcom}
\myindex{OpenWatcom}

Hier erfolgt der Aufruf mit der \q{Register-Aufruf-Konvention}.
Die ersten vier Argumente werden in den Registern \EAX, \EDX, \EBX und \ECX übergeben,
der Rest auf dem Stack.

Diese Funktionen haben einen Unterstrich an den Funktionsnamen angehängt, um sie von
den anderen Aufrufkonventionen unterscheiden zu können.

\subsection{thiscall}
\myindex{thiscall}

Hier wird der \ITthis-Zeiger des Objekts an die Methode in \Cpp übergeben.

In MSVC wird \ITthis üblicherweise im \ECX-Register übergeben.

In GCC wird der \ITthis-Zeiger im ersten Argument der Methode übergeben.
Es ist sehr offensichtlich dass intern alle Methoden ein zusätzliches Argument haben.

Als Beispiel siehe (\myref{thiscall}).

\subsection{x86-64}
\myindex{x86-64}

\subsubsection{Windows x64}
\label{sec:callingconventions_win64}

Die Art Argumente zu Übergeben ähnelt in Win64 in gewisser Weise \TT{fastcall}.
Die ersten vier Argumente werden in den Registern \RCX, \RDX, \Reg{8} und \Reg{9}
übergeben und der Rest auf dem Stack.
Der \gls{caller} muss Platz für 32 Byte, also 4 64-Bit-Werte bereitstellen, so
dass der \gls{callee} dort die ersten vier Argumente speichern kann.
Kurze Funktionen können die Werte der Argumente direkt aus den Registern lesen,
während längere Funktionen diese für späteren Gebrauch zwischenspeichern sollten.

Der \gls{caller} muss den \gls{stack pointer} auf den vorherigen Zustand zurücksetzen.

Diese Aufrufkonvention wird auch in den Windows x86-64-System-DLLs genutzt
(anstatt \emph{stdcall} in Win32).

Beispiel:

\lstinputlisting[style=customc]{OS/calling_conventions/x64.c}

%TODO Compilation of german version fails with style=customasm
%\lstinputlisting[caption=MSVC 2012 /0b,style=customasm]{OS/calling_conventions/x64_MSVC_Ob.asm}
\lstinputlisting[caption=MSVC 2012 /0b]{OS/calling_conventions/x64_MSVC_Ob.asm}

\myindex{Scratch space}

Es ist hier klar erkennbar, wie sieben Argumente übergeben werden: vier in den
Registern und die drei restlichen auf dem Stack.

Der Code des f1()-Funktionsprolog sichert die Argumente in dem \q{Scratch Space},
einer Stelle auf dem Stack, die genau für diese Zwecke existiert.

Dies ist so realisiert, weil der Compiler nicht sicher sein kann, dass genug Register
ohne diese vier genutzt werden können und sie sonst durch die Argumente verändert
werden können, bis die Funktion beendet wird.

Die Bereitstellung des \q{Scratch Space} auf dem Stack ist Aufgabe der aufrufenden Funktion.

%TODO Compilation of german version fails with style=customasm
%\lstinputlisting[caption=\Optimizing MSVC 2012 /0b,style=customasm]{OS/calling_conventions/x64_MSVC_Ox_Ob.asm}
\lstinputlisting[caption=\Optimizing MSVC 2012 /0b]{OS/calling_conventions/x64_MSVC_Ox_Ob.asm}

Wenn das Beispiel ohne Optimierung compiliert wird, ist das Ergebnis fast das gleiche,
lediglich der \q{Scratch Space} ist unnötig und wird nicht genutzt.

\myindex{x86!\Instructions!LEA}
\label{using_MOV_and_pack_of_LEA_to_load_values}

Beachtenswert ist auch, wie MSVC 2012 das Laden von einfachen Werten in Register
durch das Nutzen von \LEA (\myref{sec:LEA}) optimiert.
\INS{MOV} wäre hier ein Byte länger (5 anstatt 4).

Ein weiteres Beispiel so eines Sachverhalts ist: \myref{TaskMgr_LEA}.

\myparagraph{Windows x64: Übergeben von \ITthis (\CCpp)}

Der \ITthis-Zeiger wird in \RCX übergeben, das erste Argument der Methode ist \RDX, usw.
Ein Beispiel ist hier zu sehen: \myref{simple_CPP_MSVC_x64}.

\subsubsection{Linux x64}

Die Art wie Linux für x86-64 Argumente übergibt ist fast die Gleiche wie in Windows,
jedoch werden sechs anstatt vier Register genutzt (\RDI, \RSI, \RDX, \RCX, \Reg{8}, \Reg{9})
und es gibt keinen \q{Scratch Space}, auch wenn der \gls{callee} die Registerwerte auf
dem Stack speichern kann wenn er dies will oder muss.

%TODO Compilation of german version fails with style=customasm
%\lstinputlisting[caption=\Optimizing GCC 4.7.3,style=customasm]{OS/calling_conventions/x64_linux_O3.s}
\lstinputlisting[caption=\Optimizing GCC 4.7.3]{OS/calling_conventions/x64_linux_O3.s}

\myindex{AMD}
Zur Beachtung: die Werte werden hier in die 32-Bit-Teile der Register (z.B.EAX) geschrieben,
aber nicht in die kompletten 64-Bit-Register (RAX).
Dies wird gemacht, weil jeder Schreibzugang auf die niederwertigen 32-Bit-Teile eines
Registers automatisch den höherwertigen Teil zurücksetzt.

Vermutlich wurde dies bei AMD so eingeführt um die Portierung des Codes zu x86-64 zu vereinfachen.

\subsection{Rückgabewerte von \Tfloat- und \Tdouble-Typen}
\myindex{float}
\myindex{double}

In allen Konventionen außer in Win64, werden di Werte vom Typ \Tfloat oder \Tdouble
in dem FPU-Register \ST{0} zurückgegeben.

In Win64 werden die Werte vom Typ \Tfloat oder \Tdouble in den niederwertigen 32 oder
64 Bit des \XMM{0}-Registers zurückgegeben.

\subsection{Verändern von Argumenten}

Manchmal fragen \CCpp{}-Programmierer (obwohl nicht auf diese \ac{PL} beschränkt),
was passieren kann wenn die Funktionsargumente verändert werden.

Die Antwort ist einfach: die Argumente sind auf dem Stack gespeichert und hier
werden auch die Veränderungen vorgenommen.

Die aufzurufende Funktion wird diese nicht nach dem Verlassen des \gls{callee}
nutzen (der Autor dieser Linien hat so einen Fall in der Praxis noch nie gesehen).

\lstinputlisting[style=customc]{OS/calling_conventions/change_arguments.c}

%TODO Compilation of german version fails with style=customasm
%\lstinputlisting[caption=MSVC 2012,style=customasm]{OS/calling_conventions/change_arguments.asm}
\lstinputlisting[caption=MSVC 2012,style=customasmx86]{OS/calling_conventions/change_arguments.asm}

% TODO (OllyDbg) пример как в стеке меняется $a$

Also: ja, die Argumente können einfach modifiziert werden.
Natürlich, wenn diese keine \emph{Referenz} in  \Cpp{} (\myref{cpp_references}) ist,
und nicht die Daten verändert werden auf die der Zeiger zeigt, wird der Effekt nicht
außerhalb der aktuellen Funktion sichtbar sein.

Theoretisch kann der \gls{caller} die modifizierten Argumente auf irgendeine Weise
nutzen nachdem der \gls{callee} beendet wurde.
Vielleicht wenn dies direkt in Assembler programmiert ist.

Beispielsweise wird Code wie folgt von gewöhnlichen \CCpp-Compilern erzeugt:

\begin{lstlisting}[style=customasmx86]
	push	456	; wird gleich b sein
	push	123	; wird gleich a sein
	call	f	; f() modifiziert das erste Argument
	add	esp, 2*4
\end{lstlisting}

Der Code kann wie folgt neu geschrieben werden:

\begin{lstlisting}[style=customasmx86]
	push	456	; wird gleich b sein
	push	123	; wird gleich a sein
	call	f	; f() modifiziert das erste Argument
	pop	eax
	add	esp, 4
	; EAX=Erstes Argument von f() modifiziert in f()
\end{lstlisting}

Es ist scher vorzustellen warum jemand dies tun sollte, aber in der Praxis ist es möglich.
Nichtsdestotrotz bietet der \CCpp-Standard keinen Möglichkeit dies zu tun.

% subsections
\mysection{x86}

\subsection{Terminologie}

Geläufig für 16-Bit (8086/80286), 32-Bit (80386, etc.), 64-Bit.

\myindex{IEEE 754}
\myindex{MS-DOS}
\begin{description}
	\item[Byte] 8-Bit.
		Die DB Assembler-Direktive wird zum Definieren von Variablen und Arrays genutzt.
		Bytes werden in dem 8-Bit-Teil der folgenden Register übergeben:
		\TT{AL/BL/CL/DL/AH/BH/CH/DH/SIL/DIL/R*L}.
	\item[Wort] 16-Bit.
		DW Assembler-Direktive \dittoclosing.
		Bytes werden in dem 16-Bit-Teil der folgenden Register übergeben:
			\TT{AX/BX/CX/DX/SI/DI/R*W}.
	\item[Doppelwort] (\q{dword}) 32-Bit.
		DD Assembler-Direktive \dittoclosing.
		Doppelwörter werden in Registern (x86) oder dem 32-Bit-Teil der Register (x64) übergeben.
		In 16-Bit-Code werden Doppelwörter in 16-Bit-Registerpaaren übergeben.
	\item[zwei Doppelwörter] (\q{qword}) 64-Bit.
		DQ Assembler-Direktive \dittoclosing.
		In 32-Bit-Umgebungen werden diese in 32-Bit-Registerpaaren übergeben.
	\item[tbyte] (10 Byte) 80-Bit oder 10 Bytes (für IEEE 754 FPU Register).
	\item[paragraph] (16 Byte) --- Bezeichnung war in MS-DOS Umgebungen gebräuchlich.
\end{description}

\myindex{Windows!API}

Datentypen der selben Breite (BYTE, WORD, DWORD) entsprechen auch denen in der Windows \ac{API}.

% TODO German Translation (DSiekmeier)
%\input{appendix/x86/registers} % subsection
%\input{appendix/x86/instructions} % subsection
\subsection{npad}
\label{sec:npad}

\RU{Это макрос в ассемблере, для выравнивания некоторой метки по некоторой границе.}
\EN{It is an assembly language macro for aligning labels on a specific boundary.}
\DE{Dies ist ein Assembler-Makro um Labels an bestimmten Grenzen auszurichten.}
\FR{C'est une macro du langage d'assemblage pour aligner les labels sur une limite
spécifique.}

\RU{Это нужно для тех \emph{нагруженных} меток, куда чаще всего передается управление, например, 
начало тела цикла. 
Для того чтобы процессор мог эффективнее вытягивать данные или код из памяти, через шину с памятью, 
кэширование, итд.}
\EN{That's often needed for the busy labels to where the control flow is often passed, e.g., loop body starts.
So the CPU can load the data or code from the memory effectively, through the memory bus, cache lines, etc.}
\DE{Dies ist oft nützlich Labels, die oft Ziel einer Kotrollstruktur sind, wie Schleifenköpfe.
Somit kann die CPU Daten oder Code sehr effizient vom Speicher durch den Bus, den Cache, usw. laden.}
\FR{C'est souvent nécessaire pour des labels très utilisés, comme par exemple le
début d'un corps de boucle. Ainsi, le CPU peut charger les données ou le code depuis
la mémoire efficacement, à travers le bus mémoire, les caches, etc.}

\RU{Взято из}\EN{Taken from}\DE{Entnommen von}\FR{Pris de} \TT{listing.inc} (MSVC):

\myindex{x86!\Instructions!NOP}
\RU{Это, кстати, любопытный пример различных вариантов \NOP{}-ов. 
Все эти инструкции не дают никакого эффекта, но отличаются разной длиной.}
\EN{By the way, it is a curious example of the different \NOP variations.
All these instructions have no effects whatsoever, but have a different size.}
\DE{Dies ist übrigens ein Beispiel für die unterschiedlichen \NOP-Variationen.
Keine dieser Anweisungen hat eine Auswirkung, aber alle haben eine unterschiedliche Größe.}
\FR{À propos, c'est un exemple curieux des différentes variations de \NOP. Toutes
ces instructions n'ont pas d'effet, mais ont une taille différente.}

\RU{Цель в том, чтобы была только одна инструкция, а не набор NOP-ов, 
считается что так лучше для производительности CPU.}
\EN{Having a single idle instruction instead of couple of NOP-s,
is accepted to be better for CPU performance.}
\DE{Eine einzelne Idle-Anweisung anstatt mehrerer NOPs hat positive Auswirkungen
auf die CPU-Performance.}
\FR{Avoir une seule instruction sans effet au lieu de plusieurs est accepté comme
étant meilleur pour la performance du CPU.}

\begin{lstlisting}[style=customasmx86]
;; LISTING.INC
;;
;; This file contains assembler macros and is included by the files created
;; with the -FA compiler switch to be assembled by MASM (Microsoft Macro
;; Assembler).
;;
;; Copyright (c) 1993-2003, Microsoft Corporation. All rights reserved.

;; non destructive nops
npad macro size
if size eq 1
  nop
else
 if size eq 2
   mov edi, edi
 else
  if size eq 3
    ; lea ecx, [ecx+00]
    DB 8DH, 49H, 00H
  else
   if size eq 4
     ; lea esp, [esp+00]
     DB 8DH, 64H, 24H, 00H
   else
    if size eq 5
      add eax, DWORD PTR 0
    else
     if size eq 6
       ; lea ebx, [ebx+00000000]
       DB 8DH, 9BH, 00H, 00H, 00H, 00H
     else
      if size eq 7
	; lea esp, [esp+00000000]
	DB 8DH, 0A4H, 24H, 00H, 00H, 00H, 00H 
      else
       if size eq 8
        ; jmp .+8; .npad 6
	DB 0EBH, 06H, 8DH, 9BH, 00H, 00H, 00H, 00H
       else
        if size eq 9
         ; jmp .+9; .npad 7
         DB 0EBH, 07H, 8DH, 0A4H, 24H, 00H, 00H, 00H, 00H
        else
         if size eq 10
          ; jmp .+A; .npad 7; .npad 1
          DB 0EBH, 08H, 8DH, 0A4H, 24H, 00H, 00H, 00H, 00H, 90H
         else
          if size eq 11
           ; jmp .+B; .npad 7; .npad 2
           DB 0EBH, 09H, 8DH, 0A4H, 24H, 00H, 00H, 00H, 00H, 8BH, 0FFH
          else
           if size eq 12
            ; jmp .+C; .npad 7; .npad 3
            DB 0EBH, 0AH, 8DH, 0A4H, 24H, 00H, 00H, 00H, 00H, 8DH, 49H, 00H
           else
            if size eq 13
             ; jmp .+D; .npad 7; .npad 4
             DB 0EBH, 0BH, 8DH, 0A4H, 24H, 00H, 00H, 00H, 00H, 8DH, 64H, 24H, 00H
            else
             if size eq 14
              ; jmp .+E; .npad 7; .npad 5
              DB 0EBH, 0CH, 8DH, 0A4H, 24H, 00H, 00H, 00H, 00H, 05H, 00H, 00H, 00H, 00H
             else
              if size eq 15
               ; jmp .+F; .npad 7; .npad 6
               DB 0EBH, 0DH, 8DH, 0A4H, 24H, 00H, 00H, 00H, 00H, 8DH, 9BH, 00H, 00H, 00H, 00H
              else
	       %out error: unsupported npad size
               .err
              endif
             endif
            endif
           endif
          endif
         endif
        endif
       endif
      endif
     endif
    endif
   endif
  endif
 endif
endif
endm
\end{lstlisting}
 % subsection

%\input{OS/calling_conventions/ctypes_DE}
%\input{OS/calling_conventions/DLL_cdecl_DE}

