\subsection{\CapitalPICcode}
\myindex{\PICcode}
\myindex{Linux}
\label{sec:PIC}

While analyzing Linux shared (.so) libraries, one may frequently spot this code pattern:

\lstinputlisting[style=customasmx86,caption=libc-2.17.so x86]{OS/tmp1.asm}

All pointers to strings are corrected by some constants and the value in \EBX,
which is calculated at the beginning of each function.

This is the so-called \ac{PIC}, it is intended to be executable if placed at any random point of memory, that is why it cannot contain any absolute memory addresses.

\ac{PIC} was crucial in early computer systems and is still crucial today in embedded systems without 
virtual memory support (where all processes are placed in a single continuous memory block).

It is also still used in *NIX systems for shared libraries, since they 
are shared across many processes while loaded in memory only once.
But all these processes can 
map the same shared library at different addresses, so that is why
a shared library has to work correctly without using any absolute addresses.

Let's do a simple experiment:

\begin{lstlisting}[style=customc]
#include <stdio.h>

int global_variable=123;

int f1(int var)
{
    int rt=global_variable+var;
    printf ("returning %d\n", rt);
    return rt;
};
\end{lstlisting}

Let's compile it in GCC 4.7.3 and see the resulting .so file in \IDA:

\begin{lstlisting}
gcc -fPIC -shared -O3 -o 1.so 1.c
\end{lstlisting}

\lstinputlisting[style=customasmx86,caption=GCC 4.7.3]{OS/tmp2.asm}

\newcommand{\retstring}{\emph{<<returning \%d\textbackslash{}n>>}}
\newcommand{\globvar}{\emph{global\_variable}}

That's it: the pointers to \retstring{} and \globvar{} are to be corrected at each function execution.

\par The \TT{\_\_x86\_get\_pc\_thunk\_bx()} function returns in \EBX the address of the point after a call to itself (\TT{0x57C} here).

That's a simple way to get the value of the program counter (\EIP) at some point.
The \TT{0x1A84} constant is related to the difference between this function's start and the so-called
\emph{Global Offset Table Procedure Linkage Table} (GOT PLT), the section right after the \emph{Global Offset Table} (GOT), where the pointer to \globvar{} is.
\IDA shows these offsets in their processed form to make them easier to understand, but in fact the code is:

\begin{lstlisting}[style=customasmx86]
.text:00000577                 call    __x86_get_pc_thunk_bx
.text:0000057C                 add     ebx, 1A84h
.text:00000582                 mov     [esp+1Ch+var_4], esi
.text:00000586                 mov     eax, [ebx-0Ch]
.text:0000058C                 mov     esi, [eax]
.text:0000058E                 lea     eax, [ebx-1A30h]
\end{lstlisting}

Here \EBX points to the \TT{GOT PLT} section and to calculate a pointer to \globvar{} (which is stored in 
the \TT{GOT}), \TT{0xC} must be subtracted.

To calculate pointer to the \retstring{} string, \TT{0x1A30} must be subtracted.

\myindex{x86-64}
\myindex{x86!\Registers!RIP}

By the way, that is the reason why the AMD64 instruction set supports RIP\footnote{program counter in AMD64}-relative addressing --- to simplify PIC-code.

Let's compile the same C code using the same GCC version, but for x64.

\myindex{objdump}
\IDA would simplify the resulting code but would suppress the RIP-relative addressing details, 
so we are going to use \emph{objdump} instead of IDA to see everything:

\lstinputlisting[style=customasmx86]{OS/tmp3.asm}

\TT{0x2008b9} is the difference between the address of the instruction at \TT{0x720} and \globvar{}, and 
\TT{0x20} is the difference between the address of the instruction at 
\TT{0x72A} and the \retstring{} string.

As you might see, the need to recalculate addresses frequently makes execution slower 
(it is better in x64, though).

So it is probably better to link statically if you care about performance \InSqBrackets{see: \AgnerFogCPP}.

\subsubsection{Windows}
\myindex{Windows!Win32}

The PIC mechanism is not used in Windows DLLs. If the Windows loader needs to load DLL 
on another base address, it \q{patches} the DLL in memory (at the \emph{FIXUP} places) in order to correct 
all addresses.

This implies that several Windows processes cannot share an once loaded DLL 
at different addresses in different process' memory 
blocks --- since each instance that's loaded in memory is \emph{fixed} to work only at these addresses..

