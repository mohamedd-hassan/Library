\myindex{\CStandardLibrary!memcpy()}
\myindex{x86!\Instructions!MOVSB}
\myindex{x86!\Instructions!MOVSW}
\myindex{x86!\Instructions!MOVSD}
\myindex{x86!\Instructions!MOVSQ}
\item[MOVSB/MOVSW/MOVSD/MOVSQ]
copier l'octet/
le mot 16-bit/
le mot 32-bit/
le mot 64-bit
depuis l'adresse se trouvant dans SI/ESI/RSI vers celle se trouvant dans DI/EDI/RDI.

\label{REP_MOVSx}
\myindex{x86!\Prefixes!REP}
Avec le préfixe REP, elle est répétée en boucle, le compteur étant stocker dans le
registre CX/ECX/RCX:
ça fonctionne comme memcpy() en C.
Si la taille du bloc est connue pendant la compilation, memcpy() est souvent mise
en ligne dans un petit morceau de code en utilisant REP MOVSx, parfois même avec
plusieurs instructions.

L'équivalent de memcpy(EDI, ESI, 15) est:

\lstinputlisting[style=customasmx86]{appendix/x86/instructions/MOVSB_ex1_FR.asm}

(Apparemment, c'est plus rapide que de copier 15 octets avec un seul REP MOVSB).
