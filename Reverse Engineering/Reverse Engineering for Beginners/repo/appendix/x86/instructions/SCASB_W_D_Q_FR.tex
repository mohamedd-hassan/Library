\myindex{\CStandardLibrary!strlen()}
\myindex{\CStandardLibrary!memchr()}
\myindex{x86!\Instructions!SCASB}
\myindex{x86!\Instructions!SCASW}
\myindex{x86!\Instructions!SCASD}
\myindex{x86!\Instructions!SCASQ}
\item[SCASB/SCASW/SCASD/SCASQ] (M) compare un octet/
un mot 16-bit/
un mot 32-bit/
un mot 64-bit stocké dans AX/EAX/RAX
avec une variable dont l'adresse est dans DI/EDI/RDI.
Met les flags comme le fait \CMP.

\label{REPNE_SCASx}
\myindex{x86!\Prefixes!REPNE}
Cette instruction est souvent utilisée avec le préfixe REPNE: continue de scanner
le buffer jusqu'à ce qu'une valeur particulière stockée dans AX/EAX/RAX soit trouvée.
D'où le \q{NE} dans REPNE: continue de scanner tant que les valeurs comparées ne
sont pas égales et s'arrête lorsqu'elles le sont.

Elle est souvent utilisée comme la fonction C standard strlen(), pour déterminer
la longueur d'une chaîne \ac{ASCIIZ}:

Exemple:

\lstinputlisting[style=customasmx86]{appendix/x86/instructions/SCASB_ex1_FR.asm}

Si nous utilisons une valeur différente dans AX/EAX/RAX, la fonction se comporte
comme la fonction C standard memchr(), i.e., elle trouve un octet spécifique.

