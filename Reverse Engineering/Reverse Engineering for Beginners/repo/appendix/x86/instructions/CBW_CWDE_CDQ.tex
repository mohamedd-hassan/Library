\myindex{x86!\Instructions!CBW}
\myindex{x86!\Instructions!CWD}
\myindex{x86!\Instructions!CDQ}
\myindex{x86!\Instructions!CWDE}
\myindex{x86!\Instructions!CDQE}
\label{ins:CBW_CWD_etc}
\item[CBW/CWD/CWDE/CDQ/CDQE]

\RU{Расширить значение учитывая его знак}\EN{Sign-extend value}:

\begin{description}
\item[CBW] \RU{конвертировать байт в AL в слово в AX}\EN{convert byte in AL to word in AX}
\item[CWD] \RU{конвертировать слово в AX в двойное слово в DX:AX}\EN{convert word in AX to doubleword in DX:AX} 
\item[CWDE] \RU{конвертировать слово в AX в двойное слово в EAX}\EN{convert word in AX to doubleword in EAX} 
\item[CDQ] \RU{конвертировать двойное слово в EAX в четверное слово в EDX:EAX}\EN{convert doubleword in EAX to quadword in EDX:EAX}
\item[CDQE] (x64) \RU{конвертировать двойное слово в EAX в четверное слово в RAX}\EN{convert doubleword in EAX to quadword in RAX}
\end{description}

\RU{Эти инструкции учитывают знак значения, расширяя его в старшую часть выходного
значения. См. также:}
\EN{These instructions consider the value's sign, extending it to high part of the newly constructed 
value. See also:} \myref{subsec:sign_extending_32_to_64}.

\newcommand{\StephenMorse}{[Stephen P. Morse, \emph{The 8086 Primer}, (1980)]\footnote{\AlsoAvailableAs \url{https://archive.org/details/The8086Primer}}}

\EN{Interestingly to know these instructions was initially named as \TT{SEX} (\emph{Sign EXtend}), 
as Stephen P. Morse (one of Intel 8086 CPU designers) wrote in \StephenMorse:}
\RU{Интересно узнать, что эти инструкции назывались \TT{SEX} (\emph{Sign EXtend}),
как Stephen P. Morse (один из создателей Intel 8086 CPU) пишет в \StephenMorse:}

\begin{framed}
\begin{quotation}
The process of stretching numbers by extending the sign bit is called sign extension. 
The 8086 provides instructions (Fig. 3.29) to facilitate the task of sign extension. 
These instructions were initially named SEX (sign extend) but were later renamed to the more 
conservative CBW (convert byte to word) and CWD (convert word to double word).
\end{quotation}
\end{framed}
