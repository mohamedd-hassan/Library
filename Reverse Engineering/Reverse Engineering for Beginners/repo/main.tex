\documentclass[a4paper,oneside]{book}

% http://www.tex.ac.uk/FAQ-noroom.html
\usepackage{etex}

\usepackage[table,usenames,dvipsnames]{xcolor}

\usepackage{fontspec}
% fonts
%\setmonofont{DroidSansMono}
%\setmainfont[Ligatures=TeX]{PT Sans}
%\setmainfont{DroidSans}
\setmainfont{DejaVu Sans}
\setmonofont{DejaVu Sans Mono}
\usepackage{polyglossia}
\usepackage{ucharclasses}
\usepackage{csquotes}

\ifdefined\ENGLISH
%\wlog{main ENGLISH defined OK}
\setmainlanguage{english}
\setotherlanguage{russian}
\fi

\ifdefined\RUSSIAN
\setmainlanguage{russian}
%\newfontfamily\cyrillicfont{LiberationSans}
%\newfontfamily\cyrillicfonttt{LiberationMono}
%\newfontfamily\cyrillicfontsf{lmsans10-regular.otf}
\setotherlanguage{english}
\fi

\ifdefined\GERMAN
%\wlog{main GERMAN defined OK}
\setmainlanguage{german}
\setotherlanguage{english}
\fi

\ifdefined\SPANISH
\setmainlanguage{spanish}
\setotherlanguage{english}
\fi

\ifdefined\ITALIAN
\setmainlanguage{italian}
\setotherlanguage{english}
\fi

\ifdefined\BRAZILIAN
\setmainlanguage{portuges}
\setotherlanguage{english}
\fi

\ifdefined\POLISH
\setmainlanguage{polish}
\setotherlanguage{english}
\fi

\ifdefined\DUTCH
\setmainlanguage{dutch}
\setotherlanguage{english}
\fi

\ifdefined\TURKISH
\setmainlanguage{turkish}
\setotherlanguage{english}
\fi

\ifdefined\THAI
\setmainlanguage{thai}
%\usepackage[thai]{babel}
%\usepackage{fonts-tlwg}
\setmainfont[Script=Thai]{TH SarabunPSK}
\newfontfamily{\thaifont}[Script=Thai]{TH SarabunPSK}
\let\thaifonttt\ttfamily
\setotherlanguage{english}
\fi

\ifdefined\FRENCH
%\setmainlanguage[autospacing=false]{french}
\setmainlanguage{french}
\setotherlanguage{english}
\fi

\ifdefined\JAPANESE
\usepackage{xeCJK}
\xeCJKallowbreakbetweenpuncts
\defaultfontfeatures{Ligatures=TeX,Scale=MatchLowercase}
\setCJKmainfont{IPAGothic}
\setCJKsansfont{IPAGothic}
\setCJKmonofont{IPAGothic}
\DeclareQuoteStyle{japanese}
  {「}
  {」}
  {『}
  {』}
\setquotestyle{japanese}
\setmainlanguage{japanese}
\setotherlanguage{english}
\fi

\usepackage{microtype}
\usepackage{fancyhdr}
\usepackage{listings}
%\usepackage{ulem} % used for \sout{}...
\usepackage{url}
\usepackage{graphicx}
\usepackage{makeidx}
\usepackage[cm]{fullpage}
%\usepackage{color}
\usepackage{fancyvrb}
\usepackage{xspace}
\usepackage{tabularx}
\usepackage{framed}
\usepackage{parskip}
\usepackage{epigraph}
\usepackage{ccicons}
\usepackage[nottoc]{tocbibind}
\usepackage{longtable}
\usepackage[footnote,printonlyused,withpage]{acronym}
\usepackage[]{bookmark,hyperref} % must be last
\usepackage[official]{eurosym}
\usepackage[usestackEOL]{stackengine}

% ************** myref
% http://tex.stackexchange.com/questions/228286/how-to-mix-ref-and-pageref#228292
\ifdefined\RUSSIAN
\newcommand{\myref}[1]{%
  \ref{#1}
  (стр.~\pageref{#1})%
  }
% FIXME: I wasn't able to force varioref to output russian text...
\else
\usepackage{varioref}
\newcommand{\myref}[1]{\vref{#1}}
\fi
% ************** myref

\usepackage{glossaries}
\usepackage{tikz}
%\usepackage{fixltx2e}
\usepackage{bytefield}

\usepackage{amsmath}
\usepackage{MnSymbol}
\undef\mathdollar

\usepackage{float}

\usepackage{shorttoc}
\usetikzlibrary{calc,positioning,chains,arrows}
\usepackage{geometry}

%--------------------
% to prevent clashing of numbers and titles in TOC:
% https://tex.stackexchange.com/a/64124
\usepackage{tocloft}% http://ctan.org/pkg/tocloft
\makeatletter
\renewcommand{\numberline}[1]{%
  \@cftbsnum #1\@cftasnum~\@cftasnumb%
}
\makeatother
%--------------------

\newcommand{\footnoteref}[1]{\textsuperscript{\ref{#1}}}

%\definecolor{lstbgcolor}{rgb}{0.94,0.94,0.94}

% I don't know why this voodoo works, but without all-caps, it can't find LIGHT-GRAY color. WTF?
% see also: https://tex.stackexchange.com/questions/64298/error-with-xcolor-package
\definecolor{light-gray}{gray}{0.87}
\definecolor{LIGHT-GRAY}{gray}{0.87}
\definecolor{RED}{rgb}{1,0,0}
\makeindex

\newcommand{\RepoURL}{https://beginners.re/current-tree}

\newcommand{\ProgCalcURL}{\url{https://yurichev.com/progcalc/}}
\newcommand{\FNURLMSDNZP}{\footnote{\href{https://docs.microsoft.com/en-us/previous-versions/ms253935(v=vs.90)}{MSDN: Working with Packing Structures}}}
\newcommand{\FNURLGCCPC}{\footnote{\href{https://gcc.gnu.org/onlinedocs/gcc-4.4.4/gcc/Structure_002dPacking-Pragmas.html}{Structure-Packing Pragmas}}}
\newcommand{\MSDNscanf}{\href{https://docs.microsoft.com/en-us/cpp/c-runtime-library/reference/scanf-scanf-l-wscanf-wscanf-l?view=vs-2019}{MSDN}}

\newcommand{\WikipediaParityFlag}{\url{https://en.wikipedia.org/wiki/Parity_flag}}

\newcommand{\ESph}{\ES{{\color{red}Spanish text placeholder}}}
\newcommand{\PTBRph}{\PTBR{{\color{red}Brazilian Portuguese text placeholder}}}
\newcommand{\PLph}{\PL{{\color{red}Polish text placeholder}}}
\newcommand{\ITph}{\IT{{\color{red}Italian text placeholder}}}
\newcommand{\DEph}{\DE{{\color{red}German text placeholder}}}
\newcommand{\THAph}{\THA{{\color{red}Thai text placeholder}}}
\newcommand{\NLph}{\NL{{\color{red}Dutch text placeholder}}}
\newcommand{\FRph}{\FR{{\color{red}French text placeholder}}}
\newcommand{\JAph}{\JA{{\color{red}Japanese text placeholder}}}
\newcommand{\TRph}{\TR{{\color{red}Turkish text placeholder}}}

\newcommand{\myhrule}{\begin{center}\rule{0.5\textwidth}{.4pt}\end{center}}

% TODO: find better name:
\newcommand{\myincludegraphics}[1]{\includegraphics[width=\textwidth]{#1}}
\newcommand{\myincludegraphicsSmall}[1]{\includegraphics[width=0.3\textwidth]{#1}}

%\newcommand{\myincludegraphicsSmallOrNormalForEbook}{\myincludegraphicsSmall}

% advertisement:
%\ifdefined\ENGLISH{}
%\newcommand{\mysection}[2][]{
%    \clearpage\input{ad_EN}\clearpage
%    \ifx&#1& \section{#2}
%    \else    \section[#1]{#2}
%    \fi
%}
%\else
\newcommand{\mysection}[2][]{
    \ifx&#1& \section{#2}
    \else    \section[#1]{#2}
    \fi
}
%\fi

%\newcommand*{\TT}[1]{\texttt{#1}}
% synonyms, so far
\newcommand*{\InSqBrackets}[1]{\lbrack{}#1\rbrack{}}

% too contrast text?
%\newcommand*{\TT}[1]{\colorbox{light-gray}{\texttt{#1}}}
%\newcommand*{\GTT}[1]{\colorbox{light-gray}{\texttt{#1}}}
\newcommand*{\TT}[1]{\texttt{#1}}
\newcommand*{\GTT}[1]{\texttt{#1}}

%\newcommand*{\EN}[1]{\iflanguage{english}{#1}{}}

\newcommand{\HeaderColor}{\cellcolor{blue!25}}

\ifdefined\ENGLISH{}
\newcommand*{\EN}[1]{#1}
\else
\newcommand*{\EN}[1]{}
\fi

\ifdefined\RUSSIAN{}
%\newcommand*{\RU}[1]{\iflanguage{russian}{#1}{}}
\newcommand*{\RU}[1]{#1}
\else
\newcommand*{\RU}[1]{}
\fi

\ifdefined\CHINESE{}
%\newcommand*{\CN}[1]{\iflanguage{chinese}{#1}{}}
\newcommand*{\CN}[1]{#1}
\else
\newcommand*{\CN}[1]{}
\fi

\ifdefined\SPANISH{}
%\newcommand*{\ES}[1]{\iflanguage{spanish}{#1}{}}
\newcommand*{\ES}[1]{#1}
\else
\newcommand*{\ES}[1]{}
\fi

\ifdefined\ITALIAN{}
%\newcommand*{\IT}[1]{\iflanguage{italian}{#1}{}}
\newcommand*{\IT}[1]{#1}
\else
\newcommand*{\IT}[1]{}
\fi

\ifdefined\BRAZILIAN{}
%\newcommand*{\PTBR}[1]{\iflanguage{portuges}{#1}{}}
\newcommand*{\PTBR}[1]{#1}
\else
\newcommand*{\PTBR}[1]{}
\fi

\ifdefined\POLISH{}
%\newcommand*{\PL}[1]{\iflanguage{polish}{#1}{}}
\newcommand*{\PL}[1]{#1}
\else
\newcommand*{\PL}[1]{}
\fi

\ifdefined\GERMAN{}
%\newcommand*{\DE}[1]{\iflanguage{german}{#1}{}}
\newcommand*{\DE}[1]{#1}
\else
\newcommand*{\DE}[1]{}
\fi

\ifdefined\THAI{}
%\newcommand*{\THA}[1]{\iflanguage{thai}{#1}{}}
\newcommand*{\THA}[1]{#1}
\else
\newcommand*{\THA}[1]{}
\fi

\ifdefined\DUTCH{}
%\newcommand*{\NL}[1]{\iflanguage{dutch}{#1}{}}
\newcommand*{\NL}[1]{#1}
\else
\newcommand*{\NL}[1]{}
\fi

\ifdefined\FRENCH{}
%\newcommand*{\FR}[1]{\iflanguage{french}{#1}{}}
\newcommand*{\FR}[1]{#1}
\else
\newcommand*{\FR}[1]{}
\fi

\ifdefined\JAPANESE{}
%\newcommand*{\JPN}[1]{\iflanguage{japanese}{#1}{}}
\newcommand*{\JA}[1]{#1}
\else
\newcommand*{\JA}[1]{}
\fi

\ifdefined\TURKISH{}
%\newcommand*{\TR}[1]{\iflanguage{turkish}{#1}{}}
\newcommand*{\TR}[1]{#1}
\else
\newcommand*{\TR}[1]{}
\fi

\newcommand*{\dittoclosing}{---''---}
\newcommand*{\AsteriskOne}{${}^{*}$}
\newcommand*{\AsteriskTwo}{${}^{**}$}
\newcommand*{\AsteriskThree}{${}^{***}$}

% http://www.ctan.org/pkg/csquotes
\newcommand{\q}[1]{\enquote{#1}}

\newcommand{\var}[1]{\textit{#1}}

\newcommand{\ttf}{\GTT{f()}\xspace}
\newcommand{\ttfone}{\GTT{f1()}\xspace}

% http://tex.stackexchange.com/questions/32160/new-line-after-paragraph
\newcommand{\myparagraph}[1]{\paragraph{#1}\mbox{}\\}
\newcommand{\mysubparagraph}[1]{\subparagraph{#1}\mbox{}\\}

\newcommand{\figref}[1]{\figname{}\ref{#1}\xspace}
\newcommand{\lstref}[1]{\listingname{}\ref{#1}\xspace}
\newcommand{\MacOSX}{Mac OS X\xspace}

% FIXME TODO non-overlapping color!
% \newcommand{\headercolor}{\cellcolor{blue!25}}
\newcommand{\headercolor}{}

% FIXME: get rid of:
\newcommand{\IDA}{\ac{IDA}\xspace}

% FIXME: get rid of:
\newcommand{\tracer}{\protect\gls{tracer}\xspace}

\newcommand{\Tchar}{\emph{char}\xspace}
\newcommand{\Tint}{\emph{int}\xspace}
\newcommand{\Tbool}{\emph{bool}\xspace}
\newcommand{\Tfloat}{\emph{float}\xspace}
\newcommand{\Tdouble}{\emph{double}\xspace}
\newcommand{\Tvoid}{\emph{void}\xspace}
\newcommand{\ITthis}{\emph{this}\xspace}

\newcommand{\Ox}{\GTT{/Ox}\xspace}
\newcommand{\Obzero}{\GTT{/Ob0}\xspace}
\newcommand{\Othree}{\GTT{-O3}\xspace}

\newcommand{\oracle}{Oracle RDBMS\xspace}

\newcommand{\idevices}{iPod/iPhone/iPad\xspace}
\newcommand{\olly}{OllyDbg\xspace}

% common C functions
\newcommand{\printf}{\GTT{printf()}\xspace}
\newcommand{\puts}{\GTT{puts()}\xspace}
\newcommand{\main}{\GTT{main()}\xspace}
\newcommand{\qsort}{\GTT{qsort()}\xspace}
\newcommand{\strlen}{\GTT{strlen()}\xspace}
\newcommand{\scanf}{\GTT{scanf()}\xspace}
\newcommand{\rand}{\GTT{rand()}\xspace}


% for easier fiddling with formatting of all instructions together
\newcommand{\INS}[1]{\GTT{#1}\xspace}

% x86 instructions
\newcommand{\ADD}{\INS{ADD}}
\newcommand{\ADRP}{\INS{ADRP}}
\newcommand{\AND}{\INS{AND}}
\newcommand{\CALL}{\INS{CALL}}
\newcommand{\CPUID}{\INS{CPUID}}
\newcommand{\CMP}{\INS{CMP}}
\newcommand{\DEC}{\INS{DEC}}
\newcommand{\FADDP}{\INS{FADDP}}
\newcommand{\FCOM}{\INS{FCOM}}
\newcommand{\FCOMP}{\INS{FCOMP}}
\newcommand{\FCOMI}{\INS{FCOMI}}
\newcommand{\FCOMIP}{\INS{FCOMIP}}
\newcommand{\FUCOM}{\INS{FUCOM}}
\newcommand{\FUCOMI}{\INS{FUCOMI}}
\newcommand{\FUCOMIP}{\INS{FUCOMIP}}
\newcommand{\FUCOMPP}{\INS{FUCOMPP}}
\newcommand{\FDIVR}{\INS{FDIVR}}
\newcommand{\FDIV}{\INS{FDIV}}
\newcommand{\FLD}{\INS{FLD}}
\newcommand{\FMUL}{\INS{FMUL}}
\newcommand{\MUL}{\INS{MUL}}
\newcommand{\FSTP}{\INS{FSTP}}
\newcommand{\FDIVP}{\INS{FDIVP}}
\newcommand{\IDIV}{\INS{IDIV}}
\newcommand{\IMUL}{\INS{IMUL}}
\newcommand{\INC}{\INS{INC}}
\newcommand{\JAE}{\INS{JAE}}
%\newcommand{\JA}{\INS{JA}} % now used for Japanese text
\newcommand{\JBE}{\INS{JBE}}
\newcommand{\JB}{\INS{JB}}
\newcommand{\JE}{\INS{JE}}
\newcommand{\JGE}{\INS{JGE}}
\newcommand{\JG}{\INS{JG}}
\newcommand{\JLE}{\INS{JLE}}
\newcommand{\JL}{\INS{JL}}
\newcommand{\JMP}{\INS{JMP}}
\newcommand{\JNE}{\INS{JNE}}
\newcommand{\JNZ}{\INS{JNZ}}
\newcommand{\JNA}{\INS{JNA}}
\newcommand{\JNAE}{\INS{JNAE}}
\newcommand{\JNB}{\INS{JNB}}
\newcommand{\JNBE}{\INS{JNBE}}
\newcommand{\JZ}{\INS{JZ}}
\newcommand{\JP}{\INS{JP}}
\newcommand{\Jcc}{\INS{Jcc}}
\newcommand{\SETcc}{\INS{SETcc}}
\newcommand{\LEA}{\INS{LEA}}
\newcommand{\LOOP}{\INS{LOOP}}
\newcommand{\MOVSX}{\INS{MOVSX}}
\newcommand{\MOVZX}{\INS{MOVZX}}
\newcommand{\MOV}{\INS{MOV}}
\newcommand{\NOP}{\INS{NOP}}
\newcommand{\POP}{\INS{POP}}
\newcommand{\PUSH}{\INS{PUSH}}
\newcommand{\NOT}{\INS{NOT}}
\newcommand{\NOR}{\INS{NOR}}
\newcommand{\RET}{\INS{RET}}
\newcommand{\RETN}{\INS{RETN}}
\newcommand{\SETNZ}{\INS{SETNZ}}
\newcommand{\SETBE}{\INS{SETBE}}
\newcommand{\SETNBE}{\INS{SETNBE}}
\newcommand{\SUB}{\INS{SUB}}
\newcommand{\TEST}{\INS{TEST}}
\newcommand{\TST}{\INS{TST}}
\newcommand{\FNSTSW}{\INS{FNSTSW}}
\newcommand{\SAHF}{\INS{SAHF}}
\newcommand{\XOR}{\INS{XOR}}
\newcommand{\OR}{\INS{OR}}
\newcommand{\SHL}{\INS{SHL}}
\newcommand{\SHR}{\INS{SHR}}
\newcommand{\SAR}{\INS{SAR}}
\newcommand{\LEAVE}{\INS{LEAVE}}
\newcommand{\MOVDQA}{\INS{MOVDQA}}
\newcommand{\MOVDQU}{\INS{MOVDQU}}
\newcommand{\PADDD}{\INS{PADDD}}
\newcommand{\PCMPEQB}{\INS{PCMPEQB}}
\newcommand{\LDR}{\INS{LDR}}
\newcommand{\LSL}{\INS{LSL}}
\newcommand{\LSR}{\INS{LSR}}
\newcommand{\ASR}{\INS{ASR}}
\newcommand{\RSB}{\INS{RSB}}
\newcommand{\BTR}{\INS{BTR}}
\newcommand{\BTS}{\INS{BTS}}
\newcommand{\BTC}{\INS{BTC}}
\newcommand{\LUI}{\INS{LUI}}
\newcommand{\ORI}{\INS{ORI}}
\newcommand{\BIC}{\INS{BIC}}
\newcommand{\EOR}{\INS{EOR}}
\newcommand{\MOVS}{\INS{MOVS}}
\newcommand{\LSLS}{\INS{LSLS}}
\newcommand{\LSRS}{\INS{LSRS}}
\newcommand{\FMRS}{\INS{FMRS}}
\newcommand{\CMOVNE}{\INS{CMOVNE}}
\newcommand{\CMOVNZ}{\INS{CMOVNZ}}
\newcommand{\ROL}{\INS{ROL}}
\newcommand{\CSEL}{\INS{CSEL}}
\newcommand{\SLL}{\INS{SLL}}
\newcommand{\SLLV}{\INS{SLLV}}
\newcommand{\SW}{\INS{SW}}
\newcommand{\LW}{\INS{LW}}

% x86 flags

\newcommand{\ZF}{\GTT{ZF}\xspace}
\newcommand{\CF}{\GTT{CF}\xspace}
\newcommand{\PF}{\GTT{PF}\xspace}

% x86 registers

\newcommand{\AL}{\GTT{AL}\xspace}
\newcommand{\AH}{\GTT{AH}\xspace}
\newcommand{\AX}{\GTT{AX}\xspace}
\newcommand{\EAX}{\GTT{EAX}\xspace}
\newcommand{\EBX}{\GTT{EBX}\xspace}
\newcommand{\ECX}{\GTT{ECX}\xspace}
\newcommand{\EDX}{\GTT{EDX}\xspace}
\newcommand{\DL}{\GTT{DL}\xspace}
\newcommand{\ESI}{\GTT{ESI}\xspace}
\newcommand{\EDI}{\GTT{EDI}\xspace}
\newcommand{\EBP}{\GTT{EBP}\xspace}
\newcommand{\ESP}{\GTT{ESP}\xspace}
\newcommand{\RSP}{\GTT{RSP}\xspace}
\newcommand{\EIP}{\GTT{EIP}\xspace}
\newcommand{\RIP}{\GTT{RIP}\xspace}
\newcommand{\RAX}{\GTT{RAX}\xspace}
\newcommand{\RBX}{\GTT{RBX}\xspace}
\newcommand{\RCX}{\GTT{RCX}\xspace}
\newcommand{\RDX}{\GTT{RDX}\xspace}
\newcommand{\RBP}{\GTT{RBP}\xspace}
\newcommand{\RSI}{\GTT{RSI}\xspace}
\newcommand{\RDI}{\GTT{RDI}\xspace}
\newcommand*{\ST}[1]{\GTT{ST(#1)}\xspace}
\newcommand*{\XMM}[1]{\GTT{XMM#1}\xspace}

% ARM
\newcommand*{\Reg}[1]{\GTT{R#1}\xspace}
\newcommand*{\RegX}[1]{\GTT{X#1}\xspace}
\newcommand*{\RegW}[1]{\GTT{W#1}\xspace}
\newcommand*{\RegD}[1]{\GTT{D#1}\xspace}
\newcommand{\ADREQ}{\GTT{ADREQ}\xspace}
\newcommand{\ADRNE}{\GTT{ADRNE}\xspace}
\newcommand{\BEQ}{\GTT{BEQ}\xspace}

% FIXME tidy this pls
\newcommand{\RegTableThree}[5]{
\begin{center}
\begin{tabular}{ | l | l | l | l | l | l | l | l | l |}
\hline
\RegHeaderTop \\
\hline
\RegHeader \\
\hline
\multicolumn{8}{ | c | }{#1} \\
\hline
\multicolumn{4}{ | c | }{} & \multicolumn{4}{ c | }{#2} \\
\hline
\multicolumn{6}{ | c | }{} & \multicolumn{2}{ c | }{#3} \\
\hline
\multicolumn{6}{ | c | }{} & #4 & #5 \\
\hline
\end{tabular}
\end{center}
}

\newcommand{\RegTableOne}[5]{\RegTableThree{#1\textsuperscript{x64}}{#2}{#3}{#4}{#5}}

\newcommand{\RegTableTwo}[4]{
\begin{center}
\begin{tabular}{ | l | l | l | l | l | l | l | l | l |}
\hline
\RegHeaderTop \\
\hline
\RegHeader \\
\hline
\multicolumn{8}{ | c | }{#1\textsuperscript{x64}} \\
\hline
\multicolumn{4}{ | c | }{} & \multicolumn{4}{ c | }{#2} \\
\hline
\multicolumn{6}{ | c | }{} & \multicolumn{2}{ c | }{#3} \\
\hline
\multicolumn{7}{ | c | }{} & #4\textsuperscript{x64} \\
\hline
\end{tabular}
\end{center}
}

\newcommand{\RegTableFour}[4]{
\begin{center}
\begin{tabular}{ | l | l | l | l | l | l | l | l | l |}
\hline
\RegHeaderTop \\
\hline
\RegHeader \\
\hline
\multicolumn{8}{ | c | }{#1} \\
\hline
\multicolumn{4}{ | c | }{} & \multicolumn{4}{ c | }{#2} \\
\hline
\multicolumn{6}{ | c | }{} & \multicolumn{2}{ c | }{#3} \\
\hline
\multicolumn{7}{ | c | }{} & #4 \\
\hline
\end{tabular}
\end{center}
}

\newcommand{\NonOptimizingKeilVI}{\NonOptimizing Keil 6/2013\xspace}
\newcommand{\OptimizingKeilVI}{\Optimizing Keil 6/2013\xspace}
\newcommand{\NonOptimizingXcodeIV}{\NonOptimizing Xcode 4.6.3 (LLVM)\xspace}
\newcommand{\OptimizingXcodeIV}{\Optimizing Xcode 4.6.3 (LLVM)\xspace}

\newcommand{\OracleTablesName}{oracle tables\xspace}
\newcommand{\oracletables}{\OracleTablesName\footnote{\href{http://yurichev.com/oracle_tables.html}{yurichev.com}}\xspace}

\newcommand{\BGREPURL}{\href{https://github.com/yurichev/bgrep}{GitHub}}
\newcommand{\FNMSDNROTxURL}{\footnote{\href{http://msdn.microsoft.com/en-us/library/5cc576c4.aspx}{MSDN}}}

\newcommand{\YurichevIDAIDCScripts}{https://github.com/yurichev/IDA_scripts}

% sources: books, etc
\newcommand{\TAOCPvolI}{Donald E. Knuth, \emph{The Art of Computer Programming}, Volume 1, 3rd ed., (1997)}
\newcommand{\TAOCPvolII}{Donald E. Knuth, \emph{The Art of Computer Programming}, Volume 2, 3rd ed., (1997)}
\newcommand{\Russinovich}{Mark Russinovich, \emph{Microsoft Windows Internals}}
\newcommand{\Schneier}{Bruce Schneier, \emph{Applied Cryptography}, (John Wiley \& Sons, 1994)}
\newcommand{\AgnerFog}{Agner Fog, \emph{The microarchitecture of Intel, AMD and VIA CPUs}, (2016)}
\newcommand{\AgnerFogCPP}{Agner Fog, \emph{Optimizing software in C++} (2015)}
\newcommand{\AgnerFogCC}{Agner Fog, \emph{Calling conventions} (2015)}
\newcommand{\JavaBook}{[Tim Lindholm, Frank Yellin, Gilad Bracha, Alex Buckley, \emph{The Java(R) Virtual Machine Specification / Java SE 7 Edition}]
\footnote{\AlsoAvailableAs \url{https://docs.oracle.com/javase/specs/jvms/se7/jvms7.pdf}; \url{http://docs.oracle.com/javase/specs/jvms/se7/html/}}}

\newcommand{\ARMPCS}{\InSqBrackets{\emph{Procedure Call Standard for the ARM 64-bit Architecture (AArch64)}, (2013)}\footnote{\AlsoAvailableAs \url{http://infocenter.arm.com/help/topic/com.arm.doc.ihi0055b/IHI0055B_aapcs64.pdf}}}

\newcommand{\IgorSkochinsky}{[Igor Skochinsky, \emph{Compiler Internals: Exceptions and RTTI}, (2012)] \footnote{\AlsoAvailableAs \url{http://yurichev.com/mirrors/RE/Recon-2012-Skochinsky-Compiler-Internals.pdf}}}

\newcommand{\PietrekSEH}{[Matt Pietrek, \emph{A Crash Course on the Depths of Win32\texttrademark{} Structured Exception Handling}, (1997)]\footnote{\AlsoAvailableAs \url{http://www.microsoft.com/msj/0197/Exception/Exception.aspx}}}

\newcommand{\PietrekPE}{Matt Pietrek, \emph{An In-Depth Look into the Win32 Portable Executable File Format}, (2002)]}

\newcommand{\PietrekPEURL}{\PietrekPE\footnote{\AlsoAvailableAs \url{http://msdn.microsoft.com/en-us/magazine/bb985992.aspx}}}

\newcommand{\RitchieDevC}{[Dennis M. Ritchie, \emph{The development of the C language}, (1993)]\footnote{\AlsoAvailableAs \href{https://yurichev.com/mirrors/C/dmr-The\%20Development\%20of\%20the\%20C\%20Language-1993.pdf}{pdf}}}

\newcommand{\RitchieThompsonUNIX}{[D. M. Ritchie and K. Thompson, \emph{The UNIX Time Sharing System}, (1974)]\footnote{\AlsoAvailableAs \href{http://dl.acm.org/citation.cfm?id=361061}{URL}}}

\newcommand{\DrepperTLS}{[Ulrich Drepper, \emph{ELF Handling For Thread-Local Storage}, (2013)]\footnote{\AlsoAvailableAs \url{http://www.akkadia.org/drepper/tls.pdf}}}

\newcommand{\DrepperMemory}{[Ulrich Drepper, \emph{What Every Programmer Should Know About Memory}, (2007)]\footnote{\AlsoAvailableAs \url{http://www.akkadia.org/drepper/cpumemory.pdf}}}

\newcommand{\AlephOne}{[Aleph One, \emph{Smashing The Stack For Fun And Profit}, (1996)]\footnote{\AlsoAvailableAs \url{http://yurichev.com/mirrors/phrack/p49-0x0e.txt}}}

\ifdefined\RUSSIAN
\newcommand{\KRBook}{Брайан Керниган, Деннис Ритчи, \emph{Язык программирования Си}, второе издание, (1988, 2009)}
\newcommand{\CppOneOneStd}{Стандарт Си++11}
\newcommand{\CNotes}{Денис Юричев, \emph{Заметки о языке программирования Си/Си++}}
\else
\newcommand{\KRBook}{Brian W. Kernighan, Dennis M. Ritchie, \emph{The C Programming Language}, 2ed, (1988)}
\newcommand{\CppOneOneStd}{C++11 standard}
\newcommand{\CNotes}{Dennis Yurichev, \emph{C/C++ programming language notes}}
\fi

\newcommand{\RobPikePractice}{Brian W. Kernighan, Rob Pike, \emph{Practice of Programming}, (1999)}

\newcommand{\ARMSixFourRef}{\emph{ARM Architecture Reference Manual, ARMv8, for ARMv8-A architecture profile}, (2013)}
\newcommand{\ARMSixFourRefURL}{\InSqBrackets{\ARMSixFourRef}\footnote{\AlsoAvailableAs \url{http://yurichev.com/mirrors/ARMv8-A_Architecture_Reference_Manual_(Issue_A.a).pdf}}}

\newcommand{\SysVABI}{[Michael Matz, Jan Hubicka, Andreas Jaeger, Mark Mitchell, \emph{System V Application Binary Interface. AMD64 Architecture Processor Supplement}, (2013)]
\footnote{\AlsoAvailableAs \url{https://software.intel.com/sites/default/files/article/402129/mpx-linux64-abi.pdf}}}

\newcommand{\IOSABI}{\InSqBrackets{\emph{iOS ABI Function Call Guide}, (2010)}\footnote{\AlsoAvailableAs \url{http://developer.apple.com/library/ios/documentation/Xcode/Conceptual/iPhoneOSABIReference/iPhoneOSABIReference.pdf}}}

\newcommand{\CNineNineStd}{\emph{ISO/IEC 9899:TC3 (C C99 standard)}, (2007)}

\newcommand{\TCPPPL}{Bjarne Stroustrup, \emph{The C++ Programming Language, 4th Edition}, (2013)}

\newcommand{\AMDOptimization}{\emph{Software Optimization Guide for AMD Family 16h Processors}, (2013)}
% note = "\AlsoAvailableAs \url{http://yurichev.com/mirrors/AMD/SOG_16h_52128_PUB_Rev1_1.pdf}",

\newcommand{\IntelOptimization}{\emph{Intel® 64 and IA-32 Architectures Optimization Reference Manual}, (2014)}
% note = "\AlsoAvailableAs \url{http://www.intel.com/content/www/us/en/architecture-and-technology/64-ia-32-architectures-optimization-manual.html}",

\newcommand{\TAOUP}{Eric S. Raymond, \emph{The Art of UNIX Programming}, (2003)}
% note = "\AlsoAvailableAs \url{http://catb.org/esr/writings/taoup/html/}",

\newcommand{\HenryWarren}{Henry S. Warren, \emph{Hacker's Delight}, (2002)}

\newcommand{\ParashiftCPPFAQ}{Marshall Cline, \emph{C++ FAQ}}
% note = "\AlsoAvailableAs \url{http://www.parashift.com/c++-faq-lite/index.html}",

\newcommand{\ARMCookBook}{Advanced RISC Machines Ltd, \emph{The ARM Cookbook}, (1994)}
% note = "\AlsoAvailableAs \url{https://yurichev.com/ref/ARM%20Cookbook%20(1994)/}",

\newcommand{\ARMSevenRef}{\emph{ARM(R) Architecture Reference Manual, ARMv7-A and ARMv7-R edition}, (2012)}

\newcommand{\MAbrash}{Michael Abrash, \emph{Graphics Programming Black Book}, 1997}

\newcommand{\MathForProg}{Mathematical Recipes\footnote{\url{https://math.recipes}}}

\newcommand{\radare}{rada.re}

\newcommand{\FNURLREDDIT}{\footnote{\href{http://www.reddit.com/r/ReverseEngineering/}{reddit.com/r/ReverseEngineering/}}}

\EN{\newcommand{\AcronymsUsed}{Acronyms Used}

\newcommand{\TitleRE}{Reverse Engineering for Beginners}

\newcommand{\TitleUAL}{Understanding Assembly Language}

\newcommand{\AUTHOR}{Dennis Yurichev}

\newcommand{\figname}{fig.\xspace}
\newcommand{\listingname}{listing.\xspace}
\newcommand{\Sourcecode}{Source code\xspace}
\newcommand{\Seealso}{See also\xspace}
\newcommand{\tableheader}{\headercolor{} offset & \headercolor{} description}
% instructions descriptions
\newcommand{\ASRdesc}{arithmetic shift right}

% x86 registers tables
\newcommand{\RegHeaderTop}{ \multicolumn{8}{ | c | }{ Byte number: } }
% TODO: non-overlapping color!
\newcommand{\RegHeader}{ 7th & 6th & 5th & 4th & 3rd & 2nd & 1st & 0th }
\newcommand{\ReturnAddress}{Return Address}

\newcommand{\localVariable}{local variable}

\newcommand{\savedValueOf}{saved value of}

% for index
\newcommand{\GrepUsage}{grep usage}
\newcommand{\SyntacticSugar}{Syntactic Sugar}
\newcommand{\CompilerAnomaly}{Compiler's anomalies}
\newcommand{\CLanguageElements}{C language elements}
\newcommand{\CStandardLibrary}{C standard library}
\newcommand{\Instructions}{Instructions}
\newcommand{\Pseudoinstructions}{Pseudoinstructions}
\newcommand{\Prefixes}{Prefixes}

\newcommand{\Flags}{Flags}
\newcommand{\Registers}{Registers}
\newcommand{\registers}{registers}
\newcommand{\Stack}{Stack}
\newcommand{\Recursion}{Recursion}
\newcommand{\RAM}{RAM}
\newcommand{\ROM}{ROM}
\newcommand{\Pointers}{Pointers}
\newcommand{\BufferOverflow}{Buffer Overflow}

% DE: also "Zusammenfassung"
\newcommand{\Conclusion}{Conclusion}

\newcommand{\Exercise}{Exercise}
\newcommand{\Exercises}{Exercises}
\newcommand{\Arrays}{Arrays}
\newcommand{\Cpp}{C++\xspace}
\newcommand{\CCpp}{C/C++\xspace}
\newcommand{\NonOptimizing}{Non-optimizing\xspace}
\newcommand{\Optimizing}{Optimizing\xspace}
\newcommand{\ARMMode}{ARM mode\xspace}
\newcommand{\ThumbMode}{Thumb mode\xspace}
\newcommand{\ThumbTwoMode}{Thumb-2 mode\xspace}

\newcommand{\DataProcessingInstructionsFootNote}{These instructions are also called \q{data processing instructions}}
% for .bib files
\newcommand{\AlsoAvailableAs}{Also available as\xspace}

% section names
\newcommand{\ShiftsSectionName}{Shifts}
\newcommand{\HelloWorldSectionName}{Hello, world!}
\newcommand{\SwitchCaseDefaultSectionName}{switch()/case/default}
\newcommand{\PrintfSeveralArgumentsSectionName}{printf() with several arguments}
\newcommand{\BitfieldsChapter}{Manipulating specific bit(s)}
\newcommand{\ArithOptimizations}{Replacing arithmetic instructions to other ones}
\newcommand{\FPUChapterName}{Floating-point unit}
\newcommand{\MoreAboutStrings}{More about strings}
\newcommand{\DivisionByMultSectionName}{Division using multiplication}
\newcommand{\Answer}{Answer}
\newcommand{\WhatThisCodeDoes}{What does this code do}
\newcommand{\WorkingWithFloatAsWithStructSubSubSectionName}{Handling float data type as a structure}

\newcommand{\MinesweeperWinXPExampleChapterName}{Minesweeper (Windows XP)}
\newcommand{\StructurePackingSectionName}{Fields packing in structure}
\newcommand{\StructuresChapterName}{Structures}
\newcommand{\PICcode}{position-independent code}
\newcommand{\CapitalPICcode}{Position-independent code}
\newcommand{\Loops}{Loops}

% C
\newcommand{\PostIncrement}{Post-increment}
\newcommand{\PostDecrement}{Post-decrement}
\newcommand{\PreIncrement}{Pre-increment}
\newcommand{\PreDecrement}{Pre-decrement}

% MIPS
\newcommand{\GlobalPointer}{Global Pointer}

\newcommand{\garbage}{garbage}
\newcommand{\IntelSyntax}{Intel syntax}
\newcommand{\ATTSyntax}{AT\&T syntax}
\newcommand{\randomNoise}{random noise}
\newcommand{\Example}{Example}
\newcommand{\argument}{argument}
\newcommand{\MarkedInIDAAs}{marked in \IDA as}
\newcommand{\stepover}{step over}
\newcommand{\ShortHotKeyCheatsheet}{Hot-keys cheatsheet}

\newcommand{\assemblyOutput}{assembly output}

% ML prefix is for multi-lingual words and sentences:
\newcommand{\MLHeap}{Heap}
\newcommand{\MLStack}{Stack}
\newcommand{\MLStackOverflow}{Stack overflow}
\newcommand{\MLStartOfHeap}{Start of heap}
\newcommand{\MLStartOfStack}{Start of stack}
\newcommand{\MLinputA}{input A}
\newcommand{\MLinputB}{input B}
\newcommand{\MLoutput}{output}
\newcommand{\SoftwareCracking}{Software cracking}

\newcommand{\EMAILPRI}{\href{https://yurichev.com/contact.html}{my emails}}
\newcommand{\EMAILS}{\href{https://yurichev.com/contact.html}{my emails}}
}
\RU{\newcommand{\AcronymsUsed}{Список принятых сокращений}

\newcommand{\TitleRE}{Reverse Engineering для начинающих}

\newcommand{\TitleUAL}{Понимание языка ассемблера}

\newcommand{\AUTHOR}{Денис Юричев}

\newcommand{\figname}{илл.\xspace}
\newcommand{\listingname}{листинг.\xspace}
\newcommand{\Sourcecode}{Исходный код}
\newcommand{\Seealso}{См. также\xspace}
\newcommand{\tableheader}{\headercolor{} смещение & \headercolor{} описание }
% instructions descriptions
\newcommand{\ASRdesc}{арифметический сдвиг вправо}

% x86 registers tables
\newcommand{\RegHeaderTop}{ \multicolumn{8}{ | c | }{ Номер байта: } }
% TODO: non-overlapping color!
\newcommand{\RegHeader}{ 7-й & 6-й & 5-й & 4-й & 3-й & 2-й & 1-й & 0-й }
\newcommand{\ReturnAddress}{Адрес возврата}

\newcommand{\localVariable}{локальная переменная}

\newcommand{\savedValueOf}{сохраненное значение}

% for index
\newcommand{\GrepUsage}{Использование grep}
\newcommand{\SyntacticSugar}{Синтаксический сахар}
\newcommand{\CompilerAnomaly}{Аномалии компиляторов}
\newcommand{\CLanguageElements}{Элементы языка Си}
\newcommand{\CStandardLibrary}{Стандартная библиотека Си}
\newcommand{\Instructions}{Инструкции}
\newcommand{\Pseudoinstructions}{Псевдоинструкции}
\newcommand{\Prefixes}{Префиксы}

\newcommand{\Flags}{Флаги}
\newcommand{\Registers}{Регистры}
\newcommand{\registers}{регистры}
\newcommand{\Stack}{Стек}
\newcommand{\Recursion}{Рекурсия}
\newcommand{\RAM}{ОЗУ}
\newcommand{\ROM}{ПЗУ}
\newcommand{\Pointers}{Указатели}
\newcommand{\BufferOverflow}{Переполнение буфера}

\newcommand{\Conclusion}{Вывод}

\newcommand{\Exercise}{Упражнение}
\newcommand{\Exercises}{Упражнения}
\newcommand{\Arrays}{Массивы}
\newcommand{\Cpp}{Си++}
\newcommand{\CCpp}{Си/Си++}
\newcommand{\NonOptimizing}{Неоптимизирующий\xspace}
\newcommand{\Optimizing}{Оптимизирующий\xspace}
\newcommand{\ARMMode}{Режим ARM\xspace}
\newcommand{\ThumbMode}{Режим Thumb\xspace}
\newcommand{\ThumbTwoMode}{Режим Thumb-2\xspace}

\newcommand{\DataProcessingInstructionsFootNote}{Эти инструкции также называются \q{data processing instructions}}

% for .bib files
\newcommand{\AlsoAvailableAs}{Также доступно здесь:\xspace}

% section names
\newcommand{\ShiftsSectionName}{Сдвиги}
\newcommand{\HelloWorldSectionName}{Hello, world!}
\newcommand{\SwitchCaseDefaultSectionName}{switch()/case/default}
\newcommand{\PrintfSeveralArgumentsSectionName}{printf() с несколькими аргументами}
\newcommand{\BitfieldsChapter}{Работа с отдельными битами}
\newcommand{\ArithOptimizations}{Замена одних арифметических инструкций на другие}

\newcommand{\FPUChapterName}{Работа с FPU}
\newcommand{\MoreAboutStrings}{Еще кое-что о строках}
\newcommand{\DivisionByMultSectionName}{Деление используя умножение}
\newcommand{\Answer}{Ответ}
\newcommand{\WhatThisCodeDoes}{Что делает этот код}
\newcommand{\WorkingWithFloatAsWithStructSubSubSectionName}{Работа с типом float как со структурой}

\newcommand{\MinesweeperWinXPExampleChapterName}{Сапёр (Windows XP)}
\newcommand{\StructurePackingSectionName}{Упаковка полей в структуре}
\newcommand{\StructuresChapterName}{Структуры}
\newcommand{\PICcode}{адресно-независимый код}
\newcommand{\CapitalPICcode}{Адресно-независимый код}
\newcommand{\Loops}{Циклы}

% C
\newcommand{\PostIncrement}{Пост-инкремент}
\newcommand{\PostDecrement}{Пост-декремент}
\newcommand{\PreIncrement}{Пре-инкремент}
\newcommand{\PreDecrement}{Пре-декремент}

% MIPS
\newcommand{\GlobalPointer}{Глобальный указатель}

\newcommand{\garbage}{мусор}
\newcommand{\IntelSyntax}{Синтаксис Intel}
\newcommand{\ATTSyntax}{Синтаксис AT\&T}
\newcommand{\randomNoise}{случайный шум}
\newcommand{\Example}{Пример}
\newcommand{\argument}{аргумент}
\newcommand{\MarkedInIDAAs}{маркируется в \IDA как}
\newcommand{\stepover}{сделать шаг, не входя в функцию}
\newcommand{\ShortHotKeyCheatsheet}{Краткий справочник горячих клавиш}

\newcommand{\assemblyOutput}{вывод на ассемблере}

% ML prefix is for multi-lingual words and sentences:
\newcommand{\MLHeap}{Куча}
\newcommand{\MLStack}{Стэк}
\newcommand{\MLStackOverflow}{Переполнение стека}
\newcommand{\MLStartOfHeap}{Начало кучи}
\newcommand{\MLStartOfStack}{Вершина стека}
\newcommand{\MLinputA}{вход А}
\newcommand{\MLinputB}{вход Б}
\newcommand{\MLoutput}{выход}
\newcommand{\SoftwareCracking}{Взлом ПО}

\newcommand{\EMAILPRI}{\href{https://yurichev.com/contact.html}{мои адреса}}
\newcommand{\EMAILS}{\href{https://yurichev.com/contact.html}{мои адреса}}

}
\FR{\newcommand{\AcronymsUsed}{Acronymes utilisés}

\newcommand{\TitleRE}{Rétro-ingénierie pour Débutants}

\newcommand{\TitleUAL}{Comprendre le langage d'assemblage}

\newcommand{\AUTHOR}{Dennis Yurichev}

\newcommand{\figname}{fig.\xspace}
\newcommand{\listingname}{listado.\xspace}
\newcommand{\Sourcecode}{Code source\xspace}
\newcommand{\Seealso}{Voir également\xspace}
\newcommand{\tableheader}{\headercolor{} offset & \headercolor{} description }
% instructions descriptions
\newcommand{\ASRdesc}{décalage arithmétique vers la droite}

% x86 registers tables
\newcommand{\RegHeaderTop}{ \multicolumn{8}{ | c | }{ Octet d'indice } }
% TODO: non-overlapping color!
\newcommand{\RegHeader}{7 & 6 & 5 & 4 & 3 & 2 & 1 & 0}

\newcommand{\ReturnAddress}{Adresse de retour}

\newcommand{\localVariable}{variable locale}

\newcommand{\savedValueOf}{valeur enregistrée de}

% for index
\newcommand{\GrepUsage}{Utilisation de grep}
\newcommand{\SyntacticSugar}{Sucre syntaxique}
\newcommand{\CompilerAnomaly}{Anomalies du compilateur}
\newcommand{\CLanguageElements}{Éléments du langage C}
\newcommand{\CStandardLibrary}{Bibliothèque standard C}
\newcommand{\Instructions}{Instructions}
\newcommand{\Pseudoinstructions}{Pseudo-instructions}
\newcommand{\Prefixes}{Préfixes}

\newcommand{\Flags}{Flags}
\newcommand{\Registers}{Registres}
\newcommand{\registers}{registres}
\newcommand{\Stack}{Pile}
\newcommand{\Recursion}{Récursivité}
\newcommand{\RAM}{RAM}
\newcommand{\ROM}{ROM}
\newcommand{\Pointers}{Pointeurs}
\newcommand{\BufferOverflow}{Débordement de tampon}

\newcommand{\Conclusion}{Conclusion}

\newcommand{\Exercise}{Exercice}
\newcommand{\Exercises}{Exercices\xspace}
\newcommand{\Arrays}{Tableaux}
\newcommand{\Cpp}{C++\xspace}
\newcommand{\CCpp}{C/C++\xspace}
\newcommand{\NonOptimizing}{sans optimisation\xspace}
\newcommand{\Optimizing}{avec optimisation\xspace}
\newcommand{\ARMMode}{Mode ARM\xspace}
\newcommand{\ThumbMode}{Mode Thumb\xspace}
\newcommand{\ThumbTwoMode}{Mode Thumb-2\xspace}

\newcommand{\DataProcessingInstructionsFootNote}{Ces instructions sont également appelées \q{instructions de traitement de données}}

% for .bib files
\newcommand{\AlsoAvailableAs}{Aussi disponible en\xspace}

% section names
\newcommand{\ShiftsSectionName}{Décalages}
\newcommand{\HelloWorldSectionName}{Hello, world!}
\newcommand{\SwitchCaseDefaultSectionName}{switch()/case/default}
\newcommand{\PrintfSeveralArgumentsSectionName}{printf() avec plusieurs arguments}
\newcommand{\BitfieldsChapter}{Manipulation de bits spécifiques}
\newcommand{\ArithOptimizations}{Remplacement de certaines instructions arithmétiques par d'autres}

\newcommand{\FPUChapterName}{Unité à virgule flottante}
\newcommand{\MoreAboutStrings}{Plus d'information sur les chaînes}
\newcommand{\DivisionByMultSectionName}{Division par la multiplication}
\newcommand{\Answer}{Réponse}
\newcommand{\WhatThisCodeDoes}{Que fait ce code ?}
\newcommand{\WorkingWithFloatAsWithStructSubSubSectionName}{Travailler avec le type float comme une structure}

\newcommand{\MinesweeperWinXPExampleChapterName}{Démineur (Windows XP)}
\newcommand{\StructurePackingSectionName}{Organisation des champs dans la structure}
\newcommand{\StructuresChapterName}{Structures}
\newcommand{\PICcode}{code indépendant de la position}
\newcommand{\CapitalPICcode}{Code indépendant de la position}
\newcommand{\Loops}{Boucles}

% C
\newcommand{\PostIncrement}{Post-incrémentation}
\newcommand{\PostDecrement}{Post-décrémentation}
\newcommand{\PreIncrement}{Pré-incrémentation}
\newcommand{\PreDecrement}{Pré-décrémentation}

% MIPS
\newcommand{\GlobalPointer}{Pointeur Global}

\newcommand{\garbage}{déchets}
\newcommand{\IntelSyntax}{Syntaxe Intel}
\newcommand{\ATTSyntax}{Syntaxe AT\&T}
\newcommand{\randomNoise}{bruit aléatoire}
\newcommand{\Example}{Exemple}
\newcommand{\argument}{argument}
\newcommand{\MarkedInIDAAs}{marqué dans \IDA comme}
\newcommand{\stepover}{enjamber}
\newcommand{\ShortHotKeyCheatsheet}{Anti-sèche des touches de raccourci}

\newcommand{\assemblyOutput}{résultat en sortie de l'assembleur}

% ML prefix is for multi-lingual words and sentences:
\newcommand{\MLHeap}{Heap}
\newcommand{\MLStack}{Pile}
\newcommand{\MLStackOverflow}{Débordement de pile}
\newcommand{\MLStartOfHeap}{Début du heap}
\newcommand{\MLStartOfStack}{Début de la pile}
\newcommand{\MLinputA}{entrée A}
\newcommand{\MLinputB}{entrée B}
\newcommand{\MLoutput}{sortie}
\newcommand{\SoftwareCracking}{cracking de logiciel}

\newcommand{\EMAILPRI}{\href{https://yurichev.com/contact.html}{my emails}}
\newcommand{\EMAILS}{\href{https://yurichev.com/contact.html}{my emails}}

}
\DE{\newcommand{\AcronymsUsed}{Verwendete Abkürzungen}

\newcommand{\TitleRE}{Reverse Engineering für Einsteiger}

% TBT?
\newcommand{\TitleUAL}{Understanding Assembly Language}

\newcommand{\AUTHOR}{Dennis Yurichev}

\newcommand{\figname}{Abb.\xspace}

\newcommand{\listingname}{\DE{Listing.}\xspace}

\newcommand{\Sourcecode}{Quellcode\xspace}
\newcommand{\Seealso}{siehe auch\xspace}
\newcommand{\tableheader}{\headercolor{}Offset & \headercolor{} Beschreibung}
% instructions descriptions
\newcommand{\ASRdesc}{\DEph{}}

% x86 registers tables
\newcommand{\RegHeaderTop}{ \multicolumn{8}{ | c | }{ Byte-Nummer: }}

% TODO: non-overlapping color!
\newcommand{\RegHeader}{\DEph{}}

\newcommand{\ReturnAddress}{Rücksprungadresse}

\newcommand{\localVariable}{lokale Variable}

\newcommand{\savedValueOf}{\DEph{}}

% for index
\newcommand{\GrepUsage}{\DEph}
\newcommand{\SyntacticSugar}{\DEph{}}
\newcommand{\CompilerAnomaly}{\DEph{}}
\newcommand{\CLanguageElements}{C Sprachelemente}
\newcommand{\CStandardLibrary}{\DEph{}}
\newcommand{\Instructions}{\DEph{}}
\newcommand{\Pseudoinstructions}{\DEph{}}
\newcommand{\Prefixes}{\DEph{}}

\newcommand{\Flags}{\DEph{}}
\newcommand{\Registers}{\DEph{}}
\newcommand{\registers}{\DEph{}}
\newcommand{\Stack}{Stack}
\newcommand{\Recursion}{\DEph{}}
\newcommand{\RAM}{RAM}
\newcommand{\ROM}{ROM}
\newcommand{\Pointers}{\DEph{}}
\newcommand{\BufferOverflow}{\DEph{}}

% DE: also "Zusammenfassung"
\newcommand{\Conclusion}{Fazit}

\newcommand{\Exercise}{\DEph{}\xspace}
\newcommand{\Exercises}{Übungen\xspace}
\newcommand{\Arrays}{Arrays}
\newcommand{\Cpp}{C++\xspace}
\newcommand{\CCpp}{C/C++\xspace}
\newcommand{\NonOptimizing}{\DEph{}\xspace}
\newcommand{\Optimizing}{\DEph{}\xspace}
\newcommand{\ARMMode}{\DEph{}\xspace}
\newcommand{\ThumbMode}{\DEph{}\xspace}
\newcommand{\ThumbTwoMode}{\DEph{}\xspace}

\newcommand{\DataProcessingInstructionsFootNote}{\DEph{}}
% for .bib files
\newcommand{\AlsoAvailableAs}{\DEph{}\xspace}

% section names
\newcommand{\ShiftsSectionName}{\DEph{}}
\newcommand{\HelloWorldSectionName}{Hallo, Welt!}
\newcommand{\SwitchCaseDefaultSectionName}{switch()/case/default}
\newcommand{\PrintfSeveralArgumentsSectionName}{\DEph{}}
\newcommand{\BitfieldsChapter}{Manipulieren einzelner Bits}
\newcommand{\ArithOptimizations}{Ersetzen von arithmetischen Operationen}
\newcommand{\FPUChapterName}{Gleitkommaeinheit}
\newcommand{\MoreAboutStrings}{Mehr über Zeichenketten}
\newcommand{\DivisionByMultSectionName}{\DEph{}}
\newcommand{\Answer}{\DEph{}}
\newcommand{\WhatThisCodeDoes}{\DEph{}}
\newcommand{\WorkingWithFloatAsWithStructSubSubSectionName}{\DEph{}}

\newcommand{\MinesweeperWinXPExampleChapterName}{\DEph{} (Windows XP)}
\newcommand{\StructurePackingSectionName}{\DEph{}}
\newcommand{\StructuresChapterName}{\DEph{}}
\newcommand{\PICcode}{positionsabhängiger Code}
\newcommand{\CapitalPICcode}{Positionsabhängiger Code}
\newcommand{\Loops}{Schleifen}

% C
\newcommand{\PostIncrement}{\DEph{}}
\newcommand{\PostDecrement}{\DEph{}}
\newcommand{\PreIncrement}{\DEph{}}
\newcommand{\PreDecrement}{\DEph{}}

% MIPS
\newcommand{\GlobalPointer}{\DEph{}}

\newcommand{\garbage}{\DEph{}}
\newcommand{\IntelSyntax}{\DEph{}}
\newcommand{\ATTSyntax}{\DEph{}}
\newcommand{\randomNoise}{\DEph{}}
\newcommand{\Example}{\DEph{}}
\newcommand{\argument}{\DEph{}}
\newcommand{\MarkedInIDAAs}{\DEph{}}
\newcommand{\stepover}{\DEph{}}
\newcommand{\ShortHotKeyCheatsheet}{\DEph{}}

\newcommand{\assemblyOutput}{\DEph{}}

% ML prefix is for multi-lingual words and sentences:
\newcommand{\MLHeap}{\DEph{}}
\newcommand{\MLStack}{\DEph{}}
\newcommand{\MLStackOverflow}{\DEph{}}
\newcommand{\MLStartOfHeap}{\DEph{}}
\newcommand{\MLStartOfStack}{\DEph{}}
\newcommand{\MLinputA}{\DEph{}}
\newcommand{\MLinputB}{\DEph{}}
\newcommand{\MLoutput}{\DEph{}}
\newcommand{\SoftwareCracking}{\DEph{}}

\newcommand{\EMAILPRI}{\href{https://yurichev.com/contact.html}{my emails}}
\newcommand{\EMAILS}{\href{https://yurichev.com/contact.html}{my emails}}

}
\JA{\newcommand{\AcronymsUsed}{頭字語}

\newcommand{\TitleRE}{リバースエンジニアリング入門}

%TBT
\newcommand{\TitleUAL}{Understanding Assembly Language}

\newcommand{\AUTHOR}{Dennis Yurichev}

\newcommand{\figname}{fig.\xspace}
\newcommand{\listingname}{リスト}

\newcommand{\Sourcecode}{ソースコード\xspace}
\newcommand{\Seealso}{参照\xspace}
\newcommand{\tableheader}{\headercolor{} オフセット & \headercolor{} 記述 }
% instructions descriptions
\newcommand{\ASRdesc}{\JAph{}}

% x86 registers tables
\newcommand{\RegHeaderTop}{ \multicolumn{8}{ | c | }{ バイトの並び順 } }
% TODO: non-overlapping color!
\newcommand{\RegHeader}{ 第7 & 第6 & 第5 & 第4 & 第3 & 第2 & 第1 & 第0 }

\newcommand{\ReturnAddress}{リターンアドレス}

\newcommand{\localVariable}{ローカル変数}

\newcommand{\savedValueOf}{saved value of}

% for index
\newcommand{\GrepUsage}{\JAph{}}
\newcommand{\SyntacticSugar}{糖衣構文}
\newcommand{\CompilerAnomaly}{コンパイラアノマリ}
\newcommand{\CLanguageElements}{C言語の要素}
\newcommand{\CStandardLibrary}{C標準ライブラリ}
\newcommand{\Instructions}{命令}
\newcommand{\Pseudoinstructions}{疑似命令}
\newcommand{\Prefixes}{プリフィックス}

\newcommand{\Flags}{フラグ}
\newcommand{\Registers}{レジスタ}
\newcommand{\registers}{レジスタ}
\newcommand{\Stack}{スタック}
\newcommand{\Recursion}{再帰}
\newcommand{\RAM}{RAM}
\newcommand{\ROM}{ROM}
\newcommand{\Pointers}{ポインタ}
\newcommand{\BufferOverflow}{バッファオーバーフロー}

% DE: also "Zusammenfassung"
\newcommand{\Conclusion}{結論}

\newcommand{\Exercise}{練習問題\xspace}
\newcommand{\Exercises}{練習問題\xspace}
\newcommand{\Arrays}{配列}
\newcommand{\Cpp}{C++\xspace}
\newcommand{\CCpp}{C/C++\xspace}
\newcommand{\NonOptimizing}{非最適化\xspace}
\newcommand{\Optimizing}{最適化\xspace}
\newcommand{\ARMMode}{ARMモード\xspace}
\newcommand{\ThumbMode}{Thumbモード\xspace}
\newcommand{\ThumbTwoMode}{Thumb-2モード\xspace}

\newcommand{\DataProcessingInstructionsFootNote}{この命令は \q{データプロセス命令} とも呼ばれます}

% for .bib files
\newcommand{\AlsoAvailableAs}{以下で利用可能\xspace}

% section names
\newcommand{\ShiftsSectionName}{シフト}
\newcommand{\HelloWorldSectionName}{ハローワールド!}
\newcommand{\SwitchCaseDefaultSectionName}{switch()/case/default}
\newcommand{\PrintfSeveralArgumentsSectionName}{printf() 引数を取って}
\newcommand{\BitfieldsChapter}{特定のビットを操作する}
\newcommand{\ArithOptimizations}{算術命令を他の命令に置換する}

\newcommand{\FPUChapterName}{フローティングポイントユニット}
\newcommand{\MoreAboutStrings}{文字列に関する加筆}
\newcommand{\DivisionByMultSectionName}{乗法を使用した除算}
\newcommand{\Answer}{解答}
\newcommand{\WhatThisCodeDoes}{このコードは何をしている}
\newcommand{\WorkingWithFloatAsWithStructSubSubSectionName}{フロート型のデータを構造体として扱う}

\newcommand{\MinesweeperWinXPExampleChapterName}{マインスイーパー (Windows XP)}
\newcommand{\StructurePackingSectionName}{フィールドを構造体にパッキングする}
\newcommand{\StructuresChapterName}{構造体}
\newcommand{\PICcode}{位置独立コード}
\newcommand{\CapitalPICcode}{位置独立コード}
\newcommand{\Loops}{ループ}

% C
\newcommand{\PostIncrement}{後置インクリメント}
\newcommand{\PostDecrement}{後置デクリメント}
\newcommand{\PreIncrement}{前置インクリメント}
\newcommand{\PreDecrement}{前置デクリメント}

% MIPS
\newcommand{\GlobalPointer}{グローバルポインタ}

\newcommand{\garbage}{ガーベッジ}
\newcommand{\IntelSyntax}{インテル構文}
\newcommand{\ATTSyntax}{AT\&T構文}
\newcommand{\randomNoise}{ランダムノイズ}
\newcommand{\Example}{例}
\newcommand{\argument}{引数}
\newcommand{\MarkedInIDAAs}{\IDA にマークする}
\newcommand{\stepover}{ステップオーバー}
\newcommand{\ShortHotKeyCheatsheet}{ホットキーチートシート}

\newcommand{\assemblyOutput}{アセンブリ出力}

% ML prefix is for multi-lingual words and sentences:
\newcommand{\MLHeap}{ヒープ}
\newcommand{\MLStack}{スタック}
\newcommand{\MLStackOverflow}{スタックオーバーフロー}
\newcommand{\MLStartOfHeap}{ヒープの開始}
\newcommand{\MLStartOfStack}{スタックの開始}
\newcommand{\MLinputA}{入力A}
\newcommand{\MLinputB}{入力B}
\newcommand{\MLoutput}{出力}
\newcommand{\SoftwareCracking}{\JAph{}}

\newcommand{\EMAILPRI}{\href{https://yurichev.com/contact.html}{my emails}}
\newcommand{\EMAILS}{\href{https://yurichev.com/contact.html}{my emails}}

}
\IT{\newcommand{\AcronymsUsed}{Acronimi utilizzati}

\newcommand{\TitleRE}{Reverse Engineering per Principianti}

\newcommand{\TitleUAL}{Capire il linguaggio Assembly}

\newcommand{\AUTHOR}{Dennis Yurichev}

\newcommand{\figname}{fig.\xspace}
\newcommand{\listingname}{listato.\xspace}
\newcommand{\Sourcecode}{Codice sorgente\xspace}
\newcommand{\Seealso}{Vedi anche\xspace}
\newcommand{\tableheader}{\headercolor{} offset & \headercolor{} descrizione}

% instructions descriptions
\newcommand{\ASRdesc}{scorrimento aritmetico a destra}

% x86 registers tables
\newcommand{\RegHeaderTop}{ \multicolumn{8}{ | c | }{ Numero byte } }
% TODO: non-overlapping color!
\newcommand{\RegHeader}{ 7°  & 6°  & 5°  & 4°  & 3°  & 2°  & 1°  & 0 }
\newcommand{\ReturnAddress}{Indirizzo di Ritorno}

\newcommand{\localVariable}{variabile locale}

\newcommand{\savedValueOf}{valore memorizzato di}
% for index
\newcommand{\GrepUsage}{uso di grep}
\newcommand{\SyntacticSugar}{Syntactic Sugar}
\newcommand{\CompilerAnomaly}{Anomalie del compilatore}
\newcommand{\CLanguageElements}{Elementi del linguaggio C}
\newcommand{\CStandardLibrary}{Libreria C standard}
\newcommand{\Instructions}{Istruzioni}
\newcommand{\Pseudoinstructions}{Pseudo-istruzioni}
\newcommand{\Prefixes}{Prefissi}

\newcommand{\Flags}{Flag}
\newcommand{\Registers}{Registri}
\newcommand{\registers}{registri}
\newcommand{\Stack}{Stack}
\newcommand{\Recursion}{Ricorsione}
\newcommand{\RAM}{RAM}
\newcommand{\ROM}{ROM}
\newcommand{\Pointers}{Puntatori}
\newcommand{\BufferOverflow}{Buffer Overflow}

\newcommand{\Conclusion}{Conclusione}

\newcommand{\Exercise}{Esercizio}
\newcommand{\Exercises}{Esercizi\xspace}
\newcommand{\Arrays}{Array}
\newcommand{\Cpp}{C++\xspace}
\newcommand{\CCpp}{C/C++\xspace}
\newcommand{\NonOptimizing}{Senza ottimizzazione\xspace}
\newcommand{\Optimizing}{Con ottimizzazione\xspace}
\newcommand{\ARMMode}{Modalità ARM\xspace}
\newcommand{\ThumbMode}{Modalità Thumb\xspace}
\newcommand{\ThumbTwoMode}{Modalità Thumb-2\xspace}

\newcommand{\DataProcessingInstructionsFootNote}{\ITph{}}

% for .bib files
\newcommand{\AlsoAvailableAs}{\ITph{}}

% section names
\newcommand{\ShiftsSectionName}{\ITph{}}
\newcommand{\HelloWorldSectionName}{Hello, world!}
\newcommand{\SwitchCaseDefaultSectionName}{switch()/case/default}
\newcommand{\PrintfSeveralArgumentsSectionName}{printf() con più argomenti}
\newcommand{\BitfieldsChapter}{Manipolando dei bit speifici}
\newcommand{\ArithOptimizations}{Sostituzione di istruzioni aritmetiche con altre}
\newcommand{\FPUChapterName}{Floating-point unit}
\newcommand{\MoreAboutStrings}{Maggiori informazioni sulle stringhe}
\newcommand{\DivisionByMultSectionName}{Divisione utilizzando la moltiplicazione}
\newcommand{\Answer}{Risposta}
\newcommand{\WhatThisCodeDoes}{Cosa fa qusto codice}
\newcommand{\WorkingWithFloatAsWithStructSubSubSectionName}{Utilizzare i tipi di dato float come una struttura}

\newcommand{\MinesweeperWinXPExampleChapterName}{\ITph{} (Windows XP)}
\newcommand{\StructurePackingSectionName}{Organizzazione dei campi in una struttura}
\newcommand{\StructuresChapterName}{Strutture}
\newcommand{\PICcode}{codice indipendente dalla posizione}
\newcommand{\CapitalPICcode}{Codice indipendente dalla posizione}
\newcommand{\Loops}{Cicli}

% C
\newcommand{\PostIncrement}{Post-incremento}
\newcommand{\PostDecrement}{Post-decremento}
\newcommand{\PreIncrement}{Pre-incremento}
\newcommand{\PreDecrement}{Pre-decremento}

% MIPS
\newcommand{\GlobalPointer}{Puntatore Globale}

\newcommand{\garbage}{garbage}
\newcommand{\IntelSyntax}{Sintassi Intel}
\newcommand{\ATTSyntax}{Sintassi AT\&T}
\newcommand{\randomNoise}{rumore casuale}
\newcommand{\Example}{Esempio}
\newcommand{\argument}{argomento}
\newcommand{\MarkedInIDAAs}{marcato in IDA come}
\newcommand{\stepover}{step over}
\newcommand{\ShortHotKeyCheatsheet}{Elenco delle scorciatoie da tastiera}

\newcommand{\assemblyOutput}{risultato dell'assembly}

% ML prefix is for multi-lingual words and sentences:
\newcommand{\MLHeap}{Heap}
\newcommand{\MLStack}{Stack}
\newcommand{\MLStackOverflow}{Stack overflow}
\newcommand{\MLStartOfHeap}{Inizio dell'heap}
\newcommand{\MLStartOfStack}{Inizio dello stack}
\newcommand{\MLinputA}{input A}
\newcommand{\MLinputB}{input A}
\newcommand{\MLoutput}{output}
\newcommand{\SoftwareCracking}{\ITph{}}

\newcommand{\EMAILPRI}{\href{https://yurichev.com/contact.html}{my emails}}
\newcommand{\EMAILS}{\href{https://yurichev.com/contact.html}{my emails}}

}
\PL{\newcommand{\AcronymsUsed}{Użyte akronimy}

\newcommand{\TitleRE}{Inżynieria wsteczna dla początkujących}

\newcommand{\TitleUAL}{Rozumienie języka maszynowego}

\newcommand{\AUTHOR}{Dennis Yurichev}

\newcommand{\figname}{ilustr.\xspace}
\newcommand{\listingname}{listing.\xspace}
\newcommand{\Sourcecode}{Kod źródłowy\xspace}
\newcommand{\Seealso}{Zobacz również\xspace}
\newcommand{\tableheader}{\headercolor{} offset & \headercolor{} description}
% instructions descriptions
\newcommand{\ASRdesc}{arithmetic shift right}

% x86 registers tables
\newcommand{\RegHeaderTop}{ \multicolumn{8}{ | c | }{ Number bajtu: } }
% TODO: non-overlapping color!
\newcommand{\RegHeader}{ 7 & 6 & 5 & 4 & 3 & 2 & 1 & 0 }
\newcommand{\ReturnAddress}{adres powrotu}

\newcommand{\localVariable}{zmienna lokalna}

\newcommand{\savedValueOf}{odłożona wartość }

% for index
\newcommand{\GrepUsage}{grep usage}
\newcommand{\SyntacticSugar}{Syntactic Sugar}
\newcommand{\CompilerAnomaly}{Compiler's anomalies}
\newcommand{\CLanguageElements}{C language elements}
\newcommand{\CStandardLibrary}{C standard library}
\newcommand{\Instructions}{Instructions}
\newcommand{\Pseudoinstructions}{Pseudoinstructions}
\newcommand{\Prefixes}{Prefixes}

\newcommand{\Flags}{Flags}
\newcommand{\Registers}{Rejestry}
\newcommand{\registers}{rejestry}
\newcommand{\Stack}{Stos}
\newcommand{\Recursion}{Rekurencja}
\newcommand{\RAM}{RAM}
\newcommand{\ROM}{ROM}
\newcommand{\Pointers}{Pointers}
\newcommand{\BufferOverflow}{Buffer Overflow}

% DE: also "Zusammenfassung"
\newcommand{\Conclusion}{Wnioski}

\newcommand{\Exercise}{Ćwiczenie}
\newcommand{\Exercises}{Ćwiczenia}
\newcommand{\Arrays}{Arrays}
\newcommand{\Cpp}{C++\xspace}
\newcommand{\CCpp}{C/C++\xspace}
\newcommand{\NonOptimizing}{Nieoptymalizujący\xspace}
\newcommand{\Optimizing}{Optymalizujący\xspace}
\newcommand{\ARMMode}{tryb ARM\xspace}
\newcommand{\ThumbMode}{tryb Thumb\xspace}
\newcommand{\ThumbTwoMode}{tryb Thumb-2\xspace}

\newcommand{\DataProcessingInstructionsFootNote}{These instructions are also called \q{data processing instructions}}
% for .bib files
\newcommand{\AlsoAvailableAs}{Dostęp także przez\xspace}

% section names
\newcommand{\ShiftsSectionName}{Shifts}
\newcommand{\HelloWorldSectionName}{Hello, world!}
\newcommand{\SwitchCaseDefaultSectionName}{switch()/case/default}
\newcommand{\PrintfSeveralArgumentsSectionName}{printf() z wieloma argumentami}
\newcommand{\BitfieldsChapter}{Manipulating specific bit(s)}
\newcommand{\ArithOptimizations}{Replacing arithmetic instructions to other ones}
\newcommand{\FPUChapterName}{Floating-point unit}
\newcommand{\MoreAboutStrings}{More about strings}
\newcommand{\DivisionByMultSectionName}{Division using multiplication}
\newcommand{\Answer}{Answer}
\newcommand{\WhatThisCodeDoes}{What does this code do}
\newcommand{\WorkingWithFloatAsWithStructSubSubSectionName}{Handling float data type as a structure}

\newcommand{\MinesweeperWinXPExampleChapterName}{Minesweeper (Windows XP)}
\newcommand{\StructurePackingSectionName}{Fields packing in structure}
\newcommand{\StructuresChapterName}{Structures}
\newcommand{\PICcode}{position-independent code}
\newcommand{\CapitalPICcode}{Position-independent code}
\newcommand{\Loops}{Loops}

% C
\newcommand{\PostIncrement}{Post-increment}
\newcommand{\PostDecrement}{Post-decrement}
\newcommand{\PreIncrement}{Pre-increment}
\newcommand{\PreDecrement}{Pre-decrement}

% MIPS
\newcommand{\GlobalPointer}{Global Pointer}

\newcommand{\garbage}{śmieci}
\newcommand{\IntelSyntax}{składnia Intela}
\newcommand{\ATTSyntax}{składnia AT\&T}
\newcommand{\randomNoise}{random noise}
\newcommand{\Example}{Example}
\newcommand{\argument}{argument}
\newcommand{\MarkedInIDAAs}{oznaczony w programie \IDA jako}
\newcommand{\MarkedInIDAAsFem}{oznaczona w programie \IDA jako}
\newcommand{\stepover}{step over}
\newcommand{\ShortHotKeyCheatsheet}{Hot-keys cheatsheet}

\newcommand{\assemblyOutput}{wyjście w asemblerze}

% ML prefix is for multi-lingual words and sentences:
\newcommand{\MLHeap}{Sterta}
\newcommand{\MLStack}{Stos}
\newcommand{\MLStackOverflow}{Stack overflow}
\newcommand{\MLStartOfHeap}{Początek sterty}
\newcommand{\MLStartOfStack}{Początek stosu}
\newcommand{\MLinputA}{input A}
\newcommand{\MLinputB}{input B}
\newcommand{\MLoutput}{output}
\newcommand{\SoftwareCracking}{Software cracking}

\newcommand{\EMAILPRI}{\href{https://yurichev.com/contact.html}{my emails}}
\newcommand{\EMAILS}{\href{https://yurichev.com/contact.html}{my emails}}

}
\ES{\newcommand{\AcronymsUsed}{Acr\'onimos utilizados}

\newcommand{\TitleRE}{Ingenier\'ia Inversa para Principiantes}

% TBT?
\newcommand{\TitleUAL}{Understanding Assembly Language}

\newcommand{\AUTHOR}{Dennis Yurichev}

\newcommand{\figname}{fig.\xspace}
\newcommand{\listingname}{listado.\xspace}
\newcommand{\Sourcecode}{C\'odigo fuente\xspace}
\newcommand{\Seealso}{V\'ease tambi\'en\xspace}
\newcommand{\tableheader}{\headercolor{} offset & \headercolor{} descripci\'on }
% instructions descriptions
\newcommand{\ASRdesc}{desplazamiento aritm\'etico a la derecha}

% x86 registers tables
\newcommand{\RegHeaderTop}{ \multicolumn{8}{ | c | }{ \ESph{} } }
% TODO: non-overlapping color!
\newcommand{\RegHeader}{ 7mo & 6to & 5to & 4to & 3ro & 2do & 1ro & 0 }
\newcommand{\ReturnAddress}{ Direcci\'on de Retorno }

\newcommand{\localVariable}{\ESph{}}

\newcommand{\savedValueOf}{\ESph{}}

% for index
\newcommand{\GrepUsage}{Uso de grep}
\newcommand{\SyntacticSugar}{Azúcar sintáctica}
\newcommand{\CompilerAnomaly}{Anomalías del compilador}
\newcommand{\CLanguageElements}{Elementos del lenguaje C}
\newcommand{\CStandardLibrary}{Librería estándar C}
\newcommand{\Instructions}{Instrucciones}
\newcommand{\Pseudoinstructions}{Pseudo-instrucciones}
\newcommand{\Prefixes}{Prefijos}

\newcommand{\Flags}{Flags}
\newcommand{\Registers}{Registros}
\newcommand{\registers}{registros}
\newcommand{\Stack}{Pila}
\newcommand{\Recursion}{Recursión}
\newcommand{\RAM}{RAM}
\newcommand{\ROM}{ROM}
\newcommand{\Pointers}{Apuntadores}
\newcommand{\BufferOverflow}{Desbordamiento de buffer}

% DE: also "Zusammenfassung"
\newcommand{\Conclusion}{Conclusión}

\newcommand{\Exercise}{Ejercicio\xspace}
\newcommand{\Exercises}{Ejercicios\xspace}
\newcommand{\Arrays}{Matriz}
\newcommand{\Cpp}{C++}
\newcommand{\CCpp}{C/C++}
\newcommand{\NonOptimizing}{Sin optimización\xspace}
\newcommand{\Optimizing}{Con optimización\xspace}
\newcommand{\ARMMode}{Modo ARM\xspace}
\newcommand{\ThumbMode}{Modo Thumb\xspace}
\newcommand{\ThumbTwoMode}{Modo Thumb-2\xspace}

\newcommand{\DataProcessingInstructionsFootNote}{Estas instrucciones también son llamadas \q{instrucciones de procesamiento de datos}}

% for .bib files
\newcommand{\AlsoAvailableAs}{También disponible como\xspace}

% section names
\newcommand{\ShiftsSectionName}{Desplazamientos}
\newcommand{\HelloWorldSectionName}{!`Hola, mundo!}
\newcommand{\SwitchCaseDefaultSectionName}{switch()/case/default}
\newcommand{\PrintfSeveralArgumentsSectionName}{printf() con varios argumentos}
\newcommand{\BitfieldsChapter}{Manipulando bit(s) específicos}
\newcommand{\ArithOptimizations}{Substituición de instrucciones aritméticas por otras}
	
\newcommand{\FPUChapterName}{Unidad de punto flotante}
\newcommand{\MoreAboutStrings}{\ESph{}}
\newcommand{\DivisionByMultSectionName}{División entre 9}
\newcommand{\Answer}{Respuesta}
\newcommand{\WhatThisCodeDoes}{?`Qu\'e hace este código?}
\newcommand{\WorkingWithFloatAsWithStructSubSubSectionName}{Trabajando con el tipo float como una estructura}

\newcommand{\MinesweeperWinXPExampleChapterName}{Buscaminas (Windows XP)}
\newcommand{\StructurePackingSectionName}{Organización de campos en la estructura}
\newcommand{\StructuresChapterName}{Estructuras}
\newcommand{\PICcode}{código independiente de la posición}
\newcommand{\CapitalPICcode}{Código independiente de lá posición}
\newcommand{\Loops}{Bucles}

% C
\newcommand{\PostIncrement}{Post-incremento}
\newcommand{\PostDecrement}{Post-decremento}
\newcommand{\PreIncrement}{Pre-incremento}
\newcommand{\PreDecrement}{Pre-decremento}

% MIPS
\newcommand{\GlobalPointer}{Apuntador Global}

\newcommand{\garbage}{Basura}
\newcommand{\IntelSyntax}{Sintaxis Intel}
\newcommand{\ATTSyntax}{Sintaxis AT\&T}
\newcommand{\randomNoise}{Ruido aleatorio}
\newcommand{\Example}{Ejemplo}
\newcommand{\argument}{argumento}
\newcommand{\MarkedInIDAAs}{Marcado en \IDA como}
\newcommand{\stepover}{pasar por encima}
\newcommand{\ShortHotKeyCheatsheet}{Cheatsheet de teclas de acceso rápido}

\newcommand{\assemblyOutput}{salida del ensamblador}

% ML prefix is for multi-lingual words and sentences:
\newcommand{\MLHeap}{\ESph{}}
\newcommand{\MLStack}{\ESph{}}
\newcommand{\MLStackOverflow}{\ESph{}}
\newcommand{\MLStartOfHeap}{\ESph{}}
\newcommand{\MLStartOfStack}{\ESph{}}
\newcommand{\MLinputA}{\ESph{}}
\newcommand{\MLinputB}{\ESph{}}
\newcommand{\MLoutput}{\ESph{}}
\newcommand{\SoftwareCracking}{\ESph{}}

\newcommand{\EMAILPRI}{\href{https://yurichev.com/contact.html}{my emails}}
\newcommand{\EMAILS}{\href{https://yurichev.com/contact.html}{my emails}}
}
\NL{\newcommand{\AcronymsUsed}{Gebruikte afkortingen}

\newcommand{\TitleRE}{\NLph{}}

\newcommand{\TitleUAL}{\NLph{}}

\newcommand{\AUTHOR}{Dennis Yurichev}

% x86 registers tables
\newcommand{\RegHeaderTop}{ \multicolumn{8}{ | c | }{ \NLph{} } }
% TODO: non-overlapping color!
% TBT
\newcommand{\RegHeader}{ 7th & 6th & 5th & 4th & 3rd & 2nd & 1st & 0th }

\newcommand{\ReturnAddress}{Return Adres}
\newcommand{\SoftwareCracking}{\NLph{}}

\newcommand{\EMAILPRI}{\href{https://yurichev.com/contact.html}{my emails}}
\newcommand{\EMAILS}{\href{https://yurichev.com/contact.html}{my emails}}

}
\TR{\newcommand{\TitleRE}{Acemiler için Tersine Mühendislik}
\newcommand{\SoftwareCracking}{\TRph{}}

%\newcommand{\EMAILPRI}{<first\_name @ last\_name . com>}
%\newcommand{\EMAILS}{<first\_name @ last\_name . com> / <first\_name . last\_name @ gmail . com>}
%\newcommand{\EMAILPRI}{<dennis@yurichev.com>}
%\newcommand{\EMAILS}{<dennis@yurichev.com> / <dennis.yurichev@gmail.com>}
\newcommand{\EMAILPRI}{<book@beginners.re>}
\newcommand{\EMAILS}{<book@beginners.re>}

\newcommand{\EMAILPRI}{\href{https://yurichev.com/contact.html}{my emails}}
\newcommand{\EMAILS}{\href{https://yurichev.com/contact.html}{my emails}}

}
\PTBR{\newcommand{\ReturnAddress}{Endereço de retorno}

\newcommand{\localVariable}{Variável local}

\newcommand{\savedValueOf}{Valor salvo de}

% for index
\newcommand{\GrepUsage}{Uso do grep}
\newcommand{\SyntacticSugar}{Açúcar sintático}
\newcommand{\CompilerAnomaly}{Anomalias do compilador}
\newcommand{\CLanguageElements}{Elementos da linguagem C}
\newcommand{\CStandardLibrary}{Biblioteca padrão C}
\newcommand{\Instructions}{Instruções}
\newcommand{\Pseudoinstructions}{Pseudo-instruções}
\newcommand{\Prefixes}{Prefixos}

\newcommand{\Flags}{Flags}
\newcommand{\Registers}{Registradores}
\newcommand{\registers}{registradores}
\newcommand{\Stack}{Pilha}
\newcommand{\Recursion}{Recursividade}
\newcommand{\RAM}{RAM}
\newcommand{\ROM}{ROM}
\newcommand{\Pointers}{Ponteiros}
\newcommand{\BufferOverflow}{Buffer Overflow}

% DE: also "Zusammenfassung"
\newcommand{\Conclusion}{Conclusão}

\newcommand{\Exercise}{Exercício}
\newcommand{\Exercises}{Exercícios}
\newcommand{\Arrays}{Matriz}
\newcommand{\Cpp}{C++\xspace}
\newcommand{\CCpp}{C/C++\xspace}
\newcommand{\NonOptimizing}{Sem otimização\xspace}
\newcommand{\Optimizing}{Com otimização\xspace}
\newcommand{\ARMMode}{Modo ARM\xspace}
\newcommand{\ThumbMode}{Modo Thumb\xspace}
\newcommand{\ThumbTwoMode}{Modo Thumb-2\xspace}

\newcommand{\DataProcessingInstructionsFootNote}{Estas intruções também são chamadas \q{data processing instructions}}

% for .bib files
\newcommand{\AlsoAvailableAs}{Também disponível como\xspace}

% section names
\newcommand{\ShiftsSectionName}{Shifts}
\newcommand{\SwitchCaseDefaultSectionName}{switch()/case/default}
\newcommand{\BitfieldsChapter}{Manipulando bit(s) específicos}
\newcommand{\ArithOptimizations}{Substituição de instruções aritiméticas por outras}
\newcommand{\FPUChapterName}{Unidade de Ponto flutuante}
\newcommand{\DivisionByMultSectionName}{Divisão por 9}
\newcommand{\Answer}{Responda}
\newcommand{\WhatThisCodeDoes}{O que este código faz}
\newcommand{\WorkingWithFloatAsWithStructSubSubSectionName}{Trabalhando com o tipo float como uma estrutura}

\newcommand{\MinesweeperWinXPExampleChapterName}{Campo minado (Windows XP)}
\newcommand{\StructurePackingSectionName}{Organização de campos na estrutura}
\newcommand{\StructuresChapterName}{Estruturas}
\newcommand{\PICcode}{código independente de posição}
\newcommand{\CapitalPICcode}{Código independente de posição}
\newcommand{\Loops}{Laços}

% C
\newcommand{\PostIncrement}{Pós-incremento}
\newcommand{\PostDecrement}{Pós-decremento}
\newcommand{\PreIncrement}{Pré-incremento}
\newcommand{\PreDecrement}{Pré-decremento}

% MIPS
\newcommand{\GlobalPointer}{Ponteiro Global}

\newcommand{\garbage}{Lixo}
\newcommand{\IntelSyntax}{Sintaxe Intel}
\newcommand{\ATTSyntax}{Sintaxe AT\&T}
\newcommand{\randomNoise}{Ruído aleatório}
\newcommand{\Example}{Exemplo}
\newcommand{\argument}{argumento}
\newcommand{\MarkedInIDAAs}{Marcado no \IDA como}
\newcommand{\stepover}{passar por cima}
\newcommand{\ShortHotKeyCheatsheet}{Cheatsheet de teclas de atalho}

\newcommand{\assemblyOutput}{saída do assembly}

% ML prefix is for multi-lingual words and sentences:
\newcommand{\MLHeap}{Heap}
\newcommand{\MLStack}{Pilha}
\newcommand{\MLStackOverflow}{\PTBRph{}}
\newcommand{\MLStartOfHeap}{começo da heap}
\newcommand{\MLStartOfStack}{começo da pilha}
\newcommand{\SoftwareCracking}{\PTBRph{}}

\newcommand{\EMAILPRI}{\href{https://yurichev.com/contact.html}{my emails}}
\newcommand{\EMAILS}{\href{https://yurichev.com/contact.html}{my emails}}

}

\ifdefined\UAL
\newcommand{\TitleMain}{\TitleUAL}
\newcommand{\TitleAux}{\TitleRE}
\else
\newcommand{\TitleMain}{\TitleRE}
\newcommand{\TitleAux}{\TitleUAL}
\fi

\newcommand{\CURPATH}{DUMMY}

\EN{\newglossaryentry{tail call}
{
  name=tail call,
  description={It is when the compiler (or interpreter) transforms the recursion
  (\emph{tail recursion}) into an iteration for efficiency}
}

\newglossaryentry{endianness}
{
  name=endianness,
  description={Byte order}
}

\newglossaryentry{caller}
{
  name=caller,
  description={A function calling another}
}

\newglossaryentry{callee}
{
  name=callee,
  description={A function being called by another}
}

\newglossaryentry{debuggee}
{
  name=debuggee,
  description={A program being debugged}
}

\newglossaryentry{leaf function}
{
  name=leaf function,
  description={A function which does not call any other function}
}

\newglossaryentry{link register}
{
  name=link register,
  description=(RISC) {A register where the return address is usually stored.
  This makes it possible to call leaf functions without using the stack, i.e., faster}
}

\newglossaryentry{anti-pattern}
{
  name=anti-pattern,
  description={Generally considered as bad practice}
}

\newglossaryentry{stack pointer}
{
  name=stack pointer,
  description={A register pointing to a place in the stack}
}

\newglossaryentry{decrement}
{
  name=decrement,
  description={Decrease by 1}
}

\newglossaryentry{increment}
{
  name=increment,
  description={Increase by 1}
}

\newglossaryentry{loop unwinding}
{
  name=loop unwinding,
  description={It is when a compiler, instead of generating loop code for $n$ iterations, generates just $n$ copies of the
  loop body, in order to get rid of the instructions for loop maintenance}
}

\newglossaryentry{register allocator}
{
  name=register allocator,
  description=
  {The part of the compiler that assigns CPU registers to local variables}
}

\newglossaryentry{quotient}
{
  name=quotient,
  description={Division result}
}

\newglossaryentry{product}
{
  name=product,
  description={Multiplication result}
}

\newglossaryentry{NOP}
{
  name=NOP,
  description={\q{no operation}, idle instruction}
}

\newglossaryentry{POKE}
{
  name=POKE,
  description={BASIC language instruction for writing a byte at a specific address}
}

\newglossaryentry{keygenme}
{
  name=keygenme,
  description={A program which imitates software protection,
  for which one needs to make a key/license generator}
} % TODO clarify: A software which generate key/license value to bypass sotfware protection?

\newglossaryentry{dongle}
{
  name=dongle,
  description={Dongle is a small piece of hardware connected to LPT printer port (in past) or to USB}
}

\newglossaryentry{thunk function}
{
  name=thunk function,
  description={Tiny function with a single role: call another function}
}

\newglossaryentry{user mode}
{
  name=user mode,
  description={A restricted CPU mode in which it all application software code is executed. cf. \gls{kernel mode}}
}

\newglossaryentry{kernel mode}
{
  name=kernel mode,
  description={A restrictions-free CPU mode in which the OS kernel and drivers execute. cf. \gls{user mode}}
}

\newglossaryentry{Windows NT}
{
  name=Windows NT,
  description={Windows NT, 2000, XP, Vista, 7, 8, 10}
}

\newglossaryentry{atomic operation}
{
  name=atomic operation,
  description={
  \q{$\alpha{}\tau{}o\mu{}o\varsigma{}$}
  %\q{atomic}
  stands for \q{indivisible} in Greek, so an atomic operation is guaranteed not
  to be interrupted by other threads}
}

% to be proofreaded (begin)
\newglossaryentry{NaN}
{
  name=NaN,
  description={not a number: 
  	a special cases for floating point numbers, usually signaling about errors}
}

\newglossaryentry{basic block}
{
  name=basic block,
  description={
	a group of 
	instructions that do not have jump/branch instructions, and also don't have
	jumps inside the block from the outside.
	In \IDA it looks just like as a list of instructions without empty lines}
}

\newglossaryentry{NEON}
{
  name=NEON,
  description={\ac{AKA} \q{Advanced SIMD}---\ac{SIMD} from ARM}
}

\newglossaryentry{reverse engineering}
{
  name=reverse engineering,
  description={act of understanding how the thing works, sometimes in order to clone it}
}

\newglossaryentry{compiler intrinsic}
{
  name=compiler intrinsic,
  description={A function specific to a compiler which is not an usual library function.
	The compiler generates a specific machine code instead of a call to it.
	Often, it's a pseudofunction for a specific \ac{CPU} instruction. Read more:}
 (\myref{sec:compiler_intrinsic})
}

\newglossaryentry{heap}
{
  name=heap,
  description={usually, a big chunk of memory provided by the \ac{OS} so that applications can divide it by themselves as they wish.
  malloc()/free() work with the heap}
}

\newglossaryentry{name mangling}
{
  name=name mangling,
  description={used at least in \Cpp, where the compiler needs to encode the name of class, method and argument types in one string,
  which will become the internal name of the function. You can read more about it here: \myref{namemangling}}
}

\newglossaryentry{xoring}
{
  name=xoring,
  description={often used in the English language, which implying applying the \ac{XOR} operation}
}

\newglossaryentry{security cookie}
{
  name=security cookie,
  description={A random value, different at each execution. You can read more about it here:
  \myref{subsec:BO_protection}}
}

\newglossaryentry{tracer}
{
  name=tracer,
  description={My own simple debugging tool. You can read more about it here: \myref{tracer}}
}

\newglossaryentry{GiB}
{
  name=GiB,
  description={Gibibyte: $2^{30}$ or 1024 mebibytes or 1073741824 bytes}
}

\newglossaryentry{CP/M}
{
  name=CP/M,
  description={Control Program for Microcomputers: a very basic disk \ac{OS} used before MS-DOS}
}

\newglossaryentry{stack frame}
{
  name=stack frame,
  description={A part of the stack that contains information specific to the current function:
  local variables, function arguments, \ac{RA}, etc.}
}

\newglossaryentry{jump offset}
{
  name=jump offset,
  description={a part of the JMP or Jcc instruction's opcode, 
  to be added to the address
  of the next instruction, and this is how the new \ac{PC} is calculated. May be negative as well}
}

\newglossaryentry{integral type}
{
  name=integral data type,
  description={usual numbers, but not a real ones. may be used for passing variables of boolean data type and enumerations}
}

\newglossaryentry{real number}
{
  name={real number},
  description={numbers which may contain a dot. this is \Tfloat and \Tdouble in \CCpp}
}

\newglossaryentry{PDB}
{
  name=PDB,
  description={(Win32) Debugging information file, usually just function names, but sometimes also function
  arguments and local variables names}
}

\newglossaryentry{NTAPI}
{
  name=NTAPI,
  description={\ac{API} available only in the Windows NT line.  Largely not documented by Microsoft}
}

\newglossaryentry{stdout}
{
  name=stdout,
  description={standard output}
}

\newglossaryentry{word}
{
  name=word,
  description={data type fitting in \ac{GPR}.
  In the computers older than PCs, 
  the memory size was often measured in words rather than bytes.}
}

\newglossaryentry{arithmetic mean}
{
  name=arithmetic mean,
  description={a sum of all values divided by their count}
}
\newglossaryentry{padding}
{
  name=padding,
  description={
  \emph{Padding} in English language means to stuff a pillow with something
  to give it a desired (bigger) form.
  In computer science, padding means to add more bytes to a block so it will have desired size, like $2^n$ bytes.}
}

}
\RU{\newglossaryentry{tail call}
{
  name=хвостовая рекурсия,
  description={Это когда компилятор или интерпретатор превращает рекурсию 
  (с которой возможно это проделать, т.е. \emph{хвостовую}) в итерацию для эффективности}
}

\newglossaryentry{endianness}
{
  name=endianness,
  description={Порядок байт}
}

\newglossaryentry{caller}
{
  name=caller,
  description={Функция вызывающая другую функцию}
}

\newglossaryentry{callee}
{
  name=callee,
  description={Вызываемая функция}
}

\newglossaryentry{debuggee}
{
  name=debuggee,
  description={Отлаживаемая программа}
}

\newglossaryentry{leaf function}
{
  name=leaf function,
  description={Функция не вызывающая больше никаких функций}
}

\newglossaryentry{link register}
{
  name=link register,
  description=(RISC) {Регистр в котором обычно записан адрес возврата.
  Это позволяет вызывать leaf-функции без использования стека, т.е. быстрее}
}

\newglossaryentry{anti-pattern}
{
  name=anti-pattern,
  description={Нечто широко известное как плохое решение}
}

\newglossaryentry{stack pointer}
{
  name=указатель стека,
  description={Регистр указывающий на место в стеке}
}

\newglossaryentry{decrement}
{
  name=декремент,
  description={Уменьшение на 1}
}

\newglossaryentry{increment}
{
  name=инкремент,
  description={Увеличение на 1}
}

\newglossaryentry{loop unwinding}
{
  name=loop unwinding,
  description={Это когда вместо организации цикла на $n$ итераций, компилятор генерирует $n$ копий тела
  цикла, для экономии на инструкциях, обеспечивающих сам цикл}
}

\newglossaryentry{register allocator}
{
  name=register allocator,
  description={Функция компилятора распределяющая локальные переменные по регистрам процессора}
}

\newglossaryentry{quotient}
{
  name=частное,
  description={Результат деления}
}

\newglossaryentry{product}
{
  name={произведение},
  description={Результат умножения}
}

\newglossaryentry{NOP}
{
  name=NOP,
  description={\q{no operation}, холостая инструкция}
}

\newglossaryentry{POKE}
{
  name=POKE,
  description={Инструкция языка BASIC записывающая байт по определенному адресу}
}

\newglossaryentry{keygenme}
{
  name=keygenme,
  description={Программа, имитирующая защиту вымышленной программы, для которой нужно сделать 
  генератор ключей/лицензий}
} % TODO clarify: A software which generate key/license value to bypass sotfware protection?

\newglossaryentry{dongle}
{
  name=dongle,
  description={Небольшое устройство подключаемое к LPT-порту для принтера (в прошлом) или к USB}
}

\newglossaryentry{thunk function}
{
  name=thunk function,
  description={Крохотная функция делающая только одно: вызывающая другую функцию}
}

\newglossaryentry{user mode}
{
  name=user mode,
  description={Режим CPU с ограниченными возможностями в котором он исполняет прикладное ПО. ср.}
}

\newglossaryentry{kernel mode}
{
  name=kernel mode,
  description={Режим CPU с неограниченными возможностями в котором он исполняет ядро OS и драйвера. ср.}
}

\newglossaryentry{Windows NT}
{
  name=Windows NT,
  description={Windows NT, 2000, XP, Vista, 7, 8, 10}
}

\newglossaryentry{atomic operation}
{
  name=atomic operation,
  description={
  \q{$\alpha{}\tau{}o\mu{}o\varsigma{}$}
  %\q{atomic}
  означает \q{неделимый} в греческом языке, так что атомарная операция ---
  это операция которая гарантированно не будет прервана другими тредами}
}

% to be proofreaded (begin)
\newglossaryentry{NaN}
{
  name=NaN,
  description={
  	не число: специальные случаи чисел с плавающей запятой, 
  	обычно сигнализирующие об ошибках
  }
}

\newglossaryentry{basic block}
{
  name=basic block,
  description={
  	группа инструкций, не имеющая инструкций переходов,
	а также не имеющая переходов в середину блока извне.
	В \IDA он выглядит как просто список инструкций без строк-разрывов
  }
}

\newglossaryentry{NEON}
{
  name=NEON,
  description={\ac{AKA} \q{Advanced SIMD} --- от ARM}
}

\newglossaryentry{reverse engineering}
{
  name=reverse engineering,
  description={процесс понимания как устроена некая вещь, иногда, с целью клонирования оной}
}

\newglossaryentry{compiler intrinsic}
{
  name=compiler intrinsic,
  description={Специфичная для компилятора функция не являющаяся обычной библиотечной функцией.
	Компилятор вместо её вызова генерирует определенный машинный код.
	Нередко, это псевдофункции для определенной инструкции \ac{CPU}. Читайте больше:}
}

\newglossaryentry{heap}
{
  name=heap,
  description={(куча) обычно, большой кусок памяти предоставляемый \ac{OS}, так что прикладное ПО может делить его
  как захочет. malloc()/free() работают с кучей}
}

\newglossaryentry{name mangling}
{
  name=name mangling,
  description={применяется как минимум в \Cpp, где компилятору нужно закодировать имя класса,
  метода и типы аргументов в одной
  строке, которая будет внутренним именем функции. читайте также здесь: \myref{namemangling}}
}

\newglossaryentry{xoring}
{
  name=xoring,
  description={нередко применяемое в английском языке, означает применение операции 
  \ac{XOR}}
}

\newglossaryentry{security cookie}
{
  name=security cookie,
  description={Случайное значение, разное при каждом исполнении. Читайте больше об этом тут}
}

\newglossaryentry{tracer}
{
  name=tracer,
  description={Моя простейшая утилита для отладки. Читайте больше об этом тут: \myref{tracer}}
}

\newglossaryentry{GiB}
{
  name=GiB,
  description={Гибибайт: $2^{30}$ или 1024 мебибайт или 1073741824 байт}
}

\newglossaryentry{CP/M}
{
  name=CP/M,
  description={Control Program for Microcomputers: очень простая дисковая \ac{OS} использовавшаяся перед MS-DOS}
}

\newglossaryentry{stack frame}
{
  name=stack frame,
  description={Часть стека, в которой хранится информация, связанная с текущей функцией: локальные переменные,
  аргументы функции, \ac{RA}, итд.}
}

\newglossaryentry{jump offset}
{
  name=jump offset,
  description={Часть опкода JMP или Jcc инструкции, просто прибавляется к адресу следующей инструкции,
  и так вычисляется новый \ac{PC}. Может быть отрицательным}
}

\newglossaryentry{integral type}
{
  name=интегральный тип данных,
  description={обычные числа, но не вещественные. могут использоваться для передачи булевых типов и перечислений (enumerations)}
}

\newglossaryentry{real number}
{
  name=вещественное число,
  description={числа, которые могут иметь точку. в \CCpp это \Tfloat и \Tdouble}
}

\newglossaryentry{PDB}
{
  name=PDB,
  description={(Win32) Файл с отладочной информацией, обычно просто имена функций, 
  но иногда имена аргументов функций и локальных переменных}
}

\newglossaryentry{NTAPI}
{
  name=NTAPI,
  description={\ac{API} доступное только в линии Windows NT. 
  Большей частью не документировано Microsoft-ом}
}

\newglossaryentry{stdout}
{
  name=stdout,
  description={standard output}
}

\newglossaryentry{word}
{
  name=word,
  description={(слово) тип данных помещающийся в \ac{GPR}. 
  В компьютерах старше персональных, память часто измерялась не в байтах, 
  а в словах}
}

\newglossaryentry{arithmetic mean}
{
  name=среднее арифметическое,
  description={сумма всех значений, разделенная на их количество}
}
\newglossaryentry{padding}
{
  name=padding,
  description={\emph{Padding} в английском языке означает набивание подушки чем-либо для придания ей желаемой (большей)
  формы. В компьютерных науках, \emph{padding} означает добавление к блоку дополнительных байт, чтобы он имел нужный
  размер, например, $2^n$ байт.}
}

}
\FR{\newglossaryentry{tail call}
{
  name={tail call},
  description={C'est lorsque le compilateur (ou l'interpréteur) transforme la récursion (ce qui est possible: \emph{tail recursion})
  en une itération pour l'efficacité: \href{http://en.wikipedia.org/wiki/Tail_call}{wikipedia}}
}

\newglossaryentry{endianness}
{
  name=endianness,
  description={Ordre des octets: \myref{sec:endianness}}
}

\newglossaryentry{caller}
{
  name=caller,
  description={Une fonction en appelant une autre}
}

\newglossaryentry{callee}
{
  name=callee,
  description={Une fonction appelée par une autre}
}

\newglossaryentry{debuggee}
{
  name=debuggee,
  description={Un programme en train d'être débogué}
}

\newglossaryentry{leaf function}
{
  name=leaf function,
  description={Une fonction qui n'appelle pas d'autre fonction}
}

\newglossaryentry{link register}
{
  name=link register,
  description={(RISC) Un registre où l'adresse de retour est en général stockée. Ceci permet
  d'appeler une fonction leaf sans utiliser la pile, i.e, plus rapidemment}
}

\newglossaryentry{anti-pattern}
{
  name=anti-pattern,
  description={En général considéré comme une mauvaise pratique}
}

\newglossaryentry{stack pointer}
{
  name=pointeur de pile,
  description={Un registre qui pointe dans la pile}
}

\newglossaryentry{decrement}
{
  name=décrémenter,
  description={Décrémenter de 1}
}

\newglossaryentry{increment}
{
  name=incrémenter,
  description={Incrémenter de 1}
}

\newglossaryentry{loop unwinding}
{
  name=loop unwinding,
  description={C'est lorsqu'un compilateur, au lieu de générer du code pour une boucle de
  $n$ itérations, génère juste $n$ copies du corps de la boucle, afin de supprimer
  les instructions pour la gestion de la boucle}
}

\newglossaryentry{register allocator}
{
  name=register allocator,
  description={La partie du compilateur qui assigne des registes du CPU aux variables locales}
}

\newglossaryentry{quotient}
{
  name=quotient,
  description={Résultat de la division}
}

\newglossaryentry{product}
{
  name=produit,
  description={Résultat d'une multiplication}
}

\newglossaryentry{NOP}
{
  name=NOP,
  description={\q{no operation}, instruction ne faisant rien}
}

\newglossaryentry{POKE}
{
  name=POKE,
  description={instruction du langage BASIC pour écrire un octet à une adresse spécifique}
}

\newglossaryentry{keygenme}
{
  name=keygenme,
  description={Un programme qui imite la protection des logiciels pour lesquels on a besoin d'un générateur de clef/licence}
} % TODO clarify: A software which generate key/license value to bypass sotfware protection?

\newglossaryentry{dongle}
{
  name=dongle,
  description={Un dongle est un petit périphérique se connectant sur un port d'imprimante LPT (par le passé) ou USB.
  Sa fonction est similaire au tokens de sécurité, il a de la mémoire et, parfois, un algorithme secret de (crypto-)hachage.}
}

\newglossaryentry{thunk function}
{
  name=thunk function,
  description={Minuscule fonction qui a un seul rôle: appeler une autre fonction}
}

\newglossaryentry{user mode}
{
  name=user mode,
  description={Un mode CPU restreint dans lequel le code de toutes les applications est exécuté. cf. \gls{kernel mode}}
}

\newglossaryentry{kernel mode}
{
  name=kernel mode,
  description={Un mode CPU sans restriction dans lequel le noyau de l'OS et les drivers sont exécutés. cf. \gls{user mode}}
}

\newglossaryentry{Windows NT}
{
  name=Windows NT,
  description={Windows NT, 2000, XP, Vista, 7, 8, 10}
}

\newglossaryentry{atomic operation}
{
  name=atomic operation,
  description={
  \q{$\alpha{}\tau{}o\mu{}o\varsigma{}$}
  %\q{atomic}
  signifie \q{indivisible} en grec, donc il est garantie qu'une opération atomique ne sera pas interrompue par d'autres threads}
}

% to be proofreaded (begin)
\newglossaryentry{NaN}
{
  name=NaN,
  description={pas un nombre: un cas particulier pour les nombres à virgule flottante, indiquant généralement une erreur}
}

\newglossaryentry{NEON}
{
  name=NEON,
  description={\ac{AKA} \q{Advanced SIMD} --- \ac{SIMD} de ARM}
}

\newglossaryentry{reverse engineering}
{
  name=reverse engineering,
  description={action d'examiner et de comprendre comment quelque chose fonctionne, parfois dans le but de le reproduire}
}

\newglossaryentry{compiler intrinsic}
{
  name=compiler intrinsic,
  description={Une foncion spécifique d'un compilateur, qui n'est pas une fonction usuelle de bibliothèque.
    Le compilateur génère du code machine spécifique au lieu d'un appel à celui-ci.
    Souvent il s'agit d'une pseudo-fonction pour une instruction \ac{CPU} spécifique. Lire plus:
    (\myref{sec:compiler_intrinsic})}
}

\newglossaryentry{heap}
{
  name={tas},
  description={Généralement c'est un gros bout de mémoire fournit par l'\ac{OS} et utilisé par
  les applications pour le diviser comme elles le souhaitent. malloc()/free() fonctionnent en utilisant le tas}
}

\newglossaryentry{name mangling}
{
  name=name mangling,
  description={utilisé au moins en \Cpp, où le compilateur doit encoder le nom de la classe, la méthode et le type des arguments dans une chaîne,
  qui devient le nom interne de la fonction. Vous pouvez en lire plus à ce propos ici: \myref{namemangling}}
}

\newglossaryentry{xoring}
{
  name=xoring,
  description={souvent utilisé en anglais, qui signifie appliquer l'opération \ac{XOR}}
}

\newglossaryentry{security cookie}
{
  name=security cookie,
  description={Une valeur aléatoire, différente à chaque exécution. Vous pouvez en lire plus à ce propos ici: \myref{subsec:BO_protection}}
}

\newglossaryentry{tracer}
{
  name=tracer,
  description={Mon propre outil de debugging. Vous pouvez en lire plus à son propos ici: \myref{tracer}}
}

\newglossaryentry{GiB}
{
  name=GiB,
  description={Gibioctet: $2^{30}$ or 1024 mebioctets ou 1073741824 octets}
}

\newglossaryentry{CP/M}
{
  name=CP/M,
  description={Control Program for Microcomputers:
  un \ac{OS} de disque très basique utilisé avant MS-DOS}
}

\newglossaryentry{stack frame}
{
  name=stack frame,
  description={Une partie de la pile qui contient des informations spécifiques à la fonction courante:
  variables locales, arguments de la fonction, \ac{RA}, etc.}
}

\newglossaryentry{jump offset}
{
  name=jump offset,
  description={une partie de l'opcode de l'instruction JMP ou Jcc, qui doit être ajoutée à l'adresse de l'instruction suivante,
  et c'est ainsi que le nouveau \ac{PC} est calculé. Peut être négatif}
}

\newglossaryentry{integral type}
{
  name={type de donnée intégral},
  description={nombre usuel, mais pas un réel. peut être utilisé pour passer des variables de type booléen et des énumérations}
}

\newglossaryentry{real number}
{
  name={nombre réel},
  description={nombre qui peut contenir un point. comme \Tfloat et \Tdouble en \CCpp}
}

\newglossaryentry{PDB}
{
  name=PDB,
  description={(Win32) Fichier contenant des informations de débogage, en général seulement les noms des fonctions,
  mais aussi parfois les arguments des fonctions et le nom des variables locales}
}

\newglossaryentry{NTAPI}
{
  name=NTAPI,
  description={\ac{API} disponible seulement dans la série de Windows NT. Très peu documentée par Microsoft}
}

\newglossaryentry{stdout}
{
  name=stdout,
  description={standard output, sortie standard}
}

\newglossaryentry{word}
{
  name=word,
  description={Dans les ordinateurs plus vieux que les PCs, la taille de la mémoire était
  souvent mesurée en mots plutôt qu'en octet}
}

\newglossaryentry{arithmetic mean}
{
  name={moyenne arithmétique},
  description={la somme de toutes les valeurs, divisé par leur nombre}
}
\newglossaryentry{padding}
{
  name=padding,
  description={\emph{Padding} en anglais signifie rembourrer un oreiller, un matelas, etc. avec quelque chose afin de lui donner la forme désirée.
  En informatique, padding signifie ajouter des octets à un bloc, afin qu'il ait une certaine taille, comme $2^n$ octets.}
}

\newglossaryentry{basic block}
{
  name=basic block,
  description={
	un groupe d'instructions qui n'a pas d'instruction de saut/branchement,
	et donc n'as pas de saut de l'intérieur du bloc vers l'extérieur.
	Dans \IDA il ressemble à une liste d'instructions sans ligne vide}
}
}
\DE{\newglossaryentry{caller}
{
  name=caller,
  description={aufrufende Funktion}
}

\newglossaryentry{callee}
{
  name=callee,
  description={aufgerufene Funktion}
}

\newglossaryentry{stack pointer}
{
  name={Stapel-Zeiger},
  description={Ein Register das auf eine Stelle im Stack zeigt}
}

\newglossaryentry{product}
{
  name={Produkt},
  description={Ergebnis einer Multiplikation}
}

\newglossaryentry{GiB}
{
  name={GiB},
  description={\DEph{}}
}

\newglossaryentry{increment}
{
  name={\DEph},
  description={\DEph{}}
}

\newglossaryentry{decrement}
{
  name={\DEph},
  description={\DEph{}}
}

\newglossaryentry{stdout}
{
  name={\DEph},
  description={\DEph{}}
}

\newglossaryentry{endianness}
{
  name={\DEph},
  description={\DEph{}}
}

\newglossaryentry{thunk function}
{
  name={\DEph},
  description={\DEph{}}
}

\newglossaryentry{leaf function}
{
  name={\DEph},
  description={\DEph{}}
}

\newglossaryentry{heap}
{
  name={\DEph},
  description={\DEph{}}
}

\newglossaryentry{link register}
{
  name={\DEph},
  description={\DEph{}}
}

\newglossaryentry{anti-pattern}
{
  name={\DEph},
  description={\DEph{}}
}

\newglossaryentry{stack frame}
{
  name={\DEph},
  description={\DEph{}}
}

\newglossaryentry{jump offset}
{
  name={\DEph},
  description={\DEph{}}
}

\newglossaryentry{loop unwinding}
{
  name={\DEph},
  description={\DEph{}}
}

\newglossaryentry{tracer}
{
  name={\DEph},
  description={\DEph{}}
}

\newglossaryentry{register allocator}
{
  name={\DEph},
  description={\DEph{}}
}

\newglossaryentry{quotient}
{
  name={\DEph},
  description={\DEph{}}
}

\newglossaryentry{real number}
{
  name={\DEph},
  description={\DEph{}}
}

\newglossaryentry{NaN}
{
  name={\DEph},
  description={\DEph{}}
}

\newglossaryentry{Windows NT}
{
  name={\DEph},
  description={\DEph{}}
}

\newglossaryentry{word}
{
  name={\DEph},
  description={\DEph{}}
}

\newglossaryentry{PDB}
{
  name={\DEph},
  description={\DEph{}}
}

\newglossaryentry{name mangling}
{
  name={\DEph},
  description={\DEph{}}
}

\newglossaryentry{padding}
{
  name={\DEph},
  description={\DEph{}}
}

\newglossaryentry{NOP}
{
  name={\DEph},
  description={\DEph{}}
}

\newglossaryentry{POKE}
{
  name={\DEph},
  description={\DEph{}}
}

\newglossaryentry{xoring}
{
  name={\DEph},
  description={\DEph{}}
}

\newglossaryentry{atomic operation}
{
  name={\DEph},
  description={\DEph{}}
}

\newglossaryentry{basic block}
{
  name={\DEph},
  description={\DEph{}}
}

\newglossaryentry{reverse engineering}
{
  name={\DEph},
  description={\DEph{}}
}

\newglossaryentry{compiler intrinsic}
{
  name={\DEph},
  description={\DEph{}}
}


}
\IT{\newglossaryentry{caller}
{
  name=chiamante,
  description={Una funzione chiamante}
}

\newglossaryentry{callee}
{
  name=chiamata,
  description={Una funzione chiamata}
}

\newglossaryentry{stack pointer}
{
  name={stack pointer},
  description={Un registro che punta nello stack}
}

\newglossaryentry{prodotto}
{
  name={prodotto},
  description={Risultato di una moltiplicazione}
}

\newglossaryentry{GiB}
{
  name={GiB},
  description={Gibibyte: $2^{30}$ o 1024 megabyte o 1073741824 byte}
}

\newglossaryentry{increment}
{
  name={incrementa},
  description={Incrementa di 1}
}

\newglossaryentry{decrement}
{
  name={decrementa},
  description={Decrementa di 1}
}

\newglossaryentry{stdout}
{
  name={stdout},
  description={Standard output}
}

\newglossaryentry{endianness}
{
  name={endianness},
  description={L'ordine dei byte}
}

\newglossaryentry{thunk function}
{
  name={Funzione thunk},
  description={Piccola funzione con un solo scopo: chiamare un' altra funzione}
}

\newglossaryentry{leaf function}
{
  name={funzione foglia},
  description={Una funzione che non chiama nessun' altra funzione}
}

\newglossaryentry{heap}
{
  name={heap},
  description={di solito, una grossa locazione di memoria fornito da \ ac {OS} in modo che le applicazioni possano dividerla da sole come desiderano. malloc () / free () lavorano con l'heap}
}

\newglossaryentry{link register}
{
  name={registro link},
  description={(RISC) {Un registro in cui generalmente l'indirizzo di ritorno viene salvato.
Ciò rende possibile chiamare funzioni foglia più velocemente, ovvero senza l'utilizzo dello stack}}
}

\newglossaryentry{anti-pattern}
{
  name={anti-pattern},
  description={Generalmente considerata una cattiva pratica}
}

\newglossaryentry{stack frame}
{
  name={stack frame},
  description={\ITph{}}
}

\newglossaryentry{jump offset}
{
  name={offset di salto},
  description={\ITph{}}
}

\newglossaryentry{loop unwinding}
{
  name={srotolamente del ciclo},
  description={E' quando un compilatore, anzichè generare il codice di un ciclo per $n$ iterazioni, genera $n$ copie del corpo del ciclo, al fine di sbarazzarsi delle istruzioni per la manutenzione del ciclo}
}

\newglossaryentry{tracer}
{
  name={\ITph},
  description={\ITph{}}
}

\newglossaryentry{register allocator}
{
  name={registro allocatore},
  description={La parte del compilatore che assegna i registri della CPU alle variabili locali}
}

\newglossaryentry{quotient}
{
  name={quoziente},
  description={Risultato di una divisione}
}

\newglossaryentry{real number}
{
  name={numero reale},
  description={\ITph{}}
}

\newglossaryentry{NaN}
{
  name={Nan},
  description={non un numero:
	un caso speciale per i numeri a virgola mobile, generalmente segnalano errori}
}

\newglossaryentry{Windows NT}
{
  name={Windows NT},
  description={Windows NT, 2000, XP, Vista, 7, 8, 10}
}

\newglossaryentry{word}
{
  name={\ITph},
  description={\ITph{}}
}

\newglossaryentry{PDB}
{
  name={\ITph},
  description={\ITph{}}
}

\newglossaryentry{name mangling}
{
  name={\ITph},
  description={\ITph{}}
}

\newglossaryentry{padding}
{
  name={padding},
  description={in inglese significa riempire un cuscino con qualcosa per dargli una forma desiderata (più grande). In informatica, significa aggiungere più byte a un blocco in modo che abbia la dimensione desiderata, come $ 2 ^ n $ byte.}
}

\newglossaryentry{NOP}
{
  name={NOP},
  description={nessuna operazione (no operation)}
}

\newglossaryentry{POKE}
{
  name={POKE},
  description={Istruzione del linguaggio BASIC per scrivere un byte ad uno specifico indirizzo}
}

\newglossaryentry{xoring}
{
  name={\ITph},
  description={\ITph{}}
}

\newglossaryentry{atomic operation}
{
  name={operazione atomica},
  description={
  \q{$\alpha{}\tau{}o\mu{}o\varsigma{}$}
  %\q{atomic} 
  sta per \q{indivisibile} in Greco, quindi è garantito che un' operazione atomica non venga interrotta da altri thread}
}

\newglossaryentry{basic block}
{
  name={\ITph},
  description={Un gruppo di
	       istruzioni che non hanno salti / diramazioni e nemmeno salti da dentro il blocco verso fuori.
In \IDA è una lista di istruzioni senza linee vuote}
}

\newglossaryentry{reverse engineering}
{
  name={ingegneria inversa},
  description={L' atto di comprendere come una cosa funziona, a volte per clonarla}
}

\newglossaryentry{compiler intrinsic}
{
  name={\ITph},
  description={Una 	}
}
}
\JA{\newglossaryentry{tail call}
{
  name=tail call,
  description={コンパイラ(またはインタプリタ)が再帰(\emph{末尾再帰}が可能なとき)を効率のため、反復に変換するときです}
}

\newglossaryentry{endianness}
{
  name=endianness,
  description={バイトオーダー: \myref{sec:endianness}}
}

\newglossaryentry{caller}
{
  name=caller,
  description={呼び出し元の関数}
}

\newglossaryentry{callee}
{
  name=callee,
  description={呼び出された関数}
}

\newglossaryentry{debuggee}
{
  name=debuggee,
  description={デバッグされるプログラム}
}

\newglossaryentry{leaf function}
{
  name=leaf function,
  description={他の関数から呼び出されない関数}
}

\newglossaryentry{link register}
{
  name=link register,
  description=(RISC) リターンアドレスが保存されるレジスタ。これはleaf functionをスタックを使わずに呼び出すのを可能にする
}

\newglossaryentry{anti-pattern}
{
  name=anti-pattern,
  description=
  {一般に、よくないと考えられるやり方}
}

\newglossaryentry{stack pointer}
{
  name={スタックポインタ},
  description=
  {スタックの場所を示すレジスタ}
}

\newglossaryentry{decrement}
{
  name={デクリメント},
  description={1の減算}
}

\newglossaryentry{increment}
{
  name={インクリメント},
  description={1の加算}
}

\newglossaryentry{loop unwinding}
{
  name=loop unwinding,
  description={
  $n$回のイテレーションのループコードをコンパイラが生成する代わりに、ループボディを$n$回コピーするコードを生成する。
  ループに使用する命令を削除するため}
}

\newglossaryentry{register allocator}
{
  name=register allocator,
  description=
  {CPUレジスタをローカル変数に割り当てるコンパイラの機構}
}

\newglossaryentry{quotient}
{
  name=商,
  description={除算結果}
}

\newglossaryentry{product}
{
  name=積,
  description={乗算結果}
}

\newglossaryentry{NOP}
{
  name=NOP,
  description={何もしない命令}
}

\newglossaryentry{POKE}
{
  name=POKE,
  description={特定のアドレスにバイトを書き込むための基本的な言語命令}
}

\newglossaryentry{keygenme}
{
  name=keygenme,
  description={キー/ライセンスジェネレータのようなソフトウェア保護を模倣するプログラム}
} % TODO clarify: A software which generate key/license value to bypass sotfware protection?

\newglossaryentry{dongle}
{
  name=dongle,
  description={LPTプリンタポートやUSBにさす小さなハードウェア。セキュリティトークンと似て、メモリを持ち、時には秘密の(暗号学的)ハッシュアルゴリズムを持つ}
}

\newglossaryentry{thunk function}
{
  name=thunk function,
  description={単一の役割だけ持つ小さな関数:他の関数を呼び出す等}
}

\newglossaryentry{user mode}
{
  name=user mode,
  description={アプリケーションソフトウェアが実行される制限されたCPUモード \gls{kernel mode}}
}

\newglossaryentry{kernel mode}
{
  name=kernel mode,
  description={OSカーネルやドライバが実行される制限のないCPUモード \gls{user mode}}
}

\newglossaryentry{Windows NT}
{
  name=Windows NT,
  description={Windows NT, 2000, XP, Vista, 7, 8, 10}
}

\newglossaryentry{atomic operation}
{
  name=atomic operation,
  description={
  \q{$\alpha{}\tau{}o\mu{}o\varsigma{}$}
  %\q{atomic}
  ギリシャ語で分割不可能を意味し、処理は他のスレッドに割り込まれないことが保証される
  }
}

% to be proofreaded (begin)
\newglossaryentry{NaN}
{
  name=NaN,
  description=
    {非数:float型の数の特殊なケースで、エラーが通知される}
}

\newglossaryentry{basic block}
{
  name=basic block,
  description=
    {jump/branch命令を持たない命令群で、ブロックの外側から内側へのjumpも持たない}
}

\newglossaryentry{NEON}
{
  name=NEON,
  description={\ac{AKA} \q{Advanced SIMD} --- \ac{SIMD} 拡張、ARM における}
}

\newglossaryentry{reverse engineering}
{
  name=reverse engineering,
  description={時にはクローンするため、どうやって動いているのかを理解しようとする行為}
}

\newglossaryentry{compiler intrinsic}
{
  name=compiler intrinsic,
  description={
    通常のライブラリ関数ではない、コンパイラ特有の関数。コンパイラは特定の機械語を生成する。しばしば、特定のCPU命令のための疑似関数 (\myref{sec:compiler_intrinsic})
  }
}

\newglossaryentry{heap}
{
  name=ヒープ,
  description={\ac{OS}が提供する大きなメモリの塊のことで、アプリケーションが好きなように分割することができる。malloc()/free()を呼び出して使用する}
}

\newglossaryentry{name mangling}
{
  name=name mangling,
  description={コンパイラがクラス、メソッド、および引数型の名前を1つの文字列にエンコードする必要がある少なくとも \CCpp で使用され、
  関数の内部名になります。 以下で詳細を読むことができます: \myref{namemangling}}
}

\newglossaryentry{xoring}
{
  name=xoring,
  description={英語圏でしばしばみられ、\ac{XOR}操作を適用する意味になる}
}

\newglossaryentry{security cookie}
{
  name=security cookie,
  description={
  ランダムな値で、実行の度に異なった値になります。以下で詳細を読むことができます: \myref{subsec:BO_protection}}
}

\newglossaryentry{tracer}
{
  name=tracer,
  description={シンプルなデバッグツールです。以下で詳細を読むことができます: \myref{tracer}}
}

\newglossaryentry{GiB}
{
  name=GiB,
  description={ギガバイト:$2^{30}$ または1024メガバイトまたは1073741824バイト}
}

\newglossaryentry{CP/M}
{
  name=CP/M,
  description={Control Program for Microcomputers: 以前使用されていたとても基本的なディスク\ac{OS}です MS-DOS}
}

\newglossaryentry{stack frame}
{
  name=stack frame,
  description=
  {現在の関数に固有の情報(ローカル変数、関数の引数、\ac{RA}など)を含むスタックの一部}
}

\newglossaryentry{jump offset}
{
  name=jump offset,
  description=
  {JMP命令またはJcc命令のオペコードの一部を次の命令のアドレスに追加する必要があります。これが新しい\ac{PC}の計算方法です。 負となる場合もあります}
}

\newglossaryentry{integral type}
{
  name=整数型,
  description=
  {通常の数字ですが、実際の数字はありません。 ブールと列挙型の変数を渡すために使用できます}
}

\newglossaryentry{real number}
{
  name=実数,
  description={
  小数点以下を含む可能性のある数字。 これは \CCpp で \Tfloat と \Tdouble です
  }
}

\newglossaryentry{PDB}
{
  name=PDB,
  description={(Win32) 
  デバッグ情報ファイルで、通常は関数名だけではなく、関数の引数とローカル変数名を含む}
}

\newglossaryentry{NTAPI}
{
  name=NTAPI,
  description=
  {APIはWindows NT系列でのみ使用できます。 Microsoftは大部分で文書化していません}
}

\newglossaryentry{stdout}
{
  name=stdout,
  description={standard output}
}

\newglossaryentry{word}
{
  name=word,
  description=
  {PCよりも昔のコンピュータでは、メモリサイズはバイトではなくワードで測定されることがしばしばでした}
}

\newglossaryentry{arithmetic mean}
{
  name={算術平均},
  description=
  {すべての値の合計を数で割った値}
}
\newglossaryentry{padding}
{
  name=padding,
  description=
  {英語で\emph{パディング}とは、(より大きな)望ましい形状にするために、あるものを枕に詰めることを意味します。
  コンピュータサイエンスでは、パディングとはブロックにバイトを追加することで、$2^n$バイトのようなサイズにすることを意味します。}
}

}
\PL{\newglossaryentry{tail call}
{
  name={\PLph},
  description={\PLph}
}

\newglossaryentry{endianness}
{
  name={endianess},
  description={kolejność bajtów}
}

\newglossaryentry{caller}
{
  name={caller},
  description={funkcja wywołująca}
}

\newglossaryentry{callee}
{
  name={callee},
  description={funkcja wywoływana}
}

\newglossaryentry{debuggee}
{
  name={\PLph},
  description={\PLph}
}

\newglossaryentry{leaf function}
{
  name={funkcja liść},
  description={Funkcja, która nie wywołuje żadnej innej}
}

\newglossaryentry{link register}
{
  name={rejestr powrotu},
  description=(RISC) {Rejestr, w który zwykle przechowywany jest adres powrotu. Dzięki temu można wywoływać funkcje-liście (leaf functions) bez używania stosu - a więc szybciej}
}

\newglossaryentry{anti-pattern}
{
  name={anti-pattern},
  description={coś powszechnie uznanego jako zła praktyka}
}

\newglossaryentry{stack pointer}
{
  name=wskaźnik stosu,
  description={rejestr pokazujący na miejsce na stosie}
}

\newglossaryentry{decrement}
{
  name={\PLph},
  description={\PLph}
}

\newglossaryentry{increment}
{
  name={inkrementować},
  description={zwiększać o 1}
}

\newglossaryentry{loop unwinding}
{
  name={\PLph},
  description={\PLph}
}

\newglossaryentry{register allocator}
{
  name={\PLph},
  description={\PLph}
}

\newglossaryentry{quotient}
{
  name={\PLph},
  description={\PLph}
}

\newglossaryentry{product}
{
  name={\PLph},
  description={\PLph}
}

\newglossaryentry{NOP}
{
  name={\PLph},
  description={\PLph}
}

\newglossaryentry{POKE}
{
  name={\PLph},
  description={\PLph}
}

\newglossaryentry{keygenme}
{
  name={\PLph},
  description={\PLph}
} 

\newglossaryentry{dongle}
{
  name={\PLph},
  description={\PLph}
}

\newglossaryentry{thunk function}
{
  name={thunk function},
  description={prosta funkcja, której jedynym zadaniem jest wywołanie innej funkcji}
}

\newglossaryentry{user mode}
{
  name={\PLph},
  description={\PLph}
}

\newglossaryentry{kernel mode}
{
  name={\PLph},
  description={\PLph}
}

\newglossaryentry{Windows NT}
{
  name={\PLph},
  description={\PLph}
}

\newglossaryentry{atomic operation}
{
  name={\PLph},
  description={\PLph}
}

\newglossaryentry{NaN}
{
  name={\PLph},
  description={\PLph}
}

\newglossaryentry{basic block}
{
  name={\PLph},
  description={\PLph}
}

\newglossaryentry{NEON}
{
  name={\PLph},
  description={\PLph}
}

\newglossaryentry{reverse engineering}
{
  name={inżynieria wsteczna},
  description={proces odkrywania jak dana rzecz działa, czasami w celu jej sklonowania}
}

\newglossaryentry{compiler intrinsic}
{
  name={\PLph},
  description={\PLph}
}

\newglossaryentry{heap}
{
  name={heap},
  description={(kopiec, sterta) - przeważnie duży kawałek pamięci, zapewniony aplikacji przez \ac{OS} na jej własne potrzeby. malloc()/free() pracują ze stertą}
}

\newglossaryentry{name mangling}
{
  name={\PLph},
  description={\PLph}
}

\newglossaryentry{xoring}
{
  name={\PLph},
  description={\PLph}
}

\newglossaryentry{security cookie}
{
  name={\PLph},
  description={\PLph}
}

\newglossaryentry{tracer}
{
  name={\PLph},
  description={\PLph}
}

\newglossaryentry{GiB}
{
  name=GiB,
  description={gibibajt: $2^{10}$ (1024) mebibajtów, $2^{20}$ (1048576) kibibajtów lub $2^{30}$ (1073741824) bajtów}
}

\newglossaryentry{CP/M}
{
  name=CP/M,
  description={\PLph}
}

\newglossaryentry{stack frame}
{
  name=ramka stosu,
  description={Część stosu, która zawiera informacje specyficzne dla bieżącej funkcji:
  zmienne lokalne, argumenty funkcji, \ac{RA}, etc.}
}

\newglossaryentry{jump offset}
{
  name=przesunięcie skoku,
  description={część kodu operacji instrukcji JMP i Jcc,
  która jest dodawana do adresu
  kolejnej instrukcji by wyliczyć nową wartość \ac{PC}. Może mieć wartość ujemną}
}

\newglossaryentry{integral type}
{
  name={\PLph},
  description={\PLph}
}

\newglossaryentry{real number}
{
  name={real number},
  description={\PLph}
}

\newglossaryentry{PDB}
{
  name={\PLph},
  description={\PLph}
}

\newglossaryentry{NTAPI}
{
  name={\PLph},
  description={\PLph}
}

\newglossaryentry{stdout}
{
  name={stdout},
  description={standardowe wyjście}
}

\newglossaryentry{word}
{
  name={\PLph},
  description={\PLph}
}

\newglossaryentry{arithmetic mean}
{
  name={\PLph},
  description={\PLph}
}

\newglossaryentry{padding}
{
  name={\PLph},
  description={\PLph}
}

}

\makeglossaries

\hypersetup{
    colorlinks=true,
    allcolors=blue,
    pdfauthor={\AUTHOR},
    pdftitle={\TitleMain}
    }

%\ifdefined\RUSSIAN
\newcommand{\LstStyle}{\ttfamily\small}
%\else
%\newcommand{\LstStyle}{\ttfamily}
%\fi

% inspired by http://prismjs.com/
\definecolor{digits}{RGB}{0,0,0}
\definecolor{bg}{RGB}{255,255,255}
%\definecolor{bg}{RGB}{255,252,250}
\definecolor{col1}{RGB}{154,20,150}
\definecolor{col2}{RGB}{112,128,144}
\definecolor{col3}{RGB}{10,120,180}
\definecolor{col4}{RGB}{106,164,108}

\lstset{
    %backgroundcolor=\color{lstbgcolor},
    %backgroundcolor=\color{light-gray},
    backgroundcolor=\color{bg},
    basicstyle=\LstStyle,
    breaklines=true,
    %prebreak=\raisebox{0ex}[0ex][0ex]{->},
    %postbreak=\raisebox{0ex}[0ex][0ex]{->},
    prebreak=\raisebox{0ex}[0ex][0ex]{\ensuremath{\rhookswarrow}},
    postbreak=\raisebox{0ex}[0ex][0ex]{\ensuremath{\rcurvearrowse\space}},
    frame=single,
    columns=fullflexible,keepspaces,
    escapeinside=§§,
    inputencoding=utf8
}

\definecolor{digits}{RGB}{0,0,0}
\definecolor{colkw}{RGB}{0,0,0}

\definecolor{colstr}{RGB}{0,0,0}
\definecolor{colid}{RGB}{0,0,0}

\definecolor{colcomment}{RGB}{130,130,130}

\definecolor{colbg}{RGB}{255,255,255}

\lstdefinestyle{custompy}{
  texcl=true,
  language=Python,
  showstringspaces=false,
  backgroundcolor=\color{colbg},
  keywordstyle=\color{colkw},
  commentstyle=\color{colcomment},
  identifierstyle=\color{colid},
  stringstyle=\color{colstr}
}

\lstdefinestyle{customc}{
  texcl=true,
  language=C,
  showstringspaces=false,
  backgroundcolor=\color{colbg},
  keywordstyle=\color{colkw},
  commentstyle=\color{colcomment},
  identifierstyle=\color{colid},
  stringstyle=\color{colstr}
}

\lstdefinestyle{custommath}{
  texcl=true,
  language=Mathematica,
  showstringspaces=false,
  backgroundcolor=\color{colbg},
  keywordstyle=\color{colkw},
  commentstyle=\color{colcomment},
  identifierstyle=\color{colid},
  stringstyle=\color{colstr}
}

\lstdefinestyle{customjava}{
  texcl=true,
  language=Java,
  showstringspaces=false,
  backgroundcolor=\color{colbg},
  keywordstyle=\color{colkw},
  commentstyle=\color{colcomment},
  identifierstyle=\color{colid},
  stringstyle=\color{colstr}
}

\lstdefinestyle{customasmx86}{
  texcl=true,
  morecomment=[l]{;},
  morestring=[b]",
  morestring=[d]',
  showstringspaces=false,
  backgroundcolor=\color{colbg},
  keywordstyle=\color{colkw},
  commentstyle=\color{colcomment},
  identifierstyle=\color{colid},
  stringstyle=\color{colstr}
}

\lstdefinestyle{customasmMIPS}{
  texcl=true,
  morecomment=[l]{;},
  morecomment=[l]{\#},
  morestring=[b]",
  morestring=[d]',
  showstringspaces=false,
  backgroundcolor=\color{colbg},
  keywordstyle=\color{colkw},
  commentstyle=\color{colcomment},
  identifierstyle=\color{colid},
  stringstyle=\color{colstr}
}

\lstdefinestyle{customasmARM}{
  texcl=true,
  morecomment=[l]{;},
  morestring=[b]",
  morestring=[d]',
  showstringspaces=false,
  backgroundcolor=\color{colbg},
  keywordstyle=\color{colkw},
  commentstyle=\color{colcomment},
  identifierstyle=\color{colid},
  stringstyle=\color{colstr}
}

\lstdefinestyle{customasmPPC}{
  texcl=true,
  morecomment=[l]{;},
  morecomment=[l]{\#},
  morestring=[b]",
  morestring=[d]',
  showstringspaces=false,
  backgroundcolor=\color{colbg},
  keywordstyle=\color{colkw},
  commentstyle=\color{colcomment},
  identifierstyle=\color{colid},
  stringstyle=\color{colstr}
}



\ifdefined\RUSSIAN
\renewcommand\lstlistingname{Листинг}
\renewcommand\lstlistlistingname{Листинг}
\fi

\DeclareMathSizes{12}{30}{16}{12}%

% see also:
% http://tex.stackexchange.com/questions/129225/how-can-i-get-get-makeindex-to-ignore-capital-letters
% http://tex.stackexchange.com/questions/18336/correct-sorting-of-index-entries-containing-macros
\def\myindex#1{\expandafter\index\expandafter{#1}}

\begin{document}

% fancyhdr =============================================================================================================================
\pagestyle{fancy}
\setlength{\headheight}{13pt}
% https://tex.stackexchange.com/questions/10043/page-number-position
\fancyhf{}
\fancyhead[R]{\thepage} % suppress chapter name, add page number (upper right corner)

\ifdefined\ENGLISH
%\cfoot{\small Please become my patron, so I can spend more time writing this: \href{http://link.yurichev.com/17400?\today p.\thepage}{patreon.com}. Thanks!\normalsize}
\cfoot{\small If you noticed a typo, error or have any suggestions, do not hesitate to drop me a note: \EMAILPRI{}.
Thanks!\normalsize}
% https://tex.stackexchange.com/questions/13406/how-to-add-a-horizontal-line-above-the-footer-with-fancyhdr
\renewcommand{\footrulewidth}{0.4pt}
\fi

\ifdefined\RUSSIAN
\cfoot{\small Если вы заметили опечатку, ошибку или имеете какие-то либо соображения, пожелания, пожалуйста, напишите мне: \EMAILPRI{}.
Спасибо!\normalsize}
% https://tex.stackexchange.com/questions/13406/how-to-add-a-horizontal-line-above-the-footer-with-fancyhdr
\renewcommand{\footrulewidth}{0.4pt}
\fi

% ======================================================================================================================================

\VerbatimFootnotes

\frontmatter

%\RU{\input{1st_page_RU}}
%\EN{\input{1st_page_EN}}
%\DE{\input{1st_page_DE}}
%\ES{\input{1st_page_ES}}
%\CN{\input{1st_page_CN}}

\iffalse
\RU{\begin{titlepage}
\vspace*{\fill}

\begin{flushright}
\emph{Эта книга посвящается \\
компьютеру Intertec Superbrain II}
\end{flushright}

\vspace*{\fill}
\end{titlepage}

}
\EN{\begin{titlepage}
\vspace*{\fill}

\begin{flushright}
\emph{This book is dedicated to \\
Intertec Superbrain II computer}

\end{flushright}

\vspace*{\fill}
\end{titlepage}

}
\FR{\begin{titlepage}
\vspace*{\fill}

\begin{flushright}
\emph{Ce livre est dédié à \\
l'ordinateur Intertec Superbrain II}
\end{flushright}

\vspace*{\fill}
\end{titlepage}

}
\JA{\begin{titlepage}
\vspace*{\fill}

\begin{flushright}
\emph{この本は \\
Intertec Superbrain II computer専用です}

\end{flushright}

\vspace*{\fill}
\end{titlepage}

}
\DE{\begin{titlepage}
\vspace*{\fill}

\begin{flushright}
\emph{Diese Buch ist gewidmet: \\
Intertec Superbrain II computer}

\end{flushright}

\vspace*{\fill}
\end{titlepage}

}
\IT{\begin{titlepage}
\vspace*{\fill}

\begin{flushright}
\emph{Questo libro è dedicato al \\
Intertec Superbrain II computer}

\end{flushright}

\vspace*{\fill}
\end{titlepage}
}
\PL{\begin{titlepage}
\vspace*{\fill}

\begin{flushright}
\emph{Książkę poświęcam \\
komputerowi Intertec Superbrain II}

\end{flushright}

\vspace*{\fill}
\end{titlepage}

}
\fi

\EN{\begin{titlepage}

\begin{center}
\vspace*{\fill}

\ifdefined\UAL{}

\begin{center}
\begin{lstlisting}[basicstyle=\ttfamily\small,frame=none]
  _   _           _               _                  _ _             
 | | | |_ __   __| | ___ _ __ ___| |_ __ _ _ __   __| (_)_ __   __ _ 
 | | | | '_ \ / _` |/ _ \ '__/ __| __/ _` | '_ \ / _` | | '_ \ / _` |
 | |_| | | | | (_| |  __/ |  \__ \ || (_| | | | | (_| | | | | | (_| |
  \___/|_| |_|\__,_|\___|_|  |___/\__\__,_|_| |_|\__,_|_|_| |_|\__, |
                                                               |___/ 
                _                           _     _                  
               / \   ___ ___  ___ _ __ ___ | |__ | |_   _            
              / _ \ / __/ __|/ _ \ '_ ` _ \| '_ \| | | | |           
             / ___ \\__ \__ \  __/ | | | | | |_) | | |_| |           
            /_/   \_\___/___/\___|_| |_| |_|_.__/|_|\__, |           
                                                    |___/            
            _                                                        
           | |    __ _ _ __   __ _ _   _  __ _  __ _  ___            
           | |   / _` | '_ \ / _` | | | |/ _` |/ _` |/ _ \           
           | |__| (_| | | | | (_| | |_| | (_| | (_| |  __/           
           |_____\__,_|_| |_|\__, |\__,_|\__,_|\__, |\___|           
                             |___/             |___/                 
\end{lstlisting}
\end{center}

\else

\begin{center}
\begin{lstlisting}[basicstyle=\ttfamily,frame=none]
                     ____                                        
                    |  _ \ _____   _____ _ __ ___  ___           
                    | |_) / _ \ \ / / _ \ '__/ __|/ _ \          
                    |  _ <  __/\ V /  __/ |  \__ \  __/          
                    |_| \_\___| \_/ \___|_|  |___/\___|          
                                                                 
          _____             _                      _             
         | ____|_ __   __ _(_)_ __   ___  ___ _ __(_)_ __   __ _ 
         |  _| | '_ \ / _` | | '_ \ / _ \/ _ \ '__| | '_ \ / _` |
         | |___| | | | (_| | | | | |  __/  __/ |  | | | | | (_| |
         |_____|_| |_|\__, |_|_| |_|\___|\___|_|  |_|_| |_|\__, |
                      |___/                                |___/ 
                              __                                 
                             / _| ___  _ __                      
                            | |_ / _ \| '__|                     
                            |  _| (_) | |                        
                            |_|  \___/|_|                        
                                                                 
               ____             _                                
              | __ )  ___  __ _(_)_ __  _ __   ___ _ __ ___      
              |  _ \ / _ \/ _` | | '_ \| '_ \ / _ \ '__/ __|     
              | |_) |  __/ (_| | | | | | | | |  __/ |  \__ \     
              |____/ \___|\__, |_|_| |_|_| |_|\___|_|  |___/     
                          |___/                                  
\end{lstlisting}
\end{center}

\fi

\vspace*{\fill}
\end{center}

\end{titlepage}

\newpage

\begin{center}
\vspace*{\fill}
{\LARGE \TitleMain}

\bigskip

{\large (\TitleAux)}

\bigskip
\bigskip
Why two titles? Read here: \myref{TwoTitles}.

\vspace*{\fill}

{\large \AUTHOR}

{\large \TT{\EMAILPRI}}
\vspace*{\fill}
\vfill

\ccbysa

\textcopyright 2013-2021, \AUTHOR. 

This work is licensed under the Creative Commons Attribution-ShareAlike 4.0 International (CC BY-SA 4.0) license.
To view a copy of this license, visit \url{https://creativecommons.org/licenses/by-sa/4.0/}.

Text version ({\large \today}).

The latest version (and Russian edition) of this text is accessible at \url{https://beginners.re/}.

\end{center}
}
\RU{\begin{titlepage}

\begin{center}
\vspace*{\fill}

\ifdefined\UAL{}

\begin{center}
\begin{lstlisting}[basicstyle=\ttfamily\small,frame=none]
  _   _           _               _                  _ _             
 | | | |_ __   __| | ___ _ __ ___| |_ __ _ _ __   __| (_)_ __   __ _ 
 | | | | '_ \ / _` |/ _ \ '__/ __| __/ _` | '_ \ / _` | | '_ \ / _` |
 | |_| | | | | (_| |  __/ |  \__ \ || (_| | | | | (_| | | | | | (_| |
  \___/|_| |_|\__,_|\___|_|  |___/\__\__,_|_| |_|\__,_|_|_| |_|\__, |
                                                               |___/ 
                _                           _     _                  
               / \   ___ ___  ___ _ __ ___ | |__ | |_   _            
              / _ \ / __/ __|/ _ \ '_ ` _ \| '_ \| | | | |           
             / ___ \\__ \__ \  __/ | | | | | |_) | | |_| |           
            /_/   \_\___/___/\___|_| |_| |_|_.__/|_|\__, |           
                                                    |___/            
            _                                                        
           | |    __ _ _ __   __ _ _   _  __ _  __ _  ___            
           | |   / _` | '_ \ / _` | | | |/ _` |/ _` |/ _ \           
           | |__| (_| | | | | (_| | |_| | (_| | (_| |  __/           
           |_____\__,_|_| |_|\__, |\__,_|\__,_|\__, |\___|           
                             |___/             |___/                 
\end{lstlisting}
\end{center}

\else

\begin{center}
\begin{lstlisting}[basicstyle=\ttfamily,frame=none]
                     ____                                        
                    |  _ \ _____   _____ _ __ ___  ___           
                    | |_) / _ \ \ / / _ \ '__/ __|/ _ \          
                    |  _ <  __/\ V /  __/ |  \__ \  __/          
                    |_| \_\___| \_/ \___|_|  |___/\___|          
                                                                 
          _____             _                      _             
         | ____|_ __   __ _(_)_ __   ___  ___ _ __(_)_ __   __ _ 
         |  _| | '_ \ / _` | | '_ \ / _ \/ _ \ '__| | '_ \ / _` |
         | |___| | | | (_| | | | | |  __/  __/ |  | | | | | (_| |
         |_____|_| |_|\__, |_|_| |_|\___|\___|_|  |_|_| |_|\__, |
                      |___/                                |___/ 
                              __                                 
                             / _| ___  _ __                      
                            | |_ / _ \| '__|                     
                            |  _| (_) | |                        
                            |_|  \___/|_|                        
                                                                 
               ____             _                                
              | __ )  ___  __ _(_)_ __  _ __   ___ _ __ ___      
              |  _ \ / _ \/ _` | | '_ \| '_ \ / _ \ '__/ __|     
              | |_) |  __/ (_| | | | | | | | |  __/ |  \__ \     
              |____/ \___|\__, |_|_| |_|_| |_|\___|_|  |___/     
                          |___/                                  
\end{lstlisting}
\end{center}

\fi

\vspace*{\fill}
\end{center}

\end{titlepage}

\newpage

\begin{center}
\vspace*{\fill}
{\LARGE \TitleMain}

\bigskip

{\large (\TitleAux)}

\bigskip
\bigskip
Почему два названия? Читайте здесь: \myref{TwoTitles}.

\vspace*{\fill}

{\large \AUTHOR}

{\large \TT{\EMAILPRI}}
\vspace*{\fill}
\vfill

\ccbysa

\textcopyright 2013-2021, \AUTHOR. 

Это произведение доступно по лицензии Creative Commons «Attribution-ShareAlike 4.0 International» (CC BY-SA 4.0).
Чтобы увидеть копию этой лицензии, посетите \url{https://creativecommons.org/licenses/by-sa/4.0/}.

Версия этого текста ({\large \today}).

Самая новая версия текста (а также англоязычная версия) доступна на сайте \url{https://beginners.re/}.

\end{center}
}
\ES{\begin{titlepage}

\begin{center}
\vspace*{\fill}

\ifdefined\UAL{}

\begin{center}
\begin{lstlisting}[basicstyle=\ttfamily\small,frame=none]
  _   _           _               _                  _ _             
 | | | |_ __   __| | ___ _ __ ___| |_ __ _ _ __   __| (_)_ __   __ _ 
 | | | | '_ \ / _` |/ _ \ '__/ __| __/ _` | '_ \ / _` | | '_ \ / _` |
 | |_| | | | | (_| |  __/ |  \__ \ || (_| | | | | (_| | | | | | (_| |
  \___/|_| |_|\__,_|\___|_|  |___/\__\__,_|_| |_|\__,_|_|_| |_|\__, |
                                                               |___/ 
                _                           _     _                  
               / \   ___ ___  ___ _ __ ___ | |__ | |_   _            
              / _ \ / __/ __|/ _ \ '_ ` _ \| '_ \| | | | |           
             / ___ \\__ \__ \  __/ | | | | | |_) | | |_| |           
            /_/   \_\___/___/\___|_| |_| |_|_.__/|_|\__, |           
                                                    |___/            
            _                                                        
           | |    __ _ _ __   __ _ _   _  __ _  __ _  ___            
           | |   / _` | '_ \ / _` | | | |/ _` |/ _` |/ _ \           
           | |__| (_| | | | | (_| | |_| | (_| | (_| |  __/           
           |_____\__,_|_| |_|\__, |\__,_|\__,_|\__, |\___|           
                             |___/             |___/                 
\end{lstlisting}
\end{center}

\else

\begin{center}
\begin{lstlisting}[basicstyle=\ttfamily,frame=none]
                     ____                                        
                    |  _ \ _____   _____ _ __ ___  ___           
                    | |_) / _ \ \ / / _ \ '__/ __|/ _ \          
                    |  _ <  __/\ V /  __/ |  \__ \  __/          
                    |_| \_\___| \_/ \___|_|  |___/\___|          
                                                                 
          _____             _                      _             
         | ____|_ __   __ _(_)_ __   ___  ___ _ __(_)_ __   __ _ 
         |  _| | '_ \ / _` | | '_ \ / _ \/ _ \ '__| | '_ \ / _` |
         | |___| | | | (_| | | | | |  __/  __/ |  | | | | | (_| |
         |_____|_| |_|\__, |_|_| |_|\___|\___|_|  |_|_| |_|\__, |
                      |___/                                |___/ 
                              __                                 
                             / _| ___  _ __                      
                            | |_ / _ \| '__|                     
                            |  _| (_) | |                        
                            |_|  \___/|_|                        
                                                                 
               ____             _                                
              | __ )  ___  __ _(_)_ __  _ __   ___ _ __ ___      
              |  _ \ / _ \/ _` | | '_ \| '_ \ / _ \ '__/ __|     
              | |_) |  __/ (_| | | | | | | | |  __/ |  \__ \     
              |____/ \___|\__, |_|_| |_|_| |_|\___|_|  |___/     
                          |___/                                  
\end{lstlisting}
\end{center}

\fi

\vspace*{\fill}
\end{center}

\end{titlepage}

\newpage

\begin{center}
\vspace*{\fill}
{\LARGE \TitleMain}

\bigskip

{\large (\TitleAux)}

\bigskip
\bigskip
Why two titles? Read here: \myref{TwoTitles}. % TBT

\vspace*{\fill}

{\large \AUTHOR}

{\large \TT{\EMAILPRI}}
\vspace*{\fill}
\vfill

\ccbysa

\textcopyright 2013-2021, \AUTHOR. 

Esta obra est\'a bajo una Licencia Creative Commons ``Attribution-ShareAlike 4.0 International'' (CC BY-SA 4.0)
Para ver una copia de esta licencia, visita \url{https://creativecommons.org/licenses/by-sa/4.0/}.

Versi\'on del texto ({\large \today}).

La \'ultima versi\'on (as\'i como las versiones en ingl\'es y ruso) de este texto est\'a disponible en
\url{https://beginners.re/}.

\end{center}
}
\IT{\begin{titlepage}

\begin{center}
\vspace*{\fill}

\ifdefined\UAL{}

\begin{center}
\begin{lstlisting}[basicstyle=\ttfamily\small,frame=none]
  _   _           _               _                  _ _             
 | | | |_ __   __| | ___ _ __ ___| |_ __ _ _ __   __| (_)_ __   __ _ 
 | | | | '_ \ / _` |/ _ \ '__/ __| __/ _` | '_ \ / _` | | '_ \ / _` |
 | |_| | | | | (_| |  __/ |  \__ \ || (_| | | | | (_| | | | | | (_| |
  \___/|_| |_|\__,_|\___|_|  |___/\__\__,_|_| |_|\__,_|_|_| |_|\__, |
                                                               |___/ 
                _                           _     _                  
               / \   ___ ___  ___ _ __ ___ | |__ | |_   _            
              / _ \ / __/ __|/ _ \ '_ ` _ \| '_ \| | | | |           
             / ___ \\__ \__ \  __/ | | | | | |_) | | |_| |           
            /_/   \_\___/___/\___|_| |_| |_|_.__/|_|\__, |           
                                                    |___/            
            _                                                        
           | |    __ _ _ __   __ _ _   _  __ _  __ _  ___            
           | |   / _` | '_ \ / _` | | | |/ _` |/ _` |/ _ \           
           | |__| (_| | | | | (_| | |_| | (_| | (_| |  __/           
           |_____\__,_|_| |_|\__, |\__,_|\__,_|\__, |\___|           
                             |___/             |___/                 
\end{lstlisting}
\end{center}

\else

\begin{center}
\begin{lstlisting}[basicstyle=\ttfamily,frame=none]
                     ____                                        
                    |  _ \ _____   _____ _ __ ___  ___           
                    | |_) / _ \ \ / / _ \ '__/ __|/ _ \          
                    |  _ <  __/\ V /  __/ |  \__ \  __/          
                    |_| \_\___| \_/ \___|_|  |___/\___|          
                                                                 
          _____             _                      _             
         | ____|_ __   __ _(_)_ __   ___  ___ _ __(_)_ __   __ _ 
         |  _| | '_ \ / _` | | '_ \ / _ \/ _ \ '__| | '_ \ / _` |
         | |___| | | | (_| | | | | |  __/  __/ |  | | | | | (_| |
         |_____|_| |_|\__, |_|_| |_|\___|\___|_|  |_|_| |_|\__, |
                      |___/                                |___/ 
                              __                                 
                             / _| ___  _ __                      
                            | |_ / _ \| '__|                     
                            |  _| (_) | |                        
                            |_|  \___/|_|                        
                                                                 
               ____             _                                
              | __ )  ___  __ _(_)_ __  _ __   ___ _ __ ___      
              |  _ \ / _ \/ _` | | '_ \| '_ \ / _ \ '__/ __|     
              | |_) |  __/ (_| | | | | | | | |  __/ |  \__ \     
              |____/ \___|\__, |_|_| |_|_| |_|\___|_|  |___/     
                          |___/                                  
\end{lstlisting}
\end{center}

\fi

\vspace*{\fill}
\end{center}

\end{titlepage}

\newpage

\begin{center}
\vspace*{\fill}
{\LARGE \TitleMain}

\bigskip

{\large (\TitleAux)}

\bigskip
\bigskip
Perchè due titoli? Leggi qua: \myref{TwoTitles}.

\vspace*{\fill}

{\large \AUTHOR}

{\large \TT{\EMAILPRI}}
\vspace*{\fill}
\vfill

\ccbysa

\textcopyright 2013-2021, \AUTHOR.

Questo lavoro è rilasciato sotto licenza Creative Commons Attribution-ShareAlike 4.0 International (CC BY-SA 4.0).
Per vedere la copia di questa licenza visita il sito \url{https://creativecommons.org/licenses/by-sa/4.0/}.

Teston ({\large \today}).

L'ultima versione (e l'edizione Russa) del testo sono accessibili al seguente sito: \url{https://beginners.re/}.

\end{center}
}
\FR{\begin{titlepage}

\begin{center}
\vspace*{\fill}

\ifdefined\UAL{}

\begin{center}
\begin{lstlisting}[basicstyle=\ttfamily\small,frame=none]
  _   _           _               _                  _ _             
 | | | |_ __   __| | ___ _ __ ___| |_ __ _ _ __   __| (_)_ __   __ _ 
 | | | | '_ \ / _` |/ _ \ '__/ __| __/ _` | '_ \ / _` | | '_ \ / _` |
 | |_| | | | | (_| |  __/ |  \__ \ || (_| | | | | (_| | | | | | (_| |
  \___/|_| |_|\__,_|\___|_|  |___/\__\__,_|_| |_|\__,_|_|_| |_|\__, |
                                                               |___/ 
                _                           _     _                  
               / \   ___ ___  ___ _ __ ___ | |__ | |_   _            
              / _ \ / __/ __|/ _ \ '_ ` _ \| '_ \| | | | |           
             / ___ \\__ \__ \  __/ | | | | | |_) | | |_| |           
            /_/   \_\___/___/\___|_| |_| |_|_.__/|_|\__, |           
                                                    |___/            
            _                                                        
           | |    __ _ _ __   __ _ _   _  __ _  __ _  ___            
           | |   / _` | '_ \ / _` | | | |/ _` |/ _` |/ _ \           
           | |__| (_| | | | | (_| | |_| | (_| | (_| |  __/           
           |_____\__,_|_| |_|\__, |\__,_|\__,_|\__, |\___|           
                             |___/             |___/                 
\end{lstlisting}
\end{center}

\else

\begin{center}
\begin{lstlisting}[basicstyle=\ttfamily,frame=none]
                     ____                                        
                    |  _ \ _____   _____ _ __ ___  ___           
                    | |_) / _ \ \ / / _ \ '__/ __|/ _ \          
                    |  _ <  __/\ V /  __/ |  \__ \  __/          
                    |_| \_\___| \_/ \___|_|  |___/\___|          
                                                                 
          _____             _                      _             
         | ____|_ __   __ _(_)_ __   ___  ___ _ __(_)_ __   __ _ 
         |  _| | '_ \ / _` | | '_ \ / _ \/ _ \ '__| | '_ \ / _` |
         | |___| | | | (_| | | | | |  __/  __/ |  | | | | | (_| |
         |_____|_| |_|\__, |_|_| |_|\___|\___|_|  |_|_| |_|\__, |
                      |___/                                |___/ 
                              __                                 
                             / _| ___  _ __                      
                            | |_ / _ \| '__|                     
                            |  _| (_) | |                        
                            |_|  \___/|_|                        
                                                                 
               ____             _                                
              | __ )  ___  __ _(_)_ __  _ __   ___ _ __ ___      
              |  _ \ / _ \/ _` | | '_ \| '_ \ / _ \ '__/ __|     
              | |_) |  __/ (_| | | | | | | | |  __/ |  \__ \     
              |____/ \___|\__, |_|_| |_|_| |_|\___|_|  |___/     
                          |___/                                  
\end{lstlisting}
\end{center}

\fi

\vspace*{\fill}
\end{center}

\end{titlepage}

\newpage

\begin{center}
\vspace*{\fill}
{\LARGE \TitleMain}

\bigskip

{\large (\TitleAux)}

\bigskip
\bigskip
Pourquoi deux titres? Lire ici: \myref{TwoTitles}.

\vspace*{\fill}

{\large \AUTHOR}

{\large \TT{\EMAILPRI}}
\vspace*{\fill}
\vfill

\ccbysa

\textcopyright 2013-2021, \AUTHOR.

Ce travail est sous licence Creative Commons Attribution-ShareAlike 4.0 International (CC BY-SA 4.0).
Pour voir une copie de cette licence, rendez vous sur \url{https://creativecommons.org/licenses/by-sa/4.0/}.

Version du texte ({\large \today}).

La dernière version (et l'édition en russe) de ce texte est accessible sur \url{https://beginners.re/}. 

\end{center}
}
\DE{\begin{titlepage}

\begin{center}
\vspace*{\fill}

\ifdefined\UAL{}

\begin{center}
\begin{lstlisting}[basicstyle=\ttfamily\small,frame=none]
  _   _           _               _                  _ _             
 | | | |_ __   __| | ___ _ __ ___| |_ __ _ _ __   __| (_)_ __   __ _ 
 | | | | '_ \ / _` |/ _ \ '__/ __| __/ _` | '_ \ / _` | | '_ \ / _` |
 | |_| | | | | (_| |  __/ |  \__ \ || (_| | | | | (_| | | | | | (_| |
  \___/|_| |_|\__,_|\___|_|  |___/\__\__,_|_| |_|\__,_|_|_| |_|\__, |
                                                               |___/ 
                _                           _     _                  
               / \   ___ ___  ___ _ __ ___ | |__ | |_   _            
              / _ \ / __/ __|/ _ \ '_ ` _ \| '_ \| | | | |           
             / ___ \\__ \__ \  __/ | | | | | |_) | | |_| |           
            /_/   \_\___/___/\___|_| |_| |_|_.__/|_|\__, |           
                                                    |___/            
            _                                                        
           | |    __ _ _ __   __ _ _   _  __ _  __ _  ___            
           | |   / _` | '_ \ / _` | | | |/ _` |/ _` |/ _ \           
           | |__| (_| | | | | (_| | |_| | (_| | (_| |  __/           
           |_____\__,_|_| |_|\__, |\__,_|\__,_|\__, |\___|           
                             |___/             |___/                 
\end{lstlisting}
\end{center}

\else

\begin{center}
\begin{lstlisting}[basicstyle=\ttfamily,frame=none]
                     ____                                        
                    |  _ \ _____   _____ _ __ ___  ___           
                    | |_) / _ \ \ / / _ \ '__/ __|/ _ \          
                    |  _ <  __/\ V /  __/ |  \__ \  __/          
                    |_| \_\___| \_/ \___|_|  |___/\___|          
                                                                 
          _____             _                      _             
         | ____|_ __   __ _(_)_ __   ___  ___ _ __(_)_ __   __ _ 
         |  _| | '_ \ / _` | | '_ \ / _ \/ _ \ '__| | '_ \ / _` |
         | |___| | | | (_| | | | | |  __/  __/ |  | | | | | (_| |
         |_____|_| |_|\__, |_|_| |_|\___|\___|_|  |_|_| |_|\__, |
                      |___/                                |___/ 
                              __                                 
                             / _| ___  _ __                      
                            | |_ / _ \| '__|                     
                            |  _| (_) | |                        
                            |_|  \___/|_|                        
                                                                 
               ____             _                                
              | __ )  ___  __ _(_)_ __  _ __   ___ _ __ ___      
              |  _ \ / _ \/ _` | | '_ \| '_ \ / _ \ '__/ __|     
              | |_) |  __/ (_| | | | | | | | |  __/ |  \__ \     
              |____/ \___|\__, |_|_| |_|_| |_|\___|_|  |___/     
                          |___/                                  
\end{lstlisting}
\end{center}

\fi

\vspace*{\fill}
\end{center}

\end{titlepage}

\newpage

\begin{center}
\vspace*{\fill}
{\LARGE \TitleMain}

\bigskip

{\large (\TitleAux)}

\bigskip
\bigskip
Why two titles? Read here: \myref{TwoTitles}. % TBT

\vspace*{\fill}

{\large \AUTHOR}

{\large \TT{\EMAILPRI}}
\vspace*{\fill}
\vfill

\ccbysa

\textcopyright 2013-2021, \AUTHOR. 

Diese Arbeit ist lizenziert unter der Creative Commons Attribution-ShareAlike 4.0 International (CC BY-SA 4.0) Lizenz.
Um eine Kopie dieser Lizenz zu lesen, besuchen Sie \url{https://creativecommons.org/licenses/by-sa/4.0/}.

Text-Version ({\large \today}).

Die aktuellste Version (und eine russische Ausgabe) dieses Textes ist auf \url{https://beginners.re/} verfügbar.
Eine Version für den E-Book-Leser ist dort ebenfalls erhältlich.

\end{center}
}
\JA{\begin{titlepage}

\begin{center}
\vspace*{\fill}

\ifdefined\UAL{}

\begin{center}
\begin{lstlisting}[basicstyle=\ttfamily\small,frame=none]
  _   _           _               _                  _ _             
 | | | |_ __   __| | ___ _ __ ___| |_ __ _ _ __   __| (_)_ __   __ _ 
 | | | | '_ \ / _` |/ _ \ '__/ __| __/ _` | '_ \ / _` | | '_ \ / _` |
 | |_| | | | | (_| |  __/ |  \__ \ || (_| | | | | (_| | | | | | (_| |
  \___/|_| |_|\__,_|\___|_|  |___/\__\__,_|_| |_|\__,_|_|_| |_|\__, |
                                                               |___/ 
                _                           _     _                  
               / \   ___ ___  ___ _ __ ___ | |__ | |_   _            
              / _ \ / __/ __|/ _ \ '_ ` _ \| '_ \| | | | |           
             / ___ \\__ \__ \  __/ | | | | | |_) | | |_| |           
            /_/   \_\___/___/\___|_| |_| |_|_.__/|_|\__, |           
                                                    |___/            
            _                                                        
           | |    __ _ _ __   __ _ _   _  __ _  __ _  ___            
           | |   / _` | '_ \ / _` | | | |/ _` |/ _` |/ _ \           
           | |__| (_| | | | | (_| | |_| | (_| | (_| |  __/           
           |_____\__,_|_| |_|\__, |\__,_|\__,_|\__, |\___|           
                             |___/             |___/                 
\end{lstlisting}
\end{center}

\else

\begin{center}
\begin{lstlisting}[basicstyle=\ttfamily,frame=none]
                     ____                                        
                    |  _ \ _____   _____ _ __ ___  ___           
                    | |_) / _ \ \ / / _ \ '__/ __|/ _ \          
                    |  _ <  __/\ V /  __/ |  \__ \  __/          
                    |_| \_\___| \_/ \___|_|  |___/\___|          
                                                                 
          _____             _                      _             
         | ____|_ __   __ _(_)_ __   ___  ___ _ __(_)_ __   __ _ 
         |  _| | '_ \ / _` | | '_ \ / _ \/ _ \ '__| | '_ \ / _` |
         | |___| | | | (_| | | | | |  __/  __/ |  | | | | | (_| |
         |_____|_| |_|\__, |_|_| |_|\___|\___|_|  |_|_| |_|\__, |
                      |___/                                |___/ 
                              __                                 
                             / _| ___  _ __                      
                            | |_ / _ \| '__|                     
                            |  _| (_) | |                        
                            |_|  \___/|_|                        
                                                                 
               ____             _                                
              | __ )  ___  __ _(_)_ __  _ __   ___ _ __ ___      
              |  _ \ / _ \/ _` | | '_ \| '_ \ / _ \ '__/ __|     
              | |_) |  __/ (_| | | | | | | | |  __/ |  \__ \     
              |____/ \___|\__, |_|_| |_|_| |_|\___|_|  |___/     
                          |___/                                  
\end{lstlisting}
\end{center}

\fi

\vspace*{\fill}
\end{center}

\end{titlepage}

\newpage

\begin{center}
\vspace*{\fill}
{\LARGE \TitleMain}

\bigskip

{\large (\TitleAux)}

\bigskip
\bigskip
どうしてタイトルが2つあるの? 参照: \myref{TwoTitles}

\vspace*{\fill}

{\large \AUTHOR}

{\large \TT{\EMAILPRI}}
\vspace*{\fill}
\vfill

\ccbysa

\textcopyright 2013-2021, \AUTHOR. 

この作品は、クリエイティブ・コモンズの表示 - 継承 4.0 国際(CC BY-SA 4.0)の下でライセンスされています。
このライセンスのコピーを表示するには、\url{https://creativecommons.org/licenses/by-sa/4.0/}を訪れてください。

Text version ({\large \today}).

この本の最新版(そしてロシア版)は以下で見ることができます: \url{https://beginners.re/}.

\end{center}
}
\PL{\begin{titlepage}

\begin{center}
\vspace*{\fill}

\ifdefined\UAL{}

\begin{center}
\begin{lstlisting}[basicstyle=\ttfamily\small,frame=none]
  _   _           _               _                  _ _             
 | | | |_ __   __| | ___ _ __ ___| |_ __ _ _ __   __| (_)_ __   __ _ 
 | | | | '_ \ / _` |/ _ \ '__/ __| __/ _` | '_ \ / _` | | '_ \ / _` |
 | |_| | | | | (_| |  __/ |  \__ \ || (_| | | | | (_| | | | | | (_| |
  \___/|_| |_|\__,_|\___|_|  |___/\__\__,_|_| |_|\__,_|_|_| |_|\__, |
                                                               |___/ 
                _                           _     _                  
               / \   ___ ___  ___ _ __ ___ | |__ | |_   _            
              / _ \ / __/ __|/ _ \ '_ ` _ \| '_ \| | | | |           
             / ___ \\__ \__ \  __/ | | | | | |_) | | |_| |           
            /_/   \_\___/___/\___|_| |_| |_|_.__/|_|\__, |           
                                                    |___/            
            _                                                        
           | |    __ _ _ __   __ _ _   _  __ _  __ _  ___            
           | |   / _` | '_ \ / _` | | | |/ _` |/ _` |/ _ \           
           | |__| (_| | | | | (_| | |_| | (_| | (_| |  __/           
           |_____\__,_|_| |_|\__, |\__,_|\__,_|\__, |\___|           
                             |___/             |___/                 
\end{lstlisting}
\end{center}

\else

\begin{center}
\begin{lstlisting}[basicstyle=\ttfamily,frame=none]
                     ____                                        
                    |  _ \ _____   _____ _ __ ___  ___           
                    | |_) / _ \ \ / / _ \ '__/ __|/ _ \          
                    |  _ <  __/\ V /  __/ |  \__ \  __/          
                    |_| \_\___| \_/ \___|_|  |___/\___|          
                                                                 
          _____             _                      _             
         | ____|_ __   __ _(_)_ __   ___  ___ _ __(_)_ __   __ _ 
         |  _| | '_ \ / _` | | '_ \ / _ \/ _ \ '__| | '_ \ / _` |
         | |___| | | | (_| | | | | |  __/  __/ |  | | | | | (_| |
         |_____|_| |_|\__, |_|_| |_|\___|\___|_|  |_|_| |_|\__, |
                      |___/                                |___/ 
                              __                                 
                             / _| ___  _ __                      
                            | |_ / _ \| '__|                     
                            |  _| (_) | |                        
                            |_|  \___/|_|                        
                                                                 
               ____             _                                
              | __ )  ___  __ _(_)_ __  _ __   ___ _ __ ___      
              |  _ \ / _ \/ _` | | '_ \| '_ \ / _ \ '__/ __|     
              | |_) |  __/ (_| | | | | | | | |  __/ |  \__ \     
              |____/ \___|\__, |_|_| |_|_| |_|\___|_|  |___/     
                          |___/                                  
\end{lstlisting}
\end{center}

\fi

\vspace*{\fill}
\end{center}

\end{titlepage}

\newpage

\begin{center}
\vspace*{\fill}
{\LARGE \TitleMain}

\bigskip

{\large (\TitleAux)}

\bigskip
\bigskip
Dlaczego aż dwa tytuły? Przeczytaj tutaj: \myref{TwoTitles}.

\vspace*{\fill}

{\large \AUTHOR}

{\large \TT{\EMAILPRI}}
\vspace*{\fill}
\vfill

\ccbysa

\textcopyright 2013-2021, \AUTHOR. 

To dzieło jest na licencji Creative Commons Attribution-ShareAlike 4.0 International (CC BY-SA 4.0).
Kopia licencji do wglądu znajduje się na \url{https://creativecommons.org/licenses/by-sa/4.0/}.

Wersja tekstu ({\large \today}).

Aktualna wersja (a także wersja rosyjska) tego tekstu znajduje się na \url{https://beginners.re/}.

\end{center}
}

\EN{\vspace*{\fill}

\Huge Call for translators!

\normalsize

\bigskip
\bigskip
\bigskip

You may want to help me with translating this work into languages other than English and Russian.
Just send me any piece of translated text (no matter how short) and I'll put it into my LaTeX source code.

Do not ask, if you should translate. Just do something. I stopped responding ``what should I do'' emails.

\href{\RepoURL/Translation.md}{Also, read here}.

The language statistics is available right here: \url{https://beginners.re/}.

Speed isn't important, because this is an open-source project, after all.
Your name will be mentioned as a project contributor.
Korean, Chinese, and Persian languages are reserved by publishers.
English and Russian versions I do by myself, but my English is still that horrible, so I'm very grateful for any notes about grammar, etc.
Even my Russian is flawed, so I'm grateful for notes about Russian text as well!%

So do not hesitate to contact me: \GTT{\EMAILS}.

\vspace*{\fill}
\vfill
}
\RU{\vspace*{\fill}

\Huge Нужны переводчики!
\normalsize

\bigskip
\bigskip
\bigskip

Возможно, вы захотите мне помочь с переводом этой работы на другие языки, кроме английского и русского.
Просто пришлите мне любой фрагмент переведенного текста (не важно, насколько короткий), и я добавлю его в исходный код на LaTeX.

Не спрашивайте, нужно ли переводить. Просто делайте хоть что-нибудь. Я уже перестал отвечать на емейлы вроде ``что нужно сделать?''

\href{\RepoURL/Translation.md}{Также, прочитайте это}.

Посмотреть статистику языков можно прямо здесь: \url{https://beginners.re/}.

Скорость не важна, потому что это опен-сорсный проект все-таки.
Ваше имя будет указано в числе участников проекта.
Корейский, китайский и персидский языки зарезервированы издателями.
Английскую и русскую версии я делаю сам, но английский у меня все еще ужасный, так что я буду очень признателен за коррективы, итд.
Даже мой русский несовершенный, так что я благодарен за коррективы и русского текста!

Не стесняйтесь писать мне: \GTT{\EMAILS}.

\vspace*{\fill}
\vfill
}
\DE{\vspace*{\fill}

\Huge Übersetzer gesucht!

\normalsize

\bigskip
\bigskip
\bigskip

Vielleicht möchten Sie mir bei der Übersetzung dieser Arbeit in andere Sprachen (außer Englisch und
Russisch) helfen. Senden Sie mir einen übersetzten Textteil, egal wie kurz und ich arbeite ihn in den
\LaTeX{}-Quellcode ein.

% TBT
% Do not ask, if you should translate. Just do something. I stopped responding ``what should I do'' emails.

\href{\RepoURL/Translation.md}{Hier lesen}.

% TBT
% The language statistics is available right here: \url{https://beginners.re/}.

Geschwindigkeit ist nicht wichtig, denn im Endeffekt ist es ein Open-Source-Projekt.
Ihr Name wird als Mitwirkender des Projekts erwähnt.
Koreanisch, Chinesisch und Persisch sind für Verleger reserviert.
Eine englische und russische Version mache ich selber, allerdings ist mein Englisch immer noch schrecklich.
Ich bin also dankbar über alle Anmerkungen bezüglich der Grammatik und Ähnlichem.
Auch mein Russisch ist teilweise fehlerhaft, also bin ich auch hier dankbar über Anmerkungen!%

Also zögern Sie nicht mir zu schreiben: \GTT{\EMAILS}.

\vspace*{\fill}
\vfill
}
\FR{\vspace*{\fill}

\Huge À la recherche de traducteurs !

\normalsize

\bigskip
\bigskip
\bigskip

Vous souhaitez peut-être m'aider en traduisant ce projet dans d'autres langues, autres que l'anglais et le russe.
Il vous suffit de m'envoyer les portions de texte que vous avez traduites (peu importe leur longueur) et je les intégrerai à mon code source LaTeX.

Ne demandez pas si vous pouvez traduire. Faîtes simplement quelque chose. J'ai
arrêté de répondre aux courriels ``Que puis-je faire''.

\href{\RepoURL/Translation.md}{Lire ici}.

Les statistiques par langage sont disponible ici: \url{https://beginners.re/}.

La vitesse de traduction n'est pas importante, puisqu'il s'agit d'un projet open-source après tout.
Votre nom sera mentionné en tant que contributeur au projet.
Les traductions en coréen, chinois et persan sont réservées par mes éditeurs.

Les versions anglaise et russe ont été réalisées par moi-même.
Toutefois, mon anglais est toujours horrible et je vous serais très reconnaissant pour toute éventuelle remarque sur la grammaire, etc...
Même mon russe est imparfait, donc je vous serais également reconnaissant pour toute remarque sur la traduction en russe !%

N'hésitez donc pas à me contacter : \GTT{\EMAILS}.

\vspace*{\fill}
\vfill
}
\IT{\vspace*{\fill}

\Huge Cerchiamo traduttori!

\normalsize

\bigskip
\bigskip
\bigskip

Puoi aiutare a tradurre questo progetto in linguaggi diversi dall'Inglese ed il Russo.
Basta inviarmi un qualsiasi pezzo di testo tradotto (non importa quanto è lungo) e lo aggiungerò al mio codice sorgente scritto in LaTeX.

% TBT
% Do not ask, if you should translate. Just do something. I stopped responding ``what should I do'' emails.

\href{\RepoURL/Translation.md}{Leggi qui}.

% TBT
% The language statistics is available right here: \url{https://beginners.re/}.

La velocità non è importante, perchè è un progetto Open Source, dopo tutto.
Il tuo nome sarà menzionato come Contributore del progetto.
Coreano, Cinese e Persiano sono linguaggi reservati ai publisher.
Le versioni in Inglese e Russo le traduco da solo, ma il mio Inglese è ancora terribile, quindi vi sono grato ad ogni correzione sulla mia grammatica, ecc...
Anche il mio Russo non è ancora perfetto, quindi sono felice di ricevere correzioni anche per il Russo!%

Non esitate a contattarmi: \GTT{\EMAILS}.

\vspace*{\fill}
\vfill
}
\PL{\vspace*{\fill}

\Huge Potrzebujemy tłumaczy!
\normalsize

\bigskip
\bigskip
\bigskip

Jeśli chcesz pomóc z tłumaczeniem tej książki na języki inne niż angielski i rosyjski,
prześlij mailem fragment przetłumaczonego tekstu (obojętnie jakiej długości) a ja go dodam do kodu źródłowego.

Nie pytaj czy powinieneś tłumaczyć. Po prostu zrób coś. Przestałem odpowiadać na wiadomości ``co powinienem zrobić''.

\href{\RepoURL/Translation.md}{Czytać tutaj}.

Statystyka tłumaczeń na inne języki znajduje się tutaj: \url{https://beginners.re/}.

Prędkość nie gra roli, jako że jest to projekt open-source.
Twoje imię pojawi się obok imion innych uczestników projektu.
Koreański, chiński i perski są zarezerwowane przez wydawców.
Wersją angielską i rosyjską zajmuję się sam, ale mój angielski nadal jest daleki od ideału, więc będę wdzięczny za korekty.
Nawet mój rosyjski nie jest idealny, więc będę wdzięczny również za korekty tekstu w języku rosyjskim!

Śmiało piszcie do mnie: \GTT{\EMAILS}.

\vspace*{\fill}
\vfill

}
\JA{\vspace*{\fill}

\Huge 翻訳者求む!

\normalsize

\bigskip
\bigskip
\bigskip

この作品を英語とロシア語以外の言語に翻訳するのを手伝ってください。
どのように翻訳されたテキストを私に送っても(どれほど短くても)、私はLaTeXのソースコードに入れます。

% TBT
% Do not ask, if you should translate. Just do something. I stopped responding ``what should I do'' emails.

\href{\RepoURL/Translation.md}{ここを読んでください}.

% TBT
% The language statistics is available right here: \url{https://beginners.re/}.

スピードは重要ではありません。なぜなら、これはオープンソースプロジェクトなのですから。
あなたの名前はプロジェクト寄稿者として言及されます。
韓国語、中国語、ペルシャ語は出版社によって予約されています。
英語とロシア語のバージョンは自分でやっていますが、私の英語はまだひどいので、文法などに関するメモにはとても感謝しています。
私のロシア語にも欠陥があるので、ロシア語のテキストについての注釈にも感謝しています!

だから私に連絡するのをためらうことはありません: \GTT{\EMAILS}

\vspace*{\fill}
\vfill
}



\shorttoc{%
    \RU{Краткое оглавление}%
    \EN{Abridged contents}%
    \ES{Contenidos abreviados}%
    \PTBRph{}%
    \DE{Inhaltsverzeichnis (gekürzt)}%
    \PL{Skrócony spis treści}%
    \IT{Sommario}%
    \THAph{}\NLph{}%
    \FR{Contenus abrégés}%
    \JA{簡略版}
    \TR{İçindekiler}
}{0}

\tableofcontents
\cleardoublepage

\cleardoublepage
\EN{\section*{Preface}

\subsection*{What is with two titles?}
\label{TwoTitles}

The book was named ``Reverse Engineering for Beginners'' in 2014-2018, but I always suspected this makes readership too narrow.

Infosec people know about ``reverse engineering'', but I've rarely hear the ``assembler'' word from them.

Likewise, the ``reverse engineering'' term is somewhat cryptic to a general audience of programmers, but they know about ``assembler''.

In July 2018, as an experiment, I've changed the title to ``Assembly Language for Beginners''
and posted the link to Hacker News website\footnote{\url{https://news.ycombinator.com/item?id=17549050}}, and the book was received generally well.

So let it be, the book now has two titles.

However, I've changed the second title to ``Understanding Assembly Language'',
because someone had already written ``Assembly Language for Beginners'' book.
Also, people say ``for Beginners'' sounds a bit sarcastic for a book of \textasciitilde{}1000 pages.

The two books differ only by title, filename (UAL-XX.pdf versus RE4B-XX.pdf),
URL and a couple of the first pages.

\subsection*{About reverse engineering}

There are several popular meanings of the term \q{\gls{reverse engineering}}:

1) The reverse engineering of software; researching compiled programs

2) The scanning of 3D structures and the subsequent digital manipulation required in order to duplicate them

3) Recreating \ac{DBMS} structure

This book is about the first meaning.

\subsection*{Prerequisites}

Basic knowledge of the C \ac{PL}.
Recommended reading: \myref{CCppBooks}.

\subsection*{Exercises and tasks}

\dots
can be found at: \url{http://challenges.re}.

\iffalse
\subsection*{About the author}
\begin{tabularx}{\textwidth}{ l X }

\raisebox{-\totalheight}{
\includegraphics[scale=0.60]{Dennis_Yurichev.jpg}
}

&
Dennis Yurichev is an experienced reverse engineer and programmer.
He can be contacted by email: \textbf{\EMAILS{}}.

% FIXME: no link. \tablefootnote doesn't work
\end{tabularx}
\fi

% subsections:
\subsection*{Praise for this book}

\url{https://beginners.re/\#praise}.


\subsection*{Universities}

The book is recommended at least at these universities:
\url{https://beginners.re/\#uni}.


\ifdefined\RUSSIAN
\newcommand{\PeopleMistakesInaccuraciesRusEng}{Александр Лысенко, Федерико Рамондино, Марк Уилсон, Разихова Мейрамгуль Кайратовна, Анатолий Прокофьев, Костя Бегунец, Валентин ``netch'' Нечаев, Александр Плахов, Артем Метла, Александр Ястребов, Влад Головкин\footnote{goto-vlad@github}, Евгений Прошин, Александр Мясников, Алексей Третьяков, Олег Песков, Павел Шахов}
\else
\newcommand{\PeopleMistakesInaccuraciesRusEng}{Alexander Lysenko, Federico Ramondino, Mark Wilson, Razikhova Meiramgul Kayratovna, Anatoly Prokofiev, Kostya Begunets, Valentin ``netch'' Nechayev, Aleksandr Plakhov, Artem Metla, Alexander Yastrebov, Vlad Golovkin\footnote{goto-vlad@github}, Evgeny Proshin, Alexander Myasnikov, Alexey Tretiakov, Oleg Peskov, Pavel Shakhov}
\fi

\newcommand{\PeopleMistakesInaccuracies}{\PeopleMistakesInaccuraciesRusEng{}, Zhu Ruijin, Changmin Heo, Vitor Vidal, Stijn Crevits, Jean-Gregoire Foulon\footnote{\url{https://github.com/pixjuan}}, Ben L., Etienne Khan, Norbert Szetei\footnote{\url{https://github.com/73696e65}}, Marc Remy, Michael Hansen, Derk Barten, The Renaissance\footnote{\url{https://github.com/TheRenaissance}}, Hugo Chan, Emil Mursalimov, Tanner Hoke, Tan90909090@GitHub, Ole Petter Orhagen, Sourav Punoriyar, Vitor Oliveira, Alexis Ehret, Maxim Shlochiski,
Greg Paton, Pierrick Lebourgeois, Abdullah Alomair.}

\newcommand{\PeopleItalianTranslators}{Federico Ramondino\footnote{\url{https://github.com/pinkrab}},
Paolo Stivanin\footnote{\url{https://github.com/paolostivanin}}, twyK, Fabrizio Bertone, Matteo Sticco, Marco Negro\footnote{\url{https://github.com/Internaut401}}, bluepulsar}

\newcommand{\PeopleFrenchTranslators}{Florent Besnard\footnote{\url{https://github.com/besnardf}}, Marc Remy\footnote{\url{https://github.com/mremy}}, Baudouin Landais, Téo Dacquet\footnote{\url{https://github.com/T30rix}}, BlueSkeye@GitHub\footnote{\url{https://github.com/BlueSkeye}}}

\newcommand{\PeopleGermanTranslators}{Dennis Siekmeier\footnote{\url{https://github.com/DSiekmeier}},
Julius Angres\footnote{\url{https://github.com/JAngres}}, Dirk Loser\footnote{\url{https://github.com/PolymathMonkey}}, Clemens Tamme, Philipp Schweinzer}

\newcommand{\PeopleSpanishTranslators}{Diego Boy, Luis Alberto Espinosa Calvo, Fernando Guida, Diogo Mussi, Patricio Galdames,
Emiliano Estevarena}

\newcommand{\PeoplePTBRTranslators}{Thales Stevan de A. Gois, Diogo Mussi, Luiz Filipe, Primo David Santini}

\newcommand{\PeoplePolishTranslators}{Kateryna Rozanova, Aleksander Mistewicz, Wiktoria Lewicka, Marcin Sokołowski}

\newcommand{\PeopleJapaneseTranslators}{%
shmz@github\footnote{\url{https://github.com/shmz}},%
4ryuJP@github\footnote{\url{https://github.com/4ryuJP}}}

\EN{\input{thanks_EN}}
\ES{\input{thanks_ES}}
\NL{\input{thanks_NL}}
\RU{\input{thanks_RU}}
\IT{\input{thanks_IT}}
\FR{\input{thanks_FR}}
\DE{\input{thanks_DE}}
%\CN{\input{thanks_CN}}
\JA{\input{thanks_JA}}
\PL{\input{thanks_PL}}
\CN{\input{thanks_CN}}


\subsection*{mini-FAQ}

\par Q: Is this book simpler/easier than others?
\par A: No, it is at about the same level as other books of this subject.

\par Q: I'm too frightened to start reading this book, there are more than 1000 pages.
"...for Beginners" in the name sounds a bit sarcastic.
\par A: All sorts of listings are the bulk of the book.
The book is indeed for beginners, there is a lot missing (yet).

\par Q: What are the prerequisites for reading this book?
\par A: A basic understanding of C/C++ is desirable.

\par Q: Should I really learn x86/x64/ARM and MIPS at once? Isn't it too much?
\par A: Starters can read about just x86/x64, while skipping or skimming the ARM and MIPS parts.

\par Q: Can I buy a Russian or English hard copy/paper book?
\par A: Unfortunately, no. No publisher got interested in publishing a Russian or English version so far.
Meanwhile, you can ask your favorite copy shop to print and bind it.
\url{https://yurichev.com/news/20200222_printed_RE4B/}.

\par Q: Is there an epub or mobi version?
\par A: No. The book is highly dependent on TeX/LaTeX-specific hacks, so converting to HTML (epub/mobi are a set of HTMLs)
would not be easy.

\par Q: Why should one learn assembly language these days?
\par A: Unless you are an \ac{OS} developer, you probably don't need to code in assembly\textemdash{}the latest compilers (2010s) are much better at performing optimizations than humans \footnote{A very good text on this topic: \InSqBrackets{\AgnerFog}}.

Also, the latest \ac{CPU}s are very complex devices, and assembly knowledge doesn't really help towards understand their internals.

That being said, there are at least two areas where a good understanding of assembly can be helpful:
First and foremost, for security/malware research. It is also a good way to gain a better understanding of your compiled code while debugging.
This book is therefore intended for those who want to understand assembly language rather
than to code in it, which is why there are many examples of compiler output contained within.

\par Q: I clicked on a hyperlink inside a PDF-document, how do I go back?
\par A: In Adobe Acrobat Reader click Alt+LeftArrow. In Evince click ``<'' button.

\par Q: May I print this book / use it for teaching?
\par A: Of course! That's why the book is licensed under the Creative Commons license (CC BY-SA 4.0).

\par Q: Why is this book free? You've done great job. This is suspicious, as with many other free things.
\par A: In my own experience, authors of technical literature write mostly for self-advertisement purposes.
It's not possible to make any decent money from such work.

\par Q: How does one get a job in reverse engineering?
\par A: There are hiring threads that appear from time to time on reddit, devoted to RE\FNURLREDDIT{}.
Try looking there.

A somewhat related hiring thread can be found in the \q{netsec} subreddit.

\par Q: I have a question...
\par A: Send it to me by email (\EMAILS).


\subsection*{About the Korean translation}

In January 2015, the Acorn publishing company (\href{http://www.acornpub.co.kr}{www.acornpub.co.kr}) in South Korea did a huge amount of work in translating and publishing
this book (as it was in August 2014) into Korean.

It's available now at \href{http://www.acornpub.co.kr/book/reversing-for-beginners}{their website}.

\iffalse
\begin{figure}[H]
\centering
\includegraphics[scale=0.3]{acorn_cover.jpg}
\end{figure}
\fi

The translator is Byungho Min (\href{https://twitter.com/tais9}{twitter/tais9}).
The cover art was done by the artistic Andy Nechaevsky, a friend of the author:
\href{https://www.facebook.com/andydinka}{facebook/andydinka}.
Acorn also holds the copyright to the Korean translation.

So, if you want to have a \emph{real} book on your shelf in Korean and
want to support this work, it is now available for purchase.

\subsection*{About the Persian/Farsi translation}

In 2016 the book was translated by Mohsen Mostafa Jokar (who is also known to Iranian community for his translation of Radare manual\footnote{\url{http://rada.re/get/radare2book-persian.pdf}}).
It is available on the publisher’s website\footnote{\url{http://goo.gl/2Tzx0H}} (Pendare Pars).

Here is a link to a 40-page excerpt: \url{https://beginners.re/farsi.pdf}.

National Library of Iran registration information: \url{http://opac.nlai.ir/opac-prod/bibliographic/4473995}.

\subsection*{About the Chinese translation}

In April 2017, translation to Chinese was completed by Chinese PTPress. They are also the Chinese translation copyright holders.

 The Chinese version is available for order here: \url{http://www.epubit.com.cn/book/details/4174}. A partial review and history behind the translation can be found here: \url{http://www.cptoday.cn/news/detail/3155}.

The principal translator is Archer, to whom the author owes very much. He was extremely meticulous (in a good sense) and reported most of the known mistakes and bugs, which is very important in literature such as this book.
The author would recommend his services to any other author!

The guys from \href{http://www.antiy.net/}{Antiy Labs} has also helped with translation. \href{http://www.epubit.com.cn/book/onlinechapter/51413}{Here is preface} written by them.

}
\RU{\section*{Предисловие}

\subsection*{Почему два названия?}
\label{TwoTitles}

В 2014-2018 книга называлась ``Reverse Engineering для начинающих'', но я всегда подозревал что это слишком сужает аудиторию.

Люди от инфобезопасности знают о ``reverse engineering'', но я от них редко слышу слово ``ассемблер''.

Точно также, термин ``reverse engineering'' слишком незнакомый для общей аудитории программистов, но они знают про ``ассемблер''.

В июле 2018, для эксперимента, я заменил название на ``Assembly Language for Beginners''
и запостил ссылку на сайт Hacker News\footnote{\url{https://news.ycombinator.com/item?id=17549050}}, и книгу приняли, в общем, хорошо.

Так что, пусть так и будет, у книги будет два названия.

Хотя, я поменял второе название на ``Understanding Assembly Language'' (``Понимание языка ассемблера''), потому что кто-то уже написал книгу ``Assembly Language for Beginners''.
Также, люди говорят что ``для начинающих'' уже звучит немного саркастично для книги объемом в \textasciitilde{}1000 страниц.

Книги отличаются только названием, именем файла (UAL-XX.pdf и RE4B-XX.pdf), URL-ом и парой первых страниц.

\subsection*{О reverse engineering}

У термина \q{\gls{reverse engineering}} несколько популярных значений:
1) исследование скомпилированных
программ;
2) сканирование трехмерной модели для последующего копирования;
3) восстановление структуры СУБД.

Настоящая книга связана с первым значением.

\subsection*{Желательные знания перед началом чтения}

Очень желательно базовое знание \ac{PL} Си.
Рекомендуемые материалы: \myref{CCppBooks}.

\subsection*{Упражнения и задачи}

\dots 
все перемещены на отдельный сайт: \url{http://challenges.re}.

\iffalse
\subsection*{Об авторе}
\begin{tabularx}{\textwidth}{ l X }

\raisebox{-\totalheight}{
\includegraphics[scale=0.60]{Dennis_Yurichev.jpg}
}

&
Денис Юричев~--- опытный reverse engineer и программист.
С ним можно контактировать по емейлу: \textbf{\EMAILS{}}.

% FIXME: no link. \tablefootnote doesn't work
\end{tabularx}
\fi

% subsections:
\subsection*{Отзывы об этой книге}

\url{https://beginners.re/\#praise}.


\subsection*{Университеты}

Эта книга рекомендуется по крайне мере в этих университетах:
\url{https://beginners.re/\#uni}.


\ifdefined\RUSSIAN
\newcommand{\PeopleMistakesInaccuraciesRusEng}{Александр Лысенко, Федерико Рамондино, Марк Уилсон, Разихова Мейрамгуль Кайратовна, Анатолий Прокофьев, Костя Бегунец, Валентин ``netch'' Нечаев, Александр Плахов, Артем Метла, Александр Ястребов, Влад Головкин\footnote{goto-vlad@github}, Евгений Прошин, Александр Мясников, Алексей Третьяков, Олег Песков, Павел Шахов}
\else
\newcommand{\PeopleMistakesInaccuraciesRusEng}{Alexander Lysenko, Federico Ramondino, Mark Wilson, Razikhova Meiramgul Kayratovna, Anatoly Prokofiev, Kostya Begunets, Valentin ``netch'' Nechayev, Aleksandr Plakhov, Artem Metla, Alexander Yastrebov, Vlad Golovkin\footnote{goto-vlad@github}, Evgeny Proshin, Alexander Myasnikov, Alexey Tretiakov, Oleg Peskov, Pavel Shakhov}
\fi

\newcommand{\PeopleMistakesInaccuracies}{\PeopleMistakesInaccuraciesRusEng{}, Zhu Ruijin, Changmin Heo, Vitor Vidal, Stijn Crevits, Jean-Gregoire Foulon\footnote{\url{https://github.com/pixjuan}}, Ben L., Etienne Khan, Norbert Szetei\footnote{\url{https://github.com/73696e65}}, Marc Remy, Michael Hansen, Derk Barten, The Renaissance\footnote{\url{https://github.com/TheRenaissance}}, Hugo Chan, Emil Mursalimov, Tanner Hoke, Tan90909090@GitHub, Ole Petter Orhagen, Sourav Punoriyar, Vitor Oliveira, Alexis Ehret, Maxim Shlochiski,
Greg Paton, Pierrick Lebourgeois, Abdullah Alomair.}

\newcommand{\PeopleItalianTranslators}{Federico Ramondino\footnote{\url{https://github.com/pinkrab}},
Paolo Stivanin\footnote{\url{https://github.com/paolostivanin}}, twyK, Fabrizio Bertone, Matteo Sticco, Marco Negro\footnote{\url{https://github.com/Internaut401}}, bluepulsar}

\newcommand{\PeopleFrenchTranslators}{Florent Besnard\footnote{\url{https://github.com/besnardf}}, Marc Remy\footnote{\url{https://github.com/mremy}}, Baudouin Landais, Téo Dacquet\footnote{\url{https://github.com/T30rix}}, BlueSkeye@GitHub\footnote{\url{https://github.com/BlueSkeye}}}

\newcommand{\PeopleGermanTranslators}{Dennis Siekmeier\footnote{\url{https://github.com/DSiekmeier}},
Julius Angres\footnote{\url{https://github.com/JAngres}}, Dirk Loser\footnote{\url{https://github.com/PolymathMonkey}}, Clemens Tamme, Philipp Schweinzer}

\newcommand{\PeopleSpanishTranslators}{Diego Boy, Luis Alberto Espinosa Calvo, Fernando Guida, Diogo Mussi, Patricio Galdames,
Emiliano Estevarena}

\newcommand{\PeoplePTBRTranslators}{Thales Stevan de A. Gois, Diogo Mussi, Luiz Filipe, Primo David Santini}

\newcommand{\PeoplePolishTranslators}{Kateryna Rozanova, Aleksander Mistewicz, Wiktoria Lewicka, Marcin Sokołowski}

\newcommand{\PeopleJapaneseTranslators}{%
shmz@github\footnote{\url{https://github.com/shmz}},%
4ryuJP@github\footnote{\url{https://github.com/4ryuJP}}}

\EN{\input{thanks_EN}}
\ES{\input{thanks_ES}}
\NL{\input{thanks_NL}}
\RU{\input{thanks_RU}}
\IT{\input{thanks_IT}}
\FR{\input{thanks_FR}}
\DE{\input{thanks_DE}}
%\CN{\input{thanks_CN}}
\JA{\input{thanks_JA}}
\PL{\input{thanks_PL}}
\CN{\input{thanks_CN}}


\subsection*{mini-ЧаВО}

\par Q: Эта книга проще/легче других?
\par A: Нет, примерно на таком же уровне, как и остальные книги посвященные этой теме.

\par Q: Мне страшно начинать читать эту книгу, здесь более 1000 страниц.
"... для начинающих" в названии звучит слегка саркастично.
\par A: Основная часть книги это масса разных листингов.
И эта книга действительно для начинающих, тут многого (пока) не хватает.

\par Q: Что необходимо знать перед чтением книги?
\par A: Желательно иметь базовое понимание Си/Си++.

\par Q: Должен ли я изучать сразу x86/x64/ARM и MIPS? Это не многовато?
\par A: Для начала, вы можете читать только о x86/x64, пропуская/пролистывая части о ARM/MIPS.

\par Q: Возможно ли купить русскую/английскую бумажную книгу?
\par A: К сожалению нет, пока ни один издатель не заинтересовался в издании русской или английской версии.
А пока вы можете распечатать/переплести её в вашем любимом копи-шопе/копи-центре.
\url{https://yurichev.com/news/20200222_printed_RE4B/}.

\par Q: Существует ли версия epub/mobi?
\par A: Книга очень сильно завязана на специфические для TeX/LaTeX хаки, поэтому преобразование в HTML (epub/mobi это набор HTML)
легким не будет.

\par Q: Зачем в наше время нужно изучать язык ассемблера?
\par A: Если вы не разработчик \ac{OS}, вам наверное не нужно писать на ассемблере: современные компиляторы (2010-ые) оптимизируют код намного лучше человека
\footnote{Очень хороший текст на эту тему: \InSqBrackets{\AgnerFog}}.

К тому же, современные \ac{CPU} это крайне сложные устройства и знание ассемблера вряд ли
поможет узнать их внутренности.

Но все-таки остается по крайней мере две области, где знание ассемблера может хорошо помочь:
1) исследование malware (\emph{зловредов}) с целью анализа; 2) лучшее понимание
вашего скомпилированного кода в процессе отладки.
Таким образом, эта книга предназначена для тех, кто хочет скорее понимать ассемблер,
нежели писать на нем, и вот почему здесь масса примеров, связанных с результатами
работы компиляторов.

\par Q: Я кликнул на ссылку внутри PDF-документа, как теперь вернуться назад?
\par A: В Adobe Acrobat Reader нажмите сочетание Alt+LeftArrow. В Evince кликните на ``<''.

\par Q: Могу ли я распечатать эту книгу? Использовать её для обучения?
\par A: Конечно, поэтому книга и лицензирована под лицензией Creative Commons (CC BY-SA 4.0).

\par Q: Почему эта книга бесплатная? Вы проделали большую работу. Это подозрительно, как и многие другие бесплатные вещи.
\par A: По моему опыту, авторы технической литературы делают это, в основном ради саморекламы.
Такой работой заработать приличные деньги невозможно.

\par Q: Как можно найти работу reverse engineer-а?
\par A: На reddit, посвященному RE\FNURLREDDIT, время от времени бывают hiring thread.
Посмотрите там.

В смежном субреддите \q{netsec} имеется похожий тред.

\par Q: У меня есть вопрос...
\par A: Напишите мне его емейлом (\EMAILS).


\subsection*{О переводе на корейский язык}

В январе 2015, издательство Acorn в Южной Корее сделало много работы в переводе 
и издании моей книги (по состоянию на август 2014) на корейский язык.
Она теперь доступна на \href{http://www.acornpub.co.kr/book/reversing-for-beginners}{их сайте}.

\iffalse
\begin{figure}[H]
\centering
\includegraphics[scale=0.3]{acorn_cover.jpg}
\end{figure}
\fi

Переводил Byungho Min (\href{https://twitter.com/tais9}{twitter/tais9}).
Обложку нарисовал мой хороший знакомый художник Андрей Нечаевский
\href{https://www.facebook.com/andydinka}{facebook/andydinka}.
Они также имеют права на издание книги на корейском языке.
Так что если вы хотите иметь \emph{настоящую} книгу на полке на корейском языке и
хотите поддержать мою работу, вы можете купить её.

\subsection*{О переводе на персидский язык (фарси)}

В 2016 году книга была переведена Mohsen Mostafa Jokar (который также известен иранскому сообществу по переводу руководства Radare\footnote{\url{http://rada.re/get/radare2book-persian.pdf}}).
Книга доступна на сайте издательства\footnote{\url{http://goo.gl/2Tzx0H}} (Pendare Pars).

Первые 40 страниц: \url{https://beginners.re/farsi.pdf}.

Регистрация книги в Национальной Библиотеке Ирана: \url{http://opac.nlai.ir/opac-prod/bibliographic/4473995}.

\subsection*{О переводе на китайский язык}

В апреле 2017, перевод на китайский был закончен китайским издательством PTPress. Они также имеют права на издание книги на китайском языке.

Она доступна для заказа здесь: \url{http://www.epubit.com.cn/book/details/4174}. Что-то вроде рецензии и история о переводе: \url{http://www.cptoday.cn/news/detail/3155}.

Основным переводчиком был Archer, перед которым я теперь в долгу.
Он был крайне дотошным (в хорошем смысле) и сообщил о большинстве известных ошибок и баг, что крайне важно для литературы вроде этой книги.
Я буду рекомендовать его услуги всем остальным авторам!

Ребята из \href{http://www.antiy.net/}{Antiy Labs} также помогли с переводом. \href{http://www.epubit.com.cn/book/onlinechapter/51413}{Здесь предисловие} написанное ими.

}
\ES{% TODO to be synced with EN version
\section*{Pr\'ologo}

Existen muchos significados populares para el t\'ermino \q{\gls{reverse engineering}}:
1) La ingenier\'ia inversa de software: la investigaci\'on de programas compilados;
2) El escaneo de estructuras 3D y la manipulaci\'on digital subsecuente requerida para duplicarlas;
3) La recreaci\'on de la estructura de un \ac{DBMS}.
Este libro es acerca del primer significado.

\subsection*{Ejercicios y tareas}

\dots 
fueron movidos al sitio web: \url{http://challenges.re}.

\iffalse
\subsection*{Sobre el autor}
\begin{tabularx}{\textwidth}{ l X }

\raisebox{-\totalheight}{
\includegraphics[scale=0.60]{Dennis_Yurichev.jpg}
}

&
Dennis Yurichev es un reverser y programador experimentado.
Puede ser contactado por email: \textbf{\EMAILS{}}.

% FIXME: no link. \tablefootnote doesn't work
\end{tabularx}
\fi

% subsections:
\subsection*{Elogios para}

\url{https://beginners.re/\#praise}.


% TBT \input{uni_ES}
\ifdefined\RUSSIAN
\newcommand{\PeopleMistakesInaccuraciesRusEng}{Александр Лысенко, Федерико Рамондино, Марк Уилсон, Разихова Мейрамгуль Кайратовна, Анатолий Прокофьев, Костя Бегунец, Валентин ``netch'' Нечаев, Александр Плахов, Артем Метла, Александр Ястребов, Влад Головкин\footnote{goto-vlad@github}, Евгений Прошин, Александр Мясников, Алексей Третьяков, Олег Песков, Павел Шахов}
\else
\newcommand{\PeopleMistakesInaccuraciesRusEng}{Alexander Lysenko, Federico Ramondino, Mark Wilson, Razikhova Meiramgul Kayratovna, Anatoly Prokofiev, Kostya Begunets, Valentin ``netch'' Nechayev, Aleksandr Plakhov, Artem Metla, Alexander Yastrebov, Vlad Golovkin\footnote{goto-vlad@github}, Evgeny Proshin, Alexander Myasnikov, Alexey Tretiakov, Oleg Peskov, Pavel Shakhov}
\fi

\newcommand{\PeopleMistakesInaccuracies}{\PeopleMistakesInaccuraciesRusEng{}, Zhu Ruijin, Changmin Heo, Vitor Vidal, Stijn Crevits, Jean-Gregoire Foulon\footnote{\url{https://github.com/pixjuan}}, Ben L., Etienne Khan, Norbert Szetei\footnote{\url{https://github.com/73696e65}}, Marc Remy, Michael Hansen, Derk Barten, The Renaissance\footnote{\url{https://github.com/TheRenaissance}}, Hugo Chan, Emil Mursalimov, Tanner Hoke, Tan90909090@GitHub, Ole Petter Orhagen, Sourav Punoriyar, Vitor Oliveira, Alexis Ehret, Maxim Shlochiski,
Greg Paton, Pierrick Lebourgeois, Abdullah Alomair.}

\newcommand{\PeopleItalianTranslators}{Federico Ramondino\footnote{\url{https://github.com/pinkrab}},
Paolo Stivanin\footnote{\url{https://github.com/paolostivanin}}, twyK, Fabrizio Bertone, Matteo Sticco, Marco Negro\footnote{\url{https://github.com/Internaut401}}, bluepulsar}

\newcommand{\PeopleFrenchTranslators}{Florent Besnard\footnote{\url{https://github.com/besnardf}}, Marc Remy\footnote{\url{https://github.com/mremy}}, Baudouin Landais, Téo Dacquet\footnote{\url{https://github.com/T30rix}}, BlueSkeye@GitHub\footnote{\url{https://github.com/BlueSkeye}}}

\newcommand{\PeopleGermanTranslators}{Dennis Siekmeier\footnote{\url{https://github.com/DSiekmeier}},
Julius Angres\footnote{\url{https://github.com/JAngres}}, Dirk Loser\footnote{\url{https://github.com/PolymathMonkey}}, Clemens Tamme, Philipp Schweinzer}

\newcommand{\PeopleSpanishTranslators}{Diego Boy, Luis Alberto Espinosa Calvo, Fernando Guida, Diogo Mussi, Patricio Galdames,
Emiliano Estevarena}

\newcommand{\PeoplePTBRTranslators}{Thales Stevan de A. Gois, Diogo Mussi, Luiz Filipe, Primo David Santini}

\newcommand{\PeoplePolishTranslators}{Kateryna Rozanova, Aleksander Mistewicz, Wiktoria Lewicka, Marcin Sokołowski}

\newcommand{\PeopleJapaneseTranslators}{%
shmz@github\footnote{\url{https://github.com/shmz}},%
4ryuJP@github\footnote{\url{https://github.com/4ryuJP}}}

\EN{\input{thanks_EN}}
\ES{\input{thanks_ES}}
\NL{\input{thanks_NL}}
\RU{\input{thanks_RU}}
\IT{\input{thanks_IT}}
\FR{\input{thanks_FR}}
\DE{\input{thanks_DE}}
%\CN{\input{thanks_CN}}
\JA{\input{thanks_JA}}
\PL{\input{thanks_PL}}
\CN{\input{thanks_CN}}


% TODO to be resyncronized with EN-version
\subsection*{mini-FAQ}

% TBT
%\par Q: Is this book simpler/easier than others?
%\par A: No, it is at about the same level as other books of this subject.
% TBT
%\par Q: I'm too frightened to start reading this book, there are more than 1000 pages.
%\par A: All sorts of listings are the bulk of the book.

\par Q: ?`Por qu\'e deber\'ia aprender lenguaje ensamblador hoy en d\'ia?
\par A: A menos que seas un desarrollador de \ac{OS}, probablemente no necesitas programar en ensamblador\textemdash{}los compiladores modernos
son mucho mejores generando optimizaciones que los humanos
\footnote{Un buen texto acerca de este tema: \InSqBrackets{\AgnerFog}}.
Adem\'as, los \ac{CPU}s modernos son dispositivos muy complejos y el conocimiento de ensamblador en realidad no ayuda a comprender su funcionamiento interno.

Una vez dicho eso, hay al menos dos \'areas donde un buen entendimiento de ensamblador puede ser \'util:
Antes que nada, la investigaci\'on de seguridad/malware. Tambi\'en es una buena manera de obtener un mejor entendimiento de tu c\'odigo compilado mientras es depurado.

Por lo tanto, este libro est\'a dirigido a aquellos que desean comprender el lenguaje ensamblador en vez de codificar en \'el,
raz\'on por la cual contiene tantos ejemplos de c\'odigo generado por compilador.

\par Q: Di click en un link dentro del documento PDF, ?`c\'omo regreso?
\par A: En Acrobat Reader, presiona Alt+FlechaIzquierda.

\par Q: ?`Puedo imprimir este libro / usarlo para ense\~nanza?
\par A: !`Por supuesto! Por eso es que el libro est\'a registrado bajo Creative Commons.

\par Q: ?`C\'omo se consigue un trabajo en ingenier\'ia inversa?
\par A: Existen threads de contrataci\'on que aparecen de vez en cuando en reddit, dedicados a reversing\FNURLREDDIT{}.
Intenta buscando ah\'i.

Un thread en ocasiones relacionado con contrataciones puede ser encontrado en el subreddit \q{netsec}.

\par Q: Tengo una pregunta...
\par A: Env\'iamela por email (\EMAILS).



\subsection*{Acerca de la traducci\'on al Coreano}

En enero del 2015, la editorial Acorn (\href{http://www.acornpub.co.kr}{www.acornpub.co.kr}) en Corea del Sur realiz\'o una enorme cantidad de trabajo
traduciendo y publicando mi libro (como era en agosto del 2014) en Coreano.
Ahora se encuentra disponible en
\href{http://www.acornpub.co.kr/book/reversing-for-beginners}{su sitio web}.

\iffalse
\begin{figure}[H]
\centering
\includegraphics[scale=0.3]{acorn_cover.jpg}
\end{figure}
\fi

El traductor es Byungho Min (\href{https://twitter.com/tais9}{twitter/tais9}).
El arte de la portada fue hecho por mi art\'istico amigo, Andy Nechaevsky
\href{https://www.facebook.com/andydinka}{facebook/andydinka}.
Ellos tambi\'en poseen los derechos de autor de la traducci\'on al coreano.
As\'i que, si quieren tener un libro \emph{real} en coreano en su estante
y quieren apoyar mi trabajo, ya se encuentra disponible a la venta.

%\subsection*{About the Persian/Farsi translation}
%TBT

}
\NL{% TODO to be synced with EN version
\section*{Voorwoord}

Er zijn verschillende populaire betekenissen voor de term \q{\gls{reverse engineering}}:
1) Reverse engineeren van software: gecompileerde programma\'s onderzoeken;
2) Scannen van 3D structuren en de onderliggende digitale bewerkingen om deze te kunnen dupliceren;
3) Het nabootsen van een \ac{DBMS} structuur.
Dit boek gaat over de eerste betekenis.

\subsection*{Oefeningen en opdrachten}

\dots 
zijn allen verplaatst naar de website: \url{http://challenges.re}.

\iffalse
\subsection*{Over de auteur}
\begin{tabularx}{\textwidth}{ l X }

\raisebox{-\totalheight}{
\includegraphics[scale=0.60]{Dennis_Yurichev.jpg}
}

&
Dennis Yurichev is een ervaren reverse engineer en programmeur.
Je kan hem contacteren via email: \textbf{\EMAILS{}}.

% FIXME: no link. \tablefootnote doesn't work
\end{tabularx}
\fi

% subsections:
% TBT \input{praise_NL}
% TBT \input{uni_NL}
\ifdefined\RUSSIAN
\newcommand{\PeopleMistakesInaccuraciesRusEng}{Александр Лысенко, Федерико Рамондино, Марк Уилсон, Разихова Мейрамгуль Кайратовна, Анатолий Прокофьев, Костя Бегунец, Валентин ``netch'' Нечаев, Александр Плахов, Артем Метла, Александр Ястребов, Влад Головкин\footnote{goto-vlad@github}, Евгений Прошин, Александр Мясников, Алексей Третьяков, Олег Песков, Павел Шахов}
\else
\newcommand{\PeopleMistakesInaccuraciesRusEng}{Alexander Lysenko, Federico Ramondino, Mark Wilson, Razikhova Meiramgul Kayratovna, Anatoly Prokofiev, Kostya Begunets, Valentin ``netch'' Nechayev, Aleksandr Plakhov, Artem Metla, Alexander Yastrebov, Vlad Golovkin\footnote{goto-vlad@github}, Evgeny Proshin, Alexander Myasnikov, Alexey Tretiakov, Oleg Peskov, Pavel Shakhov}
\fi

\newcommand{\PeopleMistakesInaccuracies}{\PeopleMistakesInaccuraciesRusEng{}, Zhu Ruijin, Changmin Heo, Vitor Vidal, Stijn Crevits, Jean-Gregoire Foulon\footnote{\url{https://github.com/pixjuan}}, Ben L., Etienne Khan, Norbert Szetei\footnote{\url{https://github.com/73696e65}}, Marc Remy, Michael Hansen, Derk Barten, The Renaissance\footnote{\url{https://github.com/TheRenaissance}}, Hugo Chan, Emil Mursalimov, Tanner Hoke, Tan90909090@GitHub, Ole Petter Orhagen, Sourav Punoriyar, Vitor Oliveira, Alexis Ehret, Maxim Shlochiski,
Greg Paton, Pierrick Lebourgeois, Abdullah Alomair.}

\newcommand{\PeopleItalianTranslators}{Federico Ramondino\footnote{\url{https://github.com/pinkrab}},
Paolo Stivanin\footnote{\url{https://github.com/paolostivanin}}, twyK, Fabrizio Bertone, Matteo Sticco, Marco Negro\footnote{\url{https://github.com/Internaut401}}, bluepulsar}

\newcommand{\PeopleFrenchTranslators}{Florent Besnard\footnote{\url{https://github.com/besnardf}}, Marc Remy\footnote{\url{https://github.com/mremy}}, Baudouin Landais, Téo Dacquet\footnote{\url{https://github.com/T30rix}}, BlueSkeye@GitHub\footnote{\url{https://github.com/BlueSkeye}}}

\newcommand{\PeopleGermanTranslators}{Dennis Siekmeier\footnote{\url{https://github.com/DSiekmeier}},
Julius Angres\footnote{\url{https://github.com/JAngres}}, Dirk Loser\footnote{\url{https://github.com/PolymathMonkey}}, Clemens Tamme, Philipp Schweinzer}

\newcommand{\PeopleSpanishTranslators}{Diego Boy, Luis Alberto Espinosa Calvo, Fernando Guida, Diogo Mussi, Patricio Galdames,
Emiliano Estevarena}

\newcommand{\PeoplePTBRTranslators}{Thales Stevan de A. Gois, Diogo Mussi, Luiz Filipe, Primo David Santini}

\newcommand{\PeoplePolishTranslators}{Kateryna Rozanova, Aleksander Mistewicz, Wiktoria Lewicka, Marcin Sokołowski}

\newcommand{\PeopleJapaneseTranslators}{%
shmz@github\footnote{\url{https://github.com/shmz}},%
4ryuJP@github\footnote{\url{https://github.com/4ryuJP}}}

\EN{\input{thanks_EN}}
\ES{\input{thanks_ES}}
\NL{\input{thanks_NL}}
\RU{\input{thanks_RU}}
\IT{\input{thanks_IT}}
\FR{\input{thanks_FR}}
\DE{\input{thanks_DE}}
%\CN{\input{thanks_CN}}
\JA{\input{thanks_JA}}
\PL{\input{thanks_PL}}
\CN{\input{thanks_CN}}


%\input{FAQ_NL} % to be translated

% {\RU{Целевая аудитория}\EN{Target audience}}

\subsection*{Over de Koreaanse vertaling}

In Januari 2015 heeft de Acorn uitgeverij (\href{http://www.acornpub.co.kr}{www.acornpub.co.kr}) in Zuid Korea een enorme hoeveelheid werk verricht in het vertalen en uitgeven
van mijn boek (zoals het was in augustus 2014) in het Koreaans.

Het is nu beschikbaar op
\href{http://www.acornpub.co.kr/book/reversing-for-beginners}{hun website}

\iffalse
\begin{figure}[H]
\centering
\includegraphics[scale=0.3]{acorn_cover.jpg}
\end{figure}
\fi

De vertaler is Byungho Min (\href{https://twitter.com/tais9}{twitter/tais9}).
De cover art is verzorgd door mijn artistieke vriend, Andy Nechaevsky
\href{https://www.facebook.com/andydinka}{facebook/andydinka}.
Zij bezitten ook de auteursrechten voor de Koreaanse vertaling.
Dus, als je een \emph{echt} boek op je kast wil in het Koreaans en je
wil mijn werk steunen, is het nu beschikbaar voor verkoop.

%\subsection*{About the Persian/Farsi translation}
%TBT

}
\IT{\section*{Prefazione}

\subsection*{Da cosa derivano i due titoli?}
\label{TwoTitles}

Il libro era chiamato ``Reverse Engineering for Beginners'' nel periodo 2014-2018, ma ho sempre sospettato che questo restringesse troppo i potenziali lettori.

Nel campo Infosec le persone conoscono il ``reverse engineering'', ma raramente ho sentito la parola ``assembler'' da parte loro.

Similmente, il termine ``reverse engineering'' è in qualche modo criptico per il resto dei programmatori, ma sanno cos'è l'``assembler''.

A luglio 2018, per esperimento, ho cambiato il titolo in ``Assembly Language for Beginners''
e postato il link sul sito Hacker News \footnote{\url{https://news.ycombinator.com/item?id=17549050}}, ed il libro ha avuto un buon successo.

Quindi è così, il libro adesso ha due titoli.

Tuttavia, ho modificato il secondo titolo in ``Understanding Assembly Language'', perchè qualcuno aveva già scritto un libro ``Assembly Language for Beginners''.
Inoltre la gente dice che ``for Beginners'' sembra un po' sarcastico per un libro di \textasciitilde{}1000 pagine.

I due libri sono differenti solo per il titolo, il nome del file (UAL-XX.pdf versus RE4B-XX.pdf), l'URL ed un paio di pagine iniziali.

\subsection*{Sul reverse engineering}

Esistono diversi significati per il termine \q{\gls{reverse engineering}}:

1) Il reverse engineering del software; riguardo la ricerca su programmi compilati

2) La scansione di strutture 3D e la successiva manipolazione digitale necessaria alla loro riproduzione

3) Ricreare strutture in \ac{DBMS}

Questo libro riguarda il primo significato.

\subsection*{Prerequisiti}

Conoscenza di base del C \ac{PL}.
Letture raccomandate: \myref{CCppBooks}.

\subsection*{Esercizi e compiti}

\dots
possono essere trovati su: \url{http://challenges.re}.

\iffalse
\subsection*{L'Autore}
\begin{tabularx}{\textwidth}{ l X }

\raisebox{-\totalheight}{
\includegraphics[scale=0.60]{Dennis_Yurichev.jpg}
}

&
Dennis Yurichev è un reverse engineer e programmatore.
Può essere contattato via mail: \textbf{\EMAILS{}}.

% FIXME: no link. \tablefootnote doesn't work
\end{tabularx}
\fi

% subsections:
\subsection*{Elogi per questo libro}

\url{https://beginners.re/\#praise}.


% TBT \input{uni_IT}
\ifdefined\RUSSIAN
\newcommand{\PeopleMistakesInaccuraciesRusEng}{Александр Лысенко, Федерико Рамондино, Марк Уилсон, Разихова Мейрамгуль Кайратовна, Анатолий Прокофьев, Костя Бегунец, Валентин ``netch'' Нечаев, Александр Плахов, Артем Метла, Александр Ястребов, Влад Головкин\footnote{goto-vlad@github}, Евгений Прошин, Александр Мясников, Алексей Третьяков, Олег Песков, Павел Шахов}
\else
\newcommand{\PeopleMistakesInaccuraciesRusEng}{Alexander Lysenko, Federico Ramondino, Mark Wilson, Razikhova Meiramgul Kayratovna, Anatoly Prokofiev, Kostya Begunets, Valentin ``netch'' Nechayev, Aleksandr Plakhov, Artem Metla, Alexander Yastrebov, Vlad Golovkin\footnote{goto-vlad@github}, Evgeny Proshin, Alexander Myasnikov, Alexey Tretiakov, Oleg Peskov, Pavel Shakhov}
\fi

\newcommand{\PeopleMistakesInaccuracies}{\PeopleMistakesInaccuraciesRusEng{}, Zhu Ruijin, Changmin Heo, Vitor Vidal, Stijn Crevits, Jean-Gregoire Foulon\footnote{\url{https://github.com/pixjuan}}, Ben L., Etienne Khan, Norbert Szetei\footnote{\url{https://github.com/73696e65}}, Marc Remy, Michael Hansen, Derk Barten, The Renaissance\footnote{\url{https://github.com/TheRenaissance}}, Hugo Chan, Emil Mursalimov, Tanner Hoke, Tan90909090@GitHub, Ole Petter Orhagen, Sourav Punoriyar, Vitor Oliveira, Alexis Ehret, Maxim Shlochiski,
Greg Paton, Pierrick Lebourgeois, Abdullah Alomair.}

\newcommand{\PeopleItalianTranslators}{Federico Ramondino\footnote{\url{https://github.com/pinkrab}},
Paolo Stivanin\footnote{\url{https://github.com/paolostivanin}}, twyK, Fabrizio Bertone, Matteo Sticco, Marco Negro\footnote{\url{https://github.com/Internaut401}}, bluepulsar}

\newcommand{\PeopleFrenchTranslators}{Florent Besnard\footnote{\url{https://github.com/besnardf}}, Marc Remy\footnote{\url{https://github.com/mremy}}, Baudouin Landais, Téo Dacquet\footnote{\url{https://github.com/T30rix}}, BlueSkeye@GitHub\footnote{\url{https://github.com/BlueSkeye}}}

\newcommand{\PeopleGermanTranslators}{Dennis Siekmeier\footnote{\url{https://github.com/DSiekmeier}},
Julius Angres\footnote{\url{https://github.com/JAngres}}, Dirk Loser\footnote{\url{https://github.com/PolymathMonkey}}, Clemens Tamme, Philipp Schweinzer}

\newcommand{\PeopleSpanishTranslators}{Diego Boy, Luis Alberto Espinosa Calvo, Fernando Guida, Diogo Mussi, Patricio Galdames,
Emiliano Estevarena}

\newcommand{\PeoplePTBRTranslators}{Thales Stevan de A. Gois, Diogo Mussi, Luiz Filipe, Primo David Santini}

\newcommand{\PeoplePolishTranslators}{Kateryna Rozanova, Aleksander Mistewicz, Wiktoria Lewicka, Marcin Sokołowski}

\newcommand{\PeopleJapaneseTranslators}{%
shmz@github\footnote{\url{https://github.com/shmz}},%
4ryuJP@github\footnote{\url{https://github.com/4ryuJP}}}

\EN{\input{thanks_EN}}
\ES{\input{thanks_ES}}
\NL{\input{thanks_NL}}
\RU{\input{thanks_RU}}
\IT{\input{thanks_IT}}
\FR{\input{thanks_FR}}
\DE{\input{thanks_DE}}
%\CN{\input{thanks_CN}}
\JA{\input{thanks_JA}}
\PL{\input{thanks_PL}}
\CN{\input{thanks_CN}}


\subsection*{mini-FAQ}

% TBT
%\par Q: Is this book simpler/easier than others?
%\par A: No, it is at about the same level as other books of this subject.
% TBT
%\par Q: I'm too frightened to start reading this book, there are more than 1000 pages.
%\par A: All sorts of listings are the bulk of the book.

\par Q: Quali sono i prerequisiti per leggere questo libro?
\par A: È consigliato avere almeno una conoscenza base di C/C++.

\par Q: Dovrei veramente imparare x86/x64/ARM e MIPS allo stesso tempo? Non è troppo?
\par A: Chi inizia può leggere semplicemente su x86/x64, e saltare o sfogliare velocemente le parti su ARM e MIPS.

\par Q: Posso acquistare la versione in Russo/Inglese del libro?
\par A: Sfortunatamente no, nessun editore (al momento) è interessato nel pubblicare questo libro.
Nel frattempo puoi chiedere alla tua copisteria di fiducia di stamparlo.
\url{https://yurichev.com/news/20200222_printed_RE4B/}.

\par Q: C'è una versione EPUB/MOBI?
\par A: Il libro dipende fortemente da alcuni hacks in TeX/LaTeX, quindi convertire il tutto in HTML (EPUB/MOBI è un set di HTMLs)
non sarebbe facile.

\par Q: Perchè qualcuno dovrebbe studiare assembly al giorno d'oggi?
\par A: A meno che tu non sia uno sviluppatore di \ac{sistemi operativi}, probabilmente non avrai mai bisogno di scrivere codice assembly \textemdash{}i compilatori moderni sono migliori dell'uomo nell'effettuare ottimizzazioni \footnote{Un testo consigliato relativamente a questo argomento: \InSqBrackets{\AgnerFog}}.

Inoltre, le \ac{CPU} moderne sono dispositivi molto complessi e la semplice conoscenza di assembly non basta per capire il loro funzionamento interno.

Ci sono però almeno due aree in cui una buona conoscenza di assembly può tornare utile: analisi malware/ricercatore in ambito sicurezza e per avere una miglior comprensione del codice compilato durante il debugging di un programma.
Questo libro è perciò pensato per quelle persone che vogliono capire il linguaggio assembly piuttosto che imparare a programmare con esso.

\par Q: Ho cliccato su un link all'interno del PDF, come torno indietro?
\par A: In Adobe Acrobat Reader clicca Alt+FrecciaSinistra. In Evince clicca il pulsante ``<''.

\par Q: Posso stampare questo libro / usarlo per insegnare?
\par A: Certamente! Il libro è rilasciato sotto licenza Creative Commons (CC BY-SA 4.0).

\par Q: Perchè questo libro è gratis? Hai svolto un ottimo lavoro. È sospetto come molte altre cose gratis.
\par A: Per mia esperienza, gli autori di libri tecnici fanno queste cose per auto-pubblicizzarsi. Non è possibile ottenere un buon ricavato da un lavoro così oneroso.

\par Q: Come si fa ad ottenere un lavoro nel campo del reverse engineering?
\par A: Ci sono threads di assunzione che appaiono di tanto in tanto su reddit RE\FNURLREDDIT{}.
Prova a guardare lì.

Qualcosa di simile si può anche trovare nel subreddit \q{netsec}.

\par Q: Avrei una domanda...
\par A: Inviamela tramite e-mail (\EMAILS).


\subsection*{La traduzione in Coreano}

A gennaio 2015, la Acorn publishing company (\href{http://www.acornpub.co.kr}{www.acornpub.co.kr}) in Corea del Sud ha compiuto un enorme lavoro traducendo e pubblicando
questo libro (aggiornato ad agosto 2014) in Coreano.

Adesso è disponibile sul \href{http://www.acornpub.co.kr/book/reversing-for-beginners}{loro sito}.

\iffalse
\begin{figure}[H]
\centering
\includegraphics[scale=0.3]{acorn_cover.jpg}
\end{figure}
\fi

Il traduttore è Byungho Min (\href{https://twitter.com/tais9}{twitter/tais9}).
La copertina è stata creata dall'artistico Andy Nechaevsky, un amico dell'autore:
\href{https://www.facebook.com/andydinka}{facebook/andydinka}.
Acorn detiene inoltre i diritti della traduzione in Coreano.

Quindi se vuoi un \emph{vero} libro in Coreano nella tua libreria e
vuoi supportare questo lavoro, è disponibile per l'acquisto.

\subsection*{La traduzione in Persiano/Farsi}

Nel 2016 il libro è stato tradotto da Mohsen Mostafa Jokar (che è anche conosciuto nell comunità iraniana per la traduzione del manuale di Radare \footnote{\url{http://rada.re/get/radare2book-persian.pdf}}).
Potete trovarlo sul sito dell'editore\footnote{\url{http://goo.gl/2Tzx0H}} (Pendare Pars).

Qua c'è il link ad un estratto di 40 pagine: \url{https://beginners.re/farsi.pdf}.

Informazioni nella Libreria Nazionale dell'Iran: \url{http://opac.nlai.ir/opac-prod/bibliographic/4473995}.

\subsection*{La traduzione Cinese}

Ad aprile 2017, la traduzione in Cinese è stata completata da Chinese PTPress. Possiedono inoltre i diritti della traduzione in Cinese.

La versione cinese può essere ordinata a questo indirizzo: \url{http://www.epubit.com.cn/book/details/4174}. Una recensione parziale, con informazioni sulla traduzione è disponibile qua: \url{http://www.cptoday.cn/news/detail/3155}.

Il traduttore principale è Archer, al quale l'autore deve molto. E' stato estremamente meticoloso (in senso buono) ed ha segnalato buona parte degli errori e bug, il che è molto importante in un libro come questo.
L'autore raccomanderebbe i suoi servizi a qualsiasi altro autore!

I ragazzi di \href{http://www.antiy.net/}{Antiy Labs} hanno inoltre aiutato nella traduzione. \href{http://www.epubit.com.cn/book/onlinechapter/51413}{Qua c'è la prefazione} scritta da loro.
}
\DE{\section*{Vorwort}

\subsection*{Warum zwei Titel?}
\label{TwoTitles}

Dieses Buch hieß von 2014-2018 ``Reverse Engineering for Beginners'', jedoch hatte ich immer die Befürchtung, dass es den Leserkreis zu sehr einengen würde.

Infosec Leute kennen sich mit ``Reverse Engineering'' aus, jedoch hörte ich selten das Wort ``Assembler'' von Ihnen.

Desweiteren ist der Begriff ``Reverse Engineering'' etwas zu kryptisch für den Großteil von Programmierern, diesen ist jedoch ``Assembler'' geläufig.

Im Juli 2018 änderte ich als Experiment den Titel zu ``Assembly Language for Beginners'' und veröffentlichte den Link auf der Hacker News-Website\footnote{\url{https://news.ycombinator.com/item?id=17549050}}. Das Buch kam allgemein gut an.

Aus diesem Grund hat das Buch nun zwei Titel.

Ich habe den zweiten Titel zu ``Understanding Assembly Language'' geändert, da es bereits eine Erscheinung mit dem Titel ``Assembly Language for Beginners'' gab. Einige Leute sind der Meinung, dass ``for Beginners'' etwas sarkastisch klingt, für ein Buch mit \textasciitilde{}1000 Seiten.

Die beiden Bücher unterscheiden sich lediglich im Titel, dem Dateinamen (UAL-XX.pdf beziehungsweise RE4B-XX.pdf), URL und ein paar der einleitenden Seiten.

\subsection*{Über Reverse Engineering}

Es gibt verschiedene verbreitete Interpretationen des Begriffs Reverse Engineering:\\
1) Reverse Engineering von Software: Rückgewinnung des Quellcodes bereits kompilierter Programme;\\
2) Das Erfassen von 3D Strukturen und die digitalen Manipulationen die zur Duplizierung notwendig sind;\\
3) Nachbilden von \ac{DBMS}-Strukturen.\\
Dieses Buch behandelt die erste Interpretation.

\subsection*{Voraussetzungen}

Grundlegende Kenntnisse der Programmiersprache C.
Empfohlene Literatur: \myref{CCppBooks}.

\subsection*{Übungen und Aufgaben}
\dots 
befinden sich nun alle auf der Website: \url{http://challenges.re}.

\iffalse
\subsection*{Über den Autor}
\begin{tabularx}{\textwidth}{ l X }

\raisebox{-\totalheight}{
\includegraphics[scale=0.60]{Dennis_Yurichev.jpg}
}

&
Dennis Yurichev ist ein erfahrener Reverse Engineer und Programmierer.
Er kann per E-Mail kontaktiert werden: \textbf{\EMAILS{}}.

% FIXME: no link. \tablefootnote doesn't work
\end{tabularx}
\fi

% subsections:
\subsection*{Lob für}

\url{https://beginners.re/\#praise}.


% TBT \input{uni_DE}
\ifdefined\RUSSIAN
\newcommand{\PeopleMistakesInaccuraciesRusEng}{Александр Лысенко, Федерико Рамондино, Марк Уилсон, Разихова Мейрамгуль Кайратовна, Анатолий Прокофьев, Костя Бегунец, Валентин ``netch'' Нечаев, Александр Плахов, Артем Метла, Александр Ястребов, Влад Головкин\footnote{goto-vlad@github}, Евгений Прошин, Александр Мясников, Алексей Третьяков, Олег Песков, Павел Шахов}
\else
\newcommand{\PeopleMistakesInaccuraciesRusEng}{Alexander Lysenko, Federico Ramondino, Mark Wilson, Razikhova Meiramgul Kayratovna, Anatoly Prokofiev, Kostya Begunets, Valentin ``netch'' Nechayev, Aleksandr Plakhov, Artem Metla, Alexander Yastrebov, Vlad Golovkin\footnote{goto-vlad@github}, Evgeny Proshin, Alexander Myasnikov, Alexey Tretiakov, Oleg Peskov, Pavel Shakhov}
\fi

\newcommand{\PeopleMistakesInaccuracies}{\PeopleMistakesInaccuraciesRusEng{}, Zhu Ruijin, Changmin Heo, Vitor Vidal, Stijn Crevits, Jean-Gregoire Foulon\footnote{\url{https://github.com/pixjuan}}, Ben L., Etienne Khan, Norbert Szetei\footnote{\url{https://github.com/73696e65}}, Marc Remy, Michael Hansen, Derk Barten, The Renaissance\footnote{\url{https://github.com/TheRenaissance}}, Hugo Chan, Emil Mursalimov, Tanner Hoke, Tan90909090@GitHub, Ole Petter Orhagen, Sourav Punoriyar, Vitor Oliveira, Alexis Ehret, Maxim Shlochiski,
Greg Paton, Pierrick Lebourgeois, Abdullah Alomair.}

\newcommand{\PeopleItalianTranslators}{Federico Ramondino\footnote{\url{https://github.com/pinkrab}},
Paolo Stivanin\footnote{\url{https://github.com/paolostivanin}}, twyK, Fabrizio Bertone, Matteo Sticco, Marco Negro\footnote{\url{https://github.com/Internaut401}}, bluepulsar}

\newcommand{\PeopleFrenchTranslators}{Florent Besnard\footnote{\url{https://github.com/besnardf}}, Marc Remy\footnote{\url{https://github.com/mremy}}, Baudouin Landais, Téo Dacquet\footnote{\url{https://github.com/T30rix}}, BlueSkeye@GitHub\footnote{\url{https://github.com/BlueSkeye}}}

\newcommand{\PeopleGermanTranslators}{Dennis Siekmeier\footnote{\url{https://github.com/DSiekmeier}},
Julius Angres\footnote{\url{https://github.com/JAngres}}, Dirk Loser\footnote{\url{https://github.com/PolymathMonkey}}, Clemens Tamme, Philipp Schweinzer}

\newcommand{\PeopleSpanishTranslators}{Diego Boy, Luis Alberto Espinosa Calvo, Fernando Guida, Diogo Mussi, Patricio Galdames,
Emiliano Estevarena}

\newcommand{\PeoplePTBRTranslators}{Thales Stevan de A. Gois, Diogo Mussi, Luiz Filipe, Primo David Santini}

\newcommand{\PeoplePolishTranslators}{Kateryna Rozanova, Aleksander Mistewicz, Wiktoria Lewicka, Marcin Sokołowski}

\newcommand{\PeopleJapaneseTranslators}{%
shmz@github\footnote{\url{https://github.com/shmz}},%
4ryuJP@github\footnote{\url{https://github.com/4ryuJP}}}

\EN{\input{thanks_EN}}
\ES{\input{thanks_ES}}
\NL{\input{thanks_NL}}
\RU{\input{thanks_RU}}
\IT{\input{thanks_IT}}
\FR{\input{thanks_FR}}
\DE{\input{thanks_DE}}
%\CN{\input{thanks_CN}}
\JA{\input{thanks_JA}}
\PL{\input{thanks_PL}}
\CN{\input{thanks_CN}}


\subsection*{Mini-FAQ}

% TBT
%\par Q: Is this book simpler/easier than others?
%\par A: No, it is at about the same level as other books of this subject.
% TBT
%\par Q: I'm too frightened to start reading this book, there are more than 1000 pages.
%\par A: All sorts of listings are the bulk of the book.

\par F: Was sind die Voraussetzungen die der Leser dieses Buchs erfüllen sollte?
\par A: Grundlagenwissen der Programmiersprachen C und / oder C++ sind wünschenswert.

\par F: Sollte ich wirklich x86/x64/ARM und MIPS auf einmal lernen? Ist das nicht zuviel?
\par A: Anfänger können erstmal nur über x86/x64 lesen und den ARM- und MIPS-Teil überspringen oder überfliegen.

\par F: Kann ich eine russische oder englische Version als Druckausgabe kaufen?
\par A: Leider nicht, bisher ist kein Verleger an einer russischen oder englischen Version interessiert.
Bis es soweit ist, können Sie Ihren Lieblings-Copy-Shop bitten es zu drucken und zu binden.
\url{https://yurichev.com/news/20200222_printed_RE4B/}.

\par F: Gibt es eine EPUB- oder MOBI-Version?
\par A: Dieses Buch ist in hohem Maße abhängig von \TeX{}- / \LaTeX{}-spezifischen Techniken,
was das Konvertieren zu HTML schwierig macht (EPUB und MOBI basieren auf HTML).

\par F: Warum sollte ich heutzutage noch Assembler lernen?
\par A: Falls Sie kein \ac{OS}-Entwickler sind, werden Sie vermutlich nie in Assembler programmieren müssen \textemdash{}
aktuelle Compiler (2010 und später) können sehr viel besser optimieren als Menschen\footnote{Ein lesenswerter Artikel zu diesem Thema: \InSqBrackets{\AgnerFog}}.

Auch sind aktuelle \ac{CPU}s sehr komplexe Komponenten und Wissen über Assembler hilft nicht wirklich
um die Interna zu verstehen.

Davon abgesehen, gibt es mindestens zwei Bereiche in denen ein gutes Verständnis von Assembler hilfreich
sein kann: Zuallererst, bei der Security- und Malware-Forschung, aber auch um ein besseres Verständnis des kompilierten
Codes zu bekommen.
Dieses Buch ist somit für diejenigen geschrieben, die Assembler eher verstehen als darin programmieren wollen.
Das ist der Grund, warum viele Ausgabe-Beispiele des Compilers in diesem Buch enthalten sind.

\par F: Ich habe in der PDF-Datei auf einen Link geklickt. Wie komme ich zurück?
\par A: Im Adobe Acrobat Reader geht dies durch betätigen von Alt+CursorLinks.
In Evince durch die ``<''-Taste.

\par F: Darf ich dieses Buch drucken / für Lehrzwecke benutzen?
\par A: Selbstverständlich! Das ist der Grund warum es unter der Creative Commons Lizenz (CC BY-SA 4.0) veröffentlicht wird.

\par F: Warum ist dieses Buch kostenlos? Du hast gute Arbeit geleistet. Das ist verdächtig, wie bei vielen anderen kostenlosen Dingen.
\par A: Meiner Erfahrung nach schreiben Autoren von technischer Literatur diese des Lernens willen. Es ist nicht möglich angemessen
viel Geld hierfür zu bekommen.

\par F: Wie kann man einen Job im Bereich des Reverse Engineering bekommen?
\par A: Von Zeit zu Zeit gibt es Threads zu Jobangeboten auf \FNURLREDDIT{}.
Versuchen Sie es dort einmal.

Ein ähnlicher Job-Thread ist unter\q{netsec} subreddit zu finden.

\par F: Ich habe eine Frage...
\par A: Senden Sie sie mir per E-Mail (\EMAILS).


\subsection*{Über die koreanische Übersetzung}

Im Januar 2015 hat die Acorn Publishing Company (\href{http://www.acornpub.co.kr}{www.acornpub.co.kr}) in Süd-Korea
viel Aufwand in die Übersetzung und Veröffentlichung meines Buchs ins Koreanische (mit Stand August 2014) investiert.

Es ist jetzt unter dieser \href{http://www.acornpub.co.kr/book/reversing-for-beginners}{Webseite} verfügbar.

\iffalse
\begin{figure}[H]
\centering
\includegraphics[scale=0.3]{acorn_cover.jpg}
\end{figure}
\fi

Der Übersetzer ist Byungho Min (\href{https://twitter.com/tais9}{twitter/tais9}).
Die Cover-Gestaltung wurde von meinem künstlerisch begabten Freund Andy Nechaevsky erstellt:
\href{https://www.facebook.com/andydinka}{facebook/andydinka}.
Die Acorn Publishing Company besetzt die Urheberrechte an der koreanischen Übersetzung.

Wenn Sie also ein \emph{echtes} Buch in Ihrem Buchregal auf koreanisch haben und 
mich bei meiner Arbeit unterstützen wollen, können Sie das Buch nun kaufen.

\subsection*{Über die persische Übersetzung (Farsi)}

In 2016 wurde das Buch von Mohsen Mostafa Jokar übersetzt, der in der iranischen Community
auch für die Übersetzung des Radare
Handbuchs\footnote{\url{http://rada.re/get/radare2book-persian.pdf}} bekannt ist).
Es ist auf der Homepage des Verlegers\footnote{\url{http://goo.gl/2Tzx0H}} (Pendare Pars)
verfügbar.

Hier ist ein Link zu einem 40-seitigen Auszug: \url{https://beginners.re/farsi.pdf}.

National Library of Iran registration information: \url{http://opac.nlai.ir/opac-prod/bibliographic/4473995}.

\subsection*{Über die chinesische Übersetzung}

Im April 2017 wurde die chinesische Übersetzung von Chinese PTPress fertiggestellt, bei denen
auch das Copyright liegt.

Die chinesische Version kann hier bestellt werden: \url{http://www.epubit.com.cn/book/details/4174}.
Ein Auszug und die Geschichte der Übersetzung kann hier gefunden werden: \url{http://www.cptoday.cn/news/detail/3155}.

Der Hauptübersetzer ist Archer, dem der Autor sehr iel verdankt.
Archer war extrem akribisch (im positiven Sinne) und meldete
den Großteil der bekannten Fehler, was für Literatur wie dieses Buch extrem wichtig ist.
Der Autor empfiehlt diesen Service jedem anderen Autor!

Die Mitarbeiter von \href{http://www.antiy.net/}{Antiy Labs} halfen ebenfalls bei der Übersetzung.
\href{http://www.epubit.com.cn/book/onlinechapter/51413}{Hier ist das Vorwort} von ihnen.
}
\FR{\section*{Préface}

\subsection*{C'est quoi ces deux titres?}
\label{TwoTitles}

Le livre a été appelé ``Reverse Engineering for Beginners'' en 2014-2018, mais j'ai
toujours suspecté que ça rendait son audience trop réduite.

Les gens de l'infosec connaissent le ``reverse engineering'', mais j'ai rarement
entendu le mot ``assembleur'' de leur part.

De même, le terme ``reverse engineering'' est quelque peu cryptique pour une audience
générale de programmeurs, mais qui ont des connaissances à propos de l'``assembleur''.

En juillet 2018, à titre d'expérience, j'ai changé le titre en ``Assembly Language for Beginners''
et publié le lien sur le site Hacker News\footnote{\url{https://news.ycombinator.com/item?id=17549050}},
et le livre a été plutôt bien accueilli.

Donc, c'est ainsi que le livre a maintenant deux titres.

Toutefois, j'ai changé le second titre à ``Understanding Assembly Language'', car
quelqu'un a déjà écrit le livre ``Assembly Language for Beginners''.
De même, des gens disent que ``for Beginners'' sonne sarcastique pour un livre de
\textasciitilde{}1000 pages.

Les deux livres diffèrent seulement par le titre, le nom du fichier (UAL-XX.pdf versus
RE4B-XX.pdf), l'URL et quelques-une des première pages.

\subsection*{À propos de la rétro-ingénierie}

Il existe plusieurs définitions pour l'expression \q{ingénierie inverse ou rétro-ingénierie \gls{reverse engineering}} :

1) L'ingénierie inverse de logiciels : examiner des programmes compilés;

2) Le balayage des structures en 3D et la manipulation numérique nécessaire afin de les reproduire;

3) Recréer une structure de base de données.

Ce livre concerne la première définition.

\subsection*{Prérequis}

Connaissance basique du C \ac{PL}.
Il est recommandé de lire: \myref{CCppBooks}.

\subsection*{Exercices et tâches}

\dots
ont été déplacés sur un site différent : \url{http://challenges.re}.

\iffalse
\subsection*{A propos de l'auteur}
\begin{tabularx}{\textwidth}{ l X }

\raisebox{-\totalheight}{
\includegraphics[scale=0.60]{Dennis_Yurichev.jpg}
}

&
Dennis Yurichev est un ingénieur expérimenté en rétro-ingénierie et un programmeur.
Il peut être contacté par email : \textbf{\EMAILS{}}.

% FIXME: no link. \tablefootnote doesn't work
\end{tabularx}
\fi

% subsections:
\subsection*{Éloges de ce livre}

\url{https://beginners.re/\#praise}.


\subsection*{Universités}

Ce livre est recommandé par au moins ces universités:
\url{https://beginners.re/\#uni}.


\ifdefined\RUSSIAN
\newcommand{\PeopleMistakesInaccuraciesRusEng}{Александр Лысенко, Федерико Рамондино, Марк Уилсон, Разихова Мейрамгуль Кайратовна, Анатолий Прокофьев, Костя Бегунец, Валентин ``netch'' Нечаев, Александр Плахов, Артем Метла, Александр Ястребов, Влад Головкин\footnote{goto-vlad@github}, Евгений Прошин, Александр Мясников, Алексей Третьяков, Олег Песков, Павел Шахов}
\else
\newcommand{\PeopleMistakesInaccuraciesRusEng}{Alexander Lysenko, Federico Ramondino, Mark Wilson, Razikhova Meiramgul Kayratovna, Anatoly Prokofiev, Kostya Begunets, Valentin ``netch'' Nechayev, Aleksandr Plakhov, Artem Metla, Alexander Yastrebov, Vlad Golovkin\footnote{goto-vlad@github}, Evgeny Proshin, Alexander Myasnikov, Alexey Tretiakov, Oleg Peskov, Pavel Shakhov}
\fi

\newcommand{\PeopleMistakesInaccuracies}{\PeopleMistakesInaccuraciesRusEng{}, Zhu Ruijin, Changmin Heo, Vitor Vidal, Stijn Crevits, Jean-Gregoire Foulon\footnote{\url{https://github.com/pixjuan}}, Ben L., Etienne Khan, Norbert Szetei\footnote{\url{https://github.com/73696e65}}, Marc Remy, Michael Hansen, Derk Barten, The Renaissance\footnote{\url{https://github.com/TheRenaissance}}, Hugo Chan, Emil Mursalimov, Tanner Hoke, Tan90909090@GitHub, Ole Petter Orhagen, Sourav Punoriyar, Vitor Oliveira, Alexis Ehret, Maxim Shlochiski,
Greg Paton, Pierrick Lebourgeois, Abdullah Alomair.}

\newcommand{\PeopleItalianTranslators}{Federico Ramondino\footnote{\url{https://github.com/pinkrab}},
Paolo Stivanin\footnote{\url{https://github.com/paolostivanin}}, twyK, Fabrizio Bertone, Matteo Sticco, Marco Negro\footnote{\url{https://github.com/Internaut401}}, bluepulsar}

\newcommand{\PeopleFrenchTranslators}{Florent Besnard\footnote{\url{https://github.com/besnardf}}, Marc Remy\footnote{\url{https://github.com/mremy}}, Baudouin Landais, Téo Dacquet\footnote{\url{https://github.com/T30rix}}, BlueSkeye@GitHub\footnote{\url{https://github.com/BlueSkeye}}}

\newcommand{\PeopleGermanTranslators}{Dennis Siekmeier\footnote{\url{https://github.com/DSiekmeier}},
Julius Angres\footnote{\url{https://github.com/JAngres}}, Dirk Loser\footnote{\url{https://github.com/PolymathMonkey}}, Clemens Tamme, Philipp Schweinzer}

\newcommand{\PeopleSpanishTranslators}{Diego Boy, Luis Alberto Espinosa Calvo, Fernando Guida, Diogo Mussi, Patricio Galdames,
Emiliano Estevarena}

\newcommand{\PeoplePTBRTranslators}{Thales Stevan de A. Gois, Diogo Mussi, Luiz Filipe, Primo David Santini}

\newcommand{\PeoplePolishTranslators}{Kateryna Rozanova, Aleksander Mistewicz, Wiktoria Lewicka, Marcin Sokołowski}

\newcommand{\PeopleJapaneseTranslators}{%
shmz@github\footnote{\url{https://github.com/shmz}},%
4ryuJP@github\footnote{\url{https://github.com/4ryuJP}}}

\EN{\input{thanks_EN}}
\ES{\input{thanks_ES}}
\NL{\input{thanks_NL}}
\RU{\input{thanks_RU}}
\IT{\input{thanks_IT}}
\FR{\input{thanks_FR}}
\DE{\input{thanks_DE}}
%\CN{\input{thanks_CN}}
\JA{\input{thanks_JA}}
\PL{\input{thanks_PL}}
\CN{\input{thanks_CN}}


\subsection*{mini-FAQ}

\par Q: Est-ce que ce livre est plus simple/facile que les autres?
\par R: Non, c'est à peu près le même niveau que les autres livres sur ce sujet.

\par Q: J'ai trop peur de commencer à lire ce livre, il fait plus de 1000 pages.
"...for Beginners" dans le nom sonne un peu sarcastique.
\par R: Toutes sortes de listings constituent le gros de ce livre.
Le livre est en effet pour les débutants, il manque (encore) beaucoup de choses.

\par Q: Quels sont les pré-requis nécessaires avant de lire ce livre ?
\par R: Une compréhension de base du C/C++ serait l'idéal.

\par Q: Dois-je apprendre x86/x64/ARM et MIPS en même temps ? N'est-ce pas un peu trop ?
\par R: Je pense que les débutants peuvent seulement lire les parties x86/x64, tout en passant/feuilletant celles ARM/MIPS.

\par Q: Puis-je acheter une version papier du livre en russe / anglais ?
\par R: Malheureusement non, aucune maison d'édition n'a été intéressée pour publier une version en russe ou en anglais du livre jusqu'à présent.
Cependant, vous pouvez demander à votre imprimerie préférée de l'imprimer et de le relier.
\url{https://yurichev.com/news/20200222_printed_RE4B/}.

\par Q: Y a-il une version ePub/Mobi ?
\par R: Le livre dépend majoritairement de TeX/LaTeX, il n'est donc pas évident de le convertir en version ePub/Mobi.

\par Q: Pourquoi devrait-on apprendre l'assembleur de nos jours ?
\par R: A moins d'être un développeur d'\ac{OS}, vous n'aurez probablement pas besoin d'écrire en assembleur\textemdash{}les derniers compilateurs (ceux de notre décennie) sont meilleurs que les êtres humains en terme d'optimisation. \footnote{Un très bon article à ce sujet : \InSqBrackets{\AgnerFog}}.

De plus, les derniers \ac{CPU}s sont des appareils complexes et la connaissance de l'assembleur n'aide pas vraiment à comprendre leurs mécanismes internes.

Cela dit, il existe au moins deux domaines dans lesquels une bonne connaissance de l'assembleur peut être utile : 
Tout d'abord, pour de la recherche en sécurité ou sur des malwares. C'est également un bon moyen de comprendre un code compilé lorsqu'on le debug.
Ce livre est donc destiné à ceux qui veulent comprendre l'assembleur plutôt que d'écrire en assembleur, ce qui explique pourquoi il y a de nombreux exemples de résultats issus de compilateurs dans ce livre. 

\par Q: J'ai cliqué sur un lien dans le document PDF, comment puis-je retourner en arrière ?
\par R: Dans Adobe Acrobat Reader, appuyez sur Alt + Flèche gauche. Dans Evince, appuyez sur le bouton ``<''.

\par Q: Puis-je imprimer ce livre / l'utiliser pour de l'enseignement ?
\par R: Bien sûr ! C'est la raison pour laquelle le livre est sous licence Creative Commons (CC BY-SA 4.0).

\par Q: Pourquoi ce livre est-il gratuit ? Vous avez fait du bon boulot. C'est suspect, comme nombre de choses gratuites.
\par R: D'après ma propre expérience, les auteurs d'ouvrages techniques font cela pour l'auto-publicité. Il n'est pas possible de se faire beaucoup d'argent d'une telle manière.

\par Q: Comment trouver du travail dans le domaine de la rétro-ingénierie ?
\par R: Il existe des sujets d'embauche qui apparaissent de temps en temps sur Reddit, dédiés à la rétro-ingénierie (cf. reverse engineering ou RE)\FNURLREDDIT{}.
Jetez un \oe{}il ici.

Un sujet d'embauche quelque peu lié peut être trouvé dans le subreddit \q{netsec}.

\par Q: J'ai une question...
\par R: Envoyez-la moi par email (\EMAILS).



\subsection*{À propos de la traduction en Coréen}

En Janvier 2015, la maison d'édition Acorn (\href{http://www.acornpub.co.kr}{www.acornpub.co.kr}) en Corée du Sud a réalisé un énorme travail en traduisant et en publiant mon livre (dans son état en Août 2014) en Coréen.

Il est désormais disponible sur \href{http://www.acornpub.co.kr/book/reversing-for-beginners}{leur site web}.

\iffalse
\begin{figure}[H]
\centering
\includegraphics[scale=0.3]{acorn_cover.jpg}
\end{figure}
\fi

Le traducteur est Byungho Min (\href{https://twitter.com/tais9}{twitter/tais9}).
L'illustration de couverture a été réalisée l'artiste, Andy Nechaevsky, un ami de l'auteur:
\href{https://www.facebook.com/andydinka}{facebook/andydinka}.
Ils détiennent également les droits d'auteurs sur la traduction coréenne.

Donc si vous souhaitez avoir un livre \emph{réel} en coréen sur votre étagère et que vous souhaitez soutenir ce travail, il est désormais disponible à l'achat.

\subsection*{Á propos de la traduction en Farsi/Perse}

En 2016, ce livre a été traduit par Mohsen Mostafa Jokar (qui est aussi connu dans
la communauté iranienne pour sa traduction du manuel de Radare\footnote{\url{http://rada.re/get/radare2book-persian.pdf}}).
Il est disponible sur le site web de l'éditeur\footnote{\url{http://goo.gl/2Tzx0H}}
(Pendare Pars).

Extrait de 40 pages: \url{https://beginners.re/farsi.pdf}.

Enregistrement du livre à la Bibliothèque Nationale d'Iran: \url{http://opac.nlai.ir/opac-prod/bibliographic/4473995}.

\subsection*{Á propos de la traduction en Chinois}

En avril 2017, la traduction en Chinois a été terminée par Chinese PTPress. Ils sont
également les détenteurs des droits de la traduction en Chinois.

La version chinoise est disponible à l'achat ici: \url{http://www.epubit.com.cn/book/details/4174}.
Une revue partielle et l'historique de la traduction peut être trouvé ici: \url{http://www.cptoday.cn/news/detail/3155}.


Le traducteur principal est Archer, à qui je dois beaucoup. Il a été très méticuleux
(dans le bon sens du terme) et a signalé la plupart des erreurs et bugs connus, ce
qui est très important dans le genre de littérature de ce livre.
Je recommanderais ses services à tout autre auteur!

Les gens de \href{http://www.antiy.net/}{Antiy Labs} ont aussi aidé pour la traduction.
\href{http://www.epubit.com.cn/book/onlinechapter/51413}{Voici la préface} écrite par eux.

}
\PL{\section*{Przedmowa}

\subsection*{Skąd dwa tytuły?}
\label{TwoTitles}

W latach 2014-2018 książka nosiła tytuł "Inżynieria wsteczna dla początkujących" ale zawsze podejrzewałem, że zawęża to grono czytelników.

Ludzie zajmujący się bezpieczeństwem informacji (infosec) wiedzą o inżynierii wstecznej, jednak rzadko kiedy słyszałem, by używali słowa asembler.

Podobnie, termin ``inżynieria wsteczna'' jest nieco tajemniczy dla ogólnego grona programistów, jednak wiedzą oni o istnieniu asemblera.

W lipcu 2018 roku, w ramach eksperymentu, zmieniłem tytuł na ``Język maszynowy dla początkujących''
i umieściłem link na portalu Hacker News\footnote{\url{https://news.ycombinator.com/item?id=17549050}}, i książka została ogólnie dobrze przyjęta.

Niech więc tak będzie, książka ma teraz dwa tytuły.

Zmieniłem jednak drugi tytuł na "Rozumienie kodu maszynowego",
ponieważ ktoś już napisał książkę o tytule "Język maszynowy dla początkujących".
Ludzie twierdzą, że "dla początkujących" brzmi odrobinę ironicznie jak na \textasciitilde{}1000 stronicową książkę.

Dwie książki różnią się jedynie tytułem, nazwą pliku (UAL-XX.pdf versus RE4B-XX.pdf),
URLem i kilkoma pierwszymi stronami.

\subsection*{O inżynierii wstecznej}

Termin \q{\gls{reverse engineering}} ma kilka popularnych definicji:

1) inżynieria wsteczna oprogramowania; analiza skompilowanych programów; 

2) skanowanie modelu w 3D, żeby następnie go skopiować;

3) odzyskiwanie struktury \ac{DBMS}.

Nasza książka będzie powiązana z tą pierwszą definicją.

\subsection*{Pożądana wiedza}

Podstawowa znajomość C \ac{PL} .
Polecane materiały: \myref{CCppBooks}.

\subsection*{Ćwiczenia i zadania}

\dots 
wszystkie są na osobnej stronie: \url{http://challenges.re}.

\iffalse
\subsection*{O autorze}
\begin{tabularx}{\textwidth}{ l X }

\raisebox{-\totalheight}{
\includegraphics[scale=0.60]{Dennis_Yurichev.jpg}
}

&
Dennis Yurichev~--- reverse engineer i programista z dużym doświadczeniem.
Można się z nim skontaktować mailowo: \textbf{\EMAILS{}}.

% FIXME: no link. \tablefootnote doesn't work
\end{tabularx}
\fi

% subsections:
\subsection*{Pochwały dla książki}

\url{https://beginners.re/\#praise}.


\subsection*{Uczelnie}

Ta książka jest polecana przynajmniej na poniższych uczelniach:
\url{https://beginners.re/\#uni}.


\ifdefined\RUSSIAN
\newcommand{\PeopleMistakesInaccuraciesRusEng}{Александр Лысенко, Федерико Рамондино, Марк Уилсон, Разихова Мейрамгуль Кайратовна, Анатолий Прокофьев, Костя Бегунец, Валентин ``netch'' Нечаев, Александр Плахов, Артем Метла, Александр Ястребов, Влад Головкин\footnote{goto-vlad@github}, Евгений Прошин, Александр Мясников, Алексей Третьяков, Олег Песков, Павел Шахов}
\else
\newcommand{\PeopleMistakesInaccuraciesRusEng}{Alexander Lysenko, Federico Ramondino, Mark Wilson, Razikhova Meiramgul Kayratovna, Anatoly Prokofiev, Kostya Begunets, Valentin ``netch'' Nechayev, Aleksandr Plakhov, Artem Metla, Alexander Yastrebov, Vlad Golovkin\footnote{goto-vlad@github}, Evgeny Proshin, Alexander Myasnikov, Alexey Tretiakov, Oleg Peskov, Pavel Shakhov}
\fi

\newcommand{\PeopleMistakesInaccuracies}{\PeopleMistakesInaccuraciesRusEng{}, Zhu Ruijin, Changmin Heo, Vitor Vidal, Stijn Crevits, Jean-Gregoire Foulon\footnote{\url{https://github.com/pixjuan}}, Ben L., Etienne Khan, Norbert Szetei\footnote{\url{https://github.com/73696e65}}, Marc Remy, Michael Hansen, Derk Barten, The Renaissance\footnote{\url{https://github.com/TheRenaissance}}, Hugo Chan, Emil Mursalimov, Tanner Hoke, Tan90909090@GitHub, Ole Petter Orhagen, Sourav Punoriyar, Vitor Oliveira, Alexis Ehret, Maxim Shlochiski,
Greg Paton, Pierrick Lebourgeois, Abdullah Alomair.}

\newcommand{\PeopleItalianTranslators}{Federico Ramondino\footnote{\url{https://github.com/pinkrab}},
Paolo Stivanin\footnote{\url{https://github.com/paolostivanin}}, twyK, Fabrizio Bertone, Matteo Sticco, Marco Negro\footnote{\url{https://github.com/Internaut401}}, bluepulsar}

\newcommand{\PeopleFrenchTranslators}{Florent Besnard\footnote{\url{https://github.com/besnardf}}, Marc Remy\footnote{\url{https://github.com/mremy}}, Baudouin Landais, Téo Dacquet\footnote{\url{https://github.com/T30rix}}, BlueSkeye@GitHub\footnote{\url{https://github.com/BlueSkeye}}}

\newcommand{\PeopleGermanTranslators}{Dennis Siekmeier\footnote{\url{https://github.com/DSiekmeier}},
Julius Angres\footnote{\url{https://github.com/JAngres}}, Dirk Loser\footnote{\url{https://github.com/PolymathMonkey}}, Clemens Tamme, Philipp Schweinzer}

\newcommand{\PeopleSpanishTranslators}{Diego Boy, Luis Alberto Espinosa Calvo, Fernando Guida, Diogo Mussi, Patricio Galdames,
Emiliano Estevarena}

\newcommand{\PeoplePTBRTranslators}{Thales Stevan de A. Gois, Diogo Mussi, Luiz Filipe, Primo David Santini}

\newcommand{\PeoplePolishTranslators}{Kateryna Rozanova, Aleksander Mistewicz, Wiktoria Lewicka, Marcin Sokołowski}

\newcommand{\PeopleJapaneseTranslators}{%
shmz@github\footnote{\url{https://github.com/shmz}},%
4ryuJP@github\footnote{\url{https://github.com/4ryuJP}}}

\EN{\input{thanks_EN}}
\ES{\input{thanks_ES}}
\NL{\input{thanks_NL}}
\RU{\input{thanks_RU}}
\IT{\input{thanks_IT}}
\FR{\input{thanks_FR}}
\DE{\input{thanks_DE}}
%\CN{\input{thanks_CN}}
\JA{\input{thanks_JA}}
\PL{\input{thanks_PL}}
\CN{\input{thanks_CN}}


\subsection*{mini-FAQ}

\par Q: Czy ta książka jest prostsza niż inne?
\par A: Nie, poziom trudności jest mniej więcej taki sam jak innych książek na ten temat.

\par Q: Obawiam się zacząć czytać tę książkę, ma ponad 1000 stron.
"... dla początkujących" w nazwie brzmi nieco ironicznie.
\par A: Wszelkiego rodzaju kody źródłowe stanowią większość tej książki.
Ta książka naprawdę jest dla początkujących, wiele w niej (jeszcze) brakuje.

\par Q: Co trzeba wiedzieć zanim się przystąpi do czytania książki?
\par A: Umiejętności С/С++ są pożądane, ale nie są niezbędne.

\par Q: Czy powinienem uczyć się jednocześnie x86/x64/ARM i MIPS? Czy to nie za dużo?
\par A: Myślę, że na początek wystarczy czytać tylko o x86/x64, części o ARM i MIPS można pominąć.

\par Q: Czy można zakupić książki w wersji papierowej w języku rosyjskim lub angielskim?
\par A: Niestety nie, żaden wydawca jeszcze się nie zainteresował wydaniem rosyjskiej lub angielskiej wersji. Natomiast można ją wydrukować i zbindować w każdym ksero.
\url{https://yurichev.com/news/20200222_printed_RE4B/}.

\par Q: Czy istnieje wersja epub/mobi?
\par A: Nie. W wielu miejscach książka korzysta z hacków specyficznych dla TeXa/LaTeXa, dlatego przerobienie jej na HTML
(epub/mobi to jest HTML) nie jest łatwe.

\par Q: Po co uczyć się asemblera w dzisiejszych czasach?
\par A: Jeśli nie jest się programistą \ac{OS}, to prawdopodobnie nie trzeba nic pisać w asemblerze: współczesne kompilatory optymalizują kod lepiej niż człowiek \footnote{Bardzo ciekawy artykuł na ten temat: \InSqBrackets{\AgnerFog}}.

Do tego współczesne \ac{CPU} są bardzo skomplikowanymi urządzeniami i znajomość asemblera nie pomoże poznać ich mechanizmów wewnętrznych.

Jednak zostają dwa obszary, w których dobra znajomość asemblera może być pomocna:
1) badanie malware (złośliwego oprogramowania) w celu jego analizy ; 2) lepsze zrozumienie skompilowanego kodu w trakcie debuggowania.

Wobec tego ta książka jest napisana dla tych ludzi, którzy raczej chcą rozumieć assembler, a nie w nim pisać. Stąd jest w niej bardzo dużo przykładów - wyjść kompilatora.

\par Q: Kliknąłem w odnośnik wewnątrz pliku PDF, jak teraz wrócić?
\par A: W Adobe Acrobat Reader trzeba wcisnąć Alt+LeftArrow. W Evince wcisnąć "<".

\par Q: Czy mogę wydrukować tę książkę? Korzystać z niej do nauczania?
\par A: Oczywiście, właśnie dlatego ta książka ma licencję Creative Commons (CC BY-SA 4.0).

\par Q: Dlaczego ta książka jest darmowa? Wykonałeś świetną robotę. To podejrzane, podobnie jak z innymi rzeczami za darmo.
\par A: Moim zdaniem autorzy literatury technicznej robią to dla autoreklamy. Taka praca nie przynosi za dużo pieniędzy.

\par Q: Jak znaleźć pracę w zawodzie reverse engineeraа?
\par A: Na reddit (RE\FNURLREDDIT), od czasu od czasu pojawiają się wątki poszukiwania pracowników.
Możesz spróbować tam poszukać.

\par Q: Mam pytanie...
\par A: Napisz do mnie maila (\EMAILS).



\subsection*{O tłumaczeniu na język koreański}

W styczniu 2015, wydawnictwo Acorn (\href{http://www.acornpub.co.kr}{www.acornpub.co.kr}) z Korei Południowej wykonało ciężką pracę, żeby przetłumaczyć i wydać moją książkę  (stanem na sierpień 2014) w języku koreańskim.
Jest ona teraz dostępna na \href{http://www.acornpub.co.kr/book/reversing-for-beginners}{ich stronie}.

\iffalse
\begin{figure}[H]
\centering
\includegraphics[scale=0.3]{acorn_cover.jpg}
\end{figure}
\fi

Tłumaczył Byungho Min (\href{https://twitter.com/tais9}{twitter/tais9}).
Okładka namalował mój dobry przyjaciel, artysta, Andy Nechaevsky
\href{https://www.facebook.com/andydinka}{facebook/andydinka}.

Acorn również ma prawa autorskie do tłumaczenia koreańskiego.
Jakbyście chcieli mieć \emph{prawdziwą} książkę w języku koreańskim i chcielibyście wesprzeć moją pracę, możecie ją kupić.

\subsection*{O tłumaczeniu na język perski (farsi)}

W roku 2016 książkę przetłumaczył Mohsen Mostafa Jokar (znany w irańskiej społeczności z tłumaczenia instrukcji do Radare\footnote{\url{http://rada.re/get/radare2book-persian.pdf}})
Książka jest dostępna na stronie wydawnictwa \footnote{\url{http://goo.gl/2Tzx0H}} (Pendare Pars).

Pierwsze 40 stron: \url{https://beginners.re/farsi.pdf}.

Pozycja książki w Narodowej Bibliotece Iranu: \url{http://opac.nlai.ir/opac-prod/bibliographic/4473995}.

\subsection*{O tłumaczeniu na język chiński}

W kwietniu 2017, wydawnictwo PTPress skończyło tłumaczenie mojej książki na język chiński. Mają również prawo autorskie do tłumaczenia chińskiego.

Chińskie tłumaczenie można zamówić tutaj: \url{http://www.epubit.com.cn/book/details/4174}. Recenzje i historię tłumaczenia można znaleźć tutaj: \url{http://www.cptoday.cn/news/detail/3155}.

Głównym tłumaczem był Archer, u którego mam teraz dług wdzięczności.
Był bardzo dociekliwy i znalazł w książce sporo bugów i błędów, co jest szczególnie ważne w literaturze, której dotyczy ta książka.

Będę polecał go również innym autorom!

Chłopaki z \href{http://www.antiy.net/}{Antiy Labs} również pomogli z tłumaczeniem. \href{http://www.epubit.com.cn/book/onlinechapter/51413}{Tutaj słowo wstępne} napisane przez nich.


}
%\CN{% !TEX program = XeLaTeX
% !TEX encoding = UTF-8
\documentclass[UTF8,nofonts]{ctexart}
\setCJKmainfont[BoldFont=STHeiti,ItalicFont=STKaiti]{STSong}
\setCJKsansfont[BoldFont=STHeiti]{STXihei}
\setCJKmonofont{STFangsong}

\begin{document}

%daveti: translated on Dec 28, 2016
%NOTE: above is needed for MacTex.

\section*{前言 Preface}

术语\q{\gls{逆向工程 reverse engineering}}有好多种流行的说法:
1) 软件逆向工程: 研究编译器产生的程序;
2) 3D结构扫描和随后的用来复制该结构的数字模型重建;
3) Recreating \ac{DBMS} structure.
This book is about the first meaning.

\subsection*{前提条件 Prerequisites}

基本的C \ac{语言 PL}知识.
推荐阅读: \myref{CCppBooks}.

\subsection*{练习和项目 Exercises and tasks}

\dots 
都已经转移到了一个单独的网站: \url{http://challenges.re}.

\iffalse
\subsection*{关于作者 About the author}
\begin{tabularx}{\textwidth}{ l X }

\raisebox{-\totalheight}{
\includegraphics[scale=0.60]{Dennis_Yurichev.jpg}
}

&
Dennis Yurichev是一个经验丰富的逆向工程师和程序员.
可以通过邮件联系他: \textbf{\EMAILS{}}.

% FIXME: no link. \tablefootnote doesn't work
\end{tabularx}
\fi

% subsections:
% TBT \input{praise_CN}
% TBT \input{uni_CN}
\ifdefined\RUSSIAN
\newcommand{\PeopleMistakesInaccuraciesRusEng}{Александр Лысенко, Федерико Рамондино, Марк Уилсон, Разихова Мейрамгуль Кайратовна, Анатолий Прокофьев, Костя Бегунец, Валентин ``netch'' Нечаев, Александр Плахов, Артем Метла, Александр Ястребов, Влад Головкин\footnote{goto-vlad@github}, Евгений Прошин, Александр Мясников, Алексей Третьяков, Олег Песков, Павел Шахов}
\else
\newcommand{\PeopleMistakesInaccuraciesRusEng}{Alexander Lysenko, Federico Ramondino, Mark Wilson, Razikhova Meiramgul Kayratovna, Anatoly Prokofiev, Kostya Begunets, Valentin ``netch'' Nechayev, Aleksandr Plakhov, Artem Metla, Alexander Yastrebov, Vlad Golovkin\footnote{goto-vlad@github}, Evgeny Proshin, Alexander Myasnikov, Alexey Tretiakov, Oleg Peskov, Pavel Shakhov}
\fi

\newcommand{\PeopleMistakesInaccuracies}{\PeopleMistakesInaccuraciesRusEng{}, Zhu Ruijin, Changmin Heo, Vitor Vidal, Stijn Crevits, Jean-Gregoire Foulon\footnote{\url{https://github.com/pixjuan}}, Ben L., Etienne Khan, Norbert Szetei\footnote{\url{https://github.com/73696e65}}, Marc Remy, Michael Hansen, Derk Barten, The Renaissance\footnote{\url{https://github.com/TheRenaissance}}, Hugo Chan, Emil Mursalimov, Tanner Hoke, Tan90909090@GitHub, Ole Petter Orhagen, Sourav Punoriyar, Vitor Oliveira, Alexis Ehret, Maxim Shlochiski,
Greg Paton, Pierrick Lebourgeois, Abdullah Alomair.}

\newcommand{\PeopleItalianTranslators}{Federico Ramondino\footnote{\url{https://github.com/pinkrab}},
Paolo Stivanin\footnote{\url{https://github.com/paolostivanin}}, twyK, Fabrizio Bertone, Matteo Sticco, Marco Negro\footnote{\url{https://github.com/Internaut401}}, bluepulsar}

\newcommand{\PeopleFrenchTranslators}{Florent Besnard\footnote{\url{https://github.com/besnardf}}, Marc Remy\footnote{\url{https://github.com/mremy}}, Baudouin Landais, Téo Dacquet\footnote{\url{https://github.com/T30rix}}, BlueSkeye@GitHub\footnote{\url{https://github.com/BlueSkeye}}}

\newcommand{\PeopleGermanTranslators}{Dennis Siekmeier\footnote{\url{https://github.com/DSiekmeier}},
Julius Angres\footnote{\url{https://github.com/JAngres}}, Dirk Loser\footnote{\url{https://github.com/PolymathMonkey}}, Clemens Tamme, Philipp Schweinzer}

\newcommand{\PeopleSpanishTranslators}{Diego Boy, Luis Alberto Espinosa Calvo, Fernando Guida, Diogo Mussi, Patricio Galdames,
Emiliano Estevarena}

\newcommand{\PeoplePTBRTranslators}{Thales Stevan de A. Gois, Diogo Mussi, Luiz Filipe, Primo David Santini}

\newcommand{\PeoplePolishTranslators}{Kateryna Rozanova, Aleksander Mistewicz, Wiktoria Lewicka, Marcin Sokołowski}

\newcommand{\PeopleJapaneseTranslators}{%
shmz@github\footnote{\url{https://github.com/shmz}},%
4ryuJP@github\footnote{\url{https://github.com/4ryuJP}}}

\EN{\input{thanks_EN}}
\ES{\input{thanks_ES}}
\NL{\input{thanks_NL}}
\RU{\input{thanks_RU}}
\IT{\input{thanks_IT}}
\FR{\input{thanks_FR}}
\DE{\input{thanks_DE}}
%\CN{\input{thanks_CN}}
\JA{\input{thanks_JA}}
\PL{\input{thanks_PL}}
\CN{\input{thanks_CN}}


% !TEX program = XeLaTeX
% !TEX encoding = UTF-8
\documentclass[UTF8,nofonts]{ctexart}
\setCJKmainfont[BoldFont=STHeiti,ItalicFont=STKaiti]{STSong}
\setCJKsansfont[BoldFont=STHeiti]{STXihei}
\setCJKmonofont{STFangsong}

\begin{document}

%daveti: translated on Dec 25, 2016 (Merry Xmas!)
%NOTE: above is needed for MacTex.

\subsection*{迷你答疑 mini-FAQ}

% TBT
%\par Q: Is this book simpler/easier than others?
%\par A: No, it is at about the same level as other books of this subject.

% TBT
%\par Q: I'm too frightened to start reading this book, there are more than 1000 pages.
%\par A: All sorts of listings are the bulk of the book.

\par Q: 阅读本书的预备知识是什么?
\par A: 最好对C/C++有基本了解。
%TBT
\par Q: 我能买到俄语或者英语的纸板印刷品吗?
\par A: 抱歉,买不到,因为没有出版商对俄语或者英语版本有兴趣。当然,你可以自己找地方打印装订成书。

\par Q: 有epub/mobi版本吗?
\par A: 本书是用TeX/LaTeX编写和编译, 所以转换成HTML (epub/mobi是一种HTMLs)不是那么容易。\footnote{译者语:国内有一个早先的翻译版本,可以从这个链接下载 (\url{https://github.com/veficos/reverse-engineering-for-beginners})。注意,该版本基于某个较老版本翻译,已经不再和目前版本同步。}


\par Q: 这年头为啥还要学汇编呢?
\par A: 除非你是 \ac{OS} 程序员, 否则你基本不需要写汇编\textemdash{}最新的编译器 (2010s) 已经能产生比手动汇编优化更好的优化代码。 \footnote{推荐一个相关扩展阅读: \InSqBrackets{\AgnerFog}}.

而且,最新的 \ac{CPU}s 相当复杂,汇编知识已经不能帮助理解其内部构造。

然而懂汇编至少对2个领域依然有帮助:
第一个也是最重要的一个,安全/病毒研究 (security/malware research)。另外,在调试程序的时候,对汇编的理解也能帮助理解编译器产生的代码。
因此,本书着重于帮助读者理解汇编语言而不是手动写汇编语言。
这也是为什么本书含有大量的编译器产生的汇编代码。

\par Q: 我点击了一个PDF里的链接,然后如何回到PDF?
\par A: 在Adobe Acrobat Reader点击Alt+LeftArrow。在Evince点击 ``<'' 按键。

\par Q: 我能打印本书或者作为教材教课吗?
\par A: 必须的!本书使用Creative Commons license (CC BY-SA 4.0)就是为了这个目的。

\par Q: 为啥本书免费?这本书很牛比,而且免费,让人生疑。
\par A:  根据我的经验,技术类书籍作者通常通过免费来打广告。而且通常写这类书赚不到什么银子。

\par Q: 如何能找到一个逆向工程师的工作呢?How does one get a job in reverse engineering?
\par A: Reddit上有一个专门的用来招聘逆向工程师的论坛\FNURLREDDIT{}。去那儿看看。

另一个相关的招聘信息可以在\q{netsec} subreddit找到。

\par Q: 我想提问。。。
\par A: 给俺发邮件 (\EMAILS)。

\end{document}



\subsection*{关于韩语翻译 About the Korean translation}

2015一月,南韩的Acorn出版社 (\href{http://www.acornpub.co.kr}{www.acornpub.co.kr}) 做了大量的翻译工作并且最终把这本书(2014年8月的版本)韩语版出版。

现在可以从这里买到\href{http://www.acornpub.co.kr/book/reversing-for-beginners}{出版社网站}.

\iffalse
\begin{figure}[H]
\centering
\includegraphics[scale=0.3]{acorn_cover.jpg}
\end{figure}
\fi

译者是Byungho Min (\href{https://twitter.com/tais9}{twitter/tais9}).
封面依然由我的艺术家哥们Andy Nechaevsky设计:
\href{https://www.facebook.com/andydinka}{facebook/andydinka}.
他们共同持有韩语翻译版的版权。

所以,如果你想有一本\emph{真正的 real}韩语版本书而且顺便支持我的工作,请购买该书。

\end{document}

%\subsection*{About the Persian/Farsi translation}
%TBT

}
\JA{\section*{はじめに}

\subsection*{どうしてタイトルが2つあるの?}
\label{TwoTitles}

2014年~2018年に``Reverse Engineering for Beginners''という名前を付けていましたが、読者層が狭すぎるといつも思っていました。

情報セキュリティの人々は ``リバースエンジニアリング''について知っていますが、私はめったに ``アセンブラ'' という言葉を聞きませんでした。

同様に、``リバースエンジニアリング''という用語は、一般的なプログラマの読者にとってはやや暗黙の言葉ですが、``アセンブラ''については知っています。

2018年7月、実験として``初心者のためのアセンブリ言語''のタイトルを変更し、Hacker Newsのウェブサイトへのリンク\footnote{\url{https://news.ycombinator.com/item?id=17549050}}を掲載しました。
この本は一般によく受け入れられました。

そんなわけでなすがままにして、タイトルを2つにしてあります。

しかし、2番目のタイトルを``アセンブリ言語を理解する''に変更しました。これは、すでに誰かが``初心者のためのアセンブリ言語''という本を既に書いているからです。 
また、``初心者のため''という言葉は\textasciitilde{}1000ページ以上ある本だと皮肉に聞こえます。

2つの書籍は、タイトルとファイル名(UAL-XX.pdfに対してRE4B-XX.pdf)、URLと最初の数ページのみ異なります。

\subsection*{リバースエンジニアリングについて}

\q{\gls{reverse engineering}}にはよく知られた意味がいくつかあります。

1)ソフトウェアのリバースエンジニアリング、コンパイルされたプログラムの研究

2)3D構造のスキャンと、それらを複製するために必要なその後のデジタル操作

3)\ac{DBMS} 構造の再構成

本書は最初の意味についての本です。

\subsection*{前提条件}

Cプログラミング言語( \ac{PL} )の基礎知識。
推奨図書: \myref{CCppBooks}

\subsection*{練習問題やタスク}

\dots
\url{http://challenges.re} にあります。

\iffalse
\subsection*{著者について}
\begin{tabularx}{\textwidth}{ l X }

\raisebox{-\totalheight}{
\includegraphics[scale=0.60]{Dennis_Yurichev.jpg}
}

&
Dennis Yurichevは経験豊富なリバースエンジニアでありまたプログラマです。
彼には メールアドレス \textbf{\EMAILS{}}.

% FIXME: no link. \tablefootnote doesn't work
\end{tabularx}
\fi

% subsections:
\subsection*{賛辞}

\url{https://beginners.re/\#praise}.


% TBT \input{uni_JA}
\ifdefined\RUSSIAN
\newcommand{\PeopleMistakesInaccuraciesRusEng}{Александр Лысенко, Федерико Рамондино, Марк Уилсон, Разихова Мейрамгуль Кайратовна, Анатолий Прокофьев, Костя Бегунец, Валентин ``netch'' Нечаев, Александр Плахов, Артем Метла, Александр Ястребов, Влад Головкин\footnote{goto-vlad@github}, Евгений Прошин, Александр Мясников, Алексей Третьяков, Олег Песков, Павел Шахов}
\else
\newcommand{\PeopleMistakesInaccuraciesRusEng}{Alexander Lysenko, Federico Ramondino, Mark Wilson, Razikhova Meiramgul Kayratovna, Anatoly Prokofiev, Kostya Begunets, Valentin ``netch'' Nechayev, Aleksandr Plakhov, Artem Metla, Alexander Yastrebov, Vlad Golovkin\footnote{goto-vlad@github}, Evgeny Proshin, Alexander Myasnikov, Alexey Tretiakov, Oleg Peskov, Pavel Shakhov}
\fi

\newcommand{\PeopleMistakesInaccuracies}{\PeopleMistakesInaccuraciesRusEng{}, Zhu Ruijin, Changmin Heo, Vitor Vidal, Stijn Crevits, Jean-Gregoire Foulon\footnote{\url{https://github.com/pixjuan}}, Ben L., Etienne Khan, Norbert Szetei\footnote{\url{https://github.com/73696e65}}, Marc Remy, Michael Hansen, Derk Barten, The Renaissance\footnote{\url{https://github.com/TheRenaissance}}, Hugo Chan, Emil Mursalimov, Tanner Hoke, Tan90909090@GitHub, Ole Petter Orhagen, Sourav Punoriyar, Vitor Oliveira, Alexis Ehret, Maxim Shlochiski,
Greg Paton, Pierrick Lebourgeois, Abdullah Alomair.}

\newcommand{\PeopleItalianTranslators}{Federico Ramondino\footnote{\url{https://github.com/pinkrab}},
Paolo Stivanin\footnote{\url{https://github.com/paolostivanin}}, twyK, Fabrizio Bertone, Matteo Sticco, Marco Negro\footnote{\url{https://github.com/Internaut401}}, bluepulsar}

\newcommand{\PeopleFrenchTranslators}{Florent Besnard\footnote{\url{https://github.com/besnardf}}, Marc Remy\footnote{\url{https://github.com/mremy}}, Baudouin Landais, Téo Dacquet\footnote{\url{https://github.com/T30rix}}, BlueSkeye@GitHub\footnote{\url{https://github.com/BlueSkeye}}}

\newcommand{\PeopleGermanTranslators}{Dennis Siekmeier\footnote{\url{https://github.com/DSiekmeier}},
Julius Angres\footnote{\url{https://github.com/JAngres}}, Dirk Loser\footnote{\url{https://github.com/PolymathMonkey}}, Clemens Tamme, Philipp Schweinzer}

\newcommand{\PeopleSpanishTranslators}{Diego Boy, Luis Alberto Espinosa Calvo, Fernando Guida, Diogo Mussi, Patricio Galdames,
Emiliano Estevarena}

\newcommand{\PeoplePTBRTranslators}{Thales Stevan de A. Gois, Diogo Mussi, Luiz Filipe, Primo David Santini}

\newcommand{\PeoplePolishTranslators}{Kateryna Rozanova, Aleksander Mistewicz, Wiktoria Lewicka, Marcin Sokołowski}

\newcommand{\PeopleJapaneseTranslators}{%
shmz@github\footnote{\url{https://github.com/shmz}},%
4ryuJP@github\footnote{\url{https://github.com/4ryuJP}}}

\EN{\input{thanks_EN}}
\ES{\input{thanks_ES}}
\NL{\input{thanks_NL}}
\RU{\input{thanks_RU}}
\IT{\input{thanks_IT}}
\FR{\input{thanks_FR}}
\DE{\input{thanks_DE}}
%\CN{\input{thanks_CN}}
\JA{\input{thanks_JA}}
\PL{\input{thanks_PL}}
\CN{\input{thanks_CN}}


\subsection*{mini-FAQ}

% TBT
%\par Q: Is this book simpler/easier than others?
%\par A: No, it is at about the same level as other books of this subject.
% TBT
%\par Q: I'm too frightened to start reading this book, there are more than 1000 pages.
%\par A: All sorts of listings are the bulk of the book.

\par Q:この本を読むための前提条件は何ですか?
\par A:C/C ++の基本的な理解があるのが望ましいです。

\par Q:x86/x64/ARMとMIPSを本当にすぐに学ぶべきでしょうか?それはあまりにも大変ではないですか?
\par A:初心者は、ARMとMIPSの部分をスキップまたはスキミングしながら、x86/x64だけを読んでもいいです。

\par Q:ロシア語または英語のハードカバー/ペーパーブックを購入できますか?
\par A:残念ながら、いいえ。これまでにロシア語版や英語版を出版することに興味を持った出版社はいませんでした。その間に、お気に入りのコピーショップに印刷してバインドするよう依頼することができます。
\url{https://yurichev.com/news/20200222_printed_RE4B/}.

\par Q:epubまたはmobiのバージョンはありますか?
\par A:いいえ。本はTeX / LaTeX固有のハッキングに強く依存しているので、HTML(epub / mobiはHTMLの集合です)への変換は簡単ではありません。

\par Q:最近、アセンブリ言語を学ばなければならないのはなぜですか?
\par A:あなたが\ac{OS}開発者でない限り、アセンブラでコード化する必要はないでしょう。最新のコンパイラ(2010s)は、人間よりも最適化を実行する方がはるかに優れています\footnote{このトピックに関する非常に良いテキスト:\InSqBrackets{\AgnerFog}}

また、最新のCPUは非常に複雑なデバイスであり、アセンブリの知識は内部を理解するのに役立つものではありません。

それは、少なくとも2つの領域があり、アセンブリの理解を深めることが役立つことがあります。まず、セキュリティ/マルウェアの研究に役立つことです。 また、デバッグ中にコンパイルされたコードをよりよく理解するための良い方法です。 したがって、この本は、アセンブリ言語を記述するのではなく、アセンブリ言語を理解したい人のために用意されています。そのため、コンパイラ出力の例が多数含まれています。

\par Q:PDF文書内のハイパーリンクをクリックしましたが、どのように戻ってきますか?
\par A:Adobe Acrobat ReaderでAlt+左矢印をクリックします。 Evinceで `` <''ボタンをクリックしてください。

\par Q:この本を印刷して教えてもいいですか?
\par A:もちろん!だからこの本はクリエイティブコモンズライセンス(CC BY-SA 4.0)に基づいてライセンスされています。

\par Q:なぜこの本は無料ですか?あなたは素晴らしい仕事をしてきました。これは他の多くの自由なものと同様に疑わしいものです。
\par A:私自身の経験では、技術文献の著者は主に自己宣伝の目的で書いています。そのような仕事からまともな金を稼ぐことはできません。

\par Q:リバースエンジニアリングではどのように仕事をしていますか?
\par A:redditには時々現れる採用スレッドがあり、RE\FNURLREDDIT{}に専念しています。そこを見てみてください。

多少関連する採用スレッドは、netsecサブディレクトリにあります。

\par Q:質問があります...
\par A:私にメールしてください(\EMAILS)


\subsection*{韓国語版について}

2015年1月、韓国のAcorn出版社(\href{http://www.acornpub.co.kr}{www.acornpub.co.kr})は、この書籍を2014年8月に韓国語に翻訳して出版する際に膨大な作業をしました。

現在、\href{http://www.acornpub.co.kr/book/reversing-for-beginners}{彼らのウェブサイト}で利用可能です。

\iffalse
\begin{figure}[H]
\centering
\includegraphics[scale=0.3]{acorn_cover.jpg}
\end{figure}
\fi

翻訳者はByungho Min (\href{https://twitter.com/tais9}{twitter/tais9})です。 カバーアートは、作家の友人Andy Nechaevsky(\href{https://www.facebook.com/andydinka}{facebook/andydinka})によって行われました。 Acornには、韓国語翻訳の著作権もあります。

そして、あなたが韓国語で本棚にリアルな本がほしくて、この作品をサポートしたいなら、現在購入可能です。

\subsection*{ペルシア/ファルシ語版について}

2016年にこの本はMohsen Mostafa Jokar(Radadeのマニュアルを翻訳していて、イランのコミュニティにも知られています)によって翻訳されました。 出版社のウェブサイト(Pendare Pars)で入手できます。

ここに40ページの抜粋へのリンクがあります:\url{https://beginners.re/farsi.pdf}

イラン国立図書館登録情報:\url{http://opac.nlai.ir/opac-prod/bibliographic/4473995}

\subsection*{中国語版について}

2017年4月、中国語への翻訳は中国のPTPressによって完了しました。 中国語の著作権者でもあります。

中国語版はこちらから注文できます:\url{http://www.epubit.com.cn/book/details/4174}

翻訳の背後にある部分的なレビューと履歴は、ここにあります:\url{http://www.cptoday.cn/news/detail/3155}

主な翻訳者はArcherです。 彼は非常に細心の注意を払っており、この本のような文献では非常に重要な既知の間違いやバグのほとんどを報告していました。 私は彼のサービスを他の著者に推薦します!

\href{http://www.antiy.net/}{Antiy Labs}のスタッフも翻訳を手伝ってくれました。 \href{http://www.epubit.com.cn/book/onlinechapter/51413}{ここ}に彼らが書いた序文があります。
}


\mainmatter

% only chapters here!
\EN{\mysection{Task manager practical joke (Windows Vista)}
\myindex{Windows!Windows Vista}

Let's see if it's possible to hack Task Manager slightly so it would detect more \ac{CPU} cores.

\myindex{Windows!NTAPI}

Let us first think, how does the Task Manager know the number of cores?

There is the \TT{GetSystemInfo()} win32 function present in win32 userspace which can tell us this.
But it's not imported in \TT{taskmgr.exe}.

There is, however, another one in \gls{NTAPI}, \TT{NtQuerySystemInformation()}, 
which is used in \TT{taskmgr.exe} in several places.

To get the number of cores, one has to call this function with the \TT{SystemBasicInformation} constant
as a first argument (which is zero
\footnote{\href{http://msdn.microsoft.com/en-us/library/windows/desktop/ms724509(v=vs.85).aspx}{MSDN}}).

The second argument has to point to the buffer which is getting all the information.

So we have to find all calls to the \\
\TT{NtQuerySystemInformation(0, ?, ?, ?)} function.
Let's open \TT{taskmgr.exe} in IDA. 
\myindex{Windows!PDB}

What is always good about Microsoft executables is that IDA can download the corresponding \gls{PDB} 
file for this executable and show all function names.

It is visible that Task Manager is written in \Cpp and some of the function names and classes are really 
speaking for themselves.
There are classes CAdapter, CNetPage, CPerfPage, CProcInfo, CProcPage, CSvcPage, 
CTaskPage, CUserPage.

Apparently, each class corresponds to each tab in Task Manager.

Let's visit each call and add comment with the value which is passed as the first function argument.
We will write \q{not zero} at some places, because the value there was clearly not zero, 
but something really different (more about this in the second part of this chapter).

And we are looking for zero passed as argument, after all.

\begin{figure}[H]
\centering
\myincludegraphics{examples/taskmgr/IDA_xrefs.png}
\caption{IDA: cross references to NtQuerySystemInformation()}
\end{figure}

Yes, the names are really speaking for themselves.

When we closely investigate each place where\\
\TT{NtQuerySystemInformation(0, ?, ?, ?)} is called,
we quickly find what we need in the \TT{InitPerfInfo()} function:

\lstinputlisting[caption=taskmgr.exe (Windows Vista),style=customasmx86]{examples/taskmgr/taskmgr.lst}

\TT{g\_cProcessors} is a global variable, and this name has been assigned by 
IDA according to the \gls{PDB} loaded from Microsoft's symbol server.

The byte is taken from \TT{var\_C20}. 
And \TT{var\_C58} is passed to\\
\TT{NtQuerySystemInformation()} 
as a pointer to the receiving buffer.
The difference between 0xC20 and 0xC58 is 0x38 (56).

Let's take a look at format of the return structure, which we can find in MSDN:

\begin{lstlisting}[style=customc]
typedef struct _SYSTEM_BASIC_INFORMATION {
    BYTE Reserved1[24];
    PVOID Reserved2[4];
    CCHAR NumberOfProcessors;
} SYSTEM_BASIC_INFORMATION;
\end{lstlisting}

This is a x64 system, so each PVOID takes 8 bytes.

All \emph{reserved} fields in the structure take $24+4*8=56$ bytes.

Oh yes, this implies that \TT{var\_C20} is the local stack is exactly the
\TT{NumberOfProcessors} field of the \TT{SYSTEM\_BASIC\_INFORMATION} structure.

Let's check our guess.
Copy \TT{taskmgr.exe} from \TT{C:\textbackslash{}Windows\textbackslash{}System32} 
to some other folder 
(so the \emph{Windows Resource Protection} 
will not try to restore the patched \TT{taskmgr.exe}).

Let's open it in Hiew and find the place:

\begin{figure}[H]
\centering
\myincludegraphics{examples/taskmgr/hiew2.png}
\caption{Hiew: find the place to be patched}
\end{figure}

Let's replace the \TT{MOVZX} instruction with ours.
Let's pretend we've got 64 CPU cores.

Add one additional \ac{NOP} (because our instruction is shorter than the original one):

\begin{figure}[H]
\centering
\myincludegraphics{examples/taskmgr/hiew1.png}
\caption{Hiew: patch it}
\end{figure}

And it works!
Of course, the data in the graphs is not correct.

At times, Task Manager even shows an overall CPU load of more than 100\%.

\begin{figure}[H]
\centering
\myincludegraphics{examples/taskmgr/taskmgr_64cpu_crop.png}
\caption{Fooled Windows Task Manager}
\end{figure}

The biggest number Task Manager does not crash with is 64.

Apparently, Task Manager in Windows Vista was not tested on computers with a large number of cores.

So there are probably some static data structure(s) inside it limited to 64 cores.

\subsection{Using LEA to load values}
\label{TaskMgr_LEA}

Sometimes, \TT{LEA} is used in \TT{taskmgr.exe} instead of \TT{MOV} to set the first argument of \\
\TT{NtQuerySystemInformation()}:

\lstinputlisting[caption=taskmgr.exe (Windows Vista),style=customasmx86]{examples/taskmgr/taskmgr2.lst}

\myindex{x86!\Instructions!LEA}

Perhaps \ac{MSVC} did so because machine code of \INS{LEA} is shorter than \INS{MOV REG, 5} (would be 5 instead of 4).

\INS{LEA} with offset in $-128..127$ range (offset will occupy 1 byte in opcode) with 32-bit registers is even shorter (for lack of REX prefix)---3 bytes.

Another example of such thing is: \myref{using_MOV_and_pack_of_LEA_to_load_values}.
}%
\RU{\subsection{Обменять входные значения друг с другом}

Вот так:

\lstinputlisting[style=customc]{patterns/061_pointers/swap/5_RU.c}

Как видим, байты загружаются в младшие 8-битные части регистров \TT{ECX} и \TT{EBX} используя \INS{MOVZX}
(так что старшие части регистров очищаются), затем байты записываются назад в другом порядке.

\lstinputlisting[style=customasmx86,caption=Optimizing GCC 5.4]{patterns/061_pointers/swap/5_GCC_O3_x86.s}

Адреса обоих байтов берутся из аргументов и во время исполнения ф-ции находятся в регистрах \TT{EDX} и \TT{EAX}.

Так что исопльзуем указатели --- вероятно, без них нет способа решить эту задачу лучше.

}%
\FR{\subsection{Exemple \#2: SCO OpenServer}

\label{examples_SCO}
\myindex{SCO OpenServer}
Un ancien logiciel pour SCO OpenServer de 1997 développé par une société qui a disparue
depuis longtemps.

Il y a un driver de dongle special à installer dans le système, qui contient les
chaînes de texte suivantes:
\q{Copyright 1989, Rainbow Technologies, Inc., Irvine, CA}
et
\q{Sentinel Integrated Driver Ver. 3.0 }.

Après l'installation du driver dans SCO OpenServer, ces fichiers apparaissent dans
l'arborescence /dev:

\begin{lstlisting}
/dev/rbsl8
/dev/rbsl9
/dev/rbsl10
\end{lstlisting}

Le programme renvoie une erreur lorsque le dongle n'est pas connecté, mais le message
d'erreur n'est pas trouvé dans les exécutables.

\myindex{COFF}

Grâce à \ac{IDA}, il est facile de charger l'exécutable COFF utilisé dans SCO OpenServer.

Essayons de trouver la chaîne \q{rbsl} et en effet, elle se trouve dans ce morceau
de code:

\lstinputlisting[style=customasmx86]{examples/dongles/2/1.lst}

Oui, en effet, le programme doit communiquer d'une façon ou d'une autre avec le driver.

\myindex{thunk-functions}
Le seul endroit où la fonction \TT{SSQC()} est appelée est dans la \glslink{thunk
 function}{fonction thunk}:

\lstinputlisting[style=customasmx86]{examples/dongles/2/2.lst}

SSQ() peut être appelé depuis au moins 2 fonctions.

L'une d'entre elles est:

\lstinputlisting[style=customasmx86]{examples/dongles/2/check1_EN.lst}

\q{\TT{3C}} et \q{\TT{3E}} semblent familiers: il y avait un dongle Sentinel Pro de
Rainbow sans mémoire, fournissant seulement une fonction de crypto-hachage secrète.

Vous pouvez lire une courte description de la fonction de hachage dont il s'agit
ici: \myref{hash_func}.

Mais retournons au programme.

Donc le programme peut seulement tester si un dongle est connecté ou s'il est absent.

Aucune autre information ne peut être écrite dans un tel dongle, puisqu'il n'a pas
de mémoire.
Les codes sur deux caractères sont des commandes (nous pouvons voir comment les commandes
sont traitées dans la fonction \TT{SSQC()}) et toutes les autres chaînes sont hachées
dans le dongle, transformées en un nombre 16-bit.
L'algorithme était secret, donc il n'était pas possible d'écrire un driver de remplacement
ou de refaire un dongle matériel qui l'émulerait parfaitement.

Toutefois, il est toujours possible d'intercepter tous les accès au dongle et de
trouver les constantes auxquelles les résultats de la fonction de hachage sont comparées.

Mais nous devons dire qu'il est possible de construire un schéma de logiciel de protection
de copie robuste basé sur une fonction secrète de hachage cryptographique: il suffit
qu'elle chiffre/déchiffre les fichiers de données utilisés par votre logiciel.

Mais retournons au code:

Les codes 51/52/53 sont utilisés pour choisir le port imprimante LPT.
3x/4x sont utilisés pour le choix de la \q{famille} (c'est ainsi que les dongles
Sentinel Pro sont différenciés les uns des autres: plus d'un dongle peut être connecté
sur un port LPT).

La seule chaîne passée à la fonction qui ne fasse pas 2 caractères est "0123456789".

Ensuite, le résultat est comparé à l'ensemble des résultats valides.

Si il est correct, 0xC ou 0xB est écrit dans la variable globale \TT{ctl\_model}.%

Une autre chaîne de texte qui est passée est
"PRESS ANY KEY TO CONTINUE: ", mais le résultat n'est pas testé.
Difficile de dire pourquoi, probablement une erreur\footnote{C'est un sentiment
étrange de trouver un bug dans un logiciel aussi ancien.}.

Voyons où la valeur de la variable globale \TT{ctl\_model} est utilisée.

Un tel endroit est:

\lstinputlisting[style=customasmx86]{examples/dongles/2/4.lst}

Si c'est 0, un message d'erreur chiffré est passé à une routine de déchiffrement
et affiché.

\myindex{x86!\Instructions!XOR}

La routine de déchiffrement de la chaîne semble être un simple \glslink{xoring}{xor}:

\lstinputlisting[style=customasmx86]{examples/dongles/2/err_warn.lst}

C'est pourquoi nous étions incapable de trouver le message d'erreur dans les fichiers
exécutable, car ils sont chiffrés (ce qui est une pratique courante).

Un autre appel à la fonction de hachage \TT{SSQ()} lui passe la chaîne \q{offln}
et le résultat est comparé avec \TT{0xFE81} et \TT{0x12A9}.

Si ils ne correspondent pas, ça se comporte comme une sorte de fonction \TT{timer()}
(peut-être en attente qu'un dongle mal connecté soit reconnecté et re-testé?) et ensuite
déchiffre un autre message d'erreur à afficher.

\lstinputlisting[style=customasmx86]{examples/dongles/2/check2_EN.lst}

Passer outre le dongle est assez facile: il suffit de patcher tous les sauts après
les instructions \CMP pertinentes.

Une autre option est d'écrire notre propre driver SCO OpenServer, contenant une table
de questions et de réponses, toutes celles qui sont présentent dans le programme.

\subsubsection{Déchiffrer les messages d'erreur}

À propos, nous pouvons aussi essayer de déchiffrer tous les messages d'erreurs.
L'algorithme qui se trouve dans la fonction \TT{err\_warn()} est très simple, en effet:

\lstinputlisting[caption=Decryption function,style=customasmx86]{examples/dongles/2/decrypting_FR.lst}

Comme on le voit, non seulement la chaîne est transmise à la fonction de déchiffrement
mais aussi la clef:

\lstinputlisting[style=customasmx86]{examples/dongles/2/tmp1_EN.asm}

L'algorithme est un simple \glslink{xoring}{xor}: chaque octet est xoré avec la clef, mais
la clef est incrémentée de 3 après le traitement de chaque octet.

Nous pouvons écrire un petit script Python pour vérifier notre hypothèse:

\lstinputlisting[caption=Python 3.x]{examples/dongles/2/decr1.py}

Et il affiche: \q{check security device connection}.
Donc oui, ceci est le message déchiffré.

Il y a d'autres messages chiffrés, avec leur clef correspondante.
Mais inutile de dire qu'il est possible de les déchiffrer sans leur clef.
Premièrement, nous voyons que le clef est en fait un octet.
C'est parce que l'instruction principale de déchiffrement (\XOR) fonctionne au niveau
de l'octet.
La clef se trouve dans le registre \ESI, mais seulement une partie de \ESI d'un octet
est utilisée.
Ainsi, une clef pourrait être plus grande que 255, mais sa valeur est toujours arrondie.

En conséquence, nous pouvons simplement essayer de brute-forcer, en essayant toutes
les clefs possible dans l'intervalle 0..255.
Nous allons aussi écarter les messages comportants des caractères non-imprimable.

\lstinputlisting[caption=Python 3.x]{examples/dongles/2/decr2.py}

Et nous obtenons:

\lstinputlisting[caption=Results]{examples/dongles/2/decr2_result.txt}

Ici il y a un peu de déchet, mais nous pouvons rapidement trouver les messages en
anglais.

À propos, puisque l'algorithme est un simple chiffrement xor, la même fonction peut
être utilisée pour chiffrer les messages.
Si besoin, nous pouvons chiffrer nos propres messages, et patcher le programme en les insérant.
}


\EN{\mysection{Task manager practical joke (Windows Vista)}
\myindex{Windows!Windows Vista}

Let's see if it's possible to hack Task Manager slightly so it would detect more \ac{CPU} cores.

\myindex{Windows!NTAPI}

Let us first think, how does the Task Manager know the number of cores?

There is the \TT{GetSystemInfo()} win32 function present in win32 userspace which can tell us this.
But it's not imported in \TT{taskmgr.exe}.

There is, however, another one in \gls{NTAPI}, \TT{NtQuerySystemInformation()}, 
which is used in \TT{taskmgr.exe} in several places.

To get the number of cores, one has to call this function with the \TT{SystemBasicInformation} constant
as a first argument (which is zero
\footnote{\href{http://msdn.microsoft.com/en-us/library/windows/desktop/ms724509(v=vs.85).aspx}{MSDN}}).

The second argument has to point to the buffer which is getting all the information.

So we have to find all calls to the \\
\TT{NtQuerySystemInformation(0, ?, ?, ?)} function.
Let's open \TT{taskmgr.exe} in IDA. 
\myindex{Windows!PDB}

What is always good about Microsoft executables is that IDA can download the corresponding \gls{PDB} 
file for this executable and show all function names.

It is visible that Task Manager is written in \Cpp and some of the function names and classes are really 
speaking for themselves.
There are classes CAdapter, CNetPage, CPerfPage, CProcInfo, CProcPage, CSvcPage, 
CTaskPage, CUserPage.

Apparently, each class corresponds to each tab in Task Manager.

Let's visit each call and add comment with the value which is passed as the first function argument.
We will write \q{not zero} at some places, because the value there was clearly not zero, 
but something really different (more about this in the second part of this chapter).

And we are looking for zero passed as argument, after all.

\begin{figure}[H]
\centering
\myincludegraphics{examples/taskmgr/IDA_xrefs.png}
\caption{IDA: cross references to NtQuerySystemInformation()}
\end{figure}

Yes, the names are really speaking for themselves.

When we closely investigate each place where\\
\TT{NtQuerySystemInformation(0, ?, ?, ?)} is called,
we quickly find what we need in the \TT{InitPerfInfo()} function:

\lstinputlisting[caption=taskmgr.exe (Windows Vista),style=customasmx86]{examples/taskmgr/taskmgr.lst}

\TT{g\_cProcessors} is a global variable, and this name has been assigned by 
IDA according to the \gls{PDB} loaded from Microsoft's symbol server.

The byte is taken from \TT{var\_C20}. 
And \TT{var\_C58} is passed to\\
\TT{NtQuerySystemInformation()} 
as a pointer to the receiving buffer.
The difference between 0xC20 and 0xC58 is 0x38 (56).

Let's take a look at format of the return structure, which we can find in MSDN:

\begin{lstlisting}[style=customc]
typedef struct _SYSTEM_BASIC_INFORMATION {
    BYTE Reserved1[24];
    PVOID Reserved2[4];
    CCHAR NumberOfProcessors;
} SYSTEM_BASIC_INFORMATION;
\end{lstlisting}

This is a x64 system, so each PVOID takes 8 bytes.

All \emph{reserved} fields in the structure take $24+4*8=56$ bytes.

Oh yes, this implies that \TT{var\_C20} is the local stack is exactly the
\TT{NumberOfProcessors} field of the \TT{SYSTEM\_BASIC\_INFORMATION} structure.

Let's check our guess.
Copy \TT{taskmgr.exe} from \TT{C:\textbackslash{}Windows\textbackslash{}System32} 
to some other folder 
(so the \emph{Windows Resource Protection} 
will not try to restore the patched \TT{taskmgr.exe}).

Let's open it in Hiew and find the place:

\begin{figure}[H]
\centering
\myincludegraphics{examples/taskmgr/hiew2.png}
\caption{Hiew: find the place to be patched}
\end{figure}

Let's replace the \TT{MOVZX} instruction with ours.
Let's pretend we've got 64 CPU cores.

Add one additional \ac{NOP} (because our instruction is shorter than the original one):

\begin{figure}[H]
\centering
\myincludegraphics{examples/taskmgr/hiew1.png}
\caption{Hiew: patch it}
\end{figure}

And it works!
Of course, the data in the graphs is not correct.

At times, Task Manager even shows an overall CPU load of more than 100\%.

\begin{figure}[H]
\centering
\myincludegraphics{examples/taskmgr/taskmgr_64cpu_crop.png}
\caption{Fooled Windows Task Manager}
\end{figure}

The biggest number Task Manager does not crash with is 64.

Apparently, Task Manager in Windows Vista was not tested on computers with a large number of cores.

So there are probably some static data structure(s) inside it limited to 64 cores.

\subsection{Using LEA to load values}
\label{TaskMgr_LEA}

Sometimes, \TT{LEA} is used in \TT{taskmgr.exe} instead of \TT{MOV} to set the first argument of \\
\TT{NtQuerySystemInformation()}:

\lstinputlisting[caption=taskmgr.exe (Windows Vista),style=customasmx86]{examples/taskmgr/taskmgr2.lst}

\myindex{x86!\Instructions!LEA}

Perhaps \ac{MSVC} did so because machine code of \INS{LEA} is shorter than \INS{MOV REG, 5} (would be 5 instead of 4).

\INS{LEA} with offset in $-128..127$ range (offset will occupy 1 byte in opcode) with 32-bit registers is even shorter (for lack of REX prefix)---3 bytes.

Another example of such thing is: \myref{using_MOV_and_pack_of_LEA_to_load_values}.
}%
\RU{\subsection{Обменять входные значения друг с другом}

Вот так:

\lstinputlisting[style=customc]{patterns/061_pointers/swap/5_RU.c}

Как видим, байты загружаются в младшие 8-битные части регистров \TT{ECX} и \TT{EBX} используя \INS{MOVZX}
(так что старшие части регистров очищаются), затем байты записываются назад в другом порядке.

\lstinputlisting[style=customasmx86,caption=Optimizing GCC 5.4]{patterns/061_pointers/swap/5_GCC_O3_x86.s}

Адреса обоих байтов берутся из аргументов и во время исполнения ф-ции находятся в регистрах \TT{EDX} и \TT{EAX}.

Так что исопльзуем указатели --- вероятно, без них нет способа решить эту задачу лучше.

}%
\FR{\subsection{Exemple \#2: SCO OpenServer}

\label{examples_SCO}
\myindex{SCO OpenServer}
Un ancien logiciel pour SCO OpenServer de 1997 développé par une société qui a disparue
depuis longtemps.

Il y a un driver de dongle special à installer dans le système, qui contient les
chaînes de texte suivantes:
\q{Copyright 1989, Rainbow Technologies, Inc., Irvine, CA}
et
\q{Sentinel Integrated Driver Ver. 3.0 }.

Après l'installation du driver dans SCO OpenServer, ces fichiers apparaissent dans
l'arborescence /dev:

\begin{lstlisting}
/dev/rbsl8
/dev/rbsl9
/dev/rbsl10
\end{lstlisting}

Le programme renvoie une erreur lorsque le dongle n'est pas connecté, mais le message
d'erreur n'est pas trouvé dans les exécutables.

\myindex{COFF}

Grâce à \ac{IDA}, il est facile de charger l'exécutable COFF utilisé dans SCO OpenServer.

Essayons de trouver la chaîne \q{rbsl} et en effet, elle se trouve dans ce morceau
de code:

\lstinputlisting[style=customasmx86]{examples/dongles/2/1.lst}

Oui, en effet, le programme doit communiquer d'une façon ou d'une autre avec le driver.

\myindex{thunk-functions}
Le seul endroit où la fonction \TT{SSQC()} est appelée est dans la \glslink{thunk
 function}{fonction thunk}:

\lstinputlisting[style=customasmx86]{examples/dongles/2/2.lst}

SSQ() peut être appelé depuis au moins 2 fonctions.

L'une d'entre elles est:

\lstinputlisting[style=customasmx86]{examples/dongles/2/check1_EN.lst}

\q{\TT{3C}} et \q{\TT{3E}} semblent familiers: il y avait un dongle Sentinel Pro de
Rainbow sans mémoire, fournissant seulement une fonction de crypto-hachage secrète.

Vous pouvez lire une courte description de la fonction de hachage dont il s'agit
ici: \myref{hash_func}.

Mais retournons au programme.

Donc le programme peut seulement tester si un dongle est connecté ou s'il est absent.

Aucune autre information ne peut être écrite dans un tel dongle, puisqu'il n'a pas
de mémoire.
Les codes sur deux caractères sont des commandes (nous pouvons voir comment les commandes
sont traitées dans la fonction \TT{SSQC()}) et toutes les autres chaînes sont hachées
dans le dongle, transformées en un nombre 16-bit.
L'algorithme était secret, donc il n'était pas possible d'écrire un driver de remplacement
ou de refaire un dongle matériel qui l'émulerait parfaitement.

Toutefois, il est toujours possible d'intercepter tous les accès au dongle et de
trouver les constantes auxquelles les résultats de la fonction de hachage sont comparées.

Mais nous devons dire qu'il est possible de construire un schéma de logiciel de protection
de copie robuste basé sur une fonction secrète de hachage cryptographique: il suffit
qu'elle chiffre/déchiffre les fichiers de données utilisés par votre logiciel.

Mais retournons au code:

Les codes 51/52/53 sont utilisés pour choisir le port imprimante LPT.
3x/4x sont utilisés pour le choix de la \q{famille} (c'est ainsi que les dongles
Sentinel Pro sont différenciés les uns des autres: plus d'un dongle peut être connecté
sur un port LPT).

La seule chaîne passée à la fonction qui ne fasse pas 2 caractères est "0123456789".

Ensuite, le résultat est comparé à l'ensemble des résultats valides.

Si il est correct, 0xC ou 0xB est écrit dans la variable globale \TT{ctl\_model}.%

Une autre chaîne de texte qui est passée est
"PRESS ANY KEY TO CONTINUE: ", mais le résultat n'est pas testé.
Difficile de dire pourquoi, probablement une erreur\footnote{C'est un sentiment
étrange de trouver un bug dans un logiciel aussi ancien.}.

Voyons où la valeur de la variable globale \TT{ctl\_model} est utilisée.

Un tel endroit est:

\lstinputlisting[style=customasmx86]{examples/dongles/2/4.lst}

Si c'est 0, un message d'erreur chiffré est passé à une routine de déchiffrement
et affiché.

\myindex{x86!\Instructions!XOR}

La routine de déchiffrement de la chaîne semble être un simple \glslink{xoring}{xor}:

\lstinputlisting[style=customasmx86]{examples/dongles/2/err_warn.lst}

C'est pourquoi nous étions incapable de trouver le message d'erreur dans les fichiers
exécutable, car ils sont chiffrés (ce qui est une pratique courante).

Un autre appel à la fonction de hachage \TT{SSQ()} lui passe la chaîne \q{offln}
et le résultat est comparé avec \TT{0xFE81} et \TT{0x12A9}.

Si ils ne correspondent pas, ça se comporte comme une sorte de fonction \TT{timer()}
(peut-être en attente qu'un dongle mal connecté soit reconnecté et re-testé?) et ensuite
déchiffre un autre message d'erreur à afficher.

\lstinputlisting[style=customasmx86]{examples/dongles/2/check2_EN.lst}

Passer outre le dongle est assez facile: il suffit de patcher tous les sauts après
les instructions \CMP pertinentes.

Une autre option est d'écrire notre propre driver SCO OpenServer, contenant une table
de questions et de réponses, toutes celles qui sont présentent dans le programme.

\subsubsection{Déchiffrer les messages d'erreur}

À propos, nous pouvons aussi essayer de déchiffrer tous les messages d'erreurs.
L'algorithme qui se trouve dans la fonction \TT{err\_warn()} est très simple, en effet:

\lstinputlisting[caption=Decryption function,style=customasmx86]{examples/dongles/2/decrypting_FR.lst}

Comme on le voit, non seulement la chaîne est transmise à la fonction de déchiffrement
mais aussi la clef:

\lstinputlisting[style=customasmx86]{examples/dongles/2/tmp1_EN.asm}

L'algorithme est un simple \glslink{xoring}{xor}: chaque octet est xoré avec la clef, mais
la clef est incrémentée de 3 après le traitement de chaque octet.

Nous pouvons écrire un petit script Python pour vérifier notre hypothèse:

\lstinputlisting[caption=Python 3.x]{examples/dongles/2/decr1.py}

Et il affiche: \q{check security device connection}.
Donc oui, ceci est le message déchiffré.

Il y a d'autres messages chiffrés, avec leur clef correspondante.
Mais inutile de dire qu'il est possible de les déchiffrer sans leur clef.
Premièrement, nous voyons que le clef est en fait un octet.
C'est parce que l'instruction principale de déchiffrement (\XOR) fonctionne au niveau
de l'octet.
La clef se trouve dans le registre \ESI, mais seulement une partie de \ESI d'un octet
est utilisée.
Ainsi, une clef pourrait être plus grande que 255, mais sa valeur est toujours arrondie.

En conséquence, nous pouvons simplement essayer de brute-forcer, en essayant toutes
les clefs possible dans l'intervalle 0..255.
Nous allons aussi écarter les messages comportants des caractères non-imprimable.

\lstinputlisting[caption=Python 3.x]{examples/dongles/2/decr2.py}

Et nous obtenons:

\lstinputlisting[caption=Results]{examples/dongles/2/decr2_result.txt}

Ici il y a un peu de déchet, mais nous pouvons rapidement trouver les messages en
anglais.

À propos, puisque l'algorithme est un simple chiffrement xor, la même fonction peut
être utilisée pour chiffrer les messages.
Si besoin, nous pouvons chiffrer nos propres messages, et patcher le programme en les insérant.
}


\EN{\mysection{Task manager practical joke (Windows Vista)}
\myindex{Windows!Windows Vista}

Let's see if it's possible to hack Task Manager slightly so it would detect more \ac{CPU} cores.

\myindex{Windows!NTAPI}

Let us first think, how does the Task Manager know the number of cores?

There is the \TT{GetSystemInfo()} win32 function present in win32 userspace which can tell us this.
But it's not imported in \TT{taskmgr.exe}.

There is, however, another one in \gls{NTAPI}, \TT{NtQuerySystemInformation()}, 
which is used in \TT{taskmgr.exe} in several places.

To get the number of cores, one has to call this function with the \TT{SystemBasicInformation} constant
as a first argument (which is zero
\footnote{\href{http://msdn.microsoft.com/en-us/library/windows/desktop/ms724509(v=vs.85).aspx}{MSDN}}).

The second argument has to point to the buffer which is getting all the information.

So we have to find all calls to the \\
\TT{NtQuerySystemInformation(0, ?, ?, ?)} function.
Let's open \TT{taskmgr.exe} in IDA. 
\myindex{Windows!PDB}

What is always good about Microsoft executables is that IDA can download the corresponding \gls{PDB} 
file for this executable and show all function names.

It is visible that Task Manager is written in \Cpp and some of the function names and classes are really 
speaking for themselves.
There are classes CAdapter, CNetPage, CPerfPage, CProcInfo, CProcPage, CSvcPage, 
CTaskPage, CUserPage.

Apparently, each class corresponds to each tab in Task Manager.

Let's visit each call and add comment with the value which is passed as the first function argument.
We will write \q{not zero} at some places, because the value there was clearly not zero, 
but something really different (more about this in the second part of this chapter).

And we are looking for zero passed as argument, after all.

\begin{figure}[H]
\centering
\myincludegraphics{examples/taskmgr/IDA_xrefs.png}
\caption{IDA: cross references to NtQuerySystemInformation()}
\end{figure}

Yes, the names are really speaking for themselves.

When we closely investigate each place where\\
\TT{NtQuerySystemInformation(0, ?, ?, ?)} is called,
we quickly find what we need in the \TT{InitPerfInfo()} function:

\lstinputlisting[caption=taskmgr.exe (Windows Vista),style=customasmx86]{examples/taskmgr/taskmgr.lst}

\TT{g\_cProcessors} is a global variable, and this name has been assigned by 
IDA according to the \gls{PDB} loaded from Microsoft's symbol server.

The byte is taken from \TT{var\_C20}. 
And \TT{var\_C58} is passed to\\
\TT{NtQuerySystemInformation()} 
as a pointer to the receiving buffer.
The difference between 0xC20 and 0xC58 is 0x38 (56).

Let's take a look at format of the return structure, which we can find in MSDN:

\begin{lstlisting}[style=customc]
typedef struct _SYSTEM_BASIC_INFORMATION {
    BYTE Reserved1[24];
    PVOID Reserved2[4];
    CCHAR NumberOfProcessors;
} SYSTEM_BASIC_INFORMATION;
\end{lstlisting}

This is a x64 system, so each PVOID takes 8 bytes.

All \emph{reserved} fields in the structure take $24+4*8=56$ bytes.

Oh yes, this implies that \TT{var\_C20} is the local stack is exactly the
\TT{NumberOfProcessors} field of the \TT{SYSTEM\_BASIC\_INFORMATION} structure.

Let's check our guess.
Copy \TT{taskmgr.exe} from \TT{C:\textbackslash{}Windows\textbackslash{}System32} 
to some other folder 
(so the \emph{Windows Resource Protection} 
will not try to restore the patched \TT{taskmgr.exe}).

Let's open it in Hiew and find the place:

\begin{figure}[H]
\centering
\myincludegraphics{examples/taskmgr/hiew2.png}
\caption{Hiew: find the place to be patched}
\end{figure}

Let's replace the \TT{MOVZX} instruction with ours.
Let's pretend we've got 64 CPU cores.

Add one additional \ac{NOP} (because our instruction is shorter than the original one):

\begin{figure}[H]
\centering
\myincludegraphics{examples/taskmgr/hiew1.png}
\caption{Hiew: patch it}
\end{figure}

And it works!
Of course, the data in the graphs is not correct.

At times, Task Manager even shows an overall CPU load of more than 100\%.

\begin{figure}[H]
\centering
\myincludegraphics{examples/taskmgr/taskmgr_64cpu_crop.png}
\caption{Fooled Windows Task Manager}
\end{figure}

The biggest number Task Manager does not crash with is 64.

Apparently, Task Manager in Windows Vista was not tested on computers with a large number of cores.

So there are probably some static data structure(s) inside it limited to 64 cores.

\subsection{Using LEA to load values}
\label{TaskMgr_LEA}

Sometimes, \TT{LEA} is used in \TT{taskmgr.exe} instead of \TT{MOV} to set the first argument of \\
\TT{NtQuerySystemInformation()}:

\lstinputlisting[caption=taskmgr.exe (Windows Vista),style=customasmx86]{examples/taskmgr/taskmgr2.lst}

\myindex{x86!\Instructions!LEA}

Perhaps \ac{MSVC} did so because machine code of \INS{LEA} is shorter than \INS{MOV REG, 5} (would be 5 instead of 4).

\INS{LEA} with offset in $-128..127$ range (offset will occupy 1 byte in opcode) with 32-bit registers is even shorter (for lack of REX prefix)---3 bytes.

Another example of such thing is: \myref{using_MOV_and_pack_of_LEA_to_load_values}.
}%
\RU{\subsection{Обменять входные значения друг с другом}

Вот так:

\lstinputlisting[style=customc]{patterns/061_pointers/swap/5_RU.c}

Как видим, байты загружаются в младшие 8-битные части регистров \TT{ECX} и \TT{EBX} используя \INS{MOVZX}
(так что старшие части регистров очищаются), затем байты записываются назад в другом порядке.

\lstinputlisting[style=customasmx86,caption=Optimizing GCC 5.4]{patterns/061_pointers/swap/5_GCC_O3_x86.s}

Адреса обоих байтов берутся из аргументов и во время исполнения ф-ции находятся в регистрах \TT{EDX} и \TT{EAX}.

Так что исопльзуем указатели --- вероятно, без них нет способа решить эту задачу лучше.

}%
\FR{\subsection{Exemple \#2: SCO OpenServer}

\label{examples_SCO}
\myindex{SCO OpenServer}
Un ancien logiciel pour SCO OpenServer de 1997 développé par une société qui a disparue
depuis longtemps.

Il y a un driver de dongle special à installer dans le système, qui contient les
chaînes de texte suivantes:
\q{Copyright 1989, Rainbow Technologies, Inc., Irvine, CA}
et
\q{Sentinel Integrated Driver Ver. 3.0 }.

Après l'installation du driver dans SCO OpenServer, ces fichiers apparaissent dans
l'arborescence /dev:

\begin{lstlisting}
/dev/rbsl8
/dev/rbsl9
/dev/rbsl10
\end{lstlisting}

Le programme renvoie une erreur lorsque le dongle n'est pas connecté, mais le message
d'erreur n'est pas trouvé dans les exécutables.

\myindex{COFF}

Grâce à \ac{IDA}, il est facile de charger l'exécutable COFF utilisé dans SCO OpenServer.

Essayons de trouver la chaîne \q{rbsl} et en effet, elle se trouve dans ce morceau
de code:

\lstinputlisting[style=customasmx86]{examples/dongles/2/1.lst}

Oui, en effet, le programme doit communiquer d'une façon ou d'une autre avec le driver.

\myindex{thunk-functions}
Le seul endroit où la fonction \TT{SSQC()} est appelée est dans la \glslink{thunk
 function}{fonction thunk}:

\lstinputlisting[style=customasmx86]{examples/dongles/2/2.lst}

SSQ() peut être appelé depuis au moins 2 fonctions.

L'une d'entre elles est:

\lstinputlisting[style=customasmx86]{examples/dongles/2/check1_EN.lst}

\q{\TT{3C}} et \q{\TT{3E}} semblent familiers: il y avait un dongle Sentinel Pro de
Rainbow sans mémoire, fournissant seulement une fonction de crypto-hachage secrète.

Vous pouvez lire une courte description de la fonction de hachage dont il s'agit
ici: \myref{hash_func}.

Mais retournons au programme.

Donc le programme peut seulement tester si un dongle est connecté ou s'il est absent.

Aucune autre information ne peut être écrite dans un tel dongle, puisqu'il n'a pas
de mémoire.
Les codes sur deux caractères sont des commandes (nous pouvons voir comment les commandes
sont traitées dans la fonction \TT{SSQC()}) et toutes les autres chaînes sont hachées
dans le dongle, transformées en un nombre 16-bit.
L'algorithme était secret, donc il n'était pas possible d'écrire un driver de remplacement
ou de refaire un dongle matériel qui l'émulerait parfaitement.

Toutefois, il est toujours possible d'intercepter tous les accès au dongle et de
trouver les constantes auxquelles les résultats de la fonction de hachage sont comparées.

Mais nous devons dire qu'il est possible de construire un schéma de logiciel de protection
de copie robuste basé sur une fonction secrète de hachage cryptographique: il suffit
qu'elle chiffre/déchiffre les fichiers de données utilisés par votre logiciel.

Mais retournons au code:

Les codes 51/52/53 sont utilisés pour choisir le port imprimante LPT.
3x/4x sont utilisés pour le choix de la \q{famille} (c'est ainsi que les dongles
Sentinel Pro sont différenciés les uns des autres: plus d'un dongle peut être connecté
sur un port LPT).

La seule chaîne passée à la fonction qui ne fasse pas 2 caractères est "0123456789".

Ensuite, le résultat est comparé à l'ensemble des résultats valides.

Si il est correct, 0xC ou 0xB est écrit dans la variable globale \TT{ctl\_model}.%

Une autre chaîne de texte qui est passée est
"PRESS ANY KEY TO CONTINUE: ", mais le résultat n'est pas testé.
Difficile de dire pourquoi, probablement une erreur\footnote{C'est un sentiment
étrange de trouver un bug dans un logiciel aussi ancien.}.

Voyons où la valeur de la variable globale \TT{ctl\_model} est utilisée.

Un tel endroit est:

\lstinputlisting[style=customasmx86]{examples/dongles/2/4.lst}

Si c'est 0, un message d'erreur chiffré est passé à une routine de déchiffrement
et affiché.

\myindex{x86!\Instructions!XOR}

La routine de déchiffrement de la chaîne semble être un simple \glslink{xoring}{xor}:

\lstinputlisting[style=customasmx86]{examples/dongles/2/err_warn.lst}

C'est pourquoi nous étions incapable de trouver le message d'erreur dans les fichiers
exécutable, car ils sont chiffrés (ce qui est une pratique courante).

Un autre appel à la fonction de hachage \TT{SSQ()} lui passe la chaîne \q{offln}
et le résultat est comparé avec \TT{0xFE81} et \TT{0x12A9}.

Si ils ne correspondent pas, ça se comporte comme une sorte de fonction \TT{timer()}
(peut-être en attente qu'un dongle mal connecté soit reconnecté et re-testé?) et ensuite
déchiffre un autre message d'erreur à afficher.

\lstinputlisting[style=customasmx86]{examples/dongles/2/check2_EN.lst}

Passer outre le dongle est assez facile: il suffit de patcher tous les sauts après
les instructions \CMP pertinentes.

Une autre option est d'écrire notre propre driver SCO OpenServer, contenant une table
de questions et de réponses, toutes celles qui sont présentent dans le programme.

\subsubsection{Déchiffrer les messages d'erreur}

À propos, nous pouvons aussi essayer de déchiffrer tous les messages d'erreurs.
L'algorithme qui se trouve dans la fonction \TT{err\_warn()} est très simple, en effet:

\lstinputlisting[caption=Decryption function,style=customasmx86]{examples/dongles/2/decrypting_FR.lst}

Comme on le voit, non seulement la chaîne est transmise à la fonction de déchiffrement
mais aussi la clef:

\lstinputlisting[style=customasmx86]{examples/dongles/2/tmp1_EN.asm}

L'algorithme est un simple \glslink{xoring}{xor}: chaque octet est xoré avec la clef, mais
la clef est incrémentée de 3 après le traitement de chaque octet.

Nous pouvons écrire un petit script Python pour vérifier notre hypothèse:

\lstinputlisting[caption=Python 3.x]{examples/dongles/2/decr1.py}

Et il affiche: \q{check security device connection}.
Donc oui, ceci est le message déchiffré.

Il y a d'autres messages chiffrés, avec leur clef correspondante.
Mais inutile de dire qu'il est possible de les déchiffrer sans leur clef.
Premièrement, nous voyons que le clef est en fait un octet.
C'est parce que l'instruction principale de déchiffrement (\XOR) fonctionne au niveau
de l'octet.
La clef se trouve dans le registre \ESI, mais seulement une partie de \ESI d'un octet
est utilisée.
Ainsi, une clef pourrait être plus grande que 255, mais sa valeur est toujours arrondie.

En conséquence, nous pouvons simplement essayer de brute-forcer, en essayant toutes
les clefs possible dans l'intervalle 0..255.
Nous allons aussi écarter les messages comportants des caractères non-imprimable.

\lstinputlisting[caption=Python 3.x]{examples/dongles/2/decr2.py}

Et nous obtenons:

\lstinputlisting[caption=Results]{examples/dongles/2/decr2_result.txt}

Ici il y a un peu de déchet, mais nous pouvons rapidement trouver les messages en
anglais.

À propos, puisque l'algorithme est un simple chiffrement xor, la même fonction peut
être utilisée pour chiffrer les messages.
Si besoin, nous pouvons chiffrer nos propres messages, et patcher le programme en les insérant.
}



\EN{\mysection{Task manager practical joke (Windows Vista)}
\myindex{Windows!Windows Vista}

Let's see if it's possible to hack Task Manager slightly so it would detect more \ac{CPU} cores.

\myindex{Windows!NTAPI}

Let us first think, how does the Task Manager know the number of cores?

There is the \TT{GetSystemInfo()} win32 function present in win32 userspace which can tell us this.
But it's not imported in \TT{taskmgr.exe}.

There is, however, another one in \gls{NTAPI}, \TT{NtQuerySystemInformation()}, 
which is used in \TT{taskmgr.exe} in several places.

To get the number of cores, one has to call this function with the \TT{SystemBasicInformation} constant
as a first argument (which is zero
\footnote{\href{http://msdn.microsoft.com/en-us/library/windows/desktop/ms724509(v=vs.85).aspx}{MSDN}}).

The second argument has to point to the buffer which is getting all the information.

So we have to find all calls to the \\
\TT{NtQuerySystemInformation(0, ?, ?, ?)} function.
Let's open \TT{taskmgr.exe} in IDA. 
\myindex{Windows!PDB}

What is always good about Microsoft executables is that IDA can download the corresponding \gls{PDB} 
file for this executable and show all function names.

It is visible that Task Manager is written in \Cpp and some of the function names and classes are really 
speaking for themselves.
There are classes CAdapter, CNetPage, CPerfPage, CProcInfo, CProcPage, CSvcPage, 
CTaskPage, CUserPage.

Apparently, each class corresponds to each tab in Task Manager.

Let's visit each call and add comment with the value which is passed as the first function argument.
We will write \q{not zero} at some places, because the value there was clearly not zero, 
but something really different (more about this in the second part of this chapter).

And we are looking for zero passed as argument, after all.

\begin{figure}[H]
\centering
\myincludegraphics{examples/taskmgr/IDA_xrefs.png}
\caption{IDA: cross references to NtQuerySystemInformation()}
\end{figure}

Yes, the names are really speaking for themselves.

When we closely investigate each place where\\
\TT{NtQuerySystemInformation(0, ?, ?, ?)} is called,
we quickly find what we need in the \TT{InitPerfInfo()} function:

\lstinputlisting[caption=taskmgr.exe (Windows Vista),style=customasmx86]{examples/taskmgr/taskmgr.lst}

\TT{g\_cProcessors} is a global variable, and this name has been assigned by 
IDA according to the \gls{PDB} loaded from Microsoft's symbol server.

The byte is taken from \TT{var\_C20}. 
And \TT{var\_C58} is passed to\\
\TT{NtQuerySystemInformation()} 
as a pointer to the receiving buffer.
The difference between 0xC20 and 0xC58 is 0x38 (56).

Let's take a look at format of the return structure, which we can find in MSDN:

\begin{lstlisting}[style=customc]
typedef struct _SYSTEM_BASIC_INFORMATION {
    BYTE Reserved1[24];
    PVOID Reserved2[4];
    CCHAR NumberOfProcessors;
} SYSTEM_BASIC_INFORMATION;
\end{lstlisting}

This is a x64 system, so each PVOID takes 8 bytes.

All \emph{reserved} fields in the structure take $24+4*8=56$ bytes.

Oh yes, this implies that \TT{var\_C20} is the local stack is exactly the
\TT{NumberOfProcessors} field of the \TT{SYSTEM\_BASIC\_INFORMATION} structure.

Let's check our guess.
Copy \TT{taskmgr.exe} from \TT{C:\textbackslash{}Windows\textbackslash{}System32} 
to some other folder 
(so the \emph{Windows Resource Protection} 
will not try to restore the patched \TT{taskmgr.exe}).

Let's open it in Hiew and find the place:

\begin{figure}[H]
\centering
\myincludegraphics{examples/taskmgr/hiew2.png}
\caption{Hiew: find the place to be patched}
\end{figure}

Let's replace the \TT{MOVZX} instruction with ours.
Let's pretend we've got 64 CPU cores.

Add one additional \ac{NOP} (because our instruction is shorter than the original one):

\begin{figure}[H]
\centering
\myincludegraphics{examples/taskmgr/hiew1.png}
\caption{Hiew: patch it}
\end{figure}

And it works!
Of course, the data in the graphs is not correct.

At times, Task Manager even shows an overall CPU load of more than 100\%.

\begin{figure}[H]
\centering
\myincludegraphics{examples/taskmgr/taskmgr_64cpu_crop.png}
\caption{Fooled Windows Task Manager}
\end{figure}

The biggest number Task Manager does not crash with is 64.

Apparently, Task Manager in Windows Vista was not tested on computers with a large number of cores.

So there are probably some static data structure(s) inside it limited to 64 cores.

\subsection{Using LEA to load values}
\label{TaskMgr_LEA}

Sometimes, \TT{LEA} is used in \TT{taskmgr.exe} instead of \TT{MOV} to set the first argument of \\
\TT{NtQuerySystemInformation()}:

\lstinputlisting[caption=taskmgr.exe (Windows Vista),style=customasmx86]{examples/taskmgr/taskmgr2.lst}

\myindex{x86!\Instructions!LEA}

Perhaps \ac{MSVC} did so because machine code of \INS{LEA} is shorter than \INS{MOV REG, 5} (would be 5 instead of 4).

\INS{LEA} with offset in $-128..127$ range (offset will occupy 1 byte in opcode) with 32-bit registers is even shorter (for lack of REX prefix)---3 bytes.

Another example of such thing is: \myref{using_MOV_and_pack_of_LEA_to_load_values}.
}%
\RU{\subsection{Обменять входные значения друг с другом}

Вот так:

\lstinputlisting[style=customc]{patterns/061_pointers/swap/5_RU.c}

Как видим, байты загружаются в младшие 8-битные части регистров \TT{ECX} и \TT{EBX} используя \INS{MOVZX}
(так что старшие части регистров очищаются), затем байты записываются назад в другом порядке.

\lstinputlisting[style=customasmx86,caption=Optimizing GCC 5.4]{patterns/061_pointers/swap/5_GCC_O3_x86.s}

Адреса обоих байтов берутся из аргументов и во время исполнения ф-ции находятся в регистрах \TT{EDX} и \TT{EAX}.

Так что исопльзуем указатели --- вероятно, без них нет способа решить эту задачу лучше.

}%
\FR{\subsection{Exemple \#2: SCO OpenServer}

\label{examples_SCO}
\myindex{SCO OpenServer}
Un ancien logiciel pour SCO OpenServer de 1997 développé par une société qui a disparue
depuis longtemps.

Il y a un driver de dongle special à installer dans le système, qui contient les
chaînes de texte suivantes:
\q{Copyright 1989, Rainbow Technologies, Inc., Irvine, CA}
et
\q{Sentinel Integrated Driver Ver. 3.0 }.

Après l'installation du driver dans SCO OpenServer, ces fichiers apparaissent dans
l'arborescence /dev:

\begin{lstlisting}
/dev/rbsl8
/dev/rbsl9
/dev/rbsl10
\end{lstlisting}

Le programme renvoie une erreur lorsque le dongle n'est pas connecté, mais le message
d'erreur n'est pas trouvé dans les exécutables.

\myindex{COFF}

Grâce à \ac{IDA}, il est facile de charger l'exécutable COFF utilisé dans SCO OpenServer.

Essayons de trouver la chaîne \q{rbsl} et en effet, elle se trouve dans ce morceau
de code:

\lstinputlisting[style=customasmx86]{examples/dongles/2/1.lst}

Oui, en effet, le programme doit communiquer d'une façon ou d'une autre avec le driver.

\myindex{thunk-functions}
Le seul endroit où la fonction \TT{SSQC()} est appelée est dans la \glslink{thunk
 function}{fonction thunk}:

\lstinputlisting[style=customasmx86]{examples/dongles/2/2.lst}

SSQ() peut être appelé depuis au moins 2 fonctions.

L'une d'entre elles est:

\lstinputlisting[style=customasmx86]{examples/dongles/2/check1_EN.lst}

\q{\TT{3C}} et \q{\TT{3E}} semblent familiers: il y avait un dongle Sentinel Pro de
Rainbow sans mémoire, fournissant seulement une fonction de crypto-hachage secrète.

Vous pouvez lire une courte description de la fonction de hachage dont il s'agit
ici: \myref{hash_func}.

Mais retournons au programme.

Donc le programme peut seulement tester si un dongle est connecté ou s'il est absent.

Aucune autre information ne peut être écrite dans un tel dongle, puisqu'il n'a pas
de mémoire.
Les codes sur deux caractères sont des commandes (nous pouvons voir comment les commandes
sont traitées dans la fonction \TT{SSQC()}) et toutes les autres chaînes sont hachées
dans le dongle, transformées en un nombre 16-bit.
L'algorithme était secret, donc il n'était pas possible d'écrire un driver de remplacement
ou de refaire un dongle matériel qui l'émulerait parfaitement.

Toutefois, il est toujours possible d'intercepter tous les accès au dongle et de
trouver les constantes auxquelles les résultats de la fonction de hachage sont comparées.

Mais nous devons dire qu'il est possible de construire un schéma de logiciel de protection
de copie robuste basé sur une fonction secrète de hachage cryptographique: il suffit
qu'elle chiffre/déchiffre les fichiers de données utilisés par votre logiciel.

Mais retournons au code:

Les codes 51/52/53 sont utilisés pour choisir le port imprimante LPT.
3x/4x sont utilisés pour le choix de la \q{famille} (c'est ainsi que les dongles
Sentinel Pro sont différenciés les uns des autres: plus d'un dongle peut être connecté
sur un port LPT).

La seule chaîne passée à la fonction qui ne fasse pas 2 caractères est "0123456789".

Ensuite, le résultat est comparé à l'ensemble des résultats valides.

Si il est correct, 0xC ou 0xB est écrit dans la variable globale \TT{ctl\_model}.%

Une autre chaîne de texte qui est passée est
"PRESS ANY KEY TO CONTINUE: ", mais le résultat n'est pas testé.
Difficile de dire pourquoi, probablement une erreur\footnote{C'est un sentiment
étrange de trouver un bug dans un logiciel aussi ancien.}.

Voyons où la valeur de la variable globale \TT{ctl\_model} est utilisée.

Un tel endroit est:

\lstinputlisting[style=customasmx86]{examples/dongles/2/4.lst}

Si c'est 0, un message d'erreur chiffré est passé à une routine de déchiffrement
et affiché.

\myindex{x86!\Instructions!XOR}

La routine de déchiffrement de la chaîne semble être un simple \glslink{xoring}{xor}:

\lstinputlisting[style=customasmx86]{examples/dongles/2/err_warn.lst}

C'est pourquoi nous étions incapable de trouver le message d'erreur dans les fichiers
exécutable, car ils sont chiffrés (ce qui est une pratique courante).

Un autre appel à la fonction de hachage \TT{SSQ()} lui passe la chaîne \q{offln}
et le résultat est comparé avec \TT{0xFE81} et \TT{0x12A9}.

Si ils ne correspondent pas, ça se comporte comme une sorte de fonction \TT{timer()}
(peut-être en attente qu'un dongle mal connecté soit reconnecté et re-testé?) et ensuite
déchiffre un autre message d'erreur à afficher.

\lstinputlisting[style=customasmx86]{examples/dongles/2/check2_EN.lst}

Passer outre le dongle est assez facile: il suffit de patcher tous les sauts après
les instructions \CMP pertinentes.

Une autre option est d'écrire notre propre driver SCO OpenServer, contenant une table
de questions et de réponses, toutes celles qui sont présentent dans le programme.

\subsubsection{Déchiffrer les messages d'erreur}

À propos, nous pouvons aussi essayer de déchiffrer tous les messages d'erreurs.
L'algorithme qui se trouve dans la fonction \TT{err\_warn()} est très simple, en effet:

\lstinputlisting[caption=Decryption function,style=customasmx86]{examples/dongles/2/decrypting_FR.lst}

Comme on le voit, non seulement la chaîne est transmise à la fonction de déchiffrement
mais aussi la clef:

\lstinputlisting[style=customasmx86]{examples/dongles/2/tmp1_EN.asm}

L'algorithme est un simple \glslink{xoring}{xor}: chaque octet est xoré avec la clef, mais
la clef est incrémentée de 3 après le traitement de chaque octet.

Nous pouvons écrire un petit script Python pour vérifier notre hypothèse:

\lstinputlisting[caption=Python 3.x]{examples/dongles/2/decr1.py}

Et il affiche: \q{check security device connection}.
Donc oui, ceci est le message déchiffré.

Il y a d'autres messages chiffrés, avec leur clef correspondante.
Mais inutile de dire qu'il est possible de les déchiffrer sans leur clef.
Premièrement, nous voyons que le clef est en fait un octet.
C'est parce que l'instruction principale de déchiffrement (\XOR) fonctionne au niveau
de l'octet.
La clef se trouve dans le registre \ESI, mais seulement une partie de \ESI d'un octet
est utilisée.
Ainsi, une clef pourrait être plus grande que 255, mais sa valeur est toujours arrondie.

En conséquence, nous pouvons simplement essayer de brute-forcer, en essayant toutes
les clefs possible dans l'intervalle 0..255.
Nous allons aussi écarter les messages comportants des caractères non-imprimable.

\lstinputlisting[caption=Python 3.x]{examples/dongles/2/decr2.py}

Et nous obtenons:

\lstinputlisting[caption=Results]{examples/dongles/2/decr2_result.txt}

Ici il y a un peu de déchet, mais nous pouvons rapidement trouver les messages en
anglais.

À propos, puisque l'algorithme est un simple chiffrement xor, la même fonction peut
être utilisée pour chiffrer les messages.
Si besoin, nous pouvons chiffrer nos propres messages, et patcher le programme en les insérant.
}



\EN{\mysection{Task manager practical joke (Windows Vista)}
\myindex{Windows!Windows Vista}

Let's see if it's possible to hack Task Manager slightly so it would detect more \ac{CPU} cores.

\myindex{Windows!NTAPI}

Let us first think, how does the Task Manager know the number of cores?

There is the \TT{GetSystemInfo()} win32 function present in win32 userspace which can tell us this.
But it's not imported in \TT{taskmgr.exe}.

There is, however, another one in \gls{NTAPI}, \TT{NtQuerySystemInformation()}, 
which is used in \TT{taskmgr.exe} in several places.

To get the number of cores, one has to call this function with the \TT{SystemBasicInformation} constant
as a first argument (which is zero
\footnote{\href{http://msdn.microsoft.com/en-us/library/windows/desktop/ms724509(v=vs.85).aspx}{MSDN}}).

The second argument has to point to the buffer which is getting all the information.

So we have to find all calls to the \\
\TT{NtQuerySystemInformation(0, ?, ?, ?)} function.
Let's open \TT{taskmgr.exe} in IDA. 
\myindex{Windows!PDB}

What is always good about Microsoft executables is that IDA can download the corresponding \gls{PDB} 
file for this executable and show all function names.

It is visible that Task Manager is written in \Cpp and some of the function names and classes are really 
speaking for themselves.
There are classes CAdapter, CNetPage, CPerfPage, CProcInfo, CProcPage, CSvcPage, 
CTaskPage, CUserPage.

Apparently, each class corresponds to each tab in Task Manager.

Let's visit each call and add comment with the value which is passed as the first function argument.
We will write \q{not zero} at some places, because the value there was clearly not zero, 
but something really different (more about this in the second part of this chapter).

And we are looking for zero passed as argument, after all.

\begin{figure}[H]
\centering
\myincludegraphics{examples/taskmgr/IDA_xrefs.png}
\caption{IDA: cross references to NtQuerySystemInformation()}
\end{figure}

Yes, the names are really speaking for themselves.

When we closely investigate each place where\\
\TT{NtQuerySystemInformation(0, ?, ?, ?)} is called,
we quickly find what we need in the \TT{InitPerfInfo()} function:

\lstinputlisting[caption=taskmgr.exe (Windows Vista),style=customasmx86]{examples/taskmgr/taskmgr.lst}

\TT{g\_cProcessors} is a global variable, and this name has been assigned by 
IDA according to the \gls{PDB} loaded from Microsoft's symbol server.

The byte is taken from \TT{var\_C20}. 
And \TT{var\_C58} is passed to\\
\TT{NtQuerySystemInformation()} 
as a pointer to the receiving buffer.
The difference between 0xC20 and 0xC58 is 0x38 (56).

Let's take a look at format of the return structure, which we can find in MSDN:

\begin{lstlisting}[style=customc]
typedef struct _SYSTEM_BASIC_INFORMATION {
    BYTE Reserved1[24];
    PVOID Reserved2[4];
    CCHAR NumberOfProcessors;
} SYSTEM_BASIC_INFORMATION;
\end{lstlisting}

This is a x64 system, so each PVOID takes 8 bytes.

All \emph{reserved} fields in the structure take $24+4*8=56$ bytes.

Oh yes, this implies that \TT{var\_C20} is the local stack is exactly the
\TT{NumberOfProcessors} field of the \TT{SYSTEM\_BASIC\_INFORMATION} structure.

Let's check our guess.
Copy \TT{taskmgr.exe} from \TT{C:\textbackslash{}Windows\textbackslash{}System32} 
to some other folder 
(so the \emph{Windows Resource Protection} 
will not try to restore the patched \TT{taskmgr.exe}).

Let's open it in Hiew and find the place:

\begin{figure}[H]
\centering
\myincludegraphics{examples/taskmgr/hiew2.png}
\caption{Hiew: find the place to be patched}
\end{figure}

Let's replace the \TT{MOVZX} instruction with ours.
Let's pretend we've got 64 CPU cores.

Add one additional \ac{NOP} (because our instruction is shorter than the original one):

\begin{figure}[H]
\centering
\myincludegraphics{examples/taskmgr/hiew1.png}
\caption{Hiew: patch it}
\end{figure}

And it works!
Of course, the data in the graphs is not correct.

At times, Task Manager even shows an overall CPU load of more than 100\%.

\begin{figure}[H]
\centering
\myincludegraphics{examples/taskmgr/taskmgr_64cpu_crop.png}
\caption{Fooled Windows Task Manager}
\end{figure}

The biggest number Task Manager does not crash with is 64.

Apparently, Task Manager in Windows Vista was not tested on computers with a large number of cores.

So there are probably some static data structure(s) inside it limited to 64 cores.

\subsection{Using LEA to load values}
\label{TaskMgr_LEA}

Sometimes, \TT{LEA} is used in \TT{taskmgr.exe} instead of \TT{MOV} to set the first argument of \\
\TT{NtQuerySystemInformation()}:

\lstinputlisting[caption=taskmgr.exe (Windows Vista),style=customasmx86]{examples/taskmgr/taskmgr2.lst}

\myindex{x86!\Instructions!LEA}

Perhaps \ac{MSVC} did so because machine code of \INS{LEA} is shorter than \INS{MOV REG, 5} (would be 5 instead of 4).

\INS{LEA} with offset in $-128..127$ range (offset will occupy 1 byte in opcode) with 32-bit registers is even shorter (for lack of REX prefix)---3 bytes.

Another example of such thing is: \myref{using_MOV_and_pack_of_LEA_to_load_values}.
}
\RU{\subsection{Обменять входные значения друг с другом}

Вот так:

\lstinputlisting[style=customc]{patterns/061_pointers/swap/5_RU.c}

Как видим, байты загружаются в младшие 8-битные части регистров \TT{ECX} и \TT{EBX} используя \INS{MOVZX}
(так что старшие части регистров очищаются), затем байты записываются назад в другом порядке.

\lstinputlisting[style=customasmx86,caption=Optimizing GCC 5.4]{patterns/061_pointers/swap/5_GCC_O3_x86.s}

Адреса обоих байтов берутся из аргументов и во время исполнения ф-ции находятся в регистрах \TT{EDX} и \TT{EAX}.

Так что исопльзуем указатели --- вероятно, без них нет способа решить эту задачу лучше.

}
\DE{\mysection{x86}

\subsection{Terminologie}

Geläufig für 16-Bit (8086/80286), 32-Bit (80386, etc.), 64-Bit.

\myindex{IEEE 754}
\myindex{MS-DOS}
\begin{description}
	\item[Byte] 8-Bit.
		Die DB Assembler-Direktive wird zum Definieren von Variablen und Arrays genutzt.
		Bytes werden in dem 8-Bit-Teil der folgenden Register übergeben:
		\TT{AL/BL/CL/DL/AH/BH/CH/DH/SIL/DIL/R*L}.
	\item[Wort] 16-Bit.
		DW Assembler-Direktive \dittoclosing.
		Bytes werden in dem 16-Bit-Teil der folgenden Register übergeben:
			\TT{AX/BX/CX/DX/SI/DI/R*W}.
	\item[Doppelwort] (\q{dword}) 32-Bit.
		DD Assembler-Direktive \dittoclosing.
		Doppelwörter werden in Registern (x86) oder dem 32-Bit-Teil der Register (x64) übergeben.
		In 16-Bit-Code werden Doppelwörter in 16-Bit-Registerpaaren übergeben.
	\item[zwei Doppelwörter] (\q{qword}) 64-Bit.
		DQ Assembler-Direktive \dittoclosing.
		In 32-Bit-Umgebungen werden diese in 32-Bit-Registerpaaren übergeben.
	\item[tbyte] (10 Byte) 80-Bit oder 10 Bytes (für IEEE 754 FPU Register).
	\item[paragraph] (16 Byte) --- Bezeichnung war in MS-DOS Umgebungen gebräuchlich.
\end{description}

\myindex{Windows!API}

Datentypen der selben Breite (BYTE, WORD, DWORD) entsprechen auch denen in der Windows \ac{API}.

% TODO German Translation (DSiekmeier)
%\input{appendix/x86/registers} % subsection
%\input{appendix/x86/instructions} % subsection
\subsection{npad}
\label{sec:npad}

\RU{Это макрос в ассемблере, для выравнивания некоторой метки по некоторой границе.}
\EN{It is an assembly language macro for aligning labels on a specific boundary.}
\DE{Dies ist ein Assembler-Makro um Labels an bestimmten Grenzen auszurichten.}
\FR{C'est une macro du langage d'assemblage pour aligner les labels sur une limite
spécifique.}

\RU{Это нужно для тех \emph{нагруженных} меток, куда чаще всего передается управление, например, 
начало тела цикла. 
Для того чтобы процессор мог эффективнее вытягивать данные или код из памяти, через шину с памятью, 
кэширование, итд.}
\EN{That's often needed for the busy labels to where the control flow is often passed, e.g., loop body starts.
So the CPU can load the data or code from the memory effectively, through the memory bus, cache lines, etc.}
\DE{Dies ist oft nützlich Labels, die oft Ziel einer Kotrollstruktur sind, wie Schleifenköpfe.
Somit kann die CPU Daten oder Code sehr effizient vom Speicher durch den Bus, den Cache, usw. laden.}
\FR{C'est souvent nécessaire pour des labels très utilisés, comme par exemple le
début d'un corps de boucle. Ainsi, le CPU peut charger les données ou le code depuis
la mémoire efficacement, à travers le bus mémoire, les caches, etc.}

\RU{Взято из}\EN{Taken from}\DE{Entnommen von}\FR{Pris de} \TT{listing.inc} (MSVC):

\myindex{x86!\Instructions!NOP}
\RU{Это, кстати, любопытный пример различных вариантов \NOP{}-ов. 
Все эти инструкции не дают никакого эффекта, но отличаются разной длиной.}
\EN{By the way, it is a curious example of the different \NOP variations.
All these instructions have no effects whatsoever, but have a different size.}
\DE{Dies ist übrigens ein Beispiel für die unterschiedlichen \NOP-Variationen.
Keine dieser Anweisungen hat eine Auswirkung, aber alle haben eine unterschiedliche Größe.}
\FR{À propos, c'est un exemple curieux des différentes variations de \NOP. Toutes
ces instructions n'ont pas d'effet, mais ont une taille différente.}

\RU{Цель в том, чтобы была только одна инструкция, а не набор NOP-ов, 
считается что так лучше для производительности CPU.}
\EN{Having a single idle instruction instead of couple of NOP-s,
is accepted to be better for CPU performance.}
\DE{Eine einzelne Idle-Anweisung anstatt mehrerer NOPs hat positive Auswirkungen
auf die CPU-Performance.}
\FR{Avoir une seule instruction sans effet au lieu de plusieurs est accepté comme
étant meilleur pour la performance du CPU.}

\begin{lstlisting}[style=customasmx86]
;; LISTING.INC
;;
;; This file contains assembler macros and is included by the files created
;; with the -FA compiler switch to be assembled by MASM (Microsoft Macro
;; Assembler).
;;
;; Copyright (c) 1993-2003, Microsoft Corporation. All rights reserved.

;; non destructive nops
npad macro size
if size eq 1
  nop
else
 if size eq 2
   mov edi, edi
 else
  if size eq 3
    ; lea ecx, [ecx+00]
    DB 8DH, 49H, 00H
  else
   if size eq 4
     ; lea esp, [esp+00]
     DB 8DH, 64H, 24H, 00H
   else
    if size eq 5
      add eax, DWORD PTR 0
    else
     if size eq 6
       ; lea ebx, [ebx+00000000]
       DB 8DH, 9BH, 00H, 00H, 00H, 00H
     else
      if size eq 7
	; lea esp, [esp+00000000]
	DB 8DH, 0A4H, 24H, 00H, 00H, 00H, 00H 
      else
       if size eq 8
        ; jmp .+8; .npad 6
	DB 0EBH, 06H, 8DH, 9BH, 00H, 00H, 00H, 00H
       else
        if size eq 9
         ; jmp .+9; .npad 7
         DB 0EBH, 07H, 8DH, 0A4H, 24H, 00H, 00H, 00H, 00H
        else
         if size eq 10
          ; jmp .+A; .npad 7; .npad 1
          DB 0EBH, 08H, 8DH, 0A4H, 24H, 00H, 00H, 00H, 00H, 90H
         else
          if size eq 11
           ; jmp .+B; .npad 7; .npad 2
           DB 0EBH, 09H, 8DH, 0A4H, 24H, 00H, 00H, 00H, 00H, 8BH, 0FFH
          else
           if size eq 12
            ; jmp .+C; .npad 7; .npad 3
            DB 0EBH, 0AH, 8DH, 0A4H, 24H, 00H, 00H, 00H, 00H, 8DH, 49H, 00H
           else
            if size eq 13
             ; jmp .+D; .npad 7; .npad 4
             DB 0EBH, 0BH, 8DH, 0A4H, 24H, 00H, 00H, 00H, 00H, 8DH, 64H, 24H, 00H
            else
             if size eq 14
              ; jmp .+E; .npad 7; .npad 5
              DB 0EBH, 0CH, 8DH, 0A4H, 24H, 00H, 00H, 00H, 00H, 05H, 00H, 00H, 00H, 00H
             else
              if size eq 15
               ; jmp .+F; .npad 7; .npad 6
               DB 0EBH, 0DH, 8DH, 0A4H, 24H, 00H, 00H, 00H, 00H, 8DH, 9BH, 00H, 00H, 00H, 00H
              else
	       %out error: unsupported npad size
               .err
              endif
             endif
            endif
           endif
          endif
         endif
        endif
       endif
      endif
     endif
    endif
   endif
  endif
 endif
endif
endm
\end{lstlisting}
 % subsection
}
\FR{\subsection{Exemple \#2: SCO OpenServer}

\label{examples_SCO}
\myindex{SCO OpenServer}
Un ancien logiciel pour SCO OpenServer de 1997 développé par une société qui a disparue
depuis longtemps.

Il y a un driver de dongle special à installer dans le système, qui contient les
chaînes de texte suivantes:
\q{Copyright 1989, Rainbow Technologies, Inc., Irvine, CA}
et
\q{Sentinel Integrated Driver Ver. 3.0 }.

Après l'installation du driver dans SCO OpenServer, ces fichiers apparaissent dans
l'arborescence /dev:

\begin{lstlisting}
/dev/rbsl8
/dev/rbsl9
/dev/rbsl10
\end{lstlisting}

Le programme renvoie une erreur lorsque le dongle n'est pas connecté, mais le message
d'erreur n'est pas trouvé dans les exécutables.

\myindex{COFF}

Grâce à \ac{IDA}, il est facile de charger l'exécutable COFF utilisé dans SCO OpenServer.

Essayons de trouver la chaîne \q{rbsl} et en effet, elle se trouve dans ce morceau
de code:

\lstinputlisting[style=customasmx86]{examples/dongles/2/1.lst}

Oui, en effet, le programme doit communiquer d'une façon ou d'une autre avec le driver.

\myindex{thunk-functions}
Le seul endroit où la fonction \TT{SSQC()} est appelée est dans la \glslink{thunk
 function}{fonction thunk}:

\lstinputlisting[style=customasmx86]{examples/dongles/2/2.lst}

SSQ() peut être appelé depuis au moins 2 fonctions.

L'une d'entre elles est:

\lstinputlisting[style=customasmx86]{examples/dongles/2/check1_EN.lst}

\q{\TT{3C}} et \q{\TT{3E}} semblent familiers: il y avait un dongle Sentinel Pro de
Rainbow sans mémoire, fournissant seulement une fonction de crypto-hachage secrète.

Vous pouvez lire une courte description de la fonction de hachage dont il s'agit
ici: \myref{hash_func}.

Mais retournons au programme.

Donc le programme peut seulement tester si un dongle est connecté ou s'il est absent.

Aucune autre information ne peut être écrite dans un tel dongle, puisqu'il n'a pas
de mémoire.
Les codes sur deux caractères sont des commandes (nous pouvons voir comment les commandes
sont traitées dans la fonction \TT{SSQC()}) et toutes les autres chaînes sont hachées
dans le dongle, transformées en un nombre 16-bit.
L'algorithme était secret, donc il n'était pas possible d'écrire un driver de remplacement
ou de refaire un dongle matériel qui l'émulerait parfaitement.

Toutefois, il est toujours possible d'intercepter tous les accès au dongle et de
trouver les constantes auxquelles les résultats de la fonction de hachage sont comparées.

Mais nous devons dire qu'il est possible de construire un schéma de logiciel de protection
de copie robuste basé sur une fonction secrète de hachage cryptographique: il suffit
qu'elle chiffre/déchiffre les fichiers de données utilisés par votre logiciel.

Mais retournons au code:

Les codes 51/52/53 sont utilisés pour choisir le port imprimante LPT.
3x/4x sont utilisés pour le choix de la \q{famille} (c'est ainsi que les dongles
Sentinel Pro sont différenciés les uns des autres: plus d'un dongle peut être connecté
sur un port LPT).

La seule chaîne passée à la fonction qui ne fasse pas 2 caractères est "0123456789".

Ensuite, le résultat est comparé à l'ensemble des résultats valides.

Si il est correct, 0xC ou 0xB est écrit dans la variable globale \TT{ctl\_model}.%

Une autre chaîne de texte qui est passée est
"PRESS ANY KEY TO CONTINUE: ", mais le résultat n'est pas testé.
Difficile de dire pourquoi, probablement une erreur\footnote{C'est un sentiment
étrange de trouver un bug dans un logiciel aussi ancien.}.

Voyons où la valeur de la variable globale \TT{ctl\_model} est utilisée.

Un tel endroit est:

\lstinputlisting[style=customasmx86]{examples/dongles/2/4.lst}

Si c'est 0, un message d'erreur chiffré est passé à une routine de déchiffrement
et affiché.

\myindex{x86!\Instructions!XOR}

La routine de déchiffrement de la chaîne semble être un simple \glslink{xoring}{xor}:

\lstinputlisting[style=customasmx86]{examples/dongles/2/err_warn.lst}

C'est pourquoi nous étions incapable de trouver le message d'erreur dans les fichiers
exécutable, car ils sont chiffrés (ce qui est une pratique courante).

Un autre appel à la fonction de hachage \TT{SSQ()} lui passe la chaîne \q{offln}
et le résultat est comparé avec \TT{0xFE81} et \TT{0x12A9}.

Si ils ne correspondent pas, ça se comporte comme une sorte de fonction \TT{timer()}
(peut-être en attente qu'un dongle mal connecté soit reconnecté et re-testé?) et ensuite
déchiffre un autre message d'erreur à afficher.

\lstinputlisting[style=customasmx86]{examples/dongles/2/check2_EN.lst}

Passer outre le dongle est assez facile: il suffit de patcher tous les sauts après
les instructions \CMP pertinentes.

Une autre option est d'écrire notre propre driver SCO OpenServer, contenant une table
de questions et de réponses, toutes celles qui sont présentent dans le programme.

\subsubsection{Déchiffrer les messages d'erreur}

À propos, nous pouvons aussi essayer de déchiffrer tous les messages d'erreurs.
L'algorithme qui se trouve dans la fonction \TT{err\_warn()} est très simple, en effet:

\lstinputlisting[caption=Decryption function,style=customasmx86]{examples/dongles/2/decrypting_FR.lst}

Comme on le voit, non seulement la chaîne est transmise à la fonction de déchiffrement
mais aussi la clef:

\lstinputlisting[style=customasmx86]{examples/dongles/2/tmp1_EN.asm}

L'algorithme est un simple \glslink{xoring}{xor}: chaque octet est xoré avec la clef, mais
la clef est incrémentée de 3 après le traitement de chaque octet.

Nous pouvons écrire un petit script Python pour vérifier notre hypothèse:

\lstinputlisting[caption=Python 3.x]{examples/dongles/2/decr1.py}

Et il affiche: \q{check security device connection}.
Donc oui, ceci est le message déchiffré.

Il y a d'autres messages chiffrés, avec leur clef correspondante.
Mais inutile de dire qu'il est possible de les déchiffrer sans leur clef.
Premièrement, nous voyons que le clef est en fait un octet.
C'est parce que l'instruction principale de déchiffrement (\XOR) fonctionne au niveau
de l'octet.
La clef se trouve dans le registre \ESI, mais seulement une partie de \ESI d'un octet
est utilisée.
Ainsi, une clef pourrait être plus grande que 255, mais sa valeur est toujours arrondie.

En conséquence, nous pouvons simplement essayer de brute-forcer, en essayant toutes
les clefs possible dans l'intervalle 0..255.
Nous allons aussi écarter les messages comportants des caractères non-imprimable.

\lstinputlisting[caption=Python 3.x]{examples/dongles/2/decr2.py}

Et nous obtenons:

\lstinputlisting[caption=Results]{examples/dongles/2/decr2_result.txt}

Ici il y a un peu de déchet, mais nous pouvons rapidement trouver les messages en
anglais.

À propos, puisque l'algorithme est un simple chiffrement xor, la même fonction peut
être utilisée pour chiffrer les messages.
Si besoin, nous pouvons chiffrer nos propres messages, et patcher le programme en les insérant.
}

\EN{\mysection{Task manager practical joke (Windows Vista)}
\myindex{Windows!Windows Vista}

Let's see if it's possible to hack Task Manager slightly so it would detect more \ac{CPU} cores.

\myindex{Windows!NTAPI}

Let us first think, how does the Task Manager know the number of cores?

There is the \TT{GetSystemInfo()} win32 function present in win32 userspace which can tell us this.
But it's not imported in \TT{taskmgr.exe}.

There is, however, another one in \gls{NTAPI}, \TT{NtQuerySystemInformation()}, 
which is used in \TT{taskmgr.exe} in several places.

To get the number of cores, one has to call this function with the \TT{SystemBasicInformation} constant
as a first argument (which is zero
\footnote{\href{http://msdn.microsoft.com/en-us/library/windows/desktop/ms724509(v=vs.85).aspx}{MSDN}}).

The second argument has to point to the buffer which is getting all the information.

So we have to find all calls to the \\
\TT{NtQuerySystemInformation(0, ?, ?, ?)} function.
Let's open \TT{taskmgr.exe} in IDA. 
\myindex{Windows!PDB}

What is always good about Microsoft executables is that IDA can download the corresponding \gls{PDB} 
file for this executable and show all function names.

It is visible that Task Manager is written in \Cpp and some of the function names and classes are really 
speaking for themselves.
There are classes CAdapter, CNetPage, CPerfPage, CProcInfo, CProcPage, CSvcPage, 
CTaskPage, CUserPage.

Apparently, each class corresponds to each tab in Task Manager.

Let's visit each call and add comment with the value which is passed as the first function argument.
We will write \q{not zero} at some places, because the value there was clearly not zero, 
but something really different (more about this in the second part of this chapter).

And we are looking for zero passed as argument, after all.

\begin{figure}[H]
\centering
\myincludegraphics{examples/taskmgr/IDA_xrefs.png}
\caption{IDA: cross references to NtQuerySystemInformation()}
\end{figure}

Yes, the names are really speaking for themselves.

When we closely investigate each place where\\
\TT{NtQuerySystemInformation(0, ?, ?, ?)} is called,
we quickly find what we need in the \TT{InitPerfInfo()} function:

\lstinputlisting[caption=taskmgr.exe (Windows Vista),style=customasmx86]{examples/taskmgr/taskmgr.lst}

\TT{g\_cProcessors} is a global variable, and this name has been assigned by 
IDA according to the \gls{PDB} loaded from Microsoft's symbol server.

The byte is taken from \TT{var\_C20}. 
And \TT{var\_C58} is passed to\\
\TT{NtQuerySystemInformation()} 
as a pointer to the receiving buffer.
The difference between 0xC20 and 0xC58 is 0x38 (56).

Let's take a look at format of the return structure, which we can find in MSDN:

\begin{lstlisting}[style=customc]
typedef struct _SYSTEM_BASIC_INFORMATION {
    BYTE Reserved1[24];
    PVOID Reserved2[4];
    CCHAR NumberOfProcessors;
} SYSTEM_BASIC_INFORMATION;
\end{lstlisting}

This is a x64 system, so each PVOID takes 8 bytes.

All \emph{reserved} fields in the structure take $24+4*8=56$ bytes.

Oh yes, this implies that \TT{var\_C20} is the local stack is exactly the
\TT{NumberOfProcessors} field of the \TT{SYSTEM\_BASIC\_INFORMATION} structure.

Let's check our guess.
Copy \TT{taskmgr.exe} from \TT{C:\textbackslash{}Windows\textbackslash{}System32} 
to some other folder 
(so the \emph{Windows Resource Protection} 
will not try to restore the patched \TT{taskmgr.exe}).

Let's open it in Hiew and find the place:

\begin{figure}[H]
\centering
\myincludegraphics{examples/taskmgr/hiew2.png}
\caption{Hiew: find the place to be patched}
\end{figure}

Let's replace the \TT{MOVZX} instruction with ours.
Let's pretend we've got 64 CPU cores.

Add one additional \ac{NOP} (because our instruction is shorter than the original one):

\begin{figure}[H]
\centering
\myincludegraphics{examples/taskmgr/hiew1.png}
\caption{Hiew: patch it}
\end{figure}

And it works!
Of course, the data in the graphs is not correct.

At times, Task Manager even shows an overall CPU load of more than 100\%.

\begin{figure}[H]
\centering
\myincludegraphics{examples/taskmgr/taskmgr_64cpu_crop.png}
\caption{Fooled Windows Task Manager}
\end{figure}

The biggest number Task Manager does not crash with is 64.

Apparently, Task Manager in Windows Vista was not tested on computers with a large number of cores.

So there are probably some static data structure(s) inside it limited to 64 cores.

\subsection{Using LEA to load values}
\label{TaskMgr_LEA}

Sometimes, \TT{LEA} is used in \TT{taskmgr.exe} instead of \TT{MOV} to set the first argument of \\
\TT{NtQuerySystemInformation()}:

\lstinputlisting[caption=taskmgr.exe (Windows Vista),style=customasmx86]{examples/taskmgr/taskmgr2.lst}

\myindex{x86!\Instructions!LEA}

Perhaps \ac{MSVC} did so because machine code of \INS{LEA} is shorter than \INS{MOV REG, 5} (would be 5 instead of 4).

\INS{LEA} with offset in $-128..127$ range (offset will occupy 1 byte in opcode) with 32-bit registers is even shorter (for lack of REX prefix)---3 bytes.

Another example of such thing is: \myref{using_MOV_and_pack_of_LEA_to_load_values}.
}%
\RU{\subsection{Обменять входные значения друг с другом}

Вот так:

\lstinputlisting[style=customc]{patterns/061_pointers/swap/5_RU.c}

Как видим, байты загружаются в младшие 8-битные части регистров \TT{ECX} и \TT{EBX} используя \INS{MOVZX}
(так что старшие части регистров очищаются), затем байты записываются назад в другом порядке.

\lstinputlisting[style=customasmx86,caption=Optimizing GCC 5.4]{patterns/061_pointers/swap/5_GCC_O3_x86.s}

Адреса обоих байтов берутся из аргументов и во время исполнения ф-ции находятся в регистрах \TT{EDX} и \TT{EAX}.

Так что исопльзуем указатели --- вероятно, без них нет способа решить эту задачу лучше.

}%
\FR{\subsection{Exemple \#2: SCO OpenServer}

\label{examples_SCO}
\myindex{SCO OpenServer}
Un ancien logiciel pour SCO OpenServer de 1997 développé par une société qui a disparue
depuis longtemps.

Il y a un driver de dongle special à installer dans le système, qui contient les
chaînes de texte suivantes:
\q{Copyright 1989, Rainbow Technologies, Inc., Irvine, CA}
et
\q{Sentinel Integrated Driver Ver. 3.0 }.

Après l'installation du driver dans SCO OpenServer, ces fichiers apparaissent dans
l'arborescence /dev:

\begin{lstlisting}
/dev/rbsl8
/dev/rbsl9
/dev/rbsl10
\end{lstlisting}

Le programme renvoie une erreur lorsque le dongle n'est pas connecté, mais le message
d'erreur n'est pas trouvé dans les exécutables.

\myindex{COFF}

Grâce à \ac{IDA}, il est facile de charger l'exécutable COFF utilisé dans SCO OpenServer.

Essayons de trouver la chaîne \q{rbsl} et en effet, elle se trouve dans ce morceau
de code:

\lstinputlisting[style=customasmx86]{examples/dongles/2/1.lst}

Oui, en effet, le programme doit communiquer d'une façon ou d'une autre avec le driver.

\myindex{thunk-functions}
Le seul endroit où la fonction \TT{SSQC()} est appelée est dans la \glslink{thunk
 function}{fonction thunk}:

\lstinputlisting[style=customasmx86]{examples/dongles/2/2.lst}

SSQ() peut être appelé depuis au moins 2 fonctions.

L'une d'entre elles est:

\lstinputlisting[style=customasmx86]{examples/dongles/2/check1_EN.lst}

\q{\TT{3C}} et \q{\TT{3E}} semblent familiers: il y avait un dongle Sentinel Pro de
Rainbow sans mémoire, fournissant seulement une fonction de crypto-hachage secrète.

Vous pouvez lire une courte description de la fonction de hachage dont il s'agit
ici: \myref{hash_func}.

Mais retournons au programme.

Donc le programme peut seulement tester si un dongle est connecté ou s'il est absent.

Aucune autre information ne peut être écrite dans un tel dongle, puisqu'il n'a pas
de mémoire.
Les codes sur deux caractères sont des commandes (nous pouvons voir comment les commandes
sont traitées dans la fonction \TT{SSQC()}) et toutes les autres chaînes sont hachées
dans le dongle, transformées en un nombre 16-bit.
L'algorithme était secret, donc il n'était pas possible d'écrire un driver de remplacement
ou de refaire un dongle matériel qui l'émulerait parfaitement.

Toutefois, il est toujours possible d'intercepter tous les accès au dongle et de
trouver les constantes auxquelles les résultats de la fonction de hachage sont comparées.

Mais nous devons dire qu'il est possible de construire un schéma de logiciel de protection
de copie robuste basé sur une fonction secrète de hachage cryptographique: il suffit
qu'elle chiffre/déchiffre les fichiers de données utilisés par votre logiciel.

Mais retournons au code:

Les codes 51/52/53 sont utilisés pour choisir le port imprimante LPT.
3x/4x sont utilisés pour le choix de la \q{famille} (c'est ainsi que les dongles
Sentinel Pro sont différenciés les uns des autres: plus d'un dongle peut être connecté
sur un port LPT).

La seule chaîne passée à la fonction qui ne fasse pas 2 caractères est "0123456789".

Ensuite, le résultat est comparé à l'ensemble des résultats valides.

Si il est correct, 0xC ou 0xB est écrit dans la variable globale \TT{ctl\_model}.%

Une autre chaîne de texte qui est passée est
"PRESS ANY KEY TO CONTINUE: ", mais le résultat n'est pas testé.
Difficile de dire pourquoi, probablement une erreur\footnote{C'est un sentiment
étrange de trouver un bug dans un logiciel aussi ancien.}.

Voyons où la valeur de la variable globale \TT{ctl\_model} est utilisée.

Un tel endroit est:

\lstinputlisting[style=customasmx86]{examples/dongles/2/4.lst}

Si c'est 0, un message d'erreur chiffré est passé à une routine de déchiffrement
et affiché.

\myindex{x86!\Instructions!XOR}

La routine de déchiffrement de la chaîne semble être un simple \glslink{xoring}{xor}:

\lstinputlisting[style=customasmx86]{examples/dongles/2/err_warn.lst}

C'est pourquoi nous étions incapable de trouver le message d'erreur dans les fichiers
exécutable, car ils sont chiffrés (ce qui est une pratique courante).

Un autre appel à la fonction de hachage \TT{SSQ()} lui passe la chaîne \q{offln}
et le résultat est comparé avec \TT{0xFE81} et \TT{0x12A9}.

Si ils ne correspondent pas, ça se comporte comme une sorte de fonction \TT{timer()}
(peut-être en attente qu'un dongle mal connecté soit reconnecté et re-testé?) et ensuite
déchiffre un autre message d'erreur à afficher.

\lstinputlisting[style=customasmx86]{examples/dongles/2/check2_EN.lst}

Passer outre le dongle est assez facile: il suffit de patcher tous les sauts après
les instructions \CMP pertinentes.

Une autre option est d'écrire notre propre driver SCO OpenServer, contenant une table
de questions et de réponses, toutes celles qui sont présentent dans le programme.

\subsubsection{Déchiffrer les messages d'erreur}

À propos, nous pouvons aussi essayer de déchiffrer tous les messages d'erreurs.
L'algorithme qui se trouve dans la fonction \TT{err\_warn()} est très simple, en effet:

\lstinputlisting[caption=Decryption function,style=customasmx86]{examples/dongles/2/decrypting_FR.lst}

Comme on le voit, non seulement la chaîne est transmise à la fonction de déchiffrement
mais aussi la clef:

\lstinputlisting[style=customasmx86]{examples/dongles/2/tmp1_EN.asm}

L'algorithme est un simple \glslink{xoring}{xor}: chaque octet est xoré avec la clef, mais
la clef est incrémentée de 3 après le traitement de chaque octet.

Nous pouvons écrire un petit script Python pour vérifier notre hypothèse:

\lstinputlisting[caption=Python 3.x]{examples/dongles/2/decr1.py}

Et il affiche: \q{check security device connection}.
Donc oui, ceci est le message déchiffré.

Il y a d'autres messages chiffrés, avec leur clef correspondante.
Mais inutile de dire qu'il est possible de les déchiffrer sans leur clef.
Premièrement, nous voyons que le clef est en fait un octet.
C'est parce que l'instruction principale de déchiffrement (\XOR) fonctionne au niveau
de l'octet.
La clef se trouve dans le registre \ESI, mais seulement une partie de \ESI d'un octet
est utilisée.
Ainsi, une clef pourrait être plus grande que 255, mais sa valeur est toujours arrondie.

En conséquence, nous pouvons simplement essayer de brute-forcer, en essayant toutes
les clefs possible dans l'intervalle 0..255.
Nous allons aussi écarter les messages comportants des caractères non-imprimable.

\lstinputlisting[caption=Python 3.x]{examples/dongles/2/decr2.py}

Et nous obtenons:

\lstinputlisting[caption=Results]{examples/dongles/2/decr2_result.txt}

Ici il y a un peu de déchet, mais nous pouvons rapidement trouver les messages en
anglais.

À propos, puisque l'algorithme est un simple chiffrement xor, la même fonction peut
être utilisée pour chiffrer les messages.
Si besoin, nous pouvons chiffrer nos propres messages, et patcher le programme en les insérant.
}


\EN{\chapter{Tools}

\epigraph{Now that Dennis Yurichev has made this book free (libre), it is a
contribution to the world of free knowledge and free education.
However, for our freedom's sake, we need free (libre) reverse
engineering tools to replace the proprietary tools described in this book.}{Richard M. Stallman}

\mysection{Binary analysis}

Tools you use when you don't run any process.

\myindex{Hiew}
\myindex{GHex}
\myindex{UNIX!strings}
\myindex{UNIX!xxd}
\myindex{UNIX!od}

\begin{itemize}
\item
(Free, open-source) \emph{ent}\footnote{\url{http://www.fourmilab.ch/random/}}: entropy analyzing tool.
Read more about entropy: \myref{entropy}.

\item
\label{Hiew}
\emph{Hiew}\footnote{\href{http://www.hiew.ru/}{hiew.ru}}:
for small modifications of code in binary files.
Has assembler/disassembler.

\item (Free, open-source) \emph{GHex}\footnote{\url{https://wiki.gnome.org/Apps/Ghex}}: simple hexadecimal editor for Linux.

\item (Free, open-source) \emph{xxd} and \emph{od}: standard UNIX utilities for dumping.

\item (Free, open-source) \emph{strings}: *NIX tool for searching for ASCII strings in binary files, including executable ones.
Sysinternals has alternative\footnote{\url{https://technet.microsoft.com/en-us/sysinternals/strings}}
supporting wide char strings (UTF-16, widely used in Windows).

\item (Free, open-source) \emph{Binwalk}\footnote{\url{http://binwalk.org/}}: analyzing firmware images.

\item
\myindex{binary grep}
(Free, open-source) \emph{binary grep}:
a small utility for searching any byte sequence in a big pile of files,
including non-executable ones: \BGREPURL.
\myindex{rafind2}
There is also rafind2 in rada.re for the same purpose.
\end{itemize}

\subsection{Disassemblers}

\myindex{IDA}
\myindex{Binary Ninja}
\myindex{BinNavi}
\myindex{objdump}

\begin{itemize}
\item \emph{IDA}. An older freeware version is available for download
\footnote{\href{http://www.hex-rays.com/idapro/idadownfreeware.htm}{hex-rays.com/products/ida/support/download\_freeware.shtml}}.
\ShortHotKeyCheatsheet: \myref{sec:IDA_cheatsheet}

\item \emph{Ghidra}\footnote{\url{https://ghidra-sre.org/}} --- free alternative to IDA from \ac{NSA}.

\item \emph{Binary Ninja}\footnote{\url{http://binary.ninja/}}

\item (Free, open-source) \emph{zynamics BinNavi}\footnote{\url{https://www.zynamics.com/binnavi.html}}

\item (Free, open-source) \emph{objdump}: simple command-line utility for dumping and disassembling.

\item (Free, open-source) \emph{readelf}\footnote{\url{https://sourceware.org/binutils/docs/binutils/readelf.html}}:
dump information about ELF file.
\end{itemize}

\subsection{Decompilers}

The best known is \emph{Hex-Rays}: \url{http://hex-rays.com/products/decompiler/}.
Read more about it: \myref{hex_rays}.

There is also a free alternative from \ac{NSA}: \emph{Ghidra}\footnote{\url{https://ghidra-sre.org/}}.

\subsection{Patch comparison/diffing}

You may want to use it when you compare original version of some executable and patched one, in order to find
what has been patched and why.

\begin{itemize}
\item (Free) \emph{zynamics BinDiff}\footnote{\url{https://www.zynamics.com/software.html}}

\item (Free, open-source) \emph{Diaphora}\footnote{\url{https://github.com/joxeankoret/diaphora}}
\end{itemize}

\mysection{Live analysis}

Tools you use on a live system or during running of a process.

\subsection{Debuggers}

\myindex{\olly}
\myindex{Radare}
\myindex{GDB}
\myindex{tracer}
\myindex{LLDB}
\myindex{WinDbg}
\myindex{IDA}

\begin{itemize}
\item (Free) \emph{OllyDbg}.
Very popular user-mode win32 debugger\footnote{\href{http://www.ollydbg.de/}{ollydbg.de}}.
\ShortHotKeyCheatsheet: \myref{sec:Olly_cheatsheet}

\item (Free, open-source) \emph{GDB}.
Not quite popular debugger among reverse engineers, because it's intended mostly for programmers.
Some commands: \myref{sec:GDB_cheatsheet}.
There is a visual interface for GDB, ``GDB dashboard''\footnote{\url{https://github.com/cyrus-and/gdb-dashboard}}.

\item (Free, open-source) \emph{LLDB}\footnote{\url{http://lldb.llvm.org/}}.

\item \emph{WinDbg}\footnote{\url{https://developer.microsoft.com/en-us/windows/hardware/windows-driver-kit}}:
kernel debugger for Windows.

\item \emph{IDA} has internal debugger.

\item (Free, open-source) \emph{Radare} \ac{AKA} rada.re \ac{AKA} r2\footnote{\url{http://rada.re/r/}}.
A GUI also exists: \emph{ragui}\footnote{\url{http://radare.org/ragui/}}.

\item (Free, open-source) \emph{tracer}.
\label{tracer}
The author often uses \emph{tracer}
\footnote{\href{http://yurichev.com/tracer-en.html}{yurichev.com}}
instead of a debugger.

The author of these lines stopped using a debugger eventually, since all he needs from it is to spot function arguments while
executing, or registers state at some point.
Loading a debugger each time is too much, so a small utility called \emph{tracer} was born.
It works from command line, allows intercepting function execution,
setting breakpoints at arbitrary places, reading and changing registers state, etc.

N.B.: the \emph{tracer} isn't evolving, because it was developed as a demonstration tool for this book, not as everyday tool.
\end{itemize}

\subsection{Library calls tracing}

\emph{ltrace}\footnote{\url{http://www.ltrace.org/}}.

\subsection{System calls tracing}

\label{strace}
\myindex{strace}
\myindex{dtruss}
\subsubsection{strace / dtruss}

\myindex{syscall}
It shows which system calls (syscalls(\myref{syscalls})) are called by a process right now.

For example:

\begin{lstlisting}
# strace df -h

...

access("/etc/ld.so.nohwcap", F_OK)      = -1 ENOENT (No such file or directory)
open("/lib/i386-linux-gnu/libc.so.6", O_RDONLY|O_CLOEXEC) = 3
read(3, "\177ELF\1\1\1\0\0\0\0\0\0\0\0\0\3\0\3\0\1\0\0\0\220\232\1\0004\0\0\0"..., 512) = 512
fstat64(3, {st_mode=S_IFREG|0755, st_size=1770984, ...}) = 0
mmap2(NULL, 1780508, PROT_READ|PROT_EXEC, MAP_PRIVATE|MAP_DENYWRITE, 3, 0) = 0xb75b3000
\end{lstlisting}

\myindex{\MacOSX}
\MacOSX has dtruss for doing the same.

\myindex{Cygwin}
Cygwin also has strace, but as far as it's known, it works only for .exe-files
compiled for the cygwin environment itself.

\subsection{Network sniffing}

\emph{Sniffing} is intercepting some information you may be interested in.

(Free, open-source) \emph{Wireshark}\footnote{\url{https://www.wireshark.org/}} for network sniffing.
It has also capability for USB sniffing\footnote{\url{https://wiki.wireshark.org/CaptureSetup/USB}}.

Wireshark has a younger (or older) brother \emph{tcpdump}\footnote{\url{http://www.tcpdump.org/}}, simpler command-line tool.

\subsection{Sysinternals}

\myindex{Sysinternals}
(Free) Sysinternals (developed by Mark Russinovich)
\footnote{\url{https://technet.microsoft.com/en-us/sysinternals/bb842062}}.
At least these tools are important and worth studying: Process Explorer, Handle, VMMap, TCPView, Process Monitor.

\subsection{Valgrind}

(Free, open-source) a powerful tool for detecting memory leaks: \url{http://valgrind.org/}.
Due to its powerful \ac{JIT} mechanism, Valgrind is used as a framework for other tools.

% TODO network fuzzing

\subsection{Emulators}

\begin{itemize}
\item (Free, open-source) \emph{QEMU}\footnote{\url{http://qemu.org}}: emulator for various CPUs and architectures.

\item (Free, open-source) \emph{DosBox}\footnote{\url{https://www.dosbox.com/}}: MS-DOS emulator, mostly used for retrogaming.

\item (Free, open-source) \emph{SimH}\footnote{\url{http://simh.trailing-edge.com/}}: emulator of ancient computers, mainframes, etc.
\end{itemize}

\mysection{Other tools}

\emph{Microsoft Visual Studio Express}
\footnote{\href{http://www.microsoft.com/express/Downloads/}{visualstudio.com/en-US/products/visual-studio-express-vs}}:
Stripped-down free version of Visual Studio, convenient for simple experiments.

Some useful options: \myref{sec:MSVC_options}.

There is a website named ``Compiler Explorer'', allowing to compile small code snippets and see output
in various GCC versions and architectures
(at least x86, ARM, MIPS): \url{http://godbolt.org/}---I would have used it myself for the book if I would know about it!

\subsection{SMT solvers}

From the reverse engineer's perspective, SMT solvers are used when dealing with
amateur cryptography,
symbolic/concolic execution,
ROP chains generation.

For more information, read: \url{https://yurichev.com/writings/SAT_SMT_by_example.pdf}.

\subsection{Calculators}

Good calculator for reverse engineer's needs should support at least decimal, hexadecimal and binary bases,
as well as many important operations like XOR and shifts.

\begin{itemize}

\item IDA has built-in calculator (``?'').

\item rada.re has \emph{rax2}.

\item \ProgCalcURL

\item As a last resort, standard calculator in Windows has programmer's mode.

\end{itemize}

\mysection{Do You Think Something Is Missing Here?}

If you know a great tool not listed here, please drop a note:\\
\TT{\EMAILS}.

}
\ES{% TODO sync with English version
\chapter{Herramientas}

\mysection{Desensamblador}

\subsection{IDA}

\label{IDA}
Una versi\'on freeware anterior est\'a disponible para descargar
\footnote{\href{http://www.hex-rays.com/idapro/idadownfreeware.htm}{hex-rays.com/products/ida/support/download\_freeware.shtml}}.

\ShortHotKeyCheatsheet: \myref{sec:IDA_cheatsheet}

\mysection{Depurador}

\subsection{\olly}
\myindex{\olly}

Un depurador muy popular para win32 en modo usuario: \href{http://www.ollydbg.de/}{ollydbg.de}.

\ShortHotKeyCheatsheet: \myref{sec:Olly_cheatsheet}

\subsection{GDB}
\myindex{GDB}

No es muy popular entre reversers, aunqu es muy c\'omoda. % even though there's a direct translation for 'reverse engineer', the word commonly used is 'reverser' 

Algunos comandos: \myref{sec:GDB_cheatsheet}.

\subsection{tracer}

\myindex{tracer}
\label{tracer}
El autor suele utilizar \emph{tracer}
\footnote{\href{http://yurichev.com/tracer-en.html}{yurichev.com}} english link
(recurso en ingl\'es) en vez de un depurador.% we explicitly say that the link is for the english version

El autor de estas l\'ineas eventualmente dej\'o de utilizar un depurador, ya que lo \'unico que necesita es hallar los argumentos
de las funciones durante la ejecuci\'on, o el estado de los registros en alg\'un punto.
Cargar un depurador en cada ocasi\'on es demasiado, y fue as\'i como naci\'o una utiler\'ia llamada \emph{tracer}.
Funciona a trav\'es de la l\'inea de comandos, permitiendo interceptar la ejecuc\'on de una funci\'on,
colocar breakpoints en lugares arbitrarios, leer y cambiar el estado de los registros, etc.

Sin embargo, con fines de aprendizaje es altamente recomendable trazar el c\'odigo manualmente en un depurador,
observar c\'omo cambia el estado de los registros (e.g. los cl\'asicos SoftICE, OllyDbg, WinDbg subrayan los registros modificado),
de las banderas, de los datos, modificarlos, observar la reacci\'on, etc.

\mysection{Trazado de llamadas al sistema}

\label{strace}
\myindex{strace}
\myindex{dtruss}
\subsubsection{strace / dtruss}

\myindex{syscall}
Muestra cu\'ales llamadas al sistema (llamadas al sistema(\myref{syscalls})) son llamadas por un proceso en este momento.

Por ejemplo:

\begin{lstlisting}
# strace df -h

...

access("/etc/ld.so.nohwcap", F_OK)      = -1 ENOENT (No such file or directory)
open("/lib/i386-linux-gnu/libc.so.6", O_RDONLY|O_CLOEXEC) = 3
read(3, "\177ELF\1\1\1\0\0\0\0\0\0\0\0\0\3\0\3\0\1\0\0\0\220\232\1\0004\0\0\0"..., 512) = 512
fstat64(3, {st_mode=S_IFREG|0755, st_size=1770984, ...}) = 0
mmap2(NULL, 1780508, PROT_READ|PROT_EXEC, MAP_PRIVATE|MAP_DENYWRITE, 3, 0) = 0xb75b3000
\end{lstlisting}

\myindex{\MacOSX}
\RU{В \MacOSX для этого же имеется dtruss.}%
\EN{\MacOSX has dtruss for doing the same.}%
\ES{\MacOSX tiene dtruss para hacer lo mismo.}%
\PTBRph{}%
\DEph{}\PLph{}%
\ITph{}

\myindex{Cygwin}
Cygwin tambi\'en tiene strace, para hasta donde se sabe, s\'olo trabaja con archivos .exe
compilados para el mismo ambiente cygwin.

\mysection{Decompiladores}

Existe un solo decompilador a c\'odigo C de alta calidad y disponible p\'ublicamente: Hex-Rays:
\href{https://www.hex-rays.com/products/decompiler/}{hex-rays.com/products/decompiler/}

% TBT

% TODO Java, .NET, VB, etc

\mysection{Otras herramientas}

\begin{itemize}
\item
Microsoft Visual Studio Express\footnote{\href{http://www.microsoft.com/express/Downloads/}{visualstudio.com/en-US/products/visual-studio-express-vs}}:
La versi\'on m\'inima y gratuita de Visual Studio, conveniente para experimentos sencillos.
Algunas opciones \'utiles: \myref{sec:MSVC_options}.

\item
\label{Hiew}
Hiew\footnote{\href{http://www.hiew.ru/}{hiew.ru}}:
para realizar modificaciones de c\'odigo peque\~nas en archivos binarios.
	
\item
\myindex{binary grep}
binary grep: 
una peque\~na utiler\'ia para buscar cualquier secuencia de bytes en un mont\'on de archivos, 
incluyendo archivos no ejecutables: \BGREPURL.
% TBT
\end{itemize}

% TBT

}
\RU{% TODO sync with English version
\chapter{Инструменты}

\mysection{Дизассемблеры}

\subsection{IDA}

\label{IDA}
Старая бесплатная версия доступна для скачивания
\footnote{\href{http://www.hex-rays.com/idapro/idadownfreeware.htm}{hex-rays.com/products/ida/support/download\_freeware.shtml}}.

\ShortHotKeyCheatsheet: \myref{sec:IDA_cheatsheet}

\mysection{Отладчики}

\subsection{\olly}
\myindex{\olly}

Очень популярный отладчик пользовательской среды win32: \href{http://www.ollydbg.de/}{ollydbg.de}.

\ShortHotKeyCheatsheet: \myref{sec:Olly_cheatsheet}

\subsection{GDB}
\myindex{GDB}

Не очень популярный отладчик у реверсеров, тем не менее, крайне удобный.

Некоторые команды: \myref{sec:GDB_cheatsheet}.

\subsection{tracer}

\myindex{tracer}
\label{tracer}
Автор часто использует \emph{tracer}
\footnote{\href{http://yurichev.com/tracer-ru.html}{yurichev.com}}
вместо отладчика.

Со временем, автор этих строк отказался использовать отладчик, потому что всё что ему нужно от него это иногда подсмотреть 
какие-либо аргументы какой-либо функции во время исполнения или состояние регистров в определенном месте. 
Каждый раз загружать отладчик для этого это слишком, поэтому родилась очень простая утилита \emph{tracer}. 
Она консольная, запускается из командной строки, позволяет перехватывать исполнение функций, 
ставить точки останова на произвольные места, смотреть состояние регистров, модифицировать их, итд.

Но для учебы очень полезно трассировать код руками в отладчике, наблюдать как меняются значения регистров 
(например, как минимум классический SoftICE, OllyDbg, WinDbg подсвечивают измененные регистры), 
флагов, данные, менять их самому, смотреть реакцию, итд.

\mysection{Трассировка системных вызовов}

\label{strace}
\myindex{strace}
\myindex{dtruss}
\subsubsection{strace / dtruss}

\myindex{syscall}
Позволяет показать, какие системные вызовы (syscalls(\myref{syscalls})) прямо сейчас вызывает процесс.

Например:

\begin{lstlisting}
# strace df -h

...

access("/etc/ld.so.nohwcap", F_OK)      = -1 ENOENT (No such file or directory)
open("/lib/i386-linux-gnu/libc.so.6", O_RDONLY|O_CLOEXEC) = 3
read(3, "\177ELF\1\1\1\0\0\0\0\0\0\0\0\0\3\0\3\0\1\0\0\0\220\232\1\0004\0\0\0"..., 512) = 512
fstat64(3, {st_mode=S_IFREG|0755, st_size=1770984, ...}) = 0
mmap2(NULL, 1780508, PROT_READ|PROT_EXEC, MAP_PRIVATE|MAP_DENYWRITE, 3, 0) = 0xb75b3000
\end{lstlisting}

\myindex{\MacOSX}
В \MacOSX для этого же имеется dtruss.

\myindex{Cygwin}
В Cygwin также есть strace, впрочем, насколько известно, 
он показывает результаты только для .exe-файлов скомпилированных для среды самого cygwin.

\mysection{Декомпиляторы}

Пока существует только один публично доступный декомпилятор в Си высокого качества: Hex-Rays:
\href{https://www.hex-rays.com/products/decompiler/}{hex-rays.com/products/decompiler/}

Чиайте больше о нем: \myref{hex_rays}.

% TODO Java, .NET, VB, etc

\mysection{Прочие инструменты}

\begin{itemize}
\item
Microsoft Visual Studio Express\footnote{\href{http://www.microsoft.com/express/Downloads/}{visualstudio.com/en-US/products/visual-studio-express-vs}}:
Усеченная бесплатная версия Visual Studio, пригодная для простых экспериментов.
Некоторые полезные опции: \myref{sec:MSVC_options}.

\item
\label{Hiew}
Hiew\footnote{\href{http://www.hiew.ru/}{hiew.ru}}:
для мелкой модификации кода в исполняемых файлах.
	
\item
\myindex{binary grep}
binary grep: 
небольшая утилита для поиска констант (либо просто последовательности байт)
в большом количестве файлов, включая неисполняемые: \BGREPURL.
\myindex{rafind2}
В rada.re имеется также rafind2 для тех же целей.

\end{itemize}

\subsection{Калькуляторы}

Хороший калькулятор для нужд реверс-инженера должен поддерживать как минимум десятичную, шестнадцатеричную и двоичную системы
счисления, а также многие важные операции как ``исключающее ИЛИ'' и сдвиги.

\begin{itemize}

\item В IDA есть встроенный калькулятор (``?'').

\item В rada.re есть \emph{rax2}.

\item \ProgCalcURL

\item Стандартный калькулятор в Windows имеет режим \emph{программистского калькулятора}.

\end{itemize}

\mysection{Чего-то здесь недостает?}

Если вы знаете о хорошем инструменте, которого не хватает здесь в этом списке, пожалуйста сообщите мне об этом:\\
\TT{\EMAILS}.

}
\IT{\chapter{Strumenti}

\epigraph{Ora che Dennis Yurichev ha reso questo libro free (libre), è un
contributo a tutto il mondo della libera informazione ed educazione.
Comunque, per il bene della nostra libertà, abbiamo bisogno di strumenti free (libre) per il reverse
engineering in modo da rimpiazzare quelli proprietari descritti in questo libro.}{Richard M. Stallman}

\mysection{Analisi di Binari}

Strumenti da utilizzare senza eseguire nessun processo:

\myindex{Hiew}
\myindex{UNIX!strings}
\myindex{UNIX!xxd}
\myindex{UNIX!od}

\begin{itemize}
\item
(Free, open-source) \emph{ent}\footnote{\url{http://www.fourmilab.ch/random/}}: strumento per analizzare l'entropia.
Leggi di più riguardo l'entropia: \myref{entropy}.

\item
\label{Hiew}
\emph{Hiew}\footnote{\href{http://www.hiew.ru/}{hiew.ru}}:
per piccole modifiche del codice dei file binari.

\item (Free, open-source) \emph{xxd} e \emph{od}: standard UNIX utility per effettuare il dump nel formato desiderato.

\item (Free, open-source) \emph{strings}: strumento *NIX per cercare stringhe ASCII all'interno di file binari, eseguibili compresi.
Sysinternals ha un'alternativa\footnote{\url{https://technet.microsoft.com/en-us/sysinternals/strings}}
che supporta la stringhe con caratteri di tipo "wide" (UTF-16, ampiamente utilizzati in Windows).

\item (Free, open-source) \emph{Binwalk}\footnote{\url{http://binwalk.org/}}: analisi di immagini firmware.

\item
\myindex{binary grep}
(Free, open-source) \emph{binary grep}:
piccola utility per cercare una sequenza di byte in molti file,
incluso file non eseguibili: \BGREPURL.
\myindex{rafind2}
C'è anche rafind2 in rada.re allo stesso scopo.
\end{itemize}

\subsection{Disassemblers}

\myindex{IDA}
\myindex{Binary Ninja}
\myindex{BinNavi}
\myindex{objdump}

\begin{itemize}
\item \emph{IDA}. Una versione più vecchia è liberamente disponibile per lo scaricamento
\footnote{\href{http://www.hex-rays.com/idapro/idadownfreeware.htm}{hex-rays.com/products/ida/support/download\_freeware.shtml}}.
\ShortHotKeyCheatsheet: \myref{sec:IDA_cheatsheet}

% TBT \item \emph{Ghidra}\footnote{\url{https://ghidra-sre.org/}} --- free alternative to IDA from \ac{NSA}.

\item \emph{Binary Ninja}\footnote{\url{http://binary.ninja/}}

\item (Free, open-source) \emph{zynamics BinNavi}\footnote{\url{https://www.zynamics.com/binnavi.html}}

\item (Free, open-source) \emph{objdump}: semplice utility command-line per effettuare il dump e il disassembling.

\item (Free, open-ssource) \emph{readelf}\footnote{\url{https://sourceware.org/binutils/docs/binutils/readelf.html}}:
dump delle informazioni dei file ELF.
\end{itemize}

\subsection{Decompilers}

C'è solo un decompiler conosciuto, pubblicamente disponibile e di elevata qualità per decompilare in C: \emph{Hex-Rays}:
\href{https://www.hex-rays.com/products/decompiler/}{hex-rays.com/products/decompiler/}

Più informazioni su: \myref{hex_rays}.
% TBT: to be synced with EN version

\subsection{Comparazione Patch/diffing}

Potresti voler utilizzare questi strumenti quando devi comparare la versione originale di un eseguibile con quella patchata,
in modo da trovare cos'è stato patchato e perchè.

\begin{itemize}
\item (Free) \emph{zynamics BinDiff}\footnote{\url{https://www.zynamics.com/software.html}}

\item (Free, open-source) \emph{Diaphora}\footnote{\url{https://github.com/joxeankoret/diaphora}}
\end{itemize}

\mysection{Analisi live}

Strumenti da utilizzare per effettuare un'analisi live del sistema o di un processo in esecuzione.

\subsection{Debuggers}

\myindex{\olly}
\myindex{Radare}
\myindex{GDB}
\myindex{tracer}
\myindex{LLDB}
\myindex{WinDbg}

\begin{itemize}
\item (Free) \emph{OllyDbg}.
Popolare win32 debugger\footnote{\href{http://www.ollydbg.de/}{ollydbg.de}}.
\ShortHotKeyCheatsheet: \myref{sec:Olly_cheatsheet}

\item (Free, open-source) \emph{GDB}.
Strumento non molto popolare tra reverse engineers perchè è per lo più inteso per programmatori.
Alcuni comandi: \myref{sec:GDB_cheatsheet}.
C'è anche un'interfaccia per GDB, ``GDB dashboard''\footnote{\url{https://github.com/cyrus-and/gdb-dashboard}}.

\item (Free, open-source) \emph{LLDB}\footnote{\url{http://lldb.llvm.org/}}.

\item \emph{WinDbg}\footnote{\url{https://developer.microsoft.com/en-us/windows/hardware/windows-driver-kit}}:
kernel debugger per Windows.

\item (Free, open-source) \emph{Radare} \ac{AKA} rada.re \ac{AKA} r2\footnote{\url{http://rada.re/r/}}.
Esiste anche una GUI: \emph{ragui}\footnote{\url{http://radare.org/ragui/}}.

\item (Free, open-source) \emph{tracer}.
\label{tracer}
L'autore usa spesso \emph{tracer}
\footnote{\href{http://yurichev.com/tracer-en.html}{yurichev.com}}
invece di un debugger.

L'autore di queste righe ha smesso di utilizzare un debugger dato che l'unica cosa di cui aveva bisogno era di trovare gli
argomenti delle funzioni durante l'esecuzione o lo stato dei registri ad un determinato punto.
Caricare un debugger ogni volta risultava essere non ottimale, perciò è nacque una nuova utility chiamata \emph{tracer}.
Funziona dalla linea di comando e permette di intercettare l'esecuzione di funzioni,
impostare breakpoint in posizioni arbitrarie, leggere e modificare lo stato dei registri, ecc.

N.B.: \emph{tracer} non sta evolvendo perchè è nato principalmente come strumento di dimostrazione per questo libro, non come strumento di ogni giorno.
\end{itemize}

\subsection{Tracciare chiamate alle librerie}

\emph{ltrace}\footnote{\url{http://www.ltrace.org/}}.

\subsection{Tracciare chiamate di sistema}

\label{strace}
\myindex{strace}
\myindex{dtruss}
\subsubsection{strace / dtruss}

\myindex{syscall}
Mostra quali chiamate di sistema sono chiamate da un processo. (syscalls(\myref{syscalls}))

Per esempio:

\begin{lstlisting}
# strace df -h

...

access("/etc/ld.so.nohwcap", F_OK)      = -1 ENOENT (No such file or directory)
open("/lib/i386-linux-gnu/libc.so.6", O_RDONLY|O_CLOEXEC) = 3
read(3, "\177ELF\1\1\1\0\0\0\0\0\0\0\0\0\3\0\3\0\1\0\0\0\220\232\1\0004\0\0\0"..., 512) = 512
fstat64(3, {st_mode=S_IFREG|0755, st_size=1770984, ...}) = 0
mmap2(NULL, 1780508, PROT_READ|PROT_EXEC, MAP_PRIVATE|MAP_DENYWRITE, 3, 0) = 0xb75b3000
\end{lstlisting}

\myindex{\MacOSX}
\MacOSX ha dtruss per lo stesso compito.

\myindex{Cygwin}
Cygwin ha strace ma, per quanto ne so, funziona solo per file .exe compilati all'interno dell'ambiente
cygwin.

\subsection{Network sniffing}

\emph{Sniffing} significa intercettare informazioni di cui si potrebbe essere interessati.

(Free, open-source) \emph{Wireshark}\footnote{\url{https://www.wireshark.org/}} per lo sniffing di rete.
Puà sniffare anche USB\footnote{\url{https://wiki.wireshark.org/CaptureSetup/USB}}.

Wireshark ha un fratello chiamato \emph{tcpdump}\footnote{\url{http://www.tcpdump.org/}}, un semplice strumento a linea di comando.

\subsection{Sysinternals}

\myindex{Sysinternals}
(Free) Sysinternals (developed by Mark Russinovich)
\footnote{\url{https://technet.microsoft.com/en-us/sysinternals/bb842062}}.
Questi strumenti sono importanti e vale la pena studiarli: Process Explorer, Handle, VMMap, TCPView, Process Monitor.

\subsection{Valgrind}

(Free, open-source) strumento per rilevare memory leak: \url{http://valgrind.org/}.
A causa del suo potente meccanismo \ac{JIT}, Valgrind è utilizzato come framework per altri strumenti.

% TODO network fuzzing

\subsection{Emulatori}

\begin{itemize}
\item (Free, open-source) \emph{QEMU}\footnote{\url{http://qemu.org}}: emulatore per differenti tipi di CPU e architetture.

\item (Free, open-source) \emph{DosBox}\footnote{\url{https://www.dosbox.com/}}: emulatore MS-DOS, usato soprattutto per il retro-gaming.

\item (Free, open-source) \emph{SimH}\footnote{\url{http://simh.trailing-edge.com/}}: emulatore di antichi computer, mainframe, ecc.
\end{itemize}

\mysection{Altri strumenti}

\emph{Microsoft Visual Studio Express}
\footnote{\href{http://www.microsoft.com/express/Downloads/}{visualstudio.com/en-US/products/visual-studio-express-vs}}:
Versione gratuita di Visual Studio, conveniente per semplici esperimenti.

Alcune opzioni utili: \myref{sec:MSVC_options}.

C'è un sito chiamato ``Compiler Explorer'', che permette di compilare piccoli pezzi di codice e vederne l'output
in varie versioni di GCC ed architetture differenti
(almeno x86, ARM, MIPS): \url{http://godbolt.org/}---lo avrei utilizzato io stesso per il libro se lo avessi saputo!

% TBT
%\subsection{SMT solvers}

%From the reverse engineer's perspective, SMT solvers are used when dealing with
%amateur cryptography,
%symbolic/concolic execution,
%ROP chains generation.

%For more information, read: \url{https://yurichev.com/writings/SAT_SMT_by_example.pdf}.

\subsection{Calcolatrici}

Una buona calcolatrice, per le esigenze del reverse engineer, dovrebbe supportare almeno le basi decimale, esadecimale e binaria,
ed operazioni importanti come XOR e gli shift.

\begin{itemize}

\item IDA ha una calcolatrice integrata (``?'').

\item rada.re ha \emph{rax2}.

\item \ProgCalcURL

\item Come ultima opzione, la calcolatrice standard di Windows ha una modalità programmatore.

\end{itemize}

\mysection{Manca qualcosa qui?}

Se conosci qualche strumento non elencato qui, per favore segnalamelo tramite e-mail al seguente indirizzo:\\
\TT{\EMAILS}.
}
\FR{\chapter{Outils}

\epigraph{Maintenant que Dennis Yurichev a réalisé ce livre gratuit, il s'agit d'une contribution au monde de la connaissance et de l'éducation gratuite.
Cependant, pour l'amour de la liberté, nous avons besoin d'outils de rétro-ingénierie (libres) afin de remplacer les outils propriétaires mentionnés dans ce livre.}{Richard M. Stallman}

\mysection{Analyse statique}

Outils à utiliser lorsqu'aucun processus n'est en cours d'exécution.

\myindex{Hiew}
\myindex{UNIX!strings}
\myindex{UNIX!xxd}
\myindex{UNIX!od}

\begin{itemize}
\item
(Gratuit, open-source) \emph{ent}\footnote{\url{http://www.fourmilab.ch/random/}}: outil d'analyse d'entropie.
En savoir plus sur l'entropie : \myref{entropy}.

\item
\label{Hiew}
\emph{Hiew}\footnote{\href{http://www.hiew.ru/}{hiew.ru}}:
pour de petites modifications de code dans les fichiers binaires.
Inclut un assembleur/désassembleur.

\item {Libre, open-source} \emph{GHex}\footnote{\url{https://wiki.gnome.org/Apps/Ghex}}: éditeur
hexadécimal simple pour Linux.

\item (Libre, open-source) \emph{xxd} et \emph{od}: utilitaires standards UNIX pour réaliser un dump.

\item (Libre, open-source) \emph{strings}: outil *NIX pour rechercher des chaînes ASCII dans des fichiers binaires, fichiers exécutables inclus.
Sysinternals ont une alternative \footnote{\url{https://technet.microsoft.com/en-us/sysinternals/strings}}
qui supporte les larges chaînes de caractères (UTF-16, très utilisé dans Windows).

\item (Libre, open-source) \emph{Binwalk}\footnote{\url{http://binwalk.org/}}: analyser les images firmware.

\item
\myindex{binary grep}
(Libre, open-source) \emph{binary grep}:
un petit utilitaire pour rechercher une séquence d'octets dans un paquet de fichiers,
incluant ceux non exécutables : \BGREPURL.
\myindex{rafind2}
Il y a aussi rafind2 dans rada.re pour le même usage.
\end{itemize}

\subsection{Désassembleurs}

\myindex{IDA}
\myindex{Binary Ninja}
\myindex{BinNavi}
\myindex{objdump}

\begin{itemize}
\item \emph{IDA}. Une ancienne version Freeware est disponible via téléchargement
\footnote{\href{http://www.hex-rays.com/idapro/idadownfreeware.htm}{hex-rays.com/products/ida/support/download\_Freeware.shtml}}.
\ShortHotKeyCheatsheet: \myref{sec:IDA_cheatsheet}

\item (Gratuit, open-source) \emph{Ghidra}\footnote{\url{https://ghidra-sre.org/}} --- une alternative
	libre et open-source de IDA développée par la \ac{NSA}.

\item \emph{Binary Ninja}\footnote{\url{http://binary.ninja/}}

\item (Gratuit, open-source) \emph{zynamics BinNavi}\footnote{\url{https://www.zynamics.com/binnavi.html}}

\item (Gratuit, open-source) \emph{objdump}: simple utilitaire en ligne de commandes pour désassembler et réaliser des dumps.

\item (Gratuit, open-source) \emph{readelf}\footnote{\url{https://sourceware.org/binutils/docs/binutils/readelf.html}}:
réaliser des dumps d'informations sur des fichiers ELF.
\end{itemize}

\subsection{Décompilateurs}

Il n'existe qu'un seul décompilateur connu en C, d'excellente qualité et disponible au public \emph{Hex-Rays}:\\
\href{https://www.hex-rays.com/products/decompiler/}{hex-rays.com/products/decompiler/}

Pour en savoir plus: \myref{hex_rays}.

Il y a une alternative libre développée par la \ac{NSA}: \emph{Ghidra}\footnote{\url{https://ghidra-sre.org/}}.

\subsection{Comparaison de versions}

Vous pouvez éventuellement les utiliser lorsque vous comparez la version originale d'un exécutable et une version remaniée, pour déterminer ce qui a été corrigé et en déterminer la raison.

\begin{itemize}
\item (Gratuit) \emph{zynamics BinDiff}\footnote{\url{https://www.zynamics.com/software.html}}

\item (Gratuit, open-source) \emph{Diaphora}\footnote{\url{https://github.com/joxeankoret/diaphora}}
\end{itemize}

\mysection{Analyse dynamique}

Outils à utiliser lorsque que le système est en cours d'exploitation ou lorsqu'un processus est en cours d'exécution.

\subsection{Débogueurs}

\myindex{\olly}
\myindex{Radare}
\myindex{GDB}
\myindex{tracer}
\myindex{LLDB}
\myindex{WinDbg}

\begin{itemize}
\item (Gratuit) \emph{OllyDbg}.
Débogueur Win32 très populaire \footnote{\href{http://www.ollydbg.de/}{ollydbg.de}}.
\ShortHotKeyCheatsheet: \myref{sec:Olly_cheatsheet}

\item (Gratuit, open-source) \emph{GDB}.
Débogueur peu populaire parmi les ingénieurs en rétro-ingénierie, car il est principalement destiné aux programmeurs.
Quelques commandes : \myref{sec:GDB_cheatsheet}.
Il y a une interface graphique pour GDB, ``GDB dashboard''\footnote{\url{https://github.com/cyrus-and/gdb-dashboard}}.

\item (Gratuit, open-source) \emph{LLDB}\footnote{\url{http://lldb.llvm.org/}}.

\item \emph{WinDbg}\footnote{\url{https://developer.microsoft.com/en-us/windows/hardware/windows-driver-kit}}:
débogueur pour le noyau Windows.

\item (Gratuit, open-source) \emph{Radare} \ac{AKA} rada.re \ac{AKA} r2\footnote{\url{http://rada.re/r/}}.
Une interface graphique existe aussi : \emph{ragui}\footnote{\url{http://radare.org/ragui/}}.

\item (Gratuit, open-source) \emph{tracer}.
\label{tracer}
L'auteur utilise souvent \emph{tracer}
\footnote{\href{http://yurichev.com/tracer-en.html}{yurichev.com}}
au lieu d'un débogueur.

L'auteur de ces lignes a finalement arrêté d'utiliser un débogueur, depuis que tout ce dont il a besoin est de repérer les arguments d'une fonction lorsque cette dernière est exécutée, ou l'état des registres à un instant donné.
Le temps de chargement d'un débogueur étant trop long, un petit utilitaire sous le nom de \emph{tracer} a été conçu.
Il fonctionne depuis la ligne de commandes, permettant d'intercepter l'exécution d'une fonction,
en plaçant des breakpoints à des endroits définis, en lisant et en changeant l'état des registres, etc...

N.B.: \emph{tracer} n'évolue pas, parce qu'il a été développé en tant qu'outil de démonstration pour ce livre, et non pas comme un outil dont on se servirait au quotidien.
\end{itemize}

\subsection{Tracer les appels de librairies}

\emph{ltrace}\footnote{\url{http://www.ltrace.org/}}.

\subsection{Tracer les appels système}

\label{strace}
\myindex{strace}
\myindex{dtruss}
\subsubsection{strace / dtruss}

\myindex{syscall}
Montre les appels système (syscalls(\myref{syscalls})) effectués dans l'immédiat.

Par exemple:

\begin{lstlisting}
# strace df -h

...

access("/etc/ld.so.nohwcap", F_OK)      = -1 ENOENT (No such file or directory)
open("/lib/i386-linux-gnu/libc.so.6", O_RDONLY|O_CLOEXEC) = 3
read(3, "\177ELF\1\1\1\0\0\0\0\0\0\0\0\0\3\0\3\0\1\0\0\0\220\232\1\0004\0\0\0"..., 512) = 512
fstat64(3, {st_mode=S_IFREG|0755, st_size=1770984, ...}) = 0
mmap2(NULL, 1780508, PROT_READ|PROT_EXEC, MAP_PRIVATE|MAP_DENYWRITE, 3, 0) = 0xb75b3000
\end{lstlisting}

\myindex{\MacOSX}
\MacOSX a dtruss pour faire la même chose.

\myindex{Cygwin}
Cygwin a également strace, mais de ce que je sais, cela ne fonctionne que pour les fichiers .exe
compilés pour l'environnement Cygwin lui-même.

\subsection{Sniffer le réseau}

\emph{Sniffer} signifie intercepter des informations qui peuvent vous intéresser.

(Gratuit, open-source) \emph{Wireshark}\footnote{\url{https://www.wireshark.org/}} pour sniffer le réseau.
Peut également sniffer les protocoles USB \footnote{\url{https://wiki.wireshark.org/CaptureSetup/USB}}.

Wireshark a un petit (ou vieux) frère \emph{tcpdump}\footnote{\url{http://www.tcpdump.org/}}, outil simple en ligne de commandes.

\subsection{Sysinternals}

\myindex{Sysinternals}
(Gratuit) Sysinternals (développé par Mark Russinovich)
\footnote{\url{https://technet.microsoft.com/en-us/sysinternals/bb842062}}.
Ces outils sont importants et valent la peine d'être étudiés : Process Explorer, Handle, VMMap, TCPView, Process Monitor.

\subsection{Valgrind}

(Gratuit, open-source) un puissant outil pour détecter les fuites mémoire : \url{http://valgrind.org/}.
Grâce à ses puissants mécanismes \ac{JIT} ("Just In Time"), Valgrind est utilisé comme un framework pour d'autres outils.

% TODO network fuzzing

\subsection{Emulateurs}

\begin{itemize}
\item (Gratuit, open-source) \emph{QEMU}\footnote{\url{http://qemu.org}}: émulateur pour différents CPUs et architectures.

\item (Gratuit, open-source) \emph{DosBox}\footnote{\url{https://www.dosbox.com/}}: émulateur MS-DOS, principalement utilisé pour le rétro-gaming.

\item (Gratuit, open-source) \emph{SimH}\footnote{\url{http://simh.trailing-edge.com/}}: émulateur d'anciens ordinateurs, unités centrales, etc...
\end{itemize}

\mysection{Autres outils}

\emph{Microsoft Visual Studio Express}
\footnote{\href{http://www.microsoft.com/express/Downloads/}{visualstudio.com/en-US/products/visual-studio-express-vs}}:
Version gratuite simplifiée de Visual Studio, pratique pour des études de cas simples.

Quelques options utiles : \myref{sec:MSVC_options}.

Il y a un site web appelé ``Compiler Explorer'', permettant de compiler des petits
morceaux de code et de voir le résultat avec des versions variées de GCC et d'architectures
(au moins x86, ARM, MIPS): \url{http://godbolt.org/}---Je l'aurais utilisé
pour le livre si je l'avais connu!

\subsection{Solveurs SMT}

Du point de vue de rétro-ingénieur, les solveurs SAT sont utilisés lorsque l'on
fait face à de la cryptogrphie amateur, de l'exécution symoblique/concolique,
de la génération de chaîne ROP.

Pour plus d'information, lire: \url{https://yurichev.com/writings/SAT_SMT_by_example.pdf}.

\subsection{Calculatrices}

Une bonne calculatrice pour les besoins des rétro-ingénieurs doit au moins supporter
les bases décimale, hexadécimale et binaire, ainsi que plusieurs opérations importantes
comme XOR et les décalages.

\begin{itemize}

\item IDA possède une calculatrice intégrée (``?'').

\item rada.re a \emph{rax2}.

\item \ProgCalcURL

\item En dernier recours, la calculatrice standard de Windows dispose d'un mode
programmeur.

\end{itemize}

\mysection{Un outil manquant ?}

Si vous connaissez un bon outil non listé précédemment, n'hésitez pas à m'en faire la remarque : \\
\TT{\EMAILS}.

}
\DE{\chapter{Tools}

\epigraph{Now that Dennis Yurichev has made this book free (libre), it is a
contribution to the world of free knowledge and free education.
However, for our freedom's sake, we need free (libre) reverse
engineering tools to replace the proprietary tools described in this book.}{Richard M. Stallman}

\mysection{Binäre Analyse}

Tools die genutzt werden können, wenn kein Prozess gestartet wurde.

\myindex{Hiew}
\myindex{GHex}
\myindex{UNIX!strings}
\myindex{UNIX!xxd}
\myindex{UNIX!od}

\begin{itemize}
\item
(kostenlos, Open Source) \emph{ent}\footnote{\url{http://www.fourmilab.ch/random/}}: Entropie-Analyse-Tool.
Mehr über Entropie: \myref{entropy}.

\item
\label{Hiew}
\emph{Hiew}\footnote{\href{http://www.hiew.ru/}{hiew.ru}}:
für kleinere Modifikationen von Code in Binärdateien.
Beinhaltet einen Assembler / Dissassembler.

\item (kostenlos, Open Source) \emph{GHex}\footnote{\url{https://wiki.gnome.org/Apps/Ghex}}: Einfacher Hex-Editor für Linux.

\item (kostenlos, Open Source) \emph{xxd} und \emph{od}: Standard UNIX-Tools für Dumping.

\item (kostenlos, Open Source) \emph{strings}: *NIX-Tool für das Suchen von ASCII-Zeichenketten in Binärdateien,
inklusive ausführbaren Dateien.
Sysinternals hat eine Alternative\footnote{\url{https://technet.microsoft.com/en-us/sysinternals/strings}}
die Wide-Charakter-Zeichenketten unterstützt (UTF-16, unter Windows weit verbreitet).

\item (kostenlos, Open Source) \emph{Binwalk}\footnote{\url{http://binwalk.org/}}: Analyse von Firmware-Images.

\item
\myindex{binary grep}
(kostenlos, Open Source) \emph{binary grep}:
ein kleines Tool um jede Byte-Sequenz in einer großen Anzahl von Dateien zu suchen,
inklusive nicht-ausführbaren Dateien: \BGREPURL.
\myindex{rafind2}
Es gibt auch rafind2 in rada.re mit dem gleichen Verwendungszweck.
\end{itemize}

\subsection{Disassembler}

\myindex{IDA}
\myindex{Binary Ninja}
\myindex{BinNavi}
\myindex{objdump}

\begin{itemize}
\item \emph{IDA}. Eine ältere Freeware-Version ist online erhältlich
\footnote{\href{http://www.hex-rays.com/idapro/idadownfreeware.htm}{hex-rays.com/products/ida/support/download\_freeware.shtml}}.
\ShortHotKeyCheatsheet: \myref{sec:IDA_cheatsheet}

% TBT \item \emph{Ghidra}\footnote{\url{https://ghidra-sre.org/}} --- free alternative to IDA from \ac{NSA}.

\item \emph{Binary Ninja}\footnote{\url{http://binary.ninja/}}

\item (kostenlos, Open Source) \emph{zynamics BinNavi}\footnote{\url{https://www.zynamics.com/binnavi.html}}

\item (kostenlos, Open Source) \emph{objdump}: Einfaches Kommandozeilen-Tool für Dumping und zum disassemblieren.

\item (kostenlos, Open Source) \emph{readelf}\footnote{\url{https://sourceware.org/binutils/docs/binutils/readelf.html}}:
Gibt Informationen über ELF-Dateien aus.
\end{itemize}

\subsection{Decompiler}

Es gibt lediglich einen bekannten, öffentlich verfügbaren Decompiler für C-Code in
hoher Qualität: \emph{Hex-Rays}:\\
\href{https://www.hex-rays.com/products/decompiler/}{hex-rays.com/products/decompiler/}

Mehr darüber: \myref{hex_rays}.
% TBT: to be synced with EN version

\subsection{Vergleichen von Patches}

Diese Tools können genutzt werden wenn die Original-Version einer ausführbaren Datei
mit einer veränderten Version verglichen werden soll, oder um herauszufinden was
verändert wurde und warum.

\begin{itemize}
\item (kostenlos) \emph{zynamics BinDiff}\footnote{\url{https://www.zynamics.com/software.html}}

\item (kostenlos, Open Source) \emph{Diaphora}\footnote{\url{https://github.com/joxeankoret/diaphora}}
\end{itemize}

\mysection{Live-Analyse}

Tools die im Live-System oder auf laufende Prozesse angewandt werden können.

\subsection{Debugger}

\myindex{\olly}
\myindex{Radare}
\myindex{GDB}
\myindex{tracer}
\myindex{LLDB}
\myindex{WinDbg}
\myindex{IDA}

\begin{itemize}
\item (kostenlos) \emph{OllyDbg}.
Sehr populärer user-mode Debugger für die Win32-Architektur\footnote{\href{http://www.ollydbg.de/}{ollydbg.de}}.
\ShortHotKeyCheatsheet: \myref{sec:Olly_cheatsheet}

\item (kostenlos, Open Source) \emph{GDB}.
Nicht sehr populärer Debugger unter Reverse Engineers, da eher für Programmierer gemacht.
Einige Kommandos: \myref{sec:GDB_cheatsheet}.
Es gibt eine grafische Oberfläche für GDB, ``GDB dashboard''\footnote{\url{https://github.com/cyrus-and/gdb-dashboard}}.

\item (kostenlos, Open Source) \emph{LLDB}\footnote{\url{http://lldb.llvm.org/}}.

\item \emph{WinDbg}\footnote{\url{https://developer.microsoft.com/en-us/windows/hardware/windows-driver-kit}}:
Kernel-Debugger für Windows.

\item \emph{IDA} hat einen internen Debugger.

\item (kostenlos, Open Source) \emph{Radare} \ac{AKA} rada.re \ac{AKA} r2\footnote{\url{http://rada.re/r/}}.
Es existiert auch eine GUI: \emph{ragui}\footnote{\url{http://radare.org/ragui/}}.

\item (kostenlos, Open Source) \emph{tracer}.
\label{tracer}
Der Autor benutzt oft \emph{tracer}
\footnote{\href{http://yurichev.com/tracer-en.html}{yurichev.com}}
anstatt Debugger.

Der Autor dieses Buchs hat irgendwann aufgehört Debugger zu nutzen, da alles was er von diesen
brauchte, die Funktionsargumente während der Ausführung oder die Zustände der Register an einem
bestimmten Punkt, waren.
Jedes Mal den Debugger zu starten ist zu aufwändig, deswegen entstand das kleine Tool \emph{tracer}.
Es funktioniert in der Kommandozeile und erlaubt es Funktionsausführungen abzufangen,
Breakpoints an beliebigen Stellen zu setzen und Register-Zustände zu lesen und ändern.

\emph{tracer} wird nicht weiterentwickelt, weil es als Demonstrationstool für dieses Buch entstand
und nicht als Tool für den Alltag.
\end{itemize}

\subsection{Tracen von Bibliotheksaufrufen}

\emph{ltrace}\footnote{\url{http://www.ltrace.org/}}.

\subsection{Tracen von Systemaufrufe}

\label{strace}
\myindex{strace}
\myindex{dtruss}
\subsubsection{strace / dtruss}

\myindex{syscall}
Dies zeigt welche Systemaufrufe (syscalls(\myref{syscalls})) vom aktuellen Prozess aufgerufen werden.

Zum Beispiel:

\begin{lstlisting}
# strace df -h

...

access("/etc/ld.so.nohwcap", F_OK)      = -1 ENOENT (No such file or directory)
open("/lib/i386-linux-gnu/libc.so.6", O_RDONLY|O_CLOEXEC) = 3
read(3, "\177ELF\1\1\1\0\0\0\0\0\0\0\0\0\3\0\3\0\1\0\0\0\220\232\1\0004\0\0\0"..., 512) = 512
fstat64(3, {st_mode=S_IFREG|0755, st_size=1770984, ...}) = 0
mmap2(NULL, 1780508, PROT_READ|PROT_EXEC, MAP_PRIVATE|MAP_DENYWRITE, 3, 0) = 0xb75b3000
\end{lstlisting}

\myindex{\MacOSX}
\MacOSX hat dtruss für den Selben Verwendungszweck.

\myindex{Cygwin}
Cygwin beinhaltet ebenso strace, funktioniert aber soweit bekannt nur mit .exe-Dateien
die für die Cygwin-Umgebung kompiliert wurden.

\subsection{Netzwerk-Analyse (Sniffing)}

\emph{Sniffing} ist das Abfangen einiger Informationen die interessant sein könnten.

(kostenlos, Open Source) \emph{Wireshark}\footnote{\url{https://www.wireshark.org/}} für Netzwerk-Analyse.
Stellt ebenfalls die Möglichkeit USB-Schnittstellen zu analysieren\footnote{\url{https://wiki.wireshark.org/CaptureSetup/USB}}.

Wireshark hat einen jüngeren (oder älteren) Bruder \emph{tcpdump}\footnote{\url{http://www.tcpdump.org/}},
bei dem es sich um ein simples Kommandozeilen-Tool handelt.

\subsection{Sysinternals}

\myindex{Sysinternals}
(kostenlos) Sysinternals (entwickelt von Mark Russinovich)
\footnote{\url{https://technet.microsoft.com/en-us/sysinternals/bb842062}}.
Zumindest die folgenden Tools sind wichtig und wert sich damit zu beschäftigen:
Process Explorer, Handle, VMMap, TCPView, Process Monitor.

\subsection{Valgrind}

(kostenlos, Open Source) ein mächtiges Tool um Speicherlecks zu finden: \url{http://valgrind.org/}.
Wegen des ausgeklügelten \ac{JIT}-Mechanismus wird Valgrind oft als Framework für andere Tools genutzt.

% TODO network fuzzing

\subsection{Emulatoren}

\begin{itemize}
\item (kostenlos, Open Source) \emph{QEMU}\footnote{\url{http://qemu.org}}: Emulator für verschiedene CPUs und Architekturen.

\item (kostenlos, Open Source) \emph{DosBox}\footnote{\url{https://www.dosbox.com/}}: MS-DOS-Emulator, meist genutzt für Retro-Gaming.

\item (kostenlos, Open Source) \emph{SimH}\footnote{\url{http://simh.trailing-edge.com/}}: Emulator für ältere Computer, Mainframes, etc.
\end{itemize}

\mysection{Andere Tools}

\emph{Microsoft Visual Studio Express}
\footnote{\href{http://www.microsoft.com/express/Downloads/}{visualstudio.com/en-US/products/visual-studio-express-vs}}:
Abgespeckte, freie Variante von Visual Studio, praktisch für einfache Experimente.

Einige nützliche Optionen: \myref{sec:MSVC_options}.

Es gibt eine Website die ``Compiler Explorer'' heißt und es erlaubt kleine Code-Teile zu kompilieren
und den Output verschiedener GCC-Versionen und Architekturen anzusehen (zumindest x86, ARM, MIPS):
\url{http://godbolt.org/}---Ich hätte es für dieses Buch selber genutzt wenn ich davon gewusst hätte!

\subsection{Rechner}

Gute Rechner für Reverse Engineering sollten zumindest Unterstützung für Dezimal, Hexadezimal und binär-Basen,
sowie wichtige Operationen wie XOR oder Schiebeoperationen haben.

\begin{itemize}

\item IDA hat einen eingebauten Rechner (``?'').

\item rada.re hat \emph{rax2}.

\item \ProgCalcURL

\item Als letzte Rettung hat der Standard-Rechner von Windows einen Programmierer-Modus.

\end{itemize}

\mysection{Fehlt etwas?}

Wenn Sie ein gutes Tool kennen, was hier nicht aufgelistet ist, schreiben Sie mir:\\
\TT{\EMAILS}.
}
%\CN{% !TEX program = XeLaTeX
% !TEX encoding = UTF-8
\documentclass[UTF8,nofonts]{ctexart}
\setCJKmainfont[BoldFont=STHeiti,ItalicFont=STKaiti]{STSong}
\setCJKsansfont[BoldFont=STHeiti]{STXihei}
\setCJKmonofont{STFangsong}

\begin{document}

%daveti: translated on Dec 24, 2016
%NOTE: above is needed for MacTex.

\chapter{工具库 Tools}

\epigraph{
Dennis Yurichev已经把这本书免费和开源 (libre),作为对免费知识和教育的贡献。
同样,因为免费和开源的原则,我们需要免费的 (libre) 逆向工程工具来替代本书中提到
商业的专有工具。}{Richard M. Stallman}

\mysection{二进制文件分析 Binary analysis}

不需要运行程序的静态分析工具。

\myindex{Hiew}
\myindex{UNIX!strings}
\myindex{UNIX!xxd}
\myindex{UNIX!od}

\begin{itemize}
\item
(免费, 开源) \emph{ent}\footnote{\url{http://www.fourmilab.ch/random/}}: 熵 (entropy) 分析工具。
更多关于熵: \myref{entropy}.

\item
\label{Hiew}
\emph{Hiew}\footnote{\href{http://www.hiew.ru/}{hiew.ru}}:
针对于对二进制文件的小改动。

\item (免费, 开源) \emph{xxd} and \emph{od}: UNIX环境下用来显示二进制文件的标准工具。

\item (免费, 开源) \emph{strings}: *NIX环境下用来在二进制文件中寻找ASCII字符串的工具。
Sysinternals提供另一个版本\footnote{\url{https://technet.microsoft.com/en-us/sysinternals/strings}}
支持宽字符串 (UTF-16, Windows中广泛使用)。

\item (免费, 开源) \emph{Binwalk}\footnote{\url{http://binwalk.org/}}: 分析固件镜像的工具。

\item
\myindex{binary grep}
(免费, 开源) \emph{binary grep}:
用来在众多文件中寻找任意字节序列的小工具,
包括非执行文件: \BGREPURL.
% TBT
\end{itemize}

\subsection{反汇编器 Disassemblers}

\myindex{IDA}
\myindex{Binary Ninja}
\myindex{BinNavi}
\myindex{objdump}

\begin{itemize}
\item \emph{IDA}. 一个老的但是免费的版本可以在这里下载
\footnote{\href{http://www.hex-rays.com/idapro/idadownfreeware.htm}{hex-rays.com/products/ida/support/download\_freeware.shtml}}.
\ShortHotKeyCheatsheet: \myref{sec:IDA_cheatsheet}

% TBT \item \emph{Ghidra}\footnote{\url{https://ghidra-sre.org/}} --- free alternative to IDA from \ac{NSA}.

\item \emph{Binary Ninja}\footnote{\url{http://binary.ninja/}}

\item (免费, 开源) \emph{zynamics BinNavi}\footnote{\url{https://www.zynamics.com/binnavi.html}}

\item (免费, 开源) \emph{objdump}: 显示和反汇编二进制文件的小命令行工具。

\item (免费, 开源) \emph{readelf}\footnote{\url{https://sourceware.org/binutils/docs/binutils/readelf.html}}:
显示ELF文件信息。
\end{itemize}

\subsection{反编译器 Decompilers}

事实上,只有一个众所周知,可以下载到的高质量C代码反编译器: \emph{Hex-Rays}:\\
\href{https://www.hex-rays.com/products/decompiler/}{hex-rays.com/products/decompiler/}

% TBT: to be synced with EN version

\subsection{补丁比较工具 Patch comparison/diffing}

为了弄明白补丁都干了些什么或者为什么这个补丁修补了某个漏洞,你可能需要用补丁比较工具来比较原执行文件和打过补丁文件的区别。

\begin{itemize}
\item (免费) \emph{zynamics BinDiff}\footnote{\url{https://www.zynamics.com/software.html}}

\item (免费, 开源) \emph{Diaphora}\footnote{\url{https://github.com/joxeankoret/diaphora}}
\end{itemize}

\mysection{现场分析 Live analysis}

用来对一个运行系统或者进程的分析工具。

\subsection{调试器 Debuggers}

\myindex{\olly}
\myindex{Radare}
\myindex{GDB}
\myindex{tracer}
\myindex{LLDB}
\myindex{WinDbg}

\begin{itemize}
\item (免费) \emph{OllyDbg}.
非常流行的win32用户模式调试器\footnote{\href{http://www.ollydbg.de/}{ollydbg.de}}.
\ShortHotKeyCheatsheet: \myref{sec:Olly_cheatsheet}

\item (免费, 开源) \emph{GDB}.
因为主要是给程序员开发使用的工具,所以在逆向工程师中不是太流行。
部分命令: \myref{sec:GDB_cheatsheet}.
GDB也有一个图形界面接口, ``GDB dashboard''\footnote{\url{https://github.com/cyrus-and/gdb-dashboard}}.

\item (免费, 开源) \emph{LLDB}\footnote{\url{http://lldb.llvm.org/}}.

\item \emph{WinDbg}\footnote{\url{https://developer.microsoft.com/en-us/windows/hardware/windows-driver-kit}}:
Windows内核调试器。

\item (免费, 开源) \emph{Radare} \ac{AKA} rada.re \ac{AKA} r2\footnote{\url{http://rada.re/r/}}.
也有图形界面程序: \emph{ragui}\footnote{\url{http://radare.org/ragui/}}.

\item (免费, 开源) \emph{tracer}.
\label{tracer}
本文作者通常使用 \emph{tracer}
\footnote{\href{http://yurichev.com/tracer-en.html}{yurichev.com}}
而不是调试器。

本文作者事实上已经不再使用调试器,因为用调试器的主要目的是查看函数运行中的传参,或者是某个时刻的寄存器状态。
每次启动运行调试器太费时,所以一个小工具 \emph{tracer} 由此而生。
这是个命令行工具,支持截断函数运行,在任意位置设置断点,读取和改变寄存器状态等。

注意: 没有对 \emph{tracer} 长期维护的计划,因为这个工具是为了用来展示书中例子而开发的,而不是一个每天必备的工具。
\end{itemize}

\subsection{库函数跟踪 Library calls tracing}

\emph{ltrace}\footnote{\url{http://www.ltrace.org/}}.

\subsection{系统调用跟踪 System calls tracing}

\label{strace}
\myindex{strace}
\myindex{dtruss}
\subsubsection{strace / dtruss}

\myindex{syscall}
显示进程在运行中调用的系统调用 (syscalls(\myref{syscalls}))。

例如:

\begin{lstlisting}
# strace df -h

...

access("/etc/ld.so.nohwcap", F_OK)      = -1 ENOENT (No such file or directory)
open("/lib/i386-linux-gnu/libc.so.6", O_RDONLY|O_CLOEXEC) = 3
read(3, "\177ELF\1\1\1\0\0\0\0\0\0\0\0\0\3\0\3\0\1\0\0\0\220\232\1\0004\0\0\0"..., 512) = 512
fstat64(3, {st_mode=S_IFREG|0755, st_size=1770984, ...}) = 0
mmap2(NULL, 1780508, PROT_READ|PROT_EXEC, MAP_PRIVATE|MAP_DENYWRITE, 3, 0) = 0xb75b3000
\end{lstlisting}

\myindex{\MacOSX}
\MacOSX 中的``strace''是dtruss。

\myindex{Cygwin}
Cygwin也提供strace,但是只能用于在cygwin环境编译下的可执行文件。

\subsection{网络嗅探 Network sniffing}

\emph{嗅探 Sniffing} 是截断和跟踪你感兴趣的信息。

(免费, 开源) \emph{Wireshark}\footnote{\url{https://www.wireshark.org/}} 用于网络嗅探。
该工具也支持USB嗅探\footnote{\url{https://wiki.wireshark.org/CaptureSetup/USB}}。

Wireshark还有一个古老的兄弟工具 \emph{tcpdump}\footnote{\url{http://www.tcpdump.org/}}, 一个简单的命令行工具。

\subsection{Sysinternals}

\myindex{Sysinternals}
(免费) Sysinternals (由Mark Russinovich开发)
\footnote{\url{https://technet.microsoft.com/en-us/sysinternals/bb842062}}。
至少以下重要工具值得学习和研究: Process Explorer, Handle, VMMap, TCPView, Process Monitor。

\subsection{Valgrind}

(免费, 开源) 寻找内存泄漏的强大工具: \url{http://valgrind.org/}。
因为其强大的 \ac{JIT} 机制, Valgrind也是其他各种工具的运行框架。

% TODO network fuzzing

\subsection{模拟器 Emulators}

\begin{itemize}
\item (免费, 开源) \emph{QEMU}\footnote{\url{http://qemu.org}}: 模拟各种CPU架构。

\item (免费, 开源) \emph{DosBox}\footnote{\url{https://www.dosbox.com/}}: MS-DOS模拟器,主要用来运行老游戏。 

\item (免费, 开源) \emph{SimH}\footnote{\url{http://simh.trailing-edge.com/}}: 恐龙机,大型机等模拟器。
\end{itemize}

\mysection{其他 Other tools}

\emph{Microsoft Visual Studio Express}
\footnote{\href{http://www.microsoft.com/express/Downloads/}{visualstudio.com/en-US/products/visual-studio-express-vs}}:
Visual Studio的简化免费版本,对于一些简单的实验,已经足够。

一些有用的选项配置: \myref{sec:MSVC_options}。

% TBT
\mysection{遗漏 Something missing here?}

如果你知道还有什么其他有用的工具,和我说一声:\\
\TT{\EMAILS}.

\end{document}

}
\JA{\chapter{ツール}

\epigraph{Now that Dennis Yurichev has made this book free (libre), it is a
contribution to the world of free knowledge and free education.
However, for our freedom's sake, we need free (libre) reverse
engineering tools to replace the proprietary tools described in this book.}{Richard M. Stallman}

\mysection{バイナリ解析}

プロセスを実行しないときに使用するツール。

\myindex{Hiew}
\myindex{GHex}
\myindex{UNIX!strings}
\myindex{UNIX!xxd}
\myindex{UNIX!od}

\begin{itemize}
\item
(フリー、オープンソース) \emph{ent}\footnote{\url{http://www.fourmilab.ch/random/}}: エントロピー分析ツール。
エントロピーについての詳細: \myref{entropy}.

\item
\label{Hiew}
\emph{Hiew}\footnote{\href{http://www.hiew.ru/}{hiew.ru}}:
バイナリファイルでのコードの小さな変更を追う。
アセンブラ/ディスアセンブラを内蔵。

\item (フリー、オープンソース) \emph{GHex}\footnote{\url{https://wiki.gnome.org/Apps/Ghex}}: Linux用の単純な16進エディタ

\item (フリー、オープンソース) \emph{xxd} and \emph{od}: ダンプするための標準的なUNIXユーティリティ

\item (フリー、オープンソース) \emph{strings}: 実行可能ファイルを含むバイナリファイルでASCII文字列を検索するための*NIXツール
Sysinternalsには、ワイド文字列をサポートする代替機能\footnote{\url{https://technet.microsoft.com/en-us/sysinternals/strings}}
があります(Windowsで広く使用されているUTF-16)

\item (フリー、オープンソース) \emph{Binwalk}\footnote{\url{http://binwalk.org/}}: ファームウェアイメージの分析

\item
\myindex{binary grep}
(フリー、オープンソース) \emph{binary grep}:
実行不可能なファイルを含む大きなファイルの中の
バイトシーケンスを検索するための小さなユーティリティ:\BGREPURL
\myindex{rafind2}
同じ目的のためにrada.reにrafind2もあります。
\end{itemize}

\subsection{ディスアセンブラ}

\myindex{IDA}
\myindex{Binary Ninja}
\myindex{BinNavi}
\myindex{objdump}

\begin{itemize}
\item \emph{IDA}. 古いフリーウェアバージョンがダウンロード可能です
\footnote{\href{http://www.hex-rays.com/idapro/idadownfreeware.htm}{hex-rays.com/products/ida/support/download\_freeware.shtml}}.
\ShortHotKeyCheatsheet: \myref{sec:IDA_cheatsheet}

% TBT \item \emph{Ghidra}\footnote{\url{https://ghidra-sre.org/}} --- free alternative to IDA from \ac{NSA}.

\item \emph{Binary Ninja}\footnote{\url{http://binary.ninja/}}

\item (フリー、オープンソース) \emph{zynamics BinNavi}\footnote{\url{https://www.zynamics.com/binnavi.html}}

\item (フリー、オープンソース) \emph{objdump}: ダンプとディスアセンブルのための簡単なコマンドラインユーティリティ

\item (フリー、オープンソース) \emph{readelf}\footnote{\url{https://sourceware.org/binutils/docs/binutils/readelf.html}}:
ELFファイルに関するダンプ情報。
\end{itemize}

\subsection{デコンパイラ}

公に利用可能な既知の高品質なCコードへのデコンパイラがたった1つだけあります:\emph{Hex-Rays}:\\
\href{https://www.hex-rays.com/products/decompiler/}{hex-rays.com/products/decompiler/}

詳細を読む: \myref{hex_rays}.
% TBT: to be synced with EN version

\subsection{パッチの比較/diffing}

実行可能ファイルの一部を元のバージョンと比較し、パッチが適用されたものとその理由を調べるときに
使用することをお勧めします。

\begin{itemize}
\item (フリー) \emph{zynamics BinDiff}\footnote{\url{https://www.zynamics.com/software.html}}

\item (フリー、オープンソース) \emph{Diaphora}\footnote{\url{https://github.com/joxeankoret/diaphora}}
\end{itemize}

\mysection{ライブ解析}

稼働中のシステム上またはプロセスの実行中に使用するツール。

\subsection{デバッガ}

\myindex{\olly}
\myindex{Radare}
\myindex{GDB}
\myindex{tracer}
\myindex{LLDB}
\myindex{WinDbg}
\myindex{IDA}

\begin{itemize}
\item (フリー) \emph{OllyDbg}.
とても人気のあるユーザーモードのwin32デバッガ\footnote{\href{http://www.ollydbg.de/}{ollydbg.de}}.
\ShortHotKeyCheatsheet: \myref{sec:Olly_cheatsheet}

\item (フリー、オープンソース) \emph{GDB}.
主にプログラマ向けなので、リバースエンジニアの間ではあまり一般的ではないデバッガ。
いくつかのコマンド: \myref{sec:GDB_cheatsheet}.
GDB用のビジュアルインターフェイスがあります、 ``GDB dashboard''\footnote{\url{https://github.com/cyrus-and/gdb-dashboard}}.

\item (フリー、オープンソース) \emph{LLDB}\footnote{\url{http://lldb.llvm.org/}}.

\item \emph{WinDbg}\footnote{\url{https://developer.microsoft.com/en-us/windows/hardware/windows-driver-kit}}:
Windows用カーネルデバッガ。

\item \emph{IDA} には内部デバッガがあります。

\item (フリー、オープンソース) \emph{Radare} \ac{AKA} rada.re \ac{AKA} r2\footnote{\url{http://rada.re/r/}}.
GUIも存在します: \emph{ragui}\footnote{\url{http://radare.org/ragui/}}.

\item (フリー、オープンソース) \emph{tracer}.
\label{tracer}
著者はしばしばデバッガの代わりに
\footnote{\href{http://yurichev.com/tracer-en.html}{yurichev.com}}
IT{tracer}を使用します。

著者は、最終的にデバッガの使用を中止しました。実行している間に関数の引数を見つけること、
またはある時点でのレジスタの状態を特定するだけだったからです。
毎回デバッガをロードするのは過剰で、\emph{tracer}という小さなユーティリティが生まれました。
コマンドラインから機能し、関数の実行を傍受したり、
任意の場所でブレークポイントを設定したり、レジスタの状態を読み込んだり変更したりすることができます。

注意:\emph{tracer}はは進化していません。なぜなら、日常的なツールとしてではなく、この本のデモンストレーションツールとして開発されたからです
\end{itemize}

\subsection{ライブラリ コールトレース}

\emph{ltrace}\footnote{\url{http://www.ltrace.org/}}.

\subsection{システムコールトレース}

\label{strace}
\myindex{strace}
\myindex{dtruss}
\subsubsection{strace / dtruss}

\myindex{syscall}
これは、どのシステムコール(syscalls(\myref{syscalls}))が現在プロセスによって呼び出されているかを示します

例えば:

\begin{lstlisting}
# strace df -h

...

access("/etc/ld.so.nohwcap", F_OK)      = -1 ENOENT (No such file or directory)
open("/lib/i386-linux-gnu/libc.so.6", O_RDONLY|O_CLOEXEC) = 3
read(3, "\177ELF\1\1\1\0\0\0\0\0\0\0\0\0\3\0\3\0\1\0\0\0\220\232\1\0004\0\0\0"..., 512) = 512
fstat64(3, {st_mode=S_IFREG|0755, st_size=1770984, ...}) = 0
mmap2(NULL, 1780508, PROT_READ|PROT_EXEC, MAP_PRIVATE|MAP_DENYWRITE, 3, 0) = 0xb75b3000
\end{lstlisting}

\myindex{\MacOSX}
\MacOSX は同じことを行うために dtrussがあります。

\myindex{Cygwin}
Cygwinにはstraceもありますが、知る限り、cygwin環境用にコンパイルされた
.exeファイルに対してのみ動作します。

\subsection{ネットワーク傍受}

\emph{Sniffing}は興味のある情報を傍受します。

(フリー、オープンソース) \emph{Wireshark}\footnote{\url{https://www.wireshark.org/}} ネットワーク傍受のために。
また、USBスニッフィング機能も備えています。\footnote{\url{https://wiki.wireshark.org/CaptureSetup/USB}}.

Wiresharkには若い(または古い)兄弟がいます: \emph{tcpdump}\footnote{\url{http://www.tcpdump.org/}}、簡単なコマンドラインツールです。

\subsection{Sysinternals}

\myindex{Sysinternals}
(フリー) Sysinternals (Mark Russinovichによって開発)
\footnote{\url{https://technet.microsoft.com/en-us/sysinternals/bb842062}}.
少なくともこれらツールは重要で、検討する価値があります:プロセスエクスプローラ、Handle、VMMap、TCPView、プロセスモニタ

\subsection{Valgrind}

(フリー、オープンソース) メモリリークを検出する強力なツール: \url{http://valgrind.org/}.
強力な\ac{JIT}メカニズムのため、Valgrindは他のツールのフレームワークとして使用されています

% TODO network fuzzing

\subsection{エミュレータ}

\begin{itemize}
\item (フリー、オープンソース) \emph{QEMU}\footnote{\url{http://qemu.org}}: さまざまなCPUおよびアーキテクチャ用のエミュレータ

\item (フリー、オープンソース) \emph{DosBox}\footnote{\url{https://www.dosbox.com/}}: MS-DOSエミュレータ、主にレトロゲームに使用されます。

\item (フリー、オープンソース) \emph{SimH}\footnote{\url{http://simh.trailing-edge.com/}}: 大昔のコンピュータ、メインフレームなどのエミュレータ
\end{itemize}

\mysection{他のツール}

\emph{Microsoft Visual Studio Express}
\footnote{\href{http://www.microsoft.com/express/Downloads/}{visualstudio.com/en-US/products/visual-studio-express-vs}}:
簡単な実験に便利な、Visual Studioの無償版。

いくつかの便利なオプション: \myref{sec:MSVC_options}.

``Compiler Explorer'' というWebサイトがあり、小さなコードスニペットをコンパイルし、
さまざまなGCCのバージョンとアーキテクチャ(少なくともx86、ARM、MIPS)で
出力を見ることができます:\url{http://godbolt.org/}---もし私がそれについて知っていたら、私は本のためにそれを使ったでしょう!

% TBT
%\subsection{SMT solvers}

%From the reverse engineer's perspective, SMT solvers are used when dealing with
%amateur cryptography,
%symbolic/concolic execution,
%ROP chains generation.

%For more information, read: \url{https://yurichev.com/writings/SAT_SMT_by_example.pdf}.
\subsection{電卓}

リバースエンジニアのニーズに合った良い電卓は、少なくとも10進数、16進数、2進数ベース、
XORやシフトなどの多くの重要な演算をサポートする必要があります。

\begin{itemize}

\item IDA にはビルトインの電卓があります (``?'').

\item rada.re には \emph{rax2}があります。

\item \ProgCalcURL

\item 最後の手段として、Windowsの標準電卓にはプログラマモードがあります。

\end{itemize}

\mysection{何か足りないものは?}

ここにリストされていない素晴らしいツールが知っている場合は:\\
\TT{\EMAILS}
}

\EN{\mysection{Task manager practical joke (Windows Vista)}
\myindex{Windows!Windows Vista}

Let's see if it's possible to hack Task Manager slightly so it would detect more \ac{CPU} cores.

\myindex{Windows!NTAPI}

Let us first think, how does the Task Manager know the number of cores?

There is the \TT{GetSystemInfo()} win32 function present in win32 userspace which can tell us this.
But it's not imported in \TT{taskmgr.exe}.

There is, however, another one in \gls{NTAPI}, \TT{NtQuerySystemInformation()}, 
which is used in \TT{taskmgr.exe} in several places.

To get the number of cores, one has to call this function with the \TT{SystemBasicInformation} constant
as a first argument (which is zero
\footnote{\href{http://msdn.microsoft.com/en-us/library/windows/desktop/ms724509(v=vs.85).aspx}{MSDN}}).

The second argument has to point to the buffer which is getting all the information.

So we have to find all calls to the \\
\TT{NtQuerySystemInformation(0, ?, ?, ?)} function.
Let's open \TT{taskmgr.exe} in IDA. 
\myindex{Windows!PDB}

What is always good about Microsoft executables is that IDA can download the corresponding \gls{PDB} 
file for this executable and show all function names.

It is visible that Task Manager is written in \Cpp and some of the function names and classes are really 
speaking for themselves.
There are classes CAdapter, CNetPage, CPerfPage, CProcInfo, CProcPage, CSvcPage, 
CTaskPage, CUserPage.

Apparently, each class corresponds to each tab in Task Manager.

Let's visit each call and add comment with the value which is passed as the first function argument.
We will write \q{not zero} at some places, because the value there was clearly not zero, 
but something really different (more about this in the second part of this chapter).

And we are looking for zero passed as argument, after all.

\begin{figure}[H]
\centering
\myincludegraphics{examples/taskmgr/IDA_xrefs.png}
\caption{IDA: cross references to NtQuerySystemInformation()}
\end{figure}

Yes, the names are really speaking for themselves.

When we closely investigate each place where\\
\TT{NtQuerySystemInformation(0, ?, ?, ?)} is called,
we quickly find what we need in the \TT{InitPerfInfo()} function:

\lstinputlisting[caption=taskmgr.exe (Windows Vista),style=customasmx86]{examples/taskmgr/taskmgr.lst}

\TT{g\_cProcessors} is a global variable, and this name has been assigned by 
IDA according to the \gls{PDB} loaded from Microsoft's symbol server.

The byte is taken from \TT{var\_C20}. 
And \TT{var\_C58} is passed to\\
\TT{NtQuerySystemInformation()} 
as a pointer to the receiving buffer.
The difference between 0xC20 and 0xC58 is 0x38 (56).

Let's take a look at format of the return structure, which we can find in MSDN:

\begin{lstlisting}[style=customc]
typedef struct _SYSTEM_BASIC_INFORMATION {
    BYTE Reserved1[24];
    PVOID Reserved2[4];
    CCHAR NumberOfProcessors;
} SYSTEM_BASIC_INFORMATION;
\end{lstlisting}

This is a x64 system, so each PVOID takes 8 bytes.

All \emph{reserved} fields in the structure take $24+4*8=56$ bytes.

Oh yes, this implies that \TT{var\_C20} is the local stack is exactly the
\TT{NumberOfProcessors} field of the \TT{SYSTEM\_BASIC\_INFORMATION} structure.

Let's check our guess.
Copy \TT{taskmgr.exe} from \TT{C:\textbackslash{}Windows\textbackslash{}System32} 
to some other folder 
(so the \emph{Windows Resource Protection} 
will not try to restore the patched \TT{taskmgr.exe}).

Let's open it in Hiew and find the place:

\begin{figure}[H]
\centering
\myincludegraphics{examples/taskmgr/hiew2.png}
\caption{Hiew: find the place to be patched}
\end{figure}

Let's replace the \TT{MOVZX} instruction with ours.
Let's pretend we've got 64 CPU cores.

Add one additional \ac{NOP} (because our instruction is shorter than the original one):

\begin{figure}[H]
\centering
\myincludegraphics{examples/taskmgr/hiew1.png}
\caption{Hiew: patch it}
\end{figure}

And it works!
Of course, the data in the graphs is not correct.

At times, Task Manager even shows an overall CPU load of more than 100\%.

\begin{figure}[H]
\centering
\myincludegraphics{examples/taskmgr/taskmgr_64cpu_crop.png}
\caption{Fooled Windows Task Manager}
\end{figure}

The biggest number Task Manager does not crash with is 64.

Apparently, Task Manager in Windows Vista was not tested on computers with a large number of cores.

So there are probably some static data structure(s) inside it limited to 64 cores.

\subsection{Using LEA to load values}
\label{TaskMgr_LEA}

Sometimes, \TT{LEA} is used in \TT{taskmgr.exe} instead of \TT{MOV} to set the first argument of \\
\TT{NtQuerySystemInformation()}:

\lstinputlisting[caption=taskmgr.exe (Windows Vista),style=customasmx86]{examples/taskmgr/taskmgr2.lst}

\myindex{x86!\Instructions!LEA}

Perhaps \ac{MSVC} did so because machine code of \INS{LEA} is shorter than \INS{MOV REG, 5} (would be 5 instead of 4).

\INS{LEA} with offset in $-128..127$ range (offset will occupy 1 byte in opcode) with 32-bit registers is even shorter (for lack of REX prefix)---3 bytes.

Another example of such thing is: \myref{using_MOV_and_pack_of_LEA_to_load_values}.
}
\RU{\subsection{Обменять входные значения друг с другом}

Вот так:

\lstinputlisting[style=customc]{patterns/061_pointers/swap/5_RU.c}

Как видим, байты загружаются в младшие 8-битные части регистров \TT{ECX} и \TT{EBX} используя \INS{MOVZX}
(так что старшие части регистров очищаются), затем байты записываются назад в другом порядке.

\lstinputlisting[style=customasmx86,caption=Optimizing GCC 5.4]{patterns/061_pointers/swap/5_GCC_O3_x86.s}

Адреса обоих байтов берутся из аргументов и во время исполнения ф-ции находятся в регистрах \TT{EDX} и \TT{EAX}.

Так что исопльзуем указатели --- вероятно, без них нет способа решить эту задачу лучше.

}
%\DE{\mysection{x86}

\subsection{Terminologie}

Geläufig für 16-Bit (8086/80286), 32-Bit (80386, etc.), 64-Bit.

\myindex{IEEE 754}
\myindex{MS-DOS}
\begin{description}
	\item[Byte] 8-Bit.
		Die DB Assembler-Direktive wird zum Definieren von Variablen und Arrays genutzt.
		Bytes werden in dem 8-Bit-Teil der folgenden Register übergeben:
		\TT{AL/BL/CL/DL/AH/BH/CH/DH/SIL/DIL/R*L}.
	\item[Wort] 16-Bit.
		DW Assembler-Direktive \dittoclosing.
		Bytes werden in dem 16-Bit-Teil der folgenden Register übergeben:
			\TT{AX/BX/CX/DX/SI/DI/R*W}.
	\item[Doppelwort] (\q{dword}) 32-Bit.
		DD Assembler-Direktive \dittoclosing.
		Doppelwörter werden in Registern (x86) oder dem 32-Bit-Teil der Register (x64) übergeben.
		In 16-Bit-Code werden Doppelwörter in 16-Bit-Registerpaaren übergeben.
	\item[zwei Doppelwörter] (\q{qword}) 64-Bit.
		DQ Assembler-Direktive \dittoclosing.
		In 32-Bit-Umgebungen werden diese in 32-Bit-Registerpaaren übergeben.
	\item[tbyte] (10 Byte) 80-Bit oder 10 Bytes (für IEEE 754 FPU Register).
	\item[paragraph] (16 Byte) --- Bezeichnung war in MS-DOS Umgebungen gebräuchlich.
\end{description}

\myindex{Windows!API}

Datentypen der selben Breite (BYTE, WORD, DWORD) entsprechen auch denen in der Windows \ac{API}.

% TODO German Translation (DSiekmeier)
%\input{appendix/x86/registers} % subsection
%\input{appendix/x86/instructions} % subsection
\subsection{npad}
\label{sec:npad}

\RU{Это макрос в ассемблере, для выравнивания некоторой метки по некоторой границе.}
\EN{It is an assembly language macro for aligning labels on a specific boundary.}
\DE{Dies ist ein Assembler-Makro um Labels an bestimmten Grenzen auszurichten.}
\FR{C'est une macro du langage d'assemblage pour aligner les labels sur une limite
spécifique.}

\RU{Это нужно для тех \emph{нагруженных} меток, куда чаще всего передается управление, например, 
начало тела цикла. 
Для того чтобы процессор мог эффективнее вытягивать данные или код из памяти, через шину с памятью, 
кэширование, итд.}
\EN{That's often needed for the busy labels to where the control flow is often passed, e.g., loop body starts.
So the CPU can load the data or code from the memory effectively, through the memory bus, cache lines, etc.}
\DE{Dies ist oft nützlich Labels, die oft Ziel einer Kotrollstruktur sind, wie Schleifenköpfe.
Somit kann die CPU Daten oder Code sehr effizient vom Speicher durch den Bus, den Cache, usw. laden.}
\FR{C'est souvent nécessaire pour des labels très utilisés, comme par exemple le
début d'un corps de boucle. Ainsi, le CPU peut charger les données ou le code depuis
la mémoire efficacement, à travers le bus mémoire, les caches, etc.}

\RU{Взято из}\EN{Taken from}\DE{Entnommen von}\FR{Pris de} \TT{listing.inc} (MSVC):

\myindex{x86!\Instructions!NOP}
\RU{Это, кстати, любопытный пример различных вариантов \NOP{}-ов. 
Все эти инструкции не дают никакого эффекта, но отличаются разной длиной.}
\EN{By the way, it is a curious example of the different \NOP variations.
All these instructions have no effects whatsoever, but have a different size.}
\DE{Dies ist übrigens ein Beispiel für die unterschiedlichen \NOP-Variationen.
Keine dieser Anweisungen hat eine Auswirkung, aber alle haben eine unterschiedliche Größe.}
\FR{À propos, c'est un exemple curieux des différentes variations de \NOP. Toutes
ces instructions n'ont pas d'effet, mais ont une taille différente.}

\RU{Цель в том, чтобы была только одна инструкция, а не набор NOP-ов, 
считается что так лучше для производительности CPU.}
\EN{Having a single idle instruction instead of couple of NOP-s,
is accepted to be better for CPU performance.}
\DE{Eine einzelne Idle-Anweisung anstatt mehrerer NOPs hat positive Auswirkungen
auf die CPU-Performance.}
\FR{Avoir une seule instruction sans effet au lieu de plusieurs est accepté comme
étant meilleur pour la performance du CPU.}

\begin{lstlisting}[style=customasmx86]
;; LISTING.INC
;;
;; This file contains assembler macros and is included by the files created
;; with the -FA compiler switch to be assembled by MASM (Microsoft Macro
;; Assembler).
;;
;; Copyright (c) 1993-2003, Microsoft Corporation. All rights reserved.

;; non destructive nops
npad macro size
if size eq 1
  nop
else
 if size eq 2
   mov edi, edi
 else
  if size eq 3
    ; lea ecx, [ecx+00]
    DB 8DH, 49H, 00H
  else
   if size eq 4
     ; lea esp, [esp+00]
     DB 8DH, 64H, 24H, 00H
   else
    if size eq 5
      add eax, DWORD PTR 0
    else
     if size eq 6
       ; lea ebx, [ebx+00000000]
       DB 8DH, 9BH, 00H, 00H, 00H, 00H
     else
      if size eq 7
	; lea esp, [esp+00000000]
	DB 8DH, 0A4H, 24H, 00H, 00H, 00H, 00H 
      else
       if size eq 8
        ; jmp .+8; .npad 6
	DB 0EBH, 06H, 8DH, 9BH, 00H, 00H, 00H, 00H
       else
        if size eq 9
         ; jmp .+9; .npad 7
         DB 0EBH, 07H, 8DH, 0A4H, 24H, 00H, 00H, 00H, 00H
        else
         if size eq 10
          ; jmp .+A; .npad 7; .npad 1
          DB 0EBH, 08H, 8DH, 0A4H, 24H, 00H, 00H, 00H, 00H, 90H
         else
          if size eq 11
           ; jmp .+B; .npad 7; .npad 2
           DB 0EBH, 09H, 8DH, 0A4H, 24H, 00H, 00H, 00H, 00H, 8BH, 0FFH
          else
           if size eq 12
            ; jmp .+C; .npad 7; .npad 3
            DB 0EBH, 0AH, 8DH, 0A4H, 24H, 00H, 00H, 00H, 00H, 8DH, 49H, 00H
           else
            if size eq 13
             ; jmp .+D; .npad 7; .npad 4
             DB 0EBH, 0BH, 8DH, 0A4H, 24H, 00H, 00H, 00H, 00H, 8DH, 64H, 24H, 00H
            else
             if size eq 14
              ; jmp .+E; .npad 7; .npad 5
              DB 0EBH, 0CH, 8DH, 0A4H, 24H, 00H, 00H, 00H, 00H, 05H, 00H, 00H, 00H, 00H
             else
              if size eq 15
               ; jmp .+F; .npad 7; .npad 6
               DB 0EBH, 0DH, 8DH, 0A4H, 24H, 00H, 00H, 00H, 00H, 8DH, 9BH, 00H, 00H, 00H, 00H
              else
	       %out error: unsupported npad size
               .err
              endif
             endif
            endif
           endif
          endif
         endif
        endif
       endif
      endif
     endif
    endif
   endif
  endif
 endif
endif
endm
\end{lstlisting}
 % subsection
}
\FR{\subsection{Exemple \#2: SCO OpenServer}

\label{examples_SCO}
\myindex{SCO OpenServer}
Un ancien logiciel pour SCO OpenServer de 1997 développé par une société qui a disparue
depuis longtemps.

Il y a un driver de dongle special à installer dans le système, qui contient les
chaînes de texte suivantes:
\q{Copyright 1989, Rainbow Technologies, Inc., Irvine, CA}
et
\q{Sentinel Integrated Driver Ver. 3.0 }.

Après l'installation du driver dans SCO OpenServer, ces fichiers apparaissent dans
l'arborescence /dev:

\begin{lstlisting}
/dev/rbsl8
/dev/rbsl9
/dev/rbsl10
\end{lstlisting}

Le programme renvoie une erreur lorsque le dongle n'est pas connecté, mais le message
d'erreur n'est pas trouvé dans les exécutables.

\myindex{COFF}

Grâce à \ac{IDA}, il est facile de charger l'exécutable COFF utilisé dans SCO OpenServer.

Essayons de trouver la chaîne \q{rbsl} et en effet, elle se trouve dans ce morceau
de code:

\lstinputlisting[style=customasmx86]{examples/dongles/2/1.lst}

Oui, en effet, le programme doit communiquer d'une façon ou d'une autre avec le driver.

\myindex{thunk-functions}
Le seul endroit où la fonction \TT{SSQC()} est appelée est dans la \glslink{thunk
 function}{fonction thunk}:

\lstinputlisting[style=customasmx86]{examples/dongles/2/2.lst}

SSQ() peut être appelé depuis au moins 2 fonctions.

L'une d'entre elles est:

\lstinputlisting[style=customasmx86]{examples/dongles/2/check1_EN.lst}

\q{\TT{3C}} et \q{\TT{3E}} semblent familiers: il y avait un dongle Sentinel Pro de
Rainbow sans mémoire, fournissant seulement une fonction de crypto-hachage secrète.

Vous pouvez lire une courte description de la fonction de hachage dont il s'agit
ici: \myref{hash_func}.

Mais retournons au programme.

Donc le programme peut seulement tester si un dongle est connecté ou s'il est absent.

Aucune autre information ne peut être écrite dans un tel dongle, puisqu'il n'a pas
de mémoire.
Les codes sur deux caractères sont des commandes (nous pouvons voir comment les commandes
sont traitées dans la fonction \TT{SSQC()}) et toutes les autres chaînes sont hachées
dans le dongle, transformées en un nombre 16-bit.
L'algorithme était secret, donc il n'était pas possible d'écrire un driver de remplacement
ou de refaire un dongle matériel qui l'émulerait parfaitement.

Toutefois, il est toujours possible d'intercepter tous les accès au dongle et de
trouver les constantes auxquelles les résultats de la fonction de hachage sont comparées.

Mais nous devons dire qu'il est possible de construire un schéma de logiciel de protection
de copie robuste basé sur une fonction secrète de hachage cryptographique: il suffit
qu'elle chiffre/déchiffre les fichiers de données utilisés par votre logiciel.

Mais retournons au code:

Les codes 51/52/53 sont utilisés pour choisir le port imprimante LPT.
3x/4x sont utilisés pour le choix de la \q{famille} (c'est ainsi que les dongles
Sentinel Pro sont différenciés les uns des autres: plus d'un dongle peut être connecté
sur un port LPT).

La seule chaîne passée à la fonction qui ne fasse pas 2 caractères est "0123456789".

Ensuite, le résultat est comparé à l'ensemble des résultats valides.

Si il est correct, 0xC ou 0xB est écrit dans la variable globale \TT{ctl\_model}.%

Une autre chaîne de texte qui est passée est
"PRESS ANY KEY TO CONTINUE: ", mais le résultat n'est pas testé.
Difficile de dire pourquoi, probablement une erreur\footnote{C'est un sentiment
étrange de trouver un bug dans un logiciel aussi ancien.}.

Voyons où la valeur de la variable globale \TT{ctl\_model} est utilisée.

Un tel endroit est:

\lstinputlisting[style=customasmx86]{examples/dongles/2/4.lst}

Si c'est 0, un message d'erreur chiffré est passé à une routine de déchiffrement
et affiché.

\myindex{x86!\Instructions!XOR}

La routine de déchiffrement de la chaîne semble être un simple \glslink{xoring}{xor}:

\lstinputlisting[style=customasmx86]{examples/dongles/2/err_warn.lst}

C'est pourquoi nous étions incapable de trouver le message d'erreur dans les fichiers
exécutable, car ils sont chiffrés (ce qui est une pratique courante).

Un autre appel à la fonction de hachage \TT{SSQ()} lui passe la chaîne \q{offln}
et le résultat est comparé avec \TT{0xFE81} et \TT{0x12A9}.

Si ils ne correspondent pas, ça se comporte comme une sorte de fonction \TT{timer()}
(peut-être en attente qu'un dongle mal connecté soit reconnecté et re-testé?) et ensuite
déchiffre un autre message d'erreur à afficher.

\lstinputlisting[style=customasmx86]{examples/dongles/2/check2_EN.lst}

Passer outre le dongle est assez facile: il suffit de patcher tous les sauts après
les instructions \CMP pertinentes.

Une autre option est d'écrire notre propre driver SCO OpenServer, contenant une table
de questions et de réponses, toutes celles qui sont présentent dans le programme.

\subsubsection{Déchiffrer les messages d'erreur}

À propos, nous pouvons aussi essayer de déchiffrer tous les messages d'erreurs.
L'algorithme qui se trouve dans la fonction \TT{err\_warn()} est très simple, en effet:

\lstinputlisting[caption=Decryption function,style=customasmx86]{examples/dongles/2/decrypting_FR.lst}

Comme on le voit, non seulement la chaîne est transmise à la fonction de déchiffrement
mais aussi la clef:

\lstinputlisting[style=customasmx86]{examples/dongles/2/tmp1_EN.asm}

L'algorithme est un simple \glslink{xoring}{xor}: chaque octet est xoré avec la clef, mais
la clef est incrémentée de 3 après le traitement de chaque octet.

Nous pouvons écrire un petit script Python pour vérifier notre hypothèse:

\lstinputlisting[caption=Python 3.x]{examples/dongles/2/decr1.py}

Et il affiche: \q{check security device connection}.
Donc oui, ceci est le message déchiffré.

Il y a d'autres messages chiffrés, avec leur clef correspondante.
Mais inutile de dire qu'il est possible de les déchiffrer sans leur clef.
Premièrement, nous voyons que le clef est en fait un octet.
C'est parce que l'instruction principale de déchiffrement (\XOR) fonctionne au niveau
de l'octet.
La clef se trouve dans le registre \ESI, mais seulement une partie de \ESI d'un octet
est utilisée.
Ainsi, une clef pourrait être plus grande que 255, mais sa valeur est toujours arrondie.

En conséquence, nous pouvons simplement essayer de brute-forcer, en essayant toutes
les clefs possible dans l'intervalle 0..255.
Nous allons aussi écarter les messages comportants des caractères non-imprimable.

\lstinputlisting[caption=Python 3.x]{examples/dongles/2/decr2.py}

Et nous obtenons:

\lstinputlisting[caption=Results]{examples/dongles/2/decr2_result.txt}

Ici il y a un peu de déchet, mais nous pouvons rapidement trouver les messages en
anglais.

À propos, puisque l'algorithme est un simple chiffrement xor, la même fonction peut
être utilisée pour chiffrer les messages.
Si besoin, nous pouvons chiffrer nos propres messages, et patcher le programme en les insérant.
}
\EN{\mysection{Task manager practical joke (Windows Vista)}
\myindex{Windows!Windows Vista}

Let's see if it's possible to hack Task Manager slightly so it would detect more \ac{CPU} cores.

\myindex{Windows!NTAPI}

Let us first think, how does the Task Manager know the number of cores?

There is the \TT{GetSystemInfo()} win32 function present in win32 userspace which can tell us this.
But it's not imported in \TT{taskmgr.exe}.

There is, however, another one in \gls{NTAPI}, \TT{NtQuerySystemInformation()}, 
which is used in \TT{taskmgr.exe} in several places.

To get the number of cores, one has to call this function with the \TT{SystemBasicInformation} constant
as a first argument (which is zero
\footnote{\href{http://msdn.microsoft.com/en-us/library/windows/desktop/ms724509(v=vs.85).aspx}{MSDN}}).

The second argument has to point to the buffer which is getting all the information.

So we have to find all calls to the \\
\TT{NtQuerySystemInformation(0, ?, ?, ?)} function.
Let's open \TT{taskmgr.exe} in IDA. 
\myindex{Windows!PDB}

What is always good about Microsoft executables is that IDA can download the corresponding \gls{PDB} 
file for this executable and show all function names.

It is visible that Task Manager is written in \Cpp and some of the function names and classes are really 
speaking for themselves.
There are classes CAdapter, CNetPage, CPerfPage, CProcInfo, CProcPage, CSvcPage, 
CTaskPage, CUserPage.

Apparently, each class corresponds to each tab in Task Manager.

Let's visit each call and add comment with the value which is passed as the first function argument.
We will write \q{not zero} at some places, because the value there was clearly not zero, 
but something really different (more about this in the second part of this chapter).

And we are looking for zero passed as argument, after all.

\begin{figure}[H]
\centering
\myincludegraphics{examples/taskmgr/IDA_xrefs.png}
\caption{IDA: cross references to NtQuerySystemInformation()}
\end{figure}

Yes, the names are really speaking for themselves.

When we closely investigate each place where\\
\TT{NtQuerySystemInformation(0, ?, ?, ?)} is called,
we quickly find what we need in the \TT{InitPerfInfo()} function:

\lstinputlisting[caption=taskmgr.exe (Windows Vista),style=customasmx86]{examples/taskmgr/taskmgr.lst}

\TT{g\_cProcessors} is a global variable, and this name has been assigned by 
IDA according to the \gls{PDB} loaded from Microsoft's symbol server.

The byte is taken from \TT{var\_C20}. 
And \TT{var\_C58} is passed to\\
\TT{NtQuerySystemInformation()} 
as a pointer to the receiving buffer.
The difference between 0xC20 and 0xC58 is 0x38 (56).

Let's take a look at format of the return structure, which we can find in MSDN:

\begin{lstlisting}[style=customc]
typedef struct _SYSTEM_BASIC_INFORMATION {
    BYTE Reserved1[24];
    PVOID Reserved2[4];
    CCHAR NumberOfProcessors;
} SYSTEM_BASIC_INFORMATION;
\end{lstlisting}

This is a x64 system, so each PVOID takes 8 bytes.

All \emph{reserved} fields in the structure take $24+4*8=56$ bytes.

Oh yes, this implies that \TT{var\_C20} is the local stack is exactly the
\TT{NumberOfProcessors} field of the \TT{SYSTEM\_BASIC\_INFORMATION} structure.

Let's check our guess.
Copy \TT{taskmgr.exe} from \TT{C:\textbackslash{}Windows\textbackslash{}System32} 
to some other folder 
(so the \emph{Windows Resource Protection} 
will not try to restore the patched \TT{taskmgr.exe}).

Let's open it in Hiew and find the place:

\begin{figure}[H]
\centering
\myincludegraphics{examples/taskmgr/hiew2.png}
\caption{Hiew: find the place to be patched}
\end{figure}

Let's replace the \TT{MOVZX} instruction with ours.
Let's pretend we've got 64 CPU cores.

Add one additional \ac{NOP} (because our instruction is shorter than the original one):

\begin{figure}[H]
\centering
\myincludegraphics{examples/taskmgr/hiew1.png}
\caption{Hiew: patch it}
\end{figure}

And it works!
Of course, the data in the graphs is not correct.

At times, Task Manager even shows an overall CPU load of more than 100\%.

\begin{figure}[H]
\centering
\myincludegraphics{examples/taskmgr/taskmgr_64cpu_crop.png}
\caption{Fooled Windows Task Manager}
\end{figure}

The biggest number Task Manager does not crash with is 64.

Apparently, Task Manager in Windows Vista was not tested on computers with a large number of cores.

So there are probably some static data structure(s) inside it limited to 64 cores.

\subsection{Using LEA to load values}
\label{TaskMgr_LEA}

Sometimes, \TT{LEA} is used in \TT{taskmgr.exe} instead of \TT{MOV} to set the first argument of \\
\TT{NtQuerySystemInformation()}:

\lstinputlisting[caption=taskmgr.exe (Windows Vista),style=customasmx86]{examples/taskmgr/taskmgr2.lst}

\myindex{x86!\Instructions!LEA}

Perhaps \ac{MSVC} did so because machine code of \INS{LEA} is shorter than \INS{MOV REG, 5} (would be 5 instead of 4).

\INS{LEA} with offset in $-128..127$ range (offset will occupy 1 byte in opcode) with 32-bit registers is even shorter (for lack of REX prefix)---3 bytes.

Another example of such thing is: \myref{using_MOV_and_pack_of_LEA_to_load_values}.
}%
\RU{\subsection{Обменять входные значения друг с другом}

Вот так:

\lstinputlisting[style=customc]{patterns/061_pointers/swap/5_RU.c}

Как видим, байты загружаются в младшие 8-битные части регистров \TT{ECX} и \TT{EBX} используя \INS{MOVZX}
(так что старшие части регистров очищаются), затем байты записываются назад в другом порядке.

\lstinputlisting[style=customasmx86,caption=Optimizing GCC 5.4]{patterns/061_pointers/swap/5_GCC_O3_x86.s}

Адреса обоих байтов берутся из аргументов и во время исполнения ф-ции находятся в регистрах \TT{EDX} и \TT{EAX}.

Так что исопльзуем указатели --- вероятно, без них нет способа решить эту задачу лучше.

}%
\FR{\subsection{Exemple \#2: SCO OpenServer}

\label{examples_SCO}
\myindex{SCO OpenServer}
Un ancien logiciel pour SCO OpenServer de 1997 développé par une société qui a disparue
depuis longtemps.

Il y a un driver de dongle special à installer dans le système, qui contient les
chaînes de texte suivantes:
\q{Copyright 1989, Rainbow Technologies, Inc., Irvine, CA}
et
\q{Sentinel Integrated Driver Ver. 3.0 }.

Après l'installation du driver dans SCO OpenServer, ces fichiers apparaissent dans
l'arborescence /dev:

\begin{lstlisting}
/dev/rbsl8
/dev/rbsl9
/dev/rbsl10
\end{lstlisting}

Le programme renvoie une erreur lorsque le dongle n'est pas connecté, mais le message
d'erreur n'est pas trouvé dans les exécutables.

\myindex{COFF}

Grâce à \ac{IDA}, il est facile de charger l'exécutable COFF utilisé dans SCO OpenServer.

Essayons de trouver la chaîne \q{rbsl} et en effet, elle se trouve dans ce morceau
de code:

\lstinputlisting[style=customasmx86]{examples/dongles/2/1.lst}

Oui, en effet, le programme doit communiquer d'une façon ou d'une autre avec le driver.

\myindex{thunk-functions}
Le seul endroit où la fonction \TT{SSQC()} est appelée est dans la \glslink{thunk
 function}{fonction thunk}:

\lstinputlisting[style=customasmx86]{examples/dongles/2/2.lst}

SSQ() peut être appelé depuis au moins 2 fonctions.

L'une d'entre elles est:

\lstinputlisting[style=customasmx86]{examples/dongles/2/check1_EN.lst}

\q{\TT{3C}} et \q{\TT{3E}} semblent familiers: il y avait un dongle Sentinel Pro de
Rainbow sans mémoire, fournissant seulement une fonction de crypto-hachage secrète.

Vous pouvez lire une courte description de la fonction de hachage dont il s'agit
ici: \myref{hash_func}.

Mais retournons au programme.

Donc le programme peut seulement tester si un dongle est connecté ou s'il est absent.

Aucune autre information ne peut être écrite dans un tel dongle, puisqu'il n'a pas
de mémoire.
Les codes sur deux caractères sont des commandes (nous pouvons voir comment les commandes
sont traitées dans la fonction \TT{SSQC()}) et toutes les autres chaînes sont hachées
dans le dongle, transformées en un nombre 16-bit.
L'algorithme était secret, donc il n'était pas possible d'écrire un driver de remplacement
ou de refaire un dongle matériel qui l'émulerait parfaitement.

Toutefois, il est toujours possible d'intercepter tous les accès au dongle et de
trouver les constantes auxquelles les résultats de la fonction de hachage sont comparées.

Mais nous devons dire qu'il est possible de construire un schéma de logiciel de protection
de copie robuste basé sur une fonction secrète de hachage cryptographique: il suffit
qu'elle chiffre/déchiffre les fichiers de données utilisés par votre logiciel.

Mais retournons au code:

Les codes 51/52/53 sont utilisés pour choisir le port imprimante LPT.
3x/4x sont utilisés pour le choix de la \q{famille} (c'est ainsi que les dongles
Sentinel Pro sont différenciés les uns des autres: plus d'un dongle peut être connecté
sur un port LPT).

La seule chaîne passée à la fonction qui ne fasse pas 2 caractères est "0123456789".

Ensuite, le résultat est comparé à l'ensemble des résultats valides.

Si il est correct, 0xC ou 0xB est écrit dans la variable globale \TT{ctl\_model}.%

Une autre chaîne de texte qui est passée est
"PRESS ANY KEY TO CONTINUE: ", mais le résultat n'est pas testé.
Difficile de dire pourquoi, probablement une erreur\footnote{C'est un sentiment
étrange de trouver un bug dans un logiciel aussi ancien.}.

Voyons où la valeur de la variable globale \TT{ctl\_model} est utilisée.

Un tel endroit est:

\lstinputlisting[style=customasmx86]{examples/dongles/2/4.lst}

Si c'est 0, un message d'erreur chiffré est passé à une routine de déchiffrement
et affiché.

\myindex{x86!\Instructions!XOR}

La routine de déchiffrement de la chaîne semble être un simple \glslink{xoring}{xor}:

\lstinputlisting[style=customasmx86]{examples/dongles/2/err_warn.lst}

C'est pourquoi nous étions incapable de trouver le message d'erreur dans les fichiers
exécutable, car ils sont chiffrés (ce qui est une pratique courante).

Un autre appel à la fonction de hachage \TT{SSQ()} lui passe la chaîne \q{offln}
et le résultat est comparé avec \TT{0xFE81} et \TT{0x12A9}.

Si ils ne correspondent pas, ça se comporte comme une sorte de fonction \TT{timer()}
(peut-être en attente qu'un dongle mal connecté soit reconnecté et re-testé?) et ensuite
déchiffre un autre message d'erreur à afficher.

\lstinputlisting[style=customasmx86]{examples/dongles/2/check2_EN.lst}

Passer outre le dongle est assez facile: il suffit de patcher tous les sauts après
les instructions \CMP pertinentes.

Une autre option est d'écrire notre propre driver SCO OpenServer, contenant une table
de questions et de réponses, toutes celles qui sont présentent dans le programme.

\subsubsection{Déchiffrer les messages d'erreur}

À propos, nous pouvons aussi essayer de déchiffrer tous les messages d'erreurs.
L'algorithme qui se trouve dans la fonction \TT{err\_warn()} est très simple, en effet:

\lstinputlisting[caption=Decryption function,style=customasmx86]{examples/dongles/2/decrypting_FR.lst}

Comme on le voit, non seulement la chaîne est transmise à la fonction de déchiffrement
mais aussi la clef:

\lstinputlisting[style=customasmx86]{examples/dongles/2/tmp1_EN.asm}

L'algorithme est un simple \glslink{xoring}{xor}: chaque octet est xoré avec la clef, mais
la clef est incrémentée de 3 après le traitement de chaque octet.

Nous pouvons écrire un petit script Python pour vérifier notre hypothèse:

\lstinputlisting[caption=Python 3.x]{examples/dongles/2/decr1.py}

Et il affiche: \q{check security device connection}.
Donc oui, ceci est le message déchiffré.

Il y a d'autres messages chiffrés, avec leur clef correspondante.
Mais inutile de dire qu'il est possible de les déchiffrer sans leur clef.
Premièrement, nous voyons que le clef est en fait un octet.
C'est parce que l'instruction principale de déchiffrement (\XOR) fonctionne au niveau
de l'octet.
La clef se trouve dans le registre \ESI, mais seulement une partie de \ESI d'un octet
est utilisée.
Ainsi, une clef pourrait être plus grande que 255, mais sa valeur est toujours arrondie.

En conséquence, nous pouvons simplement essayer de brute-forcer, en essayant toutes
les clefs possible dans l'intervalle 0..255.
Nous allons aussi écarter les messages comportants des caractères non-imprimable.

\lstinputlisting[caption=Python 3.x]{examples/dongles/2/decr2.py}

Et nous obtenons:

\lstinputlisting[caption=Results]{examples/dongles/2/decr2_result.txt}

Ici il y a un peu de déchet, mais nous pouvons rapidement trouver les messages en
anglais.

À propos, puisque l'algorithme est un simple chiffrement xor, la même fonction peut
être utilisée pour chiffrer les messages.
Si besoin, nous pouvons chiffrer nos propres messages, et patcher le programme en les insérant.
}


\EN{\mysection{Task manager practical joke (Windows Vista)}
\myindex{Windows!Windows Vista}

Let's see if it's possible to hack Task Manager slightly so it would detect more \ac{CPU} cores.

\myindex{Windows!NTAPI}

Let us first think, how does the Task Manager know the number of cores?

There is the \TT{GetSystemInfo()} win32 function present in win32 userspace which can tell us this.
But it's not imported in \TT{taskmgr.exe}.

There is, however, another one in \gls{NTAPI}, \TT{NtQuerySystemInformation()}, 
which is used in \TT{taskmgr.exe} in several places.

To get the number of cores, one has to call this function with the \TT{SystemBasicInformation} constant
as a first argument (which is zero
\footnote{\href{http://msdn.microsoft.com/en-us/library/windows/desktop/ms724509(v=vs.85).aspx}{MSDN}}).

The second argument has to point to the buffer which is getting all the information.

So we have to find all calls to the \\
\TT{NtQuerySystemInformation(0, ?, ?, ?)} function.
Let's open \TT{taskmgr.exe} in IDA. 
\myindex{Windows!PDB}

What is always good about Microsoft executables is that IDA can download the corresponding \gls{PDB} 
file for this executable and show all function names.

It is visible that Task Manager is written in \Cpp and some of the function names and classes are really 
speaking for themselves.
There are classes CAdapter, CNetPage, CPerfPage, CProcInfo, CProcPage, CSvcPage, 
CTaskPage, CUserPage.

Apparently, each class corresponds to each tab in Task Manager.

Let's visit each call and add comment with the value which is passed as the first function argument.
We will write \q{not zero} at some places, because the value there was clearly not zero, 
but something really different (more about this in the second part of this chapter).

And we are looking for zero passed as argument, after all.

\begin{figure}[H]
\centering
\myincludegraphics{examples/taskmgr/IDA_xrefs.png}
\caption{IDA: cross references to NtQuerySystemInformation()}
\end{figure}

Yes, the names are really speaking for themselves.

When we closely investigate each place where\\
\TT{NtQuerySystemInformation(0, ?, ?, ?)} is called,
we quickly find what we need in the \TT{InitPerfInfo()} function:

\lstinputlisting[caption=taskmgr.exe (Windows Vista),style=customasmx86]{examples/taskmgr/taskmgr.lst}

\TT{g\_cProcessors} is a global variable, and this name has been assigned by 
IDA according to the \gls{PDB} loaded from Microsoft's symbol server.

The byte is taken from \TT{var\_C20}. 
And \TT{var\_C58} is passed to\\
\TT{NtQuerySystemInformation()} 
as a pointer to the receiving buffer.
The difference between 0xC20 and 0xC58 is 0x38 (56).

Let's take a look at format of the return structure, which we can find in MSDN:

\begin{lstlisting}[style=customc]
typedef struct _SYSTEM_BASIC_INFORMATION {
    BYTE Reserved1[24];
    PVOID Reserved2[4];
    CCHAR NumberOfProcessors;
} SYSTEM_BASIC_INFORMATION;
\end{lstlisting}

This is a x64 system, so each PVOID takes 8 bytes.

All \emph{reserved} fields in the structure take $24+4*8=56$ bytes.

Oh yes, this implies that \TT{var\_C20} is the local stack is exactly the
\TT{NumberOfProcessors} field of the \TT{SYSTEM\_BASIC\_INFORMATION} structure.

Let's check our guess.
Copy \TT{taskmgr.exe} from \TT{C:\textbackslash{}Windows\textbackslash{}System32} 
to some other folder 
(so the \emph{Windows Resource Protection} 
will not try to restore the patched \TT{taskmgr.exe}).

Let's open it in Hiew and find the place:

\begin{figure}[H]
\centering
\myincludegraphics{examples/taskmgr/hiew2.png}
\caption{Hiew: find the place to be patched}
\end{figure}

Let's replace the \TT{MOVZX} instruction with ours.
Let's pretend we've got 64 CPU cores.

Add one additional \ac{NOP} (because our instruction is shorter than the original one):

\begin{figure}[H]
\centering
\myincludegraphics{examples/taskmgr/hiew1.png}
\caption{Hiew: patch it}
\end{figure}

And it works!
Of course, the data in the graphs is not correct.

At times, Task Manager even shows an overall CPU load of more than 100\%.

\begin{figure}[H]
\centering
\myincludegraphics{examples/taskmgr/taskmgr_64cpu_crop.png}
\caption{Fooled Windows Task Manager}
\end{figure}

The biggest number Task Manager does not crash with is 64.

Apparently, Task Manager in Windows Vista was not tested on computers with a large number of cores.

So there are probably some static data structure(s) inside it limited to 64 cores.

\subsection{Using LEA to load values}
\label{TaskMgr_LEA}

Sometimes, \TT{LEA} is used in \TT{taskmgr.exe} instead of \TT{MOV} to set the first argument of \\
\TT{NtQuerySystemInformation()}:

\lstinputlisting[caption=taskmgr.exe (Windows Vista),style=customasmx86]{examples/taskmgr/taskmgr2.lst}

\myindex{x86!\Instructions!LEA}

Perhaps \ac{MSVC} did so because machine code of \INS{LEA} is shorter than \INS{MOV REG, 5} (would be 5 instead of 4).

\INS{LEA} with offset in $-128..127$ range (offset will occupy 1 byte in opcode) with 32-bit registers is even shorter (for lack of REX prefix)---3 bytes.

Another example of such thing is: \myref{using_MOV_and_pack_of_LEA_to_load_values}.
}%
\RU{\subsection{Обменять входные значения друг с другом}

Вот так:

\lstinputlisting[style=customc]{patterns/061_pointers/swap/5_RU.c}

Как видим, байты загружаются в младшие 8-битные части регистров \TT{ECX} и \TT{EBX} используя \INS{MOVZX}
(так что старшие части регистров очищаются), затем байты записываются назад в другом порядке.

\lstinputlisting[style=customasmx86,caption=Optimizing GCC 5.4]{patterns/061_pointers/swap/5_GCC_O3_x86.s}

Адреса обоих байтов берутся из аргументов и во время исполнения ф-ции находятся в регистрах \TT{EDX} и \TT{EAX}.

Так что исопльзуем указатели --- вероятно, без них нет способа решить эту задачу лучше.

}%
\FR{\subsection{Exemple \#2: SCO OpenServer}

\label{examples_SCO}
\myindex{SCO OpenServer}
Un ancien logiciel pour SCO OpenServer de 1997 développé par une société qui a disparue
depuis longtemps.

Il y a un driver de dongle special à installer dans le système, qui contient les
chaînes de texte suivantes:
\q{Copyright 1989, Rainbow Technologies, Inc., Irvine, CA}
et
\q{Sentinel Integrated Driver Ver. 3.0 }.

Après l'installation du driver dans SCO OpenServer, ces fichiers apparaissent dans
l'arborescence /dev:

\begin{lstlisting}
/dev/rbsl8
/dev/rbsl9
/dev/rbsl10
\end{lstlisting}

Le programme renvoie une erreur lorsque le dongle n'est pas connecté, mais le message
d'erreur n'est pas trouvé dans les exécutables.

\myindex{COFF}

Grâce à \ac{IDA}, il est facile de charger l'exécutable COFF utilisé dans SCO OpenServer.

Essayons de trouver la chaîne \q{rbsl} et en effet, elle se trouve dans ce morceau
de code:

\lstinputlisting[style=customasmx86]{examples/dongles/2/1.lst}

Oui, en effet, le programme doit communiquer d'une façon ou d'une autre avec le driver.

\myindex{thunk-functions}
Le seul endroit où la fonction \TT{SSQC()} est appelée est dans la \glslink{thunk
 function}{fonction thunk}:

\lstinputlisting[style=customasmx86]{examples/dongles/2/2.lst}

SSQ() peut être appelé depuis au moins 2 fonctions.

L'une d'entre elles est:

\lstinputlisting[style=customasmx86]{examples/dongles/2/check1_EN.lst}

\q{\TT{3C}} et \q{\TT{3E}} semblent familiers: il y avait un dongle Sentinel Pro de
Rainbow sans mémoire, fournissant seulement une fonction de crypto-hachage secrète.

Vous pouvez lire une courte description de la fonction de hachage dont il s'agit
ici: \myref{hash_func}.

Mais retournons au programme.

Donc le programme peut seulement tester si un dongle est connecté ou s'il est absent.

Aucune autre information ne peut être écrite dans un tel dongle, puisqu'il n'a pas
de mémoire.
Les codes sur deux caractères sont des commandes (nous pouvons voir comment les commandes
sont traitées dans la fonction \TT{SSQC()}) et toutes les autres chaînes sont hachées
dans le dongle, transformées en un nombre 16-bit.
L'algorithme était secret, donc il n'était pas possible d'écrire un driver de remplacement
ou de refaire un dongle matériel qui l'émulerait parfaitement.

Toutefois, il est toujours possible d'intercepter tous les accès au dongle et de
trouver les constantes auxquelles les résultats de la fonction de hachage sont comparées.

Mais nous devons dire qu'il est possible de construire un schéma de logiciel de protection
de copie robuste basé sur une fonction secrète de hachage cryptographique: il suffit
qu'elle chiffre/déchiffre les fichiers de données utilisés par votre logiciel.

Mais retournons au code:

Les codes 51/52/53 sont utilisés pour choisir le port imprimante LPT.
3x/4x sont utilisés pour le choix de la \q{famille} (c'est ainsi que les dongles
Sentinel Pro sont différenciés les uns des autres: plus d'un dongle peut être connecté
sur un port LPT).

La seule chaîne passée à la fonction qui ne fasse pas 2 caractères est "0123456789".

Ensuite, le résultat est comparé à l'ensemble des résultats valides.

Si il est correct, 0xC ou 0xB est écrit dans la variable globale \TT{ctl\_model}.%

Une autre chaîne de texte qui est passée est
"PRESS ANY KEY TO CONTINUE: ", mais le résultat n'est pas testé.
Difficile de dire pourquoi, probablement une erreur\footnote{C'est un sentiment
étrange de trouver un bug dans un logiciel aussi ancien.}.

Voyons où la valeur de la variable globale \TT{ctl\_model} est utilisée.

Un tel endroit est:

\lstinputlisting[style=customasmx86]{examples/dongles/2/4.lst}

Si c'est 0, un message d'erreur chiffré est passé à une routine de déchiffrement
et affiché.

\myindex{x86!\Instructions!XOR}

La routine de déchiffrement de la chaîne semble être un simple \glslink{xoring}{xor}:

\lstinputlisting[style=customasmx86]{examples/dongles/2/err_warn.lst}

C'est pourquoi nous étions incapable de trouver le message d'erreur dans les fichiers
exécutable, car ils sont chiffrés (ce qui est une pratique courante).

Un autre appel à la fonction de hachage \TT{SSQ()} lui passe la chaîne \q{offln}
et le résultat est comparé avec \TT{0xFE81} et \TT{0x12A9}.

Si ils ne correspondent pas, ça se comporte comme une sorte de fonction \TT{timer()}
(peut-être en attente qu'un dongle mal connecté soit reconnecté et re-testé?) et ensuite
déchiffre un autre message d'erreur à afficher.

\lstinputlisting[style=customasmx86]{examples/dongles/2/check2_EN.lst}

Passer outre le dongle est assez facile: il suffit de patcher tous les sauts après
les instructions \CMP pertinentes.

Une autre option est d'écrire notre propre driver SCO OpenServer, contenant une table
de questions et de réponses, toutes celles qui sont présentent dans le programme.

\subsubsection{Déchiffrer les messages d'erreur}

À propos, nous pouvons aussi essayer de déchiffrer tous les messages d'erreurs.
L'algorithme qui se trouve dans la fonction \TT{err\_warn()} est très simple, en effet:

\lstinputlisting[caption=Decryption function,style=customasmx86]{examples/dongles/2/decrypting_FR.lst}

Comme on le voit, non seulement la chaîne est transmise à la fonction de déchiffrement
mais aussi la clef:

\lstinputlisting[style=customasmx86]{examples/dongles/2/tmp1_EN.asm}

L'algorithme est un simple \glslink{xoring}{xor}: chaque octet est xoré avec la clef, mais
la clef est incrémentée de 3 après le traitement de chaque octet.

Nous pouvons écrire un petit script Python pour vérifier notre hypothèse:

\lstinputlisting[caption=Python 3.x]{examples/dongles/2/decr1.py}

Et il affiche: \q{check security device connection}.
Donc oui, ceci est le message déchiffré.

Il y a d'autres messages chiffrés, avec leur clef correspondante.
Mais inutile de dire qu'il est possible de les déchiffrer sans leur clef.
Premièrement, nous voyons que le clef est en fait un octet.
C'est parce que l'instruction principale de déchiffrement (\XOR) fonctionne au niveau
de l'octet.
La clef se trouve dans le registre \ESI, mais seulement une partie de \ESI d'un octet
est utilisée.
Ainsi, une clef pourrait être plus grande que 255, mais sa valeur est toujours arrondie.

En conséquence, nous pouvons simplement essayer de brute-forcer, en essayant toutes
les clefs possible dans l'intervalle 0..255.
Nous allons aussi écarter les messages comportants des caractères non-imprimable.

\lstinputlisting[caption=Python 3.x]{examples/dongles/2/decr2.py}

Et nous obtenons:

\lstinputlisting[caption=Results]{examples/dongles/2/decr2_result.txt}

Ici il y a un peu de déchet, mais nous pouvons rapidement trouver les messages en
anglais.

À propos, puisque l'algorithme est un simple chiffrement xor, la même fonction peut
être utilisée pour chiffrer les messages.
Si besoin, nous pouvons chiffrer nos propres messages, et patcher le programme en les insérant.
}


\EN{\mysection{Task manager practical joke (Windows Vista)}
\myindex{Windows!Windows Vista}

Let's see if it's possible to hack Task Manager slightly so it would detect more \ac{CPU} cores.

\myindex{Windows!NTAPI}

Let us first think, how does the Task Manager know the number of cores?

There is the \TT{GetSystemInfo()} win32 function present in win32 userspace which can tell us this.
But it's not imported in \TT{taskmgr.exe}.

There is, however, another one in \gls{NTAPI}, \TT{NtQuerySystemInformation()}, 
which is used in \TT{taskmgr.exe} in several places.

To get the number of cores, one has to call this function with the \TT{SystemBasicInformation} constant
as a first argument (which is zero
\footnote{\href{http://msdn.microsoft.com/en-us/library/windows/desktop/ms724509(v=vs.85).aspx}{MSDN}}).

The second argument has to point to the buffer which is getting all the information.

So we have to find all calls to the \\
\TT{NtQuerySystemInformation(0, ?, ?, ?)} function.
Let's open \TT{taskmgr.exe} in IDA. 
\myindex{Windows!PDB}

What is always good about Microsoft executables is that IDA can download the corresponding \gls{PDB} 
file for this executable and show all function names.

It is visible that Task Manager is written in \Cpp and some of the function names and classes are really 
speaking for themselves.
There are classes CAdapter, CNetPage, CPerfPage, CProcInfo, CProcPage, CSvcPage, 
CTaskPage, CUserPage.

Apparently, each class corresponds to each tab in Task Manager.

Let's visit each call and add comment with the value which is passed as the first function argument.
We will write \q{not zero} at some places, because the value there was clearly not zero, 
but something really different (more about this in the second part of this chapter).

And we are looking for zero passed as argument, after all.

\begin{figure}[H]
\centering
\myincludegraphics{examples/taskmgr/IDA_xrefs.png}
\caption{IDA: cross references to NtQuerySystemInformation()}
\end{figure}

Yes, the names are really speaking for themselves.

When we closely investigate each place where\\
\TT{NtQuerySystemInformation(0, ?, ?, ?)} is called,
we quickly find what we need in the \TT{InitPerfInfo()} function:

\lstinputlisting[caption=taskmgr.exe (Windows Vista),style=customasmx86]{examples/taskmgr/taskmgr.lst}

\TT{g\_cProcessors} is a global variable, and this name has been assigned by 
IDA according to the \gls{PDB} loaded from Microsoft's symbol server.

The byte is taken from \TT{var\_C20}. 
And \TT{var\_C58} is passed to\\
\TT{NtQuerySystemInformation()} 
as a pointer to the receiving buffer.
The difference between 0xC20 and 0xC58 is 0x38 (56).

Let's take a look at format of the return structure, which we can find in MSDN:

\begin{lstlisting}[style=customc]
typedef struct _SYSTEM_BASIC_INFORMATION {
    BYTE Reserved1[24];
    PVOID Reserved2[4];
    CCHAR NumberOfProcessors;
} SYSTEM_BASIC_INFORMATION;
\end{lstlisting}

This is a x64 system, so each PVOID takes 8 bytes.

All \emph{reserved} fields in the structure take $24+4*8=56$ bytes.

Oh yes, this implies that \TT{var\_C20} is the local stack is exactly the
\TT{NumberOfProcessors} field of the \TT{SYSTEM\_BASIC\_INFORMATION} structure.

Let's check our guess.
Copy \TT{taskmgr.exe} from \TT{C:\textbackslash{}Windows\textbackslash{}System32} 
to some other folder 
(so the \emph{Windows Resource Protection} 
will not try to restore the patched \TT{taskmgr.exe}).

Let's open it in Hiew and find the place:

\begin{figure}[H]
\centering
\myincludegraphics{examples/taskmgr/hiew2.png}
\caption{Hiew: find the place to be patched}
\end{figure}

Let's replace the \TT{MOVZX} instruction with ours.
Let's pretend we've got 64 CPU cores.

Add one additional \ac{NOP} (because our instruction is shorter than the original one):

\begin{figure}[H]
\centering
\myincludegraphics{examples/taskmgr/hiew1.png}
\caption{Hiew: patch it}
\end{figure}

And it works!
Of course, the data in the graphs is not correct.

At times, Task Manager even shows an overall CPU load of more than 100\%.

\begin{figure}[H]
\centering
\myincludegraphics{examples/taskmgr/taskmgr_64cpu_crop.png}
\caption{Fooled Windows Task Manager}
\end{figure}

The biggest number Task Manager does not crash with is 64.

Apparently, Task Manager in Windows Vista was not tested on computers with a large number of cores.

So there are probably some static data structure(s) inside it limited to 64 cores.

\subsection{Using LEA to load values}
\label{TaskMgr_LEA}

Sometimes, \TT{LEA} is used in \TT{taskmgr.exe} instead of \TT{MOV} to set the first argument of \\
\TT{NtQuerySystemInformation()}:

\lstinputlisting[caption=taskmgr.exe (Windows Vista),style=customasmx86]{examples/taskmgr/taskmgr2.lst}

\myindex{x86!\Instructions!LEA}

Perhaps \ac{MSVC} did so because machine code of \INS{LEA} is shorter than \INS{MOV REG, 5} (would be 5 instead of 4).

\INS{LEA} with offset in $-128..127$ range (offset will occupy 1 byte in opcode) with 32-bit registers is even shorter (for lack of REX prefix)---3 bytes.

Another example of such thing is: \myref{using_MOV_and_pack_of_LEA_to_load_values}.
}%
\RU{\subsection{Обменять входные значения друг с другом}

Вот так:

\lstinputlisting[style=customc]{patterns/061_pointers/swap/5_RU.c}

Как видим, байты загружаются в младшие 8-битные части регистров \TT{ECX} и \TT{EBX} используя \INS{MOVZX}
(так что старшие части регистров очищаются), затем байты записываются назад в другом порядке.

\lstinputlisting[style=customasmx86,caption=Optimizing GCC 5.4]{patterns/061_pointers/swap/5_GCC_O3_x86.s}

Адреса обоих байтов берутся из аргументов и во время исполнения ф-ции находятся в регистрах \TT{EDX} и \TT{EAX}.

Так что исопльзуем указатели --- вероятно, без них нет способа решить эту задачу лучше.

}%
\FR{\subsection{Exemple \#2: SCO OpenServer}

\label{examples_SCO}
\myindex{SCO OpenServer}
Un ancien logiciel pour SCO OpenServer de 1997 développé par une société qui a disparue
depuis longtemps.

Il y a un driver de dongle special à installer dans le système, qui contient les
chaînes de texte suivantes:
\q{Copyright 1989, Rainbow Technologies, Inc., Irvine, CA}
et
\q{Sentinel Integrated Driver Ver. 3.0 }.

Après l'installation du driver dans SCO OpenServer, ces fichiers apparaissent dans
l'arborescence /dev:

\begin{lstlisting}
/dev/rbsl8
/dev/rbsl9
/dev/rbsl10
\end{lstlisting}

Le programme renvoie une erreur lorsque le dongle n'est pas connecté, mais le message
d'erreur n'est pas trouvé dans les exécutables.

\myindex{COFF}

Grâce à \ac{IDA}, il est facile de charger l'exécutable COFF utilisé dans SCO OpenServer.

Essayons de trouver la chaîne \q{rbsl} et en effet, elle se trouve dans ce morceau
de code:

\lstinputlisting[style=customasmx86]{examples/dongles/2/1.lst}

Oui, en effet, le programme doit communiquer d'une façon ou d'une autre avec le driver.

\myindex{thunk-functions}
Le seul endroit où la fonction \TT{SSQC()} est appelée est dans la \glslink{thunk
 function}{fonction thunk}:

\lstinputlisting[style=customasmx86]{examples/dongles/2/2.lst}

SSQ() peut être appelé depuis au moins 2 fonctions.

L'une d'entre elles est:

\lstinputlisting[style=customasmx86]{examples/dongles/2/check1_EN.lst}

\q{\TT{3C}} et \q{\TT{3E}} semblent familiers: il y avait un dongle Sentinel Pro de
Rainbow sans mémoire, fournissant seulement une fonction de crypto-hachage secrète.

Vous pouvez lire une courte description de la fonction de hachage dont il s'agit
ici: \myref{hash_func}.

Mais retournons au programme.

Donc le programme peut seulement tester si un dongle est connecté ou s'il est absent.

Aucune autre information ne peut être écrite dans un tel dongle, puisqu'il n'a pas
de mémoire.
Les codes sur deux caractères sont des commandes (nous pouvons voir comment les commandes
sont traitées dans la fonction \TT{SSQC()}) et toutes les autres chaînes sont hachées
dans le dongle, transformées en un nombre 16-bit.
L'algorithme était secret, donc il n'était pas possible d'écrire un driver de remplacement
ou de refaire un dongle matériel qui l'émulerait parfaitement.

Toutefois, il est toujours possible d'intercepter tous les accès au dongle et de
trouver les constantes auxquelles les résultats de la fonction de hachage sont comparées.

Mais nous devons dire qu'il est possible de construire un schéma de logiciel de protection
de copie robuste basé sur une fonction secrète de hachage cryptographique: il suffit
qu'elle chiffre/déchiffre les fichiers de données utilisés par votre logiciel.

Mais retournons au code:

Les codes 51/52/53 sont utilisés pour choisir le port imprimante LPT.
3x/4x sont utilisés pour le choix de la \q{famille} (c'est ainsi que les dongles
Sentinel Pro sont différenciés les uns des autres: plus d'un dongle peut être connecté
sur un port LPT).

La seule chaîne passée à la fonction qui ne fasse pas 2 caractères est "0123456789".

Ensuite, le résultat est comparé à l'ensemble des résultats valides.

Si il est correct, 0xC ou 0xB est écrit dans la variable globale \TT{ctl\_model}.%

Une autre chaîne de texte qui est passée est
"PRESS ANY KEY TO CONTINUE: ", mais le résultat n'est pas testé.
Difficile de dire pourquoi, probablement une erreur\footnote{C'est un sentiment
étrange de trouver un bug dans un logiciel aussi ancien.}.

Voyons où la valeur de la variable globale \TT{ctl\_model} est utilisée.

Un tel endroit est:

\lstinputlisting[style=customasmx86]{examples/dongles/2/4.lst}

Si c'est 0, un message d'erreur chiffré est passé à une routine de déchiffrement
et affiché.

\myindex{x86!\Instructions!XOR}

La routine de déchiffrement de la chaîne semble être un simple \glslink{xoring}{xor}:

\lstinputlisting[style=customasmx86]{examples/dongles/2/err_warn.lst}

C'est pourquoi nous étions incapable de trouver le message d'erreur dans les fichiers
exécutable, car ils sont chiffrés (ce qui est une pratique courante).

Un autre appel à la fonction de hachage \TT{SSQ()} lui passe la chaîne \q{offln}
et le résultat est comparé avec \TT{0xFE81} et \TT{0x12A9}.

Si ils ne correspondent pas, ça se comporte comme une sorte de fonction \TT{timer()}
(peut-être en attente qu'un dongle mal connecté soit reconnecté et re-testé?) et ensuite
déchiffre un autre message d'erreur à afficher.

\lstinputlisting[style=customasmx86]{examples/dongles/2/check2_EN.lst}

Passer outre le dongle est assez facile: il suffit de patcher tous les sauts après
les instructions \CMP pertinentes.

Une autre option est d'écrire notre propre driver SCO OpenServer, contenant une table
de questions et de réponses, toutes celles qui sont présentent dans le programme.

\subsubsection{Déchiffrer les messages d'erreur}

À propos, nous pouvons aussi essayer de déchiffrer tous les messages d'erreurs.
L'algorithme qui se trouve dans la fonction \TT{err\_warn()} est très simple, en effet:

\lstinputlisting[caption=Decryption function,style=customasmx86]{examples/dongles/2/decrypting_FR.lst}

Comme on le voit, non seulement la chaîne est transmise à la fonction de déchiffrement
mais aussi la clef:

\lstinputlisting[style=customasmx86]{examples/dongles/2/tmp1_EN.asm}

L'algorithme est un simple \glslink{xoring}{xor}: chaque octet est xoré avec la clef, mais
la clef est incrémentée de 3 après le traitement de chaque octet.

Nous pouvons écrire un petit script Python pour vérifier notre hypothèse:

\lstinputlisting[caption=Python 3.x]{examples/dongles/2/decr1.py}

Et il affiche: \q{check security device connection}.
Donc oui, ceci est le message déchiffré.

Il y a d'autres messages chiffrés, avec leur clef correspondante.
Mais inutile de dire qu'il est possible de les déchiffrer sans leur clef.
Premièrement, nous voyons que le clef est en fait un octet.
C'est parce que l'instruction principale de déchiffrement (\XOR) fonctionne au niveau
de l'octet.
La clef se trouve dans le registre \ESI, mais seulement une partie de \ESI d'un octet
est utilisée.
Ainsi, une clef pourrait être plus grande que 255, mais sa valeur est toujours arrondie.

En conséquence, nous pouvons simplement essayer de brute-forcer, en essayant toutes
les clefs possible dans l'intervalle 0..255.
Nous allons aussi écarter les messages comportants des caractères non-imprimable.

\lstinputlisting[caption=Python 3.x]{examples/dongles/2/decr2.py}

Et nous obtenons:

\lstinputlisting[caption=Results]{examples/dongles/2/decr2_result.txt}

Ici il y a un peu de déchet, mais nous pouvons rapidement trouver les messages en
anglais.

À propos, puisque l'algorithme est un simple chiffrement xor, la même fonction peut
être utilisée pour chiffrer les messages.
Si besoin, nous pouvons chiffrer nos propres messages, et patcher le programme en les insérant.
}



\EN{\subsubsection{Reading outside array bounds}

So, array indexing is just \emph{array\lbrack{}index\rbrack}.
If you study the generated code closely, you'll probably note the missing index bounds checking,
which could check \emph{if it is less than 20}.
What if the index is 20 or greater?
That's the one \CCpp feature it is often blamed for.

Here is a code that successfully compiles and works:

\lstinputlisting[style=customc]{patterns/13_arrays/2_BO/r.c}

Compilation results (MSVC 2008):

\lstinputlisting[caption=\NonOptimizing MSVC 2008,style=customasmx86]{patterns/13_arrays/2_BO/r_msvc.asm}

The code produced this result:

\lstinputlisting[caption=\olly: console output]{patterns/13_arrays/2_BO/console.txt}

It is just \emph{something} that has been lying in the stack near to the array, 80 bytes away from its first element.

\clearpage
\myindex{\olly}
Let's try to find out where did this value come from, using \olly.

Let's load and find the value located right after the last array element:

\begin{figure}[H]
\centering
\myincludegraphics{patterns/13_arrays/2_BO/olly_r1.png}
\caption{\olly: reading of the 20th element and execution of \printf}
\label{fig:array_BO_olly_r1}
\end{figure}

What is this? 
Judging by the stack layout,
this is the saved value of the EBP register.
\clearpage
Let's trace further and see how it gets restored:

\begin{figure}[H]
\centering
\myincludegraphics{patterns/13_arrays/2_BO/olly_r2.png}
\caption{\olly: restoring value of EBP}
\label{fig:array_BO_olly_r2}
\end{figure}

Indeed, how it could be different?
The compiler may generate some additional code to check the index value to be always
in the array's bounds (like in higher-level programming languages\footnote{Java, Python, etc.})
but this makes the code slower.

}
\ES{% TODO resync with EN version
\chapter{Libros/blogs que merecen lectura}

\mysection{Libros}

\subsection{Reverse Engineering}

\begin{itemize}
\item Eldad Eilam, \emph{Reversing: Secrets of Reverse Engineering}, (2005)

\item Bruce Dang, Alexandre Gazet, Elias Bachaalany, Sebastien Josse, \emph{Practical Reverse Engineering: x86, x64, ARM, Windows Kernel, Reversing Tools, and Obfuscation}, (2014)

\item Michael Sikorski, Andrew Honig, \emph{Practical Malware Analysis: The Hands-On Guide to Dissecting Malicious Software}, (2012)

\item Chris Eagle, \emph{IDA Pro Book}, (2011)

\item Reginald Wong, \emph{Mastering Reverse Engineering: Re-engineer your ethical hacking skills}, (2018)

\end{itemize}


% TBT

\subsection{Windows}

\begin{itemize}
\item \Russinovich
\item Peter Ferrie -- The ``Ultimate'' Anti-Debugging Reference\footnote\url{http://pferrie.host22.com/papers/antidebug.pdf}}
\end{itemize}

\EN{Blogs}\ES{Blogs}\RU{Блоги}\FR{Blogs}\DE{Blogs}\PL{Blogi}:

\begin{itemize}
\item \href{http://blogs.msdn.com/oldnewthing/}{Microsoft: Raymond Chen}
\item \href{http://www.nynaeve.net/}{nynaeve.net}
\end{itemize}



\subsection{\CCpp}

\label{CCppBooks}

\begin{itemize}

\item \KRBook

\item \CNineNineStd\footnote{\AlsoAvailableAs \url{http://www.open-std.org/jtc1/sc22/WG14/www/docs/n1256.pdf}}

\item \TCPPPL

\item \CppOneOneStd\footnote{\AlsoAvailableAs \url{http://www.open-std.org/jtc1/sc22/wg21/docs/papers/2013/n3690.pdf}.}

\item \AgnerFogCPP\footnote{\AlsoAvailableAs \url{http://agner.org/optimize/optimizing_cpp.pdf}.}

\item \ParashiftCPPFAQ\footnote{\AlsoAvailableAs \url{http://www.parashift.com/c++-faq-lite/index.html}}

\item \CNotes\footnote{\AlsoAvailableAs \url{http://yurichev.com/C-book.html}}

\item JPL Institutional Coding Standard for the C Programming Language\footnote{\AlsoAvailableAs \url{https://yurichev.com/mirrors/C/JPL_Coding_Standard_C.pdf}}

\RU{\item Евгений Зуев --- Редкая профессия\footnote{\AlsoAvailableAs \url{https://yurichev.com/mirrors/C++/Redkaya_professiya.pdf}}}

\end{itemize}



\label{x86_manuals}
\begin{itemize}
\item Intel manuals\footnote{\AlsoAvailableAs \url{http://www.intel.com/content/www/us/en/processors/architectures-software-developer-manuals.html}}

\item AMD manuals\footnote{\AlsoAvailableAs \url{http://developer.amd.com/resources/developer-guides-manuals/}}

\item \AgnerFog{}\footnote{\AlsoAvailableAs \url{http://agner.org/optimize/microarchitecture.pdf}}

\item \AgnerFogCC{}\footnote{\AlsoAvailableAs \url{http://www.agner.org/optimize/calling_conventions.pdf}}

\item \IntelOptimization

\item \AMDOptimization
\end{itemize}

\subsection{ARM}

\begin{itemize}
\item Manuales de ARM\footnote{\AlsoAvailableAs \url{http://infocenter.arm.com/help/index.jsp?topic=/com.arm.doc.subset.architecture.reference/index.html}}

\item \ARMSevenRef

\item \ARMSixFourRefURL

\item \ARMCookBook\footnote{\AlsoAvailableAs \url{https://yurichev.com/ref/ARM%20Cookbook%20(1994)/}}
\end{itemize}

% TBT

\subsection{Java}

\JavaBook.

\subsection{UNIX}

\TAOUP

% subsection:
\subsection{\EN{Cryptography}\ES{Criptograf\'ia}\IT{Crittografia}\RU{Криптография}\FR{Cryptographie}\DE{Kryptografie}\JA{暗号学}}
\label{crypto_books}

\begin{itemize}
\item \Schneier{}

\item (Free) lvh, \emph{Crypto 101}\footnote{\AlsoAvailableAs \url{https://www.crypto101.io/}}

\item (Free) Dan Boneh, Victor Shoup, \emph{A Graduate Course in Applied Cryptography}\footnote{\AlsoAvailableAs \url{https://crypto.stanford.edu/~dabo/cryptobook/}}.
\end{itemize}



\mysection{Otros}

\HenryWarren.

Existen dos excelentes subreddits relacionados con \ac{RE} en reddit.com:
\href{http://www.reddit.com/r/ReverseEngineering/}{reddit.com/r/ReverseEngineering/} \ESph{}
\href{http://www.reddit.com/r/remath}{reddit.com/r/remath}
(en los t\'opicos de la intersecci\'on de \ac{RE} y matem\'aticas).

Tambi\'en hay una secci\'on sobre \ac{RE} en el sitio web de Stack Exchange:

\par
\href{http://reverseengineering.stackexchange.com/}{reverseengineering.stackexchange.com}.

En IRC hay un canal \#\#re en
FreeNode\footnote{\href{https://freenode.net/}{freenode.net}}.

% TBT
}
\RU{% TODO sync with English version
\chapter{Что стоит почитать}

\mysection{Книги и прочие материалы}

\subsection{Reverse Engineering}

\begin{itemize}
\item Eldad Eilam, \emph{Reversing: Secrets of Reverse Engineering}, (2005)

\item Bruce Dang, Alexandre Gazet, Elias Bachaalany, Sebastien Josse, \emph{Practical Reverse Engineering: x86, x64, ARM, Windows Kernel, Reversing Tools, and Obfuscation}, (2014)

\item Michael Sikorski, Andrew Honig, \emph{Practical Malware Analysis: The Hands-On Guide to Dissecting Malicious Software}, (2012)

\item Chris Eagle, \emph{IDA Pro Book}, (2011)

\item Reginald Wong, \emph{Mastering Reverse Engineering: Re-engineer your ethical hacking skills}, (2018)

\end{itemize}


(Старое, но всё равно интересное) Pavol Cerven, \emph{Crackproof Your Software: Protect Your Software Against Crackers}, (2002).

Дмитрий Скляров --- ``Искусство защиты и взлома информации''.

Также, книги Криса Касперски.

\subsection{Windows}

\begin{itemize}
\item \Russinovich
\item Peter Ferrie -- The ``Ultimate'' Anti-Debugging Reference\footnote\url{http://pferrie.host22.com/papers/antidebug.pdf}}
\end{itemize}

\EN{Blogs}\ES{Blogs}\RU{Блоги}\FR{Blogs}\DE{Blogs}\PL{Blogi}:

\begin{itemize}
\item \href{http://blogs.msdn.com/oldnewthing/}{Microsoft: Raymond Chen}
\item \href{http://www.nynaeve.net/}{nynaeve.net}
\end{itemize}



\subsection{\CCpp}

\label{CCppBooks}

\begin{itemize}

\item \KRBook

\item \CNineNineStd\footnote{\AlsoAvailableAs \url{http://www.open-std.org/jtc1/sc22/WG14/www/docs/n1256.pdf}}

\item \TCPPPL

\item \CppOneOneStd\footnote{\AlsoAvailableAs \url{http://www.open-std.org/jtc1/sc22/wg21/docs/papers/2013/n3690.pdf}.}

\item \AgnerFogCPP\footnote{\AlsoAvailableAs \url{http://agner.org/optimize/optimizing_cpp.pdf}.}

\item \ParashiftCPPFAQ\footnote{\AlsoAvailableAs \url{http://www.parashift.com/c++-faq-lite/index.html}}

\item \CNotes\footnote{\AlsoAvailableAs \url{http://yurichev.com/C-book.html}}

\item JPL Institutional Coding Standard for the C Programming Language\footnote{\AlsoAvailableAs \url{https://yurichev.com/mirrors/C/JPL_Coding_Standard_C.pdf}}

\RU{\item Евгений Зуев --- Редкая профессия\footnote{\AlsoAvailableAs \url{https://yurichev.com/mirrors/C++/Redkaya_professiya.pdf}}}

\end{itemize}



\subsection{x86 / x86-64}

\label{x86_manuals}
\begin{itemize}
\item Документация от Intel\footnote{\AlsoAvailableAs \url{http://www.intel.com/content/www/us/en/processors/architectures-software-developer-manuals.html}}

\item Документация от AMD\footnote{\AlsoAvailableAs \url{http://developer.amd.com/resources/developer-guides-manuals/}}

\item \AgnerFog{}\footnote{\AlsoAvailableAs \url{http://agner.org/optimize/microarchitecture.pdf}}

\item \AgnerFogCC{}\footnote{\AlsoAvailableAs \url{http://www.agner.org/optimize/calling_conventions.pdf}}

\item \IntelOptimization

\item \AMDOptimization
\end{itemize}

Немного устарело, но всё равно интересно почитать:

\MAbrash\footnote{\AlsoAvailableAs \url{https://github.com/jagregory/abrash-black-book}}
(он известен своей работой над низкоуровневой оптимизацией в таких проектах как Windows NT 3.1 и id Quake).

\subsection{ARM}

\begin{itemize}
\item Документация от ARM\footnote{\AlsoAvailableAs \url{http://infocenter.arm.com/help/index.jsp?topic=/com.arm.doc.subset.architecture.reference/index.html}}

\item \ARMSevenRef

\item \ARMSixFourRefURL

\item \ARMCookBook\footnote{\AlsoAvailableAs \url{https://yurichev.com/ref/ARM%20Cookbook%20(1994)/}}
\end{itemize}

\subsection{Язык ассемблера}

Richard Blum --- Professional Assembly Language.

\subsection{Java}

\JavaBook.

\subsection{UNIX}

\TAOUP

\subsection{Программирование}

\begin{itemize}

\item \RobPikePractice

\item Александр Шень\footnote{\url{http://imperium.lenin.ru/~verbit/Shen.dir/shen-progra.html}}

\item \HenryWarren.
Некоторые люди говорят, что трюки и хаки из этой книги уже не нужны, потому что годились только для \ac{RISC}-процессоров,
где инструкции перехода слишком дорогие.
Тем не менее, всё это здорово помогает лучше понять булеву алгебру и всю математику рядом.

\end{itemize}

% subsection:
\subsection{\EN{Cryptography}\ES{Criptograf\'ia}\IT{Crittografia}\RU{Криптография}\FR{Cryptographie}\DE{Kryptografie}\JA{暗号学}}
\label{crypto_books}

\begin{itemize}
\item \Schneier{}

\item (Free) lvh, \emph{Crypto 101}\footnote{\AlsoAvailableAs \url{https://www.crypto101.io/}}

\item (Free) Dan Boneh, Victor Shoup, \emph{A Graduate Course in Applied Cryptography}\footnote{\AlsoAvailableAs \url{https://crypto.stanford.edu/~dabo/cryptobook/}}.
\end{itemize}



\subsection{Что-то попроще}

Для тех кто находит эту книгу слишком трудной и технической,
вот еще более легкое введение в низкоуровневые внутренности компьютерных устройств:
Чарльз Петцольд -- ``Код: тайный язык информатики'' (также переведена на русский).

И еще книга комиксов (1983-й год) для детей\footnote{\url{https://yurichev.com/mirrors/machine-code-for-beginners.pdf}},
посвященная процессорам 6502 и Z80.

}
\FR{\subsubsection{Lire en dehors des bornes du tableau}

Donc, indexer un tableau est juste \emph{array\lbrack{}index\rbrack}.
Si vous étudiez le code généré avec soin, vous remarquerez sans doute l'absence de
test sur les bornes de l'index, qui devrait vérifier \emph{si il est inférieur à 20}.
Que ce passe-t-il si l'index est supérieur à 20?
C'est une des caractéristiques de \CCpp qui est souvent critiquée.

Voici un code qui compile et fonctionne:

\lstinputlisting[style=customc]{patterns/13_arrays/2_BO/r.c}

Résultat de la compilation (MSVC 2008):

\lstinputlisting[caption=MSVC 2008 \NonOptimizing,style=customasmx86]{patterns/13_arrays/2_BO/r_msvc.asm}

Le code produit ce résultat:

\lstinputlisting[caption=\olly: sortie sur la console]{patterns/13_arrays/2_BO/console.txt}

C'est juste \emph{quelque chose} qui se trouvait sur la pile à côté du tableau, 80 octets
après le début de son premier élément.

\clearpage
\myindex{\olly}
Essayons de trouver d'où vient cette valeur, en utilisant \olly.

Chargeons et trouvons la valeur située juste après le dernier élément du tableau:

\begin{figure}[H]
\centering
\myincludegraphics{patterns/13_arrays/2_BO/olly_r1.png}
\caption{\olly: lecture du 20ème élément et exécution de \printf}
\label{fig:array_BO_olly_r1}
\end{figure}

Qu'est-ce que c'est?
D'après le schéma de la pile, c'est la valeur sauvegardée du registre EBP.
\clearpage
Exécutons encore et voyons comment il est restauré:

\begin{figure}[H]
\centering
\myincludegraphics{patterns/13_arrays/2_BO/olly_r2.png}
\caption{\olly: restaurer la valeur de EBP}
\label{fig:array_BO_olly_r2}
\end{figure}

En effet, comment est-ce ça pourrait être différent?
Le compilateur pourrait générer du code supplémentaire pour vérifier que la valeur
de l'index est toujours entre les bornes du tableau (comme dans les langages de
programmation de plus haut-niveau\footnote{Java, Python, etc.}) mais cela rendrait
le code plus lent.

}
\DE{\subsubsection{Lesezugriff außerhalb von Arraygrenzen}
Der indizierte Zugriff auf ein Array wird durch \emph{array\lbrack{}index\rbrack} realisiert.
Wenn man sich den erzeugten Code genau ansieht, bemerkt man, dass eine Prüfung der Indexgrenzen fehlt, welche die
Bedingung \emph{kleiner als 20} validiert.
Was also passiert, wenn der Index 20 oder größer ist? 
Hier haben wir es mit einem unschönen Feature von \CCpp zu tun

Hier ein Beipsielcode der erfolgreich kompiliert wurde und funktioniert:

\lstinputlisting[style=customc]{patterns/13_arrays/2_BO/r.c}

Ergebnis des Kompiliervorgangs (MSVC 2008):

\lstinputlisting[caption=\NonOptimizing MSVC 2008,style=customasmx86]{patterns/13_arrays/2_BO/r_msvc.asm}

Der Code produziert dieses Ergebnis:

\lstinputlisting[caption=\olly: console output]{patterns/13_arrays/2_BO/console.txt}
Es handelt sich um \emph{irgendetwas}, das auf dem Stack in der Nähe des Arrays gelegen hat, 80 Byte von dessen erstem
Element entfernt.

\clearpage
\myindex{\olly}
Versuchen wir mit \olly herauszufinden, woher dieser Wert kommt.

Laden und finden wir also den Wert, der sich direkt hinter dem letzten Arrayelement befindet:

\begin{figure}[H]
\centering
\myincludegraphics{patterns/13_arrays/2_BO/olly_r1.png}
\caption{\olly: das 20. Element lesen und \printf ausführen}
\label{fig:array_BO_olly_r1}
\end{figure}

Worum handelt es sich? 
Dem Stacklayout nach zu urteilen ist dies der gespeicherte Wert des EBP Registers.
\clearpage
Verfolgen wir das ganze weiter und schauen uns an, wie dieser wiederhergestellt wird:

\begin{figure}[H]
\centering
\myincludegraphics{patterns/13_arrays/2_BO/olly_r2.png}
\caption{\olly: Wert von EBP wiederherstellen}
\label{fig:array_BO_olly_r2}
\end{figure}
Wie könnte es anders gelöst werden?
Der Compiler könnte zusätzlichen Code erzeugen, der sicherstellt, dass der Index sich stets innerhalb der Arraygrenzen
befindet (wie in höheren Programmiersprachen\footnote{Java, Python, etc.}), aber das würde den Code langsamer machen.
}
\IT{\chapter{Libri/blog da leggere}

\mysection{Libri ed altro materiale}

\subsection{Reverse Engineering}

\begin{itemize}
\item Eldad Eilam, \emph{Reversing: Secrets of Reverse Engineering}, (2005)

\item Bruce Dang, Alexandre Gazet, Elias Bachaalany, Sebastien Josse, \emph{Practical Reverse Engineering: x86, x64, ARM, Windows Kernel, Reversing Tools, and Obfuscation}, (2014)

\item Michael Sikorski, Andrew Honig, \emph{Practical Malware Analysis: The Hands-On Guide to Dissecting Malicious Software}, (2012)

\item Chris Eagle, \emph{IDA Pro Book}, (2011)

\item Reginald Wong, \emph{Mastering Reverse Engineering: Re-engineer your ethical hacking skills}, (2018)

\end{itemize}


% TBT
% (Outdated, but still interesting) Pavol Cerven, \emph{Crackproof Your Software: Protect Your Software Against Crackers}, (2002).

Inoltre, i libri di Kris Kaspersky.

\subsection{Windows}

\begin{itemize}
\item \Russinovich
\item Peter Ferrie -- The ``Ultimate'' Anti-Debugging Reference\footnote\url{http://pferrie.host22.com/papers/antidebug.pdf}}
\end{itemize}

\EN{Blogs}\ES{Blogs}\RU{Блоги}\FR{Blogs}\DE{Blogs}\PL{Blogi}:

\begin{itemize}
\item \href{http://blogs.msdn.com/oldnewthing/}{Microsoft: Raymond Chen}
\item \href{http://www.nynaeve.net/}{nynaeve.net}
\end{itemize}



\subsection{\CCpp}

\label{CCppBooks}

\begin{itemize}

\item \KRBook

\item \CNineNineStd\footnote{\AlsoAvailableAs \url{http://www.open-std.org/jtc1/sc22/WG14/www/docs/n1256.pdf}}

\item \TCPPPL

\item \CppOneOneStd\footnote{\AlsoAvailableAs \url{http://www.open-std.org/jtc1/sc22/wg21/docs/papers/2013/n3690.pdf}.}

\item \AgnerFogCPP\footnote{\AlsoAvailableAs \url{http://agner.org/optimize/optimizing_cpp.pdf}.}

\item \ParashiftCPPFAQ\footnote{\AlsoAvailableAs \url{http://www.parashift.com/c++-faq-lite/index.html}}

\item \CNotes\footnote{\AlsoAvailableAs \url{http://yurichev.com/C-book.html}}

\item JPL Institutional Coding Standard for the C Programming Language\footnote{\AlsoAvailableAs \url{https://yurichev.com/mirrors/C/JPL_Coding_Standard_C.pdf}}

\RU{\item Евгений Зуев --- Редкая профессия\footnote{\AlsoAvailableAs \url{https://yurichev.com/mirrors/C++/Redkaya_professiya.pdf}}}

\end{itemize}



\subsection{x86 / x86-64}

\label{x86_manuals}
\begin{itemize}
\item Manuali Intel\footnote{\AlsoAvailableAs \url{http://www.intel.com/content/www/us/en/processors/architectures-software-developer-manuals.html}}

\item Manuali AMD\footnote{\AlsoAvailableAs \url{http://developer.amd.com/resources/developer-guides-manuals/}}

\item \AgnerFog{}\footnote{\AlsoAvailableAs \url{http://agner.org/optimize/microarchitecture.pdf}}

\item \AgnerFogCC{}\footnote{\AlsoAvailableAs \url{http://www.agner.org/optimize/calling_conventions.pdf}}

\item \IntelOptimization

\item \AMDOptimization
\end{itemize}

Un po' datati ma sempre interessanti:

\MAbrash\footnote{\AlsoAvailableAs \url{https://github.com/jagregory/abrash-black-book}}
(è conosciuto per i suoi lavori di ottimizzazione a basso livello su progetti come Windows NT 3.1 e id Quake).

\subsection{ARM}

\begin{itemize}
\item Manuali ARM\footnote{\AlsoAvailableAs \url{http://infocenter.arm.com/help/index.jsp?topic=/com.arm.doc.subset.architecture.reference/index.html}}

\item \ARMSevenRef

\item \ARMSixFourRefURL

\item \ARMCookBook\footnote{\AlsoAvailableAs \url{https://yurichev.com/ref/ARM%20Cookbook%20(1994)/}}
\end{itemize}

\subsection{Assembly}

Richard Blum --- Professional Assembly Language.

\subsection{Java}

\JavaBook.

\subsection{UNIX}

\TAOUP

\subsection{Programmazione in generale}

\begin{itemize}

\item \RobPikePractice

\item \HenryWarren.
Alcune persone sostengono che i trucchi e gli hack di questo libro non siano più attuali adesso perchè erano validi solo per le \ac{CPU} \ac{RISC},
dove le istruzioni di branching sono costose.
Ad ogni modo, possono aiutare enormemente a comprendere l'algebra booleana e tutta la matematica coinvolta.

\end{itemize}

% subsection:
\subsection{\EN{Cryptography}\ES{Criptograf\'ia}\IT{Crittografia}\RU{Криптография}\FR{Cryptographie}\DE{Kryptografie}\JA{暗号学}}
\label{crypto_books}

\begin{itemize}
\item \Schneier{}

\item (Free) lvh, \emph{Crypto 101}\footnote{\AlsoAvailableAs \url{https://www.crypto101.io/}}

\item (Free) Dan Boneh, Victor Shoup, \emph{A Graduate Course in Applied Cryptography}\footnote{\AlsoAvailableAs \url{https://crypto.stanford.edu/~dabo/cryptobook/}}.
\end{itemize}



% TBT
}
\JA{\subsubsection{配列の範囲外の読み込み}

配列のインデックス化は単に\emph{array\lbrack{}index\rbrack}です。
生成されたコードを詳しく研究したなら、\emph{20未満であるか}チェックするような
インデックスの境界チェックがないことに気づくでしょう。
もしインデックスが20以上だったらどうでしょうか。
これは \CCpp が批判される1つの特徴です。

コンパイルされて動作するコードがあります。

\lstinputlisting[style=customc]{patterns/13_arrays/2_BO/r.c}

コンパイル結果(MSVC 2008)

\lstinputlisting[caption=\NonOptimizing MSVC 2008,style=customasmx86]{patterns/13_arrays/2_BO/r_msvc.asm}

コードは次の結果を生成します。

\lstinputlisting[caption=\olly: console output]{patterns/13_arrays/2_BO/console.txt}

これは単に配列のそばのスタックにある \emph{何か} です。配列の最初の要素から80バイト離れています。

\clearpage
\myindex{\olly}
この値がどこから来るのか \olly を使って見つけてみましょう。

最後の配列の要素のすぐあとに配置された値をロードして見つけましょう。

\begin{figure}[H]
\centering
\myincludegraphics{patterns/13_arrays/2_BO/olly_r1.png}
\caption{\olly: 20番目の要素を読み込み、 \printf を実行する}
\label{fig:array_BO_olly_r1}
\end{figure}

これは何でしょうか?
スタックレイアウトで判断すると、
これは保存されたEBPレジスタの値です。
\clearpage
もっとトレースしてどのようにリストアされるか見てみましょう。

\begin{figure}[H]
\centering
\myincludegraphics{patterns/13_arrays/2_BO/olly_r2.png}
\caption{\olly: EBPの値をリストア}
\label{fig:array_BO_olly_r2}
\end{figure}

本当に、異なっていますか?
コンパイラはインデックス値が配列の境界内かを常にチェックする追加のコードを生成するかもしれません。
(高水準プログラミング言語\footnote{Java, Pythonなど}のように)
しかし、これはコードを遅くします。
}
\PL{\chapter{Książki/blogi warte przeczytania}

\mysection{Książki i inne materiały}

\subsection{Inżynieria wsteczna}

\begin{itemize}
\item Eldad Eilam, \emph{Reversing: Secrets of Reverse Engineering}, (2005)

\item Bruce Dang, Alexandre Gazet, Elias Bachaalany, Sebastien Josse, \emph{Practical Reverse Engineering: x86, x64, ARM, Windows Kernel, Reversing Tools, and Obfuscation}, (2014)

\item Michael Sikorski, Andrew Honig, \emph{Practical Malware Analysis: The Hands-On Guide to Dissecting Malicious Software}, (2012)

\item Chris Eagle, \emph{IDA Pro Book}, (2011)

\item Reginald Wong, \emph{Mastering Reverse Engineering: Re-engineer your ethical hacking skills}, (2018)

\end{itemize}


(Stara, ale wciąż interesująca) Pavol Cerven, \emph{Crackproof Your Software: Protect Your Software Against Crackers}, (2002).

Oraz książki Krisa Kaspersky'ego.

\subsection{Windows}

\begin{itemize}
\item \Russinovich
\item Peter Ferrie -- The ``Ultimate'' Anti-Debugging Reference\footnote\url{http://pferrie.host22.com/papers/antidebug.pdf}}
\end{itemize}

\EN{Blogs}\ES{Blogs}\RU{Блоги}\FR{Blogs}\DE{Blogs}\PL{Blogi}:

\begin{itemize}
\item \href{http://blogs.msdn.com/oldnewthing/}{Microsoft: Raymond Chen}
\item \href{http://www.nynaeve.net/}{nynaeve.net}
\end{itemize}



\subsection{\CCpp}

\label{CCppBooks}

\begin{itemize}

\item \KRBook

\item \CNineNineStd\footnote{\AlsoAvailableAs \url{http://www.open-std.org/jtc1/sc22/WG14/www/docs/n1256.pdf}}

\item \TCPPPL

\item \CppOneOneStd\footnote{\AlsoAvailableAs \url{http://www.open-std.org/jtc1/sc22/wg21/docs/papers/2013/n3690.pdf}.}

\item \AgnerFogCPP\footnote{\AlsoAvailableAs \url{http://agner.org/optimize/optimizing_cpp.pdf}.}

\item \ParashiftCPPFAQ\footnote{\AlsoAvailableAs \url{http://www.parashift.com/c++-faq-lite/index.html}}

\item \CNotes\footnote{\AlsoAvailableAs \url{http://yurichev.com/C-book.html}}

\item JPL Institutional Coding Standard for the C Programming Language\footnote{\AlsoAvailableAs \url{https://yurichev.com/mirrors/C/JPL_Coding_Standard_C.pdf}}

\RU{\item Евгений Зуев --- Редкая профессия\footnote{\AlsoAvailableAs \url{https://yurichev.com/mirrors/C++/Redkaya_professiya.pdf}}}

\end{itemize}



\subsection{x86 / x86-64}

\label{x86_manuals}
\begin{itemize}
\item manuale Intela\footnote{\AlsoAvailableAs \url{http://www.intel.com/content/www/us/en/processors/architectures-software-developer-manuals.html}}

\item manuale AMD \footnote{\AlsoAvailableAs \url{http://developer.amd.com/resources/developer-guides-manuals/}}

\item \AgnerFog{}\footnote{\AlsoAvailableAs \url{http://agner.org/optimize/microarchitecture.pdf}}

\item \AgnerFogCC{}\footnote{\AlsoAvailableAs \url{http://www.agner.org/optimize/calling_conventions.pdf}}

\item \IntelOptimization

\item \AMDOptimization
\end{itemize}

Trochę stare, ale wciąż interesujące:

\MAbrash\footnote{\AlsoAvailableAs \url{https://github.com/jagregory/abrash-black-book}}
(znany z pracy nad niskopoziomową optymalizacją w takich projektach jak Windows NT 3.1 i id Quake).

\subsection{ARM}

\begin{itemize}
\item manuale ARM\footnote{\AlsoAvailableAs \url{http://infocenter.arm.com/help/index.jsp?topic=/com.arm.doc.subset.architecture.reference/index.html}}

\item \ARMSevenRef

\item \ARMSixFourRefURL

\item \ARMCookBook\footnote{\AlsoAvailableAs \url{https://yurichev.com/ref/ARM%20Cookbook%20(1994)/}}
\end{itemize}

\subsection{Język maszynowy}

Richard Blum --- Professional Assembly Language.

\subsection{Java}

\JavaBook.

\subsection{UNIX}

\TAOUP

\subsection{Programowanie}

\begin{itemize}

\item \RobPikePractice

\item \HenryWarren.
Niektórzy twierdzą, że sztuczki z tej książki nie mają dzisiaj znaczenia, ponieważ miały zastosowanie wyłącznie w procesorach \ac{RISC},
gdzie instrukcje typu branch są kosztowne.
Niemniej jednak, wszystko to znacząco ułatwia zrozumienie algebry Boole'a i całej matematyki wokół tego.

\end{itemize}

% subsection:
\subsection{\EN{Cryptography}\ES{Criptograf\'ia}\IT{Crittografia}\RU{Криптография}\FR{Cryptographie}\DE{Kryptografie}\JA{暗号学}}
\label{crypto_books}

\begin{itemize}
\item \Schneier{}

\item (Free) lvh, \emph{Crypto 101}\footnote{\AlsoAvailableAs \url{https://www.crypto101.io/}}

\item (Free) Dan Boneh, Victor Shoup, \emph{A Graduate Course in Applied Cryptography}\footnote{\AlsoAvailableAs \url{https://crypto.stanford.edu/~dabo/cryptobook/}}.
\end{itemize}



\subsection{Coś jeszcze prostszego}

Osobom, dla których ta książka jest zbyt trudna i techniczna,
polecam łagodne wprowadzenie do niskopoziomowych zagadnień związanych z maszynami liczącymi:
``Code: The Hidden Language of Computer Hardware and Software'' Charlesa Petzolda.

Inną prostą książką jest komiks dla dzieci\footnote{\url{https://yurichev.com/mirrors/machine-code-for-beginners.pdf}} z 1983 roku,
poświęcony mikroprocesorom 6502 i Z80.

}
\CN{% !TEX program = XeLaTeX
% !TEX encoding = UTF-8
\documentclass[UTF8,nofonts]{ctexart}
\setCJKmainfont[BoldFont=STHeiti,ItalicFont=STKaiti]{STSong}
\setCJKsansfont[BoldFont=STHeiti]{STXihei}
\setCJKmonofont{STFangsong}

\begin{document}

%daveti: translated on Dec 28, 2016
%NOTE: above is needed for MacTex.

\chapter{值得一看的书和博客 Books/blogs worth reading}

\mysection{书和其他材料 Books and other materials}

\subsection{逆向工程 Reverse Engineering}

\begin{itemize}
\item Eldad Eilam, \emph{Reversing: Secrets of Reverse Engineering}, (2005)

\item Bruce Dang, Alexandre Gazet, Elias Bachaalany, Sebastien Josse, \emph{Practical Reverse Engineering: x86, x64, ARM, Windows Kernel, Reversing Tools, and Obfuscation}, (2014)

\item Michael Sikorski, Andrew Honig, \emph{Practical Malware Analysis: The Hands-On Guide to Dissecting Malicious Software}, (2012)

\item Chris Eagle, \emph{IDA Pro Book}, (2011)

\item Reginald Wong, \emph{Mastering Reverse Engineering: Re-engineer your ethical hacking skills}, (2018)

\end{itemize}


% TBT

\subsection{Windows}

\begin{itemize}
\item \Russinovich
\item Peter Ferrie -- The ``Ultimate'' Anti-Debugging Reference\footnote\url{http://pferrie.host22.com/papers/antidebug.pdf}}
\end{itemize}

\EN{Blogs}\ES{Blogs}\RU{Блоги}\FR{Blogs}\DE{Blogs}\PL{Blogi}:

\begin{itemize}
\item \href{http://blogs.msdn.com/oldnewthing/}{Microsoft: Raymond Chen}
\item \href{http://www.nynaeve.net/}{nynaeve.net}
\end{itemize}



\subsection{\CCpp}

\label{CCppBooks}

\begin{itemize}

\item \KRBook

\item \CNineNineStd\footnote{\AlsoAvailableAs \url{http://www.open-std.org/jtc1/sc22/WG14/www/docs/n1256.pdf}}

\item \TCPPPL

\item \CppOneOneStd\footnote{\AlsoAvailableAs \url{http://www.open-std.org/jtc1/sc22/wg21/docs/papers/2013/n3690.pdf}.}

\item \AgnerFogCPP\footnote{\AlsoAvailableAs \url{http://agner.org/optimize/optimizing_cpp.pdf}.}

\item \ParashiftCPPFAQ\footnote{\AlsoAvailableAs \url{http://www.parashift.com/c++-faq-lite/index.html}}

\item \CNotes\footnote{\AlsoAvailableAs \url{http://yurichev.com/C-book.html}}

\item JPL Institutional Coding Standard for the C Programming Language\footnote{\AlsoAvailableAs \url{https://yurichev.com/mirrors/C/JPL_Coding_Standard_C.pdf}}

\RU{\item Евгений Зуев --- Редкая профессия\footnote{\AlsoAvailableAs \url{https://yurichev.com/mirrors/C++/Redkaya_professiya.pdf}}}

\end{itemize}



\subsection{x86 / x86-64}

\label{x86_manuals}
\begin{itemize}
\item Intel手册\footnote{\AlsoAvailableAs \url{http://www.intel.com/content/www/us/en/processors/architectures-software-developer-manuals.html}}

\item AMD手册\footnote{\AlsoAvailableAs \url{http://developer.amd.com/resources/developer-guides-manuals/}}

\item \AgnerFog{}\footnote{\AlsoAvailableAs \url{http://agner.org/optimize/microarchitecture.pdf}}

\item \AgnerFogCC{}\footnote{\AlsoAvailableAs \url{http://www.agner.org/optimize/calling_conventions.pdf}}

\item \IntelOptimization

\item \AMDOptimization
\end{itemize}

虽然有点过时担任值得一读:

\MAbrash\footnote{\AlsoAvailableAs \url{https://github.com/jagregory/abrash-black-book}}
(这个哥们以针对Windows NT 3.1和id Quake的底层优化而出名)。

\subsection{ARM}

\begin{itemize}
\item ARM手册\footnote{\AlsoAvailableAs \url{http://infocenter.arm.com/help/index.jsp?topic=/com.arm.doc.subset.architecture.reference/index.html}}

\item \ARMSevenRef

\item \ARMSixFourRefURL

\item \ARMCookBook\footnote{\AlsoAvailableAs \url{https://yurichev.com/ref/ARM%20Cookbook%20(1994)/}}
\end{itemize}

% TBT

\subsection{Java}

\JavaBook.

\subsection{UNIX}

\TAOUP

\subsection{编程相关 Programming in general}

\begin{itemize}

\item \RobPikePractice

\item \HenryWarren.
% TBT
\item (仅献给具有计算机科学和数学背景的硬壳(hard-core)geek) Donald E. Knuth, \emph{计算机编程艺术 The Art of Computer Programming}.

\end{itemize}

% subsection:
\subsection{\EN{Cryptography}\ES{Criptograf\'ia}\IT{Crittografia}\RU{Криптография}\FR{Cryptographie}\DE{Kryptografie}\JA{暗号学}}
\label{crypto_books}

\begin{itemize}
\item \Schneier{}

\item (Free) lvh, \emph{Crypto 101}\footnote{\AlsoAvailableAs \url{https://www.crypto101.io/}}

\item (Free) Dan Boneh, Victor Shoup, \emph{A Graduate Course in Applied Cryptography}\footnote{\AlsoAvailableAs \url{https://crypto.stanford.edu/~dabo/cryptobook/}}.
\end{itemize}



\end{document}

% TBT

}

\EN{\chapter{Communities}

There are two excellent \ac{RE}-related subreddits on reddit.com:\\
\href{http://www.reddit.com/r/ReverseEngineering/}{reddit.com/r/ReverseEngineering/} and
\href{http://www.reddit.com/r/remath}{reddit.com/r/remath}
(on the topics for the intersection of \ac{RE} and mathematics).

There is also a \ac{RE} part of the Stack Exchange website:\\
\href{http://reverseengineering.stackexchange.com/}{reverseengineering.stackexchange.com}.

On IRC there are \#\#re and \#\#asm channels on
FreeNode\footnote{\href{https://freenode.net/}{freenode.net}}.

}
\RU{\chapter{Сообщества}

Имеются два отличных субреддита на reddit.com посвященных \ac{RE}:\\
\href{http://www.reddit.com/r/ReverseEngineering/}{reddit.com/r/ReverseEngineering/} и
\href{http://www.reddit.com/r/remath}{reddit.com/r/remath}

Имеется также часть сайта Stack Exchange посвященная \ac{RE}:\\
\href{http://reverseengineering.stackexchange.com/}{reverseengineering.stackexchange.com}.

На IRC есть каналы \#\#re и \#\#asm
FreeNode\footnote{\href{https://freenode.net/}{freenode.net}}.

}
\DE{\chapter{Communities}

Es gibt zwei exzellente Subreddits auf reddit.com mit \ac{RE}-relevanten Themen:\\
\href{http://www.reddit.com/r/ReverseEngineering/}{reddit.com/r/ReverseEngineering/} und
\href{http://www.reddit.com/r/remath}{reddit.com/r/remath}
(für Themen mit der Schnittmenge \ac{RE} und Mathematik).

Es gibt außerdem einen \ac{RE}-relevanten Teil auf der Stack Exchange-Seite:\\
\href{http://reverseengineering.stackexchange.com/}{reverseengineering.stackexchange.com}.

%TBT
%On IRC there are \#\#re and \#\#asm channels on
Im IRC gibt es einen \#\#re-Kanal auf FreeNode\footnote{\href{https://freenode.net/}{freenode.net}}.

}
\FR{\chapter{Communautés}

Il existe deux excellents subreddits liés à la \ac{RE} (rétro-ingénierie) sur reddit.com :\\
\href{http://www.reddit.com/r/ReverseEngineering/}{reddit.com/r/ReverseEngineering/} et
\href{http://www.reddit.com/r/remath}{reddit.com/r/remath}
(en lien avec la liaison de la \ac{RE} et des mathématiques).

Il y a également une section sur l'\ac{RE} sur le site Stack Exchange:\\
\href{http://reverseengineering.stackexchange.com/}{reverseengineering.stackexchange.com}.

Sur IRC, il y a les canaux \#\#re et \#\#asm sur
FreeNode\footnote{\href{https://freenode.net/}{freenode.net}}.

}
\IT{\chapter{Community}

Esistono due eccellenti subreddit riguardo il \ac{RE} su reddit.com:\\
\href{http://www.reddit.com/r/ReverseEngineering/}{reddit.com/r/ReverseEngineering/} e
\href{http://www.reddit.com/r/remath}{reddit.com/r/remath}
(Sugli argomenti di intersezione fra \ac{RE} e matematica).

Esiste anche una sezione relativa a \ac{RE} sul sito Stack Exchange:\\
\href{http://reverseengineering.stackexchange.com/}{reverseengineering.stackexchange.com}.

%TBT
%On IRC there are \#\#re and \#\#asm channels on
In IRC c'è un canale \#\#re su
FreeNode\footnote{\href{https://freenode.net/}{freenode.net}}.
}

\EN{\part*{Afterword}
\addcontentsline{toc}{part}{Afterword}

\mysection{Questions?}

Do not hesitate to mail any questions to the author: \\
\GTT{\EMAILS}.
Do you have any suggestion on new content for to the book?
Please do not hesitate to send any corrections (including grammar (you see how horrible my English is?)), etc.

The author is working on the book a lot, so the page and listing numbers, etc., are changing very rapidly.
Please do not refer to page and listing numbers in your emails to me.
There is a much simpler method: make a screenshot of the page, in a graphics editor underline the place where you see the error,
and send it to the author. He'll fix it much faster.
And if you familiar with git and \LaTeX\, you can fix the error right in the source code: 

\url{https://beginners.re/src/}.

Do not worry to bother me while writing me about any petty mistakes you found, even if you are not very confident.
I'm writing for beginners, after all, so beginners' opinions and comments are crucial for my job.
}
\ES{\part*{\ESph{}}
\addcontentsline{toc}{part}{\ESph{}}

\mysection{\ESph{}}

No dudes en enviar cualquier pregunta por email al autor: \\
\GTT{\EMAILS}.
?`Tienes alguna sugerencia sobre nuevas cosas que puedan agregarse al libro?
Por favor, no dudes en enviar correcciones (incluyendo gram\'atica), etc.

El autor est\'a trabajando arduamente en el libro, as\'i que los n\'umeros de p\'agina, de listado, etc., cambian con frecuencia.
Por favor, no utilices como referencia n\'umeros de p\'agina o de listado en tus emails.
Existe un m\'etodo mucho m\'as simple: toma una impresi\'on de pantalla de la p\'agina, subraya el lugar donde se encuentra el error en un editor de gr\'aficos,
y env\'ialo. De este modo ser\'a arreglado m\'as r\'apido.
Y si est\'as fmiliarizado con git y \LaTeX\, puedes arreglar el error directo en el c\'odigo fuente:

\url{https://beginners.re/src/}.

No te preocupes por escribirme acerca de errores insignificantes que encuentras, incluso si no te sientes seguro.
Estoy escribiendo para principiantes, despu\'es de todo, por lo tanto la opini\'on de los principiantes y sus comentarios son cruciales para mi trabajo.

}
\RU{\part*{Послесловие}
\addcontentsline{toc}{part}{Послесловие}

\mysection{Вопросы?}

Совершенно по любым вопросам вы можете не раздумывая писать автору: \\
\GTT{\EMAILS}
Есть идеи о том, что ещё можно добавить в эту книгу?
Пожалуйста, присылайте мне информацию о замеченных ошибках (включая грамматические), итд.

Автор много работает над книгой, поэтому номера страниц, листингов, итд, очень часто меняются.
Пожалуйста, в своих письмах мне не ссылайтесь на номера страниц и листингов.
Есть метод проще: сделайте скриншот страницы, затем в графическом редакторе подчеркните место, где вы видите
ошибку, и отправьте автору. Так он может исправить её намного быстрее.
Ну а если вы знакомы с git и \LaTeX, вы можете исправить ошибку прямо в исходных текстах:

\url{https://beginners.re/src/}.

Не бойтесь побеспокоить меня написав мне о какой-то мелкой ошибке, даже если вы не очень уверены.
Я всё-таки пишу для начинающих, поэтому мнение и комментарии именно начинающих очень важны для моей работы.

}
\NL{\part*{Nawoord}
\addcontentsline{toc}{part}{Nawoord}

\mysection{\NLph{}}

Aarzel niet om de auteur te mailen met eventuele vragen: \\
\GTT{\EMAILS}.
Heb je een suggestie of nieuwe content voor het boek?
Alsjeblieft, aarzel niet om verbeteringen (inclusief grammatica), etc.

De auteur werkt vaak aan het boek, dus de pagina- en lijstnummering, etc., veranderen zeer frequent.
Gelieve dus geen paginanummering te vermelden in de emails die je me stuurt.
Een veel betere methode is: neem een screenshot van de pagina, onderlijn het gedeelte waar je de fout ziet in een grafische editor,
en stuur deze mij toe. Zo zal de wijziging sneller worden opgenomen.
Als je bekend bent met git en \LaTeX\, kan je de fout rechtstreeks in de programmacode aanpassen:

\url{https://beginners.re/src/}.

Voel je zeker niet schuldig om me te storen bij het schrijven over de fouten die je hebt gevonden, zelfs als je niet echt zeker bent.
Uiteindelijk schrijf ik dit voor beginners, en de meningen en opmerkingen van beginners zijn dus cruciaal voor dit werk.

}
\IT{\part*{Afterword}
\addcontentsline{toc}{part}{Afterword}

\mysection{Domande?}

Non esitate a inviare una e-mail all'autore in caso abbiate una domanda o un dubbio o altro: \\
\GTT{\EMAILS}.
Hai dei suggerimenti su dei contenuti da aggiungere al libro?
Per favore, non esitare ad inviare qualsiasi correzione (comprese quelle grammaticali), ecc.

L'autore sta lavorando molto sul libro quindi i numeri delle pagine e gli elenchi numerati, etc., cambiano molto rapidamente.
Per favore non riferitevi a numeri di pagine quando mi scrivere un'e-mail.
C'è un metodo molto più semplice: fai uno screenshot della pagina e, tramite un editor per le foto, sottolinea il posto in cui hai visto l'errore
e inviamelo. Lo sistemerò molto più velocemente!
E se hai familiarità con git e \LaTeX\, allora puoi correggere l'errore direttamente nei sorgenti:

\url{https://beginners.re/src/}.

Non farti scrupoli nell'inviarmi gli errori che hai trovato anche nel caso non fossi sicuro(a).
Sto scrivendo per principianti e quindi la vostra opinione è estremamente importante per me.
}
\FR{\part*{Épilogue}
\addcontentsline{toc}{part}{Épilogue}

\mysection{Des questions ?}

Pour toute question, n'hésitez pas à m'envoyer un mail : \\
\GTT{\EMAILS}.
Vous avez une suggestion ou une proposition de contenu supplémentaire pour le livre ?
N'hésitez pas à envoyer toute correction (grammaire incluse), etc...

Je travaille beaucoup sur cette \oe{}uvre, c'est pourquoi les numéros de pages
et les numéros de parties peuvent changer rapidement.
S'il vous plaît, ne vous fiez pas à ces derniers si vous m'envoyez un email. 
Il existe une méthode plus simple : faites une capture d'écran de la page, puis dans
un éditeur graphique, soulignez l'endroit où il y a une erreur et envoyez-moi l'image.
Je la corrigerai plus rapidement.
Et si vous êtes familier avec git et \LaTeX\, vous pouvez corriger l'erreur directement
dans le code source :

\url{https://beginners.re/src/}.

N'hésitez surtout pas à m'envoyer la moindre erreur que vous pourriez trouver, aussi
petite soit-elle, même si vous n'êtes pas certain que ce soit une erreur.
J'écris pour les débutants après tout, il est donc crucial pour mon travail d'avoir
les retours de débutants.

}
\DE{\part*{Nachwort}
\addcontentsline{toc}{part}{Nachwort}

\mysection{Fragen?}

Zögern Sie nicht dem Autor Ihre Fragen per \GTT{\EMAILS} zu schicken.
Haben Sie irgendwelche Vorschläge für neue Inhalte in diesem Buch?
Gerne können Sie Korrekturen (auch grammatischer Art, vor allem im Englischen) zuschicken.

Der Autor arbeitet sehr viel an diesem Buch, so dass sich Seitenzahlen, Nummerierungen und so weiter
schnell ändern können. Bitte beziehen Sie sich also nicht auf diese Angaben.
Einfacher ist es einen Screenshot von der entsprechenden Seite zu machen und den Fehler in einer Bildbearbeitung
zu markieren. Der Autor kann so den Fehler sehr viel schneller korrigieren.
Wenn Sie sich mit git und \LaTeX\ auskennen, können Sie den Fehler direkt im Quellcode ändern:

\url{https://beginners.re/src/}.

Auch wenn Sie sich nicht ganz sicher sind, teilen Sie bitte die kleinen oder großen Fehler mit, die Sie finden.
Dieses Buch richtet sich speziell an Anfänger, so dass deren Meinungen und Kommentare einen entscheidenden Einfluss haben.
}
%\CN{% !TEX program = XeLaTeX
% !TEX encoding = UTF-8
\documentclass[UTF8,nofonts]{ctexart}
\setCJKmainfont[BoldFont=STHeiti,ItalicFont=STKaiti]{STSong}
\setCJKsansfont[BoldFont=STHeiti]{STXihei}
\setCJKmonofont{STFangsong}

\begin{document}

%daveti: translated on Dec 26, 2016
%NOTE: above is needed for MacTex.

\part*{后话 Afterword}
\addcontentsline{toc}{part}{后话 Afterword}

\mysection{问题?}

有问题请随时给我发邮件: \\
\GTT{\EMAILS}。
有添加新内容的建议?
如果发现任何错误,请随时给我发邮件(包括语法(估计你已经发现我的英语有多烂)。

作者依然投入大量精力在本书,所以页码或者罗列项会经常更新。
所以请在邮件里不要饮用页码或者罗列项号等。
有个更简单的方法:把问题页面拍个快照,用图形编辑器圈出问题所在,然后发给我。
这样我能更快解决问题。
当让,如果你熟悉git和\LaTeX\, 那就直接在原代码文件中解决问题: 

\url{https://beginners.re/src/}。

即使是小的问题,或者你并不确定,也给我发邮件。
毕竟这本书是写给初学者,所以初学者的建议对于完善本书极为重要。

\end{document}
}
\JA{\part*{Afterword}
\addcontentsline{toc}{part}{Afterword}

\mysection{Questions?}

恥ずかしがらずに著者へメールしてみよう
\GTT{\EMAILS}.
この本への新たなコンテンツについての提案がありますか?
怖がらずにどんな訂正でも送ってください(文法ミスも含め(私の日本語がとってもひどいのを見てるでしょ?))

The author is working on the book a lot, so the page and listing numbers, etc., are changing very rapidly.
Please do not refer to page and listing numbers in your emails to me.
There is a much simpler method: make a screenshot of the page, in a graphics editor underline the place where you see the error,
and send it to the author. He'll fix it much faster.
And if you familiar with git and \LaTeX\, you can fix the error right in the source code: 

\url{https://beginners.re/src/}.

Do not worry to bother me while writing me about any petty mistakes you found, even if you are not very confident.
I'm writing for beginners, after all, so beginners' opinions and comments are crucial for my job.
}
\PTBR{\part*{Afterword}
\addcontentsline{toc}{part}{Afterword}

\mysection{Questions?}

Não hesite em enviar qualquer pergunta ao autor: \\
\GTT{\EMAILS}.
Você tem alguma sugestão sobre novos conteúdos para o livro?
Por favor, não hesite em enviar quaisquer correções (inclusive gramaticais (nota quão horrível meu INglês é?)), etc.

O autor está trabalhando muito no livro, de modo que as numerações de página e dos códigos, etc., estão mudando muito rapidamente.
Por favor, não referencie a página e aos números de código nos seus e-mails para mim.
Existe um método muito mais simples: faça uma captura de tela da página, em um editor gráfico e sublinhe o local onde você vê o erro, e envie para o autor. Ele vai corrigi-lo muito mais rápido.
E se você conhece o git e \LaTeX\, você pode corrigir o erro diretamente no código-fonte:

\url{https://beginners.re/src/}.

Não se preocupe em me incomodar ao me escrever sobre qualquer erro insignificante que você encontrar, mesmo que você não esteja muita confiante. Estou escrevendo para iniciantes, afinal, as opiniões e os comentários dos iniciantes são cruciais para o meu trabalho.
}
\TR{\part*{Önsöz}
% TBT
\iffalse
\addcontentsline{toc}{part}{Afterword}

\mysection{Sorular?}

Do not hesitate to mail any questions to the author: \\
\GTT{\EMAILS}.
Do you have any suggestion on new content for to the book?
Please do not hesitate to send any corrections (including grammar (you see how horrible my English is?)), etc.

The author is working on the book a lot, so the page and listing numbers, etc., are changing very rapidly.
Please do not refer to page and listing numbers in your emails to me.
There is a much simpler method: make a screenshot of the page, in a graphics editor underline the place where you see the error,
and send it to the author. He'll fix it much faster.
And if you familiar with git and \LaTeX\, you can fix the error right in the source code: 

\url{https://beginners.re/src/}.

Do not worry to bother me while writing me about any petty mistakes you found, even if you are not very confident.
I'm writing for beginners, after all, so beginners' opinions and comments are crucial for my job.
\fi
}
\CN{% !TEX program = XeLaTeX
% !TEX encoding = UTF-8
\documentclass[UTF8,nofonts]{ctexart}
\setCJKmainfont[BoldFont=STHeiti,ItalicFont=STKaiti]{STSong}
\setCJKsansfont[BoldFont=STHeiti]{STXihei}
\setCJKmonofont{STFangsong}

\begin{document}

%daveti: translated on Dec 26, 2016
%NOTE: above is needed for MacTex.

\part*{后话 Afterword}
\addcontentsline{toc}{part}{后话 Afterword}

\mysection{问题?}

有问题请随时给我发邮件: \\
\GTT{\EMAILS}。
有添加新内容的建议?
如果发现任何错误,请随时给我发邮件(包括语法(估计你已经发现我的英语有多烂)。

作者依然投入大量精力在本书,所以页码或者罗列项会经常更新。
所以请在邮件里不要饮用页码或者罗列项号等。
有个更简单的方法:把问题页面拍个快照,用图形编辑器圈出问题所在,然后发给我。
这样我能更快解决问题。
当让,如果你熟悉git和\LaTeX\, 那就直接在原代码文件中解决问题: 

\url{https://beginners.re/src/}。

即使是小的问题,或者你并不确定,也给我发邮件。
毕竟这本书是写给初学者,所以初学者的建议对于完善本书极为重要。

\end{document}
}





\EN{\part*{\RU{Приложение}\EN{Appendix}\DE{Anhang}\FR{Appendice}\IT{Appendice}}
\appendix
\addcontentsline{toc}{part}{\RU{Приложение}\EN{Appendix}\DE{Anhang}\FR{Appendice}\IT{Appendice}}

% chapters
\EN{\mysection{Task manager practical joke (Windows Vista)}
\myindex{Windows!Windows Vista}

Let's see if it's possible to hack Task Manager slightly so it would detect more \ac{CPU} cores.

\myindex{Windows!NTAPI}

Let us first think, how does the Task Manager know the number of cores?

There is the \TT{GetSystemInfo()} win32 function present in win32 userspace which can tell us this.
But it's not imported in \TT{taskmgr.exe}.

There is, however, another one in \gls{NTAPI}, \TT{NtQuerySystemInformation()}, 
which is used in \TT{taskmgr.exe} in several places.

To get the number of cores, one has to call this function with the \TT{SystemBasicInformation} constant
as a first argument (which is zero
\footnote{\href{http://msdn.microsoft.com/en-us/library/windows/desktop/ms724509(v=vs.85).aspx}{MSDN}}).

The second argument has to point to the buffer which is getting all the information.

So we have to find all calls to the \\
\TT{NtQuerySystemInformation(0, ?, ?, ?)} function.
Let's open \TT{taskmgr.exe} in IDA. 
\myindex{Windows!PDB}

What is always good about Microsoft executables is that IDA can download the corresponding \gls{PDB} 
file for this executable and show all function names.

It is visible that Task Manager is written in \Cpp and some of the function names and classes are really 
speaking for themselves.
There are classes CAdapter, CNetPage, CPerfPage, CProcInfo, CProcPage, CSvcPage, 
CTaskPage, CUserPage.

Apparently, each class corresponds to each tab in Task Manager.

Let's visit each call and add comment with the value which is passed as the first function argument.
We will write \q{not zero} at some places, because the value there was clearly not zero, 
but something really different (more about this in the second part of this chapter).

And we are looking for zero passed as argument, after all.

\begin{figure}[H]
\centering
\myincludegraphics{examples/taskmgr/IDA_xrefs.png}
\caption{IDA: cross references to NtQuerySystemInformation()}
\end{figure}

Yes, the names are really speaking for themselves.

When we closely investigate each place where\\
\TT{NtQuerySystemInformation(0, ?, ?, ?)} is called,
we quickly find what we need in the \TT{InitPerfInfo()} function:

\lstinputlisting[caption=taskmgr.exe (Windows Vista),style=customasmx86]{examples/taskmgr/taskmgr.lst}

\TT{g\_cProcessors} is a global variable, and this name has been assigned by 
IDA according to the \gls{PDB} loaded from Microsoft's symbol server.

The byte is taken from \TT{var\_C20}. 
And \TT{var\_C58} is passed to\\
\TT{NtQuerySystemInformation()} 
as a pointer to the receiving buffer.
The difference between 0xC20 and 0xC58 is 0x38 (56).

Let's take a look at format of the return structure, which we can find in MSDN:

\begin{lstlisting}[style=customc]
typedef struct _SYSTEM_BASIC_INFORMATION {
    BYTE Reserved1[24];
    PVOID Reserved2[4];
    CCHAR NumberOfProcessors;
} SYSTEM_BASIC_INFORMATION;
\end{lstlisting}

This is a x64 system, so each PVOID takes 8 bytes.

All \emph{reserved} fields in the structure take $24+4*8=56$ bytes.

Oh yes, this implies that \TT{var\_C20} is the local stack is exactly the
\TT{NumberOfProcessors} field of the \TT{SYSTEM\_BASIC\_INFORMATION} structure.

Let's check our guess.
Copy \TT{taskmgr.exe} from \TT{C:\textbackslash{}Windows\textbackslash{}System32} 
to some other folder 
(so the \emph{Windows Resource Protection} 
will not try to restore the patched \TT{taskmgr.exe}).

Let's open it in Hiew and find the place:

\begin{figure}[H]
\centering
\myincludegraphics{examples/taskmgr/hiew2.png}
\caption{Hiew: find the place to be patched}
\end{figure}

Let's replace the \TT{MOVZX} instruction with ours.
Let's pretend we've got 64 CPU cores.

Add one additional \ac{NOP} (because our instruction is shorter than the original one):

\begin{figure}[H]
\centering
\myincludegraphics{examples/taskmgr/hiew1.png}
\caption{Hiew: patch it}
\end{figure}

And it works!
Of course, the data in the graphs is not correct.

At times, Task Manager even shows an overall CPU load of more than 100\%.

\begin{figure}[H]
\centering
\myincludegraphics{examples/taskmgr/taskmgr_64cpu_crop.png}
\caption{Fooled Windows Task Manager}
\end{figure}

The biggest number Task Manager does not crash with is 64.

Apparently, Task Manager in Windows Vista was not tested on computers with a large number of cores.

So there are probably some static data structure(s) inside it limited to 64 cores.

\subsection{Using LEA to load values}
\label{TaskMgr_LEA}

Sometimes, \TT{LEA} is used in \TT{taskmgr.exe} instead of \TT{MOV} to set the first argument of \\
\TT{NtQuerySystemInformation()}:

\lstinputlisting[caption=taskmgr.exe (Windows Vista),style=customasmx86]{examples/taskmgr/taskmgr2.lst}

\myindex{x86!\Instructions!LEA}

Perhaps \ac{MSVC} did so because machine code of \INS{LEA} is shorter than \INS{MOV REG, 5} (would be 5 instead of 4).

\INS{LEA} with offset in $-128..127$ range (offset will occupy 1 byte in opcode) with 32-bit registers is even shorter (for lack of REX prefix)---3 bytes.

Another example of such thing is: \myref{using_MOV_and_pack_of_LEA_to_load_values}.
}
\RU{\subsection{Обменять входные значения друг с другом}

Вот так:

\lstinputlisting[style=customc]{patterns/061_pointers/swap/5_RU.c}

Как видим, байты загружаются в младшие 8-битные части регистров \TT{ECX} и \TT{EBX} используя \INS{MOVZX}
(так что старшие части регистров очищаются), затем байты записываются назад в другом порядке.

\lstinputlisting[style=customasmx86,caption=Optimizing GCC 5.4]{patterns/061_pointers/swap/5_GCC_O3_x86.s}

Адреса обоих байтов берутся из аргументов и во время исполнения ф-ции находятся в регистрах \TT{EDX} и \TT{EAX}.

Так что исопльзуем указатели --- вероятно, без них нет способа решить эту задачу лучше.

}
\DE{\mysection{x86}

\subsection{Terminologie}

Geläufig für 16-Bit (8086/80286), 32-Bit (80386, etc.), 64-Bit.

\myindex{IEEE 754}
\myindex{MS-DOS}
\begin{description}
	\item[Byte] 8-Bit.
		Die DB Assembler-Direktive wird zum Definieren von Variablen und Arrays genutzt.
		Bytes werden in dem 8-Bit-Teil der folgenden Register übergeben:
		\TT{AL/BL/CL/DL/AH/BH/CH/DH/SIL/DIL/R*L}.
	\item[Wort] 16-Bit.
		DW Assembler-Direktive \dittoclosing.
		Bytes werden in dem 16-Bit-Teil der folgenden Register übergeben:
			\TT{AX/BX/CX/DX/SI/DI/R*W}.
	\item[Doppelwort] (\q{dword}) 32-Bit.
		DD Assembler-Direktive \dittoclosing.
		Doppelwörter werden in Registern (x86) oder dem 32-Bit-Teil der Register (x64) übergeben.
		In 16-Bit-Code werden Doppelwörter in 16-Bit-Registerpaaren übergeben.
	\item[zwei Doppelwörter] (\q{qword}) 64-Bit.
		DQ Assembler-Direktive \dittoclosing.
		In 32-Bit-Umgebungen werden diese in 32-Bit-Registerpaaren übergeben.
	\item[tbyte] (10 Byte) 80-Bit oder 10 Bytes (für IEEE 754 FPU Register).
	\item[paragraph] (16 Byte) --- Bezeichnung war in MS-DOS Umgebungen gebräuchlich.
\end{description}

\myindex{Windows!API}

Datentypen der selben Breite (BYTE, WORD, DWORD) entsprechen auch denen in der Windows \ac{API}.

% TODO German Translation (DSiekmeier)
%\input{appendix/x86/registers} % subsection
%\input{appendix/x86/instructions} % subsection
\subsection{npad}
\label{sec:npad}

\RU{Это макрос в ассемблере, для выравнивания некоторой метки по некоторой границе.}
\EN{It is an assembly language macro for aligning labels on a specific boundary.}
\DE{Dies ist ein Assembler-Makro um Labels an bestimmten Grenzen auszurichten.}
\FR{C'est une macro du langage d'assemblage pour aligner les labels sur une limite
spécifique.}

\RU{Это нужно для тех \emph{нагруженных} меток, куда чаще всего передается управление, например, 
начало тела цикла. 
Для того чтобы процессор мог эффективнее вытягивать данные или код из памяти, через шину с памятью, 
кэширование, итд.}
\EN{That's often needed for the busy labels to where the control flow is often passed, e.g., loop body starts.
So the CPU can load the data or code from the memory effectively, through the memory bus, cache lines, etc.}
\DE{Dies ist oft nützlich Labels, die oft Ziel einer Kotrollstruktur sind, wie Schleifenköpfe.
Somit kann die CPU Daten oder Code sehr effizient vom Speicher durch den Bus, den Cache, usw. laden.}
\FR{C'est souvent nécessaire pour des labels très utilisés, comme par exemple le
début d'un corps de boucle. Ainsi, le CPU peut charger les données ou le code depuis
la mémoire efficacement, à travers le bus mémoire, les caches, etc.}

\RU{Взято из}\EN{Taken from}\DE{Entnommen von}\FR{Pris de} \TT{listing.inc} (MSVC):

\myindex{x86!\Instructions!NOP}
\RU{Это, кстати, любопытный пример различных вариантов \NOP{}-ов. 
Все эти инструкции не дают никакого эффекта, но отличаются разной длиной.}
\EN{By the way, it is a curious example of the different \NOP variations.
All these instructions have no effects whatsoever, but have a different size.}
\DE{Dies ist übrigens ein Beispiel für die unterschiedlichen \NOP-Variationen.
Keine dieser Anweisungen hat eine Auswirkung, aber alle haben eine unterschiedliche Größe.}
\FR{À propos, c'est un exemple curieux des différentes variations de \NOP. Toutes
ces instructions n'ont pas d'effet, mais ont une taille différente.}

\RU{Цель в том, чтобы была только одна инструкция, а не набор NOP-ов, 
считается что так лучше для производительности CPU.}
\EN{Having a single idle instruction instead of couple of NOP-s,
is accepted to be better for CPU performance.}
\DE{Eine einzelne Idle-Anweisung anstatt mehrerer NOPs hat positive Auswirkungen
auf die CPU-Performance.}
\FR{Avoir une seule instruction sans effet au lieu de plusieurs est accepté comme
étant meilleur pour la performance du CPU.}

\begin{lstlisting}[style=customasmx86]
;; LISTING.INC
;;
;; This file contains assembler macros and is included by the files created
;; with the -FA compiler switch to be assembled by MASM (Microsoft Macro
;; Assembler).
;;
;; Copyright (c) 1993-2003, Microsoft Corporation. All rights reserved.

;; non destructive nops
npad macro size
if size eq 1
  nop
else
 if size eq 2
   mov edi, edi
 else
  if size eq 3
    ; lea ecx, [ecx+00]
    DB 8DH, 49H, 00H
  else
   if size eq 4
     ; lea esp, [esp+00]
     DB 8DH, 64H, 24H, 00H
   else
    if size eq 5
      add eax, DWORD PTR 0
    else
     if size eq 6
       ; lea ebx, [ebx+00000000]
       DB 8DH, 9BH, 00H, 00H, 00H, 00H
     else
      if size eq 7
	; lea esp, [esp+00000000]
	DB 8DH, 0A4H, 24H, 00H, 00H, 00H, 00H 
      else
       if size eq 8
        ; jmp .+8; .npad 6
	DB 0EBH, 06H, 8DH, 9BH, 00H, 00H, 00H, 00H
       else
        if size eq 9
         ; jmp .+9; .npad 7
         DB 0EBH, 07H, 8DH, 0A4H, 24H, 00H, 00H, 00H, 00H
        else
         if size eq 10
          ; jmp .+A; .npad 7; .npad 1
          DB 0EBH, 08H, 8DH, 0A4H, 24H, 00H, 00H, 00H, 00H, 90H
         else
          if size eq 11
           ; jmp .+B; .npad 7; .npad 2
           DB 0EBH, 09H, 8DH, 0A4H, 24H, 00H, 00H, 00H, 00H, 8BH, 0FFH
          else
           if size eq 12
            ; jmp .+C; .npad 7; .npad 3
            DB 0EBH, 0AH, 8DH, 0A4H, 24H, 00H, 00H, 00H, 00H, 8DH, 49H, 00H
           else
            if size eq 13
             ; jmp .+D; .npad 7; .npad 4
             DB 0EBH, 0BH, 8DH, 0A4H, 24H, 00H, 00H, 00H, 00H, 8DH, 64H, 24H, 00H
            else
             if size eq 14
              ; jmp .+E; .npad 7; .npad 5
              DB 0EBH, 0CH, 8DH, 0A4H, 24H, 00H, 00H, 00H, 00H, 05H, 00H, 00H, 00H, 00H
             else
              if size eq 15
               ; jmp .+F; .npad 7; .npad 6
               DB 0EBH, 0DH, 8DH, 0A4H, 24H, 00H, 00H, 00H, 00H, 8DH, 9BH, 00H, 00H, 00H, 00H
              else
	       %out error: unsupported npad size
               .err
              endif
             endif
            endif
           endif
          endif
         endif
        endif
       endif
      endif
     endif
    endif
   endif
  endif
 endif
endif
endm
\end{lstlisting}
 % subsection
}
\FR{\subsection{Exemple \#2: SCO OpenServer}

\label{examples_SCO}
\myindex{SCO OpenServer}
Un ancien logiciel pour SCO OpenServer de 1997 développé par une société qui a disparue
depuis longtemps.

Il y a un driver de dongle special à installer dans le système, qui contient les
chaînes de texte suivantes:
\q{Copyright 1989, Rainbow Technologies, Inc., Irvine, CA}
et
\q{Sentinel Integrated Driver Ver. 3.0 }.

Après l'installation du driver dans SCO OpenServer, ces fichiers apparaissent dans
l'arborescence /dev:

\begin{lstlisting}
/dev/rbsl8
/dev/rbsl9
/dev/rbsl10
\end{lstlisting}

Le programme renvoie une erreur lorsque le dongle n'est pas connecté, mais le message
d'erreur n'est pas trouvé dans les exécutables.

\myindex{COFF}

Grâce à \ac{IDA}, il est facile de charger l'exécutable COFF utilisé dans SCO OpenServer.

Essayons de trouver la chaîne \q{rbsl} et en effet, elle se trouve dans ce morceau
de code:

\lstinputlisting[style=customasmx86]{examples/dongles/2/1.lst}

Oui, en effet, le programme doit communiquer d'une façon ou d'une autre avec le driver.

\myindex{thunk-functions}
Le seul endroit où la fonction \TT{SSQC()} est appelée est dans la \glslink{thunk
 function}{fonction thunk}:

\lstinputlisting[style=customasmx86]{examples/dongles/2/2.lst}

SSQ() peut être appelé depuis au moins 2 fonctions.

L'une d'entre elles est:

\lstinputlisting[style=customasmx86]{examples/dongles/2/check1_EN.lst}

\q{\TT{3C}} et \q{\TT{3E}} semblent familiers: il y avait un dongle Sentinel Pro de
Rainbow sans mémoire, fournissant seulement une fonction de crypto-hachage secrète.

Vous pouvez lire une courte description de la fonction de hachage dont il s'agit
ici: \myref{hash_func}.

Mais retournons au programme.

Donc le programme peut seulement tester si un dongle est connecté ou s'il est absent.

Aucune autre information ne peut être écrite dans un tel dongle, puisqu'il n'a pas
de mémoire.
Les codes sur deux caractères sont des commandes (nous pouvons voir comment les commandes
sont traitées dans la fonction \TT{SSQC()}) et toutes les autres chaînes sont hachées
dans le dongle, transformées en un nombre 16-bit.
L'algorithme était secret, donc il n'était pas possible d'écrire un driver de remplacement
ou de refaire un dongle matériel qui l'émulerait parfaitement.

Toutefois, il est toujours possible d'intercepter tous les accès au dongle et de
trouver les constantes auxquelles les résultats de la fonction de hachage sont comparées.

Mais nous devons dire qu'il est possible de construire un schéma de logiciel de protection
de copie robuste basé sur une fonction secrète de hachage cryptographique: il suffit
qu'elle chiffre/déchiffre les fichiers de données utilisés par votre logiciel.

Mais retournons au code:

Les codes 51/52/53 sont utilisés pour choisir le port imprimante LPT.
3x/4x sont utilisés pour le choix de la \q{famille} (c'est ainsi que les dongles
Sentinel Pro sont différenciés les uns des autres: plus d'un dongle peut être connecté
sur un port LPT).

La seule chaîne passée à la fonction qui ne fasse pas 2 caractères est "0123456789".

Ensuite, le résultat est comparé à l'ensemble des résultats valides.

Si il est correct, 0xC ou 0xB est écrit dans la variable globale \TT{ctl\_model}.%

Une autre chaîne de texte qui est passée est
"PRESS ANY KEY TO CONTINUE: ", mais le résultat n'est pas testé.
Difficile de dire pourquoi, probablement une erreur\footnote{C'est un sentiment
étrange de trouver un bug dans un logiciel aussi ancien.}.

Voyons où la valeur de la variable globale \TT{ctl\_model} est utilisée.

Un tel endroit est:

\lstinputlisting[style=customasmx86]{examples/dongles/2/4.lst}

Si c'est 0, un message d'erreur chiffré est passé à une routine de déchiffrement
et affiché.

\myindex{x86!\Instructions!XOR}

La routine de déchiffrement de la chaîne semble être un simple \glslink{xoring}{xor}:

\lstinputlisting[style=customasmx86]{examples/dongles/2/err_warn.lst}

C'est pourquoi nous étions incapable de trouver le message d'erreur dans les fichiers
exécutable, car ils sont chiffrés (ce qui est une pratique courante).

Un autre appel à la fonction de hachage \TT{SSQ()} lui passe la chaîne \q{offln}
et le résultat est comparé avec \TT{0xFE81} et \TT{0x12A9}.

Si ils ne correspondent pas, ça se comporte comme une sorte de fonction \TT{timer()}
(peut-être en attente qu'un dongle mal connecté soit reconnecté et re-testé?) et ensuite
déchiffre un autre message d'erreur à afficher.

\lstinputlisting[style=customasmx86]{examples/dongles/2/check2_EN.lst}

Passer outre le dongle est assez facile: il suffit de patcher tous les sauts après
les instructions \CMP pertinentes.

Une autre option est d'écrire notre propre driver SCO OpenServer, contenant une table
de questions et de réponses, toutes celles qui sont présentent dans le programme.

\subsubsection{Déchiffrer les messages d'erreur}

À propos, nous pouvons aussi essayer de déchiffrer tous les messages d'erreurs.
L'algorithme qui se trouve dans la fonction \TT{err\_warn()} est très simple, en effet:

\lstinputlisting[caption=Decryption function,style=customasmx86]{examples/dongles/2/decrypting_FR.lst}

Comme on le voit, non seulement la chaîne est transmise à la fonction de déchiffrement
mais aussi la clef:

\lstinputlisting[style=customasmx86]{examples/dongles/2/tmp1_EN.asm}

L'algorithme est un simple \glslink{xoring}{xor}: chaque octet est xoré avec la clef, mais
la clef est incrémentée de 3 après le traitement de chaque octet.

Nous pouvons écrire un petit script Python pour vérifier notre hypothèse:

\lstinputlisting[caption=Python 3.x]{examples/dongles/2/decr1.py}

Et il affiche: \q{check security device connection}.
Donc oui, ceci est le message déchiffré.

Il y a d'autres messages chiffrés, avec leur clef correspondante.
Mais inutile de dire qu'il est possible de les déchiffrer sans leur clef.
Premièrement, nous voyons que le clef est en fait un octet.
C'est parce que l'instruction principale de déchiffrement (\XOR) fonctionne au niveau
de l'octet.
La clef se trouve dans le registre \ESI, mais seulement une partie de \ESI d'un octet
est utilisée.
Ainsi, une clef pourrait être plus grande que 255, mais sa valeur est toujours arrondie.

En conséquence, nous pouvons simplement essayer de brute-forcer, en essayant toutes
les clefs possible dans l'intervalle 0..255.
Nous allons aussi écarter les messages comportants des caractères non-imprimable.

\lstinputlisting[caption=Python 3.x]{examples/dongles/2/decr2.py}

Et nous obtenons:

\lstinputlisting[caption=Results]{examples/dongles/2/decr2_result.txt}

Ici il y a un peu de déchet, mais nous pouvons rapidement trouver les messages en
anglais.

À propos, puisque l'algorithme est un simple chiffrement xor, la même fonction peut
être utilisée pour chiffrer les messages.
Si besoin, nous pouvons chiffrer nos propres messages, et patcher le programme en les insérant.
}
\IT{\subsection{Fall-through}

Un altro uso diffuso dell'operatore \TT{switch()} è il cosiddetto \q{fallthrough}.
Ecco un semplice esempio \footnote{Preso da \url{https://github.com/azonalon/prgraas/blob/master/prog1lib/lecture_examples/is_whitespace.c}}:

\lstinputlisting[numbers=left,style=customc]{patterns/08_switch/4_fallthrough/fallthrough1.c}

Uno leggermente più difficile, dal kernel di Linux \footnote{Preso da \url{https://github.com/torvalds/linux/blob/master/drivers/media/dvb-frontends/lgdt3306a.c}}:

\lstinputlisting[numbers=left,style=customc]{patterns/08_switch/4_fallthrough/fallthrough2.c}

\lstinputlisting[caption=Optimizing GCC 5.4.0 x86,numbers=left,style=customasmx86]{patterns/08_switch/4_fallthrough/fallthrough2.s}

Possiamo arrivare alla label \TT{.L5} se all'input della funzione viene dato il valore 3250.
Ma si può anche giungere allo stesso punto da un altro percorso:
notiamo che non ci sono jump tra la chiamata a \printf e la label \TT{.L5}.

Questo spiega facilmente perchè i costrutti con \emph{switch()} sono spesso fonte di bug:
è sufficiente dimenticare un \emph{break} per trasformare il costrutto \emph{switch()} in un \emph{fallthrough} , in cui vengono eseguiti
più blocchi invece di uno solo.
}

\EN{\mysection{Task manager practical joke (Windows Vista)}
\myindex{Windows!Windows Vista}

Let's see if it's possible to hack Task Manager slightly so it would detect more \ac{CPU} cores.

\myindex{Windows!NTAPI}

Let us first think, how does the Task Manager know the number of cores?

There is the \TT{GetSystemInfo()} win32 function present in win32 userspace which can tell us this.
But it's not imported in \TT{taskmgr.exe}.

There is, however, another one in \gls{NTAPI}, \TT{NtQuerySystemInformation()}, 
which is used in \TT{taskmgr.exe} in several places.

To get the number of cores, one has to call this function with the \TT{SystemBasicInformation} constant
as a first argument (which is zero
\footnote{\href{http://msdn.microsoft.com/en-us/library/windows/desktop/ms724509(v=vs.85).aspx}{MSDN}}).

The second argument has to point to the buffer which is getting all the information.

So we have to find all calls to the \\
\TT{NtQuerySystemInformation(0, ?, ?, ?)} function.
Let's open \TT{taskmgr.exe} in IDA. 
\myindex{Windows!PDB}

What is always good about Microsoft executables is that IDA can download the corresponding \gls{PDB} 
file for this executable and show all function names.

It is visible that Task Manager is written in \Cpp and some of the function names and classes are really 
speaking for themselves.
There are classes CAdapter, CNetPage, CPerfPage, CProcInfo, CProcPage, CSvcPage, 
CTaskPage, CUserPage.

Apparently, each class corresponds to each tab in Task Manager.

Let's visit each call and add comment with the value which is passed as the first function argument.
We will write \q{not zero} at some places, because the value there was clearly not zero, 
but something really different (more about this in the second part of this chapter).

And we are looking for zero passed as argument, after all.

\begin{figure}[H]
\centering
\myincludegraphics{examples/taskmgr/IDA_xrefs.png}
\caption{IDA: cross references to NtQuerySystemInformation()}
\end{figure}

Yes, the names are really speaking for themselves.

When we closely investigate each place where\\
\TT{NtQuerySystemInformation(0, ?, ?, ?)} is called,
we quickly find what we need in the \TT{InitPerfInfo()} function:

\lstinputlisting[caption=taskmgr.exe (Windows Vista),style=customasmx86]{examples/taskmgr/taskmgr.lst}

\TT{g\_cProcessors} is a global variable, and this name has been assigned by 
IDA according to the \gls{PDB} loaded from Microsoft's symbol server.

The byte is taken from \TT{var\_C20}. 
And \TT{var\_C58} is passed to\\
\TT{NtQuerySystemInformation()} 
as a pointer to the receiving buffer.
The difference between 0xC20 and 0xC58 is 0x38 (56).

Let's take a look at format of the return structure, which we can find in MSDN:

\begin{lstlisting}[style=customc]
typedef struct _SYSTEM_BASIC_INFORMATION {
    BYTE Reserved1[24];
    PVOID Reserved2[4];
    CCHAR NumberOfProcessors;
} SYSTEM_BASIC_INFORMATION;
\end{lstlisting}

This is a x64 system, so each PVOID takes 8 bytes.

All \emph{reserved} fields in the structure take $24+4*8=56$ bytes.

Oh yes, this implies that \TT{var\_C20} is the local stack is exactly the
\TT{NumberOfProcessors} field of the \TT{SYSTEM\_BASIC\_INFORMATION} structure.

Let's check our guess.
Copy \TT{taskmgr.exe} from \TT{C:\textbackslash{}Windows\textbackslash{}System32} 
to some other folder 
(so the \emph{Windows Resource Protection} 
will not try to restore the patched \TT{taskmgr.exe}).

Let's open it in Hiew and find the place:

\begin{figure}[H]
\centering
\myincludegraphics{examples/taskmgr/hiew2.png}
\caption{Hiew: find the place to be patched}
\end{figure}

Let's replace the \TT{MOVZX} instruction with ours.
Let's pretend we've got 64 CPU cores.

Add one additional \ac{NOP} (because our instruction is shorter than the original one):

\begin{figure}[H]
\centering
\myincludegraphics{examples/taskmgr/hiew1.png}
\caption{Hiew: patch it}
\end{figure}

And it works!
Of course, the data in the graphs is not correct.

At times, Task Manager even shows an overall CPU load of more than 100\%.

\begin{figure}[H]
\centering
\myincludegraphics{examples/taskmgr/taskmgr_64cpu_crop.png}
\caption{Fooled Windows Task Manager}
\end{figure}

The biggest number Task Manager does not crash with is 64.

Apparently, Task Manager in Windows Vista was not tested on computers with a large number of cores.

So there are probably some static data structure(s) inside it limited to 64 cores.

\subsection{Using LEA to load values}
\label{TaskMgr_LEA}

Sometimes, \TT{LEA} is used in \TT{taskmgr.exe} instead of \TT{MOV} to set the first argument of \\
\TT{NtQuerySystemInformation()}:

\lstinputlisting[caption=taskmgr.exe (Windows Vista),style=customasmx86]{examples/taskmgr/taskmgr2.lst}

\myindex{x86!\Instructions!LEA}

Perhaps \ac{MSVC} did so because machine code of \INS{LEA} is shorter than \INS{MOV REG, 5} (would be 5 instead of 4).

\INS{LEA} with offset in $-128..127$ range (offset will occupy 1 byte in opcode) with 32-bit registers is even shorter (for lack of REX prefix)---3 bytes.

Another example of such thing is: \myref{using_MOV_and_pack_of_LEA_to_load_values}.
}
\RU{\subsection{Обменять входные значения друг с другом}

Вот так:

\lstinputlisting[style=customc]{patterns/061_pointers/swap/5_RU.c}

Как видим, байты загружаются в младшие 8-битные части регистров \TT{ECX} и \TT{EBX} используя \INS{MOVZX}
(так что старшие части регистров очищаются), затем байты записываются назад в другом порядке.

\lstinputlisting[style=customasmx86,caption=Optimizing GCC 5.4]{patterns/061_pointers/swap/5_GCC_O3_x86.s}

Адреса обоих байтов берутся из аргументов и во время исполнения ф-ции находятся в регистрах \TT{EDX} и \TT{EAX}.

Так что исопльзуем указатели --- вероятно, без них нет способа решить эту задачу лучше.

}
\FR{\subsection{Exemple \#2: SCO OpenServer}

\label{examples_SCO}
\myindex{SCO OpenServer}
Un ancien logiciel pour SCO OpenServer de 1997 développé par une société qui a disparue
depuis longtemps.

Il y a un driver de dongle special à installer dans le système, qui contient les
chaînes de texte suivantes:
\q{Copyright 1989, Rainbow Technologies, Inc., Irvine, CA}
et
\q{Sentinel Integrated Driver Ver. 3.0 }.

Après l'installation du driver dans SCO OpenServer, ces fichiers apparaissent dans
l'arborescence /dev:

\begin{lstlisting}
/dev/rbsl8
/dev/rbsl9
/dev/rbsl10
\end{lstlisting}

Le programme renvoie une erreur lorsque le dongle n'est pas connecté, mais le message
d'erreur n'est pas trouvé dans les exécutables.

\myindex{COFF}

Grâce à \ac{IDA}, il est facile de charger l'exécutable COFF utilisé dans SCO OpenServer.

Essayons de trouver la chaîne \q{rbsl} et en effet, elle se trouve dans ce morceau
de code:

\lstinputlisting[style=customasmx86]{examples/dongles/2/1.lst}

Oui, en effet, le programme doit communiquer d'une façon ou d'une autre avec le driver.

\myindex{thunk-functions}
Le seul endroit où la fonction \TT{SSQC()} est appelée est dans la \glslink{thunk
 function}{fonction thunk}:

\lstinputlisting[style=customasmx86]{examples/dongles/2/2.lst}

SSQ() peut être appelé depuis au moins 2 fonctions.

L'une d'entre elles est:

\lstinputlisting[style=customasmx86]{examples/dongles/2/check1_EN.lst}

\q{\TT{3C}} et \q{\TT{3E}} semblent familiers: il y avait un dongle Sentinel Pro de
Rainbow sans mémoire, fournissant seulement une fonction de crypto-hachage secrète.

Vous pouvez lire une courte description de la fonction de hachage dont il s'agit
ici: \myref{hash_func}.

Mais retournons au programme.

Donc le programme peut seulement tester si un dongle est connecté ou s'il est absent.

Aucune autre information ne peut être écrite dans un tel dongle, puisqu'il n'a pas
de mémoire.
Les codes sur deux caractères sont des commandes (nous pouvons voir comment les commandes
sont traitées dans la fonction \TT{SSQC()}) et toutes les autres chaînes sont hachées
dans le dongle, transformées en un nombre 16-bit.
L'algorithme était secret, donc il n'était pas possible d'écrire un driver de remplacement
ou de refaire un dongle matériel qui l'émulerait parfaitement.

Toutefois, il est toujours possible d'intercepter tous les accès au dongle et de
trouver les constantes auxquelles les résultats de la fonction de hachage sont comparées.

Mais nous devons dire qu'il est possible de construire un schéma de logiciel de protection
de copie robuste basé sur une fonction secrète de hachage cryptographique: il suffit
qu'elle chiffre/déchiffre les fichiers de données utilisés par votre logiciel.

Mais retournons au code:

Les codes 51/52/53 sont utilisés pour choisir le port imprimante LPT.
3x/4x sont utilisés pour le choix de la \q{famille} (c'est ainsi que les dongles
Sentinel Pro sont différenciés les uns des autres: plus d'un dongle peut être connecté
sur un port LPT).

La seule chaîne passée à la fonction qui ne fasse pas 2 caractères est "0123456789".

Ensuite, le résultat est comparé à l'ensemble des résultats valides.

Si il est correct, 0xC ou 0xB est écrit dans la variable globale \TT{ctl\_model}.%

Une autre chaîne de texte qui est passée est
"PRESS ANY KEY TO CONTINUE: ", mais le résultat n'est pas testé.
Difficile de dire pourquoi, probablement une erreur\footnote{C'est un sentiment
étrange de trouver un bug dans un logiciel aussi ancien.}.

Voyons où la valeur de la variable globale \TT{ctl\_model} est utilisée.

Un tel endroit est:

\lstinputlisting[style=customasmx86]{examples/dongles/2/4.lst}

Si c'est 0, un message d'erreur chiffré est passé à une routine de déchiffrement
et affiché.

\myindex{x86!\Instructions!XOR}

La routine de déchiffrement de la chaîne semble être un simple \glslink{xoring}{xor}:

\lstinputlisting[style=customasmx86]{examples/dongles/2/err_warn.lst}

C'est pourquoi nous étions incapable de trouver le message d'erreur dans les fichiers
exécutable, car ils sont chiffrés (ce qui est une pratique courante).

Un autre appel à la fonction de hachage \TT{SSQ()} lui passe la chaîne \q{offln}
et le résultat est comparé avec \TT{0xFE81} et \TT{0x12A9}.

Si ils ne correspondent pas, ça se comporte comme une sorte de fonction \TT{timer()}
(peut-être en attente qu'un dongle mal connecté soit reconnecté et re-testé?) et ensuite
déchiffre un autre message d'erreur à afficher.

\lstinputlisting[style=customasmx86]{examples/dongles/2/check2_EN.lst}

Passer outre le dongle est assez facile: il suffit de patcher tous les sauts après
les instructions \CMP pertinentes.

Une autre option est d'écrire notre propre driver SCO OpenServer, contenant une table
de questions et de réponses, toutes celles qui sont présentent dans le programme.

\subsubsection{Déchiffrer les messages d'erreur}

À propos, nous pouvons aussi essayer de déchiffrer tous les messages d'erreurs.
L'algorithme qui se trouve dans la fonction \TT{err\_warn()} est très simple, en effet:

\lstinputlisting[caption=Decryption function,style=customasmx86]{examples/dongles/2/decrypting_FR.lst}

Comme on le voit, non seulement la chaîne est transmise à la fonction de déchiffrement
mais aussi la clef:

\lstinputlisting[style=customasmx86]{examples/dongles/2/tmp1_EN.asm}

L'algorithme est un simple \glslink{xoring}{xor}: chaque octet est xoré avec la clef, mais
la clef est incrémentée de 3 après le traitement de chaque octet.

Nous pouvons écrire un petit script Python pour vérifier notre hypothèse:

\lstinputlisting[caption=Python 3.x]{examples/dongles/2/decr1.py}

Et il affiche: \q{check security device connection}.
Donc oui, ceci est le message déchiffré.

Il y a d'autres messages chiffrés, avec leur clef correspondante.
Mais inutile de dire qu'il est possible de les déchiffrer sans leur clef.
Premièrement, nous voyons que le clef est en fait un octet.
C'est parce que l'instruction principale de déchiffrement (\XOR) fonctionne au niveau
de l'octet.
La clef se trouve dans le registre \ESI, mais seulement une partie de \ESI d'un octet
est utilisée.
Ainsi, une clef pourrait être plus grande que 255, mais sa valeur est toujours arrondie.

En conséquence, nous pouvons simplement essayer de brute-forcer, en essayant toutes
les clefs possible dans l'intervalle 0..255.
Nous allons aussi écarter les messages comportants des caractères non-imprimable.

\lstinputlisting[caption=Python 3.x]{examples/dongles/2/decr2.py}

Et nous obtenons:

\lstinputlisting[caption=Results]{examples/dongles/2/decr2_result.txt}

Ici il y a un peu de déchet, mais nous pouvons rapidement trouver les messages en
anglais.

À propos, puisque l'algorithme est un simple chiffrement xor, la même fonction peut
être utilisée pour chiffrer les messages.
Si besoin, nous pouvons chiffrer nos propres messages, et patcher le programme en les insérant.
}

\RU{\subsection{Обменять входные значения друг с другом}

Вот так:

\lstinputlisting[style=customc]{patterns/061_pointers/swap/5_RU.c}

Как видим, байты загружаются в младшие 8-битные части регистров \TT{ECX} и \TT{EBX} используя \INS{MOVZX}
(так что старшие части регистров очищаются), затем байты записываются назад в другом порядке.

\lstinputlisting[style=customasmx86,caption=Optimizing GCC 5.4]{patterns/061_pointers/swap/5_GCC_O3_x86.s}

Адреса обоих байтов берутся из аргументов и во время исполнения ф-ции находятся в регистрах \TT{EDX} и \TT{EAX}.

Так что исопльзуем указатели --- вероятно, без них нет способа решить эту задачу лучше.

}
\EN{\mysection{Task manager practical joke (Windows Vista)}
\myindex{Windows!Windows Vista}

Let's see if it's possible to hack Task Manager slightly so it would detect more \ac{CPU} cores.

\myindex{Windows!NTAPI}

Let us first think, how does the Task Manager know the number of cores?

There is the \TT{GetSystemInfo()} win32 function present in win32 userspace which can tell us this.
But it's not imported in \TT{taskmgr.exe}.

There is, however, another one in \gls{NTAPI}, \TT{NtQuerySystemInformation()}, 
which is used in \TT{taskmgr.exe} in several places.

To get the number of cores, one has to call this function with the \TT{SystemBasicInformation} constant
as a first argument (which is zero
\footnote{\href{http://msdn.microsoft.com/en-us/library/windows/desktop/ms724509(v=vs.85).aspx}{MSDN}}).

The second argument has to point to the buffer which is getting all the information.

So we have to find all calls to the \\
\TT{NtQuerySystemInformation(0, ?, ?, ?)} function.
Let's open \TT{taskmgr.exe} in IDA. 
\myindex{Windows!PDB}

What is always good about Microsoft executables is that IDA can download the corresponding \gls{PDB} 
file for this executable and show all function names.

It is visible that Task Manager is written in \Cpp and some of the function names and classes are really 
speaking for themselves.
There are classes CAdapter, CNetPage, CPerfPage, CProcInfo, CProcPage, CSvcPage, 
CTaskPage, CUserPage.

Apparently, each class corresponds to each tab in Task Manager.

Let's visit each call and add comment with the value which is passed as the first function argument.
We will write \q{not zero} at some places, because the value there was clearly not zero, 
but something really different (more about this in the second part of this chapter).

And we are looking for zero passed as argument, after all.

\begin{figure}[H]
\centering
\myincludegraphics{examples/taskmgr/IDA_xrefs.png}
\caption{IDA: cross references to NtQuerySystemInformation()}
\end{figure}

Yes, the names are really speaking for themselves.

When we closely investigate each place where\\
\TT{NtQuerySystemInformation(0, ?, ?, ?)} is called,
we quickly find what we need in the \TT{InitPerfInfo()} function:

\lstinputlisting[caption=taskmgr.exe (Windows Vista),style=customasmx86]{examples/taskmgr/taskmgr.lst}

\TT{g\_cProcessors} is a global variable, and this name has been assigned by 
IDA according to the \gls{PDB} loaded from Microsoft's symbol server.

The byte is taken from \TT{var\_C20}. 
And \TT{var\_C58} is passed to\\
\TT{NtQuerySystemInformation()} 
as a pointer to the receiving buffer.
The difference between 0xC20 and 0xC58 is 0x38 (56).

Let's take a look at format of the return structure, which we can find in MSDN:

\begin{lstlisting}[style=customc]
typedef struct _SYSTEM_BASIC_INFORMATION {
    BYTE Reserved1[24];
    PVOID Reserved2[4];
    CCHAR NumberOfProcessors;
} SYSTEM_BASIC_INFORMATION;
\end{lstlisting}

This is a x64 system, so each PVOID takes 8 bytes.

All \emph{reserved} fields in the structure take $24+4*8=56$ bytes.

Oh yes, this implies that \TT{var\_C20} is the local stack is exactly the
\TT{NumberOfProcessors} field of the \TT{SYSTEM\_BASIC\_INFORMATION} structure.

Let's check our guess.
Copy \TT{taskmgr.exe} from \TT{C:\textbackslash{}Windows\textbackslash{}System32} 
to some other folder 
(so the \emph{Windows Resource Protection} 
will not try to restore the patched \TT{taskmgr.exe}).

Let's open it in Hiew and find the place:

\begin{figure}[H]
\centering
\myincludegraphics{examples/taskmgr/hiew2.png}
\caption{Hiew: find the place to be patched}
\end{figure}

Let's replace the \TT{MOVZX} instruction with ours.
Let's pretend we've got 64 CPU cores.

Add one additional \ac{NOP} (because our instruction is shorter than the original one):

\begin{figure}[H]
\centering
\myincludegraphics{examples/taskmgr/hiew1.png}
\caption{Hiew: patch it}
\end{figure}

And it works!
Of course, the data in the graphs is not correct.

At times, Task Manager even shows an overall CPU load of more than 100\%.

\begin{figure}[H]
\centering
\myincludegraphics{examples/taskmgr/taskmgr_64cpu_crop.png}
\caption{Fooled Windows Task Manager}
\end{figure}

The biggest number Task Manager does not crash with is 64.

Apparently, Task Manager in Windows Vista was not tested on computers with a large number of cores.

So there are probably some static data structure(s) inside it limited to 64 cores.

\subsection{Using LEA to load values}
\label{TaskMgr_LEA}

Sometimes, \TT{LEA} is used in \TT{taskmgr.exe} instead of \TT{MOV} to set the first argument of \\
\TT{NtQuerySystemInformation()}:

\lstinputlisting[caption=taskmgr.exe (Windows Vista),style=customasmx86]{examples/taskmgr/taskmgr2.lst}

\myindex{x86!\Instructions!LEA}

Perhaps \ac{MSVC} did so because machine code of \INS{LEA} is shorter than \INS{MOV REG, 5} (would be 5 instead of 4).

\INS{LEA} with offset in $-128..127$ range (offset will occupy 1 byte in opcode) with 32-bit registers is even shorter (for lack of REX prefix)---3 bytes.

Another example of such thing is: \myref{using_MOV_and_pack_of_LEA_to_load_values}.
}
\FR{\subsection{Exemple \#2: SCO OpenServer}

\label{examples_SCO}
\myindex{SCO OpenServer}
Un ancien logiciel pour SCO OpenServer de 1997 développé par une société qui a disparue
depuis longtemps.

Il y a un driver de dongle special à installer dans le système, qui contient les
chaînes de texte suivantes:
\q{Copyright 1989, Rainbow Technologies, Inc., Irvine, CA}
et
\q{Sentinel Integrated Driver Ver. 3.0 }.

Après l'installation du driver dans SCO OpenServer, ces fichiers apparaissent dans
l'arborescence /dev:

\begin{lstlisting}
/dev/rbsl8
/dev/rbsl9
/dev/rbsl10
\end{lstlisting}

Le programme renvoie une erreur lorsque le dongle n'est pas connecté, mais le message
d'erreur n'est pas trouvé dans les exécutables.

\myindex{COFF}

Grâce à \ac{IDA}, il est facile de charger l'exécutable COFF utilisé dans SCO OpenServer.

Essayons de trouver la chaîne \q{rbsl} et en effet, elle se trouve dans ce morceau
de code:

\lstinputlisting[style=customasmx86]{examples/dongles/2/1.lst}

Oui, en effet, le programme doit communiquer d'une façon ou d'une autre avec le driver.

\myindex{thunk-functions}
Le seul endroit où la fonction \TT{SSQC()} est appelée est dans la \glslink{thunk
 function}{fonction thunk}:

\lstinputlisting[style=customasmx86]{examples/dongles/2/2.lst}

SSQ() peut être appelé depuis au moins 2 fonctions.

L'une d'entre elles est:

\lstinputlisting[style=customasmx86]{examples/dongles/2/check1_EN.lst}

\q{\TT{3C}} et \q{\TT{3E}} semblent familiers: il y avait un dongle Sentinel Pro de
Rainbow sans mémoire, fournissant seulement une fonction de crypto-hachage secrète.

Vous pouvez lire une courte description de la fonction de hachage dont il s'agit
ici: \myref{hash_func}.

Mais retournons au programme.

Donc le programme peut seulement tester si un dongle est connecté ou s'il est absent.

Aucune autre information ne peut être écrite dans un tel dongle, puisqu'il n'a pas
de mémoire.
Les codes sur deux caractères sont des commandes (nous pouvons voir comment les commandes
sont traitées dans la fonction \TT{SSQC()}) et toutes les autres chaînes sont hachées
dans le dongle, transformées en un nombre 16-bit.
L'algorithme était secret, donc il n'était pas possible d'écrire un driver de remplacement
ou de refaire un dongle matériel qui l'émulerait parfaitement.

Toutefois, il est toujours possible d'intercepter tous les accès au dongle et de
trouver les constantes auxquelles les résultats de la fonction de hachage sont comparées.

Mais nous devons dire qu'il est possible de construire un schéma de logiciel de protection
de copie robuste basé sur une fonction secrète de hachage cryptographique: il suffit
qu'elle chiffre/déchiffre les fichiers de données utilisés par votre logiciel.

Mais retournons au code:

Les codes 51/52/53 sont utilisés pour choisir le port imprimante LPT.
3x/4x sont utilisés pour le choix de la \q{famille} (c'est ainsi que les dongles
Sentinel Pro sont différenciés les uns des autres: plus d'un dongle peut être connecté
sur un port LPT).

La seule chaîne passée à la fonction qui ne fasse pas 2 caractères est "0123456789".

Ensuite, le résultat est comparé à l'ensemble des résultats valides.

Si il est correct, 0xC ou 0xB est écrit dans la variable globale \TT{ctl\_model}.%

Une autre chaîne de texte qui est passée est
"PRESS ANY KEY TO CONTINUE: ", mais le résultat n'est pas testé.
Difficile de dire pourquoi, probablement une erreur\footnote{C'est un sentiment
étrange de trouver un bug dans un logiciel aussi ancien.}.

Voyons où la valeur de la variable globale \TT{ctl\_model} est utilisée.

Un tel endroit est:

\lstinputlisting[style=customasmx86]{examples/dongles/2/4.lst}

Si c'est 0, un message d'erreur chiffré est passé à une routine de déchiffrement
et affiché.

\myindex{x86!\Instructions!XOR}

La routine de déchiffrement de la chaîne semble être un simple \glslink{xoring}{xor}:

\lstinputlisting[style=customasmx86]{examples/dongles/2/err_warn.lst}

C'est pourquoi nous étions incapable de trouver le message d'erreur dans les fichiers
exécutable, car ils sont chiffrés (ce qui est une pratique courante).

Un autre appel à la fonction de hachage \TT{SSQ()} lui passe la chaîne \q{offln}
et le résultat est comparé avec \TT{0xFE81} et \TT{0x12A9}.

Si ils ne correspondent pas, ça se comporte comme une sorte de fonction \TT{timer()}
(peut-être en attente qu'un dongle mal connecté soit reconnecté et re-testé?) et ensuite
déchiffre un autre message d'erreur à afficher.

\lstinputlisting[style=customasmx86]{examples/dongles/2/check2_EN.lst}

Passer outre le dongle est assez facile: il suffit de patcher tous les sauts après
les instructions \CMP pertinentes.

Une autre option est d'écrire notre propre driver SCO OpenServer, contenant une table
de questions et de réponses, toutes celles qui sont présentent dans le programme.

\subsubsection{Déchiffrer les messages d'erreur}

À propos, nous pouvons aussi essayer de déchiffrer tous les messages d'erreurs.
L'algorithme qui se trouve dans la fonction \TT{err\_warn()} est très simple, en effet:

\lstinputlisting[caption=Decryption function,style=customasmx86]{examples/dongles/2/decrypting_FR.lst}

Comme on le voit, non seulement la chaîne est transmise à la fonction de déchiffrement
mais aussi la clef:

\lstinputlisting[style=customasmx86]{examples/dongles/2/tmp1_EN.asm}

L'algorithme est un simple \glslink{xoring}{xor}: chaque octet est xoré avec la clef, mais
la clef est incrémentée de 3 après le traitement de chaque octet.

Nous pouvons écrire un petit script Python pour vérifier notre hypothèse:

\lstinputlisting[caption=Python 3.x]{examples/dongles/2/decr1.py}

Et il affiche: \q{check security device connection}.
Donc oui, ceci est le message déchiffré.

Il y a d'autres messages chiffrés, avec leur clef correspondante.
Mais inutile de dire qu'il est possible de les déchiffrer sans leur clef.
Premièrement, nous voyons que le clef est en fait un octet.
C'est parce que l'instruction principale de déchiffrement (\XOR) fonctionne au niveau
de l'octet.
La clef se trouve dans le registre \ESI, mais seulement une partie de \ESI d'un octet
est utilisée.
Ainsi, une clef pourrait être plus grande que 255, mais sa valeur est toujours arrondie.

En conséquence, nous pouvons simplement essayer de brute-forcer, en essayant toutes
les clefs possible dans l'intervalle 0..255.
Nous allons aussi écarter les messages comportants des caractères non-imprimable.

\lstinputlisting[caption=Python 3.x]{examples/dongles/2/decr2.py}

Et nous obtenons:

\lstinputlisting[caption=Results]{examples/dongles/2/decr2_result.txt}

Ici il y a un peu de déchet, mais nous pouvons rapidement trouver les messages en
anglais.

À propos, puisque l'algorithme est un simple chiffrement xor, la même fonction peut
être utilisée pour chiffrer les messages.
Si besoin, nous pouvons chiffrer nos propres messages, et patcher le programme en les insérant.
}

\EN{\mysection{Some GCC library functions}
\myindex{GCC}
\label{sec:GCC_library_func}

%__ashldi3
%__ashrdi3
%__floatundidf
%__floatdisf
%__floatdixf
%__floatundidf
%__floatundisf
%__floatundixf
%__lshrdi3
%__muldi3

\begin{center}
\begin{tabular}{ | l | l | }
\hline
\HeaderColor name & \HeaderColor meaning \\
\hline \TT{\_\_divdi3} & signed division \\
\hline \TT{\_\_moddi3} & getting remainder (modulo) of signed division \\
\hline \TT{\_\_udivdi3} & unsigned division \\
\hline \TT{\_\_umoddi3} & getting remainder (modulo) of unsigned division \\
\hline
\end{tabular}
\end{center}

}
\RU{\mysection{Некоторые библиотечные функции GCC}
\myindex{GCC}
\label{sec:GCC_library_func}

%__ashldi3
%__ashrdi3
%__floatundidf
%__floatdisf
%__floatdixf
%__floatundidf
%__floatundisf
%__floatundixf
%__lshrdi3
%__muldi3

\begin{center}
\begin{tabular}{ | l | l | }
\hline
\HeaderColor имя & \HeaderColor значение \\
\hline \TT{\_\_divdi3} & знаковое деление \\
\hline \TT{\_\_moddi3} & остаток от знакового деления \\
\hline \TT{\_\_udivdi3} & беззнаковое деление \\
\hline \TT{\_\_umoddi3} & остаток от беззнакового деления \\
\hline
\end{tabular}
\end{center}

}
\FR{\mysection{Quelques fonctions de la bibliothèque de GCC}
\myindex{GCC}
\label{sec:GCC_library_func}

%__ashldi3
%__ashrdi3
%__floatundidf
%__floatdisf
%__floatdixf
%__floatundidf
%__floatundisf
%__floatundixf
%__lshrdi3
%__muldi3

\begin{center}
\begin{tabular}{ | l | l | }
\hline
\HeaderColor nom & \HeaderColor signification \\
\hline \TT{\_\_divdi3} & division signée \\
\hline \TT{\_\_moddi3} & reste (modulo) d'une division signée \\
\hline \TT{\_\_udivdi3} & division non signée \\
\hline \TT{\_\_umoddi3} & reste (modulo) d'une division non signée \\
\hline
\end{tabular}
\end{center}

}
\DE{\mysection{Einige GCC-Bibliotheks-Funktionen}
\myindex{GCC}
\label{sec:GCC_library_func}

%__ashldi3
%__ashrdi3
%__floatundidf
%__floatdisf
%__floatdixf
%__floatundidf
%__floatundisf
%__floatundixf
%__lshrdi3
%__muldi3

\begin{center}
\begin{tabular}{ | l | l | }
\hline
\HeaderColor Name & \HeaderColor Bedeutung \\
\hline \TT{\_\_divdi3} & vorzeichenbehaftete Division \\
\hline \TT{\_\_moddi3} & Rest (Modulo) einer vorzeichenbehafteten Division \\
\hline \TT{\_\_udivdi3} & vorzeichenlose Division \\
\hline \TT{\_\_umoddi3} & Rest (Modulo) einer vorzeichenlosen Division \\
\hline
\end{tabular}
\end{center}

}


\mysection{\RU{Некоторые библиотечные функции MSVC}\EN{Some MSVC library functions}\DE{Einige MSVC-Bibliotheks-Funktionen}%
\FR{Quelques fonctions de la bibliothèque MSVC}}
\myindex{MSVC}
\label{sec:MSVC_library_func}

\TT{ll} \RU{в имени функции означает}\EN{in function name stands for}\DE{in Funktionsnamen steht für}%
\FR{dans une fontion signifie} \q{long long}, \RU{т.е. 64-битный тип данных}
\EN{e.g., a 64-bit data type}\DE{z.B. einen 64-Bit-Datentyp}\FR{i.e., type de donées 64-bit}.

\begin{center}
\begin{tabular}{ | l | l | }
\hline
\HeaderColor \RU{имя}\EN{name}\DE{Name}\FR{nom} & \HeaderColor \RU{значение}\EN{meaning}\DE{Bedeutung}\FR{signification} \\
\hline \TT{\_\_alldiv} & \RU{знаковое деление}\EN{signed division}\DE{vorzeichenbehaftete Division}\FR{division signée} \\
\hline \TT{\_\_allmul} & \RU{умножение}\EN{multiplication}\DE{Multiplikation}\FR{multiplication} \\
\hline \TT{\_\_allrem} & \RU{остаток от знакового деления}\EN{remainder of signed division}\DE{Rest einer vorzeichenbehafteten Division}%
\FR{reste de la division signée} \\
\hline \TT{\_\_allshl} & \RU{сдвиг влево}\EN{shift left}\DE{Schiebe links}\FR{décalage à gauche} \\
\hline \TT{\_\_allshr} & \RU{знаковый сдвиг вправо}\EN{signed shift right}\DE{Schiebe links, vorzeichenbehaftet}%
\FR{décalage signé à droite} \\
\hline \TT{\_\_aulldiv} & \RU{беззнаковое деление}\EN{unsigned division}\DE{vorzeichenlose Division}%
\FR{division non signée} \\
\hline \TT{\_\_aullrem} & \RU{остаток от беззнакового деления}\EN{remainder of unsigned division}\DE{Rest (Modulo) einer vorzeichenlosen Division}%
\FR{reste de la division non signée} \\
\hline \TT{\_\_aullshr} & \RU{беззнаковый сдвиг вправо}\EN{unsigned shift right}\DE{Schiebe rechts, vorzeichenlos}%
\FR{décalage non signé à droite} \\
\hline
\end{tabular}
\end{center}

\RU{Процедуры умножения и сдвига влево, одни и те же и для знаковых чисел, и для беззнаковых,
поэтому здесь только одна функция для каждой операции}
\EN{Multiplication and shift left procedures are the same for both signed and unsigned numbers, hence there is only one function 
for each operation here}
\DE{Multiplikation und Links-Schiebebefehle sind sowohl für vorzeichenbehaftete als auch vorzeichenlose Zahlen,
da hier für jede Operation nur ein Befehl existiert}
\FR{La multiplication et le décalage à gauche sont similaire pour les nombres signés
et non signés, donc il n'y a qu'une seule fonction ici}. \\
\\
\RU{Исходные коды этих функций можно найти в установленной \ac{MSVS}, в}\EN{The source code of these function
can be found in the installed \ac{MSVS}, in}%
\DE{Der Quellcode dieser Funktionen kann im Pfad des installierten \ac{MSVS}, gefunden werden: }%
\FR{Le code source des ces fonctions peut être trouvé dans l'installation de \ac{MSVS},
dans} \TT{VC/crt/src/intel/*.asm}.


\mysection{Cheatsheets}

% sections
\subsection{IDA}
\myindex{IDA}
\label{sec:IDA_cheatsheet}

\ShortHotKeyCheatsheet:

\begin{center}
\begin{tabular}{ | l | l | }
\hline
\HeaderColor \RU{клавиша}\EN{key}\DE{Taste}\FR{touche} & \HeaderColor \RU{значение}\EN{meaning}\DE{Bedeutung}\FR{signification} \\
\hline
Space 	& \RU{переключать между листингом и просмотром кода в виде графа}
            \EN{switch listing and graph view}
            \DE{Zwischen Quellcode und grafischer Ansicht wechseln}%
				\FR{échanger le listing et le mode graphique} \\
C 	& \RU{конвертировать в код}\EN{convert to code}\DE{zu Code konvertieren}%
		\FR{convertir en code} \\
D 	& \RU{конвертировать в данные}\EN{convert to data}\DE{zu Daten konvertieren}%
		\FR{convertir en données} \\
A 	& \RU{конвертировать в строку}\EN{convert to string}\DE{zu Zeichenkette konvertieren}%
		\FR{convertir en chaîne} \\
* 	& \RU{конвертировать в массив}\EN{convert to array}\DE{zu Array konvertieren}%
		\FR{convertir en tableau} \\
U 	& \RU{сделать неопределенным}\EN{undefine}\DE{undefinieren}%
		\FR{rendre indéfini}\\
O 	& \RU{сделать смещение из операнда}\EN{make offset of operand}\DE{Offset von Operanden}%
		\FR{donner l'offset d'une opérande}\\
H 	& \RU{сделать десятичное число}\EN{make decimal number}\DE{Dezimalzahl erstellen}%
		\FR{transformer en nombre décimal} \\
R 	& \RU{сделать символ}\EN{make char}\DE{Zeichen erstellen}%
		\FR{transformer en caractère} \\
B 	& \RU{сделать двоичное число}\EN{make binary number}\DE{Binärzahl erstellen}%
		\FR{transformer en nombre binaire} \\
Q 	& \RU{сделать шестнадцатеричное число}\EN{make hexadecimal number}\DE{Hexadezimalzahl erstellen}%
		\FR{transformer en nombre hexa-décimal} \\
N 	& \RU{переименовать идентификатор}\EN{rename identifier}\DE{Identifikator umbenennen}%
		\FR{renommer l'identifiant} \\
? 	& \RU{калькулятор}\EN{calculator}\DE{Rechner}\FR{calculatrice} \\
G 	& \RU{переход на адрес}\EN{jump to address}\DE{zu Adresse springen}%
		\FR{sauter à l'adresse} \\
: 	& \RU{добавить комментарий}\EN{add comment}\DE{Kommentar einfügen}\FR{ajouter un commentaire} \\
Ctrl-X 	& \RU{показать ссылки на текущую функцию, метку, переменную}%
		\EN{show references to the current function, label, variable }%
		\DE{Referenz zu aktueller Funktion, Variable, ... zeigen}%
		\FR{montrer les références à la fonction, au label, à la variable courant} \\
	& \RU{(в т.ч., в стеке)}\EN{(incl. in local stack)}\DE{(inkl. lokalem Stack)}%
		\FR{inclure dans la pile locale} \\
X 	& \RU{показать ссылки на функцию, метку, переменную, итд}\EN{show references to the function, label, variable, etc.}%
		\DE{Referenz zu Funktion, Variable, ... zeigen}%
		\FR{montrer les références à la fonction, au label, à la variable, etc.} \\
Alt-I 	& \RU{искать константу}\EN{search for constant}\DE{Konstante suchen}%
		\FR{chercher une constante} \\
Ctrl-I 	& \RU{искать следующее вхождение константы}\EN{search for the next occurrence of constant}\DE{Nächstes Auftreten der Konstante suchen}%
		\FR{chercher la prochaine occurrence d'une constante} \\
Alt-B 	& \RU{искать последовательность байт}\EN{search for byte sequence}\DE{Byte-Sequenz suchen}%
		\FR{chercher une séquence d'octets} \\
Ctrl-B 	& \RU{искать следующее вхождение последовательности байт}
		\EN{search for the next occurrence of byte sequence}
		\DE{Nächstes Auftreten der Byte-Sequenz suchen}%
		\FR{chercher l'occurrence suivante d'une séquence d'octets} \\
Alt-T 	& \RU{искать текст (включая инструкции, итд.)}%
		\EN{search for text (including instructions, etc.)}%
		\DE{Text suchen (inkl. Anweisungen, usw.)}%
		\FR{chercher du texte (instructions incluses, etc.)} \\
Ctrl-T 	& \RU{искать следующее вхождение текста}%
		\EN{search for the next occurrence of text}%
		\DE{nächstes Aufreten des Textes suchen}%
		\FR{chercher l'occurrence suivante du texte} \\
Alt-P 	& \RU{редактировать текущую функцию}%
		\EN{edit current function}%
		\DE{akutelle Funktion editieren}%
		\FR{éditer la fonction courante} \\
Enter 	& \RU{перейти к функции, переменной, итд.}%
		\EN{jump to function, variable, etc.}%
		\DE{zu Funktion, Variable, ... springen}%
		\FR{sauter à la fonction, la variable, etc.} \\
Esc 	& \RU{вернуться назад}\EN{get back}\DE{zurückgehen}%
		\FR{retourner en arrière} \\
Num -   & \RU{свернуть функцию или отмеченную область}%
		\EN{fold function or selected area}%
		\DE{Funktion oder markierten Bereich einklappen}%
		\FR{cacher/plier la fonction ou la partie sélectionnée} \\
Num + 	& \RU{снова показать функцию или область}%
		\EN{unhide function or area}%
		\DE{Funktion oder Bereich anzeigen}%
		\FR{afficher la fonction ou une partie} \\
\hline
\end{tabular}
\end{center}

\RU{Сворачивание функции или области может быть удобно чтобы прятать те части функции,
чья функция вам стала уже ясна}%
\EN{Function/area folding may be useful for hiding function parts when you realize what they do}%
\DE{Das Einklappen ist nützlich um Teile von Funktionen zu verstecken, wenn bekannt ist was sie tun}%
\FR{cacher une fonction ou une partie de code peut être utile pour cacher des parties du
code lorsque vous avez compris ce qu'elles font}.
\RU{это используется в моем скрипте\footnote{\href{\YurichevIDAIDCScripts}{GitHub}}}\EN{this is used in my}\DE{dies wird genutzt im}%
\RU{для сворачивания некоторых очень часто используемых фрагментов inline-кода}%
\EN{script\footnote{\href{\YurichevIDAIDCScripts}{GitHub}} for hiding some often used patterns of inline code}%
\DE{Script\footnote{\href{\YurichevIDAIDCScripts}{GitHub}} um häufig genutzte Inline-Code-Stellen zu verstecken}%
\FR{ceci est utilisé dans mon script\footnote{\href{\YurichevIDAIDCScripts}{GitHub}}%
pour cacher des patterns de code inline souvent utilisés}.


\subsection{\olly}
\myindex{\olly}
\label{sec:Olly_cheatsheet}

\ShortHotKeyCheatsheet:

\begin{center}
\begin{tabular}{ | l | l | }
\hline
\HeaderColor \RU{хот-кей}\EN{hot-key}\DE{Tastenkürzel}\FR{raccourci} & 
\HeaderColor \RU{значение}\EN{meaning}\DE{Bedeutung}\FR{signification} \\
\hline
F7	& \RU{трассировать внутрь}\EN{trace into}\DE{Schritt}\FR{tracer dans la fonction}\\
F8	& \stepover\\
F9	& \RU{запуск}\EN{run}\DE{starten}\FR{démarrer}\\
Ctrl-F2	& \RU{перезапуск}\EN{restart}\DE{Neustart}\FR{redémarrer}\\
\hline
\end{tabular}
\end{center}

\subsection{MSVC}
\myindex{MSVC}
\label{sec:MSVC_options}

\RU{Некоторые полезные опции, которые были использованы в книге}
\EN{Some useful options which were used through this book}.
\DE{Einige nützliche Optionen die in diesem Buch genutzt werden}.
\FR{Quelques options utiles qui ont été utilisées dans ce livre}

\begin{center}
\begin{tabular}{ | l | l | }
\hline
\HeaderColor \RU{опция}\EN{option}\DE{Option}\FR{option} & 
\HeaderColor \RU{значение}\EN{meaning}\DE{Bedeutung}\FR{signification} \\
\hline
/O1		& \RU{оптимизация по размеру кода}\EN{minimize space}\DE{Speicherplatz minimieren}%
\FR{minimiser l'espace}\\
/Ob0		& \RU{не заменять вызовы inline-функций их кодом}\EN{no inline expansion}\DE{Keine Inline-Erweiterung}%
\FR{pas de mire en ligne}\\
/Ox		& \RU{максимальная оптимизация}\EN{maximum optimizations}\DE{maximale Optimierung}%
\FR{optimisation maximale}\\
/GS-		& \RU{отключить проверки переполнений буфера}
		\EN{disable security checks (buffer overflows)}
        \DE{Sicherheitsüberprüfungen deaktivieren (Buffer Overflows)}%
		\FR{désactiver les vérifications de sécurité (buffer overflows)}\\
/Fa(file)	& \RU{генерировать листинг на ассемблере}\EN{generate assembly listing}\DE{Assembler-Quelltext erstellen}%
\FR{générer un listing assembleur}\\
/Zi		& \RU{генерировать отладочную информацию}\EN{enable debugging information}\DE{Debugging-Informationen erstellen}%
\FR{activer les informations de débogage}\\
/Zp(n)		& \RU{паковать структуры по границе в $n$ байт}\EN{pack structs on $n$-byte boundary}\DE{Strukturen an $n$-Byte-Grenze ausrichten}%
\FR{aligner les structures sur une limite de $n$-octet}\\
/MD		& \RU{выходной исполняемый файл будет использовать}
			\EN{produced executable will use}%
            \DE{ausführbare Daten nutzt}%
\FR{l'exécutable généré utilisera} \TT{MSVCR*.DLL}\\
\hline
\end{tabular}
\end{center}

\RU{Кое-как информация о версиях MSVC}\EN{Some information about MSVC versions}\DE{Informationen zu MSVC-Versionen}%
\FR{Quelques informations sur les versions de MSVC}:
\myref{MSVC_versions}.


\EN{\subsection{GCC}
\myindex{GCC}

Some useful options which were used through this book.

\begin{center}
\begin{tabular}{ | l | l | }
\hline
\HeaderColor option & 
\HeaderColor meaning \\
\hline
-Os		& code size optimization \\
-O3		& maximum optimization \\
-regparm=	& how many arguments are to be passed in registers \\
-o file		& set name of output file \\
-g		& produce debugging information in resulting executable \\
-S		& generate assembly listing file \\
-masm=intel	& produce listing in Intel syntax \\
-fno-inline	& do not inline functions \\
\hline
\end{tabular}
\end{center}


}
\RU{\myparagraph{GCC 4.4.1}

\lstinputlisting[caption=GCC 4.4.1,style=customasmx86]{patterns/12_FPU/3_comparison/x86/GCC_RU.asm}

\myindex{x86!\Instructions!FUCOMPP}
\FUCOMPP~--- это почти то же что и \FCOM, только выкидывает из стека оба значения после сравнения, 
а также несколько иначе реагирует на \q{не-числа}.

\myindex{Не-числа (NaNs)}
Немного о \emph{не-числах}.

FPU умеет работать со специальными переменными, которые числами не являются и называются \q{не числа} или 
\gls{NaN}.
Это бесконечность, результат деления на ноль, и так далее. Нечисла бывают \q{тихие} и \q{сигнализирующие}. 
С первыми можно продолжать работать и далее, а вот если вы попытаетесь совершить какую-то операцию 
с сигнализирующим нечислом, то сработает исключение.

\myindex{x86!\Instructions!FCOM}
\myindex{x86!\Instructions!FUCOM}
Так вот, \FCOM вызовет исключение если любой из операндов какое-либо нечисло.
\FUCOM же вызовет исключение только если один из операндов именно \q{сигнализирующее нечисло}.

\myindex{x86!\Instructions!SAHF}
\label{SAHF}
Далее мы видим \SAHF (\emph{Store AH into Flags})~--- это довольно редкая инструкция в коде, не использующим FPU. 
8 бит из \AH перекладываются в младшие 8 бит регистра статуса процессора в таком порядке:

\input{SAHF_LAHF}

\myindex{x86!\Instructions!FNSTSW}
Вспомним, что \FNSTSW перегружает интересующие нас биты \CThreeBits в \AH, 
и соответственно они будут в позициях 6, 2, 0 в регистре \AH:

\input{C3_in_AH}

Иными словами, пара инструкций \INS{fnstsw  ax / sahf} перекладывает биты \CThreeBits в флаги \ZF, \PF, \CF.

Теперь снова вспомним, какие значения бит \CThreeBits будут при каких результатах сравнения:

\begin{itemize}
\item Если $a$ больше $b$ в нашем случае, то биты \CThreeBits должны быть выставлены так: 0, 0, 0.
\item Если $a$ меньше $b$, то биты будут выставлены так: 0, 0, 1.
\item Если $a=b$, то так: 1, 0, 0.
\end{itemize}
% TODO: table?

Иными словами, после трех инструкций \FUCOMPP/\FNSTSW/\SAHF возможны такие состояния флагов:

\begin{itemize}
\item Если $a>b$ в нашем случае, то флаги будут выставлены так: \GTT{ZF=0, PF=0, CF=0}.
\item Если $a<b$, то флаги будут выставлены так: \GTT{ZF=0, PF=0, CF=1}.
\item Если $a=b$, то так: \GTT{ZF=1, PF=0, CF=0}.
\end{itemize}
% TODO: table?

\myindex{x86!\Instructions!SETcc}
\myindex{x86!\Instructions!JNBE}
Инструкция \SETNBE выставит в \AL единицу или ноль в зависимости от флагов и условий. 
Это почти аналог \JNBE, за тем лишь исключением, что \SETcc
\footnote{\emph{cc} это \emph{condition code}}
выставляет 1 или 0 в \AL, а \Jcc делает переход или нет. 
\SETNBE запишет 1 только если \GTT{CF=0} и \GTT{ZF=0}. Если это не так, то запишет 0 в \AL.

\CF будет 0 и \ZF будет 0 одновременно только в одном случае: если $a>b$.

Тогда в \AL будет записана 1, последующий условный переход \JZ выполнен не будет 
и функция вернет~\GTT{\_a}. 
В остальных случаях, функция вернет~\GTT{\_b}.
}
\FR{\subsection{GCC}
\myindex{GCC}

Quelques options utiles qui ont été utilisées dans ce livre.

\begin{center}
\begin{tabular}{ | l | l | }
\hline
\HeaderColor option & 
\HeaderColor signification \\
\hline
-Os		& optimiser la taille du code \\
-O3		& optimisation maximale \\
-regparm=	& nombre d'arguments devant être passés dans les registres \\
-o file		& définir le nom du fichier de sortie \\
-g		& mettre l'information de débogage dans l'exécutable généré \\
-S		& générer un fichier assembleur \\
-masm=intel	& construire le code source en syntaxe Intel \\
-fno-inline	& ne pas mettre les fonctions en ligne \\
\hline
\end{tabular}
\end{center}


}
\DE{\myparagraph{GCC 4.4.1}

\lstinputlisting[caption=GCC 4.4.1,style=customasmx86]{patterns/12_FPU/3_comparison/x86/GCC_DE.asm}

\myindex{x86!\Instructions!FUCOMPP}
\FUCOMPP{} ist fast wie like \FCOM, nimmt aber beide Werte vom Stand und
behandelt \q{undefinierte Zahlenwerte} anders.


\myindex{Non-a-numbers (NaNs)}
Ein wenig über \emph{undefinierte Zahlenwerte}.

Die FPU ist in der Lage mit speziellen undefinieten Werten, den sogenannten
\emph{not-a-number}(kurz \gls{NaN}) umzugehen. Beispiele sind etwa der Wert
unendlich, das Ergebnis einer Division durch 0, etc. Undefinierte Werte können
entwder \q{quiet} oder \q{signaling} sein. Es ist möglich mit \q{quiet} NaNs zu
arbeiten, aber beim Versuch einen Befehl auf \q{signaling} NaNs auszuführen,
wird eine Exception geworfen. 

\myindex{x86!\Instructions!FCOM}
\myindex{x86!\Instructions!FUCOM}
\FCOM erzeugt eine Exception, falls irgendein Operand ein \gls{NaN} ist.
\FUCOM erzeugt eine Exception nur dann, wenn ein Operand eine \q{signaling}
\gls{NaN} (SNaN) ist.

\myindex{x86!\Instructions!SAHF}
\label{SAHF}
Der nächste Befehl ist \SAHF (\emph{Store AH into Flags})~---es handelt sich
hierbei um einen seltenen Befehl, der nicht mit der FPU zusammenhängt.
8 Bits aus AH werden in die niederen 8 Bit der CPU Flags in der folgenden
Reihenfolge verschoben:

\input{SAHF_LAHF}

\myindex{x86!\Instructions!FNSTSW}
Erinnern wir uns, dass \FNSTSW die für uns interessanten Bits (\CThreeBits) auf
den Stellen 6,2,0 im AH Register setzt:

\input{C3_in_AH}
Mit anderen Worten: der Befehl \INS{fnstsw ax / sahf} verschiebt \CThreeBits
nach \ZF, \PF und \CF. 

Überlegen wir uns auch die Werte der \CThreeBits in unterschiedlichen Szenarien:

\begin{itemize} 
  \item Falls in unserem Beispiel $a$ größer als $b$ ist, dann werden die
  \CThreeBits auf 0,0,0 gesetzt.
  \item Falls $a$ kleiner als $b$ ist, werden die Bits auf 0,0,1 gesetzt.
  \item Falls $a=b$, dann werden die Bits auf 1,0,0 gesetzt.
\end{itemize}
% TODO: table?
Mit anderen Worten, die folgenden Zustände der CPU Flags sind nach drei
\FUCOMPP/\FNSTSW/\SAHF Befehlen möglich:

\begin{itemize}
\item Falls $a>b$, werden die CPU Flags wie folgt gesetzt \GTT{ZF=0, PF=0,
CF=0}.
\item Falls $a<b$, werden die CPU Flags wie folgt gesetzt: \GTT{ZF=0, PF=0,
CF=1}.
\item Und falls $a=b$, dann gilt: \GTT{ZF=1, PF=0, CF=0}.
\end{itemize}
% TODO: table?

\myindex{x86!\Instructions!SETcc}
\myindex{x86!\Instructions!JNBE}
Abhängig von den CPU Flags und Bedingungen, speichert \SETNBE entweder 1 oder 0
in AL.
Es ist also quasi das Gegenstück von \JNBE mit dem Unterschied, dass \SETcc

Depending on the CPU flags and conditions, \SETNBE stores 1 or 0 to AL. 
It is almost the counterpart of \JNBE, with the exception that \SETcc
\footnote{\emph{cc} is \emph{condition code}} eine 1 oder 0 in \AL speichert, aber
\Jcc tatsächlich auch springt.
\SETNBE speicher 1 nur, falls \GTT{CF=0} und \GTT{ZF=0}.
Wenn dies nicht der Fall ist, dann wird 0 in \AL gespeichert.

Nur in einem Fall sind \CF und \ZF beide 0: falls $a>b$.

In diesem Fall wird 1 in \AL gespeichert, der nachfolgende \JZ Sprung wird nicht
ausgeführt und die Funktion liefert {\_a} zurück. In allen anderen Fällen wird
{\_b} zurückgegeben.
}

\subsection{GDB}
\myindex{GDB}
\label{sec:GDB_cheatsheet}

% FIXME: in Russian table doesn't fit!

\RU{Некоторые команды, которые были использованы в книге}\EN{Some of commands we used in this book}\DE{Einige nützliche Optionen die in diesem Buch genutzt werden}%
\FR{Quelques commandes que nous avons utilisées dans ce livre}:

\small
\begin{center}
\begin{tabular}{ | l | l | }
\hline
\HeaderColor \RU{опция}\EN{option}\DE{Option}\FR{option} & 
\HeaderColor \RU{значение}\EN{meaning}\DE{Bedeutung} \\
\hline
break filename.c:number		& \RU{установить точку останова на номере строки в исходном файле}
					\EN{set a breakpoint on line number in source code}
                    \DE{Setzen eines Breakpoints in der angegebenen Zeile}%
					\FR{mettre un point d'arrêt à la ligne number du code source} \\
break function			& \RU{установить точку останова на функции}\EN{set a breakpoint on function}\DE{Setzen eines Breakpoints in der Funktion}%
\FR{mettre un point d'arrêt sur une fonction} \\
break *address			& \RU{установить точку останова на адресе}\EN{set a breakpoint on address}\DE{Setzen eines Breakpoints auf Adresse}%
\FR{mettre un point d'arrêt à une adresse} \\
b				& \dittoclosing \\
p variable			& \RU{вывести значение переменной}\EN{print value of variable}\DE{Ausgabe eines Variablenwerts}%
\FR{afficher le contenu d'une variable} \\
run				& \RU{запустить}\EN{run}\DE{Starten}\FR{démarrer} \\
r				& \dittoclosing \\
cont				& \RU{продолжить исполнение}\EN{continue execution}\DE{Ausführung fortfahren}\FR{continuer l'exécution} \\
c				& \dittoclosing \\
bt				& \RU{вывести стек}\EN{print stack}\DE{Stack ausgeben}\FR{afficher la pile} \\
set disassembly-flavor intel	& \RU{установить Intel-синтаксис}\EN{set Intel syntax}\DE{Intel-Syntax nutzen}%
\FR{utiliser la syntaxe Intel} \\
disas				& disassemble current function \\
disas function			& \RU{дизассемблировать функцию}\EN{disassemble function}\DE{Funktion disassemblieren}\FR{désassembler la fonction} \\
disas function,+50		& disassemble portion \\
disas \$eip,+0x10		& \dittoclosing \\
disas/r				& \EN{disassemble with opcodes}\RU{дизассемблировать с опкодами}\DE{mit OpCodes disassemblieren}%
\FR{désassembler avec les opcodes} \\
info registers			& \RU{вывести все регистры}\EN{print all registers}\DE{Ausgabe aller Register}\FR{afficher tous les registres} \\
info float			& \RU{вывести FPU-регистры}\EN{print FPU-registers}\DE{Ausgabe der FPU-Register}\FR{afficher les registres FPU} \\
info locals			& \RU{вывести локальные переменные (если известны)}\EN{dump local variables (if known)}\DE{(bekannte) lokale Variablen ausgeben}%
\FR{afficher les variables locales} \\
x/w ...				& \RU{вывести память как 32-битные слова}\EN{dump memory as 32-bit word}\DE{Speicher als 32-Bit-Wort ausgeben}%
\FR{afficher la mémoire en mot de 32-bit} \\
x/w \$rdi			& \RU{вывести память как 32-битные слова}\EN{dump memory as 32-bit word}\DE{Speicher als 32-Bit-Wort ausgeben}%
\FR{afficher la mémoire en mot de 32-bit} \\
				& \RU{по адресу в \TT{RDI}}\EN{at address in \TT{RDI}}\DE{an Adresse in \TT{RDI}}\FR{à l'adresse dans \TT{RDI}} \\

x/10w ...			& \RU{вывести 10 слов памяти}\EN{dump 10 memory words}\DE{10 Speicherworte ausgeben}%
\FR{afficher 10 mots de la mémoire} \\
x/s ...				& \RU{вывести строку из памяти}\EN{dump memory as string}\DE{Speicher als Zeichenkette ausgeben}%
\FR{afficher la mémoire en tant que chaîne} \\
x/i ...				& \RU{трактовать память как код}\EN{dump memory as code}\DE{Speicher als Code ausgeben}%
\FR{afficher la mémoire en tant que code} \\
x/10c ...			& \RU{вывести 10 символов}\EN{dump 10 characters}\DE{10 Zeichen ausgeben}%
\FR{afficher 10 caractères} \\
x/b ...				& \RU{вывести байты}\EN{dump bytes}\DE{Bytes ausgeben}\FR{afficher des octets} \\
x/h ...				& \RU{вывести 16-битные полуслова}\EN{dump 16-bit halfwords}\DE{16-Bit-Halbworte ausgeben}%
\FR{afficher en demi-mots de 16-bit} \\
x/g ...				& \RU{вывести 64-битные слова}\EN{dump giant (64-bit) words}\DE{große (64-Bit-) Worte ausgeben}%
\FR{afficher des mots géants (64-bit)} \\
finish				& \RU{исполнять до конца функции}\EN{execute till the end of function}\DE{bis Funktionsende fortfahren}%
\FR{exécuter jusqu'à la fin de la fonction} \\
next				& \RU{следующая инструкция (не заходить в функции)}
					\EN{next instruction (don't dive into functions)}
					\DE{Nächste Anweisung (nicht in Funktion springen)}
					\FR{instruction suivante (ne pas descendre dans les fonctions)} \\
step				& \RU{следующая инструкция (заходить в функции)}
					\EN{next instruction (dive into functions)}
					\DE{Nächste Anweisung (in Funktion springen)}
					\FR{instruction suivante (descendre dans les fonctions)} \\
set step-mode on		& \RU{не использовать информацию о номерах строк при использовании команды step}
					\EN{do not use line number information while stepping}
					\DE{Beim schrittweisen Ausführen keine Zeilennummerninfos nutzen}
					\FR{ne pas utiliser l'information du numéro de ligne en exécutant pas à pas} \\
frame n				& \RU{переключить фрейм стека}\EN{switch stack frame}\DE{Stack-Frame tauschen}\FR{échanger la stack frame} \\
info break			& \RU{список точек останова}\EN{list of breakpoints}\DE{Breakpoints schauen}%
\FR{afficher les points d'arrêt} \\
del n				& \RU{удалить точку останова}\EN{delete breakpoint}\DE{Breakpoints löschen}\FR{effacer un point d'arrêt} \\
set args ...			& \RU{установить аргументы командной строки}\EN{set command-line arguments}\DE{Aufrufparameter setzen}%
\FR{définir les arguments de la ligne de commande} \\
\hline
\end{tabular}
\end{center}
\normalsize



}
\RU{\part*{\RU{Приложение}\EN{Appendix}\DE{Anhang}\FR{Appendice}\IT{Appendice}}
\appendix
\addcontentsline{toc}{part}{\RU{Приложение}\EN{Appendix}\DE{Anhang}\FR{Appendice}\IT{Appendice}}

% chapters
\EN{\mysection{Task manager practical joke (Windows Vista)}
\myindex{Windows!Windows Vista}

Let's see if it's possible to hack Task Manager slightly so it would detect more \ac{CPU} cores.

\myindex{Windows!NTAPI}

Let us first think, how does the Task Manager know the number of cores?

There is the \TT{GetSystemInfo()} win32 function present in win32 userspace which can tell us this.
But it's not imported in \TT{taskmgr.exe}.

There is, however, another one in \gls{NTAPI}, \TT{NtQuerySystemInformation()}, 
which is used in \TT{taskmgr.exe} in several places.

To get the number of cores, one has to call this function with the \TT{SystemBasicInformation} constant
as a first argument (which is zero
\footnote{\href{http://msdn.microsoft.com/en-us/library/windows/desktop/ms724509(v=vs.85).aspx}{MSDN}}).

The second argument has to point to the buffer which is getting all the information.

So we have to find all calls to the \\
\TT{NtQuerySystemInformation(0, ?, ?, ?)} function.
Let's open \TT{taskmgr.exe} in IDA. 
\myindex{Windows!PDB}

What is always good about Microsoft executables is that IDA can download the corresponding \gls{PDB} 
file for this executable and show all function names.

It is visible that Task Manager is written in \Cpp and some of the function names and classes are really 
speaking for themselves.
There are classes CAdapter, CNetPage, CPerfPage, CProcInfo, CProcPage, CSvcPage, 
CTaskPage, CUserPage.

Apparently, each class corresponds to each tab in Task Manager.

Let's visit each call and add comment with the value which is passed as the first function argument.
We will write \q{not zero} at some places, because the value there was clearly not zero, 
but something really different (more about this in the second part of this chapter).

And we are looking for zero passed as argument, after all.

\begin{figure}[H]
\centering
\myincludegraphics{examples/taskmgr/IDA_xrefs.png}
\caption{IDA: cross references to NtQuerySystemInformation()}
\end{figure}

Yes, the names are really speaking for themselves.

When we closely investigate each place where\\
\TT{NtQuerySystemInformation(0, ?, ?, ?)} is called,
we quickly find what we need in the \TT{InitPerfInfo()} function:

\lstinputlisting[caption=taskmgr.exe (Windows Vista),style=customasmx86]{examples/taskmgr/taskmgr.lst}

\TT{g\_cProcessors} is a global variable, and this name has been assigned by 
IDA according to the \gls{PDB} loaded from Microsoft's symbol server.

The byte is taken from \TT{var\_C20}. 
And \TT{var\_C58} is passed to\\
\TT{NtQuerySystemInformation()} 
as a pointer to the receiving buffer.
The difference between 0xC20 and 0xC58 is 0x38 (56).

Let's take a look at format of the return structure, which we can find in MSDN:

\begin{lstlisting}[style=customc]
typedef struct _SYSTEM_BASIC_INFORMATION {
    BYTE Reserved1[24];
    PVOID Reserved2[4];
    CCHAR NumberOfProcessors;
} SYSTEM_BASIC_INFORMATION;
\end{lstlisting}

This is a x64 system, so each PVOID takes 8 bytes.

All \emph{reserved} fields in the structure take $24+4*8=56$ bytes.

Oh yes, this implies that \TT{var\_C20} is the local stack is exactly the
\TT{NumberOfProcessors} field of the \TT{SYSTEM\_BASIC\_INFORMATION} structure.

Let's check our guess.
Copy \TT{taskmgr.exe} from \TT{C:\textbackslash{}Windows\textbackslash{}System32} 
to some other folder 
(so the \emph{Windows Resource Protection} 
will not try to restore the patched \TT{taskmgr.exe}).

Let's open it in Hiew and find the place:

\begin{figure}[H]
\centering
\myincludegraphics{examples/taskmgr/hiew2.png}
\caption{Hiew: find the place to be patched}
\end{figure}

Let's replace the \TT{MOVZX} instruction with ours.
Let's pretend we've got 64 CPU cores.

Add one additional \ac{NOP} (because our instruction is shorter than the original one):

\begin{figure}[H]
\centering
\myincludegraphics{examples/taskmgr/hiew1.png}
\caption{Hiew: patch it}
\end{figure}

And it works!
Of course, the data in the graphs is not correct.

At times, Task Manager even shows an overall CPU load of more than 100\%.

\begin{figure}[H]
\centering
\myincludegraphics{examples/taskmgr/taskmgr_64cpu_crop.png}
\caption{Fooled Windows Task Manager}
\end{figure}

The biggest number Task Manager does not crash with is 64.

Apparently, Task Manager in Windows Vista was not tested on computers with a large number of cores.

So there are probably some static data structure(s) inside it limited to 64 cores.

\subsection{Using LEA to load values}
\label{TaskMgr_LEA}

Sometimes, \TT{LEA} is used in \TT{taskmgr.exe} instead of \TT{MOV} to set the first argument of \\
\TT{NtQuerySystemInformation()}:

\lstinputlisting[caption=taskmgr.exe (Windows Vista),style=customasmx86]{examples/taskmgr/taskmgr2.lst}

\myindex{x86!\Instructions!LEA}

Perhaps \ac{MSVC} did so because machine code of \INS{LEA} is shorter than \INS{MOV REG, 5} (would be 5 instead of 4).

\INS{LEA} with offset in $-128..127$ range (offset will occupy 1 byte in opcode) with 32-bit registers is even shorter (for lack of REX prefix)---3 bytes.

Another example of such thing is: \myref{using_MOV_and_pack_of_LEA_to_load_values}.
}
\RU{\subsection{Обменять входные значения друг с другом}

Вот так:

\lstinputlisting[style=customc]{patterns/061_pointers/swap/5_RU.c}

Как видим, байты загружаются в младшие 8-битные части регистров \TT{ECX} и \TT{EBX} используя \INS{MOVZX}
(так что старшие части регистров очищаются), затем байты записываются назад в другом порядке.

\lstinputlisting[style=customasmx86,caption=Optimizing GCC 5.4]{patterns/061_pointers/swap/5_GCC_O3_x86.s}

Адреса обоих байтов берутся из аргументов и во время исполнения ф-ции находятся в регистрах \TT{EDX} и \TT{EAX}.

Так что исопльзуем указатели --- вероятно, без них нет способа решить эту задачу лучше.

}
\DE{\mysection{x86}

\subsection{Terminologie}

Geläufig für 16-Bit (8086/80286), 32-Bit (80386, etc.), 64-Bit.

\myindex{IEEE 754}
\myindex{MS-DOS}
\begin{description}
	\item[Byte] 8-Bit.
		Die DB Assembler-Direktive wird zum Definieren von Variablen und Arrays genutzt.
		Bytes werden in dem 8-Bit-Teil der folgenden Register übergeben:
		\TT{AL/BL/CL/DL/AH/BH/CH/DH/SIL/DIL/R*L}.
	\item[Wort] 16-Bit.
		DW Assembler-Direktive \dittoclosing.
		Bytes werden in dem 16-Bit-Teil der folgenden Register übergeben:
			\TT{AX/BX/CX/DX/SI/DI/R*W}.
	\item[Doppelwort] (\q{dword}) 32-Bit.
		DD Assembler-Direktive \dittoclosing.
		Doppelwörter werden in Registern (x86) oder dem 32-Bit-Teil der Register (x64) übergeben.
		In 16-Bit-Code werden Doppelwörter in 16-Bit-Registerpaaren übergeben.
	\item[zwei Doppelwörter] (\q{qword}) 64-Bit.
		DQ Assembler-Direktive \dittoclosing.
		In 32-Bit-Umgebungen werden diese in 32-Bit-Registerpaaren übergeben.
	\item[tbyte] (10 Byte) 80-Bit oder 10 Bytes (für IEEE 754 FPU Register).
	\item[paragraph] (16 Byte) --- Bezeichnung war in MS-DOS Umgebungen gebräuchlich.
\end{description}

\myindex{Windows!API}

Datentypen der selben Breite (BYTE, WORD, DWORD) entsprechen auch denen in der Windows \ac{API}.

% TODO German Translation (DSiekmeier)
%\input{appendix/x86/registers} % subsection
%\input{appendix/x86/instructions} % subsection
\subsection{npad}
\label{sec:npad}

\RU{Это макрос в ассемблере, для выравнивания некоторой метки по некоторой границе.}
\EN{It is an assembly language macro for aligning labels on a specific boundary.}
\DE{Dies ist ein Assembler-Makro um Labels an bestimmten Grenzen auszurichten.}
\FR{C'est une macro du langage d'assemblage pour aligner les labels sur une limite
spécifique.}

\RU{Это нужно для тех \emph{нагруженных} меток, куда чаще всего передается управление, например, 
начало тела цикла. 
Для того чтобы процессор мог эффективнее вытягивать данные или код из памяти, через шину с памятью, 
кэширование, итд.}
\EN{That's often needed for the busy labels to where the control flow is often passed, e.g., loop body starts.
So the CPU can load the data or code from the memory effectively, through the memory bus, cache lines, etc.}
\DE{Dies ist oft nützlich Labels, die oft Ziel einer Kotrollstruktur sind, wie Schleifenköpfe.
Somit kann die CPU Daten oder Code sehr effizient vom Speicher durch den Bus, den Cache, usw. laden.}
\FR{C'est souvent nécessaire pour des labels très utilisés, comme par exemple le
début d'un corps de boucle. Ainsi, le CPU peut charger les données ou le code depuis
la mémoire efficacement, à travers le bus mémoire, les caches, etc.}

\RU{Взято из}\EN{Taken from}\DE{Entnommen von}\FR{Pris de} \TT{listing.inc} (MSVC):

\myindex{x86!\Instructions!NOP}
\RU{Это, кстати, любопытный пример различных вариантов \NOP{}-ов. 
Все эти инструкции не дают никакого эффекта, но отличаются разной длиной.}
\EN{By the way, it is a curious example of the different \NOP variations.
All these instructions have no effects whatsoever, but have a different size.}
\DE{Dies ist übrigens ein Beispiel für die unterschiedlichen \NOP-Variationen.
Keine dieser Anweisungen hat eine Auswirkung, aber alle haben eine unterschiedliche Größe.}
\FR{À propos, c'est un exemple curieux des différentes variations de \NOP. Toutes
ces instructions n'ont pas d'effet, mais ont une taille différente.}

\RU{Цель в том, чтобы была только одна инструкция, а не набор NOP-ов, 
считается что так лучше для производительности CPU.}
\EN{Having a single idle instruction instead of couple of NOP-s,
is accepted to be better for CPU performance.}
\DE{Eine einzelne Idle-Anweisung anstatt mehrerer NOPs hat positive Auswirkungen
auf die CPU-Performance.}
\FR{Avoir une seule instruction sans effet au lieu de plusieurs est accepté comme
étant meilleur pour la performance du CPU.}

\begin{lstlisting}[style=customasmx86]
;; LISTING.INC
;;
;; This file contains assembler macros and is included by the files created
;; with the -FA compiler switch to be assembled by MASM (Microsoft Macro
;; Assembler).
;;
;; Copyright (c) 1993-2003, Microsoft Corporation. All rights reserved.

;; non destructive nops
npad macro size
if size eq 1
  nop
else
 if size eq 2
   mov edi, edi
 else
  if size eq 3
    ; lea ecx, [ecx+00]
    DB 8DH, 49H, 00H
  else
   if size eq 4
     ; lea esp, [esp+00]
     DB 8DH, 64H, 24H, 00H
   else
    if size eq 5
      add eax, DWORD PTR 0
    else
     if size eq 6
       ; lea ebx, [ebx+00000000]
       DB 8DH, 9BH, 00H, 00H, 00H, 00H
     else
      if size eq 7
	; lea esp, [esp+00000000]
	DB 8DH, 0A4H, 24H, 00H, 00H, 00H, 00H 
      else
       if size eq 8
        ; jmp .+8; .npad 6
	DB 0EBH, 06H, 8DH, 9BH, 00H, 00H, 00H, 00H
       else
        if size eq 9
         ; jmp .+9; .npad 7
         DB 0EBH, 07H, 8DH, 0A4H, 24H, 00H, 00H, 00H, 00H
        else
         if size eq 10
          ; jmp .+A; .npad 7; .npad 1
          DB 0EBH, 08H, 8DH, 0A4H, 24H, 00H, 00H, 00H, 00H, 90H
         else
          if size eq 11
           ; jmp .+B; .npad 7; .npad 2
           DB 0EBH, 09H, 8DH, 0A4H, 24H, 00H, 00H, 00H, 00H, 8BH, 0FFH
          else
           if size eq 12
            ; jmp .+C; .npad 7; .npad 3
            DB 0EBH, 0AH, 8DH, 0A4H, 24H, 00H, 00H, 00H, 00H, 8DH, 49H, 00H
           else
            if size eq 13
             ; jmp .+D; .npad 7; .npad 4
             DB 0EBH, 0BH, 8DH, 0A4H, 24H, 00H, 00H, 00H, 00H, 8DH, 64H, 24H, 00H
            else
             if size eq 14
              ; jmp .+E; .npad 7; .npad 5
              DB 0EBH, 0CH, 8DH, 0A4H, 24H, 00H, 00H, 00H, 00H, 05H, 00H, 00H, 00H, 00H
             else
              if size eq 15
               ; jmp .+F; .npad 7; .npad 6
               DB 0EBH, 0DH, 8DH, 0A4H, 24H, 00H, 00H, 00H, 00H, 8DH, 9BH, 00H, 00H, 00H, 00H
              else
	       %out error: unsupported npad size
               .err
              endif
             endif
            endif
           endif
          endif
         endif
        endif
       endif
      endif
     endif
    endif
   endif
  endif
 endif
endif
endm
\end{lstlisting}
 % subsection
}
\FR{\subsection{Exemple \#2: SCO OpenServer}

\label{examples_SCO}
\myindex{SCO OpenServer}
Un ancien logiciel pour SCO OpenServer de 1997 développé par une société qui a disparue
depuis longtemps.

Il y a un driver de dongle special à installer dans le système, qui contient les
chaînes de texte suivantes:
\q{Copyright 1989, Rainbow Technologies, Inc., Irvine, CA}
et
\q{Sentinel Integrated Driver Ver. 3.0 }.

Après l'installation du driver dans SCO OpenServer, ces fichiers apparaissent dans
l'arborescence /dev:

\begin{lstlisting}
/dev/rbsl8
/dev/rbsl9
/dev/rbsl10
\end{lstlisting}

Le programme renvoie une erreur lorsque le dongle n'est pas connecté, mais le message
d'erreur n'est pas trouvé dans les exécutables.

\myindex{COFF}

Grâce à \ac{IDA}, il est facile de charger l'exécutable COFF utilisé dans SCO OpenServer.

Essayons de trouver la chaîne \q{rbsl} et en effet, elle se trouve dans ce morceau
de code:

\lstinputlisting[style=customasmx86]{examples/dongles/2/1.lst}

Oui, en effet, le programme doit communiquer d'une façon ou d'une autre avec le driver.

\myindex{thunk-functions}
Le seul endroit où la fonction \TT{SSQC()} est appelée est dans la \glslink{thunk
 function}{fonction thunk}:

\lstinputlisting[style=customasmx86]{examples/dongles/2/2.lst}

SSQ() peut être appelé depuis au moins 2 fonctions.

L'une d'entre elles est:

\lstinputlisting[style=customasmx86]{examples/dongles/2/check1_EN.lst}

\q{\TT{3C}} et \q{\TT{3E}} semblent familiers: il y avait un dongle Sentinel Pro de
Rainbow sans mémoire, fournissant seulement une fonction de crypto-hachage secrète.

Vous pouvez lire une courte description de la fonction de hachage dont il s'agit
ici: \myref{hash_func}.

Mais retournons au programme.

Donc le programme peut seulement tester si un dongle est connecté ou s'il est absent.

Aucune autre information ne peut être écrite dans un tel dongle, puisqu'il n'a pas
de mémoire.
Les codes sur deux caractères sont des commandes (nous pouvons voir comment les commandes
sont traitées dans la fonction \TT{SSQC()}) et toutes les autres chaînes sont hachées
dans le dongle, transformées en un nombre 16-bit.
L'algorithme était secret, donc il n'était pas possible d'écrire un driver de remplacement
ou de refaire un dongle matériel qui l'émulerait parfaitement.

Toutefois, il est toujours possible d'intercepter tous les accès au dongle et de
trouver les constantes auxquelles les résultats de la fonction de hachage sont comparées.

Mais nous devons dire qu'il est possible de construire un schéma de logiciel de protection
de copie robuste basé sur une fonction secrète de hachage cryptographique: il suffit
qu'elle chiffre/déchiffre les fichiers de données utilisés par votre logiciel.

Mais retournons au code:

Les codes 51/52/53 sont utilisés pour choisir le port imprimante LPT.
3x/4x sont utilisés pour le choix de la \q{famille} (c'est ainsi que les dongles
Sentinel Pro sont différenciés les uns des autres: plus d'un dongle peut être connecté
sur un port LPT).

La seule chaîne passée à la fonction qui ne fasse pas 2 caractères est "0123456789".

Ensuite, le résultat est comparé à l'ensemble des résultats valides.

Si il est correct, 0xC ou 0xB est écrit dans la variable globale \TT{ctl\_model}.%

Une autre chaîne de texte qui est passée est
"PRESS ANY KEY TO CONTINUE: ", mais le résultat n'est pas testé.
Difficile de dire pourquoi, probablement une erreur\footnote{C'est un sentiment
étrange de trouver un bug dans un logiciel aussi ancien.}.

Voyons où la valeur de la variable globale \TT{ctl\_model} est utilisée.

Un tel endroit est:

\lstinputlisting[style=customasmx86]{examples/dongles/2/4.lst}

Si c'est 0, un message d'erreur chiffré est passé à une routine de déchiffrement
et affiché.

\myindex{x86!\Instructions!XOR}

La routine de déchiffrement de la chaîne semble être un simple \glslink{xoring}{xor}:

\lstinputlisting[style=customasmx86]{examples/dongles/2/err_warn.lst}

C'est pourquoi nous étions incapable de trouver le message d'erreur dans les fichiers
exécutable, car ils sont chiffrés (ce qui est une pratique courante).

Un autre appel à la fonction de hachage \TT{SSQ()} lui passe la chaîne \q{offln}
et le résultat est comparé avec \TT{0xFE81} et \TT{0x12A9}.

Si ils ne correspondent pas, ça se comporte comme une sorte de fonction \TT{timer()}
(peut-être en attente qu'un dongle mal connecté soit reconnecté et re-testé?) et ensuite
déchiffre un autre message d'erreur à afficher.

\lstinputlisting[style=customasmx86]{examples/dongles/2/check2_EN.lst}

Passer outre le dongle est assez facile: il suffit de patcher tous les sauts après
les instructions \CMP pertinentes.

Une autre option est d'écrire notre propre driver SCO OpenServer, contenant une table
de questions et de réponses, toutes celles qui sont présentent dans le programme.

\subsubsection{Déchiffrer les messages d'erreur}

À propos, nous pouvons aussi essayer de déchiffrer tous les messages d'erreurs.
L'algorithme qui se trouve dans la fonction \TT{err\_warn()} est très simple, en effet:

\lstinputlisting[caption=Decryption function,style=customasmx86]{examples/dongles/2/decrypting_FR.lst}

Comme on le voit, non seulement la chaîne est transmise à la fonction de déchiffrement
mais aussi la clef:

\lstinputlisting[style=customasmx86]{examples/dongles/2/tmp1_EN.asm}

L'algorithme est un simple \glslink{xoring}{xor}: chaque octet est xoré avec la clef, mais
la clef est incrémentée de 3 après le traitement de chaque octet.

Nous pouvons écrire un petit script Python pour vérifier notre hypothèse:

\lstinputlisting[caption=Python 3.x]{examples/dongles/2/decr1.py}

Et il affiche: \q{check security device connection}.
Donc oui, ceci est le message déchiffré.

Il y a d'autres messages chiffrés, avec leur clef correspondante.
Mais inutile de dire qu'il est possible de les déchiffrer sans leur clef.
Premièrement, nous voyons que le clef est en fait un octet.
C'est parce que l'instruction principale de déchiffrement (\XOR) fonctionne au niveau
de l'octet.
La clef se trouve dans le registre \ESI, mais seulement une partie de \ESI d'un octet
est utilisée.
Ainsi, une clef pourrait être plus grande que 255, mais sa valeur est toujours arrondie.

En conséquence, nous pouvons simplement essayer de brute-forcer, en essayant toutes
les clefs possible dans l'intervalle 0..255.
Nous allons aussi écarter les messages comportants des caractères non-imprimable.

\lstinputlisting[caption=Python 3.x]{examples/dongles/2/decr2.py}

Et nous obtenons:

\lstinputlisting[caption=Results]{examples/dongles/2/decr2_result.txt}

Ici il y a un peu de déchet, mais nous pouvons rapidement trouver les messages en
anglais.

À propos, puisque l'algorithme est un simple chiffrement xor, la même fonction peut
être utilisée pour chiffrer les messages.
Si besoin, nous pouvons chiffrer nos propres messages, et patcher le programme en les insérant.
}
\IT{\subsection{Fall-through}

Un altro uso diffuso dell'operatore \TT{switch()} è il cosiddetto \q{fallthrough}.
Ecco un semplice esempio \footnote{Preso da \url{https://github.com/azonalon/prgraas/blob/master/prog1lib/lecture_examples/is_whitespace.c}}:

\lstinputlisting[numbers=left,style=customc]{patterns/08_switch/4_fallthrough/fallthrough1.c}

Uno leggermente più difficile, dal kernel di Linux \footnote{Preso da \url{https://github.com/torvalds/linux/blob/master/drivers/media/dvb-frontends/lgdt3306a.c}}:

\lstinputlisting[numbers=left,style=customc]{patterns/08_switch/4_fallthrough/fallthrough2.c}

\lstinputlisting[caption=Optimizing GCC 5.4.0 x86,numbers=left,style=customasmx86]{patterns/08_switch/4_fallthrough/fallthrough2.s}

Possiamo arrivare alla label \TT{.L5} se all'input della funzione viene dato il valore 3250.
Ma si può anche giungere allo stesso punto da un altro percorso:
notiamo che non ci sono jump tra la chiamata a \printf e la label \TT{.L5}.

Questo spiega facilmente perchè i costrutti con \emph{switch()} sono spesso fonte di bug:
è sufficiente dimenticare un \emph{break} per trasformare il costrutto \emph{switch()} in un \emph{fallthrough} , in cui vengono eseguiti
più blocchi invece di uno solo.
}

\EN{\mysection{Task manager practical joke (Windows Vista)}
\myindex{Windows!Windows Vista}

Let's see if it's possible to hack Task Manager slightly so it would detect more \ac{CPU} cores.

\myindex{Windows!NTAPI}

Let us first think, how does the Task Manager know the number of cores?

There is the \TT{GetSystemInfo()} win32 function present in win32 userspace which can tell us this.
But it's not imported in \TT{taskmgr.exe}.

There is, however, another one in \gls{NTAPI}, \TT{NtQuerySystemInformation()}, 
which is used in \TT{taskmgr.exe} in several places.

To get the number of cores, one has to call this function with the \TT{SystemBasicInformation} constant
as a first argument (which is zero
\footnote{\href{http://msdn.microsoft.com/en-us/library/windows/desktop/ms724509(v=vs.85).aspx}{MSDN}}).

The second argument has to point to the buffer which is getting all the information.

So we have to find all calls to the \\
\TT{NtQuerySystemInformation(0, ?, ?, ?)} function.
Let's open \TT{taskmgr.exe} in IDA. 
\myindex{Windows!PDB}

What is always good about Microsoft executables is that IDA can download the corresponding \gls{PDB} 
file for this executable and show all function names.

It is visible that Task Manager is written in \Cpp and some of the function names and classes are really 
speaking for themselves.
There are classes CAdapter, CNetPage, CPerfPage, CProcInfo, CProcPage, CSvcPage, 
CTaskPage, CUserPage.

Apparently, each class corresponds to each tab in Task Manager.

Let's visit each call and add comment with the value which is passed as the first function argument.
We will write \q{not zero} at some places, because the value there was clearly not zero, 
but something really different (more about this in the second part of this chapter).

And we are looking for zero passed as argument, after all.

\begin{figure}[H]
\centering
\myincludegraphics{examples/taskmgr/IDA_xrefs.png}
\caption{IDA: cross references to NtQuerySystemInformation()}
\end{figure}

Yes, the names are really speaking for themselves.

When we closely investigate each place where\\
\TT{NtQuerySystemInformation(0, ?, ?, ?)} is called,
we quickly find what we need in the \TT{InitPerfInfo()} function:

\lstinputlisting[caption=taskmgr.exe (Windows Vista),style=customasmx86]{examples/taskmgr/taskmgr.lst}

\TT{g\_cProcessors} is a global variable, and this name has been assigned by 
IDA according to the \gls{PDB} loaded from Microsoft's symbol server.

The byte is taken from \TT{var\_C20}. 
And \TT{var\_C58} is passed to\\
\TT{NtQuerySystemInformation()} 
as a pointer to the receiving buffer.
The difference between 0xC20 and 0xC58 is 0x38 (56).

Let's take a look at format of the return structure, which we can find in MSDN:

\begin{lstlisting}[style=customc]
typedef struct _SYSTEM_BASIC_INFORMATION {
    BYTE Reserved1[24];
    PVOID Reserved2[4];
    CCHAR NumberOfProcessors;
} SYSTEM_BASIC_INFORMATION;
\end{lstlisting}

This is a x64 system, so each PVOID takes 8 bytes.

All \emph{reserved} fields in the structure take $24+4*8=56$ bytes.

Oh yes, this implies that \TT{var\_C20} is the local stack is exactly the
\TT{NumberOfProcessors} field of the \TT{SYSTEM\_BASIC\_INFORMATION} structure.

Let's check our guess.
Copy \TT{taskmgr.exe} from \TT{C:\textbackslash{}Windows\textbackslash{}System32} 
to some other folder 
(so the \emph{Windows Resource Protection} 
will not try to restore the patched \TT{taskmgr.exe}).

Let's open it in Hiew and find the place:

\begin{figure}[H]
\centering
\myincludegraphics{examples/taskmgr/hiew2.png}
\caption{Hiew: find the place to be patched}
\end{figure}

Let's replace the \TT{MOVZX} instruction with ours.
Let's pretend we've got 64 CPU cores.

Add one additional \ac{NOP} (because our instruction is shorter than the original one):

\begin{figure}[H]
\centering
\myincludegraphics{examples/taskmgr/hiew1.png}
\caption{Hiew: patch it}
\end{figure}

And it works!
Of course, the data in the graphs is not correct.

At times, Task Manager even shows an overall CPU load of more than 100\%.

\begin{figure}[H]
\centering
\myincludegraphics{examples/taskmgr/taskmgr_64cpu_crop.png}
\caption{Fooled Windows Task Manager}
\end{figure}

The biggest number Task Manager does not crash with is 64.

Apparently, Task Manager in Windows Vista was not tested on computers with a large number of cores.

So there are probably some static data structure(s) inside it limited to 64 cores.

\subsection{Using LEA to load values}
\label{TaskMgr_LEA}

Sometimes, \TT{LEA} is used in \TT{taskmgr.exe} instead of \TT{MOV} to set the first argument of \\
\TT{NtQuerySystemInformation()}:

\lstinputlisting[caption=taskmgr.exe (Windows Vista),style=customasmx86]{examples/taskmgr/taskmgr2.lst}

\myindex{x86!\Instructions!LEA}

Perhaps \ac{MSVC} did so because machine code of \INS{LEA} is shorter than \INS{MOV REG, 5} (would be 5 instead of 4).

\INS{LEA} with offset in $-128..127$ range (offset will occupy 1 byte in opcode) with 32-bit registers is even shorter (for lack of REX prefix)---3 bytes.

Another example of such thing is: \myref{using_MOV_and_pack_of_LEA_to_load_values}.
}
\RU{\subsection{Обменять входные значения друг с другом}

Вот так:

\lstinputlisting[style=customc]{patterns/061_pointers/swap/5_RU.c}

Как видим, байты загружаются в младшие 8-битные части регистров \TT{ECX} и \TT{EBX} используя \INS{MOVZX}
(так что старшие части регистров очищаются), затем байты записываются назад в другом порядке.

\lstinputlisting[style=customasmx86,caption=Optimizing GCC 5.4]{patterns/061_pointers/swap/5_GCC_O3_x86.s}

Адреса обоих байтов берутся из аргументов и во время исполнения ф-ции находятся в регистрах \TT{EDX} и \TT{EAX}.

Так что исопльзуем указатели --- вероятно, без них нет способа решить эту задачу лучше.

}
\FR{\subsection{Exemple \#2: SCO OpenServer}

\label{examples_SCO}
\myindex{SCO OpenServer}
Un ancien logiciel pour SCO OpenServer de 1997 développé par une société qui a disparue
depuis longtemps.

Il y a un driver de dongle special à installer dans le système, qui contient les
chaînes de texte suivantes:
\q{Copyright 1989, Rainbow Technologies, Inc., Irvine, CA}
et
\q{Sentinel Integrated Driver Ver. 3.0 }.

Après l'installation du driver dans SCO OpenServer, ces fichiers apparaissent dans
l'arborescence /dev:

\begin{lstlisting}
/dev/rbsl8
/dev/rbsl9
/dev/rbsl10
\end{lstlisting}

Le programme renvoie une erreur lorsque le dongle n'est pas connecté, mais le message
d'erreur n'est pas trouvé dans les exécutables.

\myindex{COFF}

Grâce à \ac{IDA}, il est facile de charger l'exécutable COFF utilisé dans SCO OpenServer.

Essayons de trouver la chaîne \q{rbsl} et en effet, elle se trouve dans ce morceau
de code:

\lstinputlisting[style=customasmx86]{examples/dongles/2/1.lst}

Oui, en effet, le programme doit communiquer d'une façon ou d'une autre avec le driver.

\myindex{thunk-functions}
Le seul endroit où la fonction \TT{SSQC()} est appelée est dans la \glslink{thunk
 function}{fonction thunk}:

\lstinputlisting[style=customasmx86]{examples/dongles/2/2.lst}

SSQ() peut être appelé depuis au moins 2 fonctions.

L'une d'entre elles est:

\lstinputlisting[style=customasmx86]{examples/dongles/2/check1_EN.lst}

\q{\TT{3C}} et \q{\TT{3E}} semblent familiers: il y avait un dongle Sentinel Pro de
Rainbow sans mémoire, fournissant seulement une fonction de crypto-hachage secrète.

Vous pouvez lire une courte description de la fonction de hachage dont il s'agit
ici: \myref{hash_func}.

Mais retournons au programme.

Donc le programme peut seulement tester si un dongle est connecté ou s'il est absent.

Aucune autre information ne peut être écrite dans un tel dongle, puisqu'il n'a pas
de mémoire.
Les codes sur deux caractères sont des commandes (nous pouvons voir comment les commandes
sont traitées dans la fonction \TT{SSQC()}) et toutes les autres chaînes sont hachées
dans le dongle, transformées en un nombre 16-bit.
L'algorithme était secret, donc il n'était pas possible d'écrire un driver de remplacement
ou de refaire un dongle matériel qui l'émulerait parfaitement.

Toutefois, il est toujours possible d'intercepter tous les accès au dongle et de
trouver les constantes auxquelles les résultats de la fonction de hachage sont comparées.

Mais nous devons dire qu'il est possible de construire un schéma de logiciel de protection
de copie robuste basé sur une fonction secrète de hachage cryptographique: il suffit
qu'elle chiffre/déchiffre les fichiers de données utilisés par votre logiciel.

Mais retournons au code:

Les codes 51/52/53 sont utilisés pour choisir le port imprimante LPT.
3x/4x sont utilisés pour le choix de la \q{famille} (c'est ainsi que les dongles
Sentinel Pro sont différenciés les uns des autres: plus d'un dongle peut être connecté
sur un port LPT).

La seule chaîne passée à la fonction qui ne fasse pas 2 caractères est "0123456789".

Ensuite, le résultat est comparé à l'ensemble des résultats valides.

Si il est correct, 0xC ou 0xB est écrit dans la variable globale \TT{ctl\_model}.%

Une autre chaîne de texte qui est passée est
"PRESS ANY KEY TO CONTINUE: ", mais le résultat n'est pas testé.
Difficile de dire pourquoi, probablement une erreur\footnote{C'est un sentiment
étrange de trouver un bug dans un logiciel aussi ancien.}.

Voyons où la valeur de la variable globale \TT{ctl\_model} est utilisée.

Un tel endroit est:

\lstinputlisting[style=customasmx86]{examples/dongles/2/4.lst}

Si c'est 0, un message d'erreur chiffré est passé à une routine de déchiffrement
et affiché.

\myindex{x86!\Instructions!XOR}

La routine de déchiffrement de la chaîne semble être un simple \glslink{xoring}{xor}:

\lstinputlisting[style=customasmx86]{examples/dongles/2/err_warn.lst}

C'est pourquoi nous étions incapable de trouver le message d'erreur dans les fichiers
exécutable, car ils sont chiffrés (ce qui est une pratique courante).

Un autre appel à la fonction de hachage \TT{SSQ()} lui passe la chaîne \q{offln}
et le résultat est comparé avec \TT{0xFE81} et \TT{0x12A9}.

Si ils ne correspondent pas, ça se comporte comme une sorte de fonction \TT{timer()}
(peut-être en attente qu'un dongle mal connecté soit reconnecté et re-testé?) et ensuite
déchiffre un autre message d'erreur à afficher.

\lstinputlisting[style=customasmx86]{examples/dongles/2/check2_EN.lst}

Passer outre le dongle est assez facile: il suffit de patcher tous les sauts après
les instructions \CMP pertinentes.

Une autre option est d'écrire notre propre driver SCO OpenServer, contenant une table
de questions et de réponses, toutes celles qui sont présentent dans le programme.

\subsubsection{Déchiffrer les messages d'erreur}

À propos, nous pouvons aussi essayer de déchiffrer tous les messages d'erreurs.
L'algorithme qui se trouve dans la fonction \TT{err\_warn()} est très simple, en effet:

\lstinputlisting[caption=Decryption function,style=customasmx86]{examples/dongles/2/decrypting_FR.lst}

Comme on le voit, non seulement la chaîne est transmise à la fonction de déchiffrement
mais aussi la clef:

\lstinputlisting[style=customasmx86]{examples/dongles/2/tmp1_EN.asm}

L'algorithme est un simple \glslink{xoring}{xor}: chaque octet est xoré avec la clef, mais
la clef est incrémentée de 3 après le traitement de chaque octet.

Nous pouvons écrire un petit script Python pour vérifier notre hypothèse:

\lstinputlisting[caption=Python 3.x]{examples/dongles/2/decr1.py}

Et il affiche: \q{check security device connection}.
Donc oui, ceci est le message déchiffré.

Il y a d'autres messages chiffrés, avec leur clef correspondante.
Mais inutile de dire qu'il est possible de les déchiffrer sans leur clef.
Premièrement, nous voyons que le clef est en fait un octet.
C'est parce que l'instruction principale de déchiffrement (\XOR) fonctionne au niveau
de l'octet.
La clef se trouve dans le registre \ESI, mais seulement une partie de \ESI d'un octet
est utilisée.
Ainsi, une clef pourrait être plus grande que 255, mais sa valeur est toujours arrondie.

En conséquence, nous pouvons simplement essayer de brute-forcer, en essayant toutes
les clefs possible dans l'intervalle 0..255.
Nous allons aussi écarter les messages comportants des caractères non-imprimable.

\lstinputlisting[caption=Python 3.x]{examples/dongles/2/decr2.py}

Et nous obtenons:

\lstinputlisting[caption=Results]{examples/dongles/2/decr2_result.txt}

Ici il y a un peu de déchet, mais nous pouvons rapidement trouver les messages en
anglais.

À propos, puisque l'algorithme est un simple chiffrement xor, la même fonction peut
être utilisée pour chiffrer les messages.
Si besoin, nous pouvons chiffrer nos propres messages, et patcher le programme en les insérant.
}

\RU{\subsection{Обменять входные значения друг с другом}

Вот так:

\lstinputlisting[style=customc]{patterns/061_pointers/swap/5_RU.c}

Как видим, байты загружаются в младшие 8-битные части регистров \TT{ECX} и \TT{EBX} используя \INS{MOVZX}
(так что старшие части регистров очищаются), затем байты записываются назад в другом порядке.

\lstinputlisting[style=customasmx86,caption=Optimizing GCC 5.4]{patterns/061_pointers/swap/5_GCC_O3_x86.s}

Адреса обоих байтов берутся из аргументов и во время исполнения ф-ции находятся в регистрах \TT{EDX} и \TT{EAX}.

Так что исопльзуем указатели --- вероятно, без них нет способа решить эту задачу лучше.

}
\EN{\mysection{Task manager practical joke (Windows Vista)}
\myindex{Windows!Windows Vista}

Let's see if it's possible to hack Task Manager slightly so it would detect more \ac{CPU} cores.

\myindex{Windows!NTAPI}

Let us first think, how does the Task Manager know the number of cores?

There is the \TT{GetSystemInfo()} win32 function present in win32 userspace which can tell us this.
But it's not imported in \TT{taskmgr.exe}.

There is, however, another one in \gls{NTAPI}, \TT{NtQuerySystemInformation()}, 
which is used in \TT{taskmgr.exe} in several places.

To get the number of cores, one has to call this function with the \TT{SystemBasicInformation} constant
as a first argument (which is zero
\footnote{\href{http://msdn.microsoft.com/en-us/library/windows/desktop/ms724509(v=vs.85).aspx}{MSDN}}).

The second argument has to point to the buffer which is getting all the information.

So we have to find all calls to the \\
\TT{NtQuerySystemInformation(0, ?, ?, ?)} function.
Let's open \TT{taskmgr.exe} in IDA. 
\myindex{Windows!PDB}

What is always good about Microsoft executables is that IDA can download the corresponding \gls{PDB} 
file for this executable and show all function names.

It is visible that Task Manager is written in \Cpp and some of the function names and classes are really 
speaking for themselves.
There are classes CAdapter, CNetPage, CPerfPage, CProcInfo, CProcPage, CSvcPage, 
CTaskPage, CUserPage.

Apparently, each class corresponds to each tab in Task Manager.

Let's visit each call and add comment with the value which is passed as the first function argument.
We will write \q{not zero} at some places, because the value there was clearly not zero, 
but something really different (more about this in the second part of this chapter).

And we are looking for zero passed as argument, after all.

\begin{figure}[H]
\centering
\myincludegraphics{examples/taskmgr/IDA_xrefs.png}
\caption{IDA: cross references to NtQuerySystemInformation()}
\end{figure}

Yes, the names are really speaking for themselves.

When we closely investigate each place where\\
\TT{NtQuerySystemInformation(0, ?, ?, ?)} is called,
we quickly find what we need in the \TT{InitPerfInfo()} function:

\lstinputlisting[caption=taskmgr.exe (Windows Vista),style=customasmx86]{examples/taskmgr/taskmgr.lst}

\TT{g\_cProcessors} is a global variable, and this name has been assigned by 
IDA according to the \gls{PDB} loaded from Microsoft's symbol server.

The byte is taken from \TT{var\_C20}. 
And \TT{var\_C58} is passed to\\
\TT{NtQuerySystemInformation()} 
as a pointer to the receiving buffer.
The difference between 0xC20 and 0xC58 is 0x38 (56).

Let's take a look at format of the return structure, which we can find in MSDN:

\begin{lstlisting}[style=customc]
typedef struct _SYSTEM_BASIC_INFORMATION {
    BYTE Reserved1[24];
    PVOID Reserved2[4];
    CCHAR NumberOfProcessors;
} SYSTEM_BASIC_INFORMATION;
\end{lstlisting}

This is a x64 system, so each PVOID takes 8 bytes.

All \emph{reserved} fields in the structure take $24+4*8=56$ bytes.

Oh yes, this implies that \TT{var\_C20} is the local stack is exactly the
\TT{NumberOfProcessors} field of the \TT{SYSTEM\_BASIC\_INFORMATION} structure.

Let's check our guess.
Copy \TT{taskmgr.exe} from \TT{C:\textbackslash{}Windows\textbackslash{}System32} 
to some other folder 
(so the \emph{Windows Resource Protection} 
will not try to restore the patched \TT{taskmgr.exe}).

Let's open it in Hiew and find the place:

\begin{figure}[H]
\centering
\myincludegraphics{examples/taskmgr/hiew2.png}
\caption{Hiew: find the place to be patched}
\end{figure}

Let's replace the \TT{MOVZX} instruction with ours.
Let's pretend we've got 64 CPU cores.

Add one additional \ac{NOP} (because our instruction is shorter than the original one):

\begin{figure}[H]
\centering
\myincludegraphics{examples/taskmgr/hiew1.png}
\caption{Hiew: patch it}
\end{figure}

And it works!
Of course, the data in the graphs is not correct.

At times, Task Manager even shows an overall CPU load of more than 100\%.

\begin{figure}[H]
\centering
\myincludegraphics{examples/taskmgr/taskmgr_64cpu_crop.png}
\caption{Fooled Windows Task Manager}
\end{figure}

The biggest number Task Manager does not crash with is 64.

Apparently, Task Manager in Windows Vista was not tested on computers with a large number of cores.

So there are probably some static data structure(s) inside it limited to 64 cores.

\subsection{Using LEA to load values}
\label{TaskMgr_LEA}

Sometimes, \TT{LEA} is used in \TT{taskmgr.exe} instead of \TT{MOV} to set the first argument of \\
\TT{NtQuerySystemInformation()}:

\lstinputlisting[caption=taskmgr.exe (Windows Vista),style=customasmx86]{examples/taskmgr/taskmgr2.lst}

\myindex{x86!\Instructions!LEA}

Perhaps \ac{MSVC} did so because machine code of \INS{LEA} is shorter than \INS{MOV REG, 5} (would be 5 instead of 4).

\INS{LEA} with offset in $-128..127$ range (offset will occupy 1 byte in opcode) with 32-bit registers is even shorter (for lack of REX prefix)---3 bytes.

Another example of such thing is: \myref{using_MOV_and_pack_of_LEA_to_load_values}.
}
\FR{\subsection{Exemple \#2: SCO OpenServer}

\label{examples_SCO}
\myindex{SCO OpenServer}
Un ancien logiciel pour SCO OpenServer de 1997 développé par une société qui a disparue
depuis longtemps.

Il y a un driver de dongle special à installer dans le système, qui contient les
chaînes de texte suivantes:
\q{Copyright 1989, Rainbow Technologies, Inc., Irvine, CA}
et
\q{Sentinel Integrated Driver Ver. 3.0 }.

Après l'installation du driver dans SCO OpenServer, ces fichiers apparaissent dans
l'arborescence /dev:

\begin{lstlisting}
/dev/rbsl8
/dev/rbsl9
/dev/rbsl10
\end{lstlisting}

Le programme renvoie une erreur lorsque le dongle n'est pas connecté, mais le message
d'erreur n'est pas trouvé dans les exécutables.

\myindex{COFF}

Grâce à \ac{IDA}, il est facile de charger l'exécutable COFF utilisé dans SCO OpenServer.

Essayons de trouver la chaîne \q{rbsl} et en effet, elle se trouve dans ce morceau
de code:

\lstinputlisting[style=customasmx86]{examples/dongles/2/1.lst}

Oui, en effet, le programme doit communiquer d'une façon ou d'une autre avec le driver.

\myindex{thunk-functions}
Le seul endroit où la fonction \TT{SSQC()} est appelée est dans la \glslink{thunk
 function}{fonction thunk}:

\lstinputlisting[style=customasmx86]{examples/dongles/2/2.lst}

SSQ() peut être appelé depuis au moins 2 fonctions.

L'une d'entre elles est:

\lstinputlisting[style=customasmx86]{examples/dongles/2/check1_EN.lst}

\q{\TT{3C}} et \q{\TT{3E}} semblent familiers: il y avait un dongle Sentinel Pro de
Rainbow sans mémoire, fournissant seulement une fonction de crypto-hachage secrète.

Vous pouvez lire une courte description de la fonction de hachage dont il s'agit
ici: \myref{hash_func}.

Mais retournons au programme.

Donc le programme peut seulement tester si un dongle est connecté ou s'il est absent.

Aucune autre information ne peut être écrite dans un tel dongle, puisqu'il n'a pas
de mémoire.
Les codes sur deux caractères sont des commandes (nous pouvons voir comment les commandes
sont traitées dans la fonction \TT{SSQC()}) et toutes les autres chaînes sont hachées
dans le dongle, transformées en un nombre 16-bit.
L'algorithme était secret, donc il n'était pas possible d'écrire un driver de remplacement
ou de refaire un dongle matériel qui l'émulerait parfaitement.

Toutefois, il est toujours possible d'intercepter tous les accès au dongle et de
trouver les constantes auxquelles les résultats de la fonction de hachage sont comparées.

Mais nous devons dire qu'il est possible de construire un schéma de logiciel de protection
de copie robuste basé sur une fonction secrète de hachage cryptographique: il suffit
qu'elle chiffre/déchiffre les fichiers de données utilisés par votre logiciel.

Mais retournons au code:

Les codes 51/52/53 sont utilisés pour choisir le port imprimante LPT.
3x/4x sont utilisés pour le choix de la \q{famille} (c'est ainsi que les dongles
Sentinel Pro sont différenciés les uns des autres: plus d'un dongle peut être connecté
sur un port LPT).

La seule chaîne passée à la fonction qui ne fasse pas 2 caractères est "0123456789".

Ensuite, le résultat est comparé à l'ensemble des résultats valides.

Si il est correct, 0xC ou 0xB est écrit dans la variable globale \TT{ctl\_model}.%

Une autre chaîne de texte qui est passée est
"PRESS ANY KEY TO CONTINUE: ", mais le résultat n'est pas testé.
Difficile de dire pourquoi, probablement une erreur\footnote{C'est un sentiment
étrange de trouver un bug dans un logiciel aussi ancien.}.

Voyons où la valeur de la variable globale \TT{ctl\_model} est utilisée.

Un tel endroit est:

\lstinputlisting[style=customasmx86]{examples/dongles/2/4.lst}

Si c'est 0, un message d'erreur chiffré est passé à une routine de déchiffrement
et affiché.

\myindex{x86!\Instructions!XOR}

La routine de déchiffrement de la chaîne semble être un simple \glslink{xoring}{xor}:

\lstinputlisting[style=customasmx86]{examples/dongles/2/err_warn.lst}

C'est pourquoi nous étions incapable de trouver le message d'erreur dans les fichiers
exécutable, car ils sont chiffrés (ce qui est une pratique courante).

Un autre appel à la fonction de hachage \TT{SSQ()} lui passe la chaîne \q{offln}
et le résultat est comparé avec \TT{0xFE81} et \TT{0x12A9}.

Si ils ne correspondent pas, ça se comporte comme une sorte de fonction \TT{timer()}
(peut-être en attente qu'un dongle mal connecté soit reconnecté et re-testé?) et ensuite
déchiffre un autre message d'erreur à afficher.

\lstinputlisting[style=customasmx86]{examples/dongles/2/check2_EN.lst}

Passer outre le dongle est assez facile: il suffit de patcher tous les sauts après
les instructions \CMP pertinentes.

Une autre option est d'écrire notre propre driver SCO OpenServer, contenant une table
de questions et de réponses, toutes celles qui sont présentent dans le programme.

\subsubsection{Déchiffrer les messages d'erreur}

À propos, nous pouvons aussi essayer de déchiffrer tous les messages d'erreurs.
L'algorithme qui se trouve dans la fonction \TT{err\_warn()} est très simple, en effet:

\lstinputlisting[caption=Decryption function,style=customasmx86]{examples/dongles/2/decrypting_FR.lst}

Comme on le voit, non seulement la chaîne est transmise à la fonction de déchiffrement
mais aussi la clef:

\lstinputlisting[style=customasmx86]{examples/dongles/2/tmp1_EN.asm}

L'algorithme est un simple \glslink{xoring}{xor}: chaque octet est xoré avec la clef, mais
la clef est incrémentée de 3 après le traitement de chaque octet.

Nous pouvons écrire un petit script Python pour vérifier notre hypothèse:

\lstinputlisting[caption=Python 3.x]{examples/dongles/2/decr1.py}

Et il affiche: \q{check security device connection}.
Donc oui, ceci est le message déchiffré.

Il y a d'autres messages chiffrés, avec leur clef correspondante.
Mais inutile de dire qu'il est possible de les déchiffrer sans leur clef.
Premièrement, nous voyons que le clef est en fait un octet.
C'est parce que l'instruction principale de déchiffrement (\XOR) fonctionne au niveau
de l'octet.
La clef se trouve dans le registre \ESI, mais seulement une partie de \ESI d'un octet
est utilisée.
Ainsi, une clef pourrait être plus grande que 255, mais sa valeur est toujours arrondie.

En conséquence, nous pouvons simplement essayer de brute-forcer, en essayant toutes
les clefs possible dans l'intervalle 0..255.
Nous allons aussi écarter les messages comportants des caractères non-imprimable.

\lstinputlisting[caption=Python 3.x]{examples/dongles/2/decr2.py}

Et nous obtenons:

\lstinputlisting[caption=Results]{examples/dongles/2/decr2_result.txt}

Ici il y a un peu de déchet, mais nous pouvons rapidement trouver les messages en
anglais.

À propos, puisque l'algorithme est un simple chiffrement xor, la même fonction peut
être utilisée pour chiffrer les messages.
Si besoin, nous pouvons chiffrer nos propres messages, et patcher le programme en les insérant.
}

\EN{\mysection{Some GCC library functions}
\myindex{GCC}
\label{sec:GCC_library_func}

%__ashldi3
%__ashrdi3
%__floatundidf
%__floatdisf
%__floatdixf
%__floatundidf
%__floatundisf
%__floatundixf
%__lshrdi3
%__muldi3

\begin{center}
\begin{tabular}{ | l | l | }
\hline
\HeaderColor name & \HeaderColor meaning \\
\hline \TT{\_\_divdi3} & signed division \\
\hline \TT{\_\_moddi3} & getting remainder (modulo) of signed division \\
\hline \TT{\_\_udivdi3} & unsigned division \\
\hline \TT{\_\_umoddi3} & getting remainder (modulo) of unsigned division \\
\hline
\end{tabular}
\end{center}

}
\RU{\mysection{Некоторые библиотечные функции GCC}
\myindex{GCC}
\label{sec:GCC_library_func}

%__ashldi3
%__ashrdi3
%__floatundidf
%__floatdisf
%__floatdixf
%__floatundidf
%__floatundisf
%__floatundixf
%__lshrdi3
%__muldi3

\begin{center}
\begin{tabular}{ | l | l | }
\hline
\HeaderColor имя & \HeaderColor значение \\
\hline \TT{\_\_divdi3} & знаковое деление \\
\hline \TT{\_\_moddi3} & остаток от знакового деления \\
\hline \TT{\_\_udivdi3} & беззнаковое деление \\
\hline \TT{\_\_umoddi3} & остаток от беззнакового деления \\
\hline
\end{tabular}
\end{center}

}
\FR{\mysection{Quelques fonctions de la bibliothèque de GCC}
\myindex{GCC}
\label{sec:GCC_library_func}

%__ashldi3
%__ashrdi3
%__floatundidf
%__floatdisf
%__floatdixf
%__floatundidf
%__floatundisf
%__floatundixf
%__lshrdi3
%__muldi3

\begin{center}
\begin{tabular}{ | l | l | }
\hline
\HeaderColor nom & \HeaderColor signification \\
\hline \TT{\_\_divdi3} & division signée \\
\hline \TT{\_\_moddi3} & reste (modulo) d'une division signée \\
\hline \TT{\_\_udivdi3} & division non signée \\
\hline \TT{\_\_umoddi3} & reste (modulo) d'une division non signée \\
\hline
\end{tabular}
\end{center}

}
\DE{\mysection{Einige GCC-Bibliotheks-Funktionen}
\myindex{GCC}
\label{sec:GCC_library_func}

%__ashldi3
%__ashrdi3
%__floatundidf
%__floatdisf
%__floatdixf
%__floatundidf
%__floatundisf
%__floatundixf
%__lshrdi3
%__muldi3

\begin{center}
\begin{tabular}{ | l | l | }
\hline
\HeaderColor Name & \HeaderColor Bedeutung \\
\hline \TT{\_\_divdi3} & vorzeichenbehaftete Division \\
\hline \TT{\_\_moddi3} & Rest (Modulo) einer vorzeichenbehafteten Division \\
\hline \TT{\_\_udivdi3} & vorzeichenlose Division \\
\hline \TT{\_\_umoddi3} & Rest (Modulo) einer vorzeichenlosen Division \\
\hline
\end{tabular}
\end{center}

}


\mysection{\RU{Некоторые библиотечные функции MSVC}\EN{Some MSVC library functions}\DE{Einige MSVC-Bibliotheks-Funktionen}%
\FR{Quelques fonctions de la bibliothèque MSVC}}
\myindex{MSVC}
\label{sec:MSVC_library_func}

\TT{ll} \RU{в имени функции означает}\EN{in function name stands for}\DE{in Funktionsnamen steht für}%
\FR{dans une fontion signifie} \q{long long}, \RU{т.е. 64-битный тип данных}
\EN{e.g., a 64-bit data type}\DE{z.B. einen 64-Bit-Datentyp}\FR{i.e., type de donées 64-bit}.

\begin{center}
\begin{tabular}{ | l | l | }
\hline
\HeaderColor \RU{имя}\EN{name}\DE{Name}\FR{nom} & \HeaderColor \RU{значение}\EN{meaning}\DE{Bedeutung}\FR{signification} \\
\hline \TT{\_\_alldiv} & \RU{знаковое деление}\EN{signed division}\DE{vorzeichenbehaftete Division}\FR{division signée} \\
\hline \TT{\_\_allmul} & \RU{умножение}\EN{multiplication}\DE{Multiplikation}\FR{multiplication} \\
\hline \TT{\_\_allrem} & \RU{остаток от знакового деления}\EN{remainder of signed division}\DE{Rest einer vorzeichenbehafteten Division}%
\FR{reste de la division signée} \\
\hline \TT{\_\_allshl} & \RU{сдвиг влево}\EN{shift left}\DE{Schiebe links}\FR{décalage à gauche} \\
\hline \TT{\_\_allshr} & \RU{знаковый сдвиг вправо}\EN{signed shift right}\DE{Schiebe links, vorzeichenbehaftet}%
\FR{décalage signé à droite} \\
\hline \TT{\_\_aulldiv} & \RU{беззнаковое деление}\EN{unsigned division}\DE{vorzeichenlose Division}%
\FR{division non signée} \\
\hline \TT{\_\_aullrem} & \RU{остаток от беззнакового деления}\EN{remainder of unsigned division}\DE{Rest (Modulo) einer vorzeichenlosen Division}%
\FR{reste de la division non signée} \\
\hline \TT{\_\_aullshr} & \RU{беззнаковый сдвиг вправо}\EN{unsigned shift right}\DE{Schiebe rechts, vorzeichenlos}%
\FR{décalage non signé à droite} \\
\hline
\end{tabular}
\end{center}

\RU{Процедуры умножения и сдвига влево, одни и те же и для знаковых чисел, и для беззнаковых,
поэтому здесь только одна функция для каждой операции}
\EN{Multiplication and shift left procedures are the same for both signed and unsigned numbers, hence there is only one function 
for each operation here}
\DE{Multiplikation und Links-Schiebebefehle sind sowohl für vorzeichenbehaftete als auch vorzeichenlose Zahlen,
da hier für jede Operation nur ein Befehl existiert}
\FR{La multiplication et le décalage à gauche sont similaire pour les nombres signés
et non signés, donc il n'y a qu'une seule fonction ici}. \\
\\
\RU{Исходные коды этих функций можно найти в установленной \ac{MSVS}, в}\EN{The source code of these function
can be found in the installed \ac{MSVS}, in}%
\DE{Der Quellcode dieser Funktionen kann im Pfad des installierten \ac{MSVS}, gefunden werden: }%
\FR{Le code source des ces fonctions peut être trouvé dans l'installation de \ac{MSVS},
dans} \TT{VC/crt/src/intel/*.asm}.


\mysection{Cheatsheets}

% sections
\subsection{IDA}
\myindex{IDA}
\label{sec:IDA_cheatsheet}

\ShortHotKeyCheatsheet:

\begin{center}
\begin{tabular}{ | l | l | }
\hline
\HeaderColor \RU{клавиша}\EN{key}\DE{Taste}\FR{touche} & \HeaderColor \RU{значение}\EN{meaning}\DE{Bedeutung}\FR{signification} \\
\hline
Space 	& \RU{переключать между листингом и просмотром кода в виде графа}
            \EN{switch listing and graph view}
            \DE{Zwischen Quellcode und grafischer Ansicht wechseln}%
				\FR{échanger le listing et le mode graphique} \\
C 	& \RU{конвертировать в код}\EN{convert to code}\DE{zu Code konvertieren}%
		\FR{convertir en code} \\
D 	& \RU{конвертировать в данные}\EN{convert to data}\DE{zu Daten konvertieren}%
		\FR{convertir en données} \\
A 	& \RU{конвертировать в строку}\EN{convert to string}\DE{zu Zeichenkette konvertieren}%
		\FR{convertir en chaîne} \\
* 	& \RU{конвертировать в массив}\EN{convert to array}\DE{zu Array konvertieren}%
		\FR{convertir en tableau} \\
U 	& \RU{сделать неопределенным}\EN{undefine}\DE{undefinieren}%
		\FR{rendre indéfini}\\
O 	& \RU{сделать смещение из операнда}\EN{make offset of operand}\DE{Offset von Operanden}%
		\FR{donner l'offset d'une opérande}\\
H 	& \RU{сделать десятичное число}\EN{make decimal number}\DE{Dezimalzahl erstellen}%
		\FR{transformer en nombre décimal} \\
R 	& \RU{сделать символ}\EN{make char}\DE{Zeichen erstellen}%
		\FR{transformer en caractère} \\
B 	& \RU{сделать двоичное число}\EN{make binary number}\DE{Binärzahl erstellen}%
		\FR{transformer en nombre binaire} \\
Q 	& \RU{сделать шестнадцатеричное число}\EN{make hexadecimal number}\DE{Hexadezimalzahl erstellen}%
		\FR{transformer en nombre hexa-décimal} \\
N 	& \RU{переименовать идентификатор}\EN{rename identifier}\DE{Identifikator umbenennen}%
		\FR{renommer l'identifiant} \\
? 	& \RU{калькулятор}\EN{calculator}\DE{Rechner}\FR{calculatrice} \\
G 	& \RU{переход на адрес}\EN{jump to address}\DE{zu Adresse springen}%
		\FR{sauter à l'adresse} \\
: 	& \RU{добавить комментарий}\EN{add comment}\DE{Kommentar einfügen}\FR{ajouter un commentaire} \\
Ctrl-X 	& \RU{показать ссылки на текущую функцию, метку, переменную}%
		\EN{show references to the current function, label, variable }%
		\DE{Referenz zu aktueller Funktion, Variable, ... zeigen}%
		\FR{montrer les références à la fonction, au label, à la variable courant} \\
	& \RU{(в т.ч., в стеке)}\EN{(incl. in local stack)}\DE{(inkl. lokalem Stack)}%
		\FR{inclure dans la pile locale} \\
X 	& \RU{показать ссылки на функцию, метку, переменную, итд}\EN{show references to the function, label, variable, etc.}%
		\DE{Referenz zu Funktion, Variable, ... zeigen}%
		\FR{montrer les références à la fonction, au label, à la variable, etc.} \\
Alt-I 	& \RU{искать константу}\EN{search for constant}\DE{Konstante suchen}%
		\FR{chercher une constante} \\
Ctrl-I 	& \RU{искать следующее вхождение константы}\EN{search for the next occurrence of constant}\DE{Nächstes Auftreten der Konstante suchen}%
		\FR{chercher la prochaine occurrence d'une constante} \\
Alt-B 	& \RU{искать последовательность байт}\EN{search for byte sequence}\DE{Byte-Sequenz suchen}%
		\FR{chercher une séquence d'octets} \\
Ctrl-B 	& \RU{искать следующее вхождение последовательности байт}
		\EN{search for the next occurrence of byte sequence}
		\DE{Nächstes Auftreten der Byte-Sequenz suchen}%
		\FR{chercher l'occurrence suivante d'une séquence d'octets} \\
Alt-T 	& \RU{искать текст (включая инструкции, итд.)}%
		\EN{search for text (including instructions, etc.)}%
		\DE{Text suchen (inkl. Anweisungen, usw.)}%
		\FR{chercher du texte (instructions incluses, etc.)} \\
Ctrl-T 	& \RU{искать следующее вхождение текста}%
		\EN{search for the next occurrence of text}%
		\DE{nächstes Aufreten des Textes suchen}%
		\FR{chercher l'occurrence suivante du texte} \\
Alt-P 	& \RU{редактировать текущую функцию}%
		\EN{edit current function}%
		\DE{akutelle Funktion editieren}%
		\FR{éditer la fonction courante} \\
Enter 	& \RU{перейти к функции, переменной, итд.}%
		\EN{jump to function, variable, etc.}%
		\DE{zu Funktion, Variable, ... springen}%
		\FR{sauter à la fonction, la variable, etc.} \\
Esc 	& \RU{вернуться назад}\EN{get back}\DE{zurückgehen}%
		\FR{retourner en arrière} \\
Num -   & \RU{свернуть функцию или отмеченную область}%
		\EN{fold function or selected area}%
		\DE{Funktion oder markierten Bereich einklappen}%
		\FR{cacher/plier la fonction ou la partie sélectionnée} \\
Num + 	& \RU{снова показать функцию или область}%
		\EN{unhide function or area}%
		\DE{Funktion oder Bereich anzeigen}%
		\FR{afficher la fonction ou une partie} \\
\hline
\end{tabular}
\end{center}

\RU{Сворачивание функции или области может быть удобно чтобы прятать те части функции,
чья функция вам стала уже ясна}%
\EN{Function/area folding may be useful for hiding function parts when you realize what they do}%
\DE{Das Einklappen ist nützlich um Teile von Funktionen zu verstecken, wenn bekannt ist was sie tun}%
\FR{cacher une fonction ou une partie de code peut être utile pour cacher des parties du
code lorsque vous avez compris ce qu'elles font}.
\RU{это используется в моем скрипте\footnote{\href{\YurichevIDAIDCScripts}{GitHub}}}\EN{this is used in my}\DE{dies wird genutzt im}%
\RU{для сворачивания некоторых очень часто используемых фрагментов inline-кода}%
\EN{script\footnote{\href{\YurichevIDAIDCScripts}{GitHub}} for hiding some often used patterns of inline code}%
\DE{Script\footnote{\href{\YurichevIDAIDCScripts}{GitHub}} um häufig genutzte Inline-Code-Stellen zu verstecken}%
\FR{ceci est utilisé dans mon script\footnote{\href{\YurichevIDAIDCScripts}{GitHub}}%
pour cacher des patterns de code inline souvent utilisés}.


\subsection{\olly}
\myindex{\olly}
\label{sec:Olly_cheatsheet}

\ShortHotKeyCheatsheet:

\begin{center}
\begin{tabular}{ | l | l | }
\hline
\HeaderColor \RU{хот-кей}\EN{hot-key}\DE{Tastenkürzel}\FR{raccourci} & 
\HeaderColor \RU{значение}\EN{meaning}\DE{Bedeutung}\FR{signification} \\
\hline
F7	& \RU{трассировать внутрь}\EN{trace into}\DE{Schritt}\FR{tracer dans la fonction}\\
F8	& \stepover\\
F9	& \RU{запуск}\EN{run}\DE{starten}\FR{démarrer}\\
Ctrl-F2	& \RU{перезапуск}\EN{restart}\DE{Neustart}\FR{redémarrer}\\
\hline
\end{tabular}
\end{center}

\subsection{MSVC}
\myindex{MSVC}
\label{sec:MSVC_options}

\RU{Некоторые полезные опции, которые были использованы в книге}
\EN{Some useful options which were used through this book}.
\DE{Einige nützliche Optionen die in diesem Buch genutzt werden}.
\FR{Quelques options utiles qui ont été utilisées dans ce livre}

\begin{center}
\begin{tabular}{ | l | l | }
\hline
\HeaderColor \RU{опция}\EN{option}\DE{Option}\FR{option} & 
\HeaderColor \RU{значение}\EN{meaning}\DE{Bedeutung}\FR{signification} \\
\hline
/O1		& \RU{оптимизация по размеру кода}\EN{minimize space}\DE{Speicherplatz minimieren}%
\FR{minimiser l'espace}\\
/Ob0		& \RU{не заменять вызовы inline-функций их кодом}\EN{no inline expansion}\DE{Keine Inline-Erweiterung}%
\FR{pas de mire en ligne}\\
/Ox		& \RU{максимальная оптимизация}\EN{maximum optimizations}\DE{maximale Optimierung}%
\FR{optimisation maximale}\\
/GS-		& \RU{отключить проверки переполнений буфера}
		\EN{disable security checks (buffer overflows)}
        \DE{Sicherheitsüberprüfungen deaktivieren (Buffer Overflows)}%
		\FR{désactiver les vérifications de sécurité (buffer overflows)}\\
/Fa(file)	& \RU{генерировать листинг на ассемблере}\EN{generate assembly listing}\DE{Assembler-Quelltext erstellen}%
\FR{générer un listing assembleur}\\
/Zi		& \RU{генерировать отладочную информацию}\EN{enable debugging information}\DE{Debugging-Informationen erstellen}%
\FR{activer les informations de débogage}\\
/Zp(n)		& \RU{паковать структуры по границе в $n$ байт}\EN{pack structs on $n$-byte boundary}\DE{Strukturen an $n$-Byte-Grenze ausrichten}%
\FR{aligner les structures sur une limite de $n$-octet}\\
/MD		& \RU{выходной исполняемый файл будет использовать}
			\EN{produced executable will use}%
            \DE{ausführbare Daten nutzt}%
\FR{l'exécutable généré utilisera} \TT{MSVCR*.DLL}\\
\hline
\end{tabular}
\end{center}

\RU{Кое-как информация о версиях MSVC}\EN{Some information about MSVC versions}\DE{Informationen zu MSVC-Versionen}%
\FR{Quelques informations sur les versions de MSVC}:
\myref{MSVC_versions}.


\EN{\subsection{GCC}
\myindex{GCC}

Some useful options which were used through this book.

\begin{center}
\begin{tabular}{ | l | l | }
\hline
\HeaderColor option & 
\HeaderColor meaning \\
\hline
-Os		& code size optimization \\
-O3		& maximum optimization \\
-regparm=	& how many arguments are to be passed in registers \\
-o file		& set name of output file \\
-g		& produce debugging information in resulting executable \\
-S		& generate assembly listing file \\
-masm=intel	& produce listing in Intel syntax \\
-fno-inline	& do not inline functions \\
\hline
\end{tabular}
\end{center}


}
\RU{\myparagraph{GCC 4.4.1}

\lstinputlisting[caption=GCC 4.4.1,style=customasmx86]{patterns/12_FPU/3_comparison/x86/GCC_RU.asm}

\myindex{x86!\Instructions!FUCOMPP}
\FUCOMPP~--- это почти то же что и \FCOM, только выкидывает из стека оба значения после сравнения, 
а также несколько иначе реагирует на \q{не-числа}.

\myindex{Не-числа (NaNs)}
Немного о \emph{не-числах}.

FPU умеет работать со специальными переменными, которые числами не являются и называются \q{не числа} или 
\gls{NaN}.
Это бесконечность, результат деления на ноль, и так далее. Нечисла бывают \q{тихие} и \q{сигнализирующие}. 
С первыми можно продолжать работать и далее, а вот если вы попытаетесь совершить какую-то операцию 
с сигнализирующим нечислом, то сработает исключение.

\myindex{x86!\Instructions!FCOM}
\myindex{x86!\Instructions!FUCOM}
Так вот, \FCOM вызовет исключение если любой из операндов какое-либо нечисло.
\FUCOM же вызовет исключение только если один из операндов именно \q{сигнализирующее нечисло}.

\myindex{x86!\Instructions!SAHF}
\label{SAHF}
Далее мы видим \SAHF (\emph{Store AH into Flags})~--- это довольно редкая инструкция в коде, не использующим FPU. 
8 бит из \AH перекладываются в младшие 8 бит регистра статуса процессора в таком порядке:

\input{SAHF_LAHF}

\myindex{x86!\Instructions!FNSTSW}
Вспомним, что \FNSTSW перегружает интересующие нас биты \CThreeBits в \AH, 
и соответственно они будут в позициях 6, 2, 0 в регистре \AH:

\input{C3_in_AH}

Иными словами, пара инструкций \INS{fnstsw  ax / sahf} перекладывает биты \CThreeBits в флаги \ZF, \PF, \CF.

Теперь снова вспомним, какие значения бит \CThreeBits будут при каких результатах сравнения:

\begin{itemize}
\item Если $a$ больше $b$ в нашем случае, то биты \CThreeBits должны быть выставлены так: 0, 0, 0.
\item Если $a$ меньше $b$, то биты будут выставлены так: 0, 0, 1.
\item Если $a=b$, то так: 1, 0, 0.
\end{itemize}
% TODO: table?

Иными словами, после трех инструкций \FUCOMPP/\FNSTSW/\SAHF возможны такие состояния флагов:

\begin{itemize}
\item Если $a>b$ в нашем случае, то флаги будут выставлены так: \GTT{ZF=0, PF=0, CF=0}.
\item Если $a<b$, то флаги будут выставлены так: \GTT{ZF=0, PF=0, CF=1}.
\item Если $a=b$, то так: \GTT{ZF=1, PF=0, CF=0}.
\end{itemize}
% TODO: table?

\myindex{x86!\Instructions!SETcc}
\myindex{x86!\Instructions!JNBE}
Инструкция \SETNBE выставит в \AL единицу или ноль в зависимости от флагов и условий. 
Это почти аналог \JNBE, за тем лишь исключением, что \SETcc
\footnote{\emph{cc} это \emph{condition code}}
выставляет 1 или 0 в \AL, а \Jcc делает переход или нет. 
\SETNBE запишет 1 только если \GTT{CF=0} и \GTT{ZF=0}. Если это не так, то запишет 0 в \AL.

\CF будет 0 и \ZF будет 0 одновременно только в одном случае: если $a>b$.

Тогда в \AL будет записана 1, последующий условный переход \JZ выполнен не будет 
и функция вернет~\GTT{\_a}. 
В остальных случаях, функция вернет~\GTT{\_b}.
}
\FR{\subsection{GCC}
\myindex{GCC}

Quelques options utiles qui ont été utilisées dans ce livre.

\begin{center}
\begin{tabular}{ | l | l | }
\hline
\HeaderColor option & 
\HeaderColor signification \\
\hline
-Os		& optimiser la taille du code \\
-O3		& optimisation maximale \\
-regparm=	& nombre d'arguments devant être passés dans les registres \\
-o file		& définir le nom du fichier de sortie \\
-g		& mettre l'information de débogage dans l'exécutable généré \\
-S		& générer un fichier assembleur \\
-masm=intel	& construire le code source en syntaxe Intel \\
-fno-inline	& ne pas mettre les fonctions en ligne \\
\hline
\end{tabular}
\end{center}


}
\DE{\myparagraph{GCC 4.4.1}

\lstinputlisting[caption=GCC 4.4.1,style=customasmx86]{patterns/12_FPU/3_comparison/x86/GCC_DE.asm}

\myindex{x86!\Instructions!FUCOMPP}
\FUCOMPP{} ist fast wie like \FCOM, nimmt aber beide Werte vom Stand und
behandelt \q{undefinierte Zahlenwerte} anders.


\myindex{Non-a-numbers (NaNs)}
Ein wenig über \emph{undefinierte Zahlenwerte}.

Die FPU ist in der Lage mit speziellen undefinieten Werten, den sogenannten
\emph{not-a-number}(kurz \gls{NaN}) umzugehen. Beispiele sind etwa der Wert
unendlich, das Ergebnis einer Division durch 0, etc. Undefinierte Werte können
entwder \q{quiet} oder \q{signaling} sein. Es ist möglich mit \q{quiet} NaNs zu
arbeiten, aber beim Versuch einen Befehl auf \q{signaling} NaNs auszuführen,
wird eine Exception geworfen. 

\myindex{x86!\Instructions!FCOM}
\myindex{x86!\Instructions!FUCOM}
\FCOM erzeugt eine Exception, falls irgendein Operand ein \gls{NaN} ist.
\FUCOM erzeugt eine Exception nur dann, wenn ein Operand eine \q{signaling}
\gls{NaN} (SNaN) ist.

\myindex{x86!\Instructions!SAHF}
\label{SAHF}
Der nächste Befehl ist \SAHF (\emph{Store AH into Flags})~---es handelt sich
hierbei um einen seltenen Befehl, der nicht mit der FPU zusammenhängt.
8 Bits aus AH werden in die niederen 8 Bit der CPU Flags in der folgenden
Reihenfolge verschoben:

\input{SAHF_LAHF}

\myindex{x86!\Instructions!FNSTSW}
Erinnern wir uns, dass \FNSTSW die für uns interessanten Bits (\CThreeBits) auf
den Stellen 6,2,0 im AH Register setzt:

\input{C3_in_AH}
Mit anderen Worten: der Befehl \INS{fnstsw ax / sahf} verschiebt \CThreeBits
nach \ZF, \PF und \CF. 

Überlegen wir uns auch die Werte der \CThreeBits in unterschiedlichen Szenarien:

\begin{itemize} 
  \item Falls in unserem Beispiel $a$ größer als $b$ ist, dann werden die
  \CThreeBits auf 0,0,0 gesetzt.
  \item Falls $a$ kleiner als $b$ ist, werden die Bits auf 0,0,1 gesetzt.
  \item Falls $a=b$, dann werden die Bits auf 1,0,0 gesetzt.
\end{itemize}
% TODO: table?
Mit anderen Worten, die folgenden Zustände der CPU Flags sind nach drei
\FUCOMPP/\FNSTSW/\SAHF Befehlen möglich:

\begin{itemize}
\item Falls $a>b$, werden die CPU Flags wie folgt gesetzt \GTT{ZF=0, PF=0,
CF=0}.
\item Falls $a<b$, werden die CPU Flags wie folgt gesetzt: \GTT{ZF=0, PF=0,
CF=1}.
\item Und falls $a=b$, dann gilt: \GTT{ZF=1, PF=0, CF=0}.
\end{itemize}
% TODO: table?

\myindex{x86!\Instructions!SETcc}
\myindex{x86!\Instructions!JNBE}
Abhängig von den CPU Flags und Bedingungen, speichert \SETNBE entweder 1 oder 0
in AL.
Es ist also quasi das Gegenstück von \JNBE mit dem Unterschied, dass \SETcc

Depending on the CPU flags and conditions, \SETNBE stores 1 or 0 to AL. 
It is almost the counterpart of \JNBE, with the exception that \SETcc
\footnote{\emph{cc} is \emph{condition code}} eine 1 oder 0 in \AL speichert, aber
\Jcc tatsächlich auch springt.
\SETNBE speicher 1 nur, falls \GTT{CF=0} und \GTT{ZF=0}.
Wenn dies nicht der Fall ist, dann wird 0 in \AL gespeichert.

Nur in einem Fall sind \CF und \ZF beide 0: falls $a>b$.

In diesem Fall wird 1 in \AL gespeichert, der nachfolgende \JZ Sprung wird nicht
ausgeführt und die Funktion liefert {\_a} zurück. In allen anderen Fällen wird
{\_b} zurückgegeben.
}

\subsection{GDB}
\myindex{GDB}
\label{sec:GDB_cheatsheet}

% FIXME: in Russian table doesn't fit!

\RU{Некоторые команды, которые были использованы в книге}\EN{Some of commands we used in this book}\DE{Einige nützliche Optionen die in diesem Buch genutzt werden}%
\FR{Quelques commandes que nous avons utilisées dans ce livre}:

\small
\begin{center}
\begin{tabular}{ | l | l | }
\hline
\HeaderColor \RU{опция}\EN{option}\DE{Option}\FR{option} & 
\HeaderColor \RU{значение}\EN{meaning}\DE{Bedeutung} \\
\hline
break filename.c:number		& \RU{установить точку останова на номере строки в исходном файле}
					\EN{set a breakpoint on line number in source code}
                    \DE{Setzen eines Breakpoints in der angegebenen Zeile}%
					\FR{mettre un point d'arrêt à la ligne number du code source} \\
break function			& \RU{установить точку останова на функции}\EN{set a breakpoint on function}\DE{Setzen eines Breakpoints in der Funktion}%
\FR{mettre un point d'arrêt sur une fonction} \\
break *address			& \RU{установить точку останова на адресе}\EN{set a breakpoint on address}\DE{Setzen eines Breakpoints auf Adresse}%
\FR{mettre un point d'arrêt à une adresse} \\
b				& \dittoclosing \\
p variable			& \RU{вывести значение переменной}\EN{print value of variable}\DE{Ausgabe eines Variablenwerts}%
\FR{afficher le contenu d'une variable} \\
run				& \RU{запустить}\EN{run}\DE{Starten}\FR{démarrer} \\
r				& \dittoclosing \\
cont				& \RU{продолжить исполнение}\EN{continue execution}\DE{Ausführung fortfahren}\FR{continuer l'exécution} \\
c				& \dittoclosing \\
bt				& \RU{вывести стек}\EN{print stack}\DE{Stack ausgeben}\FR{afficher la pile} \\
set disassembly-flavor intel	& \RU{установить Intel-синтаксис}\EN{set Intel syntax}\DE{Intel-Syntax nutzen}%
\FR{utiliser la syntaxe Intel} \\
disas				& disassemble current function \\
disas function			& \RU{дизассемблировать функцию}\EN{disassemble function}\DE{Funktion disassemblieren}\FR{désassembler la fonction} \\
disas function,+50		& disassemble portion \\
disas \$eip,+0x10		& \dittoclosing \\
disas/r				& \EN{disassemble with opcodes}\RU{дизассемблировать с опкодами}\DE{mit OpCodes disassemblieren}%
\FR{désassembler avec les opcodes} \\
info registers			& \RU{вывести все регистры}\EN{print all registers}\DE{Ausgabe aller Register}\FR{afficher tous les registres} \\
info float			& \RU{вывести FPU-регистры}\EN{print FPU-registers}\DE{Ausgabe der FPU-Register}\FR{afficher les registres FPU} \\
info locals			& \RU{вывести локальные переменные (если известны)}\EN{dump local variables (if known)}\DE{(bekannte) lokale Variablen ausgeben}%
\FR{afficher les variables locales} \\
x/w ...				& \RU{вывести память как 32-битные слова}\EN{dump memory as 32-bit word}\DE{Speicher als 32-Bit-Wort ausgeben}%
\FR{afficher la mémoire en mot de 32-bit} \\
x/w \$rdi			& \RU{вывести память как 32-битные слова}\EN{dump memory as 32-bit word}\DE{Speicher als 32-Bit-Wort ausgeben}%
\FR{afficher la mémoire en mot de 32-bit} \\
				& \RU{по адресу в \TT{RDI}}\EN{at address in \TT{RDI}}\DE{an Adresse in \TT{RDI}}\FR{à l'adresse dans \TT{RDI}} \\

x/10w ...			& \RU{вывести 10 слов памяти}\EN{dump 10 memory words}\DE{10 Speicherworte ausgeben}%
\FR{afficher 10 mots de la mémoire} \\
x/s ...				& \RU{вывести строку из памяти}\EN{dump memory as string}\DE{Speicher als Zeichenkette ausgeben}%
\FR{afficher la mémoire en tant que chaîne} \\
x/i ...				& \RU{трактовать память как код}\EN{dump memory as code}\DE{Speicher als Code ausgeben}%
\FR{afficher la mémoire en tant que code} \\
x/10c ...			& \RU{вывести 10 символов}\EN{dump 10 characters}\DE{10 Zeichen ausgeben}%
\FR{afficher 10 caractères} \\
x/b ...				& \RU{вывести байты}\EN{dump bytes}\DE{Bytes ausgeben}\FR{afficher des octets} \\
x/h ...				& \RU{вывести 16-битные полуслова}\EN{dump 16-bit halfwords}\DE{16-Bit-Halbworte ausgeben}%
\FR{afficher en demi-mots de 16-bit} \\
x/g ...				& \RU{вывести 64-битные слова}\EN{dump giant (64-bit) words}\DE{große (64-Bit-) Worte ausgeben}%
\FR{afficher des mots géants (64-bit)} \\
finish				& \RU{исполнять до конца функции}\EN{execute till the end of function}\DE{bis Funktionsende fortfahren}%
\FR{exécuter jusqu'à la fin de la fonction} \\
next				& \RU{следующая инструкция (не заходить в функции)}
					\EN{next instruction (don't dive into functions)}
					\DE{Nächste Anweisung (nicht in Funktion springen)}
					\FR{instruction suivante (ne pas descendre dans les fonctions)} \\
step				& \RU{следующая инструкция (заходить в функции)}
					\EN{next instruction (dive into functions)}
					\DE{Nächste Anweisung (in Funktion springen)}
					\FR{instruction suivante (descendre dans les fonctions)} \\
set step-mode on		& \RU{не использовать информацию о номерах строк при использовании команды step}
					\EN{do not use line number information while stepping}
					\DE{Beim schrittweisen Ausführen keine Zeilennummerninfos nutzen}
					\FR{ne pas utiliser l'information du numéro de ligne en exécutant pas à pas} \\
frame n				& \RU{переключить фрейм стека}\EN{switch stack frame}\DE{Stack-Frame tauschen}\FR{échanger la stack frame} \\
info break			& \RU{список точек останова}\EN{list of breakpoints}\DE{Breakpoints schauen}%
\FR{afficher les points d'arrêt} \\
del n				& \RU{удалить точку останова}\EN{delete breakpoint}\DE{Breakpoints löschen}\FR{effacer un point d'arrêt} \\
set args ...			& \RU{установить аргументы командной строки}\EN{set command-line arguments}\DE{Aufrufparameter setzen}%
\FR{définir les arguments de la ligne de commande} \\
\hline
\end{tabular}
\end{center}
\normalsize



}
\DE{\part*{\RU{Приложение}\EN{Appendix}\DE{Anhang}\FR{Appendice}\IT{Appendice}}
\appendix
\addcontentsline{toc}{part}{\RU{Приложение}\EN{Appendix}\DE{Anhang}\FR{Appendice}\IT{Appendice}}

% chapters
\EN{\mysection{Task manager practical joke (Windows Vista)}
\myindex{Windows!Windows Vista}

Let's see if it's possible to hack Task Manager slightly so it would detect more \ac{CPU} cores.

\myindex{Windows!NTAPI}

Let us first think, how does the Task Manager know the number of cores?

There is the \TT{GetSystemInfo()} win32 function present in win32 userspace which can tell us this.
But it's not imported in \TT{taskmgr.exe}.

There is, however, another one in \gls{NTAPI}, \TT{NtQuerySystemInformation()}, 
which is used in \TT{taskmgr.exe} in several places.

To get the number of cores, one has to call this function with the \TT{SystemBasicInformation} constant
as a first argument (which is zero
\footnote{\href{http://msdn.microsoft.com/en-us/library/windows/desktop/ms724509(v=vs.85).aspx}{MSDN}}).

The second argument has to point to the buffer which is getting all the information.

So we have to find all calls to the \\
\TT{NtQuerySystemInformation(0, ?, ?, ?)} function.
Let's open \TT{taskmgr.exe} in IDA. 
\myindex{Windows!PDB}

What is always good about Microsoft executables is that IDA can download the corresponding \gls{PDB} 
file for this executable and show all function names.

It is visible that Task Manager is written in \Cpp and some of the function names and classes are really 
speaking for themselves.
There are classes CAdapter, CNetPage, CPerfPage, CProcInfo, CProcPage, CSvcPage, 
CTaskPage, CUserPage.

Apparently, each class corresponds to each tab in Task Manager.

Let's visit each call and add comment with the value which is passed as the first function argument.
We will write \q{not zero} at some places, because the value there was clearly not zero, 
but something really different (more about this in the second part of this chapter).

And we are looking for zero passed as argument, after all.

\begin{figure}[H]
\centering
\myincludegraphics{examples/taskmgr/IDA_xrefs.png}
\caption{IDA: cross references to NtQuerySystemInformation()}
\end{figure}

Yes, the names are really speaking for themselves.

When we closely investigate each place where\\
\TT{NtQuerySystemInformation(0, ?, ?, ?)} is called,
we quickly find what we need in the \TT{InitPerfInfo()} function:

\lstinputlisting[caption=taskmgr.exe (Windows Vista),style=customasmx86]{examples/taskmgr/taskmgr.lst}

\TT{g\_cProcessors} is a global variable, and this name has been assigned by 
IDA according to the \gls{PDB} loaded from Microsoft's symbol server.

The byte is taken from \TT{var\_C20}. 
And \TT{var\_C58} is passed to\\
\TT{NtQuerySystemInformation()} 
as a pointer to the receiving buffer.
The difference between 0xC20 and 0xC58 is 0x38 (56).

Let's take a look at format of the return structure, which we can find in MSDN:

\begin{lstlisting}[style=customc]
typedef struct _SYSTEM_BASIC_INFORMATION {
    BYTE Reserved1[24];
    PVOID Reserved2[4];
    CCHAR NumberOfProcessors;
} SYSTEM_BASIC_INFORMATION;
\end{lstlisting}

This is a x64 system, so each PVOID takes 8 bytes.

All \emph{reserved} fields in the structure take $24+4*8=56$ bytes.

Oh yes, this implies that \TT{var\_C20} is the local stack is exactly the
\TT{NumberOfProcessors} field of the \TT{SYSTEM\_BASIC\_INFORMATION} structure.

Let's check our guess.
Copy \TT{taskmgr.exe} from \TT{C:\textbackslash{}Windows\textbackslash{}System32} 
to some other folder 
(so the \emph{Windows Resource Protection} 
will not try to restore the patched \TT{taskmgr.exe}).

Let's open it in Hiew and find the place:

\begin{figure}[H]
\centering
\myincludegraphics{examples/taskmgr/hiew2.png}
\caption{Hiew: find the place to be patched}
\end{figure}

Let's replace the \TT{MOVZX} instruction with ours.
Let's pretend we've got 64 CPU cores.

Add one additional \ac{NOP} (because our instruction is shorter than the original one):

\begin{figure}[H]
\centering
\myincludegraphics{examples/taskmgr/hiew1.png}
\caption{Hiew: patch it}
\end{figure}

And it works!
Of course, the data in the graphs is not correct.

At times, Task Manager even shows an overall CPU load of more than 100\%.

\begin{figure}[H]
\centering
\myincludegraphics{examples/taskmgr/taskmgr_64cpu_crop.png}
\caption{Fooled Windows Task Manager}
\end{figure}

The biggest number Task Manager does not crash with is 64.

Apparently, Task Manager in Windows Vista was not tested on computers with a large number of cores.

So there are probably some static data structure(s) inside it limited to 64 cores.

\subsection{Using LEA to load values}
\label{TaskMgr_LEA}

Sometimes, \TT{LEA} is used in \TT{taskmgr.exe} instead of \TT{MOV} to set the first argument of \\
\TT{NtQuerySystemInformation()}:

\lstinputlisting[caption=taskmgr.exe (Windows Vista),style=customasmx86]{examples/taskmgr/taskmgr2.lst}

\myindex{x86!\Instructions!LEA}

Perhaps \ac{MSVC} did so because machine code of \INS{LEA} is shorter than \INS{MOV REG, 5} (would be 5 instead of 4).

\INS{LEA} with offset in $-128..127$ range (offset will occupy 1 byte in opcode) with 32-bit registers is even shorter (for lack of REX prefix)---3 bytes.

Another example of such thing is: \myref{using_MOV_and_pack_of_LEA_to_load_values}.
}
\RU{\subsection{Обменять входные значения друг с другом}

Вот так:

\lstinputlisting[style=customc]{patterns/061_pointers/swap/5_RU.c}

Как видим, байты загружаются в младшие 8-битные части регистров \TT{ECX} и \TT{EBX} используя \INS{MOVZX}
(так что старшие части регистров очищаются), затем байты записываются назад в другом порядке.

\lstinputlisting[style=customasmx86,caption=Optimizing GCC 5.4]{patterns/061_pointers/swap/5_GCC_O3_x86.s}

Адреса обоих байтов берутся из аргументов и во время исполнения ф-ции находятся в регистрах \TT{EDX} и \TT{EAX}.

Так что исопльзуем указатели --- вероятно, без них нет способа решить эту задачу лучше.

}
\DE{\mysection{x86}

\subsection{Terminologie}

Geläufig für 16-Bit (8086/80286), 32-Bit (80386, etc.), 64-Bit.

\myindex{IEEE 754}
\myindex{MS-DOS}
\begin{description}
	\item[Byte] 8-Bit.
		Die DB Assembler-Direktive wird zum Definieren von Variablen und Arrays genutzt.
		Bytes werden in dem 8-Bit-Teil der folgenden Register übergeben:
		\TT{AL/BL/CL/DL/AH/BH/CH/DH/SIL/DIL/R*L}.
	\item[Wort] 16-Bit.
		DW Assembler-Direktive \dittoclosing.
		Bytes werden in dem 16-Bit-Teil der folgenden Register übergeben:
			\TT{AX/BX/CX/DX/SI/DI/R*W}.
	\item[Doppelwort] (\q{dword}) 32-Bit.
		DD Assembler-Direktive \dittoclosing.
		Doppelwörter werden in Registern (x86) oder dem 32-Bit-Teil der Register (x64) übergeben.
		In 16-Bit-Code werden Doppelwörter in 16-Bit-Registerpaaren übergeben.
	\item[zwei Doppelwörter] (\q{qword}) 64-Bit.
		DQ Assembler-Direktive \dittoclosing.
		In 32-Bit-Umgebungen werden diese in 32-Bit-Registerpaaren übergeben.
	\item[tbyte] (10 Byte) 80-Bit oder 10 Bytes (für IEEE 754 FPU Register).
	\item[paragraph] (16 Byte) --- Bezeichnung war in MS-DOS Umgebungen gebräuchlich.
\end{description}

\myindex{Windows!API}

Datentypen der selben Breite (BYTE, WORD, DWORD) entsprechen auch denen in der Windows \ac{API}.

% TODO German Translation (DSiekmeier)
%\input{appendix/x86/registers} % subsection
%\input{appendix/x86/instructions} % subsection
\subsection{npad}
\label{sec:npad}

\RU{Это макрос в ассемблере, для выравнивания некоторой метки по некоторой границе.}
\EN{It is an assembly language macro for aligning labels on a specific boundary.}
\DE{Dies ist ein Assembler-Makro um Labels an bestimmten Grenzen auszurichten.}
\FR{C'est une macro du langage d'assemblage pour aligner les labels sur une limite
spécifique.}

\RU{Это нужно для тех \emph{нагруженных} меток, куда чаще всего передается управление, например, 
начало тела цикла. 
Для того чтобы процессор мог эффективнее вытягивать данные или код из памяти, через шину с памятью, 
кэширование, итд.}
\EN{That's often needed for the busy labels to where the control flow is often passed, e.g., loop body starts.
So the CPU can load the data or code from the memory effectively, through the memory bus, cache lines, etc.}
\DE{Dies ist oft nützlich Labels, die oft Ziel einer Kotrollstruktur sind, wie Schleifenköpfe.
Somit kann die CPU Daten oder Code sehr effizient vom Speicher durch den Bus, den Cache, usw. laden.}
\FR{C'est souvent nécessaire pour des labels très utilisés, comme par exemple le
début d'un corps de boucle. Ainsi, le CPU peut charger les données ou le code depuis
la mémoire efficacement, à travers le bus mémoire, les caches, etc.}

\RU{Взято из}\EN{Taken from}\DE{Entnommen von}\FR{Pris de} \TT{listing.inc} (MSVC):

\myindex{x86!\Instructions!NOP}
\RU{Это, кстати, любопытный пример различных вариантов \NOP{}-ов. 
Все эти инструкции не дают никакого эффекта, но отличаются разной длиной.}
\EN{By the way, it is a curious example of the different \NOP variations.
All these instructions have no effects whatsoever, but have a different size.}
\DE{Dies ist übrigens ein Beispiel für die unterschiedlichen \NOP-Variationen.
Keine dieser Anweisungen hat eine Auswirkung, aber alle haben eine unterschiedliche Größe.}
\FR{À propos, c'est un exemple curieux des différentes variations de \NOP. Toutes
ces instructions n'ont pas d'effet, mais ont une taille différente.}

\RU{Цель в том, чтобы была только одна инструкция, а не набор NOP-ов, 
считается что так лучше для производительности CPU.}
\EN{Having a single idle instruction instead of couple of NOP-s,
is accepted to be better for CPU performance.}
\DE{Eine einzelne Idle-Anweisung anstatt mehrerer NOPs hat positive Auswirkungen
auf die CPU-Performance.}
\FR{Avoir une seule instruction sans effet au lieu de plusieurs est accepté comme
étant meilleur pour la performance du CPU.}

\begin{lstlisting}[style=customasmx86]
;; LISTING.INC
;;
;; This file contains assembler macros and is included by the files created
;; with the -FA compiler switch to be assembled by MASM (Microsoft Macro
;; Assembler).
;;
;; Copyright (c) 1993-2003, Microsoft Corporation. All rights reserved.

;; non destructive nops
npad macro size
if size eq 1
  nop
else
 if size eq 2
   mov edi, edi
 else
  if size eq 3
    ; lea ecx, [ecx+00]
    DB 8DH, 49H, 00H
  else
   if size eq 4
     ; lea esp, [esp+00]
     DB 8DH, 64H, 24H, 00H
   else
    if size eq 5
      add eax, DWORD PTR 0
    else
     if size eq 6
       ; lea ebx, [ebx+00000000]
       DB 8DH, 9BH, 00H, 00H, 00H, 00H
     else
      if size eq 7
	; lea esp, [esp+00000000]
	DB 8DH, 0A4H, 24H, 00H, 00H, 00H, 00H 
      else
       if size eq 8
        ; jmp .+8; .npad 6
	DB 0EBH, 06H, 8DH, 9BH, 00H, 00H, 00H, 00H
       else
        if size eq 9
         ; jmp .+9; .npad 7
         DB 0EBH, 07H, 8DH, 0A4H, 24H, 00H, 00H, 00H, 00H
        else
         if size eq 10
          ; jmp .+A; .npad 7; .npad 1
          DB 0EBH, 08H, 8DH, 0A4H, 24H, 00H, 00H, 00H, 00H, 90H
         else
          if size eq 11
           ; jmp .+B; .npad 7; .npad 2
           DB 0EBH, 09H, 8DH, 0A4H, 24H, 00H, 00H, 00H, 00H, 8BH, 0FFH
          else
           if size eq 12
            ; jmp .+C; .npad 7; .npad 3
            DB 0EBH, 0AH, 8DH, 0A4H, 24H, 00H, 00H, 00H, 00H, 8DH, 49H, 00H
           else
            if size eq 13
             ; jmp .+D; .npad 7; .npad 4
             DB 0EBH, 0BH, 8DH, 0A4H, 24H, 00H, 00H, 00H, 00H, 8DH, 64H, 24H, 00H
            else
             if size eq 14
              ; jmp .+E; .npad 7; .npad 5
              DB 0EBH, 0CH, 8DH, 0A4H, 24H, 00H, 00H, 00H, 00H, 05H, 00H, 00H, 00H, 00H
             else
              if size eq 15
               ; jmp .+F; .npad 7; .npad 6
               DB 0EBH, 0DH, 8DH, 0A4H, 24H, 00H, 00H, 00H, 00H, 8DH, 9BH, 00H, 00H, 00H, 00H
              else
	       %out error: unsupported npad size
               .err
              endif
             endif
            endif
           endif
          endif
         endif
        endif
       endif
      endif
     endif
    endif
   endif
  endif
 endif
endif
endm
\end{lstlisting}
 % subsection
}
\FR{\subsection{Exemple \#2: SCO OpenServer}

\label{examples_SCO}
\myindex{SCO OpenServer}
Un ancien logiciel pour SCO OpenServer de 1997 développé par une société qui a disparue
depuis longtemps.

Il y a un driver de dongle special à installer dans le système, qui contient les
chaînes de texte suivantes:
\q{Copyright 1989, Rainbow Technologies, Inc., Irvine, CA}
et
\q{Sentinel Integrated Driver Ver. 3.0 }.

Après l'installation du driver dans SCO OpenServer, ces fichiers apparaissent dans
l'arborescence /dev:

\begin{lstlisting}
/dev/rbsl8
/dev/rbsl9
/dev/rbsl10
\end{lstlisting}

Le programme renvoie une erreur lorsque le dongle n'est pas connecté, mais le message
d'erreur n'est pas trouvé dans les exécutables.

\myindex{COFF}

Grâce à \ac{IDA}, il est facile de charger l'exécutable COFF utilisé dans SCO OpenServer.

Essayons de trouver la chaîne \q{rbsl} et en effet, elle se trouve dans ce morceau
de code:

\lstinputlisting[style=customasmx86]{examples/dongles/2/1.lst}

Oui, en effet, le programme doit communiquer d'une façon ou d'une autre avec le driver.

\myindex{thunk-functions}
Le seul endroit où la fonction \TT{SSQC()} est appelée est dans la \glslink{thunk
 function}{fonction thunk}:

\lstinputlisting[style=customasmx86]{examples/dongles/2/2.lst}

SSQ() peut être appelé depuis au moins 2 fonctions.

L'une d'entre elles est:

\lstinputlisting[style=customasmx86]{examples/dongles/2/check1_EN.lst}

\q{\TT{3C}} et \q{\TT{3E}} semblent familiers: il y avait un dongle Sentinel Pro de
Rainbow sans mémoire, fournissant seulement une fonction de crypto-hachage secrète.

Vous pouvez lire une courte description de la fonction de hachage dont il s'agit
ici: \myref{hash_func}.

Mais retournons au programme.

Donc le programme peut seulement tester si un dongle est connecté ou s'il est absent.

Aucune autre information ne peut être écrite dans un tel dongle, puisqu'il n'a pas
de mémoire.
Les codes sur deux caractères sont des commandes (nous pouvons voir comment les commandes
sont traitées dans la fonction \TT{SSQC()}) et toutes les autres chaînes sont hachées
dans le dongle, transformées en un nombre 16-bit.
L'algorithme était secret, donc il n'était pas possible d'écrire un driver de remplacement
ou de refaire un dongle matériel qui l'émulerait parfaitement.

Toutefois, il est toujours possible d'intercepter tous les accès au dongle et de
trouver les constantes auxquelles les résultats de la fonction de hachage sont comparées.

Mais nous devons dire qu'il est possible de construire un schéma de logiciel de protection
de copie robuste basé sur une fonction secrète de hachage cryptographique: il suffit
qu'elle chiffre/déchiffre les fichiers de données utilisés par votre logiciel.

Mais retournons au code:

Les codes 51/52/53 sont utilisés pour choisir le port imprimante LPT.
3x/4x sont utilisés pour le choix de la \q{famille} (c'est ainsi que les dongles
Sentinel Pro sont différenciés les uns des autres: plus d'un dongle peut être connecté
sur un port LPT).

La seule chaîne passée à la fonction qui ne fasse pas 2 caractères est "0123456789".

Ensuite, le résultat est comparé à l'ensemble des résultats valides.

Si il est correct, 0xC ou 0xB est écrit dans la variable globale \TT{ctl\_model}.%

Une autre chaîne de texte qui est passée est
"PRESS ANY KEY TO CONTINUE: ", mais le résultat n'est pas testé.
Difficile de dire pourquoi, probablement une erreur\footnote{C'est un sentiment
étrange de trouver un bug dans un logiciel aussi ancien.}.

Voyons où la valeur de la variable globale \TT{ctl\_model} est utilisée.

Un tel endroit est:

\lstinputlisting[style=customasmx86]{examples/dongles/2/4.lst}

Si c'est 0, un message d'erreur chiffré est passé à une routine de déchiffrement
et affiché.

\myindex{x86!\Instructions!XOR}

La routine de déchiffrement de la chaîne semble être un simple \glslink{xoring}{xor}:

\lstinputlisting[style=customasmx86]{examples/dongles/2/err_warn.lst}

C'est pourquoi nous étions incapable de trouver le message d'erreur dans les fichiers
exécutable, car ils sont chiffrés (ce qui est une pratique courante).

Un autre appel à la fonction de hachage \TT{SSQ()} lui passe la chaîne \q{offln}
et le résultat est comparé avec \TT{0xFE81} et \TT{0x12A9}.

Si ils ne correspondent pas, ça se comporte comme une sorte de fonction \TT{timer()}
(peut-être en attente qu'un dongle mal connecté soit reconnecté et re-testé?) et ensuite
déchiffre un autre message d'erreur à afficher.

\lstinputlisting[style=customasmx86]{examples/dongles/2/check2_EN.lst}

Passer outre le dongle est assez facile: il suffit de patcher tous les sauts après
les instructions \CMP pertinentes.

Une autre option est d'écrire notre propre driver SCO OpenServer, contenant une table
de questions et de réponses, toutes celles qui sont présentent dans le programme.

\subsubsection{Déchiffrer les messages d'erreur}

À propos, nous pouvons aussi essayer de déchiffrer tous les messages d'erreurs.
L'algorithme qui se trouve dans la fonction \TT{err\_warn()} est très simple, en effet:

\lstinputlisting[caption=Decryption function,style=customasmx86]{examples/dongles/2/decrypting_FR.lst}

Comme on le voit, non seulement la chaîne est transmise à la fonction de déchiffrement
mais aussi la clef:

\lstinputlisting[style=customasmx86]{examples/dongles/2/tmp1_EN.asm}

L'algorithme est un simple \glslink{xoring}{xor}: chaque octet est xoré avec la clef, mais
la clef est incrémentée de 3 après le traitement de chaque octet.

Nous pouvons écrire un petit script Python pour vérifier notre hypothèse:

\lstinputlisting[caption=Python 3.x]{examples/dongles/2/decr1.py}

Et il affiche: \q{check security device connection}.
Donc oui, ceci est le message déchiffré.

Il y a d'autres messages chiffrés, avec leur clef correspondante.
Mais inutile de dire qu'il est possible de les déchiffrer sans leur clef.
Premièrement, nous voyons que le clef est en fait un octet.
C'est parce que l'instruction principale de déchiffrement (\XOR) fonctionne au niveau
de l'octet.
La clef se trouve dans le registre \ESI, mais seulement une partie de \ESI d'un octet
est utilisée.
Ainsi, une clef pourrait être plus grande que 255, mais sa valeur est toujours arrondie.

En conséquence, nous pouvons simplement essayer de brute-forcer, en essayant toutes
les clefs possible dans l'intervalle 0..255.
Nous allons aussi écarter les messages comportants des caractères non-imprimable.

\lstinputlisting[caption=Python 3.x]{examples/dongles/2/decr2.py}

Et nous obtenons:

\lstinputlisting[caption=Results]{examples/dongles/2/decr2_result.txt}

Ici il y a un peu de déchet, mais nous pouvons rapidement trouver les messages en
anglais.

À propos, puisque l'algorithme est un simple chiffrement xor, la même fonction peut
être utilisée pour chiffrer les messages.
Si besoin, nous pouvons chiffrer nos propres messages, et patcher le programme en les insérant.
}
\IT{\subsection{Fall-through}

Un altro uso diffuso dell'operatore \TT{switch()} è il cosiddetto \q{fallthrough}.
Ecco un semplice esempio \footnote{Preso da \url{https://github.com/azonalon/prgraas/blob/master/prog1lib/lecture_examples/is_whitespace.c}}:

\lstinputlisting[numbers=left,style=customc]{patterns/08_switch/4_fallthrough/fallthrough1.c}

Uno leggermente più difficile, dal kernel di Linux \footnote{Preso da \url{https://github.com/torvalds/linux/blob/master/drivers/media/dvb-frontends/lgdt3306a.c}}:

\lstinputlisting[numbers=left,style=customc]{patterns/08_switch/4_fallthrough/fallthrough2.c}

\lstinputlisting[caption=Optimizing GCC 5.4.0 x86,numbers=left,style=customasmx86]{patterns/08_switch/4_fallthrough/fallthrough2.s}

Possiamo arrivare alla label \TT{.L5} se all'input della funzione viene dato il valore 3250.
Ma si può anche giungere allo stesso punto da un altro percorso:
notiamo che non ci sono jump tra la chiamata a \printf e la label \TT{.L5}.

Questo spiega facilmente perchè i costrutti con \emph{switch()} sono spesso fonte di bug:
è sufficiente dimenticare un \emph{break} per trasformare il costrutto \emph{switch()} in un \emph{fallthrough} , in cui vengono eseguiti
più blocchi invece di uno solo.
}

\EN{\mysection{Task manager practical joke (Windows Vista)}
\myindex{Windows!Windows Vista}

Let's see if it's possible to hack Task Manager slightly so it would detect more \ac{CPU} cores.

\myindex{Windows!NTAPI}

Let us first think, how does the Task Manager know the number of cores?

There is the \TT{GetSystemInfo()} win32 function present in win32 userspace which can tell us this.
But it's not imported in \TT{taskmgr.exe}.

There is, however, another one in \gls{NTAPI}, \TT{NtQuerySystemInformation()}, 
which is used in \TT{taskmgr.exe} in several places.

To get the number of cores, one has to call this function with the \TT{SystemBasicInformation} constant
as a first argument (which is zero
\footnote{\href{http://msdn.microsoft.com/en-us/library/windows/desktop/ms724509(v=vs.85).aspx}{MSDN}}).

The second argument has to point to the buffer which is getting all the information.

So we have to find all calls to the \\
\TT{NtQuerySystemInformation(0, ?, ?, ?)} function.
Let's open \TT{taskmgr.exe} in IDA. 
\myindex{Windows!PDB}

What is always good about Microsoft executables is that IDA can download the corresponding \gls{PDB} 
file for this executable and show all function names.

It is visible that Task Manager is written in \Cpp and some of the function names and classes are really 
speaking for themselves.
There are classes CAdapter, CNetPage, CPerfPage, CProcInfo, CProcPage, CSvcPage, 
CTaskPage, CUserPage.

Apparently, each class corresponds to each tab in Task Manager.

Let's visit each call and add comment with the value which is passed as the first function argument.
We will write \q{not zero} at some places, because the value there was clearly not zero, 
but something really different (more about this in the second part of this chapter).

And we are looking for zero passed as argument, after all.

\begin{figure}[H]
\centering
\myincludegraphics{examples/taskmgr/IDA_xrefs.png}
\caption{IDA: cross references to NtQuerySystemInformation()}
\end{figure}

Yes, the names are really speaking for themselves.

When we closely investigate each place where\\
\TT{NtQuerySystemInformation(0, ?, ?, ?)} is called,
we quickly find what we need in the \TT{InitPerfInfo()} function:

\lstinputlisting[caption=taskmgr.exe (Windows Vista),style=customasmx86]{examples/taskmgr/taskmgr.lst}

\TT{g\_cProcessors} is a global variable, and this name has been assigned by 
IDA according to the \gls{PDB} loaded from Microsoft's symbol server.

The byte is taken from \TT{var\_C20}. 
And \TT{var\_C58} is passed to\\
\TT{NtQuerySystemInformation()} 
as a pointer to the receiving buffer.
The difference between 0xC20 and 0xC58 is 0x38 (56).

Let's take a look at format of the return structure, which we can find in MSDN:

\begin{lstlisting}[style=customc]
typedef struct _SYSTEM_BASIC_INFORMATION {
    BYTE Reserved1[24];
    PVOID Reserved2[4];
    CCHAR NumberOfProcessors;
} SYSTEM_BASIC_INFORMATION;
\end{lstlisting}

This is a x64 system, so each PVOID takes 8 bytes.

All \emph{reserved} fields in the structure take $24+4*8=56$ bytes.

Oh yes, this implies that \TT{var\_C20} is the local stack is exactly the
\TT{NumberOfProcessors} field of the \TT{SYSTEM\_BASIC\_INFORMATION} structure.

Let's check our guess.
Copy \TT{taskmgr.exe} from \TT{C:\textbackslash{}Windows\textbackslash{}System32} 
to some other folder 
(so the \emph{Windows Resource Protection} 
will not try to restore the patched \TT{taskmgr.exe}).

Let's open it in Hiew and find the place:

\begin{figure}[H]
\centering
\myincludegraphics{examples/taskmgr/hiew2.png}
\caption{Hiew: find the place to be patched}
\end{figure}

Let's replace the \TT{MOVZX} instruction with ours.
Let's pretend we've got 64 CPU cores.

Add one additional \ac{NOP} (because our instruction is shorter than the original one):

\begin{figure}[H]
\centering
\myincludegraphics{examples/taskmgr/hiew1.png}
\caption{Hiew: patch it}
\end{figure}

And it works!
Of course, the data in the graphs is not correct.

At times, Task Manager even shows an overall CPU load of more than 100\%.

\begin{figure}[H]
\centering
\myincludegraphics{examples/taskmgr/taskmgr_64cpu_crop.png}
\caption{Fooled Windows Task Manager}
\end{figure}

The biggest number Task Manager does not crash with is 64.

Apparently, Task Manager in Windows Vista was not tested on computers with a large number of cores.

So there are probably some static data structure(s) inside it limited to 64 cores.

\subsection{Using LEA to load values}
\label{TaskMgr_LEA}

Sometimes, \TT{LEA} is used in \TT{taskmgr.exe} instead of \TT{MOV} to set the first argument of \\
\TT{NtQuerySystemInformation()}:

\lstinputlisting[caption=taskmgr.exe (Windows Vista),style=customasmx86]{examples/taskmgr/taskmgr2.lst}

\myindex{x86!\Instructions!LEA}

Perhaps \ac{MSVC} did so because machine code of \INS{LEA} is shorter than \INS{MOV REG, 5} (would be 5 instead of 4).

\INS{LEA} with offset in $-128..127$ range (offset will occupy 1 byte in opcode) with 32-bit registers is even shorter (for lack of REX prefix)---3 bytes.

Another example of such thing is: \myref{using_MOV_and_pack_of_LEA_to_load_values}.
}
\RU{\subsection{Обменять входные значения друг с другом}

Вот так:

\lstinputlisting[style=customc]{patterns/061_pointers/swap/5_RU.c}

Как видим, байты загружаются в младшие 8-битные части регистров \TT{ECX} и \TT{EBX} используя \INS{MOVZX}
(так что старшие части регистров очищаются), затем байты записываются назад в другом порядке.

\lstinputlisting[style=customasmx86,caption=Optimizing GCC 5.4]{patterns/061_pointers/swap/5_GCC_O3_x86.s}

Адреса обоих байтов берутся из аргументов и во время исполнения ф-ции находятся в регистрах \TT{EDX} и \TT{EAX}.

Так что исопльзуем указатели --- вероятно, без них нет способа решить эту задачу лучше.

}
\FR{\subsection{Exemple \#2: SCO OpenServer}

\label{examples_SCO}
\myindex{SCO OpenServer}
Un ancien logiciel pour SCO OpenServer de 1997 développé par une société qui a disparue
depuis longtemps.

Il y a un driver de dongle special à installer dans le système, qui contient les
chaînes de texte suivantes:
\q{Copyright 1989, Rainbow Technologies, Inc., Irvine, CA}
et
\q{Sentinel Integrated Driver Ver. 3.0 }.

Après l'installation du driver dans SCO OpenServer, ces fichiers apparaissent dans
l'arborescence /dev:

\begin{lstlisting}
/dev/rbsl8
/dev/rbsl9
/dev/rbsl10
\end{lstlisting}

Le programme renvoie une erreur lorsque le dongle n'est pas connecté, mais le message
d'erreur n'est pas trouvé dans les exécutables.

\myindex{COFF}

Grâce à \ac{IDA}, il est facile de charger l'exécutable COFF utilisé dans SCO OpenServer.

Essayons de trouver la chaîne \q{rbsl} et en effet, elle se trouve dans ce morceau
de code:

\lstinputlisting[style=customasmx86]{examples/dongles/2/1.lst}

Oui, en effet, le programme doit communiquer d'une façon ou d'une autre avec le driver.

\myindex{thunk-functions}
Le seul endroit où la fonction \TT{SSQC()} est appelée est dans la \glslink{thunk
 function}{fonction thunk}:

\lstinputlisting[style=customasmx86]{examples/dongles/2/2.lst}

SSQ() peut être appelé depuis au moins 2 fonctions.

L'une d'entre elles est:

\lstinputlisting[style=customasmx86]{examples/dongles/2/check1_EN.lst}

\q{\TT{3C}} et \q{\TT{3E}} semblent familiers: il y avait un dongle Sentinel Pro de
Rainbow sans mémoire, fournissant seulement une fonction de crypto-hachage secrète.

Vous pouvez lire une courte description de la fonction de hachage dont il s'agit
ici: \myref{hash_func}.

Mais retournons au programme.

Donc le programme peut seulement tester si un dongle est connecté ou s'il est absent.

Aucune autre information ne peut être écrite dans un tel dongle, puisqu'il n'a pas
de mémoire.
Les codes sur deux caractères sont des commandes (nous pouvons voir comment les commandes
sont traitées dans la fonction \TT{SSQC()}) et toutes les autres chaînes sont hachées
dans le dongle, transformées en un nombre 16-bit.
L'algorithme était secret, donc il n'était pas possible d'écrire un driver de remplacement
ou de refaire un dongle matériel qui l'émulerait parfaitement.

Toutefois, il est toujours possible d'intercepter tous les accès au dongle et de
trouver les constantes auxquelles les résultats de la fonction de hachage sont comparées.

Mais nous devons dire qu'il est possible de construire un schéma de logiciel de protection
de copie robuste basé sur une fonction secrète de hachage cryptographique: il suffit
qu'elle chiffre/déchiffre les fichiers de données utilisés par votre logiciel.

Mais retournons au code:

Les codes 51/52/53 sont utilisés pour choisir le port imprimante LPT.
3x/4x sont utilisés pour le choix de la \q{famille} (c'est ainsi que les dongles
Sentinel Pro sont différenciés les uns des autres: plus d'un dongle peut être connecté
sur un port LPT).

La seule chaîne passée à la fonction qui ne fasse pas 2 caractères est "0123456789".

Ensuite, le résultat est comparé à l'ensemble des résultats valides.

Si il est correct, 0xC ou 0xB est écrit dans la variable globale \TT{ctl\_model}.%

Une autre chaîne de texte qui est passée est
"PRESS ANY KEY TO CONTINUE: ", mais le résultat n'est pas testé.
Difficile de dire pourquoi, probablement une erreur\footnote{C'est un sentiment
étrange de trouver un bug dans un logiciel aussi ancien.}.

Voyons où la valeur de la variable globale \TT{ctl\_model} est utilisée.

Un tel endroit est:

\lstinputlisting[style=customasmx86]{examples/dongles/2/4.lst}

Si c'est 0, un message d'erreur chiffré est passé à une routine de déchiffrement
et affiché.

\myindex{x86!\Instructions!XOR}

La routine de déchiffrement de la chaîne semble être un simple \glslink{xoring}{xor}:

\lstinputlisting[style=customasmx86]{examples/dongles/2/err_warn.lst}

C'est pourquoi nous étions incapable de trouver le message d'erreur dans les fichiers
exécutable, car ils sont chiffrés (ce qui est une pratique courante).

Un autre appel à la fonction de hachage \TT{SSQ()} lui passe la chaîne \q{offln}
et le résultat est comparé avec \TT{0xFE81} et \TT{0x12A9}.

Si ils ne correspondent pas, ça se comporte comme une sorte de fonction \TT{timer()}
(peut-être en attente qu'un dongle mal connecté soit reconnecté et re-testé?) et ensuite
déchiffre un autre message d'erreur à afficher.

\lstinputlisting[style=customasmx86]{examples/dongles/2/check2_EN.lst}

Passer outre le dongle est assez facile: il suffit de patcher tous les sauts après
les instructions \CMP pertinentes.

Une autre option est d'écrire notre propre driver SCO OpenServer, contenant une table
de questions et de réponses, toutes celles qui sont présentent dans le programme.

\subsubsection{Déchiffrer les messages d'erreur}

À propos, nous pouvons aussi essayer de déchiffrer tous les messages d'erreurs.
L'algorithme qui se trouve dans la fonction \TT{err\_warn()} est très simple, en effet:

\lstinputlisting[caption=Decryption function,style=customasmx86]{examples/dongles/2/decrypting_FR.lst}

Comme on le voit, non seulement la chaîne est transmise à la fonction de déchiffrement
mais aussi la clef:

\lstinputlisting[style=customasmx86]{examples/dongles/2/tmp1_EN.asm}

L'algorithme est un simple \glslink{xoring}{xor}: chaque octet est xoré avec la clef, mais
la clef est incrémentée de 3 après le traitement de chaque octet.

Nous pouvons écrire un petit script Python pour vérifier notre hypothèse:

\lstinputlisting[caption=Python 3.x]{examples/dongles/2/decr1.py}

Et il affiche: \q{check security device connection}.
Donc oui, ceci est le message déchiffré.

Il y a d'autres messages chiffrés, avec leur clef correspondante.
Mais inutile de dire qu'il est possible de les déchiffrer sans leur clef.
Premièrement, nous voyons que le clef est en fait un octet.
C'est parce que l'instruction principale de déchiffrement (\XOR) fonctionne au niveau
de l'octet.
La clef se trouve dans le registre \ESI, mais seulement une partie de \ESI d'un octet
est utilisée.
Ainsi, une clef pourrait être plus grande que 255, mais sa valeur est toujours arrondie.

En conséquence, nous pouvons simplement essayer de brute-forcer, en essayant toutes
les clefs possible dans l'intervalle 0..255.
Nous allons aussi écarter les messages comportants des caractères non-imprimable.

\lstinputlisting[caption=Python 3.x]{examples/dongles/2/decr2.py}

Et nous obtenons:

\lstinputlisting[caption=Results]{examples/dongles/2/decr2_result.txt}

Ici il y a un peu de déchet, mais nous pouvons rapidement trouver les messages en
anglais.

À propos, puisque l'algorithme est un simple chiffrement xor, la même fonction peut
être utilisée pour chiffrer les messages.
Si besoin, nous pouvons chiffrer nos propres messages, et patcher le programme en les insérant.
}

\RU{\subsection{Обменять входные значения друг с другом}

Вот так:

\lstinputlisting[style=customc]{patterns/061_pointers/swap/5_RU.c}

Как видим, байты загружаются в младшие 8-битные части регистров \TT{ECX} и \TT{EBX} используя \INS{MOVZX}
(так что старшие части регистров очищаются), затем байты записываются назад в другом порядке.

\lstinputlisting[style=customasmx86,caption=Optimizing GCC 5.4]{patterns/061_pointers/swap/5_GCC_O3_x86.s}

Адреса обоих байтов берутся из аргументов и во время исполнения ф-ции находятся в регистрах \TT{EDX} и \TT{EAX}.

Так что исопльзуем указатели --- вероятно, без них нет способа решить эту задачу лучше.

}
\EN{\mysection{Task manager practical joke (Windows Vista)}
\myindex{Windows!Windows Vista}

Let's see if it's possible to hack Task Manager slightly so it would detect more \ac{CPU} cores.

\myindex{Windows!NTAPI}

Let us first think, how does the Task Manager know the number of cores?

There is the \TT{GetSystemInfo()} win32 function present in win32 userspace which can tell us this.
But it's not imported in \TT{taskmgr.exe}.

There is, however, another one in \gls{NTAPI}, \TT{NtQuerySystemInformation()}, 
which is used in \TT{taskmgr.exe} in several places.

To get the number of cores, one has to call this function with the \TT{SystemBasicInformation} constant
as a first argument (which is zero
\footnote{\href{http://msdn.microsoft.com/en-us/library/windows/desktop/ms724509(v=vs.85).aspx}{MSDN}}).

The second argument has to point to the buffer which is getting all the information.

So we have to find all calls to the \\
\TT{NtQuerySystemInformation(0, ?, ?, ?)} function.
Let's open \TT{taskmgr.exe} in IDA. 
\myindex{Windows!PDB}

What is always good about Microsoft executables is that IDA can download the corresponding \gls{PDB} 
file for this executable and show all function names.

It is visible that Task Manager is written in \Cpp and some of the function names and classes are really 
speaking for themselves.
There are classes CAdapter, CNetPage, CPerfPage, CProcInfo, CProcPage, CSvcPage, 
CTaskPage, CUserPage.

Apparently, each class corresponds to each tab in Task Manager.

Let's visit each call and add comment with the value which is passed as the first function argument.
We will write \q{not zero} at some places, because the value there was clearly not zero, 
but something really different (more about this in the second part of this chapter).

And we are looking for zero passed as argument, after all.

\begin{figure}[H]
\centering
\myincludegraphics{examples/taskmgr/IDA_xrefs.png}
\caption{IDA: cross references to NtQuerySystemInformation()}
\end{figure}

Yes, the names are really speaking for themselves.

When we closely investigate each place where\\
\TT{NtQuerySystemInformation(0, ?, ?, ?)} is called,
we quickly find what we need in the \TT{InitPerfInfo()} function:

\lstinputlisting[caption=taskmgr.exe (Windows Vista),style=customasmx86]{examples/taskmgr/taskmgr.lst}

\TT{g\_cProcessors} is a global variable, and this name has been assigned by 
IDA according to the \gls{PDB} loaded from Microsoft's symbol server.

The byte is taken from \TT{var\_C20}. 
And \TT{var\_C58} is passed to\\
\TT{NtQuerySystemInformation()} 
as a pointer to the receiving buffer.
The difference between 0xC20 and 0xC58 is 0x38 (56).

Let's take a look at format of the return structure, which we can find in MSDN:

\begin{lstlisting}[style=customc]
typedef struct _SYSTEM_BASIC_INFORMATION {
    BYTE Reserved1[24];
    PVOID Reserved2[4];
    CCHAR NumberOfProcessors;
} SYSTEM_BASIC_INFORMATION;
\end{lstlisting}

This is a x64 system, so each PVOID takes 8 bytes.

All \emph{reserved} fields in the structure take $24+4*8=56$ bytes.

Oh yes, this implies that \TT{var\_C20} is the local stack is exactly the
\TT{NumberOfProcessors} field of the \TT{SYSTEM\_BASIC\_INFORMATION} structure.

Let's check our guess.
Copy \TT{taskmgr.exe} from \TT{C:\textbackslash{}Windows\textbackslash{}System32} 
to some other folder 
(so the \emph{Windows Resource Protection} 
will not try to restore the patched \TT{taskmgr.exe}).

Let's open it in Hiew and find the place:

\begin{figure}[H]
\centering
\myincludegraphics{examples/taskmgr/hiew2.png}
\caption{Hiew: find the place to be patched}
\end{figure}

Let's replace the \TT{MOVZX} instruction with ours.
Let's pretend we've got 64 CPU cores.

Add one additional \ac{NOP} (because our instruction is shorter than the original one):

\begin{figure}[H]
\centering
\myincludegraphics{examples/taskmgr/hiew1.png}
\caption{Hiew: patch it}
\end{figure}

And it works!
Of course, the data in the graphs is not correct.

At times, Task Manager even shows an overall CPU load of more than 100\%.

\begin{figure}[H]
\centering
\myincludegraphics{examples/taskmgr/taskmgr_64cpu_crop.png}
\caption{Fooled Windows Task Manager}
\end{figure}

The biggest number Task Manager does not crash with is 64.

Apparently, Task Manager in Windows Vista was not tested on computers with a large number of cores.

So there are probably some static data structure(s) inside it limited to 64 cores.

\subsection{Using LEA to load values}
\label{TaskMgr_LEA}

Sometimes, \TT{LEA} is used in \TT{taskmgr.exe} instead of \TT{MOV} to set the first argument of \\
\TT{NtQuerySystemInformation()}:

\lstinputlisting[caption=taskmgr.exe (Windows Vista),style=customasmx86]{examples/taskmgr/taskmgr2.lst}

\myindex{x86!\Instructions!LEA}

Perhaps \ac{MSVC} did so because machine code of \INS{LEA} is shorter than \INS{MOV REG, 5} (would be 5 instead of 4).

\INS{LEA} with offset in $-128..127$ range (offset will occupy 1 byte in opcode) with 32-bit registers is even shorter (for lack of REX prefix)---3 bytes.

Another example of such thing is: \myref{using_MOV_and_pack_of_LEA_to_load_values}.
}
\FR{\subsection{Exemple \#2: SCO OpenServer}

\label{examples_SCO}
\myindex{SCO OpenServer}
Un ancien logiciel pour SCO OpenServer de 1997 développé par une société qui a disparue
depuis longtemps.

Il y a un driver de dongle special à installer dans le système, qui contient les
chaînes de texte suivantes:
\q{Copyright 1989, Rainbow Technologies, Inc., Irvine, CA}
et
\q{Sentinel Integrated Driver Ver. 3.0 }.

Après l'installation du driver dans SCO OpenServer, ces fichiers apparaissent dans
l'arborescence /dev:

\begin{lstlisting}
/dev/rbsl8
/dev/rbsl9
/dev/rbsl10
\end{lstlisting}

Le programme renvoie une erreur lorsque le dongle n'est pas connecté, mais le message
d'erreur n'est pas trouvé dans les exécutables.

\myindex{COFF}

Grâce à \ac{IDA}, il est facile de charger l'exécutable COFF utilisé dans SCO OpenServer.

Essayons de trouver la chaîne \q{rbsl} et en effet, elle se trouve dans ce morceau
de code:

\lstinputlisting[style=customasmx86]{examples/dongles/2/1.lst}

Oui, en effet, le programme doit communiquer d'une façon ou d'une autre avec le driver.

\myindex{thunk-functions}
Le seul endroit où la fonction \TT{SSQC()} est appelée est dans la \glslink{thunk
 function}{fonction thunk}:

\lstinputlisting[style=customasmx86]{examples/dongles/2/2.lst}

SSQ() peut être appelé depuis au moins 2 fonctions.

L'une d'entre elles est:

\lstinputlisting[style=customasmx86]{examples/dongles/2/check1_EN.lst}

\q{\TT{3C}} et \q{\TT{3E}} semblent familiers: il y avait un dongle Sentinel Pro de
Rainbow sans mémoire, fournissant seulement une fonction de crypto-hachage secrète.

Vous pouvez lire une courte description de la fonction de hachage dont il s'agit
ici: \myref{hash_func}.

Mais retournons au programme.

Donc le programme peut seulement tester si un dongle est connecté ou s'il est absent.

Aucune autre information ne peut être écrite dans un tel dongle, puisqu'il n'a pas
de mémoire.
Les codes sur deux caractères sont des commandes (nous pouvons voir comment les commandes
sont traitées dans la fonction \TT{SSQC()}) et toutes les autres chaînes sont hachées
dans le dongle, transformées en un nombre 16-bit.
L'algorithme était secret, donc il n'était pas possible d'écrire un driver de remplacement
ou de refaire un dongle matériel qui l'émulerait parfaitement.

Toutefois, il est toujours possible d'intercepter tous les accès au dongle et de
trouver les constantes auxquelles les résultats de la fonction de hachage sont comparées.

Mais nous devons dire qu'il est possible de construire un schéma de logiciel de protection
de copie robuste basé sur une fonction secrète de hachage cryptographique: il suffit
qu'elle chiffre/déchiffre les fichiers de données utilisés par votre logiciel.

Mais retournons au code:

Les codes 51/52/53 sont utilisés pour choisir le port imprimante LPT.
3x/4x sont utilisés pour le choix de la \q{famille} (c'est ainsi que les dongles
Sentinel Pro sont différenciés les uns des autres: plus d'un dongle peut être connecté
sur un port LPT).

La seule chaîne passée à la fonction qui ne fasse pas 2 caractères est "0123456789".

Ensuite, le résultat est comparé à l'ensemble des résultats valides.

Si il est correct, 0xC ou 0xB est écrit dans la variable globale \TT{ctl\_model}.%

Une autre chaîne de texte qui est passée est
"PRESS ANY KEY TO CONTINUE: ", mais le résultat n'est pas testé.
Difficile de dire pourquoi, probablement une erreur\footnote{C'est un sentiment
étrange de trouver un bug dans un logiciel aussi ancien.}.

Voyons où la valeur de la variable globale \TT{ctl\_model} est utilisée.

Un tel endroit est:

\lstinputlisting[style=customasmx86]{examples/dongles/2/4.lst}

Si c'est 0, un message d'erreur chiffré est passé à une routine de déchiffrement
et affiché.

\myindex{x86!\Instructions!XOR}

La routine de déchiffrement de la chaîne semble être un simple \glslink{xoring}{xor}:

\lstinputlisting[style=customasmx86]{examples/dongles/2/err_warn.lst}

C'est pourquoi nous étions incapable de trouver le message d'erreur dans les fichiers
exécutable, car ils sont chiffrés (ce qui est une pratique courante).

Un autre appel à la fonction de hachage \TT{SSQ()} lui passe la chaîne \q{offln}
et le résultat est comparé avec \TT{0xFE81} et \TT{0x12A9}.

Si ils ne correspondent pas, ça se comporte comme une sorte de fonction \TT{timer()}
(peut-être en attente qu'un dongle mal connecté soit reconnecté et re-testé?) et ensuite
déchiffre un autre message d'erreur à afficher.

\lstinputlisting[style=customasmx86]{examples/dongles/2/check2_EN.lst}

Passer outre le dongle est assez facile: il suffit de patcher tous les sauts après
les instructions \CMP pertinentes.

Une autre option est d'écrire notre propre driver SCO OpenServer, contenant une table
de questions et de réponses, toutes celles qui sont présentent dans le programme.

\subsubsection{Déchiffrer les messages d'erreur}

À propos, nous pouvons aussi essayer de déchiffrer tous les messages d'erreurs.
L'algorithme qui se trouve dans la fonction \TT{err\_warn()} est très simple, en effet:

\lstinputlisting[caption=Decryption function,style=customasmx86]{examples/dongles/2/decrypting_FR.lst}

Comme on le voit, non seulement la chaîne est transmise à la fonction de déchiffrement
mais aussi la clef:

\lstinputlisting[style=customasmx86]{examples/dongles/2/tmp1_EN.asm}

L'algorithme est un simple \glslink{xoring}{xor}: chaque octet est xoré avec la clef, mais
la clef est incrémentée de 3 après le traitement de chaque octet.

Nous pouvons écrire un petit script Python pour vérifier notre hypothèse:

\lstinputlisting[caption=Python 3.x]{examples/dongles/2/decr1.py}

Et il affiche: \q{check security device connection}.
Donc oui, ceci est le message déchiffré.

Il y a d'autres messages chiffrés, avec leur clef correspondante.
Mais inutile de dire qu'il est possible de les déchiffrer sans leur clef.
Premièrement, nous voyons que le clef est en fait un octet.
C'est parce que l'instruction principale de déchiffrement (\XOR) fonctionne au niveau
de l'octet.
La clef se trouve dans le registre \ESI, mais seulement une partie de \ESI d'un octet
est utilisée.
Ainsi, une clef pourrait être plus grande que 255, mais sa valeur est toujours arrondie.

En conséquence, nous pouvons simplement essayer de brute-forcer, en essayant toutes
les clefs possible dans l'intervalle 0..255.
Nous allons aussi écarter les messages comportants des caractères non-imprimable.

\lstinputlisting[caption=Python 3.x]{examples/dongles/2/decr2.py}

Et nous obtenons:

\lstinputlisting[caption=Results]{examples/dongles/2/decr2_result.txt}

Ici il y a un peu de déchet, mais nous pouvons rapidement trouver les messages en
anglais.

À propos, puisque l'algorithme est un simple chiffrement xor, la même fonction peut
être utilisée pour chiffrer les messages.
Si besoin, nous pouvons chiffrer nos propres messages, et patcher le programme en les insérant.
}

\EN{\mysection{Some GCC library functions}
\myindex{GCC}
\label{sec:GCC_library_func}

%__ashldi3
%__ashrdi3
%__floatundidf
%__floatdisf
%__floatdixf
%__floatundidf
%__floatundisf
%__floatundixf
%__lshrdi3
%__muldi3

\begin{center}
\begin{tabular}{ | l | l | }
\hline
\HeaderColor name & \HeaderColor meaning \\
\hline \TT{\_\_divdi3} & signed division \\
\hline \TT{\_\_moddi3} & getting remainder (modulo) of signed division \\
\hline \TT{\_\_udivdi3} & unsigned division \\
\hline \TT{\_\_umoddi3} & getting remainder (modulo) of unsigned division \\
\hline
\end{tabular}
\end{center}

}
\RU{\mysection{Некоторые библиотечные функции GCC}
\myindex{GCC}
\label{sec:GCC_library_func}

%__ashldi3
%__ashrdi3
%__floatundidf
%__floatdisf
%__floatdixf
%__floatundidf
%__floatundisf
%__floatundixf
%__lshrdi3
%__muldi3

\begin{center}
\begin{tabular}{ | l | l | }
\hline
\HeaderColor имя & \HeaderColor значение \\
\hline \TT{\_\_divdi3} & знаковое деление \\
\hline \TT{\_\_moddi3} & остаток от знакового деления \\
\hline \TT{\_\_udivdi3} & беззнаковое деление \\
\hline \TT{\_\_umoddi3} & остаток от беззнакового деления \\
\hline
\end{tabular}
\end{center}

}
\FR{\mysection{Quelques fonctions de la bibliothèque de GCC}
\myindex{GCC}
\label{sec:GCC_library_func}

%__ashldi3
%__ashrdi3
%__floatundidf
%__floatdisf
%__floatdixf
%__floatundidf
%__floatundisf
%__floatundixf
%__lshrdi3
%__muldi3

\begin{center}
\begin{tabular}{ | l | l | }
\hline
\HeaderColor nom & \HeaderColor signification \\
\hline \TT{\_\_divdi3} & division signée \\
\hline \TT{\_\_moddi3} & reste (modulo) d'une division signée \\
\hline \TT{\_\_udivdi3} & division non signée \\
\hline \TT{\_\_umoddi3} & reste (modulo) d'une division non signée \\
\hline
\end{tabular}
\end{center}

}
\DE{\mysection{Einige GCC-Bibliotheks-Funktionen}
\myindex{GCC}
\label{sec:GCC_library_func}

%__ashldi3
%__ashrdi3
%__floatundidf
%__floatdisf
%__floatdixf
%__floatundidf
%__floatundisf
%__floatundixf
%__lshrdi3
%__muldi3

\begin{center}
\begin{tabular}{ | l | l | }
\hline
\HeaderColor Name & \HeaderColor Bedeutung \\
\hline \TT{\_\_divdi3} & vorzeichenbehaftete Division \\
\hline \TT{\_\_moddi3} & Rest (Modulo) einer vorzeichenbehafteten Division \\
\hline \TT{\_\_udivdi3} & vorzeichenlose Division \\
\hline \TT{\_\_umoddi3} & Rest (Modulo) einer vorzeichenlosen Division \\
\hline
\end{tabular}
\end{center}

}


\mysection{\RU{Некоторые библиотечные функции MSVC}\EN{Some MSVC library functions}\DE{Einige MSVC-Bibliotheks-Funktionen}%
\FR{Quelques fonctions de la bibliothèque MSVC}}
\myindex{MSVC}
\label{sec:MSVC_library_func}

\TT{ll} \RU{в имени функции означает}\EN{in function name stands for}\DE{in Funktionsnamen steht für}%
\FR{dans une fontion signifie} \q{long long}, \RU{т.е. 64-битный тип данных}
\EN{e.g., a 64-bit data type}\DE{z.B. einen 64-Bit-Datentyp}\FR{i.e., type de donées 64-bit}.

\begin{center}
\begin{tabular}{ | l | l | }
\hline
\HeaderColor \RU{имя}\EN{name}\DE{Name}\FR{nom} & \HeaderColor \RU{значение}\EN{meaning}\DE{Bedeutung}\FR{signification} \\
\hline \TT{\_\_alldiv} & \RU{знаковое деление}\EN{signed division}\DE{vorzeichenbehaftete Division}\FR{division signée} \\
\hline \TT{\_\_allmul} & \RU{умножение}\EN{multiplication}\DE{Multiplikation}\FR{multiplication} \\
\hline \TT{\_\_allrem} & \RU{остаток от знакового деления}\EN{remainder of signed division}\DE{Rest einer vorzeichenbehafteten Division}%
\FR{reste de la division signée} \\
\hline \TT{\_\_allshl} & \RU{сдвиг влево}\EN{shift left}\DE{Schiebe links}\FR{décalage à gauche} \\
\hline \TT{\_\_allshr} & \RU{знаковый сдвиг вправо}\EN{signed shift right}\DE{Schiebe links, vorzeichenbehaftet}%
\FR{décalage signé à droite} \\
\hline \TT{\_\_aulldiv} & \RU{беззнаковое деление}\EN{unsigned division}\DE{vorzeichenlose Division}%
\FR{division non signée} \\
\hline \TT{\_\_aullrem} & \RU{остаток от беззнакового деления}\EN{remainder of unsigned division}\DE{Rest (Modulo) einer vorzeichenlosen Division}%
\FR{reste de la division non signée} \\
\hline \TT{\_\_aullshr} & \RU{беззнаковый сдвиг вправо}\EN{unsigned shift right}\DE{Schiebe rechts, vorzeichenlos}%
\FR{décalage non signé à droite} \\
\hline
\end{tabular}
\end{center}

\RU{Процедуры умножения и сдвига влево, одни и те же и для знаковых чисел, и для беззнаковых,
поэтому здесь только одна функция для каждой операции}
\EN{Multiplication and shift left procedures are the same for both signed and unsigned numbers, hence there is only one function 
for each operation here}
\DE{Multiplikation und Links-Schiebebefehle sind sowohl für vorzeichenbehaftete als auch vorzeichenlose Zahlen,
da hier für jede Operation nur ein Befehl existiert}
\FR{La multiplication et le décalage à gauche sont similaire pour les nombres signés
et non signés, donc il n'y a qu'une seule fonction ici}. \\
\\
\RU{Исходные коды этих функций можно найти в установленной \ac{MSVS}, в}\EN{The source code of these function
can be found in the installed \ac{MSVS}, in}%
\DE{Der Quellcode dieser Funktionen kann im Pfad des installierten \ac{MSVS}, gefunden werden: }%
\FR{Le code source des ces fonctions peut être trouvé dans l'installation de \ac{MSVS},
dans} \TT{VC/crt/src/intel/*.asm}.


\mysection{Cheatsheets}

% sections
\subsection{IDA}
\myindex{IDA}
\label{sec:IDA_cheatsheet}

\ShortHotKeyCheatsheet:

\begin{center}
\begin{tabular}{ | l | l | }
\hline
\HeaderColor \RU{клавиша}\EN{key}\DE{Taste}\FR{touche} & \HeaderColor \RU{значение}\EN{meaning}\DE{Bedeutung}\FR{signification} \\
\hline
Space 	& \RU{переключать между листингом и просмотром кода в виде графа}
            \EN{switch listing and graph view}
            \DE{Zwischen Quellcode und grafischer Ansicht wechseln}%
				\FR{échanger le listing et le mode graphique} \\
C 	& \RU{конвертировать в код}\EN{convert to code}\DE{zu Code konvertieren}%
		\FR{convertir en code} \\
D 	& \RU{конвертировать в данные}\EN{convert to data}\DE{zu Daten konvertieren}%
		\FR{convertir en données} \\
A 	& \RU{конвертировать в строку}\EN{convert to string}\DE{zu Zeichenkette konvertieren}%
		\FR{convertir en chaîne} \\
* 	& \RU{конвертировать в массив}\EN{convert to array}\DE{zu Array konvertieren}%
		\FR{convertir en tableau} \\
U 	& \RU{сделать неопределенным}\EN{undefine}\DE{undefinieren}%
		\FR{rendre indéfini}\\
O 	& \RU{сделать смещение из операнда}\EN{make offset of operand}\DE{Offset von Operanden}%
		\FR{donner l'offset d'une opérande}\\
H 	& \RU{сделать десятичное число}\EN{make decimal number}\DE{Dezimalzahl erstellen}%
		\FR{transformer en nombre décimal} \\
R 	& \RU{сделать символ}\EN{make char}\DE{Zeichen erstellen}%
		\FR{transformer en caractère} \\
B 	& \RU{сделать двоичное число}\EN{make binary number}\DE{Binärzahl erstellen}%
		\FR{transformer en nombre binaire} \\
Q 	& \RU{сделать шестнадцатеричное число}\EN{make hexadecimal number}\DE{Hexadezimalzahl erstellen}%
		\FR{transformer en nombre hexa-décimal} \\
N 	& \RU{переименовать идентификатор}\EN{rename identifier}\DE{Identifikator umbenennen}%
		\FR{renommer l'identifiant} \\
? 	& \RU{калькулятор}\EN{calculator}\DE{Rechner}\FR{calculatrice} \\
G 	& \RU{переход на адрес}\EN{jump to address}\DE{zu Adresse springen}%
		\FR{sauter à l'adresse} \\
: 	& \RU{добавить комментарий}\EN{add comment}\DE{Kommentar einfügen}\FR{ajouter un commentaire} \\
Ctrl-X 	& \RU{показать ссылки на текущую функцию, метку, переменную}%
		\EN{show references to the current function, label, variable }%
		\DE{Referenz zu aktueller Funktion, Variable, ... zeigen}%
		\FR{montrer les références à la fonction, au label, à la variable courant} \\
	& \RU{(в т.ч., в стеке)}\EN{(incl. in local stack)}\DE{(inkl. lokalem Stack)}%
		\FR{inclure dans la pile locale} \\
X 	& \RU{показать ссылки на функцию, метку, переменную, итд}\EN{show references to the function, label, variable, etc.}%
		\DE{Referenz zu Funktion, Variable, ... zeigen}%
		\FR{montrer les références à la fonction, au label, à la variable, etc.} \\
Alt-I 	& \RU{искать константу}\EN{search for constant}\DE{Konstante suchen}%
		\FR{chercher une constante} \\
Ctrl-I 	& \RU{искать следующее вхождение константы}\EN{search for the next occurrence of constant}\DE{Nächstes Auftreten der Konstante suchen}%
		\FR{chercher la prochaine occurrence d'une constante} \\
Alt-B 	& \RU{искать последовательность байт}\EN{search for byte sequence}\DE{Byte-Sequenz suchen}%
		\FR{chercher une séquence d'octets} \\
Ctrl-B 	& \RU{искать следующее вхождение последовательности байт}
		\EN{search for the next occurrence of byte sequence}
		\DE{Nächstes Auftreten der Byte-Sequenz suchen}%
		\FR{chercher l'occurrence suivante d'une séquence d'octets} \\
Alt-T 	& \RU{искать текст (включая инструкции, итд.)}%
		\EN{search for text (including instructions, etc.)}%
		\DE{Text suchen (inkl. Anweisungen, usw.)}%
		\FR{chercher du texte (instructions incluses, etc.)} \\
Ctrl-T 	& \RU{искать следующее вхождение текста}%
		\EN{search for the next occurrence of text}%
		\DE{nächstes Aufreten des Textes suchen}%
		\FR{chercher l'occurrence suivante du texte} \\
Alt-P 	& \RU{редактировать текущую функцию}%
		\EN{edit current function}%
		\DE{akutelle Funktion editieren}%
		\FR{éditer la fonction courante} \\
Enter 	& \RU{перейти к функции, переменной, итд.}%
		\EN{jump to function, variable, etc.}%
		\DE{zu Funktion, Variable, ... springen}%
		\FR{sauter à la fonction, la variable, etc.} \\
Esc 	& \RU{вернуться назад}\EN{get back}\DE{zurückgehen}%
		\FR{retourner en arrière} \\
Num -   & \RU{свернуть функцию или отмеченную область}%
		\EN{fold function or selected area}%
		\DE{Funktion oder markierten Bereich einklappen}%
		\FR{cacher/plier la fonction ou la partie sélectionnée} \\
Num + 	& \RU{снова показать функцию или область}%
		\EN{unhide function or area}%
		\DE{Funktion oder Bereich anzeigen}%
		\FR{afficher la fonction ou une partie} \\
\hline
\end{tabular}
\end{center}

\RU{Сворачивание функции или области может быть удобно чтобы прятать те части функции,
чья функция вам стала уже ясна}%
\EN{Function/area folding may be useful for hiding function parts when you realize what they do}%
\DE{Das Einklappen ist nützlich um Teile von Funktionen zu verstecken, wenn bekannt ist was sie tun}%
\FR{cacher une fonction ou une partie de code peut être utile pour cacher des parties du
code lorsque vous avez compris ce qu'elles font}.
\RU{это используется в моем скрипте\footnote{\href{\YurichevIDAIDCScripts}{GitHub}}}\EN{this is used in my}\DE{dies wird genutzt im}%
\RU{для сворачивания некоторых очень часто используемых фрагментов inline-кода}%
\EN{script\footnote{\href{\YurichevIDAIDCScripts}{GitHub}} for hiding some often used patterns of inline code}%
\DE{Script\footnote{\href{\YurichevIDAIDCScripts}{GitHub}} um häufig genutzte Inline-Code-Stellen zu verstecken}%
\FR{ceci est utilisé dans mon script\footnote{\href{\YurichevIDAIDCScripts}{GitHub}}%
pour cacher des patterns de code inline souvent utilisés}.


\subsection{\olly}
\myindex{\olly}
\label{sec:Olly_cheatsheet}

\ShortHotKeyCheatsheet:

\begin{center}
\begin{tabular}{ | l | l | }
\hline
\HeaderColor \RU{хот-кей}\EN{hot-key}\DE{Tastenkürzel}\FR{raccourci} & 
\HeaderColor \RU{значение}\EN{meaning}\DE{Bedeutung}\FR{signification} \\
\hline
F7	& \RU{трассировать внутрь}\EN{trace into}\DE{Schritt}\FR{tracer dans la fonction}\\
F8	& \stepover\\
F9	& \RU{запуск}\EN{run}\DE{starten}\FR{démarrer}\\
Ctrl-F2	& \RU{перезапуск}\EN{restart}\DE{Neustart}\FR{redémarrer}\\
\hline
\end{tabular}
\end{center}

\subsection{MSVC}
\myindex{MSVC}
\label{sec:MSVC_options}

\RU{Некоторые полезные опции, которые были использованы в книге}
\EN{Some useful options which were used through this book}.
\DE{Einige nützliche Optionen die in diesem Buch genutzt werden}.
\FR{Quelques options utiles qui ont été utilisées dans ce livre}

\begin{center}
\begin{tabular}{ | l | l | }
\hline
\HeaderColor \RU{опция}\EN{option}\DE{Option}\FR{option} & 
\HeaderColor \RU{значение}\EN{meaning}\DE{Bedeutung}\FR{signification} \\
\hline
/O1		& \RU{оптимизация по размеру кода}\EN{minimize space}\DE{Speicherplatz minimieren}%
\FR{minimiser l'espace}\\
/Ob0		& \RU{не заменять вызовы inline-функций их кодом}\EN{no inline expansion}\DE{Keine Inline-Erweiterung}%
\FR{pas de mire en ligne}\\
/Ox		& \RU{максимальная оптимизация}\EN{maximum optimizations}\DE{maximale Optimierung}%
\FR{optimisation maximale}\\
/GS-		& \RU{отключить проверки переполнений буфера}
		\EN{disable security checks (buffer overflows)}
        \DE{Sicherheitsüberprüfungen deaktivieren (Buffer Overflows)}%
		\FR{désactiver les vérifications de sécurité (buffer overflows)}\\
/Fa(file)	& \RU{генерировать листинг на ассемблере}\EN{generate assembly listing}\DE{Assembler-Quelltext erstellen}%
\FR{générer un listing assembleur}\\
/Zi		& \RU{генерировать отладочную информацию}\EN{enable debugging information}\DE{Debugging-Informationen erstellen}%
\FR{activer les informations de débogage}\\
/Zp(n)		& \RU{паковать структуры по границе в $n$ байт}\EN{pack structs on $n$-byte boundary}\DE{Strukturen an $n$-Byte-Grenze ausrichten}%
\FR{aligner les structures sur une limite de $n$-octet}\\
/MD		& \RU{выходной исполняемый файл будет использовать}
			\EN{produced executable will use}%
            \DE{ausführbare Daten nutzt}%
\FR{l'exécutable généré utilisera} \TT{MSVCR*.DLL}\\
\hline
\end{tabular}
\end{center}

\RU{Кое-как информация о версиях MSVC}\EN{Some information about MSVC versions}\DE{Informationen zu MSVC-Versionen}%
\FR{Quelques informations sur les versions de MSVC}:
\myref{MSVC_versions}.


\EN{\subsection{GCC}
\myindex{GCC}

Some useful options which were used through this book.

\begin{center}
\begin{tabular}{ | l | l | }
\hline
\HeaderColor option & 
\HeaderColor meaning \\
\hline
-Os		& code size optimization \\
-O3		& maximum optimization \\
-regparm=	& how many arguments are to be passed in registers \\
-o file		& set name of output file \\
-g		& produce debugging information in resulting executable \\
-S		& generate assembly listing file \\
-masm=intel	& produce listing in Intel syntax \\
-fno-inline	& do not inline functions \\
\hline
\end{tabular}
\end{center}


}
\RU{\myparagraph{GCC 4.4.1}

\lstinputlisting[caption=GCC 4.4.1,style=customasmx86]{patterns/12_FPU/3_comparison/x86/GCC_RU.asm}

\myindex{x86!\Instructions!FUCOMPP}
\FUCOMPP~--- это почти то же что и \FCOM, только выкидывает из стека оба значения после сравнения, 
а также несколько иначе реагирует на \q{не-числа}.

\myindex{Не-числа (NaNs)}
Немного о \emph{не-числах}.

FPU умеет работать со специальными переменными, которые числами не являются и называются \q{не числа} или 
\gls{NaN}.
Это бесконечность, результат деления на ноль, и так далее. Нечисла бывают \q{тихие} и \q{сигнализирующие}. 
С первыми можно продолжать работать и далее, а вот если вы попытаетесь совершить какую-то операцию 
с сигнализирующим нечислом, то сработает исключение.

\myindex{x86!\Instructions!FCOM}
\myindex{x86!\Instructions!FUCOM}
Так вот, \FCOM вызовет исключение если любой из операндов какое-либо нечисло.
\FUCOM же вызовет исключение только если один из операндов именно \q{сигнализирующее нечисло}.

\myindex{x86!\Instructions!SAHF}
\label{SAHF}
Далее мы видим \SAHF (\emph{Store AH into Flags})~--- это довольно редкая инструкция в коде, не использующим FPU. 
8 бит из \AH перекладываются в младшие 8 бит регистра статуса процессора в таком порядке:

\input{SAHF_LAHF}

\myindex{x86!\Instructions!FNSTSW}
Вспомним, что \FNSTSW перегружает интересующие нас биты \CThreeBits в \AH, 
и соответственно они будут в позициях 6, 2, 0 в регистре \AH:

\input{C3_in_AH}

Иными словами, пара инструкций \INS{fnstsw  ax / sahf} перекладывает биты \CThreeBits в флаги \ZF, \PF, \CF.

Теперь снова вспомним, какие значения бит \CThreeBits будут при каких результатах сравнения:

\begin{itemize}
\item Если $a$ больше $b$ в нашем случае, то биты \CThreeBits должны быть выставлены так: 0, 0, 0.
\item Если $a$ меньше $b$, то биты будут выставлены так: 0, 0, 1.
\item Если $a=b$, то так: 1, 0, 0.
\end{itemize}
% TODO: table?

Иными словами, после трех инструкций \FUCOMPP/\FNSTSW/\SAHF возможны такие состояния флагов:

\begin{itemize}
\item Если $a>b$ в нашем случае, то флаги будут выставлены так: \GTT{ZF=0, PF=0, CF=0}.
\item Если $a<b$, то флаги будут выставлены так: \GTT{ZF=0, PF=0, CF=1}.
\item Если $a=b$, то так: \GTT{ZF=1, PF=0, CF=0}.
\end{itemize}
% TODO: table?

\myindex{x86!\Instructions!SETcc}
\myindex{x86!\Instructions!JNBE}
Инструкция \SETNBE выставит в \AL единицу или ноль в зависимости от флагов и условий. 
Это почти аналог \JNBE, за тем лишь исключением, что \SETcc
\footnote{\emph{cc} это \emph{condition code}}
выставляет 1 или 0 в \AL, а \Jcc делает переход или нет. 
\SETNBE запишет 1 только если \GTT{CF=0} и \GTT{ZF=0}. Если это не так, то запишет 0 в \AL.

\CF будет 0 и \ZF будет 0 одновременно только в одном случае: если $a>b$.

Тогда в \AL будет записана 1, последующий условный переход \JZ выполнен не будет 
и функция вернет~\GTT{\_a}. 
В остальных случаях, функция вернет~\GTT{\_b}.
}
\FR{\subsection{GCC}
\myindex{GCC}

Quelques options utiles qui ont été utilisées dans ce livre.

\begin{center}
\begin{tabular}{ | l | l | }
\hline
\HeaderColor option & 
\HeaderColor signification \\
\hline
-Os		& optimiser la taille du code \\
-O3		& optimisation maximale \\
-regparm=	& nombre d'arguments devant être passés dans les registres \\
-o file		& définir le nom du fichier de sortie \\
-g		& mettre l'information de débogage dans l'exécutable généré \\
-S		& générer un fichier assembleur \\
-masm=intel	& construire le code source en syntaxe Intel \\
-fno-inline	& ne pas mettre les fonctions en ligne \\
\hline
\end{tabular}
\end{center}


}
\DE{\myparagraph{GCC 4.4.1}

\lstinputlisting[caption=GCC 4.4.1,style=customasmx86]{patterns/12_FPU/3_comparison/x86/GCC_DE.asm}

\myindex{x86!\Instructions!FUCOMPP}
\FUCOMPP{} ist fast wie like \FCOM, nimmt aber beide Werte vom Stand und
behandelt \q{undefinierte Zahlenwerte} anders.


\myindex{Non-a-numbers (NaNs)}
Ein wenig über \emph{undefinierte Zahlenwerte}.

Die FPU ist in der Lage mit speziellen undefinieten Werten, den sogenannten
\emph{not-a-number}(kurz \gls{NaN}) umzugehen. Beispiele sind etwa der Wert
unendlich, das Ergebnis einer Division durch 0, etc. Undefinierte Werte können
entwder \q{quiet} oder \q{signaling} sein. Es ist möglich mit \q{quiet} NaNs zu
arbeiten, aber beim Versuch einen Befehl auf \q{signaling} NaNs auszuführen,
wird eine Exception geworfen. 

\myindex{x86!\Instructions!FCOM}
\myindex{x86!\Instructions!FUCOM}
\FCOM erzeugt eine Exception, falls irgendein Operand ein \gls{NaN} ist.
\FUCOM erzeugt eine Exception nur dann, wenn ein Operand eine \q{signaling}
\gls{NaN} (SNaN) ist.

\myindex{x86!\Instructions!SAHF}
\label{SAHF}
Der nächste Befehl ist \SAHF (\emph{Store AH into Flags})~---es handelt sich
hierbei um einen seltenen Befehl, der nicht mit der FPU zusammenhängt.
8 Bits aus AH werden in die niederen 8 Bit der CPU Flags in der folgenden
Reihenfolge verschoben:

\input{SAHF_LAHF}

\myindex{x86!\Instructions!FNSTSW}
Erinnern wir uns, dass \FNSTSW die für uns interessanten Bits (\CThreeBits) auf
den Stellen 6,2,0 im AH Register setzt:

\input{C3_in_AH}
Mit anderen Worten: der Befehl \INS{fnstsw ax / sahf} verschiebt \CThreeBits
nach \ZF, \PF und \CF. 

Überlegen wir uns auch die Werte der \CThreeBits in unterschiedlichen Szenarien:

\begin{itemize} 
  \item Falls in unserem Beispiel $a$ größer als $b$ ist, dann werden die
  \CThreeBits auf 0,0,0 gesetzt.
  \item Falls $a$ kleiner als $b$ ist, werden die Bits auf 0,0,1 gesetzt.
  \item Falls $a=b$, dann werden die Bits auf 1,0,0 gesetzt.
\end{itemize}
% TODO: table?
Mit anderen Worten, die folgenden Zustände der CPU Flags sind nach drei
\FUCOMPP/\FNSTSW/\SAHF Befehlen möglich:

\begin{itemize}
\item Falls $a>b$, werden die CPU Flags wie folgt gesetzt \GTT{ZF=0, PF=0,
CF=0}.
\item Falls $a<b$, werden die CPU Flags wie folgt gesetzt: \GTT{ZF=0, PF=0,
CF=1}.
\item Und falls $a=b$, dann gilt: \GTT{ZF=1, PF=0, CF=0}.
\end{itemize}
% TODO: table?

\myindex{x86!\Instructions!SETcc}
\myindex{x86!\Instructions!JNBE}
Abhängig von den CPU Flags und Bedingungen, speichert \SETNBE entweder 1 oder 0
in AL.
Es ist also quasi das Gegenstück von \JNBE mit dem Unterschied, dass \SETcc

Depending on the CPU flags and conditions, \SETNBE stores 1 or 0 to AL. 
It is almost the counterpart of \JNBE, with the exception that \SETcc
\footnote{\emph{cc} is \emph{condition code}} eine 1 oder 0 in \AL speichert, aber
\Jcc tatsächlich auch springt.
\SETNBE speicher 1 nur, falls \GTT{CF=0} und \GTT{ZF=0}.
Wenn dies nicht der Fall ist, dann wird 0 in \AL gespeichert.

Nur in einem Fall sind \CF und \ZF beide 0: falls $a>b$.

In diesem Fall wird 1 in \AL gespeichert, der nachfolgende \JZ Sprung wird nicht
ausgeführt und die Funktion liefert {\_a} zurück. In allen anderen Fällen wird
{\_b} zurückgegeben.
}

\subsection{GDB}
\myindex{GDB}
\label{sec:GDB_cheatsheet}

% FIXME: in Russian table doesn't fit!

\RU{Некоторые команды, которые были использованы в книге}\EN{Some of commands we used in this book}\DE{Einige nützliche Optionen die in diesem Buch genutzt werden}%
\FR{Quelques commandes que nous avons utilisées dans ce livre}:

\small
\begin{center}
\begin{tabular}{ | l | l | }
\hline
\HeaderColor \RU{опция}\EN{option}\DE{Option}\FR{option} & 
\HeaderColor \RU{значение}\EN{meaning}\DE{Bedeutung} \\
\hline
break filename.c:number		& \RU{установить точку останова на номере строки в исходном файле}
					\EN{set a breakpoint on line number in source code}
                    \DE{Setzen eines Breakpoints in der angegebenen Zeile}%
					\FR{mettre un point d'arrêt à la ligne number du code source} \\
break function			& \RU{установить точку останова на функции}\EN{set a breakpoint on function}\DE{Setzen eines Breakpoints in der Funktion}%
\FR{mettre un point d'arrêt sur une fonction} \\
break *address			& \RU{установить точку останова на адресе}\EN{set a breakpoint on address}\DE{Setzen eines Breakpoints auf Adresse}%
\FR{mettre un point d'arrêt à une adresse} \\
b				& \dittoclosing \\
p variable			& \RU{вывести значение переменной}\EN{print value of variable}\DE{Ausgabe eines Variablenwerts}%
\FR{afficher le contenu d'une variable} \\
run				& \RU{запустить}\EN{run}\DE{Starten}\FR{démarrer} \\
r				& \dittoclosing \\
cont				& \RU{продолжить исполнение}\EN{continue execution}\DE{Ausführung fortfahren}\FR{continuer l'exécution} \\
c				& \dittoclosing \\
bt				& \RU{вывести стек}\EN{print stack}\DE{Stack ausgeben}\FR{afficher la pile} \\
set disassembly-flavor intel	& \RU{установить Intel-синтаксис}\EN{set Intel syntax}\DE{Intel-Syntax nutzen}%
\FR{utiliser la syntaxe Intel} \\
disas				& disassemble current function \\
disas function			& \RU{дизассемблировать функцию}\EN{disassemble function}\DE{Funktion disassemblieren}\FR{désassembler la fonction} \\
disas function,+50		& disassemble portion \\
disas \$eip,+0x10		& \dittoclosing \\
disas/r				& \EN{disassemble with opcodes}\RU{дизассемблировать с опкодами}\DE{mit OpCodes disassemblieren}%
\FR{désassembler avec les opcodes} \\
info registers			& \RU{вывести все регистры}\EN{print all registers}\DE{Ausgabe aller Register}\FR{afficher tous les registres} \\
info float			& \RU{вывести FPU-регистры}\EN{print FPU-registers}\DE{Ausgabe der FPU-Register}\FR{afficher les registres FPU} \\
info locals			& \RU{вывести локальные переменные (если известны)}\EN{dump local variables (if known)}\DE{(bekannte) lokale Variablen ausgeben}%
\FR{afficher les variables locales} \\
x/w ...				& \RU{вывести память как 32-битные слова}\EN{dump memory as 32-bit word}\DE{Speicher als 32-Bit-Wort ausgeben}%
\FR{afficher la mémoire en mot de 32-bit} \\
x/w \$rdi			& \RU{вывести память как 32-битные слова}\EN{dump memory as 32-bit word}\DE{Speicher als 32-Bit-Wort ausgeben}%
\FR{afficher la mémoire en mot de 32-bit} \\
				& \RU{по адресу в \TT{RDI}}\EN{at address in \TT{RDI}}\DE{an Adresse in \TT{RDI}}\FR{à l'adresse dans \TT{RDI}} \\

x/10w ...			& \RU{вывести 10 слов памяти}\EN{dump 10 memory words}\DE{10 Speicherworte ausgeben}%
\FR{afficher 10 mots de la mémoire} \\
x/s ...				& \RU{вывести строку из памяти}\EN{dump memory as string}\DE{Speicher als Zeichenkette ausgeben}%
\FR{afficher la mémoire en tant que chaîne} \\
x/i ...				& \RU{трактовать память как код}\EN{dump memory as code}\DE{Speicher als Code ausgeben}%
\FR{afficher la mémoire en tant que code} \\
x/10c ...			& \RU{вывести 10 символов}\EN{dump 10 characters}\DE{10 Zeichen ausgeben}%
\FR{afficher 10 caractères} \\
x/b ...				& \RU{вывести байты}\EN{dump bytes}\DE{Bytes ausgeben}\FR{afficher des octets} \\
x/h ...				& \RU{вывести 16-битные полуслова}\EN{dump 16-bit halfwords}\DE{16-Bit-Halbworte ausgeben}%
\FR{afficher en demi-mots de 16-bit} \\
x/g ...				& \RU{вывести 64-битные слова}\EN{dump giant (64-bit) words}\DE{große (64-Bit-) Worte ausgeben}%
\FR{afficher des mots géants (64-bit)} \\
finish				& \RU{исполнять до конца функции}\EN{execute till the end of function}\DE{bis Funktionsende fortfahren}%
\FR{exécuter jusqu'à la fin de la fonction} \\
next				& \RU{следующая инструкция (не заходить в функции)}
					\EN{next instruction (don't dive into functions)}
					\DE{Nächste Anweisung (nicht in Funktion springen)}
					\FR{instruction suivante (ne pas descendre dans les fonctions)} \\
step				& \RU{следующая инструкция (заходить в функции)}
					\EN{next instruction (dive into functions)}
					\DE{Nächste Anweisung (in Funktion springen)}
					\FR{instruction suivante (descendre dans les fonctions)} \\
set step-mode on		& \RU{не использовать информацию о номерах строк при использовании команды step}
					\EN{do not use line number information while stepping}
					\DE{Beim schrittweisen Ausführen keine Zeilennummerninfos nutzen}
					\FR{ne pas utiliser l'information du numéro de ligne en exécutant pas à pas} \\
frame n				& \RU{переключить фрейм стека}\EN{switch stack frame}\DE{Stack-Frame tauschen}\FR{échanger la stack frame} \\
info break			& \RU{список точек останова}\EN{list of breakpoints}\DE{Breakpoints schauen}%
\FR{afficher les points d'arrêt} \\
del n				& \RU{удалить точку останова}\EN{delete breakpoint}\DE{Breakpoints löschen}\FR{effacer un point d'arrêt} \\
set args ...			& \RU{установить аргументы командной строки}\EN{set command-line arguments}\DE{Aufrufparameter setzen}%
\FR{définir les arguments de la ligne de commande} \\
\hline
\end{tabular}
\end{center}
\normalsize



}
\FR{\part*{\RU{Приложение}\EN{Appendix}\DE{Anhang}\FR{Appendice}\IT{Appendice}}
\appendix
\addcontentsline{toc}{part}{\RU{Приложение}\EN{Appendix}\DE{Anhang}\FR{Appendice}\IT{Appendice}}

% chapters
\EN{\mysection{Task manager practical joke (Windows Vista)}
\myindex{Windows!Windows Vista}

Let's see if it's possible to hack Task Manager slightly so it would detect more \ac{CPU} cores.

\myindex{Windows!NTAPI}

Let us first think, how does the Task Manager know the number of cores?

There is the \TT{GetSystemInfo()} win32 function present in win32 userspace which can tell us this.
But it's not imported in \TT{taskmgr.exe}.

There is, however, another one in \gls{NTAPI}, \TT{NtQuerySystemInformation()}, 
which is used in \TT{taskmgr.exe} in several places.

To get the number of cores, one has to call this function with the \TT{SystemBasicInformation} constant
as a first argument (which is zero
\footnote{\href{http://msdn.microsoft.com/en-us/library/windows/desktop/ms724509(v=vs.85).aspx}{MSDN}}).

The second argument has to point to the buffer which is getting all the information.

So we have to find all calls to the \\
\TT{NtQuerySystemInformation(0, ?, ?, ?)} function.
Let's open \TT{taskmgr.exe} in IDA. 
\myindex{Windows!PDB}

What is always good about Microsoft executables is that IDA can download the corresponding \gls{PDB} 
file for this executable and show all function names.

It is visible that Task Manager is written in \Cpp and some of the function names and classes are really 
speaking for themselves.
There are classes CAdapter, CNetPage, CPerfPage, CProcInfo, CProcPage, CSvcPage, 
CTaskPage, CUserPage.

Apparently, each class corresponds to each tab in Task Manager.

Let's visit each call and add comment with the value which is passed as the first function argument.
We will write \q{not zero} at some places, because the value there was clearly not zero, 
but something really different (more about this in the second part of this chapter).

And we are looking for zero passed as argument, after all.

\begin{figure}[H]
\centering
\myincludegraphics{examples/taskmgr/IDA_xrefs.png}
\caption{IDA: cross references to NtQuerySystemInformation()}
\end{figure}

Yes, the names are really speaking for themselves.

When we closely investigate each place where\\
\TT{NtQuerySystemInformation(0, ?, ?, ?)} is called,
we quickly find what we need in the \TT{InitPerfInfo()} function:

\lstinputlisting[caption=taskmgr.exe (Windows Vista),style=customasmx86]{examples/taskmgr/taskmgr.lst}

\TT{g\_cProcessors} is a global variable, and this name has been assigned by 
IDA according to the \gls{PDB} loaded from Microsoft's symbol server.

The byte is taken from \TT{var\_C20}. 
And \TT{var\_C58} is passed to\\
\TT{NtQuerySystemInformation()} 
as a pointer to the receiving buffer.
The difference between 0xC20 and 0xC58 is 0x38 (56).

Let's take a look at format of the return structure, which we can find in MSDN:

\begin{lstlisting}[style=customc]
typedef struct _SYSTEM_BASIC_INFORMATION {
    BYTE Reserved1[24];
    PVOID Reserved2[4];
    CCHAR NumberOfProcessors;
} SYSTEM_BASIC_INFORMATION;
\end{lstlisting}

This is a x64 system, so each PVOID takes 8 bytes.

All \emph{reserved} fields in the structure take $24+4*8=56$ bytes.

Oh yes, this implies that \TT{var\_C20} is the local stack is exactly the
\TT{NumberOfProcessors} field of the \TT{SYSTEM\_BASIC\_INFORMATION} structure.

Let's check our guess.
Copy \TT{taskmgr.exe} from \TT{C:\textbackslash{}Windows\textbackslash{}System32} 
to some other folder 
(so the \emph{Windows Resource Protection} 
will not try to restore the patched \TT{taskmgr.exe}).

Let's open it in Hiew and find the place:

\begin{figure}[H]
\centering
\myincludegraphics{examples/taskmgr/hiew2.png}
\caption{Hiew: find the place to be patched}
\end{figure}

Let's replace the \TT{MOVZX} instruction with ours.
Let's pretend we've got 64 CPU cores.

Add one additional \ac{NOP} (because our instruction is shorter than the original one):

\begin{figure}[H]
\centering
\myincludegraphics{examples/taskmgr/hiew1.png}
\caption{Hiew: patch it}
\end{figure}

And it works!
Of course, the data in the graphs is not correct.

At times, Task Manager even shows an overall CPU load of more than 100\%.

\begin{figure}[H]
\centering
\myincludegraphics{examples/taskmgr/taskmgr_64cpu_crop.png}
\caption{Fooled Windows Task Manager}
\end{figure}

The biggest number Task Manager does not crash with is 64.

Apparently, Task Manager in Windows Vista was not tested on computers with a large number of cores.

So there are probably some static data structure(s) inside it limited to 64 cores.

\subsection{Using LEA to load values}
\label{TaskMgr_LEA}

Sometimes, \TT{LEA} is used in \TT{taskmgr.exe} instead of \TT{MOV} to set the first argument of \\
\TT{NtQuerySystemInformation()}:

\lstinputlisting[caption=taskmgr.exe (Windows Vista),style=customasmx86]{examples/taskmgr/taskmgr2.lst}

\myindex{x86!\Instructions!LEA}

Perhaps \ac{MSVC} did so because machine code of \INS{LEA} is shorter than \INS{MOV REG, 5} (would be 5 instead of 4).

\INS{LEA} with offset in $-128..127$ range (offset will occupy 1 byte in opcode) with 32-bit registers is even shorter (for lack of REX prefix)---3 bytes.

Another example of such thing is: \myref{using_MOV_and_pack_of_LEA_to_load_values}.
}
\RU{\subsection{Обменять входные значения друг с другом}

Вот так:

\lstinputlisting[style=customc]{patterns/061_pointers/swap/5_RU.c}

Как видим, байты загружаются в младшие 8-битные части регистров \TT{ECX} и \TT{EBX} используя \INS{MOVZX}
(так что старшие части регистров очищаются), затем байты записываются назад в другом порядке.

\lstinputlisting[style=customasmx86,caption=Optimizing GCC 5.4]{patterns/061_pointers/swap/5_GCC_O3_x86.s}

Адреса обоих байтов берутся из аргументов и во время исполнения ф-ции находятся в регистрах \TT{EDX} и \TT{EAX}.

Так что исопльзуем указатели --- вероятно, без них нет способа решить эту задачу лучше.

}
\DE{\mysection{x86}

\subsection{Terminologie}

Geläufig für 16-Bit (8086/80286), 32-Bit (80386, etc.), 64-Bit.

\myindex{IEEE 754}
\myindex{MS-DOS}
\begin{description}
	\item[Byte] 8-Bit.
		Die DB Assembler-Direktive wird zum Definieren von Variablen und Arrays genutzt.
		Bytes werden in dem 8-Bit-Teil der folgenden Register übergeben:
		\TT{AL/BL/CL/DL/AH/BH/CH/DH/SIL/DIL/R*L}.
	\item[Wort] 16-Bit.
		DW Assembler-Direktive \dittoclosing.
		Bytes werden in dem 16-Bit-Teil der folgenden Register übergeben:
			\TT{AX/BX/CX/DX/SI/DI/R*W}.
	\item[Doppelwort] (\q{dword}) 32-Bit.
		DD Assembler-Direktive \dittoclosing.
		Doppelwörter werden in Registern (x86) oder dem 32-Bit-Teil der Register (x64) übergeben.
		In 16-Bit-Code werden Doppelwörter in 16-Bit-Registerpaaren übergeben.
	\item[zwei Doppelwörter] (\q{qword}) 64-Bit.
		DQ Assembler-Direktive \dittoclosing.
		In 32-Bit-Umgebungen werden diese in 32-Bit-Registerpaaren übergeben.
	\item[tbyte] (10 Byte) 80-Bit oder 10 Bytes (für IEEE 754 FPU Register).
	\item[paragraph] (16 Byte) --- Bezeichnung war in MS-DOS Umgebungen gebräuchlich.
\end{description}

\myindex{Windows!API}

Datentypen der selben Breite (BYTE, WORD, DWORD) entsprechen auch denen in der Windows \ac{API}.

% TODO German Translation (DSiekmeier)
%\input{appendix/x86/registers} % subsection
%\input{appendix/x86/instructions} % subsection
\subsection{npad}
\label{sec:npad}

\RU{Это макрос в ассемблере, для выравнивания некоторой метки по некоторой границе.}
\EN{It is an assembly language macro for aligning labels on a specific boundary.}
\DE{Dies ist ein Assembler-Makro um Labels an bestimmten Grenzen auszurichten.}
\FR{C'est une macro du langage d'assemblage pour aligner les labels sur une limite
spécifique.}

\RU{Это нужно для тех \emph{нагруженных} меток, куда чаще всего передается управление, например, 
начало тела цикла. 
Для того чтобы процессор мог эффективнее вытягивать данные или код из памяти, через шину с памятью, 
кэширование, итд.}
\EN{That's often needed for the busy labels to where the control flow is often passed, e.g., loop body starts.
So the CPU can load the data or code from the memory effectively, through the memory bus, cache lines, etc.}
\DE{Dies ist oft nützlich Labels, die oft Ziel einer Kotrollstruktur sind, wie Schleifenköpfe.
Somit kann die CPU Daten oder Code sehr effizient vom Speicher durch den Bus, den Cache, usw. laden.}
\FR{C'est souvent nécessaire pour des labels très utilisés, comme par exemple le
début d'un corps de boucle. Ainsi, le CPU peut charger les données ou le code depuis
la mémoire efficacement, à travers le bus mémoire, les caches, etc.}

\RU{Взято из}\EN{Taken from}\DE{Entnommen von}\FR{Pris de} \TT{listing.inc} (MSVC):

\myindex{x86!\Instructions!NOP}
\RU{Это, кстати, любопытный пример различных вариантов \NOP{}-ов. 
Все эти инструкции не дают никакого эффекта, но отличаются разной длиной.}
\EN{By the way, it is a curious example of the different \NOP variations.
All these instructions have no effects whatsoever, but have a different size.}
\DE{Dies ist übrigens ein Beispiel für die unterschiedlichen \NOP-Variationen.
Keine dieser Anweisungen hat eine Auswirkung, aber alle haben eine unterschiedliche Größe.}
\FR{À propos, c'est un exemple curieux des différentes variations de \NOP. Toutes
ces instructions n'ont pas d'effet, mais ont une taille différente.}

\RU{Цель в том, чтобы была только одна инструкция, а не набор NOP-ов, 
считается что так лучше для производительности CPU.}
\EN{Having a single idle instruction instead of couple of NOP-s,
is accepted to be better for CPU performance.}
\DE{Eine einzelne Idle-Anweisung anstatt mehrerer NOPs hat positive Auswirkungen
auf die CPU-Performance.}
\FR{Avoir une seule instruction sans effet au lieu de plusieurs est accepté comme
étant meilleur pour la performance du CPU.}

\begin{lstlisting}[style=customasmx86]
;; LISTING.INC
;;
;; This file contains assembler macros and is included by the files created
;; with the -FA compiler switch to be assembled by MASM (Microsoft Macro
;; Assembler).
;;
;; Copyright (c) 1993-2003, Microsoft Corporation. All rights reserved.

;; non destructive nops
npad macro size
if size eq 1
  nop
else
 if size eq 2
   mov edi, edi
 else
  if size eq 3
    ; lea ecx, [ecx+00]
    DB 8DH, 49H, 00H
  else
   if size eq 4
     ; lea esp, [esp+00]
     DB 8DH, 64H, 24H, 00H
   else
    if size eq 5
      add eax, DWORD PTR 0
    else
     if size eq 6
       ; lea ebx, [ebx+00000000]
       DB 8DH, 9BH, 00H, 00H, 00H, 00H
     else
      if size eq 7
	; lea esp, [esp+00000000]
	DB 8DH, 0A4H, 24H, 00H, 00H, 00H, 00H 
      else
       if size eq 8
        ; jmp .+8; .npad 6
	DB 0EBH, 06H, 8DH, 9BH, 00H, 00H, 00H, 00H
       else
        if size eq 9
         ; jmp .+9; .npad 7
         DB 0EBH, 07H, 8DH, 0A4H, 24H, 00H, 00H, 00H, 00H
        else
         if size eq 10
          ; jmp .+A; .npad 7; .npad 1
          DB 0EBH, 08H, 8DH, 0A4H, 24H, 00H, 00H, 00H, 00H, 90H
         else
          if size eq 11
           ; jmp .+B; .npad 7; .npad 2
           DB 0EBH, 09H, 8DH, 0A4H, 24H, 00H, 00H, 00H, 00H, 8BH, 0FFH
          else
           if size eq 12
            ; jmp .+C; .npad 7; .npad 3
            DB 0EBH, 0AH, 8DH, 0A4H, 24H, 00H, 00H, 00H, 00H, 8DH, 49H, 00H
           else
            if size eq 13
             ; jmp .+D; .npad 7; .npad 4
             DB 0EBH, 0BH, 8DH, 0A4H, 24H, 00H, 00H, 00H, 00H, 8DH, 64H, 24H, 00H
            else
             if size eq 14
              ; jmp .+E; .npad 7; .npad 5
              DB 0EBH, 0CH, 8DH, 0A4H, 24H, 00H, 00H, 00H, 00H, 05H, 00H, 00H, 00H, 00H
             else
              if size eq 15
               ; jmp .+F; .npad 7; .npad 6
               DB 0EBH, 0DH, 8DH, 0A4H, 24H, 00H, 00H, 00H, 00H, 8DH, 9BH, 00H, 00H, 00H, 00H
              else
	       %out error: unsupported npad size
               .err
              endif
             endif
            endif
           endif
          endif
         endif
        endif
       endif
      endif
     endif
    endif
   endif
  endif
 endif
endif
endm
\end{lstlisting}
 % subsection
}
\FR{\subsection{Exemple \#2: SCO OpenServer}

\label{examples_SCO}
\myindex{SCO OpenServer}
Un ancien logiciel pour SCO OpenServer de 1997 développé par une société qui a disparue
depuis longtemps.

Il y a un driver de dongle special à installer dans le système, qui contient les
chaînes de texte suivantes:
\q{Copyright 1989, Rainbow Technologies, Inc., Irvine, CA}
et
\q{Sentinel Integrated Driver Ver. 3.0 }.

Après l'installation du driver dans SCO OpenServer, ces fichiers apparaissent dans
l'arborescence /dev:

\begin{lstlisting}
/dev/rbsl8
/dev/rbsl9
/dev/rbsl10
\end{lstlisting}

Le programme renvoie une erreur lorsque le dongle n'est pas connecté, mais le message
d'erreur n'est pas trouvé dans les exécutables.

\myindex{COFF}

Grâce à \ac{IDA}, il est facile de charger l'exécutable COFF utilisé dans SCO OpenServer.

Essayons de trouver la chaîne \q{rbsl} et en effet, elle se trouve dans ce morceau
de code:

\lstinputlisting[style=customasmx86]{examples/dongles/2/1.lst}

Oui, en effet, le programme doit communiquer d'une façon ou d'une autre avec le driver.

\myindex{thunk-functions}
Le seul endroit où la fonction \TT{SSQC()} est appelée est dans la \glslink{thunk
 function}{fonction thunk}:

\lstinputlisting[style=customasmx86]{examples/dongles/2/2.lst}

SSQ() peut être appelé depuis au moins 2 fonctions.

L'une d'entre elles est:

\lstinputlisting[style=customasmx86]{examples/dongles/2/check1_EN.lst}

\q{\TT{3C}} et \q{\TT{3E}} semblent familiers: il y avait un dongle Sentinel Pro de
Rainbow sans mémoire, fournissant seulement une fonction de crypto-hachage secrète.

Vous pouvez lire une courte description de la fonction de hachage dont il s'agit
ici: \myref{hash_func}.

Mais retournons au programme.

Donc le programme peut seulement tester si un dongle est connecté ou s'il est absent.

Aucune autre information ne peut être écrite dans un tel dongle, puisqu'il n'a pas
de mémoire.
Les codes sur deux caractères sont des commandes (nous pouvons voir comment les commandes
sont traitées dans la fonction \TT{SSQC()}) et toutes les autres chaînes sont hachées
dans le dongle, transformées en un nombre 16-bit.
L'algorithme était secret, donc il n'était pas possible d'écrire un driver de remplacement
ou de refaire un dongle matériel qui l'émulerait parfaitement.

Toutefois, il est toujours possible d'intercepter tous les accès au dongle et de
trouver les constantes auxquelles les résultats de la fonction de hachage sont comparées.

Mais nous devons dire qu'il est possible de construire un schéma de logiciel de protection
de copie robuste basé sur une fonction secrète de hachage cryptographique: il suffit
qu'elle chiffre/déchiffre les fichiers de données utilisés par votre logiciel.

Mais retournons au code:

Les codes 51/52/53 sont utilisés pour choisir le port imprimante LPT.
3x/4x sont utilisés pour le choix de la \q{famille} (c'est ainsi que les dongles
Sentinel Pro sont différenciés les uns des autres: plus d'un dongle peut être connecté
sur un port LPT).

La seule chaîne passée à la fonction qui ne fasse pas 2 caractères est "0123456789".

Ensuite, le résultat est comparé à l'ensemble des résultats valides.

Si il est correct, 0xC ou 0xB est écrit dans la variable globale \TT{ctl\_model}.%

Une autre chaîne de texte qui est passée est
"PRESS ANY KEY TO CONTINUE: ", mais le résultat n'est pas testé.
Difficile de dire pourquoi, probablement une erreur\footnote{C'est un sentiment
étrange de trouver un bug dans un logiciel aussi ancien.}.

Voyons où la valeur de la variable globale \TT{ctl\_model} est utilisée.

Un tel endroit est:

\lstinputlisting[style=customasmx86]{examples/dongles/2/4.lst}

Si c'est 0, un message d'erreur chiffré est passé à une routine de déchiffrement
et affiché.

\myindex{x86!\Instructions!XOR}

La routine de déchiffrement de la chaîne semble être un simple \glslink{xoring}{xor}:

\lstinputlisting[style=customasmx86]{examples/dongles/2/err_warn.lst}

C'est pourquoi nous étions incapable de trouver le message d'erreur dans les fichiers
exécutable, car ils sont chiffrés (ce qui est une pratique courante).

Un autre appel à la fonction de hachage \TT{SSQ()} lui passe la chaîne \q{offln}
et le résultat est comparé avec \TT{0xFE81} et \TT{0x12A9}.

Si ils ne correspondent pas, ça se comporte comme une sorte de fonction \TT{timer()}
(peut-être en attente qu'un dongle mal connecté soit reconnecté et re-testé?) et ensuite
déchiffre un autre message d'erreur à afficher.

\lstinputlisting[style=customasmx86]{examples/dongles/2/check2_EN.lst}

Passer outre le dongle est assez facile: il suffit de patcher tous les sauts après
les instructions \CMP pertinentes.

Une autre option est d'écrire notre propre driver SCO OpenServer, contenant une table
de questions et de réponses, toutes celles qui sont présentent dans le programme.

\subsubsection{Déchiffrer les messages d'erreur}

À propos, nous pouvons aussi essayer de déchiffrer tous les messages d'erreurs.
L'algorithme qui se trouve dans la fonction \TT{err\_warn()} est très simple, en effet:

\lstinputlisting[caption=Decryption function,style=customasmx86]{examples/dongles/2/decrypting_FR.lst}

Comme on le voit, non seulement la chaîne est transmise à la fonction de déchiffrement
mais aussi la clef:

\lstinputlisting[style=customasmx86]{examples/dongles/2/tmp1_EN.asm}

L'algorithme est un simple \glslink{xoring}{xor}: chaque octet est xoré avec la clef, mais
la clef est incrémentée de 3 après le traitement de chaque octet.

Nous pouvons écrire un petit script Python pour vérifier notre hypothèse:

\lstinputlisting[caption=Python 3.x]{examples/dongles/2/decr1.py}

Et il affiche: \q{check security device connection}.
Donc oui, ceci est le message déchiffré.

Il y a d'autres messages chiffrés, avec leur clef correspondante.
Mais inutile de dire qu'il est possible de les déchiffrer sans leur clef.
Premièrement, nous voyons que le clef est en fait un octet.
C'est parce que l'instruction principale de déchiffrement (\XOR) fonctionne au niveau
de l'octet.
La clef se trouve dans le registre \ESI, mais seulement une partie de \ESI d'un octet
est utilisée.
Ainsi, une clef pourrait être plus grande que 255, mais sa valeur est toujours arrondie.

En conséquence, nous pouvons simplement essayer de brute-forcer, en essayant toutes
les clefs possible dans l'intervalle 0..255.
Nous allons aussi écarter les messages comportants des caractères non-imprimable.

\lstinputlisting[caption=Python 3.x]{examples/dongles/2/decr2.py}

Et nous obtenons:

\lstinputlisting[caption=Results]{examples/dongles/2/decr2_result.txt}

Ici il y a un peu de déchet, mais nous pouvons rapidement trouver les messages en
anglais.

À propos, puisque l'algorithme est un simple chiffrement xor, la même fonction peut
être utilisée pour chiffrer les messages.
Si besoin, nous pouvons chiffrer nos propres messages, et patcher le programme en les insérant.
}
\IT{\subsection{Fall-through}

Un altro uso diffuso dell'operatore \TT{switch()} è il cosiddetto \q{fallthrough}.
Ecco un semplice esempio \footnote{Preso da \url{https://github.com/azonalon/prgraas/blob/master/prog1lib/lecture_examples/is_whitespace.c}}:

\lstinputlisting[numbers=left,style=customc]{patterns/08_switch/4_fallthrough/fallthrough1.c}

Uno leggermente più difficile, dal kernel di Linux \footnote{Preso da \url{https://github.com/torvalds/linux/blob/master/drivers/media/dvb-frontends/lgdt3306a.c}}:

\lstinputlisting[numbers=left,style=customc]{patterns/08_switch/4_fallthrough/fallthrough2.c}

\lstinputlisting[caption=Optimizing GCC 5.4.0 x86,numbers=left,style=customasmx86]{patterns/08_switch/4_fallthrough/fallthrough2.s}

Possiamo arrivare alla label \TT{.L5} se all'input della funzione viene dato il valore 3250.
Ma si può anche giungere allo stesso punto da un altro percorso:
notiamo che non ci sono jump tra la chiamata a \printf e la label \TT{.L5}.

Questo spiega facilmente perchè i costrutti con \emph{switch()} sono spesso fonte di bug:
è sufficiente dimenticare un \emph{break} per trasformare il costrutto \emph{switch()} in un \emph{fallthrough} , in cui vengono eseguiti
più blocchi invece di uno solo.
}

\EN{\mysection{Task manager practical joke (Windows Vista)}
\myindex{Windows!Windows Vista}

Let's see if it's possible to hack Task Manager slightly so it would detect more \ac{CPU} cores.

\myindex{Windows!NTAPI}

Let us first think, how does the Task Manager know the number of cores?

There is the \TT{GetSystemInfo()} win32 function present in win32 userspace which can tell us this.
But it's not imported in \TT{taskmgr.exe}.

There is, however, another one in \gls{NTAPI}, \TT{NtQuerySystemInformation()}, 
which is used in \TT{taskmgr.exe} in several places.

To get the number of cores, one has to call this function with the \TT{SystemBasicInformation} constant
as a first argument (which is zero
\footnote{\href{http://msdn.microsoft.com/en-us/library/windows/desktop/ms724509(v=vs.85).aspx}{MSDN}}).

The second argument has to point to the buffer which is getting all the information.

So we have to find all calls to the \\
\TT{NtQuerySystemInformation(0, ?, ?, ?)} function.
Let's open \TT{taskmgr.exe} in IDA. 
\myindex{Windows!PDB}

What is always good about Microsoft executables is that IDA can download the corresponding \gls{PDB} 
file for this executable and show all function names.

It is visible that Task Manager is written in \Cpp and some of the function names and classes are really 
speaking for themselves.
There are classes CAdapter, CNetPage, CPerfPage, CProcInfo, CProcPage, CSvcPage, 
CTaskPage, CUserPage.

Apparently, each class corresponds to each tab in Task Manager.

Let's visit each call and add comment with the value which is passed as the first function argument.
We will write \q{not zero} at some places, because the value there was clearly not zero, 
but something really different (more about this in the second part of this chapter).

And we are looking for zero passed as argument, after all.

\begin{figure}[H]
\centering
\myincludegraphics{examples/taskmgr/IDA_xrefs.png}
\caption{IDA: cross references to NtQuerySystemInformation()}
\end{figure}

Yes, the names are really speaking for themselves.

When we closely investigate each place where\\
\TT{NtQuerySystemInformation(0, ?, ?, ?)} is called,
we quickly find what we need in the \TT{InitPerfInfo()} function:

\lstinputlisting[caption=taskmgr.exe (Windows Vista),style=customasmx86]{examples/taskmgr/taskmgr.lst}

\TT{g\_cProcessors} is a global variable, and this name has been assigned by 
IDA according to the \gls{PDB} loaded from Microsoft's symbol server.

The byte is taken from \TT{var\_C20}. 
And \TT{var\_C58} is passed to\\
\TT{NtQuerySystemInformation()} 
as a pointer to the receiving buffer.
The difference between 0xC20 and 0xC58 is 0x38 (56).

Let's take a look at format of the return structure, which we can find in MSDN:

\begin{lstlisting}[style=customc]
typedef struct _SYSTEM_BASIC_INFORMATION {
    BYTE Reserved1[24];
    PVOID Reserved2[4];
    CCHAR NumberOfProcessors;
} SYSTEM_BASIC_INFORMATION;
\end{lstlisting}

This is a x64 system, so each PVOID takes 8 bytes.

All \emph{reserved} fields in the structure take $24+4*8=56$ bytes.

Oh yes, this implies that \TT{var\_C20} is the local stack is exactly the
\TT{NumberOfProcessors} field of the \TT{SYSTEM\_BASIC\_INFORMATION} structure.

Let's check our guess.
Copy \TT{taskmgr.exe} from \TT{C:\textbackslash{}Windows\textbackslash{}System32} 
to some other folder 
(so the \emph{Windows Resource Protection} 
will not try to restore the patched \TT{taskmgr.exe}).

Let's open it in Hiew and find the place:

\begin{figure}[H]
\centering
\myincludegraphics{examples/taskmgr/hiew2.png}
\caption{Hiew: find the place to be patched}
\end{figure}

Let's replace the \TT{MOVZX} instruction with ours.
Let's pretend we've got 64 CPU cores.

Add one additional \ac{NOP} (because our instruction is shorter than the original one):

\begin{figure}[H]
\centering
\myincludegraphics{examples/taskmgr/hiew1.png}
\caption{Hiew: patch it}
\end{figure}

And it works!
Of course, the data in the graphs is not correct.

At times, Task Manager even shows an overall CPU load of more than 100\%.

\begin{figure}[H]
\centering
\myincludegraphics{examples/taskmgr/taskmgr_64cpu_crop.png}
\caption{Fooled Windows Task Manager}
\end{figure}

The biggest number Task Manager does not crash with is 64.

Apparently, Task Manager in Windows Vista was not tested on computers with a large number of cores.

So there are probably some static data structure(s) inside it limited to 64 cores.

\subsection{Using LEA to load values}
\label{TaskMgr_LEA}

Sometimes, \TT{LEA} is used in \TT{taskmgr.exe} instead of \TT{MOV} to set the first argument of \\
\TT{NtQuerySystemInformation()}:

\lstinputlisting[caption=taskmgr.exe (Windows Vista),style=customasmx86]{examples/taskmgr/taskmgr2.lst}

\myindex{x86!\Instructions!LEA}

Perhaps \ac{MSVC} did so because machine code of \INS{LEA} is shorter than \INS{MOV REG, 5} (would be 5 instead of 4).

\INS{LEA} with offset in $-128..127$ range (offset will occupy 1 byte in opcode) with 32-bit registers is even shorter (for lack of REX prefix)---3 bytes.

Another example of such thing is: \myref{using_MOV_and_pack_of_LEA_to_load_values}.
}
\RU{\subsection{Обменять входные значения друг с другом}

Вот так:

\lstinputlisting[style=customc]{patterns/061_pointers/swap/5_RU.c}

Как видим, байты загружаются в младшие 8-битные части регистров \TT{ECX} и \TT{EBX} используя \INS{MOVZX}
(так что старшие части регистров очищаются), затем байты записываются назад в другом порядке.

\lstinputlisting[style=customasmx86,caption=Optimizing GCC 5.4]{patterns/061_pointers/swap/5_GCC_O3_x86.s}

Адреса обоих байтов берутся из аргументов и во время исполнения ф-ции находятся в регистрах \TT{EDX} и \TT{EAX}.

Так что исопльзуем указатели --- вероятно, без них нет способа решить эту задачу лучше.

}
\FR{\subsection{Exemple \#2: SCO OpenServer}

\label{examples_SCO}
\myindex{SCO OpenServer}
Un ancien logiciel pour SCO OpenServer de 1997 développé par une société qui a disparue
depuis longtemps.

Il y a un driver de dongle special à installer dans le système, qui contient les
chaînes de texte suivantes:
\q{Copyright 1989, Rainbow Technologies, Inc., Irvine, CA}
et
\q{Sentinel Integrated Driver Ver. 3.0 }.

Après l'installation du driver dans SCO OpenServer, ces fichiers apparaissent dans
l'arborescence /dev:

\begin{lstlisting}
/dev/rbsl8
/dev/rbsl9
/dev/rbsl10
\end{lstlisting}

Le programme renvoie une erreur lorsque le dongle n'est pas connecté, mais le message
d'erreur n'est pas trouvé dans les exécutables.

\myindex{COFF}

Grâce à \ac{IDA}, il est facile de charger l'exécutable COFF utilisé dans SCO OpenServer.

Essayons de trouver la chaîne \q{rbsl} et en effet, elle se trouve dans ce morceau
de code:

\lstinputlisting[style=customasmx86]{examples/dongles/2/1.lst}

Oui, en effet, le programme doit communiquer d'une façon ou d'une autre avec le driver.

\myindex{thunk-functions}
Le seul endroit où la fonction \TT{SSQC()} est appelée est dans la \glslink{thunk
 function}{fonction thunk}:

\lstinputlisting[style=customasmx86]{examples/dongles/2/2.lst}

SSQ() peut être appelé depuis au moins 2 fonctions.

L'une d'entre elles est:

\lstinputlisting[style=customasmx86]{examples/dongles/2/check1_EN.lst}

\q{\TT{3C}} et \q{\TT{3E}} semblent familiers: il y avait un dongle Sentinel Pro de
Rainbow sans mémoire, fournissant seulement une fonction de crypto-hachage secrète.

Vous pouvez lire une courte description de la fonction de hachage dont il s'agit
ici: \myref{hash_func}.

Mais retournons au programme.

Donc le programme peut seulement tester si un dongle est connecté ou s'il est absent.

Aucune autre information ne peut être écrite dans un tel dongle, puisqu'il n'a pas
de mémoire.
Les codes sur deux caractères sont des commandes (nous pouvons voir comment les commandes
sont traitées dans la fonction \TT{SSQC()}) et toutes les autres chaînes sont hachées
dans le dongle, transformées en un nombre 16-bit.
L'algorithme était secret, donc il n'était pas possible d'écrire un driver de remplacement
ou de refaire un dongle matériel qui l'émulerait parfaitement.

Toutefois, il est toujours possible d'intercepter tous les accès au dongle et de
trouver les constantes auxquelles les résultats de la fonction de hachage sont comparées.

Mais nous devons dire qu'il est possible de construire un schéma de logiciel de protection
de copie robuste basé sur une fonction secrète de hachage cryptographique: il suffit
qu'elle chiffre/déchiffre les fichiers de données utilisés par votre logiciel.

Mais retournons au code:

Les codes 51/52/53 sont utilisés pour choisir le port imprimante LPT.
3x/4x sont utilisés pour le choix de la \q{famille} (c'est ainsi que les dongles
Sentinel Pro sont différenciés les uns des autres: plus d'un dongle peut être connecté
sur un port LPT).

La seule chaîne passée à la fonction qui ne fasse pas 2 caractères est "0123456789".

Ensuite, le résultat est comparé à l'ensemble des résultats valides.

Si il est correct, 0xC ou 0xB est écrit dans la variable globale \TT{ctl\_model}.%

Une autre chaîne de texte qui est passée est
"PRESS ANY KEY TO CONTINUE: ", mais le résultat n'est pas testé.
Difficile de dire pourquoi, probablement une erreur\footnote{C'est un sentiment
étrange de trouver un bug dans un logiciel aussi ancien.}.

Voyons où la valeur de la variable globale \TT{ctl\_model} est utilisée.

Un tel endroit est:

\lstinputlisting[style=customasmx86]{examples/dongles/2/4.lst}

Si c'est 0, un message d'erreur chiffré est passé à une routine de déchiffrement
et affiché.

\myindex{x86!\Instructions!XOR}

La routine de déchiffrement de la chaîne semble être un simple \glslink{xoring}{xor}:

\lstinputlisting[style=customasmx86]{examples/dongles/2/err_warn.lst}

C'est pourquoi nous étions incapable de trouver le message d'erreur dans les fichiers
exécutable, car ils sont chiffrés (ce qui est une pratique courante).

Un autre appel à la fonction de hachage \TT{SSQ()} lui passe la chaîne \q{offln}
et le résultat est comparé avec \TT{0xFE81} et \TT{0x12A9}.

Si ils ne correspondent pas, ça se comporte comme une sorte de fonction \TT{timer()}
(peut-être en attente qu'un dongle mal connecté soit reconnecté et re-testé?) et ensuite
déchiffre un autre message d'erreur à afficher.

\lstinputlisting[style=customasmx86]{examples/dongles/2/check2_EN.lst}

Passer outre le dongle est assez facile: il suffit de patcher tous les sauts après
les instructions \CMP pertinentes.

Une autre option est d'écrire notre propre driver SCO OpenServer, contenant une table
de questions et de réponses, toutes celles qui sont présentent dans le programme.

\subsubsection{Déchiffrer les messages d'erreur}

À propos, nous pouvons aussi essayer de déchiffrer tous les messages d'erreurs.
L'algorithme qui se trouve dans la fonction \TT{err\_warn()} est très simple, en effet:

\lstinputlisting[caption=Decryption function,style=customasmx86]{examples/dongles/2/decrypting_FR.lst}

Comme on le voit, non seulement la chaîne est transmise à la fonction de déchiffrement
mais aussi la clef:

\lstinputlisting[style=customasmx86]{examples/dongles/2/tmp1_EN.asm}

L'algorithme est un simple \glslink{xoring}{xor}: chaque octet est xoré avec la clef, mais
la clef est incrémentée de 3 après le traitement de chaque octet.

Nous pouvons écrire un petit script Python pour vérifier notre hypothèse:

\lstinputlisting[caption=Python 3.x]{examples/dongles/2/decr1.py}

Et il affiche: \q{check security device connection}.
Donc oui, ceci est le message déchiffré.

Il y a d'autres messages chiffrés, avec leur clef correspondante.
Mais inutile de dire qu'il est possible de les déchiffrer sans leur clef.
Premièrement, nous voyons que le clef est en fait un octet.
C'est parce que l'instruction principale de déchiffrement (\XOR) fonctionne au niveau
de l'octet.
La clef se trouve dans le registre \ESI, mais seulement une partie de \ESI d'un octet
est utilisée.
Ainsi, une clef pourrait être plus grande que 255, mais sa valeur est toujours arrondie.

En conséquence, nous pouvons simplement essayer de brute-forcer, en essayant toutes
les clefs possible dans l'intervalle 0..255.
Nous allons aussi écarter les messages comportants des caractères non-imprimable.

\lstinputlisting[caption=Python 3.x]{examples/dongles/2/decr2.py}

Et nous obtenons:

\lstinputlisting[caption=Results]{examples/dongles/2/decr2_result.txt}

Ici il y a un peu de déchet, mais nous pouvons rapidement trouver les messages en
anglais.

À propos, puisque l'algorithme est un simple chiffrement xor, la même fonction peut
être utilisée pour chiffrer les messages.
Si besoin, nous pouvons chiffrer nos propres messages, et patcher le programme en les insérant.
}

\RU{\subsection{Обменять входные значения друг с другом}

Вот так:

\lstinputlisting[style=customc]{patterns/061_pointers/swap/5_RU.c}

Как видим, байты загружаются в младшие 8-битные части регистров \TT{ECX} и \TT{EBX} используя \INS{MOVZX}
(так что старшие части регистров очищаются), затем байты записываются назад в другом порядке.

\lstinputlisting[style=customasmx86,caption=Optimizing GCC 5.4]{patterns/061_pointers/swap/5_GCC_O3_x86.s}

Адреса обоих байтов берутся из аргументов и во время исполнения ф-ции находятся в регистрах \TT{EDX} и \TT{EAX}.

Так что исопльзуем указатели --- вероятно, без них нет способа решить эту задачу лучше.

}
\EN{\mysection{Task manager practical joke (Windows Vista)}
\myindex{Windows!Windows Vista}

Let's see if it's possible to hack Task Manager slightly so it would detect more \ac{CPU} cores.

\myindex{Windows!NTAPI}

Let us first think, how does the Task Manager know the number of cores?

There is the \TT{GetSystemInfo()} win32 function present in win32 userspace which can tell us this.
But it's not imported in \TT{taskmgr.exe}.

There is, however, another one in \gls{NTAPI}, \TT{NtQuerySystemInformation()}, 
which is used in \TT{taskmgr.exe} in several places.

To get the number of cores, one has to call this function with the \TT{SystemBasicInformation} constant
as a first argument (which is zero
\footnote{\href{http://msdn.microsoft.com/en-us/library/windows/desktop/ms724509(v=vs.85).aspx}{MSDN}}).

The second argument has to point to the buffer which is getting all the information.

So we have to find all calls to the \\
\TT{NtQuerySystemInformation(0, ?, ?, ?)} function.
Let's open \TT{taskmgr.exe} in IDA. 
\myindex{Windows!PDB}

What is always good about Microsoft executables is that IDA can download the corresponding \gls{PDB} 
file for this executable and show all function names.

It is visible that Task Manager is written in \Cpp and some of the function names and classes are really 
speaking for themselves.
There are classes CAdapter, CNetPage, CPerfPage, CProcInfo, CProcPage, CSvcPage, 
CTaskPage, CUserPage.

Apparently, each class corresponds to each tab in Task Manager.

Let's visit each call and add comment with the value which is passed as the first function argument.
We will write \q{not zero} at some places, because the value there was clearly not zero, 
but something really different (more about this in the second part of this chapter).

And we are looking for zero passed as argument, after all.

\begin{figure}[H]
\centering
\myincludegraphics{examples/taskmgr/IDA_xrefs.png}
\caption{IDA: cross references to NtQuerySystemInformation()}
\end{figure}

Yes, the names are really speaking for themselves.

When we closely investigate each place where\\
\TT{NtQuerySystemInformation(0, ?, ?, ?)} is called,
we quickly find what we need in the \TT{InitPerfInfo()} function:

\lstinputlisting[caption=taskmgr.exe (Windows Vista),style=customasmx86]{examples/taskmgr/taskmgr.lst}

\TT{g\_cProcessors} is a global variable, and this name has been assigned by 
IDA according to the \gls{PDB} loaded from Microsoft's symbol server.

The byte is taken from \TT{var\_C20}. 
And \TT{var\_C58} is passed to\\
\TT{NtQuerySystemInformation()} 
as a pointer to the receiving buffer.
The difference between 0xC20 and 0xC58 is 0x38 (56).

Let's take a look at format of the return structure, which we can find in MSDN:

\begin{lstlisting}[style=customc]
typedef struct _SYSTEM_BASIC_INFORMATION {
    BYTE Reserved1[24];
    PVOID Reserved2[4];
    CCHAR NumberOfProcessors;
} SYSTEM_BASIC_INFORMATION;
\end{lstlisting}

This is a x64 system, so each PVOID takes 8 bytes.

All \emph{reserved} fields in the structure take $24+4*8=56$ bytes.

Oh yes, this implies that \TT{var\_C20} is the local stack is exactly the
\TT{NumberOfProcessors} field of the \TT{SYSTEM\_BASIC\_INFORMATION} structure.

Let's check our guess.
Copy \TT{taskmgr.exe} from \TT{C:\textbackslash{}Windows\textbackslash{}System32} 
to some other folder 
(so the \emph{Windows Resource Protection} 
will not try to restore the patched \TT{taskmgr.exe}).

Let's open it in Hiew and find the place:

\begin{figure}[H]
\centering
\myincludegraphics{examples/taskmgr/hiew2.png}
\caption{Hiew: find the place to be patched}
\end{figure}

Let's replace the \TT{MOVZX} instruction with ours.
Let's pretend we've got 64 CPU cores.

Add one additional \ac{NOP} (because our instruction is shorter than the original one):

\begin{figure}[H]
\centering
\myincludegraphics{examples/taskmgr/hiew1.png}
\caption{Hiew: patch it}
\end{figure}

And it works!
Of course, the data in the graphs is not correct.

At times, Task Manager even shows an overall CPU load of more than 100\%.

\begin{figure}[H]
\centering
\myincludegraphics{examples/taskmgr/taskmgr_64cpu_crop.png}
\caption{Fooled Windows Task Manager}
\end{figure}

The biggest number Task Manager does not crash with is 64.

Apparently, Task Manager in Windows Vista was not tested on computers with a large number of cores.

So there are probably some static data structure(s) inside it limited to 64 cores.

\subsection{Using LEA to load values}
\label{TaskMgr_LEA}

Sometimes, \TT{LEA} is used in \TT{taskmgr.exe} instead of \TT{MOV} to set the first argument of \\
\TT{NtQuerySystemInformation()}:

\lstinputlisting[caption=taskmgr.exe (Windows Vista),style=customasmx86]{examples/taskmgr/taskmgr2.lst}

\myindex{x86!\Instructions!LEA}

Perhaps \ac{MSVC} did so because machine code of \INS{LEA} is shorter than \INS{MOV REG, 5} (would be 5 instead of 4).

\INS{LEA} with offset in $-128..127$ range (offset will occupy 1 byte in opcode) with 32-bit registers is even shorter (for lack of REX prefix)---3 bytes.

Another example of such thing is: \myref{using_MOV_and_pack_of_LEA_to_load_values}.
}
\FR{\subsection{Exemple \#2: SCO OpenServer}

\label{examples_SCO}
\myindex{SCO OpenServer}
Un ancien logiciel pour SCO OpenServer de 1997 développé par une société qui a disparue
depuis longtemps.

Il y a un driver de dongle special à installer dans le système, qui contient les
chaînes de texte suivantes:
\q{Copyright 1989, Rainbow Technologies, Inc., Irvine, CA}
et
\q{Sentinel Integrated Driver Ver. 3.0 }.

Après l'installation du driver dans SCO OpenServer, ces fichiers apparaissent dans
l'arborescence /dev:

\begin{lstlisting}
/dev/rbsl8
/dev/rbsl9
/dev/rbsl10
\end{lstlisting}

Le programme renvoie une erreur lorsque le dongle n'est pas connecté, mais le message
d'erreur n'est pas trouvé dans les exécutables.

\myindex{COFF}

Grâce à \ac{IDA}, il est facile de charger l'exécutable COFF utilisé dans SCO OpenServer.

Essayons de trouver la chaîne \q{rbsl} et en effet, elle se trouve dans ce morceau
de code:

\lstinputlisting[style=customasmx86]{examples/dongles/2/1.lst}

Oui, en effet, le programme doit communiquer d'une façon ou d'une autre avec le driver.

\myindex{thunk-functions}
Le seul endroit où la fonction \TT{SSQC()} est appelée est dans la \glslink{thunk
 function}{fonction thunk}:

\lstinputlisting[style=customasmx86]{examples/dongles/2/2.lst}

SSQ() peut être appelé depuis au moins 2 fonctions.

L'une d'entre elles est:

\lstinputlisting[style=customasmx86]{examples/dongles/2/check1_EN.lst}

\q{\TT{3C}} et \q{\TT{3E}} semblent familiers: il y avait un dongle Sentinel Pro de
Rainbow sans mémoire, fournissant seulement une fonction de crypto-hachage secrète.

Vous pouvez lire une courte description de la fonction de hachage dont il s'agit
ici: \myref{hash_func}.

Mais retournons au programme.

Donc le programme peut seulement tester si un dongle est connecté ou s'il est absent.

Aucune autre information ne peut être écrite dans un tel dongle, puisqu'il n'a pas
de mémoire.
Les codes sur deux caractères sont des commandes (nous pouvons voir comment les commandes
sont traitées dans la fonction \TT{SSQC()}) et toutes les autres chaînes sont hachées
dans le dongle, transformées en un nombre 16-bit.
L'algorithme était secret, donc il n'était pas possible d'écrire un driver de remplacement
ou de refaire un dongle matériel qui l'émulerait parfaitement.

Toutefois, il est toujours possible d'intercepter tous les accès au dongle et de
trouver les constantes auxquelles les résultats de la fonction de hachage sont comparées.

Mais nous devons dire qu'il est possible de construire un schéma de logiciel de protection
de copie robuste basé sur une fonction secrète de hachage cryptographique: il suffit
qu'elle chiffre/déchiffre les fichiers de données utilisés par votre logiciel.

Mais retournons au code:

Les codes 51/52/53 sont utilisés pour choisir le port imprimante LPT.
3x/4x sont utilisés pour le choix de la \q{famille} (c'est ainsi que les dongles
Sentinel Pro sont différenciés les uns des autres: plus d'un dongle peut être connecté
sur un port LPT).

La seule chaîne passée à la fonction qui ne fasse pas 2 caractères est "0123456789".

Ensuite, le résultat est comparé à l'ensemble des résultats valides.

Si il est correct, 0xC ou 0xB est écrit dans la variable globale \TT{ctl\_model}.%

Une autre chaîne de texte qui est passée est
"PRESS ANY KEY TO CONTINUE: ", mais le résultat n'est pas testé.
Difficile de dire pourquoi, probablement une erreur\footnote{C'est un sentiment
étrange de trouver un bug dans un logiciel aussi ancien.}.

Voyons où la valeur de la variable globale \TT{ctl\_model} est utilisée.

Un tel endroit est:

\lstinputlisting[style=customasmx86]{examples/dongles/2/4.lst}

Si c'est 0, un message d'erreur chiffré est passé à une routine de déchiffrement
et affiché.

\myindex{x86!\Instructions!XOR}

La routine de déchiffrement de la chaîne semble être un simple \glslink{xoring}{xor}:

\lstinputlisting[style=customasmx86]{examples/dongles/2/err_warn.lst}

C'est pourquoi nous étions incapable de trouver le message d'erreur dans les fichiers
exécutable, car ils sont chiffrés (ce qui est une pratique courante).

Un autre appel à la fonction de hachage \TT{SSQ()} lui passe la chaîne \q{offln}
et le résultat est comparé avec \TT{0xFE81} et \TT{0x12A9}.

Si ils ne correspondent pas, ça se comporte comme une sorte de fonction \TT{timer()}
(peut-être en attente qu'un dongle mal connecté soit reconnecté et re-testé?) et ensuite
déchiffre un autre message d'erreur à afficher.

\lstinputlisting[style=customasmx86]{examples/dongles/2/check2_EN.lst}

Passer outre le dongle est assez facile: il suffit de patcher tous les sauts après
les instructions \CMP pertinentes.

Une autre option est d'écrire notre propre driver SCO OpenServer, contenant une table
de questions et de réponses, toutes celles qui sont présentent dans le programme.

\subsubsection{Déchiffrer les messages d'erreur}

À propos, nous pouvons aussi essayer de déchiffrer tous les messages d'erreurs.
L'algorithme qui se trouve dans la fonction \TT{err\_warn()} est très simple, en effet:

\lstinputlisting[caption=Decryption function,style=customasmx86]{examples/dongles/2/decrypting_FR.lst}

Comme on le voit, non seulement la chaîne est transmise à la fonction de déchiffrement
mais aussi la clef:

\lstinputlisting[style=customasmx86]{examples/dongles/2/tmp1_EN.asm}

L'algorithme est un simple \glslink{xoring}{xor}: chaque octet est xoré avec la clef, mais
la clef est incrémentée de 3 après le traitement de chaque octet.

Nous pouvons écrire un petit script Python pour vérifier notre hypothèse:

\lstinputlisting[caption=Python 3.x]{examples/dongles/2/decr1.py}

Et il affiche: \q{check security device connection}.
Donc oui, ceci est le message déchiffré.

Il y a d'autres messages chiffrés, avec leur clef correspondante.
Mais inutile de dire qu'il est possible de les déchiffrer sans leur clef.
Premièrement, nous voyons que le clef est en fait un octet.
C'est parce que l'instruction principale de déchiffrement (\XOR) fonctionne au niveau
de l'octet.
La clef se trouve dans le registre \ESI, mais seulement une partie de \ESI d'un octet
est utilisée.
Ainsi, une clef pourrait être plus grande que 255, mais sa valeur est toujours arrondie.

En conséquence, nous pouvons simplement essayer de brute-forcer, en essayant toutes
les clefs possible dans l'intervalle 0..255.
Nous allons aussi écarter les messages comportants des caractères non-imprimable.

\lstinputlisting[caption=Python 3.x]{examples/dongles/2/decr2.py}

Et nous obtenons:

\lstinputlisting[caption=Results]{examples/dongles/2/decr2_result.txt}

Ici il y a un peu de déchet, mais nous pouvons rapidement trouver les messages en
anglais.

À propos, puisque l'algorithme est un simple chiffrement xor, la même fonction peut
être utilisée pour chiffrer les messages.
Si besoin, nous pouvons chiffrer nos propres messages, et patcher le programme en les insérant.
}

\EN{\mysection{Some GCC library functions}
\myindex{GCC}
\label{sec:GCC_library_func}

%__ashldi3
%__ashrdi3
%__floatundidf
%__floatdisf
%__floatdixf
%__floatundidf
%__floatundisf
%__floatundixf
%__lshrdi3
%__muldi3

\begin{center}
\begin{tabular}{ | l | l | }
\hline
\HeaderColor name & \HeaderColor meaning \\
\hline \TT{\_\_divdi3} & signed division \\
\hline \TT{\_\_moddi3} & getting remainder (modulo) of signed division \\
\hline \TT{\_\_udivdi3} & unsigned division \\
\hline \TT{\_\_umoddi3} & getting remainder (modulo) of unsigned division \\
\hline
\end{tabular}
\end{center}

}
\RU{\mysection{Некоторые библиотечные функции GCC}
\myindex{GCC}
\label{sec:GCC_library_func}

%__ashldi3
%__ashrdi3
%__floatundidf
%__floatdisf
%__floatdixf
%__floatundidf
%__floatundisf
%__floatundixf
%__lshrdi3
%__muldi3

\begin{center}
\begin{tabular}{ | l | l | }
\hline
\HeaderColor имя & \HeaderColor значение \\
\hline \TT{\_\_divdi3} & знаковое деление \\
\hline \TT{\_\_moddi3} & остаток от знакового деления \\
\hline \TT{\_\_udivdi3} & беззнаковое деление \\
\hline \TT{\_\_umoddi3} & остаток от беззнакового деления \\
\hline
\end{tabular}
\end{center}

}
\FR{\mysection{Quelques fonctions de la bibliothèque de GCC}
\myindex{GCC}
\label{sec:GCC_library_func}

%__ashldi3
%__ashrdi3
%__floatundidf
%__floatdisf
%__floatdixf
%__floatundidf
%__floatundisf
%__floatundixf
%__lshrdi3
%__muldi3

\begin{center}
\begin{tabular}{ | l | l | }
\hline
\HeaderColor nom & \HeaderColor signification \\
\hline \TT{\_\_divdi3} & division signée \\
\hline \TT{\_\_moddi3} & reste (modulo) d'une division signée \\
\hline \TT{\_\_udivdi3} & division non signée \\
\hline \TT{\_\_umoddi3} & reste (modulo) d'une division non signée \\
\hline
\end{tabular}
\end{center}

}
\DE{\mysection{Einige GCC-Bibliotheks-Funktionen}
\myindex{GCC}
\label{sec:GCC_library_func}

%__ashldi3
%__ashrdi3
%__floatundidf
%__floatdisf
%__floatdixf
%__floatundidf
%__floatundisf
%__floatundixf
%__lshrdi3
%__muldi3

\begin{center}
\begin{tabular}{ | l | l | }
\hline
\HeaderColor Name & \HeaderColor Bedeutung \\
\hline \TT{\_\_divdi3} & vorzeichenbehaftete Division \\
\hline \TT{\_\_moddi3} & Rest (Modulo) einer vorzeichenbehafteten Division \\
\hline \TT{\_\_udivdi3} & vorzeichenlose Division \\
\hline \TT{\_\_umoddi3} & Rest (Modulo) einer vorzeichenlosen Division \\
\hline
\end{tabular}
\end{center}

}


\mysection{\RU{Некоторые библиотечные функции MSVC}\EN{Some MSVC library functions}\DE{Einige MSVC-Bibliotheks-Funktionen}%
\FR{Quelques fonctions de la bibliothèque MSVC}}
\myindex{MSVC}
\label{sec:MSVC_library_func}

\TT{ll} \RU{в имени функции означает}\EN{in function name stands for}\DE{in Funktionsnamen steht für}%
\FR{dans une fontion signifie} \q{long long}, \RU{т.е. 64-битный тип данных}
\EN{e.g., a 64-bit data type}\DE{z.B. einen 64-Bit-Datentyp}\FR{i.e., type de donées 64-bit}.

\begin{center}
\begin{tabular}{ | l | l | }
\hline
\HeaderColor \RU{имя}\EN{name}\DE{Name}\FR{nom} & \HeaderColor \RU{значение}\EN{meaning}\DE{Bedeutung}\FR{signification} \\
\hline \TT{\_\_alldiv} & \RU{знаковое деление}\EN{signed division}\DE{vorzeichenbehaftete Division}\FR{division signée} \\
\hline \TT{\_\_allmul} & \RU{умножение}\EN{multiplication}\DE{Multiplikation}\FR{multiplication} \\
\hline \TT{\_\_allrem} & \RU{остаток от знакового деления}\EN{remainder of signed division}\DE{Rest einer vorzeichenbehafteten Division}%
\FR{reste de la division signée} \\
\hline \TT{\_\_allshl} & \RU{сдвиг влево}\EN{shift left}\DE{Schiebe links}\FR{décalage à gauche} \\
\hline \TT{\_\_allshr} & \RU{знаковый сдвиг вправо}\EN{signed shift right}\DE{Schiebe links, vorzeichenbehaftet}%
\FR{décalage signé à droite} \\
\hline \TT{\_\_aulldiv} & \RU{беззнаковое деление}\EN{unsigned division}\DE{vorzeichenlose Division}%
\FR{division non signée} \\
\hline \TT{\_\_aullrem} & \RU{остаток от беззнакового деления}\EN{remainder of unsigned division}\DE{Rest (Modulo) einer vorzeichenlosen Division}%
\FR{reste de la division non signée} \\
\hline \TT{\_\_aullshr} & \RU{беззнаковый сдвиг вправо}\EN{unsigned shift right}\DE{Schiebe rechts, vorzeichenlos}%
\FR{décalage non signé à droite} \\
\hline
\end{tabular}
\end{center}

\RU{Процедуры умножения и сдвига влево, одни и те же и для знаковых чисел, и для беззнаковых,
поэтому здесь только одна функция для каждой операции}
\EN{Multiplication and shift left procedures are the same for both signed and unsigned numbers, hence there is only one function 
for each operation here}
\DE{Multiplikation und Links-Schiebebefehle sind sowohl für vorzeichenbehaftete als auch vorzeichenlose Zahlen,
da hier für jede Operation nur ein Befehl existiert}
\FR{La multiplication et le décalage à gauche sont similaire pour les nombres signés
et non signés, donc il n'y a qu'une seule fonction ici}. \\
\\
\RU{Исходные коды этих функций можно найти в установленной \ac{MSVS}, в}\EN{The source code of these function
can be found in the installed \ac{MSVS}, in}%
\DE{Der Quellcode dieser Funktionen kann im Pfad des installierten \ac{MSVS}, gefunden werden: }%
\FR{Le code source des ces fonctions peut être trouvé dans l'installation de \ac{MSVS},
dans} \TT{VC/crt/src/intel/*.asm}.


\mysection{Cheatsheets}

% sections
\subsection{IDA}
\myindex{IDA}
\label{sec:IDA_cheatsheet}

\ShortHotKeyCheatsheet:

\begin{center}
\begin{tabular}{ | l | l | }
\hline
\HeaderColor \RU{клавиша}\EN{key}\DE{Taste}\FR{touche} & \HeaderColor \RU{значение}\EN{meaning}\DE{Bedeutung}\FR{signification} \\
\hline
Space 	& \RU{переключать между листингом и просмотром кода в виде графа}
            \EN{switch listing and graph view}
            \DE{Zwischen Quellcode und grafischer Ansicht wechseln}%
				\FR{échanger le listing et le mode graphique} \\
C 	& \RU{конвертировать в код}\EN{convert to code}\DE{zu Code konvertieren}%
		\FR{convertir en code} \\
D 	& \RU{конвертировать в данные}\EN{convert to data}\DE{zu Daten konvertieren}%
		\FR{convertir en données} \\
A 	& \RU{конвертировать в строку}\EN{convert to string}\DE{zu Zeichenkette konvertieren}%
		\FR{convertir en chaîne} \\
* 	& \RU{конвертировать в массив}\EN{convert to array}\DE{zu Array konvertieren}%
		\FR{convertir en tableau} \\
U 	& \RU{сделать неопределенным}\EN{undefine}\DE{undefinieren}%
		\FR{rendre indéfini}\\
O 	& \RU{сделать смещение из операнда}\EN{make offset of operand}\DE{Offset von Operanden}%
		\FR{donner l'offset d'une opérande}\\
H 	& \RU{сделать десятичное число}\EN{make decimal number}\DE{Dezimalzahl erstellen}%
		\FR{transformer en nombre décimal} \\
R 	& \RU{сделать символ}\EN{make char}\DE{Zeichen erstellen}%
		\FR{transformer en caractère} \\
B 	& \RU{сделать двоичное число}\EN{make binary number}\DE{Binärzahl erstellen}%
		\FR{transformer en nombre binaire} \\
Q 	& \RU{сделать шестнадцатеричное число}\EN{make hexadecimal number}\DE{Hexadezimalzahl erstellen}%
		\FR{transformer en nombre hexa-décimal} \\
N 	& \RU{переименовать идентификатор}\EN{rename identifier}\DE{Identifikator umbenennen}%
		\FR{renommer l'identifiant} \\
? 	& \RU{калькулятор}\EN{calculator}\DE{Rechner}\FR{calculatrice} \\
G 	& \RU{переход на адрес}\EN{jump to address}\DE{zu Adresse springen}%
		\FR{sauter à l'adresse} \\
: 	& \RU{добавить комментарий}\EN{add comment}\DE{Kommentar einfügen}\FR{ajouter un commentaire} \\
Ctrl-X 	& \RU{показать ссылки на текущую функцию, метку, переменную}%
		\EN{show references to the current function, label, variable }%
		\DE{Referenz zu aktueller Funktion, Variable, ... zeigen}%
		\FR{montrer les références à la fonction, au label, à la variable courant} \\
	& \RU{(в т.ч., в стеке)}\EN{(incl. in local stack)}\DE{(inkl. lokalem Stack)}%
		\FR{inclure dans la pile locale} \\
X 	& \RU{показать ссылки на функцию, метку, переменную, итд}\EN{show references to the function, label, variable, etc.}%
		\DE{Referenz zu Funktion, Variable, ... zeigen}%
		\FR{montrer les références à la fonction, au label, à la variable, etc.} \\
Alt-I 	& \RU{искать константу}\EN{search for constant}\DE{Konstante suchen}%
		\FR{chercher une constante} \\
Ctrl-I 	& \RU{искать следующее вхождение константы}\EN{search for the next occurrence of constant}\DE{Nächstes Auftreten der Konstante suchen}%
		\FR{chercher la prochaine occurrence d'une constante} \\
Alt-B 	& \RU{искать последовательность байт}\EN{search for byte sequence}\DE{Byte-Sequenz suchen}%
		\FR{chercher une séquence d'octets} \\
Ctrl-B 	& \RU{искать следующее вхождение последовательности байт}
		\EN{search for the next occurrence of byte sequence}
		\DE{Nächstes Auftreten der Byte-Sequenz suchen}%
		\FR{chercher l'occurrence suivante d'une séquence d'octets} \\
Alt-T 	& \RU{искать текст (включая инструкции, итд.)}%
		\EN{search for text (including instructions, etc.)}%
		\DE{Text suchen (inkl. Anweisungen, usw.)}%
		\FR{chercher du texte (instructions incluses, etc.)} \\
Ctrl-T 	& \RU{искать следующее вхождение текста}%
		\EN{search for the next occurrence of text}%
		\DE{nächstes Aufreten des Textes suchen}%
		\FR{chercher l'occurrence suivante du texte} \\
Alt-P 	& \RU{редактировать текущую функцию}%
		\EN{edit current function}%
		\DE{akutelle Funktion editieren}%
		\FR{éditer la fonction courante} \\
Enter 	& \RU{перейти к функции, переменной, итд.}%
		\EN{jump to function, variable, etc.}%
		\DE{zu Funktion, Variable, ... springen}%
		\FR{sauter à la fonction, la variable, etc.} \\
Esc 	& \RU{вернуться назад}\EN{get back}\DE{zurückgehen}%
		\FR{retourner en arrière} \\
Num -   & \RU{свернуть функцию или отмеченную область}%
		\EN{fold function or selected area}%
		\DE{Funktion oder markierten Bereich einklappen}%
		\FR{cacher/plier la fonction ou la partie sélectionnée} \\
Num + 	& \RU{снова показать функцию или область}%
		\EN{unhide function or area}%
		\DE{Funktion oder Bereich anzeigen}%
		\FR{afficher la fonction ou une partie} \\
\hline
\end{tabular}
\end{center}

\RU{Сворачивание функции или области может быть удобно чтобы прятать те части функции,
чья функция вам стала уже ясна}%
\EN{Function/area folding may be useful for hiding function parts when you realize what they do}%
\DE{Das Einklappen ist nützlich um Teile von Funktionen zu verstecken, wenn bekannt ist was sie tun}%
\FR{cacher une fonction ou une partie de code peut être utile pour cacher des parties du
code lorsque vous avez compris ce qu'elles font}.
\RU{это используется в моем скрипте\footnote{\href{\YurichevIDAIDCScripts}{GitHub}}}\EN{this is used in my}\DE{dies wird genutzt im}%
\RU{для сворачивания некоторых очень часто используемых фрагментов inline-кода}%
\EN{script\footnote{\href{\YurichevIDAIDCScripts}{GitHub}} for hiding some often used patterns of inline code}%
\DE{Script\footnote{\href{\YurichevIDAIDCScripts}{GitHub}} um häufig genutzte Inline-Code-Stellen zu verstecken}%
\FR{ceci est utilisé dans mon script\footnote{\href{\YurichevIDAIDCScripts}{GitHub}}%
pour cacher des patterns de code inline souvent utilisés}.


\subsection{\olly}
\myindex{\olly}
\label{sec:Olly_cheatsheet}

\ShortHotKeyCheatsheet:

\begin{center}
\begin{tabular}{ | l | l | }
\hline
\HeaderColor \RU{хот-кей}\EN{hot-key}\DE{Tastenkürzel}\FR{raccourci} & 
\HeaderColor \RU{значение}\EN{meaning}\DE{Bedeutung}\FR{signification} \\
\hline
F7	& \RU{трассировать внутрь}\EN{trace into}\DE{Schritt}\FR{tracer dans la fonction}\\
F8	& \stepover\\
F9	& \RU{запуск}\EN{run}\DE{starten}\FR{démarrer}\\
Ctrl-F2	& \RU{перезапуск}\EN{restart}\DE{Neustart}\FR{redémarrer}\\
\hline
\end{tabular}
\end{center}

\subsection{MSVC}
\myindex{MSVC}
\label{sec:MSVC_options}

\RU{Некоторые полезные опции, которые были использованы в книге}
\EN{Some useful options which were used through this book}.
\DE{Einige nützliche Optionen die in diesem Buch genutzt werden}.
\FR{Quelques options utiles qui ont été utilisées dans ce livre}

\begin{center}
\begin{tabular}{ | l | l | }
\hline
\HeaderColor \RU{опция}\EN{option}\DE{Option}\FR{option} & 
\HeaderColor \RU{значение}\EN{meaning}\DE{Bedeutung}\FR{signification} \\
\hline
/O1		& \RU{оптимизация по размеру кода}\EN{minimize space}\DE{Speicherplatz minimieren}%
\FR{minimiser l'espace}\\
/Ob0		& \RU{не заменять вызовы inline-функций их кодом}\EN{no inline expansion}\DE{Keine Inline-Erweiterung}%
\FR{pas de mire en ligne}\\
/Ox		& \RU{максимальная оптимизация}\EN{maximum optimizations}\DE{maximale Optimierung}%
\FR{optimisation maximale}\\
/GS-		& \RU{отключить проверки переполнений буфера}
		\EN{disable security checks (buffer overflows)}
        \DE{Sicherheitsüberprüfungen deaktivieren (Buffer Overflows)}%
		\FR{désactiver les vérifications de sécurité (buffer overflows)}\\
/Fa(file)	& \RU{генерировать листинг на ассемблере}\EN{generate assembly listing}\DE{Assembler-Quelltext erstellen}%
\FR{générer un listing assembleur}\\
/Zi		& \RU{генерировать отладочную информацию}\EN{enable debugging information}\DE{Debugging-Informationen erstellen}%
\FR{activer les informations de débogage}\\
/Zp(n)		& \RU{паковать структуры по границе в $n$ байт}\EN{pack structs on $n$-byte boundary}\DE{Strukturen an $n$-Byte-Grenze ausrichten}%
\FR{aligner les structures sur une limite de $n$-octet}\\
/MD		& \RU{выходной исполняемый файл будет использовать}
			\EN{produced executable will use}%
            \DE{ausführbare Daten nutzt}%
\FR{l'exécutable généré utilisera} \TT{MSVCR*.DLL}\\
\hline
\end{tabular}
\end{center}

\RU{Кое-как информация о версиях MSVC}\EN{Some information about MSVC versions}\DE{Informationen zu MSVC-Versionen}%
\FR{Quelques informations sur les versions de MSVC}:
\myref{MSVC_versions}.


\EN{\subsection{GCC}
\myindex{GCC}

Some useful options which were used through this book.

\begin{center}
\begin{tabular}{ | l | l | }
\hline
\HeaderColor option & 
\HeaderColor meaning \\
\hline
-Os		& code size optimization \\
-O3		& maximum optimization \\
-regparm=	& how many arguments are to be passed in registers \\
-o file		& set name of output file \\
-g		& produce debugging information in resulting executable \\
-S		& generate assembly listing file \\
-masm=intel	& produce listing in Intel syntax \\
-fno-inline	& do not inline functions \\
\hline
\end{tabular}
\end{center}


}
\RU{\myparagraph{GCC 4.4.1}

\lstinputlisting[caption=GCC 4.4.1,style=customasmx86]{patterns/12_FPU/3_comparison/x86/GCC_RU.asm}

\myindex{x86!\Instructions!FUCOMPP}
\FUCOMPP~--- это почти то же что и \FCOM, только выкидывает из стека оба значения после сравнения, 
а также несколько иначе реагирует на \q{не-числа}.

\myindex{Не-числа (NaNs)}
Немного о \emph{не-числах}.

FPU умеет работать со специальными переменными, которые числами не являются и называются \q{не числа} или 
\gls{NaN}.
Это бесконечность, результат деления на ноль, и так далее. Нечисла бывают \q{тихие} и \q{сигнализирующие}. 
С первыми можно продолжать работать и далее, а вот если вы попытаетесь совершить какую-то операцию 
с сигнализирующим нечислом, то сработает исключение.

\myindex{x86!\Instructions!FCOM}
\myindex{x86!\Instructions!FUCOM}
Так вот, \FCOM вызовет исключение если любой из операндов какое-либо нечисло.
\FUCOM же вызовет исключение только если один из операндов именно \q{сигнализирующее нечисло}.

\myindex{x86!\Instructions!SAHF}
\label{SAHF}
Далее мы видим \SAHF (\emph{Store AH into Flags})~--- это довольно редкая инструкция в коде, не использующим FPU. 
8 бит из \AH перекладываются в младшие 8 бит регистра статуса процессора в таком порядке:

\input{SAHF_LAHF}

\myindex{x86!\Instructions!FNSTSW}
Вспомним, что \FNSTSW перегружает интересующие нас биты \CThreeBits в \AH, 
и соответственно они будут в позициях 6, 2, 0 в регистре \AH:

\input{C3_in_AH}

Иными словами, пара инструкций \INS{fnstsw  ax / sahf} перекладывает биты \CThreeBits в флаги \ZF, \PF, \CF.

Теперь снова вспомним, какие значения бит \CThreeBits будут при каких результатах сравнения:

\begin{itemize}
\item Если $a$ больше $b$ в нашем случае, то биты \CThreeBits должны быть выставлены так: 0, 0, 0.
\item Если $a$ меньше $b$, то биты будут выставлены так: 0, 0, 1.
\item Если $a=b$, то так: 1, 0, 0.
\end{itemize}
% TODO: table?

Иными словами, после трех инструкций \FUCOMPP/\FNSTSW/\SAHF возможны такие состояния флагов:

\begin{itemize}
\item Если $a>b$ в нашем случае, то флаги будут выставлены так: \GTT{ZF=0, PF=0, CF=0}.
\item Если $a<b$, то флаги будут выставлены так: \GTT{ZF=0, PF=0, CF=1}.
\item Если $a=b$, то так: \GTT{ZF=1, PF=0, CF=0}.
\end{itemize}
% TODO: table?

\myindex{x86!\Instructions!SETcc}
\myindex{x86!\Instructions!JNBE}
Инструкция \SETNBE выставит в \AL единицу или ноль в зависимости от флагов и условий. 
Это почти аналог \JNBE, за тем лишь исключением, что \SETcc
\footnote{\emph{cc} это \emph{condition code}}
выставляет 1 или 0 в \AL, а \Jcc делает переход или нет. 
\SETNBE запишет 1 только если \GTT{CF=0} и \GTT{ZF=0}. Если это не так, то запишет 0 в \AL.

\CF будет 0 и \ZF будет 0 одновременно только в одном случае: если $a>b$.

Тогда в \AL будет записана 1, последующий условный переход \JZ выполнен не будет 
и функция вернет~\GTT{\_a}. 
В остальных случаях, функция вернет~\GTT{\_b}.
}
\FR{\subsection{GCC}
\myindex{GCC}

Quelques options utiles qui ont été utilisées dans ce livre.

\begin{center}
\begin{tabular}{ | l | l | }
\hline
\HeaderColor option & 
\HeaderColor signification \\
\hline
-Os		& optimiser la taille du code \\
-O3		& optimisation maximale \\
-regparm=	& nombre d'arguments devant être passés dans les registres \\
-o file		& définir le nom du fichier de sortie \\
-g		& mettre l'information de débogage dans l'exécutable généré \\
-S		& générer un fichier assembleur \\
-masm=intel	& construire le code source en syntaxe Intel \\
-fno-inline	& ne pas mettre les fonctions en ligne \\
\hline
\end{tabular}
\end{center}


}
\DE{\myparagraph{GCC 4.4.1}

\lstinputlisting[caption=GCC 4.4.1,style=customasmx86]{patterns/12_FPU/3_comparison/x86/GCC_DE.asm}

\myindex{x86!\Instructions!FUCOMPP}
\FUCOMPP{} ist fast wie like \FCOM, nimmt aber beide Werte vom Stand und
behandelt \q{undefinierte Zahlenwerte} anders.


\myindex{Non-a-numbers (NaNs)}
Ein wenig über \emph{undefinierte Zahlenwerte}.

Die FPU ist in der Lage mit speziellen undefinieten Werten, den sogenannten
\emph{not-a-number}(kurz \gls{NaN}) umzugehen. Beispiele sind etwa der Wert
unendlich, das Ergebnis einer Division durch 0, etc. Undefinierte Werte können
entwder \q{quiet} oder \q{signaling} sein. Es ist möglich mit \q{quiet} NaNs zu
arbeiten, aber beim Versuch einen Befehl auf \q{signaling} NaNs auszuführen,
wird eine Exception geworfen. 

\myindex{x86!\Instructions!FCOM}
\myindex{x86!\Instructions!FUCOM}
\FCOM erzeugt eine Exception, falls irgendein Operand ein \gls{NaN} ist.
\FUCOM erzeugt eine Exception nur dann, wenn ein Operand eine \q{signaling}
\gls{NaN} (SNaN) ist.

\myindex{x86!\Instructions!SAHF}
\label{SAHF}
Der nächste Befehl ist \SAHF (\emph{Store AH into Flags})~---es handelt sich
hierbei um einen seltenen Befehl, der nicht mit der FPU zusammenhängt.
8 Bits aus AH werden in die niederen 8 Bit der CPU Flags in der folgenden
Reihenfolge verschoben:

\input{SAHF_LAHF}

\myindex{x86!\Instructions!FNSTSW}
Erinnern wir uns, dass \FNSTSW die für uns interessanten Bits (\CThreeBits) auf
den Stellen 6,2,0 im AH Register setzt:

\input{C3_in_AH}
Mit anderen Worten: der Befehl \INS{fnstsw ax / sahf} verschiebt \CThreeBits
nach \ZF, \PF und \CF. 

Überlegen wir uns auch die Werte der \CThreeBits in unterschiedlichen Szenarien:

\begin{itemize} 
  \item Falls in unserem Beispiel $a$ größer als $b$ ist, dann werden die
  \CThreeBits auf 0,0,0 gesetzt.
  \item Falls $a$ kleiner als $b$ ist, werden die Bits auf 0,0,1 gesetzt.
  \item Falls $a=b$, dann werden die Bits auf 1,0,0 gesetzt.
\end{itemize}
% TODO: table?
Mit anderen Worten, die folgenden Zustände der CPU Flags sind nach drei
\FUCOMPP/\FNSTSW/\SAHF Befehlen möglich:

\begin{itemize}
\item Falls $a>b$, werden die CPU Flags wie folgt gesetzt \GTT{ZF=0, PF=0,
CF=0}.
\item Falls $a<b$, werden die CPU Flags wie folgt gesetzt: \GTT{ZF=0, PF=0,
CF=1}.
\item Und falls $a=b$, dann gilt: \GTT{ZF=1, PF=0, CF=0}.
\end{itemize}
% TODO: table?

\myindex{x86!\Instructions!SETcc}
\myindex{x86!\Instructions!JNBE}
Abhängig von den CPU Flags und Bedingungen, speichert \SETNBE entweder 1 oder 0
in AL.
Es ist also quasi das Gegenstück von \JNBE mit dem Unterschied, dass \SETcc

Depending on the CPU flags and conditions, \SETNBE stores 1 or 0 to AL. 
It is almost the counterpart of \JNBE, with the exception that \SETcc
\footnote{\emph{cc} is \emph{condition code}} eine 1 oder 0 in \AL speichert, aber
\Jcc tatsächlich auch springt.
\SETNBE speicher 1 nur, falls \GTT{CF=0} und \GTT{ZF=0}.
Wenn dies nicht der Fall ist, dann wird 0 in \AL gespeichert.

Nur in einem Fall sind \CF und \ZF beide 0: falls $a>b$.

In diesem Fall wird 1 in \AL gespeichert, der nachfolgende \JZ Sprung wird nicht
ausgeführt und die Funktion liefert {\_a} zurück. In allen anderen Fällen wird
{\_b} zurückgegeben.
}

\subsection{GDB}
\myindex{GDB}
\label{sec:GDB_cheatsheet}

% FIXME: in Russian table doesn't fit!

\RU{Некоторые команды, которые были использованы в книге}\EN{Some of commands we used in this book}\DE{Einige nützliche Optionen die in diesem Buch genutzt werden}%
\FR{Quelques commandes que nous avons utilisées dans ce livre}:

\small
\begin{center}
\begin{tabular}{ | l | l | }
\hline
\HeaderColor \RU{опция}\EN{option}\DE{Option}\FR{option} & 
\HeaderColor \RU{значение}\EN{meaning}\DE{Bedeutung} \\
\hline
break filename.c:number		& \RU{установить точку останова на номере строки в исходном файле}
					\EN{set a breakpoint on line number in source code}
                    \DE{Setzen eines Breakpoints in der angegebenen Zeile}%
					\FR{mettre un point d'arrêt à la ligne number du code source} \\
break function			& \RU{установить точку останова на функции}\EN{set a breakpoint on function}\DE{Setzen eines Breakpoints in der Funktion}%
\FR{mettre un point d'arrêt sur une fonction} \\
break *address			& \RU{установить точку останова на адресе}\EN{set a breakpoint on address}\DE{Setzen eines Breakpoints auf Adresse}%
\FR{mettre un point d'arrêt à une adresse} \\
b				& \dittoclosing \\
p variable			& \RU{вывести значение переменной}\EN{print value of variable}\DE{Ausgabe eines Variablenwerts}%
\FR{afficher le contenu d'une variable} \\
run				& \RU{запустить}\EN{run}\DE{Starten}\FR{démarrer} \\
r				& \dittoclosing \\
cont				& \RU{продолжить исполнение}\EN{continue execution}\DE{Ausführung fortfahren}\FR{continuer l'exécution} \\
c				& \dittoclosing \\
bt				& \RU{вывести стек}\EN{print stack}\DE{Stack ausgeben}\FR{afficher la pile} \\
set disassembly-flavor intel	& \RU{установить Intel-синтаксис}\EN{set Intel syntax}\DE{Intel-Syntax nutzen}%
\FR{utiliser la syntaxe Intel} \\
disas				& disassemble current function \\
disas function			& \RU{дизассемблировать функцию}\EN{disassemble function}\DE{Funktion disassemblieren}\FR{désassembler la fonction} \\
disas function,+50		& disassemble portion \\
disas \$eip,+0x10		& \dittoclosing \\
disas/r				& \EN{disassemble with opcodes}\RU{дизассемблировать с опкодами}\DE{mit OpCodes disassemblieren}%
\FR{désassembler avec les opcodes} \\
info registers			& \RU{вывести все регистры}\EN{print all registers}\DE{Ausgabe aller Register}\FR{afficher tous les registres} \\
info float			& \RU{вывести FPU-регистры}\EN{print FPU-registers}\DE{Ausgabe der FPU-Register}\FR{afficher les registres FPU} \\
info locals			& \RU{вывести локальные переменные (если известны)}\EN{dump local variables (if known)}\DE{(bekannte) lokale Variablen ausgeben}%
\FR{afficher les variables locales} \\
x/w ...				& \RU{вывести память как 32-битные слова}\EN{dump memory as 32-bit word}\DE{Speicher als 32-Bit-Wort ausgeben}%
\FR{afficher la mémoire en mot de 32-bit} \\
x/w \$rdi			& \RU{вывести память как 32-битные слова}\EN{dump memory as 32-bit word}\DE{Speicher als 32-Bit-Wort ausgeben}%
\FR{afficher la mémoire en mot de 32-bit} \\
				& \RU{по адресу в \TT{RDI}}\EN{at address in \TT{RDI}}\DE{an Adresse in \TT{RDI}}\FR{à l'adresse dans \TT{RDI}} \\

x/10w ...			& \RU{вывести 10 слов памяти}\EN{dump 10 memory words}\DE{10 Speicherworte ausgeben}%
\FR{afficher 10 mots de la mémoire} \\
x/s ...				& \RU{вывести строку из памяти}\EN{dump memory as string}\DE{Speicher als Zeichenkette ausgeben}%
\FR{afficher la mémoire en tant que chaîne} \\
x/i ...				& \RU{трактовать память как код}\EN{dump memory as code}\DE{Speicher als Code ausgeben}%
\FR{afficher la mémoire en tant que code} \\
x/10c ...			& \RU{вывести 10 символов}\EN{dump 10 characters}\DE{10 Zeichen ausgeben}%
\FR{afficher 10 caractères} \\
x/b ...				& \RU{вывести байты}\EN{dump bytes}\DE{Bytes ausgeben}\FR{afficher des octets} \\
x/h ...				& \RU{вывести 16-битные полуслова}\EN{dump 16-bit halfwords}\DE{16-Bit-Halbworte ausgeben}%
\FR{afficher en demi-mots de 16-bit} \\
x/g ...				& \RU{вывести 64-битные слова}\EN{dump giant (64-bit) words}\DE{große (64-Bit-) Worte ausgeben}%
\FR{afficher des mots géants (64-bit)} \\
finish				& \RU{исполнять до конца функции}\EN{execute till the end of function}\DE{bis Funktionsende fortfahren}%
\FR{exécuter jusqu'à la fin de la fonction} \\
next				& \RU{следующая инструкция (не заходить в функции)}
					\EN{next instruction (don't dive into functions)}
					\DE{Nächste Anweisung (nicht in Funktion springen)}
					\FR{instruction suivante (ne pas descendre dans les fonctions)} \\
step				& \RU{следующая инструкция (заходить в функции)}
					\EN{next instruction (dive into functions)}
					\DE{Nächste Anweisung (in Funktion springen)}
					\FR{instruction suivante (descendre dans les fonctions)} \\
set step-mode on		& \RU{не использовать информацию о номерах строк при использовании команды step}
					\EN{do not use line number information while stepping}
					\DE{Beim schrittweisen Ausführen keine Zeilennummerninfos nutzen}
					\FR{ne pas utiliser l'information du numéro de ligne en exécutant pas à pas} \\
frame n				& \RU{переключить фрейм стека}\EN{switch stack frame}\DE{Stack-Frame tauschen}\FR{échanger la stack frame} \\
info break			& \RU{список точек останова}\EN{list of breakpoints}\DE{Breakpoints schauen}%
\FR{afficher les points d'arrêt} \\
del n				& \RU{удалить точку останова}\EN{delete breakpoint}\DE{Breakpoints löschen}\FR{effacer un point d'arrêt} \\
set args ...			& \RU{установить аргументы командной строки}\EN{set command-line arguments}\DE{Aufrufparameter setzen}%
\FR{définir les arguments de la ligne de commande} \\
\hline
\end{tabular}
\end{center}
\normalsize



}
\IT{\part*{\RU{Приложение}\EN{Appendix}\DE{Anhang}\FR{Appendice}\IT{Appendice}}
\appendix
\addcontentsline{toc}{part}{\RU{Приложение}\EN{Appendix}\DE{Anhang}\FR{Appendice}\IT{Appendice}}

% chapters
\EN{\mysection{Task manager practical joke (Windows Vista)}
\myindex{Windows!Windows Vista}

Let's see if it's possible to hack Task Manager slightly so it would detect more \ac{CPU} cores.

\myindex{Windows!NTAPI}

Let us first think, how does the Task Manager know the number of cores?

There is the \TT{GetSystemInfo()} win32 function present in win32 userspace which can tell us this.
But it's not imported in \TT{taskmgr.exe}.

There is, however, another one in \gls{NTAPI}, \TT{NtQuerySystemInformation()}, 
which is used in \TT{taskmgr.exe} in several places.

To get the number of cores, one has to call this function with the \TT{SystemBasicInformation} constant
as a first argument (which is zero
\footnote{\href{http://msdn.microsoft.com/en-us/library/windows/desktop/ms724509(v=vs.85).aspx}{MSDN}}).

The second argument has to point to the buffer which is getting all the information.

So we have to find all calls to the \\
\TT{NtQuerySystemInformation(0, ?, ?, ?)} function.
Let's open \TT{taskmgr.exe} in IDA. 
\myindex{Windows!PDB}

What is always good about Microsoft executables is that IDA can download the corresponding \gls{PDB} 
file for this executable and show all function names.

It is visible that Task Manager is written in \Cpp and some of the function names and classes are really 
speaking for themselves.
There are classes CAdapter, CNetPage, CPerfPage, CProcInfo, CProcPage, CSvcPage, 
CTaskPage, CUserPage.

Apparently, each class corresponds to each tab in Task Manager.

Let's visit each call and add comment with the value which is passed as the first function argument.
We will write \q{not zero} at some places, because the value there was clearly not zero, 
but something really different (more about this in the second part of this chapter).

And we are looking for zero passed as argument, after all.

\begin{figure}[H]
\centering
\myincludegraphics{examples/taskmgr/IDA_xrefs.png}
\caption{IDA: cross references to NtQuerySystemInformation()}
\end{figure}

Yes, the names are really speaking for themselves.

When we closely investigate each place where\\
\TT{NtQuerySystemInformation(0, ?, ?, ?)} is called,
we quickly find what we need in the \TT{InitPerfInfo()} function:

\lstinputlisting[caption=taskmgr.exe (Windows Vista),style=customasmx86]{examples/taskmgr/taskmgr.lst}

\TT{g\_cProcessors} is a global variable, and this name has been assigned by 
IDA according to the \gls{PDB} loaded from Microsoft's symbol server.

The byte is taken from \TT{var\_C20}. 
And \TT{var\_C58} is passed to\\
\TT{NtQuerySystemInformation()} 
as a pointer to the receiving buffer.
The difference between 0xC20 and 0xC58 is 0x38 (56).

Let's take a look at format of the return structure, which we can find in MSDN:

\begin{lstlisting}[style=customc]
typedef struct _SYSTEM_BASIC_INFORMATION {
    BYTE Reserved1[24];
    PVOID Reserved2[4];
    CCHAR NumberOfProcessors;
} SYSTEM_BASIC_INFORMATION;
\end{lstlisting}

This is a x64 system, so each PVOID takes 8 bytes.

All \emph{reserved} fields in the structure take $24+4*8=56$ bytes.

Oh yes, this implies that \TT{var\_C20} is the local stack is exactly the
\TT{NumberOfProcessors} field of the \TT{SYSTEM\_BASIC\_INFORMATION} structure.

Let's check our guess.
Copy \TT{taskmgr.exe} from \TT{C:\textbackslash{}Windows\textbackslash{}System32} 
to some other folder 
(so the \emph{Windows Resource Protection} 
will not try to restore the patched \TT{taskmgr.exe}).

Let's open it in Hiew and find the place:

\begin{figure}[H]
\centering
\myincludegraphics{examples/taskmgr/hiew2.png}
\caption{Hiew: find the place to be patched}
\end{figure}

Let's replace the \TT{MOVZX} instruction with ours.
Let's pretend we've got 64 CPU cores.

Add one additional \ac{NOP} (because our instruction is shorter than the original one):

\begin{figure}[H]
\centering
\myincludegraphics{examples/taskmgr/hiew1.png}
\caption{Hiew: patch it}
\end{figure}

And it works!
Of course, the data in the graphs is not correct.

At times, Task Manager even shows an overall CPU load of more than 100\%.

\begin{figure}[H]
\centering
\myincludegraphics{examples/taskmgr/taskmgr_64cpu_crop.png}
\caption{Fooled Windows Task Manager}
\end{figure}

The biggest number Task Manager does not crash with is 64.

Apparently, Task Manager in Windows Vista was not tested on computers with a large number of cores.

So there are probably some static data structure(s) inside it limited to 64 cores.

\subsection{Using LEA to load values}
\label{TaskMgr_LEA}

Sometimes, \TT{LEA} is used in \TT{taskmgr.exe} instead of \TT{MOV} to set the first argument of \\
\TT{NtQuerySystemInformation()}:

\lstinputlisting[caption=taskmgr.exe (Windows Vista),style=customasmx86]{examples/taskmgr/taskmgr2.lst}

\myindex{x86!\Instructions!LEA}

Perhaps \ac{MSVC} did so because machine code of \INS{LEA} is shorter than \INS{MOV REG, 5} (would be 5 instead of 4).

\INS{LEA} with offset in $-128..127$ range (offset will occupy 1 byte in opcode) with 32-bit registers is even shorter (for lack of REX prefix)---3 bytes.

Another example of such thing is: \myref{using_MOV_and_pack_of_LEA_to_load_values}.
}
\RU{\subsection{Обменять входные значения друг с другом}

Вот так:

\lstinputlisting[style=customc]{patterns/061_pointers/swap/5_RU.c}

Как видим, байты загружаются в младшие 8-битные части регистров \TT{ECX} и \TT{EBX} используя \INS{MOVZX}
(так что старшие части регистров очищаются), затем байты записываются назад в другом порядке.

\lstinputlisting[style=customasmx86,caption=Optimizing GCC 5.4]{patterns/061_pointers/swap/5_GCC_O3_x86.s}

Адреса обоих байтов берутся из аргументов и во время исполнения ф-ции находятся в регистрах \TT{EDX} и \TT{EAX}.

Так что исопльзуем указатели --- вероятно, без них нет способа решить эту задачу лучше.

}
\DE{\mysection{x86}

\subsection{Terminologie}

Geläufig für 16-Bit (8086/80286), 32-Bit (80386, etc.), 64-Bit.

\myindex{IEEE 754}
\myindex{MS-DOS}
\begin{description}
	\item[Byte] 8-Bit.
		Die DB Assembler-Direktive wird zum Definieren von Variablen und Arrays genutzt.
		Bytes werden in dem 8-Bit-Teil der folgenden Register übergeben:
		\TT{AL/BL/CL/DL/AH/BH/CH/DH/SIL/DIL/R*L}.
	\item[Wort] 16-Bit.
		DW Assembler-Direktive \dittoclosing.
		Bytes werden in dem 16-Bit-Teil der folgenden Register übergeben:
			\TT{AX/BX/CX/DX/SI/DI/R*W}.
	\item[Doppelwort] (\q{dword}) 32-Bit.
		DD Assembler-Direktive \dittoclosing.
		Doppelwörter werden in Registern (x86) oder dem 32-Bit-Teil der Register (x64) übergeben.
		In 16-Bit-Code werden Doppelwörter in 16-Bit-Registerpaaren übergeben.
	\item[zwei Doppelwörter] (\q{qword}) 64-Bit.
		DQ Assembler-Direktive \dittoclosing.
		In 32-Bit-Umgebungen werden diese in 32-Bit-Registerpaaren übergeben.
	\item[tbyte] (10 Byte) 80-Bit oder 10 Bytes (für IEEE 754 FPU Register).
	\item[paragraph] (16 Byte) --- Bezeichnung war in MS-DOS Umgebungen gebräuchlich.
\end{description}

\myindex{Windows!API}

Datentypen der selben Breite (BYTE, WORD, DWORD) entsprechen auch denen in der Windows \ac{API}.

% TODO German Translation (DSiekmeier)
%\input{appendix/x86/registers} % subsection
%\input{appendix/x86/instructions} % subsection
\subsection{npad}
\label{sec:npad}

\RU{Это макрос в ассемблере, для выравнивания некоторой метки по некоторой границе.}
\EN{It is an assembly language macro for aligning labels on a specific boundary.}
\DE{Dies ist ein Assembler-Makro um Labels an bestimmten Grenzen auszurichten.}
\FR{C'est une macro du langage d'assemblage pour aligner les labels sur une limite
spécifique.}

\RU{Это нужно для тех \emph{нагруженных} меток, куда чаще всего передается управление, например, 
начало тела цикла. 
Для того чтобы процессор мог эффективнее вытягивать данные или код из памяти, через шину с памятью, 
кэширование, итд.}
\EN{That's often needed for the busy labels to where the control flow is often passed, e.g., loop body starts.
So the CPU can load the data or code from the memory effectively, through the memory bus, cache lines, etc.}
\DE{Dies ist oft nützlich Labels, die oft Ziel einer Kotrollstruktur sind, wie Schleifenköpfe.
Somit kann die CPU Daten oder Code sehr effizient vom Speicher durch den Bus, den Cache, usw. laden.}
\FR{C'est souvent nécessaire pour des labels très utilisés, comme par exemple le
début d'un corps de boucle. Ainsi, le CPU peut charger les données ou le code depuis
la mémoire efficacement, à travers le bus mémoire, les caches, etc.}

\RU{Взято из}\EN{Taken from}\DE{Entnommen von}\FR{Pris de} \TT{listing.inc} (MSVC):

\myindex{x86!\Instructions!NOP}
\RU{Это, кстати, любопытный пример различных вариантов \NOP{}-ов. 
Все эти инструкции не дают никакого эффекта, но отличаются разной длиной.}
\EN{By the way, it is a curious example of the different \NOP variations.
All these instructions have no effects whatsoever, but have a different size.}
\DE{Dies ist übrigens ein Beispiel für die unterschiedlichen \NOP-Variationen.
Keine dieser Anweisungen hat eine Auswirkung, aber alle haben eine unterschiedliche Größe.}
\FR{À propos, c'est un exemple curieux des différentes variations de \NOP. Toutes
ces instructions n'ont pas d'effet, mais ont une taille différente.}

\RU{Цель в том, чтобы была только одна инструкция, а не набор NOP-ов, 
считается что так лучше для производительности CPU.}
\EN{Having a single idle instruction instead of couple of NOP-s,
is accepted to be better for CPU performance.}
\DE{Eine einzelne Idle-Anweisung anstatt mehrerer NOPs hat positive Auswirkungen
auf die CPU-Performance.}
\FR{Avoir une seule instruction sans effet au lieu de plusieurs est accepté comme
étant meilleur pour la performance du CPU.}

\begin{lstlisting}[style=customasmx86]
;; LISTING.INC
;;
;; This file contains assembler macros and is included by the files created
;; with the -FA compiler switch to be assembled by MASM (Microsoft Macro
;; Assembler).
;;
;; Copyright (c) 1993-2003, Microsoft Corporation. All rights reserved.

;; non destructive nops
npad macro size
if size eq 1
  nop
else
 if size eq 2
   mov edi, edi
 else
  if size eq 3
    ; lea ecx, [ecx+00]
    DB 8DH, 49H, 00H
  else
   if size eq 4
     ; lea esp, [esp+00]
     DB 8DH, 64H, 24H, 00H
   else
    if size eq 5
      add eax, DWORD PTR 0
    else
     if size eq 6
       ; lea ebx, [ebx+00000000]
       DB 8DH, 9BH, 00H, 00H, 00H, 00H
     else
      if size eq 7
	; lea esp, [esp+00000000]
	DB 8DH, 0A4H, 24H, 00H, 00H, 00H, 00H 
      else
       if size eq 8
        ; jmp .+8; .npad 6
	DB 0EBH, 06H, 8DH, 9BH, 00H, 00H, 00H, 00H
       else
        if size eq 9
         ; jmp .+9; .npad 7
         DB 0EBH, 07H, 8DH, 0A4H, 24H, 00H, 00H, 00H, 00H
        else
         if size eq 10
          ; jmp .+A; .npad 7; .npad 1
          DB 0EBH, 08H, 8DH, 0A4H, 24H, 00H, 00H, 00H, 00H, 90H
         else
          if size eq 11
           ; jmp .+B; .npad 7; .npad 2
           DB 0EBH, 09H, 8DH, 0A4H, 24H, 00H, 00H, 00H, 00H, 8BH, 0FFH
          else
           if size eq 12
            ; jmp .+C; .npad 7; .npad 3
            DB 0EBH, 0AH, 8DH, 0A4H, 24H, 00H, 00H, 00H, 00H, 8DH, 49H, 00H
           else
            if size eq 13
             ; jmp .+D; .npad 7; .npad 4
             DB 0EBH, 0BH, 8DH, 0A4H, 24H, 00H, 00H, 00H, 00H, 8DH, 64H, 24H, 00H
            else
             if size eq 14
              ; jmp .+E; .npad 7; .npad 5
              DB 0EBH, 0CH, 8DH, 0A4H, 24H, 00H, 00H, 00H, 00H, 05H, 00H, 00H, 00H, 00H
             else
              if size eq 15
               ; jmp .+F; .npad 7; .npad 6
               DB 0EBH, 0DH, 8DH, 0A4H, 24H, 00H, 00H, 00H, 00H, 8DH, 9BH, 00H, 00H, 00H, 00H
              else
	       %out error: unsupported npad size
               .err
              endif
             endif
            endif
           endif
          endif
         endif
        endif
       endif
      endif
     endif
    endif
   endif
  endif
 endif
endif
endm
\end{lstlisting}
 % subsection
}
\FR{\subsection{Exemple \#2: SCO OpenServer}

\label{examples_SCO}
\myindex{SCO OpenServer}
Un ancien logiciel pour SCO OpenServer de 1997 développé par une société qui a disparue
depuis longtemps.

Il y a un driver de dongle special à installer dans le système, qui contient les
chaînes de texte suivantes:
\q{Copyright 1989, Rainbow Technologies, Inc., Irvine, CA}
et
\q{Sentinel Integrated Driver Ver. 3.0 }.

Après l'installation du driver dans SCO OpenServer, ces fichiers apparaissent dans
l'arborescence /dev:

\begin{lstlisting}
/dev/rbsl8
/dev/rbsl9
/dev/rbsl10
\end{lstlisting}

Le programme renvoie une erreur lorsque le dongle n'est pas connecté, mais le message
d'erreur n'est pas trouvé dans les exécutables.

\myindex{COFF}

Grâce à \ac{IDA}, il est facile de charger l'exécutable COFF utilisé dans SCO OpenServer.

Essayons de trouver la chaîne \q{rbsl} et en effet, elle se trouve dans ce morceau
de code:

\lstinputlisting[style=customasmx86]{examples/dongles/2/1.lst}

Oui, en effet, le programme doit communiquer d'une façon ou d'une autre avec le driver.

\myindex{thunk-functions}
Le seul endroit où la fonction \TT{SSQC()} est appelée est dans la \glslink{thunk
 function}{fonction thunk}:

\lstinputlisting[style=customasmx86]{examples/dongles/2/2.lst}

SSQ() peut être appelé depuis au moins 2 fonctions.

L'une d'entre elles est:

\lstinputlisting[style=customasmx86]{examples/dongles/2/check1_EN.lst}

\q{\TT{3C}} et \q{\TT{3E}} semblent familiers: il y avait un dongle Sentinel Pro de
Rainbow sans mémoire, fournissant seulement une fonction de crypto-hachage secrète.

Vous pouvez lire une courte description de la fonction de hachage dont il s'agit
ici: \myref{hash_func}.

Mais retournons au programme.

Donc le programme peut seulement tester si un dongle est connecté ou s'il est absent.

Aucune autre information ne peut être écrite dans un tel dongle, puisqu'il n'a pas
de mémoire.
Les codes sur deux caractères sont des commandes (nous pouvons voir comment les commandes
sont traitées dans la fonction \TT{SSQC()}) et toutes les autres chaînes sont hachées
dans le dongle, transformées en un nombre 16-bit.
L'algorithme était secret, donc il n'était pas possible d'écrire un driver de remplacement
ou de refaire un dongle matériel qui l'émulerait parfaitement.

Toutefois, il est toujours possible d'intercepter tous les accès au dongle et de
trouver les constantes auxquelles les résultats de la fonction de hachage sont comparées.

Mais nous devons dire qu'il est possible de construire un schéma de logiciel de protection
de copie robuste basé sur une fonction secrète de hachage cryptographique: il suffit
qu'elle chiffre/déchiffre les fichiers de données utilisés par votre logiciel.

Mais retournons au code:

Les codes 51/52/53 sont utilisés pour choisir le port imprimante LPT.
3x/4x sont utilisés pour le choix de la \q{famille} (c'est ainsi que les dongles
Sentinel Pro sont différenciés les uns des autres: plus d'un dongle peut être connecté
sur un port LPT).

La seule chaîne passée à la fonction qui ne fasse pas 2 caractères est "0123456789".

Ensuite, le résultat est comparé à l'ensemble des résultats valides.

Si il est correct, 0xC ou 0xB est écrit dans la variable globale \TT{ctl\_model}.%

Une autre chaîne de texte qui est passée est
"PRESS ANY KEY TO CONTINUE: ", mais le résultat n'est pas testé.
Difficile de dire pourquoi, probablement une erreur\footnote{C'est un sentiment
étrange de trouver un bug dans un logiciel aussi ancien.}.

Voyons où la valeur de la variable globale \TT{ctl\_model} est utilisée.

Un tel endroit est:

\lstinputlisting[style=customasmx86]{examples/dongles/2/4.lst}

Si c'est 0, un message d'erreur chiffré est passé à une routine de déchiffrement
et affiché.

\myindex{x86!\Instructions!XOR}

La routine de déchiffrement de la chaîne semble être un simple \glslink{xoring}{xor}:

\lstinputlisting[style=customasmx86]{examples/dongles/2/err_warn.lst}

C'est pourquoi nous étions incapable de trouver le message d'erreur dans les fichiers
exécutable, car ils sont chiffrés (ce qui est une pratique courante).

Un autre appel à la fonction de hachage \TT{SSQ()} lui passe la chaîne \q{offln}
et le résultat est comparé avec \TT{0xFE81} et \TT{0x12A9}.

Si ils ne correspondent pas, ça se comporte comme une sorte de fonction \TT{timer()}
(peut-être en attente qu'un dongle mal connecté soit reconnecté et re-testé?) et ensuite
déchiffre un autre message d'erreur à afficher.

\lstinputlisting[style=customasmx86]{examples/dongles/2/check2_EN.lst}

Passer outre le dongle est assez facile: il suffit de patcher tous les sauts après
les instructions \CMP pertinentes.

Une autre option est d'écrire notre propre driver SCO OpenServer, contenant une table
de questions et de réponses, toutes celles qui sont présentent dans le programme.

\subsubsection{Déchiffrer les messages d'erreur}

À propos, nous pouvons aussi essayer de déchiffrer tous les messages d'erreurs.
L'algorithme qui se trouve dans la fonction \TT{err\_warn()} est très simple, en effet:

\lstinputlisting[caption=Decryption function,style=customasmx86]{examples/dongles/2/decrypting_FR.lst}

Comme on le voit, non seulement la chaîne est transmise à la fonction de déchiffrement
mais aussi la clef:

\lstinputlisting[style=customasmx86]{examples/dongles/2/tmp1_EN.asm}

L'algorithme est un simple \glslink{xoring}{xor}: chaque octet est xoré avec la clef, mais
la clef est incrémentée de 3 après le traitement de chaque octet.

Nous pouvons écrire un petit script Python pour vérifier notre hypothèse:

\lstinputlisting[caption=Python 3.x]{examples/dongles/2/decr1.py}

Et il affiche: \q{check security device connection}.
Donc oui, ceci est le message déchiffré.

Il y a d'autres messages chiffrés, avec leur clef correspondante.
Mais inutile de dire qu'il est possible de les déchiffrer sans leur clef.
Premièrement, nous voyons que le clef est en fait un octet.
C'est parce que l'instruction principale de déchiffrement (\XOR) fonctionne au niveau
de l'octet.
La clef se trouve dans le registre \ESI, mais seulement une partie de \ESI d'un octet
est utilisée.
Ainsi, une clef pourrait être plus grande que 255, mais sa valeur est toujours arrondie.

En conséquence, nous pouvons simplement essayer de brute-forcer, en essayant toutes
les clefs possible dans l'intervalle 0..255.
Nous allons aussi écarter les messages comportants des caractères non-imprimable.

\lstinputlisting[caption=Python 3.x]{examples/dongles/2/decr2.py}

Et nous obtenons:

\lstinputlisting[caption=Results]{examples/dongles/2/decr2_result.txt}

Ici il y a un peu de déchet, mais nous pouvons rapidement trouver les messages en
anglais.

À propos, puisque l'algorithme est un simple chiffrement xor, la même fonction peut
être utilisée pour chiffrer les messages.
Si besoin, nous pouvons chiffrer nos propres messages, et patcher le programme en les insérant.
}
\IT{\subsection{Fall-through}

Un altro uso diffuso dell'operatore \TT{switch()} è il cosiddetto \q{fallthrough}.
Ecco un semplice esempio \footnote{Preso da \url{https://github.com/azonalon/prgraas/blob/master/prog1lib/lecture_examples/is_whitespace.c}}:

\lstinputlisting[numbers=left,style=customc]{patterns/08_switch/4_fallthrough/fallthrough1.c}

Uno leggermente più difficile, dal kernel di Linux \footnote{Preso da \url{https://github.com/torvalds/linux/blob/master/drivers/media/dvb-frontends/lgdt3306a.c}}:

\lstinputlisting[numbers=left,style=customc]{patterns/08_switch/4_fallthrough/fallthrough2.c}

\lstinputlisting[caption=Optimizing GCC 5.4.0 x86,numbers=left,style=customasmx86]{patterns/08_switch/4_fallthrough/fallthrough2.s}

Possiamo arrivare alla label \TT{.L5} se all'input della funzione viene dato il valore 3250.
Ma si può anche giungere allo stesso punto da un altro percorso:
notiamo che non ci sono jump tra la chiamata a \printf e la label \TT{.L5}.

Questo spiega facilmente perchè i costrutti con \emph{switch()} sono spesso fonte di bug:
è sufficiente dimenticare un \emph{break} per trasformare il costrutto \emph{switch()} in un \emph{fallthrough} , in cui vengono eseguiti
più blocchi invece di uno solo.
}

\EN{\mysection{Task manager practical joke (Windows Vista)}
\myindex{Windows!Windows Vista}

Let's see if it's possible to hack Task Manager slightly so it would detect more \ac{CPU} cores.

\myindex{Windows!NTAPI}

Let us first think, how does the Task Manager know the number of cores?

There is the \TT{GetSystemInfo()} win32 function present in win32 userspace which can tell us this.
But it's not imported in \TT{taskmgr.exe}.

There is, however, another one in \gls{NTAPI}, \TT{NtQuerySystemInformation()}, 
which is used in \TT{taskmgr.exe} in several places.

To get the number of cores, one has to call this function with the \TT{SystemBasicInformation} constant
as a first argument (which is zero
\footnote{\href{http://msdn.microsoft.com/en-us/library/windows/desktop/ms724509(v=vs.85).aspx}{MSDN}}).

The second argument has to point to the buffer which is getting all the information.

So we have to find all calls to the \\
\TT{NtQuerySystemInformation(0, ?, ?, ?)} function.
Let's open \TT{taskmgr.exe} in IDA. 
\myindex{Windows!PDB}

What is always good about Microsoft executables is that IDA can download the corresponding \gls{PDB} 
file for this executable and show all function names.

It is visible that Task Manager is written in \Cpp and some of the function names and classes are really 
speaking for themselves.
There are classes CAdapter, CNetPage, CPerfPage, CProcInfo, CProcPage, CSvcPage, 
CTaskPage, CUserPage.

Apparently, each class corresponds to each tab in Task Manager.

Let's visit each call and add comment with the value which is passed as the first function argument.
We will write \q{not zero} at some places, because the value there was clearly not zero, 
but something really different (more about this in the second part of this chapter).

And we are looking for zero passed as argument, after all.

\begin{figure}[H]
\centering
\myincludegraphics{examples/taskmgr/IDA_xrefs.png}
\caption{IDA: cross references to NtQuerySystemInformation()}
\end{figure}

Yes, the names are really speaking for themselves.

When we closely investigate each place where\\
\TT{NtQuerySystemInformation(0, ?, ?, ?)} is called,
we quickly find what we need in the \TT{InitPerfInfo()} function:

\lstinputlisting[caption=taskmgr.exe (Windows Vista),style=customasmx86]{examples/taskmgr/taskmgr.lst}

\TT{g\_cProcessors} is a global variable, and this name has been assigned by 
IDA according to the \gls{PDB} loaded from Microsoft's symbol server.

The byte is taken from \TT{var\_C20}. 
And \TT{var\_C58} is passed to\\
\TT{NtQuerySystemInformation()} 
as a pointer to the receiving buffer.
The difference between 0xC20 and 0xC58 is 0x38 (56).

Let's take a look at format of the return structure, which we can find in MSDN:

\begin{lstlisting}[style=customc]
typedef struct _SYSTEM_BASIC_INFORMATION {
    BYTE Reserved1[24];
    PVOID Reserved2[4];
    CCHAR NumberOfProcessors;
} SYSTEM_BASIC_INFORMATION;
\end{lstlisting}

This is a x64 system, so each PVOID takes 8 bytes.

All \emph{reserved} fields in the structure take $24+4*8=56$ bytes.

Oh yes, this implies that \TT{var\_C20} is the local stack is exactly the
\TT{NumberOfProcessors} field of the \TT{SYSTEM\_BASIC\_INFORMATION} structure.

Let's check our guess.
Copy \TT{taskmgr.exe} from \TT{C:\textbackslash{}Windows\textbackslash{}System32} 
to some other folder 
(so the \emph{Windows Resource Protection} 
will not try to restore the patched \TT{taskmgr.exe}).

Let's open it in Hiew and find the place:

\begin{figure}[H]
\centering
\myincludegraphics{examples/taskmgr/hiew2.png}
\caption{Hiew: find the place to be patched}
\end{figure}

Let's replace the \TT{MOVZX} instruction with ours.
Let's pretend we've got 64 CPU cores.

Add one additional \ac{NOP} (because our instruction is shorter than the original one):

\begin{figure}[H]
\centering
\myincludegraphics{examples/taskmgr/hiew1.png}
\caption{Hiew: patch it}
\end{figure}

And it works!
Of course, the data in the graphs is not correct.

At times, Task Manager even shows an overall CPU load of more than 100\%.

\begin{figure}[H]
\centering
\myincludegraphics{examples/taskmgr/taskmgr_64cpu_crop.png}
\caption{Fooled Windows Task Manager}
\end{figure}

The biggest number Task Manager does not crash with is 64.

Apparently, Task Manager in Windows Vista was not tested on computers with a large number of cores.

So there are probably some static data structure(s) inside it limited to 64 cores.

\subsection{Using LEA to load values}
\label{TaskMgr_LEA}

Sometimes, \TT{LEA} is used in \TT{taskmgr.exe} instead of \TT{MOV} to set the first argument of \\
\TT{NtQuerySystemInformation()}:

\lstinputlisting[caption=taskmgr.exe (Windows Vista),style=customasmx86]{examples/taskmgr/taskmgr2.lst}

\myindex{x86!\Instructions!LEA}

Perhaps \ac{MSVC} did so because machine code of \INS{LEA} is shorter than \INS{MOV REG, 5} (would be 5 instead of 4).

\INS{LEA} with offset in $-128..127$ range (offset will occupy 1 byte in opcode) with 32-bit registers is even shorter (for lack of REX prefix)---3 bytes.

Another example of such thing is: \myref{using_MOV_and_pack_of_LEA_to_load_values}.
}
\RU{\subsection{Обменять входные значения друг с другом}

Вот так:

\lstinputlisting[style=customc]{patterns/061_pointers/swap/5_RU.c}

Как видим, байты загружаются в младшие 8-битные части регистров \TT{ECX} и \TT{EBX} используя \INS{MOVZX}
(так что старшие части регистров очищаются), затем байты записываются назад в другом порядке.

\lstinputlisting[style=customasmx86,caption=Optimizing GCC 5.4]{patterns/061_pointers/swap/5_GCC_O3_x86.s}

Адреса обоих байтов берутся из аргументов и во время исполнения ф-ции находятся в регистрах \TT{EDX} и \TT{EAX}.

Так что исопльзуем указатели --- вероятно, без них нет способа решить эту задачу лучше.

}
\FR{\subsection{Exemple \#2: SCO OpenServer}

\label{examples_SCO}
\myindex{SCO OpenServer}
Un ancien logiciel pour SCO OpenServer de 1997 développé par une société qui a disparue
depuis longtemps.

Il y a un driver de dongle special à installer dans le système, qui contient les
chaînes de texte suivantes:
\q{Copyright 1989, Rainbow Technologies, Inc., Irvine, CA}
et
\q{Sentinel Integrated Driver Ver. 3.0 }.

Après l'installation du driver dans SCO OpenServer, ces fichiers apparaissent dans
l'arborescence /dev:

\begin{lstlisting}
/dev/rbsl8
/dev/rbsl9
/dev/rbsl10
\end{lstlisting}

Le programme renvoie une erreur lorsque le dongle n'est pas connecté, mais le message
d'erreur n'est pas trouvé dans les exécutables.

\myindex{COFF}

Grâce à \ac{IDA}, il est facile de charger l'exécutable COFF utilisé dans SCO OpenServer.

Essayons de trouver la chaîne \q{rbsl} et en effet, elle se trouve dans ce morceau
de code:

\lstinputlisting[style=customasmx86]{examples/dongles/2/1.lst}

Oui, en effet, le programme doit communiquer d'une façon ou d'une autre avec le driver.

\myindex{thunk-functions}
Le seul endroit où la fonction \TT{SSQC()} est appelée est dans la \glslink{thunk
 function}{fonction thunk}:

\lstinputlisting[style=customasmx86]{examples/dongles/2/2.lst}

SSQ() peut être appelé depuis au moins 2 fonctions.

L'une d'entre elles est:

\lstinputlisting[style=customasmx86]{examples/dongles/2/check1_EN.lst}

\q{\TT{3C}} et \q{\TT{3E}} semblent familiers: il y avait un dongle Sentinel Pro de
Rainbow sans mémoire, fournissant seulement une fonction de crypto-hachage secrète.

Vous pouvez lire une courte description de la fonction de hachage dont il s'agit
ici: \myref{hash_func}.

Mais retournons au programme.

Donc le programme peut seulement tester si un dongle est connecté ou s'il est absent.

Aucune autre information ne peut être écrite dans un tel dongle, puisqu'il n'a pas
de mémoire.
Les codes sur deux caractères sont des commandes (nous pouvons voir comment les commandes
sont traitées dans la fonction \TT{SSQC()}) et toutes les autres chaînes sont hachées
dans le dongle, transformées en un nombre 16-bit.
L'algorithme était secret, donc il n'était pas possible d'écrire un driver de remplacement
ou de refaire un dongle matériel qui l'émulerait parfaitement.

Toutefois, il est toujours possible d'intercepter tous les accès au dongle et de
trouver les constantes auxquelles les résultats de la fonction de hachage sont comparées.

Mais nous devons dire qu'il est possible de construire un schéma de logiciel de protection
de copie robuste basé sur une fonction secrète de hachage cryptographique: il suffit
qu'elle chiffre/déchiffre les fichiers de données utilisés par votre logiciel.

Mais retournons au code:

Les codes 51/52/53 sont utilisés pour choisir le port imprimante LPT.
3x/4x sont utilisés pour le choix de la \q{famille} (c'est ainsi que les dongles
Sentinel Pro sont différenciés les uns des autres: plus d'un dongle peut être connecté
sur un port LPT).

La seule chaîne passée à la fonction qui ne fasse pas 2 caractères est "0123456789".

Ensuite, le résultat est comparé à l'ensemble des résultats valides.

Si il est correct, 0xC ou 0xB est écrit dans la variable globale \TT{ctl\_model}.%

Une autre chaîne de texte qui est passée est
"PRESS ANY KEY TO CONTINUE: ", mais le résultat n'est pas testé.
Difficile de dire pourquoi, probablement une erreur\footnote{C'est un sentiment
étrange de trouver un bug dans un logiciel aussi ancien.}.

Voyons où la valeur de la variable globale \TT{ctl\_model} est utilisée.

Un tel endroit est:

\lstinputlisting[style=customasmx86]{examples/dongles/2/4.lst}

Si c'est 0, un message d'erreur chiffré est passé à une routine de déchiffrement
et affiché.

\myindex{x86!\Instructions!XOR}

La routine de déchiffrement de la chaîne semble être un simple \glslink{xoring}{xor}:

\lstinputlisting[style=customasmx86]{examples/dongles/2/err_warn.lst}

C'est pourquoi nous étions incapable de trouver le message d'erreur dans les fichiers
exécutable, car ils sont chiffrés (ce qui est une pratique courante).

Un autre appel à la fonction de hachage \TT{SSQ()} lui passe la chaîne \q{offln}
et le résultat est comparé avec \TT{0xFE81} et \TT{0x12A9}.

Si ils ne correspondent pas, ça se comporte comme une sorte de fonction \TT{timer()}
(peut-être en attente qu'un dongle mal connecté soit reconnecté et re-testé?) et ensuite
déchiffre un autre message d'erreur à afficher.

\lstinputlisting[style=customasmx86]{examples/dongles/2/check2_EN.lst}

Passer outre le dongle est assez facile: il suffit de patcher tous les sauts après
les instructions \CMP pertinentes.

Une autre option est d'écrire notre propre driver SCO OpenServer, contenant une table
de questions et de réponses, toutes celles qui sont présentent dans le programme.

\subsubsection{Déchiffrer les messages d'erreur}

À propos, nous pouvons aussi essayer de déchiffrer tous les messages d'erreurs.
L'algorithme qui se trouve dans la fonction \TT{err\_warn()} est très simple, en effet:

\lstinputlisting[caption=Decryption function,style=customasmx86]{examples/dongles/2/decrypting_FR.lst}

Comme on le voit, non seulement la chaîne est transmise à la fonction de déchiffrement
mais aussi la clef:

\lstinputlisting[style=customasmx86]{examples/dongles/2/tmp1_EN.asm}

L'algorithme est un simple \glslink{xoring}{xor}: chaque octet est xoré avec la clef, mais
la clef est incrémentée de 3 après le traitement de chaque octet.

Nous pouvons écrire un petit script Python pour vérifier notre hypothèse:

\lstinputlisting[caption=Python 3.x]{examples/dongles/2/decr1.py}

Et il affiche: \q{check security device connection}.
Donc oui, ceci est le message déchiffré.

Il y a d'autres messages chiffrés, avec leur clef correspondante.
Mais inutile de dire qu'il est possible de les déchiffrer sans leur clef.
Premièrement, nous voyons que le clef est en fait un octet.
C'est parce que l'instruction principale de déchiffrement (\XOR) fonctionne au niveau
de l'octet.
La clef se trouve dans le registre \ESI, mais seulement une partie de \ESI d'un octet
est utilisée.
Ainsi, une clef pourrait être plus grande que 255, mais sa valeur est toujours arrondie.

En conséquence, nous pouvons simplement essayer de brute-forcer, en essayant toutes
les clefs possible dans l'intervalle 0..255.
Nous allons aussi écarter les messages comportants des caractères non-imprimable.

\lstinputlisting[caption=Python 3.x]{examples/dongles/2/decr2.py}

Et nous obtenons:

\lstinputlisting[caption=Results]{examples/dongles/2/decr2_result.txt}

Ici il y a un peu de déchet, mais nous pouvons rapidement trouver les messages en
anglais.

À propos, puisque l'algorithme est un simple chiffrement xor, la même fonction peut
être utilisée pour chiffrer les messages.
Si besoin, nous pouvons chiffrer nos propres messages, et patcher le programme en les insérant.
}

\RU{\subsection{Обменять входные значения друг с другом}

Вот так:

\lstinputlisting[style=customc]{patterns/061_pointers/swap/5_RU.c}

Как видим, байты загружаются в младшие 8-битные части регистров \TT{ECX} и \TT{EBX} используя \INS{MOVZX}
(так что старшие части регистров очищаются), затем байты записываются назад в другом порядке.

\lstinputlisting[style=customasmx86,caption=Optimizing GCC 5.4]{patterns/061_pointers/swap/5_GCC_O3_x86.s}

Адреса обоих байтов берутся из аргументов и во время исполнения ф-ции находятся в регистрах \TT{EDX} и \TT{EAX}.

Так что исопльзуем указатели --- вероятно, без них нет способа решить эту задачу лучше.

}
\EN{\mysection{Task manager practical joke (Windows Vista)}
\myindex{Windows!Windows Vista}

Let's see if it's possible to hack Task Manager slightly so it would detect more \ac{CPU} cores.

\myindex{Windows!NTAPI}

Let us first think, how does the Task Manager know the number of cores?

There is the \TT{GetSystemInfo()} win32 function present in win32 userspace which can tell us this.
But it's not imported in \TT{taskmgr.exe}.

There is, however, another one in \gls{NTAPI}, \TT{NtQuerySystemInformation()}, 
which is used in \TT{taskmgr.exe} in several places.

To get the number of cores, one has to call this function with the \TT{SystemBasicInformation} constant
as a first argument (which is zero
\footnote{\href{http://msdn.microsoft.com/en-us/library/windows/desktop/ms724509(v=vs.85).aspx}{MSDN}}).

The second argument has to point to the buffer which is getting all the information.

So we have to find all calls to the \\
\TT{NtQuerySystemInformation(0, ?, ?, ?)} function.
Let's open \TT{taskmgr.exe} in IDA. 
\myindex{Windows!PDB}

What is always good about Microsoft executables is that IDA can download the corresponding \gls{PDB} 
file for this executable and show all function names.

It is visible that Task Manager is written in \Cpp and some of the function names and classes are really 
speaking for themselves.
There are classes CAdapter, CNetPage, CPerfPage, CProcInfo, CProcPage, CSvcPage, 
CTaskPage, CUserPage.

Apparently, each class corresponds to each tab in Task Manager.

Let's visit each call and add comment with the value which is passed as the first function argument.
We will write \q{not zero} at some places, because the value there was clearly not zero, 
but something really different (more about this in the second part of this chapter).

And we are looking for zero passed as argument, after all.

\begin{figure}[H]
\centering
\myincludegraphics{examples/taskmgr/IDA_xrefs.png}
\caption{IDA: cross references to NtQuerySystemInformation()}
\end{figure}

Yes, the names are really speaking for themselves.

When we closely investigate each place where\\
\TT{NtQuerySystemInformation(0, ?, ?, ?)} is called,
we quickly find what we need in the \TT{InitPerfInfo()} function:

\lstinputlisting[caption=taskmgr.exe (Windows Vista),style=customasmx86]{examples/taskmgr/taskmgr.lst}

\TT{g\_cProcessors} is a global variable, and this name has been assigned by 
IDA according to the \gls{PDB} loaded from Microsoft's symbol server.

The byte is taken from \TT{var\_C20}. 
And \TT{var\_C58} is passed to\\
\TT{NtQuerySystemInformation()} 
as a pointer to the receiving buffer.
The difference between 0xC20 and 0xC58 is 0x38 (56).

Let's take a look at format of the return structure, which we can find in MSDN:

\begin{lstlisting}[style=customc]
typedef struct _SYSTEM_BASIC_INFORMATION {
    BYTE Reserved1[24];
    PVOID Reserved2[4];
    CCHAR NumberOfProcessors;
} SYSTEM_BASIC_INFORMATION;
\end{lstlisting}

This is a x64 system, so each PVOID takes 8 bytes.

All \emph{reserved} fields in the structure take $24+4*8=56$ bytes.

Oh yes, this implies that \TT{var\_C20} is the local stack is exactly the
\TT{NumberOfProcessors} field of the \TT{SYSTEM\_BASIC\_INFORMATION} structure.

Let's check our guess.
Copy \TT{taskmgr.exe} from \TT{C:\textbackslash{}Windows\textbackslash{}System32} 
to some other folder 
(so the \emph{Windows Resource Protection} 
will not try to restore the patched \TT{taskmgr.exe}).

Let's open it in Hiew and find the place:

\begin{figure}[H]
\centering
\myincludegraphics{examples/taskmgr/hiew2.png}
\caption{Hiew: find the place to be patched}
\end{figure}

Let's replace the \TT{MOVZX} instruction with ours.
Let's pretend we've got 64 CPU cores.

Add one additional \ac{NOP} (because our instruction is shorter than the original one):

\begin{figure}[H]
\centering
\myincludegraphics{examples/taskmgr/hiew1.png}
\caption{Hiew: patch it}
\end{figure}

And it works!
Of course, the data in the graphs is not correct.

At times, Task Manager even shows an overall CPU load of more than 100\%.

\begin{figure}[H]
\centering
\myincludegraphics{examples/taskmgr/taskmgr_64cpu_crop.png}
\caption{Fooled Windows Task Manager}
\end{figure}

The biggest number Task Manager does not crash with is 64.

Apparently, Task Manager in Windows Vista was not tested on computers with a large number of cores.

So there are probably some static data structure(s) inside it limited to 64 cores.

\subsection{Using LEA to load values}
\label{TaskMgr_LEA}

Sometimes, \TT{LEA} is used in \TT{taskmgr.exe} instead of \TT{MOV} to set the first argument of \\
\TT{NtQuerySystemInformation()}:

\lstinputlisting[caption=taskmgr.exe (Windows Vista),style=customasmx86]{examples/taskmgr/taskmgr2.lst}

\myindex{x86!\Instructions!LEA}

Perhaps \ac{MSVC} did so because machine code of \INS{LEA} is shorter than \INS{MOV REG, 5} (would be 5 instead of 4).

\INS{LEA} with offset in $-128..127$ range (offset will occupy 1 byte in opcode) with 32-bit registers is even shorter (for lack of REX prefix)---3 bytes.

Another example of such thing is: \myref{using_MOV_and_pack_of_LEA_to_load_values}.
}
\FR{\subsection{Exemple \#2: SCO OpenServer}

\label{examples_SCO}
\myindex{SCO OpenServer}
Un ancien logiciel pour SCO OpenServer de 1997 développé par une société qui a disparue
depuis longtemps.

Il y a un driver de dongle special à installer dans le système, qui contient les
chaînes de texte suivantes:
\q{Copyright 1989, Rainbow Technologies, Inc., Irvine, CA}
et
\q{Sentinel Integrated Driver Ver. 3.0 }.

Après l'installation du driver dans SCO OpenServer, ces fichiers apparaissent dans
l'arborescence /dev:

\begin{lstlisting}
/dev/rbsl8
/dev/rbsl9
/dev/rbsl10
\end{lstlisting}

Le programme renvoie une erreur lorsque le dongle n'est pas connecté, mais le message
d'erreur n'est pas trouvé dans les exécutables.

\myindex{COFF}

Grâce à \ac{IDA}, il est facile de charger l'exécutable COFF utilisé dans SCO OpenServer.

Essayons de trouver la chaîne \q{rbsl} et en effet, elle se trouve dans ce morceau
de code:

\lstinputlisting[style=customasmx86]{examples/dongles/2/1.lst}

Oui, en effet, le programme doit communiquer d'une façon ou d'une autre avec le driver.

\myindex{thunk-functions}
Le seul endroit où la fonction \TT{SSQC()} est appelée est dans la \glslink{thunk
 function}{fonction thunk}:

\lstinputlisting[style=customasmx86]{examples/dongles/2/2.lst}

SSQ() peut être appelé depuis au moins 2 fonctions.

L'une d'entre elles est:

\lstinputlisting[style=customasmx86]{examples/dongles/2/check1_EN.lst}

\q{\TT{3C}} et \q{\TT{3E}} semblent familiers: il y avait un dongle Sentinel Pro de
Rainbow sans mémoire, fournissant seulement une fonction de crypto-hachage secrète.

Vous pouvez lire une courte description de la fonction de hachage dont il s'agit
ici: \myref{hash_func}.

Mais retournons au programme.

Donc le programme peut seulement tester si un dongle est connecté ou s'il est absent.

Aucune autre information ne peut être écrite dans un tel dongle, puisqu'il n'a pas
de mémoire.
Les codes sur deux caractères sont des commandes (nous pouvons voir comment les commandes
sont traitées dans la fonction \TT{SSQC()}) et toutes les autres chaînes sont hachées
dans le dongle, transformées en un nombre 16-bit.
L'algorithme était secret, donc il n'était pas possible d'écrire un driver de remplacement
ou de refaire un dongle matériel qui l'émulerait parfaitement.

Toutefois, il est toujours possible d'intercepter tous les accès au dongle et de
trouver les constantes auxquelles les résultats de la fonction de hachage sont comparées.

Mais nous devons dire qu'il est possible de construire un schéma de logiciel de protection
de copie robuste basé sur une fonction secrète de hachage cryptographique: il suffit
qu'elle chiffre/déchiffre les fichiers de données utilisés par votre logiciel.

Mais retournons au code:

Les codes 51/52/53 sont utilisés pour choisir le port imprimante LPT.
3x/4x sont utilisés pour le choix de la \q{famille} (c'est ainsi que les dongles
Sentinel Pro sont différenciés les uns des autres: plus d'un dongle peut être connecté
sur un port LPT).

La seule chaîne passée à la fonction qui ne fasse pas 2 caractères est "0123456789".

Ensuite, le résultat est comparé à l'ensemble des résultats valides.

Si il est correct, 0xC ou 0xB est écrit dans la variable globale \TT{ctl\_model}.%

Une autre chaîne de texte qui est passée est
"PRESS ANY KEY TO CONTINUE: ", mais le résultat n'est pas testé.
Difficile de dire pourquoi, probablement une erreur\footnote{C'est un sentiment
étrange de trouver un bug dans un logiciel aussi ancien.}.

Voyons où la valeur de la variable globale \TT{ctl\_model} est utilisée.

Un tel endroit est:

\lstinputlisting[style=customasmx86]{examples/dongles/2/4.lst}

Si c'est 0, un message d'erreur chiffré est passé à une routine de déchiffrement
et affiché.

\myindex{x86!\Instructions!XOR}

La routine de déchiffrement de la chaîne semble être un simple \glslink{xoring}{xor}:

\lstinputlisting[style=customasmx86]{examples/dongles/2/err_warn.lst}

C'est pourquoi nous étions incapable de trouver le message d'erreur dans les fichiers
exécutable, car ils sont chiffrés (ce qui est une pratique courante).

Un autre appel à la fonction de hachage \TT{SSQ()} lui passe la chaîne \q{offln}
et le résultat est comparé avec \TT{0xFE81} et \TT{0x12A9}.

Si ils ne correspondent pas, ça se comporte comme une sorte de fonction \TT{timer()}
(peut-être en attente qu'un dongle mal connecté soit reconnecté et re-testé?) et ensuite
déchiffre un autre message d'erreur à afficher.

\lstinputlisting[style=customasmx86]{examples/dongles/2/check2_EN.lst}

Passer outre le dongle est assez facile: il suffit de patcher tous les sauts après
les instructions \CMP pertinentes.

Une autre option est d'écrire notre propre driver SCO OpenServer, contenant une table
de questions et de réponses, toutes celles qui sont présentent dans le programme.

\subsubsection{Déchiffrer les messages d'erreur}

À propos, nous pouvons aussi essayer de déchiffrer tous les messages d'erreurs.
L'algorithme qui se trouve dans la fonction \TT{err\_warn()} est très simple, en effet:

\lstinputlisting[caption=Decryption function,style=customasmx86]{examples/dongles/2/decrypting_FR.lst}

Comme on le voit, non seulement la chaîne est transmise à la fonction de déchiffrement
mais aussi la clef:

\lstinputlisting[style=customasmx86]{examples/dongles/2/tmp1_EN.asm}

L'algorithme est un simple \glslink{xoring}{xor}: chaque octet est xoré avec la clef, mais
la clef est incrémentée de 3 après le traitement de chaque octet.

Nous pouvons écrire un petit script Python pour vérifier notre hypothèse:

\lstinputlisting[caption=Python 3.x]{examples/dongles/2/decr1.py}

Et il affiche: \q{check security device connection}.
Donc oui, ceci est le message déchiffré.

Il y a d'autres messages chiffrés, avec leur clef correspondante.
Mais inutile de dire qu'il est possible de les déchiffrer sans leur clef.
Premièrement, nous voyons que le clef est en fait un octet.
C'est parce que l'instruction principale de déchiffrement (\XOR) fonctionne au niveau
de l'octet.
La clef se trouve dans le registre \ESI, mais seulement une partie de \ESI d'un octet
est utilisée.
Ainsi, une clef pourrait être plus grande que 255, mais sa valeur est toujours arrondie.

En conséquence, nous pouvons simplement essayer de brute-forcer, en essayant toutes
les clefs possible dans l'intervalle 0..255.
Nous allons aussi écarter les messages comportants des caractères non-imprimable.

\lstinputlisting[caption=Python 3.x]{examples/dongles/2/decr2.py}

Et nous obtenons:

\lstinputlisting[caption=Results]{examples/dongles/2/decr2_result.txt}

Ici il y a un peu de déchet, mais nous pouvons rapidement trouver les messages en
anglais.

À propos, puisque l'algorithme est un simple chiffrement xor, la même fonction peut
être utilisée pour chiffrer les messages.
Si besoin, nous pouvons chiffrer nos propres messages, et patcher le programme en les insérant.
}

\EN{\mysection{Some GCC library functions}
\myindex{GCC}
\label{sec:GCC_library_func}

%__ashldi3
%__ashrdi3
%__floatundidf
%__floatdisf
%__floatdixf
%__floatundidf
%__floatundisf
%__floatundixf
%__lshrdi3
%__muldi3

\begin{center}
\begin{tabular}{ | l | l | }
\hline
\HeaderColor name & \HeaderColor meaning \\
\hline \TT{\_\_divdi3} & signed division \\
\hline \TT{\_\_moddi3} & getting remainder (modulo) of signed division \\
\hline \TT{\_\_udivdi3} & unsigned division \\
\hline \TT{\_\_umoddi3} & getting remainder (modulo) of unsigned division \\
\hline
\end{tabular}
\end{center}

}
\RU{\mysection{Некоторые библиотечные функции GCC}
\myindex{GCC}
\label{sec:GCC_library_func}

%__ashldi3
%__ashrdi3
%__floatundidf
%__floatdisf
%__floatdixf
%__floatundidf
%__floatundisf
%__floatundixf
%__lshrdi3
%__muldi3

\begin{center}
\begin{tabular}{ | l | l | }
\hline
\HeaderColor имя & \HeaderColor значение \\
\hline \TT{\_\_divdi3} & знаковое деление \\
\hline \TT{\_\_moddi3} & остаток от знакового деления \\
\hline \TT{\_\_udivdi3} & беззнаковое деление \\
\hline \TT{\_\_umoddi3} & остаток от беззнакового деления \\
\hline
\end{tabular}
\end{center}

}
\FR{\mysection{Quelques fonctions de la bibliothèque de GCC}
\myindex{GCC}
\label{sec:GCC_library_func}

%__ashldi3
%__ashrdi3
%__floatundidf
%__floatdisf
%__floatdixf
%__floatundidf
%__floatundisf
%__floatundixf
%__lshrdi3
%__muldi3

\begin{center}
\begin{tabular}{ | l | l | }
\hline
\HeaderColor nom & \HeaderColor signification \\
\hline \TT{\_\_divdi3} & division signée \\
\hline \TT{\_\_moddi3} & reste (modulo) d'une division signée \\
\hline \TT{\_\_udivdi3} & division non signée \\
\hline \TT{\_\_umoddi3} & reste (modulo) d'une division non signée \\
\hline
\end{tabular}
\end{center}

}
\DE{\mysection{Einige GCC-Bibliotheks-Funktionen}
\myindex{GCC}
\label{sec:GCC_library_func}

%__ashldi3
%__ashrdi3
%__floatundidf
%__floatdisf
%__floatdixf
%__floatundidf
%__floatundisf
%__floatundixf
%__lshrdi3
%__muldi3

\begin{center}
\begin{tabular}{ | l | l | }
\hline
\HeaderColor Name & \HeaderColor Bedeutung \\
\hline \TT{\_\_divdi3} & vorzeichenbehaftete Division \\
\hline \TT{\_\_moddi3} & Rest (Modulo) einer vorzeichenbehafteten Division \\
\hline \TT{\_\_udivdi3} & vorzeichenlose Division \\
\hline \TT{\_\_umoddi3} & Rest (Modulo) einer vorzeichenlosen Division \\
\hline
\end{tabular}
\end{center}

}


\mysection{\RU{Некоторые библиотечные функции MSVC}\EN{Some MSVC library functions}\DE{Einige MSVC-Bibliotheks-Funktionen}%
\FR{Quelques fonctions de la bibliothèque MSVC}}
\myindex{MSVC}
\label{sec:MSVC_library_func}

\TT{ll} \RU{в имени функции означает}\EN{in function name stands for}\DE{in Funktionsnamen steht für}%
\FR{dans une fontion signifie} \q{long long}, \RU{т.е. 64-битный тип данных}
\EN{e.g., a 64-bit data type}\DE{z.B. einen 64-Bit-Datentyp}\FR{i.e., type de donées 64-bit}.

\begin{center}
\begin{tabular}{ | l | l | }
\hline
\HeaderColor \RU{имя}\EN{name}\DE{Name}\FR{nom} & \HeaderColor \RU{значение}\EN{meaning}\DE{Bedeutung}\FR{signification} \\
\hline \TT{\_\_alldiv} & \RU{знаковое деление}\EN{signed division}\DE{vorzeichenbehaftete Division}\FR{division signée} \\
\hline \TT{\_\_allmul} & \RU{умножение}\EN{multiplication}\DE{Multiplikation}\FR{multiplication} \\
\hline \TT{\_\_allrem} & \RU{остаток от знакового деления}\EN{remainder of signed division}\DE{Rest einer vorzeichenbehafteten Division}%
\FR{reste de la division signée} \\
\hline \TT{\_\_allshl} & \RU{сдвиг влево}\EN{shift left}\DE{Schiebe links}\FR{décalage à gauche} \\
\hline \TT{\_\_allshr} & \RU{знаковый сдвиг вправо}\EN{signed shift right}\DE{Schiebe links, vorzeichenbehaftet}%
\FR{décalage signé à droite} \\
\hline \TT{\_\_aulldiv} & \RU{беззнаковое деление}\EN{unsigned division}\DE{vorzeichenlose Division}%
\FR{division non signée} \\
\hline \TT{\_\_aullrem} & \RU{остаток от беззнакового деления}\EN{remainder of unsigned division}\DE{Rest (Modulo) einer vorzeichenlosen Division}%
\FR{reste de la division non signée} \\
\hline \TT{\_\_aullshr} & \RU{беззнаковый сдвиг вправо}\EN{unsigned shift right}\DE{Schiebe rechts, vorzeichenlos}%
\FR{décalage non signé à droite} \\
\hline
\end{tabular}
\end{center}

\RU{Процедуры умножения и сдвига влево, одни и те же и для знаковых чисел, и для беззнаковых,
поэтому здесь только одна функция для каждой операции}
\EN{Multiplication and shift left procedures are the same for both signed and unsigned numbers, hence there is only one function 
for each operation here}
\DE{Multiplikation und Links-Schiebebefehle sind sowohl für vorzeichenbehaftete als auch vorzeichenlose Zahlen,
da hier für jede Operation nur ein Befehl existiert}
\FR{La multiplication et le décalage à gauche sont similaire pour les nombres signés
et non signés, donc il n'y a qu'une seule fonction ici}. \\
\\
\RU{Исходные коды этих функций можно найти в установленной \ac{MSVS}, в}\EN{The source code of these function
can be found in the installed \ac{MSVS}, in}%
\DE{Der Quellcode dieser Funktionen kann im Pfad des installierten \ac{MSVS}, gefunden werden: }%
\FR{Le code source des ces fonctions peut être trouvé dans l'installation de \ac{MSVS},
dans} \TT{VC/crt/src/intel/*.asm}.


\mysection{Cheatsheets}

% sections
\subsection{IDA}
\myindex{IDA}
\label{sec:IDA_cheatsheet}

\ShortHotKeyCheatsheet:

\begin{center}
\begin{tabular}{ | l | l | }
\hline
\HeaderColor \RU{клавиша}\EN{key}\DE{Taste}\FR{touche} & \HeaderColor \RU{значение}\EN{meaning}\DE{Bedeutung}\FR{signification} \\
\hline
Space 	& \RU{переключать между листингом и просмотром кода в виде графа}
            \EN{switch listing and graph view}
            \DE{Zwischen Quellcode und grafischer Ansicht wechseln}%
				\FR{échanger le listing et le mode graphique} \\
C 	& \RU{конвертировать в код}\EN{convert to code}\DE{zu Code konvertieren}%
		\FR{convertir en code} \\
D 	& \RU{конвертировать в данные}\EN{convert to data}\DE{zu Daten konvertieren}%
		\FR{convertir en données} \\
A 	& \RU{конвертировать в строку}\EN{convert to string}\DE{zu Zeichenkette konvertieren}%
		\FR{convertir en chaîne} \\
* 	& \RU{конвертировать в массив}\EN{convert to array}\DE{zu Array konvertieren}%
		\FR{convertir en tableau} \\
U 	& \RU{сделать неопределенным}\EN{undefine}\DE{undefinieren}%
		\FR{rendre indéfini}\\
O 	& \RU{сделать смещение из операнда}\EN{make offset of operand}\DE{Offset von Operanden}%
		\FR{donner l'offset d'une opérande}\\
H 	& \RU{сделать десятичное число}\EN{make decimal number}\DE{Dezimalzahl erstellen}%
		\FR{transformer en nombre décimal} \\
R 	& \RU{сделать символ}\EN{make char}\DE{Zeichen erstellen}%
		\FR{transformer en caractère} \\
B 	& \RU{сделать двоичное число}\EN{make binary number}\DE{Binärzahl erstellen}%
		\FR{transformer en nombre binaire} \\
Q 	& \RU{сделать шестнадцатеричное число}\EN{make hexadecimal number}\DE{Hexadezimalzahl erstellen}%
		\FR{transformer en nombre hexa-décimal} \\
N 	& \RU{переименовать идентификатор}\EN{rename identifier}\DE{Identifikator umbenennen}%
		\FR{renommer l'identifiant} \\
? 	& \RU{калькулятор}\EN{calculator}\DE{Rechner}\FR{calculatrice} \\
G 	& \RU{переход на адрес}\EN{jump to address}\DE{zu Adresse springen}%
		\FR{sauter à l'adresse} \\
: 	& \RU{добавить комментарий}\EN{add comment}\DE{Kommentar einfügen}\FR{ajouter un commentaire} \\
Ctrl-X 	& \RU{показать ссылки на текущую функцию, метку, переменную}%
		\EN{show references to the current function, label, variable }%
		\DE{Referenz zu aktueller Funktion, Variable, ... zeigen}%
		\FR{montrer les références à la fonction, au label, à la variable courant} \\
	& \RU{(в т.ч., в стеке)}\EN{(incl. in local stack)}\DE{(inkl. lokalem Stack)}%
		\FR{inclure dans la pile locale} \\
X 	& \RU{показать ссылки на функцию, метку, переменную, итд}\EN{show references to the function, label, variable, etc.}%
		\DE{Referenz zu Funktion, Variable, ... zeigen}%
		\FR{montrer les références à la fonction, au label, à la variable, etc.} \\
Alt-I 	& \RU{искать константу}\EN{search for constant}\DE{Konstante suchen}%
		\FR{chercher une constante} \\
Ctrl-I 	& \RU{искать следующее вхождение константы}\EN{search for the next occurrence of constant}\DE{Nächstes Auftreten der Konstante suchen}%
		\FR{chercher la prochaine occurrence d'une constante} \\
Alt-B 	& \RU{искать последовательность байт}\EN{search for byte sequence}\DE{Byte-Sequenz suchen}%
		\FR{chercher une séquence d'octets} \\
Ctrl-B 	& \RU{искать следующее вхождение последовательности байт}
		\EN{search for the next occurrence of byte sequence}
		\DE{Nächstes Auftreten der Byte-Sequenz suchen}%
		\FR{chercher l'occurrence suivante d'une séquence d'octets} \\
Alt-T 	& \RU{искать текст (включая инструкции, итд.)}%
		\EN{search for text (including instructions, etc.)}%
		\DE{Text suchen (inkl. Anweisungen, usw.)}%
		\FR{chercher du texte (instructions incluses, etc.)} \\
Ctrl-T 	& \RU{искать следующее вхождение текста}%
		\EN{search for the next occurrence of text}%
		\DE{nächstes Aufreten des Textes suchen}%
		\FR{chercher l'occurrence suivante du texte} \\
Alt-P 	& \RU{редактировать текущую функцию}%
		\EN{edit current function}%
		\DE{akutelle Funktion editieren}%
		\FR{éditer la fonction courante} \\
Enter 	& \RU{перейти к функции, переменной, итд.}%
		\EN{jump to function, variable, etc.}%
		\DE{zu Funktion, Variable, ... springen}%
		\FR{sauter à la fonction, la variable, etc.} \\
Esc 	& \RU{вернуться назад}\EN{get back}\DE{zurückgehen}%
		\FR{retourner en arrière} \\
Num -   & \RU{свернуть функцию или отмеченную область}%
		\EN{fold function or selected area}%
		\DE{Funktion oder markierten Bereich einklappen}%
		\FR{cacher/plier la fonction ou la partie sélectionnée} \\
Num + 	& \RU{снова показать функцию или область}%
		\EN{unhide function or area}%
		\DE{Funktion oder Bereich anzeigen}%
		\FR{afficher la fonction ou une partie} \\
\hline
\end{tabular}
\end{center}

\RU{Сворачивание функции или области может быть удобно чтобы прятать те части функции,
чья функция вам стала уже ясна}%
\EN{Function/area folding may be useful for hiding function parts when you realize what they do}%
\DE{Das Einklappen ist nützlich um Teile von Funktionen zu verstecken, wenn bekannt ist was sie tun}%
\FR{cacher une fonction ou une partie de code peut être utile pour cacher des parties du
code lorsque vous avez compris ce qu'elles font}.
\RU{это используется в моем скрипте\footnote{\href{\YurichevIDAIDCScripts}{GitHub}}}\EN{this is used in my}\DE{dies wird genutzt im}%
\RU{для сворачивания некоторых очень часто используемых фрагментов inline-кода}%
\EN{script\footnote{\href{\YurichevIDAIDCScripts}{GitHub}} for hiding some often used patterns of inline code}%
\DE{Script\footnote{\href{\YurichevIDAIDCScripts}{GitHub}} um häufig genutzte Inline-Code-Stellen zu verstecken}%
\FR{ceci est utilisé dans mon script\footnote{\href{\YurichevIDAIDCScripts}{GitHub}}%
pour cacher des patterns de code inline souvent utilisés}.


\subsection{\olly}
\myindex{\olly}
\label{sec:Olly_cheatsheet}

\ShortHotKeyCheatsheet:

\begin{center}
\begin{tabular}{ | l | l | }
\hline
\HeaderColor \RU{хот-кей}\EN{hot-key}\DE{Tastenkürzel}\FR{raccourci} & 
\HeaderColor \RU{значение}\EN{meaning}\DE{Bedeutung}\FR{signification} \\
\hline
F7	& \RU{трассировать внутрь}\EN{trace into}\DE{Schritt}\FR{tracer dans la fonction}\\
F8	& \stepover\\
F9	& \RU{запуск}\EN{run}\DE{starten}\FR{démarrer}\\
Ctrl-F2	& \RU{перезапуск}\EN{restart}\DE{Neustart}\FR{redémarrer}\\
\hline
\end{tabular}
\end{center}

\subsection{MSVC}
\myindex{MSVC}
\label{sec:MSVC_options}

\RU{Некоторые полезные опции, которые были использованы в книге}
\EN{Some useful options which were used through this book}.
\DE{Einige nützliche Optionen die in diesem Buch genutzt werden}.
\FR{Quelques options utiles qui ont été utilisées dans ce livre}

\begin{center}
\begin{tabular}{ | l | l | }
\hline
\HeaderColor \RU{опция}\EN{option}\DE{Option}\FR{option} & 
\HeaderColor \RU{значение}\EN{meaning}\DE{Bedeutung}\FR{signification} \\
\hline
/O1		& \RU{оптимизация по размеру кода}\EN{minimize space}\DE{Speicherplatz minimieren}%
\FR{minimiser l'espace}\\
/Ob0		& \RU{не заменять вызовы inline-функций их кодом}\EN{no inline expansion}\DE{Keine Inline-Erweiterung}%
\FR{pas de mire en ligne}\\
/Ox		& \RU{максимальная оптимизация}\EN{maximum optimizations}\DE{maximale Optimierung}%
\FR{optimisation maximale}\\
/GS-		& \RU{отключить проверки переполнений буфера}
		\EN{disable security checks (buffer overflows)}
        \DE{Sicherheitsüberprüfungen deaktivieren (Buffer Overflows)}%
		\FR{désactiver les vérifications de sécurité (buffer overflows)}\\
/Fa(file)	& \RU{генерировать листинг на ассемблере}\EN{generate assembly listing}\DE{Assembler-Quelltext erstellen}%
\FR{générer un listing assembleur}\\
/Zi		& \RU{генерировать отладочную информацию}\EN{enable debugging information}\DE{Debugging-Informationen erstellen}%
\FR{activer les informations de débogage}\\
/Zp(n)		& \RU{паковать структуры по границе в $n$ байт}\EN{pack structs on $n$-byte boundary}\DE{Strukturen an $n$-Byte-Grenze ausrichten}%
\FR{aligner les structures sur une limite de $n$-octet}\\
/MD		& \RU{выходной исполняемый файл будет использовать}
			\EN{produced executable will use}%
            \DE{ausführbare Daten nutzt}%
\FR{l'exécutable généré utilisera} \TT{MSVCR*.DLL}\\
\hline
\end{tabular}
\end{center}

\RU{Кое-как информация о версиях MSVC}\EN{Some information about MSVC versions}\DE{Informationen zu MSVC-Versionen}%
\FR{Quelques informations sur les versions de MSVC}:
\myref{MSVC_versions}.


\EN{\subsection{GCC}
\myindex{GCC}

Some useful options which were used through this book.

\begin{center}
\begin{tabular}{ | l | l | }
\hline
\HeaderColor option & 
\HeaderColor meaning \\
\hline
-Os		& code size optimization \\
-O3		& maximum optimization \\
-regparm=	& how many arguments are to be passed in registers \\
-o file		& set name of output file \\
-g		& produce debugging information in resulting executable \\
-S		& generate assembly listing file \\
-masm=intel	& produce listing in Intel syntax \\
-fno-inline	& do not inline functions \\
\hline
\end{tabular}
\end{center}


}
\RU{\myparagraph{GCC 4.4.1}

\lstinputlisting[caption=GCC 4.4.1,style=customasmx86]{patterns/12_FPU/3_comparison/x86/GCC_RU.asm}

\myindex{x86!\Instructions!FUCOMPP}
\FUCOMPP~--- это почти то же что и \FCOM, только выкидывает из стека оба значения после сравнения, 
а также несколько иначе реагирует на \q{не-числа}.

\myindex{Не-числа (NaNs)}
Немного о \emph{не-числах}.

FPU умеет работать со специальными переменными, которые числами не являются и называются \q{не числа} или 
\gls{NaN}.
Это бесконечность, результат деления на ноль, и так далее. Нечисла бывают \q{тихие} и \q{сигнализирующие}. 
С первыми можно продолжать работать и далее, а вот если вы попытаетесь совершить какую-то операцию 
с сигнализирующим нечислом, то сработает исключение.

\myindex{x86!\Instructions!FCOM}
\myindex{x86!\Instructions!FUCOM}
Так вот, \FCOM вызовет исключение если любой из операндов какое-либо нечисло.
\FUCOM же вызовет исключение только если один из операндов именно \q{сигнализирующее нечисло}.

\myindex{x86!\Instructions!SAHF}
\label{SAHF}
Далее мы видим \SAHF (\emph{Store AH into Flags})~--- это довольно редкая инструкция в коде, не использующим FPU. 
8 бит из \AH перекладываются в младшие 8 бит регистра статуса процессора в таком порядке:

\input{SAHF_LAHF}

\myindex{x86!\Instructions!FNSTSW}
Вспомним, что \FNSTSW перегружает интересующие нас биты \CThreeBits в \AH, 
и соответственно они будут в позициях 6, 2, 0 в регистре \AH:

\input{C3_in_AH}

Иными словами, пара инструкций \INS{fnstsw  ax / sahf} перекладывает биты \CThreeBits в флаги \ZF, \PF, \CF.

Теперь снова вспомним, какие значения бит \CThreeBits будут при каких результатах сравнения:

\begin{itemize}
\item Если $a$ больше $b$ в нашем случае, то биты \CThreeBits должны быть выставлены так: 0, 0, 0.
\item Если $a$ меньше $b$, то биты будут выставлены так: 0, 0, 1.
\item Если $a=b$, то так: 1, 0, 0.
\end{itemize}
% TODO: table?

Иными словами, после трех инструкций \FUCOMPP/\FNSTSW/\SAHF возможны такие состояния флагов:

\begin{itemize}
\item Если $a>b$ в нашем случае, то флаги будут выставлены так: \GTT{ZF=0, PF=0, CF=0}.
\item Если $a<b$, то флаги будут выставлены так: \GTT{ZF=0, PF=0, CF=1}.
\item Если $a=b$, то так: \GTT{ZF=1, PF=0, CF=0}.
\end{itemize}
% TODO: table?

\myindex{x86!\Instructions!SETcc}
\myindex{x86!\Instructions!JNBE}
Инструкция \SETNBE выставит в \AL единицу или ноль в зависимости от флагов и условий. 
Это почти аналог \JNBE, за тем лишь исключением, что \SETcc
\footnote{\emph{cc} это \emph{condition code}}
выставляет 1 или 0 в \AL, а \Jcc делает переход или нет. 
\SETNBE запишет 1 только если \GTT{CF=0} и \GTT{ZF=0}. Если это не так, то запишет 0 в \AL.

\CF будет 0 и \ZF будет 0 одновременно только в одном случае: если $a>b$.

Тогда в \AL будет записана 1, последующий условный переход \JZ выполнен не будет 
и функция вернет~\GTT{\_a}. 
В остальных случаях, функция вернет~\GTT{\_b}.
}
\FR{\subsection{GCC}
\myindex{GCC}

Quelques options utiles qui ont été utilisées dans ce livre.

\begin{center}
\begin{tabular}{ | l | l | }
\hline
\HeaderColor option & 
\HeaderColor signification \\
\hline
-Os		& optimiser la taille du code \\
-O3		& optimisation maximale \\
-regparm=	& nombre d'arguments devant être passés dans les registres \\
-o file		& définir le nom du fichier de sortie \\
-g		& mettre l'information de débogage dans l'exécutable généré \\
-S		& générer un fichier assembleur \\
-masm=intel	& construire le code source en syntaxe Intel \\
-fno-inline	& ne pas mettre les fonctions en ligne \\
\hline
\end{tabular}
\end{center}


}
\DE{\myparagraph{GCC 4.4.1}

\lstinputlisting[caption=GCC 4.4.1,style=customasmx86]{patterns/12_FPU/3_comparison/x86/GCC_DE.asm}

\myindex{x86!\Instructions!FUCOMPP}
\FUCOMPP{} ist fast wie like \FCOM, nimmt aber beide Werte vom Stand und
behandelt \q{undefinierte Zahlenwerte} anders.


\myindex{Non-a-numbers (NaNs)}
Ein wenig über \emph{undefinierte Zahlenwerte}.

Die FPU ist in der Lage mit speziellen undefinieten Werten, den sogenannten
\emph{not-a-number}(kurz \gls{NaN}) umzugehen. Beispiele sind etwa der Wert
unendlich, das Ergebnis einer Division durch 0, etc. Undefinierte Werte können
entwder \q{quiet} oder \q{signaling} sein. Es ist möglich mit \q{quiet} NaNs zu
arbeiten, aber beim Versuch einen Befehl auf \q{signaling} NaNs auszuführen,
wird eine Exception geworfen. 

\myindex{x86!\Instructions!FCOM}
\myindex{x86!\Instructions!FUCOM}
\FCOM erzeugt eine Exception, falls irgendein Operand ein \gls{NaN} ist.
\FUCOM erzeugt eine Exception nur dann, wenn ein Operand eine \q{signaling}
\gls{NaN} (SNaN) ist.

\myindex{x86!\Instructions!SAHF}
\label{SAHF}
Der nächste Befehl ist \SAHF (\emph{Store AH into Flags})~---es handelt sich
hierbei um einen seltenen Befehl, der nicht mit der FPU zusammenhängt.
8 Bits aus AH werden in die niederen 8 Bit der CPU Flags in der folgenden
Reihenfolge verschoben:

\input{SAHF_LAHF}

\myindex{x86!\Instructions!FNSTSW}
Erinnern wir uns, dass \FNSTSW die für uns interessanten Bits (\CThreeBits) auf
den Stellen 6,2,0 im AH Register setzt:

\input{C3_in_AH}
Mit anderen Worten: der Befehl \INS{fnstsw ax / sahf} verschiebt \CThreeBits
nach \ZF, \PF und \CF. 

Überlegen wir uns auch die Werte der \CThreeBits in unterschiedlichen Szenarien:

\begin{itemize} 
  \item Falls in unserem Beispiel $a$ größer als $b$ ist, dann werden die
  \CThreeBits auf 0,0,0 gesetzt.
  \item Falls $a$ kleiner als $b$ ist, werden die Bits auf 0,0,1 gesetzt.
  \item Falls $a=b$, dann werden die Bits auf 1,0,0 gesetzt.
\end{itemize}
% TODO: table?
Mit anderen Worten, die folgenden Zustände der CPU Flags sind nach drei
\FUCOMPP/\FNSTSW/\SAHF Befehlen möglich:

\begin{itemize}
\item Falls $a>b$, werden die CPU Flags wie folgt gesetzt \GTT{ZF=0, PF=0,
CF=0}.
\item Falls $a<b$, werden die CPU Flags wie folgt gesetzt: \GTT{ZF=0, PF=0,
CF=1}.
\item Und falls $a=b$, dann gilt: \GTT{ZF=1, PF=0, CF=0}.
\end{itemize}
% TODO: table?

\myindex{x86!\Instructions!SETcc}
\myindex{x86!\Instructions!JNBE}
Abhängig von den CPU Flags und Bedingungen, speichert \SETNBE entweder 1 oder 0
in AL.
Es ist also quasi das Gegenstück von \JNBE mit dem Unterschied, dass \SETcc

Depending on the CPU flags and conditions, \SETNBE stores 1 or 0 to AL. 
It is almost the counterpart of \JNBE, with the exception that \SETcc
\footnote{\emph{cc} is \emph{condition code}} eine 1 oder 0 in \AL speichert, aber
\Jcc tatsächlich auch springt.
\SETNBE speicher 1 nur, falls \GTT{CF=0} und \GTT{ZF=0}.
Wenn dies nicht der Fall ist, dann wird 0 in \AL gespeichert.

Nur in einem Fall sind \CF und \ZF beide 0: falls $a>b$.

In diesem Fall wird 1 in \AL gespeichert, der nachfolgende \JZ Sprung wird nicht
ausgeführt und die Funktion liefert {\_a} zurück. In allen anderen Fällen wird
{\_b} zurückgegeben.
}

\subsection{GDB}
\myindex{GDB}
\label{sec:GDB_cheatsheet}

% FIXME: in Russian table doesn't fit!

\RU{Некоторые команды, которые были использованы в книге}\EN{Some of commands we used in this book}\DE{Einige nützliche Optionen die in diesem Buch genutzt werden}%
\FR{Quelques commandes que nous avons utilisées dans ce livre}:

\small
\begin{center}
\begin{tabular}{ | l | l | }
\hline
\HeaderColor \RU{опция}\EN{option}\DE{Option}\FR{option} & 
\HeaderColor \RU{значение}\EN{meaning}\DE{Bedeutung} \\
\hline
break filename.c:number		& \RU{установить точку останова на номере строки в исходном файле}
					\EN{set a breakpoint on line number in source code}
                    \DE{Setzen eines Breakpoints in der angegebenen Zeile}%
					\FR{mettre un point d'arrêt à la ligne number du code source} \\
break function			& \RU{установить точку останова на функции}\EN{set a breakpoint on function}\DE{Setzen eines Breakpoints in der Funktion}%
\FR{mettre un point d'arrêt sur une fonction} \\
break *address			& \RU{установить точку останова на адресе}\EN{set a breakpoint on address}\DE{Setzen eines Breakpoints auf Adresse}%
\FR{mettre un point d'arrêt à une adresse} \\
b				& \dittoclosing \\
p variable			& \RU{вывести значение переменной}\EN{print value of variable}\DE{Ausgabe eines Variablenwerts}%
\FR{afficher le contenu d'une variable} \\
run				& \RU{запустить}\EN{run}\DE{Starten}\FR{démarrer} \\
r				& \dittoclosing \\
cont				& \RU{продолжить исполнение}\EN{continue execution}\DE{Ausführung fortfahren}\FR{continuer l'exécution} \\
c				& \dittoclosing \\
bt				& \RU{вывести стек}\EN{print stack}\DE{Stack ausgeben}\FR{afficher la pile} \\
set disassembly-flavor intel	& \RU{установить Intel-синтаксис}\EN{set Intel syntax}\DE{Intel-Syntax nutzen}%
\FR{utiliser la syntaxe Intel} \\
disas				& disassemble current function \\
disas function			& \RU{дизассемблировать функцию}\EN{disassemble function}\DE{Funktion disassemblieren}\FR{désassembler la fonction} \\
disas function,+50		& disassemble portion \\
disas \$eip,+0x10		& \dittoclosing \\
disas/r				& \EN{disassemble with opcodes}\RU{дизассемблировать с опкодами}\DE{mit OpCodes disassemblieren}%
\FR{désassembler avec les opcodes} \\
info registers			& \RU{вывести все регистры}\EN{print all registers}\DE{Ausgabe aller Register}\FR{afficher tous les registres} \\
info float			& \RU{вывести FPU-регистры}\EN{print FPU-registers}\DE{Ausgabe der FPU-Register}\FR{afficher les registres FPU} \\
info locals			& \RU{вывести локальные переменные (если известны)}\EN{dump local variables (if known)}\DE{(bekannte) lokale Variablen ausgeben}%
\FR{afficher les variables locales} \\
x/w ...				& \RU{вывести память как 32-битные слова}\EN{dump memory as 32-bit word}\DE{Speicher als 32-Bit-Wort ausgeben}%
\FR{afficher la mémoire en mot de 32-bit} \\
x/w \$rdi			& \RU{вывести память как 32-битные слова}\EN{dump memory as 32-bit word}\DE{Speicher als 32-Bit-Wort ausgeben}%
\FR{afficher la mémoire en mot de 32-bit} \\
				& \RU{по адресу в \TT{RDI}}\EN{at address in \TT{RDI}}\DE{an Adresse in \TT{RDI}}\FR{à l'adresse dans \TT{RDI}} \\

x/10w ...			& \RU{вывести 10 слов памяти}\EN{dump 10 memory words}\DE{10 Speicherworte ausgeben}%
\FR{afficher 10 mots de la mémoire} \\
x/s ...				& \RU{вывести строку из памяти}\EN{dump memory as string}\DE{Speicher als Zeichenkette ausgeben}%
\FR{afficher la mémoire en tant que chaîne} \\
x/i ...				& \RU{трактовать память как код}\EN{dump memory as code}\DE{Speicher als Code ausgeben}%
\FR{afficher la mémoire en tant que code} \\
x/10c ...			& \RU{вывести 10 символов}\EN{dump 10 characters}\DE{10 Zeichen ausgeben}%
\FR{afficher 10 caractères} \\
x/b ...				& \RU{вывести байты}\EN{dump bytes}\DE{Bytes ausgeben}\FR{afficher des octets} \\
x/h ...				& \RU{вывести 16-битные полуслова}\EN{dump 16-bit halfwords}\DE{16-Bit-Halbworte ausgeben}%
\FR{afficher en demi-mots de 16-bit} \\
x/g ...				& \RU{вывести 64-битные слова}\EN{dump giant (64-bit) words}\DE{große (64-Bit-) Worte ausgeben}%
\FR{afficher des mots géants (64-bit)} \\
finish				& \RU{исполнять до конца функции}\EN{execute till the end of function}\DE{bis Funktionsende fortfahren}%
\FR{exécuter jusqu'à la fin de la fonction} \\
next				& \RU{следующая инструкция (не заходить в функции)}
					\EN{next instruction (don't dive into functions)}
					\DE{Nächste Anweisung (nicht in Funktion springen)}
					\FR{instruction suivante (ne pas descendre dans les fonctions)} \\
step				& \RU{следующая инструкция (заходить в функции)}
					\EN{next instruction (dive into functions)}
					\DE{Nächste Anweisung (in Funktion springen)}
					\FR{instruction suivante (descendre dans les fonctions)} \\
set step-mode on		& \RU{не использовать информацию о номерах строк при использовании команды step}
					\EN{do not use line number information while stepping}
					\DE{Beim schrittweisen Ausführen keine Zeilennummerninfos nutzen}
					\FR{ne pas utiliser l'information du numéro de ligne en exécutant pas à pas} \\
frame n				& \RU{переключить фрейм стека}\EN{switch stack frame}\DE{Stack-Frame tauschen}\FR{échanger la stack frame} \\
info break			& \RU{список точек останова}\EN{list of breakpoints}\DE{Breakpoints schauen}%
\FR{afficher les points d'arrêt} \\
del n				& \RU{удалить точку останова}\EN{delete breakpoint}\DE{Breakpoints löschen}\FR{effacer un point d'arrêt} \\
set args ...			& \RU{установить аргументы командной строки}\EN{set command-line arguments}\DE{Aufrufparameter setzen}%
\FR{définir les arguments de la ligne de commande} \\
\hline
\end{tabular}
\end{center}
\normalsize



}
% TODO split
\part*{\AcronymsUsed}

\addcontentsline{toc}{part}{\AcronymsUsed}

\begin{acronym}

\RU{	\acro{OS}[ОС]{Операционная Система}
	\acro{FAQ}[ЧаВО]{Часто задаваемые вопросы}
	\acro{OOP}[ООП]{Объектно-Ориентированное Программирование}
	\acro{PL}[ЯП]{Язык Программирования}
	\acro{PRNG}[ГПСЧ]{Генератор псевдослучайных чисел}
	\acro{ROM}[ПЗУ]{Постоянное запоминающее устройство}
	\acro{ALU}[АЛУ]{Арифметико-логическое устройство}
	\acro{PID}{ID программы/процесса}
	\acro{LF}{Line feed (подача строки) (10 или '\textbackslash{}n' в \CCpp)}
	\acro{CR}{Carriage return (возврат каретки) (13 или '\textbackslash{}r' в \CCpp)}
	\acro{LIFO}{Last In First Out (последним вошел, первым вышел)}
	\acro{MSB}{Most significant bit (самый старший бит)} % NOT BYTE!
	\acro{LSB}{Least significant bit (самый младший бит)} % NOT BYTE!
	\acro{NSA}[АНБ]{Агентство национальной безопасности}
	\acro{CFB}{Режим обратной связи по шифротексту (Cipher Feedback)}
	\acro{CSPRNG}{Криптографически стойкий генератор псевдослучайных чисел (cryptographically secure pseudorandom number generator)}
        \acro{PC}{Program Counter. IP/EIP/RIP в x86/64. PC в ARM.}
        \acro{SP}{\gls{stack pointer}. SP/ESP/RSP в x86/x64. SP в ARM.}
}
\EN{	\acro{OS}{Operating System}
	\acro{FAQ}{Frequently Asked Questions}
	\acro{OOP}{Object-Oriented Programming}
	\acro{PL}{Programming Language}
	\acro{PRNG}{Pseudorandom Number Generator}
	\acro{ROM}{Read-Only Memory}
	\acro{ALU}{Arithmetic Logic Unit}
	\acro{PID}{Program/process ID}
	\acro{LF}{Line Feed (10 or '\textbackslash{}n' in \CCpp)}
	\acro{CR}{Carriage Return (13 or '\textbackslash{}r' in \CCpp)}
	\acro{LIFO}{Last In First Out}
	\acro{MSB}{Most Significant Bit} % NOT BYTE!
	\acro{LSB}{Least Significant Bit} % NOT BYTE!
	\acro{NSA}{National Security Agency}
	\acro{CFB}{Cipher Feedback}
	\acro{CSPRNG}{Cryptographically Secure Pseudorandom Number Generator}
	\acro{ABI}{Application Binary Interface}
        \acro{PC}{Program Counter. IP/EIP/RIP in x86/64. PC in ARM.}
        \acro{SP}{\gls{stack pointer}. SP/ESP/RSP in x86/x64. SP in ARM.}
}
\ES{	\acro{OS}[SO]{Sistema Operativo}
	\acro{FAQ}{Preguntas Frecuentes}
	\acro{OOP}[POO]{Programaci\'on Orientada a Objetos}
	\acro{PL}[LP]{Lenguaje de Programaci\'on}
	\acro{PRNG}[GPAN]{Generador Pseudo-Aleatorio de N\'umeros}
	\acro{ROM}{Memoria de Solo Lectura}
	\acro{ALU}{Unidad Aritm\'etica L\'ogica}
        \acro{PC}{Program Counter. IP/EIP/RIP en x86/64. PC en ARM.}
        \acro{SP}{\gls{stack pointer}. SP/ESP/RSP en x86/x64. SP en ARM.}
}
\DE{	\acro{OS}[BS]{Betriebssystem}
	\acro{FAQ}{Häufig gestellte Fragen}
	\acro{OOP}{Objektorientierte Programmierung}
	\acro{PL}[PS]{Programmiersprache}
	\acro{PRNG}{Pseudozufallszahlen-Generator}
	\acro{ALU}[ALE]{Arithmetisch-logische Einheit}
}
\IT{	\acro{OS}{Sistema Operativo (Operating System)}
	\acro{FAQ}{Domande Frequente (Frequently Asked Questions)}
	\acro{OOP}{Programmazione ad oggetti (Object-Oriented Programming)}
	\acro{PL}{Linguaggio di programmazione (Programming Language)}
	\acro{PRNG}{Generatore di numeri pseudo-casuali (Pseudorandom Number Generator)}
	\acro{ROM}{Memoria di sola lettura (Read-Only Memory)}
	\acro{ALU}{Unità aritmetica e logica (Arithmetic Logic Unit)}
	\acro{PID}{Identificatore di processo (Program/process ID}
	\acro{LF}{Line Feed (10 o '\textbackslash{}n' in \CCpp)}
	\acro{CR}{Carriage Return (13 o '\textbackslash{}r' in \CCpp)}
	\acro{LIFO}{Ultimo arrivato primo ad uscire (Last In First Out)}
	\acro{MSB}{Bit più significativo (Most Significant Bit)} % NOT BYTE!
	\acro{LSB}{Bit meno significativo (Least Significant Bit)} % NOT BYTE!
	\acro{NSA}{Agenzia di sicurezza nazionale(National Security Agency)}
	\acro{CFB}{Cipher Feedback}
	\acro{CSPRNG}{Cryptographically Secure Pseudorandom Number Generator}
	\acro{ABI}{Application Binary Interface}
}
\NL{	\acro{OS}{\NL{Operating System}}
	\acro{FAQ}{\NL{Veelvoorkomende vragen}}
	\acro{OOP}{\NL{Object-Oriented Programmeren}}
	\acro{PL}[PT]{\NL{Programmeertaal}}
	\acro{PRNG}{\NL{Pseudorandom number generator}}
	\acro{ROM}{\NL{Read-only memory}}
	\acro{ALU}{\NL{Arithmetic logic unit}}
}
\FR{	\acro{OS}[OS]{Système d'exploitation (Operating System)}
	\acro{FAQ}{Foire Aux Questions}
	\acro{OOP}[POO]{Programmation orientée objet}
	\acro{PL}[LP]{Langage de programmation}
	\acro{PRNG}{Nombre généré pseudo-aléatoirement}
	\acro{ROM}{Mémoire morte}
	\acro{ALU}[UAL]{Unité arithmétique et logique}
	\acro{PID}{ID d'un processus}
	\acro{LF}{Line feed (10 ou '\textbackslash{}n' en \CCpp)}
	\acro{CR}{Carriage return (13 ou '\textbackslash{}r' en \CCpp)}
	\acro{LIFO}{Dernier entré, premier sorti}
	\acro{MSB}{Bit le plus significatif} % NOT BYTE!
	\acro{LSB}{Bit le moins significatif} % NOT BYTE!
	\acro{NSA}{National Security Agency (Agence Nationale de la Sécurité)} % translation not used in French
	\acro{CFB}{Cipher Feedback}
	\acro{CSPRNG}{Cryptographically Secure Pseudorandom Number Generator (générateur de nombres pseudo-aléatoire cryptographiquement sûr)}
	\acro{ABI}{Application Binary Interface}
        \acro{PC}{Program Counter. IP/EIP/RIP dans x86/64. PC dans ARM.}
        \acro{SP}{\gls{stack pointer}. SP/ESP/RSP dans x86/x64. SP dans ARM.}
}
\JA{	\acro{OS}{オペレーティングシステム}
	\acro{FAQ}{よくある質問}
	\acro{OOP}{オブジェクト指向プログラミング}
	\acro{PL}{プログラミング言語}
	\acro{PRNG}{擬似乱数生成器}
	\acro{ROM}{読み取り専用メモリ}
	\acro{ALU}{算術論理ユニット}
	\acro{PID}{プログラム/プロセスID}
	\acro{LF}{ラインフィード (\CCpp で10 または '\textbackslash{}n')}
	\acro{CR}{キャリッジリターン (\CCpp で13 または '\textbackslash{}r')}
	\acro{LIFO}{後入れ先出し}
	\acro{MSB}{最上位ビット} % NOT BYTE!
	\acro{LSB}{最下位ビット} % NOT BYTE!
	\acro{NSA}{国家安全保障局}
	\acro{CFB}{暗号フィードバック}
	\acro{CSPRNG}{暗号論的擬似乱数生成器}
	\acro{ABI}{アプリケーション・バイナリー・インタフェース}
}
\PL{	\acro{OS}{System operacyjny (Operating System)}
	\acro{FAQ}{Najczęściej zadawane pytania (Frequently Asked Questions)}
	\acro{OOP}{Programowanie obiektowe (Object-Oriented Programming)}
	\acro{PL}{Język programowania (Programming Language)}
	\acro{PRNG}{Generator liczb pseudolosowych (Pseudorandom Number Generator)}
	\acro{ROM}{Pamięć tylko do odczytu (Read-Only Memory)}
	\acro{ALU}{Jednostka arytmetyczno-logiczna (Arithmetic Logic Unit)}
	\acro{PID}{Identyfikator procesu/programu (Program/process ID)}
	\acro{LF}{Line Feed (LF) (10 lub '\textbackslash{}n' w \CCpp)}
	\acro{CR}{Carriage Return (CR) (13 lub '\textbackslash{}r' w \CCpp)}
	\acro{LIFO}{Ostatni na wejściu, pierwszy na wyjściu (Last In First Out)}
	\acro{MSB}{Najbardziej znaczący bit (Most Significant Bit)} % NOT BYTE!
	\acro{LSB}{Najmniej znaczący bit (Least Significant Bit)} % NOT BYTE!
	\acro{NSA}{National Security Agency}
	\acro{CFB}{Tryb sprzężenia zwrotnego szyfrogramu (Cipher Feedback)}
	\acro{CSPRNG}{Kryptograficznie bezpieczny generator liczb pseudolosowych (Cryptographically Secure Pseudorandom Number Generator)}
	\acro{ABI}{Interfejs binarny aplikacji (Application Binary Interface)}
	\acro{PC}{Program Counter. IP/EIP/RIP w x86/64. PC w ARM.}
	\acro{SP}{\gls{stack pointer}. SP/ESP/RSP w x86/x64. SP w ARM.}
}

\acro{RA}{\ReturnAddress}
\acro{PE}{Portable Executable\PL{ (format plików wykonywalnych w systemach Windows)}}
\acro{DLL}{Dynamic-Link Library}
\acro{LR}{Link Register}
\acro{IDA}{
	\RU{Интерактивный дизассемблер и отладчик, разработан \href{https://hex-rays.com/}{Hex-Rays}}%
	\EN{Interactive Disassembler and Debugger developed by \href{https://hex-rays.com/}{Hex-Rays}}%
	\IT{\ac{TBT} by \href{https://hex-rays.com/}{Hex-Rays}}%
	\ES{Desensamblador Interactivo y depurador desarrollado por \href{https://hex-rays.com/}{Hex-Rays}}%
	\NL{Interactive Disassembler en debugger ontwikkeld door \href{https://hex-rays.com}{Hex-Rays}}
	\DE{Interaktiver Disassembler und Debugger entwickelt von \href{https://hex-rays.com/}{Hex-Rays}}%
	\PL{Interaktywny deasembler i debugger rozwijany przez \href{https://hex-rays.com/}{Hex-Rays}}%
	\FR{Désassembleur interactif et débogueur développé par \href{https://hex-rays.com/}{Hex-Rays}}%
	\JA{\href{https://hex-rays.com/}{Hex-Rays} によって開発されたインタラクティブなディスアセンブラ・デバッガ}%
}
\acro{IAT}{Import Address Table}
\acro{INT}{Import Name Table}
\acro{RVA}{Relative Virtual Address}
\acro{VA}{Virtual Address}
\acro{OEP}{Original Entry Point}
\acro{MSVC}{Microsoft Visual C++}
\acro{MSVS}{Microsoft Visual Studio}
\acro{ASLR}{Address Space Layout Randomization}
\acro{MFC}{Microsoft Foundation Classes}
\acro{TLS}{Thread Local Storage}
\acro{AKA}{
	\EN{Also Known As}%
	\FR{Also Known As --- Aussi connu sous le nom de}%
	\RU{Also Known As --- Также известный как}%
	\ES{Also Known As --- Tambi\'en Conocido Como}%
	\NL{Also Known As --- Ook gekend als}%
	\IT{Also Known As --- anche conosciuto come}%
	\JA{別名}%
	\DEph{}%
	\PL{Also Known As --- znany również jako}%
}
\acro{CRT}{C Runtime library}
\acro{CPU}{Central Processing Unit}
\acro{GPU}{Graphics Processing Unit}
\acro{FPU}{Floating-Point Unit}
\acro{CISC}{Complex Instruction Set Computing}
\acro{RISC}{Reduced Instruction Set Computing}
\acro{GUI}{Graphical User Interface}
\acro{RTTI}{Run-Time Type Information}
\acro{BSS}{Block Started by Symbol}
\acro{SIMD}{Single Instruction, Multiple Data}
\acro{BSOD}{Blue Screen of Death}
\acro{DBMS}{Database Management Systems}
\acro{ISA}{Instruction Set Architecture\RU{ (Архитектура набора команд)}\PL{ (architektura listy rozkazów)}}
\acro{CGI}{Common Gateway Interface}
\acro{HPC}{High-Performance Computing}
\acro{SOC}{System on Chip}
\acro{SEH}{Structured Exception Handling}
\acro{ELF}{\RU{Executable and Linkable Format: Формат исполняемых файлов, использующийся в Linux и некоторых других *NIX}
\EN{Executable and Linkable Format: Executable File format widely used in *NIX systems including Linux}
\FR{Executable and Linkable Format: Format de fichier exécutable couramment utilisé sur les systèmes *NIX, Linux inclus}
\JA{Executable and Linkable Format: Linuxを含め*NIXシステムで広く使用される実行ファイルフォーマット}
\IT{Executable and Linkable Format: Formato di file eseguibile largamente utilizzato nei sistemi *NIX, Linux incluso}
\DE{Executable and Linkable Format: \DEph{}}
\PL{Executable and Linkable Format: Format plików wykonywalnych używany w systemach z rodziny *NIX, w szczególności na Linuksie}}
\acro{TIB}{Thread Information Block}
\acro{TEA}{Tiny Encryption Algorithm}
\acro{PIC}{Position Independent Code}
\acro{NAN}{Not a Number}
\acro{NOP}{No Operation}
\acro{BEQ}{(PowerPC, ARM) Branch if Equal}
\acro{BNE}{(PowerPC, ARM) Branch if Not Equal}
\acro{BLR}{(PowerPC) Branch to Link Register}
\acro{XOR}{eXclusive OR\RU{ (исключающее \q{ИЛИ})}\FR{ (OU exclusif)}}
\acro{MCU}{Microcontroller Unit}
\acro{RAM}{Random-Access Memory}
\acro{GCC}{GNU Compiler Collection}
\acro{EGA}{Enhanced Graphics Adapter}
\acro{VGA}{Video Graphics Array}
\acro{API}{Application Programming Interface}
\acro{ASCII}{American Standard Code for Information Interchange}
\acro{ASCIIZ}{ASCII Zero (\RU{ASCII-строка заканчивающаяся нулем}\EN{null-terminated ASCII string}
\FR{chaîne ASCII terminée par un octet nul (à zéro)}\JA{ヌル終端文字列})}
\acro{IA64}{Intel Architecture 64 (Itanium)}
\acro{EPIC}{Explicitly Parallel Instruction Computing}
\acro{OOE}{Out-of-Order Execution}
\acro{MSDN}{Microsoft Developer Network}
\acro{STL}{(\Cpp) Standard Template Library}
\acro{PODT}{(\Cpp) Plain Old Data Type}
\acro{HDD}{Hard Disk Drive}
\acro{VM}{Virtual Memory\RU{ (виртуальная память)}\FR{ (mémoire virtuelle)}}
\acro{WRK}{Windows Research Kernel}
\acro{GPR}{General Purpose Registers\RU{ (регистры общего пользования)}\PL{ (rejestry ogólnego przeznaczania)}}
\acro{SSDT}{System Service Dispatch Table}
\acro{RE}{Reverse Engineering}
\acro{RAID}{Redundant Array of Independent Disks}
\acro{SSE}{Streaming SIMD Extensions}
\acro{BCD}{Binary-Coded Decimal}
\acro{BOM}{Byte Order Mark}
\acro{GDB}{GNU Debugger}
\acro{FP}{Frame Pointer}
\acro{MBR}{Master Boot Record}
\acro{JPE}{Jump Parity Even (\RU{инструкция x86}\EN{x86 instruction}\FR{instruction x86}\JA{x86命令}\DEph{})}
\acro{CIDR}{Classless Inter-Domain Routing}
\acro{STMFD}{Store Multiple Full Descending (\DEph{}\RU{инструкция ARM}\EN{ARM instruction}\FR{instruction ARM}\JA{ARM命令})}
\acro{LDMFD}{Load Multiple Full Descending (\DEph{}\RU{инструкция ARM}\EN{ARM instruction}\FR{instruction ARM}\JA{ARM命令})}
\acro{STMED}{Store Multiple Empty Descending (\DEph{}\RU{инструкция ARM}\EN{ARM instruction}\FR{instruction ARM}\JA{ARM命令})}
\acro{LDMED}{Load Multiple Empty Descending (\DEph{}\RU{инструкция ARM}\EN{ARM instruction}\FR{instruction ARM}\JA{ARM命令})}
\acro{STMFA}{Store Multiple Full Ascending (\DEph{}\RU{инструкция ARM}\EN{ARM instruction}\FR{instruction ARM}\JA{ARM命令})}
\acro{LDMFA}{Load Multiple Full Ascending (\DEph{}\RU{инструкция ARM}\EN{ARM instruction}\FR{instruction ARM}\JA{ARM命令})}
\acro{STMEA}{Store Multiple Empty Ascending (\DEph{}\RU{инструкция ARM}\EN{ARM instruction}\FR{instruction ARM}\JA{ARM命令})}
\acro{LDMEA}{Load Multiple Empty Ascending (\DEph{}\RU{инструкция ARM}\EN{ARM instruction}\FR{instruction ARM}\JA{ARM命令})}
\acro{APSR}{(ARM) Application Program Status Register}
\acro{FPSCR}{(ARM) Floating-Point Status and Control Register}
\acro{RFC}{Request for Comments}
\acro{TOS}{Top of Stack\RU{ (вершина стека)}}
\acro{LVA}{(Java) Local Variable Array\RU{ (массив локальных переменных)}}
\acro{JVM}{Java Virtual Machine}
\acro{JIT}{Just-In-Time compilation}
\acro{CDFS}{Compact Disc File System}
\acro{CD}{Compact Disc}
\acro{ADC}{Analog-to-Digital Converter}
\acro{EOF}{End of File\DEph{}\RU{ (конец файла)}\FR{ (fin de fichier)}\JA{(ファイル終端)}}
\acro{TBT}{To be Translated. The presence of this acronym in this place means that the English version has some new/modified content which is to be translated and placed right here.}
\acro{DIY}{Do It Yourself}
\acro{MMU}{Memory Management Unit}
\acro{DES}{Data Encryption Standard}
\acro{MIME}{Multipurpose Internet Mail Extensions}
\acro{DBI}{Dynamic Binary Instrumentation}
\acro{XML}{Extensible Markup Language}
\acro{JSON}{JavaScript Object Notation}
\acro{URL}{Uniform Resource Locator}
\acro{ISP}{Internet Service Provider}
\acro{IV}{Initialization Vector}
\acro{RSA}{Rivest Shamir Adleman}
\acro{CPRNG}{Cryptographically secure PseudoRandom Number Generator}
\acro{GiB}{Gibibyte}
\acro{CRC}{Cyclic redundancy check}
\acro{AES}{Advanced Encryption Standard}
\acro{GC}{Garbage Collector}
\acro{IDE}{Integrated development environment}
\acro{BB}{Basic Block}

\end{acronym}


\bookmarksetup{startatroot}

\clearpage
\phantomsection
\addcontentsline{toc}{chapter}{%
    \RU{Глоссарий}%
    \EN{Glossary}%
    \ES{Glosario}%
    \PTBRph{}%
    \DE{Glossar}%
    \PL{Słownik terminów}%
    \IT{Glossario}%
    \THAph{}\NLph{}%
    \FR{Glossaire}%
    \JA{用語}
    \TR{Bolum}
}
\printglossaries

\clearpage
\phantomsection
\printindex

\end{document}
