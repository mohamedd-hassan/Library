\section{Building Intel Pin}

Building Intel Pin for Windows may be tricky.
This is my working recipe.

\begin{itemize}

\item Unpack the latest Intel Pin to, say, \verb|C:\pin-3.20\|

\item Install latest Cygwin, to, say, \verb|c:\cygwin64|

\item Install MSVC 2015 or newer.

\item (If needed) in \verb|C:\pin-3.20\source\tools\SimpleExamples\makefile.rules|, add your pintool to the \verb|TEST_TOOL_ROOTS| list.

\item Open "VS2015 x86 Native Tools Command Prompt". Type:

\begin{lstlisting}
set PATH=%PATH%;c:\cygwin64/bin
cd c:\pin-3.20\source\tools
make all TARGET=ia32
\end{lstlisting}

(It will require many UNIX utilities, which are available under \verb|c:\cygwin64\bin| path.
Hence the \verb|PATH| variable is to be modified.)

Now pintools are now in \verb|c:\pin-3.20\source\tools\...\obj-ia32|

\item For winx64, use "x64 Native Tools Command Prompt" and run:

\begin{lstlisting}
make all TARGET=intel64
\end{lstlisting}

\item Run a pintool:

\begin{lstlisting}
c:\pin-3.20\pin.exe -t C:\pin-3.20\source\tools\SimpleExamples\obj-ia32\XOR_ins.dll -- program.exe arguments
\end{lstlisting}

\end{itemize}

Intel Pin 3.20 require at least Windows 10.
On Windows 7 you may got an error like:

\begin{lstlisting}
A: source\pincrt\injector_w\maincrt_windows.cpp: LEVEL_BASE::RootMain: 139: assertion failed: NT_SUCCESS(ntStatus)

NO STACK TRACE AVAILABLE
\end{lstlisting}

It also works in Windows Sever 2012.

