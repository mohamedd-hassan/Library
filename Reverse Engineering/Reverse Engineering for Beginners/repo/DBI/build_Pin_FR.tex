% to be synced with English version
\section{Compiler Pin}

Compiler Pin pour Windows peut s'avérer délicat.
Ceci est ma recette qui fonctionne.

\begin{itemize}

\item Décompacter le dernier Pin, disons, \verb|C:\pin-3.7\|

\item Installer le dernier Cygwin, dans, disons, \verb|c:\cygwin64|

\item Installer MSVC 2015 ou plus récent.

\item Ouvrir le fichier \verb|C:\pin-3.7\source\tools\Config\makefile.default.rules|, remplacer \verb|mkdir -p $@| par \verb|/bin/mkdir -p $@|

\item (Si nécessaire) dans \verb|C:\pin-3.7\source\tools\SimpleExamples\makefile.rules|, ajouter votre pintool à la liste \verb|TEST_TOOL_ROOTS|.

\item Ouvrir "VS2015 x86 Native Tools Command Prompt". Taper:

\begin{lstlisting}
cd c:\pin-3.7\source\tools\SimpleExamples
c:\cygwin64\bin\make all TARGET=ia32
\end{lstlisting}

Maintenant les outils pintools sont dans \verb|c:\pin-3.7\source\tools\SimpleExamples\obj-ia32|

\item Pour winx64, utiliser "x64 Native Tools Command Prompt" et lancer:

\begin{lstlisting}
c:\cygwin64\bin\make all TARGET=intel64
\end{lstlisting}

\item Lancer pintool:

\begin{lstlisting}
c:\pin-3.7\pin.exe -t C:\pin-3.7\source\tools\SimpleExamples\obj-ia32\XOR_ins.dll -- program.exe arguments
\end{lstlisting}

\end{itemize}
