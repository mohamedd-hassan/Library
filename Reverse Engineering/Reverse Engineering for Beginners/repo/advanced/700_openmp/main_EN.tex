\mysection{OpenMP}
\label{openmp}
\myindex{OpenMP}

OpenMP is one of the simplest ways to parallelize simple algorithms.

\myindex{Bitcoin}

As an example, let's try to build a program to compute a cryptographic \emph{nonce}.

In my simplistic example, 
the \emph{nonce} is a number added to the plain unencrypted text in order to produce a hash with some specific 
features.

For example, at some step, the Bitcoin protocol requires to find such \emph{nonce} so the resulting hash
contains a specific number of consecutive zeros.
This is also called \emph{proof of work}
(i.e., the system proves that it did some intensive calculations and spent some time for it).

\myindex{SHA512}
My example is not related to Bitcoin in any way, 
it will try to add numbers to the \q{hello, world!\_}
string in order to find such number that when 
\q{hello, world!\_<number>} is hashed with the SHA512 algorithm, it will contain at least 3 zero bytes.

Let's limit our brute-force to the interval in
0..INT32\_MAX-1 (i.e., \TT{0x7FFFFFFE} or 2147483646).

The algorithm is pretty straightforward:

\lstinputlisting[style=customc]{advanced/700_openmp/openmp_example.c}

The \TT{check\_nonce()} function just adds a number to the string, 
hashes it with the SHA512 algorithm and checks for 3 zero bytes in the result.

A very important part of the code is:

\begin{lstlisting}[style=customc]
	#pragma omp parallel for
	for (i=0; i<INT32_MAX; i++)
		check_nonce (i);
\end{lstlisting}

Yes, that simple, without \TT{\#pragma} 
we just call \TT{check\_nonce()} for each number from 0 to 
\TT{INT32\_MAX} (\TT{0x7fffffff} or 2147483647).
With \TT{\#pragma}, the compiler adds some special 
code which slices the loop interval into smaller ones,
to run them on all \ac{CPU} cores available
\footnote{N.B.: This is intentionally simplest possible
example, but in practice, the usage of OpenMP can be harder and more complex}.

The example can be compiled
\footnote{sha512.(c|h) and u64.h 
files can be taken from the OpenSSL library:
\url{http://www.openssl.org/source/}}
in MSVC 2012:
% FIXME1: \footnote{other source code files can be downloaded here: ...} 

\begin{lstlisting}
cl openmp_example.c sha512.obj /openmp /O1 /Zi /Faopenmp_example.asm
\end{lstlisting}

Or in GCC:

\begin{lstlisting}
gcc -fopenmp 2.c sha512.c -S -masm=intel
\end{lstlisting}

\subsection{MSVC}

Now this is how MSVC 2012 generates the main loop:

\lstinputlisting[caption=MSVC 2012,style=customasmx86]{advanced/700_openmp/MSVC_loop.asm}

All functions prefixed by \TT{vcomp} 
are OpenMP-related and are stored in the 
vcomp*.dll file.
So here a group of threads is started.

Let's take a look on \TT{\_main\$omp\$1}:

\lstinputlisting[caption=MSVC 2012,style=customasmx86]{advanced/700_openmp/MSVC_1.asm}

This function is to be started $n$ 
times in parallel, where $n$ is the number of \ac{CPU} cores.\\
\TT{vcomp\_for\_static\_simple\_init()} 
calculates the interval for the for() 
construct for the current
thread, depending on the current thread's number.

The loop's start and end values are stored in the \TT{\$T1} and \TT{\$T2} local variables.
You may also notice \TT{7ffffffeh} (or 2147483646) 
as an argument to the 
\TT{vcomp\_for\_static\_simple\_init()}
function---this is the number of iterations for the whole loop, to be divided evenly.

Then we see a new loop with a call to the 
\TT{check\_nonce()} function, which does all the work.

Let's also add some code at the beginning of the \TT{check\_nonce()} function to gather statistics about
the arguments with which the function has been called.

This is what we see when we run it:

\begin{lstlisting}
threads=4
...
checked=2800000
checked=3000000
checked=3200000
checked=3300000
found (thread 3): [hello, world!_1611446522]. seconds spent=3
__min[0]=0x00000000 __max[0]=0x1fffffff
__min[1]=0x20000000 __max[1]=0x3fffffff
__min[2]=0x40000000 __max[2]=0x5fffffff
__min[3]=0x60000000 __max[3]=0x7ffffffe
\end{lstlisting}

Yes, the result is correct, the first 3 bytes are zeros:

\begin{lstlisting}
C:\...\sha512sum test
000000f4a8fac5a4ed38794da4c1e39f54279ad5d9bb3c5465cdf57adaf60403
df6e3fe6019f5764fc9975e505a7395fed780fee50eb38dd4c0279cb114672e2 *test
\end{lstlisting}

The running time is $\approx2..3$ seconds on 4-core Intel Xeon E3-1220 3.10 GHz.
In the task manager we see 5 threads: 
1 main thread + 4 more.
No further optimizations are done to keep this example as small and clear as possible.
But probably it can be done much faster.
My \ac{CPU} has 4 cores, that is why OpenMP 
started exactly 4 threads.

By looking at the statistics table we can clearly see how the loop has been sliced into 4 even parts.
Oh well, almost even, if we don't consider the last bit.

There are also pragmas for 
\glslink{atomic operation}{atomic operations}.

Let's see how this code is compiled:

\begin{lstlisting}[style=customc]
	#pragma omp atomic
	checked++;

	#pragma omp critical
	if ((checked % 100000)==0)
		printf ("checked=%d\n", checked);
\end{lstlisting}

\lstinputlisting[caption=MSVC 2012,style=customasmx86]{advanced/700_openmp/MSVC_2.asm}

As it turns out, the \TT{vcomp\_atomic\_add\_i4()} 
function in the vcomp*.dll 
is just a tiny function
with the \TT{LOCK XADD} instruction\footnote{
Read more about LOCK prefix: \myref{x86_lock}} in it.

\TT{vcomp\_enter\_critsect()} 
eventually calling win32 \ac{API} function \\
\TT{EnterCriticalSection()}
\footnote{You can read more about critical sections 
here: \myref{critical_sections}}.

\subsection{GCC}

GCC 4.8.1 
produces a program which shows exactly the same statistics table, 

so, GCC's implementation divides the loop in parts in the same fashion.

\begin{lstlisting}[caption=GCC 4.8.1,style=customasmx86]
	mov	edi, OFFSET FLAT:main._omp_fn.0
	call	GOMP_parallel_start
	mov	edi, 0
	call	main._omp_fn.0
	call	GOMP_parallel_end
\end{lstlisting}

Unlike MSVC's implementation, what GCC code does is to start 3 threads,
and run the fourth in the current thread. So there are 4 threads instead of the 5 in MSVC.

Here is the \TT{main.\_omp\_fn.0} function:
 
\lstinputlisting[caption=GCC 4.8.1,style=customasmx86]{advanced/700_openmp/GCC_1.asm}

Here we see the division clearly: by calling \\
\TT{omp\_get\_num\_threads()} and \TT{omp\_get\_thread\_num()}

we get the number of threads running, and also the current thread's number, and then determine the loop's interval.
Then we run \TT{check\_nonce()}.

GCC also inserted the \TT{LOCK ADD} 

instruction right in the code, unlike MSVC, which generated a call to a separate DLL function:

\lstinputlisting[caption=GCC 4.8.1,style=customasmx86]{advanced/700_openmp/GCC_2.asm}

The functions prefixed with GOMP are from GNU OpenMP library.
Unlike vcomp*.dll, its source code is freely available: 
\href{https://github.com/mirrors/gcc/tree/master/libgomp}{GitHub}.

