\subsubsection{memset()}
\myindex{\CStandardLibrary!memset()}

\myparagraph{\Example \#1}

\lstinputlisting[caption=32 bytes,style=customc]{\CURPATH/str_mem/memset_32.c}

\myindex{x86!\Instructions!MOV}
De nombreux compilateurs ne génèrent pas un appel à memset() pour de petits blocs,
mais insèrent plutôt un paquet de \MOV{}s:

\lstinputlisting[caption=GCC 4.9.1 x64 \Optimizing,style=customasmx86]{\CURPATH/str_mem/memset_32_GCC491_x64_O3.s}

\myindex{Unrolled loop}
À propos, ça nous rappelle le déroulement de boucles:
\myref{ARM_unrolled_loops}.

\myparagraph{\Example \#2}

\lstinputlisting[caption=67 bytes,style=customc]{\CURPATH/str_mem/memset_67.c}

Lorsque la taille du bloc n'est pas un multiple de 4 ou 8, les compilateurs se comportent
différemment.

\myindex{x86!\Instructions!MOV}
Par exemple, MSVC 2012 continue à insérer des \TT{MOV}s:

\lstinputlisting[caption=MSVC 2012 x64 \Optimizing,style=customasmx86]{\CURPATH/str_mem/memset_67_MSVC2012_x64_Ox.asm}

\myindex{x86!\Instructions!STOSQ}
\dots tandis que GCC utilise \TT{REP STOSQ}, en concluant que cela sera plus petit
qu'un paquet de \TT{MOV}s:

\lstinputlisting[caption=GCC 4.9.1 x64 \Optimizing,style=customasmx86]{\CURPATH/str_mem/memset_67_GCC491_x64_O3.s}
