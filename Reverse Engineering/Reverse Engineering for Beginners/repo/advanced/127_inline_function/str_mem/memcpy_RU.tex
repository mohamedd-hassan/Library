\subsubsection{memcpy()}
\myindex{\CStandardLibrary!memcpy()}

\myparagraph{Короткие блоки}
\label{copying_short_blocks}

\myindex{x86!\Instructions!MOV}
Если нужно скопировать немного байт, то, нередко, 
\TT{memcpy()} заменяется на несколько инструкций \MOV.


\lstinputlisting[caption=пример с memcpy(),style=customc]{\CURPATH/str_mem/memcpy_7.c}

\lstinputlisting[caption=\Optimizing MSVC 2010,style=customasmx86]{\CURPATH/str_mem/memcpy_7_MSVC_2010_Ox.asm}

\lstinputlisting[caption=\Optimizing GCC 4.8.1,style=customasmx86]{\CURPATH/str_mem/memcpy_7_GCC_O3.s}

Обычно это происходит так: в начале копируются 4-байтные блоки, затем 16-битное слово (если нужно), 
затем последний байт (если нужно).


Точно так же при помощи \MOV копируются структуры: \myref{short_struct_copying_using_MOV}.

\myparagraph{Длинные блоки}

Здесь компиляторы ведут себя по-разному.

\lstinputlisting[caption=пример с memcpy(),style=customc]{\CURPATH/str_mem/memcpy.c}

\myindex{x86!\Instructions!MOVSD}
При копировании 128 байт, MSVC может обойтись одной инструкцией \TT{MOVSD} (ведь 128 кратно 4):


\lstinputlisting[caption=\Optimizing MSVC 2010,style=customasmx86]{\CURPATH/str_mem/memcpy_128_MSVC_2010_Ox.asm}

При копировании 123-х байт, в начале копируется 30 32-битных слов при помощи \TT{MOVSD} 
(это 120 байт), 
затем копируется 2 байта при помощи \TT{MOVSW}, 
затем еще один байт при помощи \TT{MOVSB}.


\lstinputlisting[caption=\Optimizing MSVC 2010,style=customasmx86]{\CURPATH/str_mem/memcpy_123_MSVC_2010_Ox.asm}

GCC во всех случаях вставляет большую универсальную функцию, работающую для всех размеров блоков:


\lstinputlisting[caption=\Optimizing GCC 4.8.1,style=customasmx86]{\CURPATH/str_mem/memcpy_GCC.s}

Универсальные функции копирования блоков обычно работают по следующей схеме: 
вычислить, сколько 32-битных слов
можно скопировать, затем сделать это при помощи \TT{MOVSD}, затем скопировать остатки.


\myindex{SIMD}
Более сложные функции копирования используют \ac{SIMD} и учитывают выравнивание в памяти.

Как пример функции strlen() использующую SIMD
: \myref{SIMD_strlen}.

