\mysection{setjmp/longjmp}

\myindex{\CStandardLibrary!setjmp}
\myindex{\CStandardLibrary!longjmp}

setjmp/longjmp is a mechanism in C which is very similar to throw/catch mechanism in C++ and other higher-level \ac{PL}s.
\myindex{zlib}
Here is an example from zlib:

\begin{lstlisting}[style=customc]
...

    /* return if bits() or decode() tries to read past available input */
    if (setjmp(s.env) != 0)             /* if came back here via longjmp(), */
        err = 2;                        /*  then skip decomp(), return error */
    else
	err = decomp(&s); /* decompress */

...

    /* load at least need bits into val */
    val = s->bitbuf;
    while (s->bitcnt < need) {
        if (s->left == 0) {
            s->left = s->infun(s->inhow, &(s->in));
	    if (s->left == 0) longjmp(s->env, 1); /* out of input */

...

        if (s->left == 0) {
            s->left = s->infun(s->inhow, &(s->in));
	    if (s->left == 0) longjmp(s->env, 1); /* out of input */
\end{lstlisting}
( zlib/contrib/blast/blast.c )

Call to \TT{setjmp()} saves current \ac{PC}, \ac{SP} and other registers into \TT{env} structure, then it returns 0.

In case of error, \TT{longjmp()} \emph{teleporting} you into the point after right after \TT{setjmp()} call,
as if \TT{setjmp()} call returned non-null value (which was passed to \TT{longjmp()}).
\myindex{UNIX!fork}
This reminds as \TT{fork()} syscall in UNIX.

Now let's take a look on distilled example:

\lstinputlisting[style=customc]{advanced/625_setjmp/1.c}

If we run it, we will see:

\begin{lstlisting}
f1() begin
f2() begin
Error 1234
\end{lstlisting}

\TT{jmp\_buf} structure usually comes undocumented, to preserve forward compatibility.

Let's see how setjmp() implemented in MSVC 2013 x64:

\lstinputlisting[style=customasmx86]{advanced/625_setjmp/setjmp_VC2013_x64.asm}

It just populates jmp\_buf structure with current values of almost all registers.
Also, current value of \ac{RA} is taken from the stack and saved in jmp\_buf:
it will be used as new value of \ac{PC} in future.

Now longjmp():

\lstinputlisting[style=customasmx86]{advanced/625_setjmp/longjmp_VC2013_x64.asm}

It just restores (almost) all registers, takes \ac{RA} from structure and jumps there.
This effectively works as if setjmp() returned to caller.
Also, \TT{RAX} is set to be equal to the second argument of longjmp().
This works as if setjmp() returned non-zero value at first place.

As a side effect of \ac{SP} restoration, all values in stack which has been set and used between setjmp() and longjmp()
calls are just dropped.
They will not be used anymore.
Hence, longjmp() usually jumps backwards
\footnote{However, there are some people who can use it for much more complicated things,
imitating coroutines, etc: \url{https://www.embeddedrelated.com/showarticle/455.php},
\url{http://fanf.livejournal.com/105413.html}}.

This implies that, unlike in throw/catch mechanism in C++, no memory will be freed, no destructors will be called, etc.
Hence, this technique sometimes can be dangerous.
Nevertheless, it's still quite popular. It's still used in \oracle.

It also has unexpected side-effect: if some buffer has been overflown inside of a function (maybe due to remote attack),
and a function wants to report error, and it calls longjmp(), overwritten stack part just gets unused.

As an exercise, you can try to understand, why not all registers are saved.
Why XMM0-XMM5 and other registers are skipped?

