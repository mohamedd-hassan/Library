\subsection{Как это работает}

Из школьной математики, мы можем вспомнить, что деление на 9 может быть заменено на умножение на $\frac{1}{9}$.
На самом деле, для чисел с плавающей точкой, иногда компиляторы так и делают,
например, инструкциея \INS{FDIV} в x86-коде может быть заменена на \INS{FMUL}.
По крайней мере MSVC 6.0 заменяет деление на 9 на умножение на $0.111111...$ и иногда нельзя быть уверенным в том,
какая операция была в оригинальном исходном коде.

Но когда мы работаем с целочисленными значениями и целочисленными регистрами CPU, мы не можем использовать дроби.
Но мы можем переписать дробь так:

% FIXME: equation size
\begin{center}
$result = \frac{x}{9} = x \cdot \frac{1}{9} = x \cdot \frac{1 \cdot MagicNumber}{9 \cdot MagicNumber}$
\end{center}

Учитывая тот факт, что деление на $2^n$ очень быстро (при помощи сдвигов), теперь нам нужно найти такой $MagicNumber$,
для которого следующее уравнение будет справедливо: $2^n = 9 \cdot MagicNumber$.

Деление на $2^{32}$ в каком-то смысле скрыто: младшие 32 бита произведения в EAX не используются (выкидываются),
только старшие 32 бита произведения (в EDX) используются и затем сдвигаются еще на 1 бит.

Другими словами, только что увиденный код на ассемблере умножает на {\Large $\frac{954437177}{2^{32+1}}$},
или делит на {\Large $\frac{2^{32+1}}{954437177}$}.
Чтобы найти делитель, нужно просто разделить числитель на знаменатель.
Используя Wolfram Alpha, мы получаем результат 8.99999999.... (что близко к 9).

% TODO ref to https://yurichev.com/blog/signed_division_using_shifts/

Читайте больше об этом в \InSqBrackets{\HenryWarren 10-3}.

Многие люди не замечают ``скрытое'' деление на $2^{32}$ или $2^{64}$,
когда младшая 32-битная часть произведения (или 64-битная) не используется.
Поэтому деление через умножение в коде поначалу трудно для понимания.

В \MathForProg есть еще одно объяснение.

