\subsection{Tableau non dimensionné dans une structure C}

Nous pouvons trouver certaines structures win32 avec la dernier champ défini comme
un tableau d'un élément.

\begin{lstlisting}[style=customc]
typedef struct _SYMBOL_INFO {
  ULONG   SizeOfStruct;
  ULONG   TypeIndex;
  
  ...

  ULONG   MaxNameLen;
  TCHAR   Name[1];
} SYMBOL_INFO, *PSYMBOL_INFO;
\end{lstlisting}

( \url{https://msdn.microsoft.com/en-us/library/windows/desktop/ms680686(v=vs.85).aspx} )

Ceci est une astuce, signifiant que le dernier champ est un tableau de taille inconnue,
qui doit être calculée lors de l'allocation de la structure.

Pourquoi: le champ \emph{Name} peut-être court, donc pourquoi le définir avec une sorte
de constante \emph{MAX\_NAME} qui peut être 128, 256 et même plus ?

Pourquoi ne pas utiliser un pointeur à la place ? Alors vous devez allouer deux blocs:
un pour la structure et  pour la chaîne.
Ceci peut-être plus lent et peut nécessiter un plus large surplus de mémoire.
Donc, vous devez déréférencer le pointeur (i.e., lire l'adresse de la chaîne dans
la structure)---ce n'est pas un problème, mais certains disent que c'est un coût supplémentaire.

Ceci est connu comme l'\emph{astuce struct}: \url{http://c-faq.com/struct/structhack.html}.

Exemple:

\begin{lstlisting}[style=customc]
#include <stdio.h>

struct st
{
	int a;
	int b;
	char s[];
};

void f (struct st *s)
{
	printf ("%d %d %s\n", s->a, s->b, s->s);
	// f() can't replace s[] with bigger string - size of allocated block is unknown at this point
};

int main()
{
#define STRING "Hello!"
	struct st *s=malloc(sizeof(struct st)+strlen(STRING)+1); // incl. terminating zero
	s->a=1;
	s->b=2;
	strcpy (s->s, STRING);
	f(s);
};
\end{lstlisting}

En quelques mots, ça fonctionne car le C n'a pas de vérification des bornes d'un
tableau. Tout tableau est traité comme ayant une taille infinie.

Problème: après l'allocation, la taille entière du bloc alloué pour la structure
est inconnue (excepté pour le gestionnaire de mémoire), donc vous ne pouvez pas remplacer
une chaîne par un chaîne plus large.
Vous pourriez faire quelque chose comme ça si le champ était déclaré comme quelque
chose comme \emph{s[MAX\_NAME]}.

Autrement dit, vous avez une structure plus un tableau (ou une chaîne) fusionnés
ensemble dans un bloc de mémoire alloué unique.
Un autre problème est que vous ne pouvez évidemment pas déclarer deux tableaux comme
ceci dans une structure unique, ou déclarer un autre champ après un tel tableau.

Les cieux compilateurs nécessitent de déclarer le tableau avec au moins un élément:
\emph{s[1]}, les plus récents permettent de le déclarer comme un tableau de taille
variable:\emph{s[]}.
Ceci est aussi appelé \emph{membre tableau flexible}.

En lire plus à ce sujet dans
GCC documentation\footnote{\url{https://gcc.gnu.org/onlinedocs/gcc/Zero-Length.html}},
MSDN documentation\footnote{\url{https://msdn.microsoft.com/en-us/library/b6fae073.aspx}}.

Dennis Ritchie (un des créateurs du C) a appelé ce truc \q{amabilité non voulue avec
l'implémentation du C} (peut-être pour reconnaître la nature astucieuse de cette ruse).

Aimez le ou non, utilisez le ou non:
il est encore une autre démonstration de la façon dont les structures sont stockées
dans la mémoire, c'est pourquoi j'en ai parlé.

