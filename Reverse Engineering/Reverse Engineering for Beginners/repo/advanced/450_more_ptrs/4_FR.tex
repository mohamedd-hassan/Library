\subsection{Pointeurs nuls}

\subsubsection{``Null pointer assignment'' erreur du temps de MS-DOS}

\myindex{MS-DOS}
Des anciens peuvent se souvenir d'un message d'erreur bizarre du temps de MS-DOS:
``Null pointer assignment''.
Qu'est-ce que ça signifie?

Il n'est pas possible d'écrire à l'adresse mémoire zéro avec les OSs *NIX et Windows,
mais il est possible de le faire avec MS-DOS, à cause de l'absence de protection
de la mémoire.

\myindex{Turbo C++}
\myindex{Borland C++}
Donc, j'ai sorti mon ancien Turbo C++ 3.0 (qui fût renommer plus tard en Borland C++)
d'avant les années 1990s et essayé de compiler ceci:

\begin{lstlisting}[style=customc]
#include <stdio.h>

int main()
{
	int *ptr=NULL;
	*ptr=1234;
	printf ("Now let's read at NULL\n");
	printf ("%d\n", *ptr);
};
\end{lstlisting}

C'est difficile à croire, mais ça fonctionne, sans erreur jusqu'à la sortie, toutefois:

\begin{lstlisting}[caption=Ancient Turbo C 3.0]
C:\TC30\BIN\1
Now let's read at NULL
1234
Null pointer assignment

C:\TC30\BIN>_
\end{lstlisting}

Plongeons un peu plus profondément dans le code du \ac{CRT} de Borland C++ 3.1, fichier \emph{c0.asm}:

\lstinputlisting[style=customasmx86]{advanced/450_more_ptrs/tmp2.lst}

Le modèle de mémoire de MS-DOS était vraiment bizarre (\myref{8086_memory_model}) et
ne vaut probablement pas la peine de s'y plonger, à moins d'être fan de rétro-computing
ou de rétro-gaming.
Une chose que nous devons garder à l'esprit est que le segment de mémoire (segment
de données inclus) dans MS-DOS, est un segment de mémoire dans lequel du code ou
des données sont stockés, mais contrairement au \ac{OS}s ``sérieux'', il commence
à l'adresse 0.

Et dans le \ac{CRT} de Borland C++, le segment de données commence avec 4 octets
à zéro puis la chaîne de copyright ``Borland C++ - Copyright 1991 Borland Intl.''.
L'intégrité des 4 octets à zéro et de la chaîne de texte est vérifiée en sortant,
et s'ils sont corrompus, le message d'erreur est affiché.

Mais pourquoi? Écrire à l'adresse zéro est une erreur courante en \CCpp, et si vous
faites cela sur *NIX ou Windows, votre application va planter.
MS-DOS n'a pas de protection de la mémoire, donc le \ac{CRT} doit vérifier ceci post-factum
et le signaler à la sortie.
Si vous voyez ce message, ceci signifie que votre programme à un certain point, a
écrit à l'adresse 0.

Notre programme le fait. Et ceci est pourquoi le nombre 1234 a été lu correctement:
car il a été écrit à la place des 4 premiers octets à zéro.
La somme de contrôle est incorrecte à la sortie (car le nombre y a été laissé), donc
le message a été affiché.

Ai-je raison?
J'ai récrit le programme pour vérifier mes suppositions:

\begin{lstlisting}[style=customc]
#include <stdio.h>

int main()
{
	int *ptr=NULL;
	*ptr=1234;
	printf ("Now let's read at NULL\n");
	printf ("%d\n", *ptr);
	*ptr=0; // psst, cover our tracks!
};
\end{lstlisting}

Ce programme s'exécute sans message d'erreur à la sortie.

Une telle méthode pour avertir en cas d'assignation du pointeur nul est pertinente
pour MS-DOS, peut-être qu'elle peut encore être utilisée de nos jours avec des \ac{MCU}s
à bas coût sans protection de la mémoire et/ou \ac{MMU}.

\subsubsection{Pourquoi voudrait-on écrire à l'adresse 0?}

Mais pourquoi un programmeur sain d'esprit écrirait du code écrivant quelque chose
à l'adresse 0?
Ça peut être fait accidentellement: par exemple, un pointeur doit être initialisé
au bloc de mémoire nouvellement alloué et ensuite passé à quelque fonction qui renvoie
des données à travers un pointeur.

\begin{lstlisting}[style=customc]
int *ptr=NULL;

\dots nous avons oublié d'allouer la mémoire et d'initialiser ptr

strcpy (ptr, buf); // strcpy() termine silencieusement car MS-DOS n'a pas de protection de la mémoire
\end{lstlisting}

Encore pire:

\begin{lstlisting}[style=customc]
int *ptr=malloc(1000);

\dots nous avons oublié de vérifier si la mémoire a réellement été allouée: c'est MS-DOS après tout et les ordinateurs ont une petite quantité de RAM,
\dots et être à court de RAM était très courant.
\dost si malloc() a renvoyé NULL, ptr sera aussi NULL.

strcpy (ptr, buf); // strcpy() termine silencieusement car MS-DOS n'a pas de protection de la mémoire
\end{lstlisting}

\subsubsection{Écrire sciemment à l'adresse 0}
\label{dmalloc_KILL_PROCESS}

\myindex{dmalloc}
\myindex{Core dump}
Voici un exemple tiré de dmalloc\footnote{\url{http://dmalloc.com/}},
une façon portable de générer un core dump, si les autres moyens ne sont pas disponibles:

\lstinputlisting{advanced/450_more_ptrs/dmalloc_KILL_PROCESS}

\subsubsection{NULL en \CCpp}

NULL en \CCpp est juste une macro qui est souvent définie comme ceci:

\begin{lstlisting}[style=customc]
#define NULL  ((void*)0)
\end{lstlisting}
( \href{https://github.com/wzhy90/linaro_toolchains/blob/8ff8ae680bac04558d10cc9626e12c4c2f6c1348/arm-cortex_a15-linux-gnueabihf/libc/usr/include/libio.h#L70}{fichier libio.h} )

\emph{void*} est un type de données reflétant le fait que c'est un pointeur, mais sur
une valeur d'un type de données inconnu (\emph{void}).

NULL est usuellement utilisé pour montrer l'absence d'un objet.
Par exemple, vous avez une liste simplement chaînée, et chaque n\oe{}ud a une valeur
(ou un pointeur sur une valeur) et un pointeur \emph{next}.
Pour montrer qu'il n'y a pas de n\oe{}ud suivant, 0 est stocké dans le champ \emph{next}.
(D'autres solutions sont pires.)
Peut-être pourriez-vous avoir un environnement fou où vous devriez allouer un bloc
de mémoire à l'adresse zéro. Comment indiqueriez-vous l'absence de n\oe{}ud suivant?
Avec une sorte de \emph{magic number}? Peut-être -1? Ou peut-être avec un bit additionnel?

Nous trouvons ceci dans Wikipédia:

\begin{framed}
\begin{quotation}
In fact, quite contrary to the zero page's original preferential use, some modern operating systems such as FreeBSD, Linux and Microsoft Windows[2] actually make the zero page inaccessible to trap uses of NULL pointers. 
\end{quotation}
\end{framed}
( \url{https://en.wikipedia.org/wiki/Zero_page} )

\subsubsection{Pointeur nul à une fonction}

Il est possible d'appeler une fonction avec son adresse.
Par exemple, je compile ceci avec MSVC 2010 et le lance dans Windows 7:

\begin{lstlisting}[style=customc]
#include <windows.h>
#include <stdio.h>

int main()
{
	printf ("0x%x\n", &MessageBoxA);
};
\end{lstlisting}

Le résultat est \emph{0x7578feae} et ne change pas après plusieurs lancements, car user32.dll
(où la fonction MessageBoxA se trouve) est toujours chargée à la même adresse.
Et aussi car \ac{ASLR} n'est pas activé (le résultat serait différent à chaque exécution
dans ce cas).

Appelons \emph{MessageBoxA()} par son adresse:

\begin{lstlisting}[style=customc]
#include <windows.h>
#include <stdio.h>

typedef int (*msgboxtype)(HWND hWnd, LPCTSTR lpText, LPCTSTR lpCaption,  UINT uType);

int main()
{
	msgboxtype msgboxaddr=0x7578feae;

	// force to load DLL into process memory, 
	// since our code doesn't use any function from user32.dll, 
	// and DLL is not imported
	LoadLibrary ("user32.dll");

	msgboxaddr(NULL, "Hello, world!", "hello", MB_OK);
};
\end{lstlisting}

Bizarre, mais ça fonctionne avec Windows 7 x86.

Ceci est communément utilisé dans les shellcodes, car il est difficile d'y appeler
des fonctions DLL par leur nom.
Et \ac{ASLR} est une contre-mesure.

Maintenant, ce qui est vraiment bizarre, quelques programmeurs C embarqué peuvent
être familiers avec un code comme ceci:

\begin{lstlisting}[style=customc]
int reset()
{
	void (*foo)(void) = 0;
	foo();
};
\end{lstlisting}

Qui voudrait appeler une fonction à l'adresse 0?
Ceci est un moyen portable de sauter à l'adresse zéro.
De nombreux micro-contrôleurs à bas coût n'ont pas de protection mémoire ou de \ac{MMU}
et après un reset, ils commencent à exécuter le code à l'adresse 0, où une sorte
de code d'initialisation est stocké.
Donc sauter à l'adresse 0 est un moyen de se réinitialiser.
On pourrait utiliser de l'assembleur inline, mais si ce n'est pas possible, cette
méthode portable est utilisable.

Ça compile même correctement avec mon GCC 4.8.4 sur Linux x64:

\begin{lstlisting}[style=customasmx86]
reset:
        sub     rsp, 8
        xor     eax, eax
        call    rax
        add     rsp, 8
        ret
\end{lstlisting}

Le fait que le pointeur de pile soit décalé n'est pas un problème: le code d'initialisation
dans les micro-contrôleurs ignorent en général les registres et l'état de la \ac{RAM}
et démarrent from scratch.

Et bien sûr, ce code planterait sur *NIX ou Windows à cause de la protection mémoire,
et même sans la protection mémoire, il n'y a pas de code à l'adresse 0.

GCC possède même des extensions non-standard, permettant de sauter à une adresse spécifique
plutôt qu'un appel à une fonction ici:
\url{http://gcc.gnu.org/onlinedocs/gcc/Labels-as-Values.html}.

