\mysection{Двойное отрицание}

Популярный способ\footnote{Хотя и спорный, потому что приводит к трудночитаемому коду} сконвертировать ненулевое значение
в 1 (или булево \emph{true}) и 0 в 0 (или булево \emph{false}) это \emph{!!statement}:

\begin{lstlisting}[style=customc]
int convert_to_bool(int a)
{
	return !!a;
};
\end{lstlisting}

Оптимизирующий GCC 5.4 x86:

\begin{lstlisting}[style=customasmx86]
convert_to_bool:
        mov     edx, DWORD PTR [esp+4]
        xor     eax, eax
        test    edx, edx
        setne   al
        ret
\end{lstlisting}

\INS{XOR} всегда очищает возвращаемое значение в \EAX, даже если \INS{SETNE} не сработает.
Т.е., \INS{XOR} устанавливает возвращаемое значение (по умолчанию) в 0.

\myindex{x86!\Instructions!SET}
Если входное значение не равно нулю (суффикс \TT{-NE} в инструкции \INS{SET}),
тогда 1 заносится в \AL, иначе \AL не модифицируется.

Почему \INS{SETNE} работает с младшей 8-битной частью регистра \EAX{}?
Потому что значение имеет только последний бит (0 или 1), а остальные биты были уже сброшены при помощи \INS{XOR}.

Следовательно, этот код на \CCpp может быть переписан так:

\begin{lstlisting}[style=customc]
int convert_to_bool(int a)
{
	if (a!=0)
		return 1;
	else
		return 0;
};
\end{lstlisting}

\dots или даже:

\begin{lstlisting}[style=customc]
int convert_to_bool(int a)
{
	if (a)
		return 1;
	else
		return 0;
};
\end{lstlisting}

Компиляторы, компилирующие для \ac{CPU} у которых нет инструкции близкой к
\INS{SET}, в этом случае, генерируют инструкции условного перехода, итд.

