\mysection{Использование const (const correctness)}
\myindex{\CLanguageElements!const}
\label{const_in_rdata}

Это незаслуженно малоиспользуемая возможность многих ЯП.
Тут можно почитать о её важности:
\href{https://isocpp.org/wiki/faq/const-correctness}{1},
\href{https://stackoverflow.com/questions/136880/sell-me-on-const-correctness}{2}.

Идеально, всё что вы не модифицируете, должно иметь модификатор \emph{const}.

Интересно, как \emph{const correctness} обеспечивается на низком уровне.
Локальные \emph{const}-переменные и аргументы ф-ций не проверяются во время исполнения (только во время компиляции).
Но глобальные переменные этого типа располагаются в сегментах данных только для чтения.

Вот пример который упадет, потому что, если будет скомпилирован MSVC для win32, глобальная переменная $a$ располагается
в сегменте \verb|.rdata|, который только для чтения:

\lstinputlisting[style=customc]{\CURPATH/ex1.c}

\emph{Анонимные} (не привязанные к имени переменной) строки в Си имеют тип \verb|const char*|.
Вы не можете их модифицировать:

\lstinputlisting[style=customc]{\CURPATH/ex2.c}

Это код упадет в Linux (``segmentation fault'') и в Windows, если скомпилирован MinGW.

GCC для Linux располагает все текстовые строки в сегменте данных \TT{.rodata}, который недвусмысленно защищен от записи
(``read only data''):

\lstinputlisting{\CURPATH/ex2.txt}

Когда ф-ция \verb|alter_string()| пытается там писать, срабатывает исключение.

Всё немного иначе в коде сгенерированном MSVC, строки располагаются в сегменте \TT{.data}, у которого нет флага \TT{READONLY}.
Оплошность разработчиков MSVC?

\lstinputlisting{\CURPATH/ex22.txt}

А в MinGW этой ошибки нет, и строки располагаются в сегменте \verb|.rdata|.

\subsection{Пересекающиеся const-строки}
\label{use_parts_of_C_strings}

Тот факт, что \emph{анонимная} Си-строка имеет тип \emph{const} (\myref{string_is_const_char}), 
и тот факт, что выделенные в сегменте констант Си-строки гарантировано неизменяемые (immutable), 
ведет к интересному следствию: компилятор может использовать определенную часть строки.

Вот простой пример:

\begin{lstlisting}[style=customc]

#include <stdio.h>

int f1()
{
	printf ("world\n");
}

int f2()
{
	printf ("hello world\n");
}

int main()
{
	f1();
	f2();
}
\end{lstlisting}

Среднестатистический компилятор с \CCpp (включая MSVC) выделит место для двух строк, но вот что делает GCC 4.8.1:

\lstinputlisting[caption=GCC 4.8.1 + листинг в IDA,style=customasmx86]{\CURPATH/two_strings.asm}

Действительно, когда мы выводим строку \q{hello world}, 
эти два слова расположены в памяти впритык друг к другу и \puts, вызываясь из функции \GTT{f2()}, вообще не знает,
что эти строки разделены. Они и не разделены на самом деле, они разделены
только \emph{виртуально}, в нашем листинге.

Когда \puts вызывается из \GTT{f1()}, он использует строку \q{world} плюс нулевой байт. \puts не знает, что там ещё есть какая-то строка перед этой!

Этот трюк часто используется (по крайней мере в GCC) и может сэкономить немного памяти.
Это близко к \emph{string interning}.

Еще один связанный с этим пример находится здесь: \myref{strstr_example}.



