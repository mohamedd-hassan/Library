\mysection{strstr()-Beispiel}
\label{strstr_example}
\myindex{\CStandardLibrary!strstr()}

Erinnern wir uns an die Tatsache, dass GCC manchmal Teile einer Zeichenkette nutzen kann: \myref{use_parts_of_C_strings}.

Die \emph{strstr()} \CCpp-Standard-Bibliotheksfunktion wird genutzt um das Auftreten einer Zeichenkette
in einer anderen zu finden.
Nachfolgend ein Beispiel für die Anwendung:

\begin{lstlisting}[style=customc]
#include <string.h>
#include <stdio.h>

int main()
{
	char *s="Hello, world!";
	char *w=strstr(s, "world");

	printf ("%p, [%s]\n", s, s);
	printf ("%p, [%s]\n", w, w);
};
\end{lstlisting}

Die Ausgabe ist:

\begin{lstlisting}
0x8048530, [Hello, world!]
0x8048537, [world!]
\end{lstlisting}

Der Unterschied zwischen der Adresse der Original-Zeichenkette und der Adresse des Substrings die
\emph{strstr()} zurück gibt ist 7.
Tatsächlich hat \q{Hello, } ja auch eine Länge von sieben Zeichen.

Der zweite \printf-Aufruf weiß nicht, dass weitere Zeichen vor der Zeichenkette sind und gibt lediglich die
Zeichen von der Mitte der Original-Zeichenkette bis zum Ende aus (markiert durch ein Null-Byte).
