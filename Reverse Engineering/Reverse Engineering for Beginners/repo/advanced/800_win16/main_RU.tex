\mysection{Windows 16-bit}
\myindex{Windows!Windows 3.x}

16-битные программы под Windows в наше время редки, хотя иногда можно поработать с ними, в смысле ретрокомпьютинга,
либо которые защищенные донглами (\myref{dongles}).

16-битные версии Windows были вплоть до 3.11.
95/98/ME также поддерживает 16-битный код, как и все 32-битные OS линейки \gls{Windows NT}.
64-битные версии \gls{Windows NT} не поддерживают 16-битный код вообще.

Код напоминает тот что под MS-DOS.

Исполняемые файлы имеют NE-тип (так называемый \q{new executable}).

Все рассмотренные здесь примеры скомпилированы компилятором OpenWatcom 1.9 используя эти опции:

\begin{lstlisting}
wcl.exe -i=C:/WATCOM/h/win/ -s -os -bt=windows -bcl=windows example.c
\end{lstlisting}

\subsection{\Example \#1}

\begin{lstlisting}[style=customc]
#include <windows.h>

int PASCAL WinMain( HINSTANCE hInstance,
                    HINSTANCE hPrevInstance,
                    LPSTR lpCmdLine,
                    int nCmdShow )
{
	MessageBeep(MB_ICONEXCLAMATION);
	return 0;
};
\end{lstlisting}

\begin{lstlisting}[style=customasmx86]
WinMain         proc near
                push    bp
                mov     bp, sp
                mov     ax, 30h ; '0'   ; MB\_ICONEXCLAMATION constant
                push    ax
                call    MESSAGEBEEP
                xor     ax, ax          ; return 0
                pop     bp
                retn    0Ah
WinMain         endp
\end{lstlisting}

\RU{Пока всё просто}\EN{Seems to be easy, so far}\FR{Ça semble facile, jusqu'ici}.

\subsection{\Example{} \#2}
\label{win16_messagebox}

\begin{lstlisting}[style=customc]
#include <windows.h>

int PASCAL WinMain( HINSTANCE hInstance,
                    HINSTANCE hPrevInstance,
                    LPSTR lpCmdLine,
                    int nCmdShow )
{
	MessageBox (NULL, "hello, world", "caption", MB_YESNOCANCEL);
	return 0;
};
\end{lstlisting}

\begin{lstlisting}[style=customasmx86]
WinMain         proc near
                push    bp
                mov     bp, sp
                xor     ax, ax          ; NULL
                push    ax
                push    ds
                mov     ax, offset aHelloWorld ; 0x18. "hello, world"
                push    ax
                push    ds
                mov     ax, offset aCaption ; 0x10. "caption"
                push    ax
                mov     ax, 3           ; MB\_YESNOCANCEL
                push    ax
                call    MESSAGEBOX
                xor     ax, ax          ; return 0
                pop     bp
                retn    0Ah
WinMain         endp

dseg02:0010 aCaption        db 'caption',0
dseg02:0018 aHelloWorld     db 'hello, world',0
\end{lstlisting}

\myindex{x86!\Instructions!RET}
\RU{Пара важных моментов: соглашение о передаче аргументов здесь \TT{PASCAL}: оно указывает что самый
первый аргумент должен передаваться первым}
\EN{Couple important things here: the \TT{PASCAL} calling convention dictates passing the first argument first} 
(\TT{MB\_YESNOCANCEL}), \RU{а самый последний аргумент --- последним}\EN{and the last argument---last} (NULL).
\RU{Это соглашение также указывает вызываемой функции восстановить}
\EN{This convention also tells the \gls{callee} to restore the} \gls{stack pointer}:
\RU{поэтому инструкция}\EN{hence the} \TT{RETN} \RU{имеет аргумент}\EN{instruction has} \TT{0Ah} 
\RU{означая что указатель нужно сдвинуть вперед на 10 байт во время возврата из функции}
\EN{as argument, which implies that the pointer has to be increased by 10 bytes when the function exits}.
\RU{Это как}\EN{It is like} stdcall (\myref{sec:stdcall}), \EN{but the arguments are passed in 
\q{natural} order}\RU{только аргументы передаются в \q{естественном} порядке}.

\RU{Указатели передаются парами: сначала сегмент данных, потом указатель внутри сегмента}
\EN{The pointers are passed in pairs: first the data segment is passed, then the pointer inside the segment}.
\RU{В этом примере только один сегмент, так что \TT{DS} всегда указывает на сегмент данных в исполняемом
файле}\EN{There is only one segment in this example, so \TT{DS} always points to the data segment of the executable}.


\subsection{\Example{} \#3}

\lstinputlisting[style=customc]{\CURPATH/ex3.c}

\lstinputlisting[style=customasmx86]{\CURPATH/ex3.lst}

\RU{Немного расширенная версия примера из предыдущей секции}
\EN{Somewhat extended example from the previous section}
\FR{Exemple un peu plus long de la section précédente}.

\subsection{\Example{} \#4}

\label{win16_32bit_values}

\lstinputlisting[style=customc]{\CURPATH/ex4.c}

\lstinputlisting[style=customasmx86]{\CURPATH/ex4.lst}

\myindex{MS-DOS}
\RU{32-битные значения (тип данных \TT{long} означает 32-бита, а \Tint здесь 16-битный) 
в 16-битном коде (и в MS-DOS и в Win16) передаются парами)}
\EN{32-bit values (the \TT{long} data type implies 32 bits, while \Tint is 16-bit)
in 16-bit code (both MS-DOS and Win16) are passed in pairs}.
\RU{Это так же как и 64-битные значения передаются в 32-битной среде}
\EN{It is just like when 64-bit values are used in a 32-bit environment} (\myref{sec:64bit_in_32_env}).

\TT{sub\_B2} \RU{здесь это библиотечная функция написанная разработчиками компилятора, делающая}
\EN{here is a library function written by the compiler's developers that does} \q{long multiplication}, \RU{т.е. перемножает
два 32-битных значения}\EN{i.e., multiplies two 32-bit values}.
\RU{Другие функции компиляторов делающие то же самое перечислены здесь}
\EN{Other compiler functions that do the same are listed here}: \myref{sec:MSVC_library_func}, \myref{sec:GCC_library_func}.

\myindex{x86!\Instructions!ADD}
\myindex{x86!\Instructions!ADC}
\EN{The}\RU{Пара инструкций} \TT{ADD}/\TT{ADC} \RU{используется для сложения этих составных значений}
\EN{instruction pair is used for addition of compound values}: 
\RU{\TT{ADD} может установить или сбросить флаг \TT{CF}, а \TT{ADC} будет использовать его после.}
\EN{\TT{ADD} may set/clear the \TT{CF} flag, and \TT{ADC} uses it after.}

\EN{The}\RU{Пара инструкций} \TT{SUB}/\TT{SBB} \RU{используется для вычитания}\EN{instruction pair is used for subtraction}: 
\RU{\TT{SUB} может установить или сбросить флаг \TT{CF}, \TT{SBB} будет использовать его после.}
\EN{\TT{SUB} may set/clear the \TT{CF} flag, \TT{SBB} uses it after.}

\RU{32-битные значения возвращаются из функций в паре регистров \TT{DX:AX}}
\EN{32-bit values are returned from functions in the \TT{DX:AX} register pair}.

\RU{Константы так же передаются как пары в}\EN{Constants are also passed in pairs in} \TT{WinMain()}\EN{ here}.

\myindex{x86!\Instructions!CWD}
\RU{Константа 123 типа \Tint в начале конвертируется (учитывая знак) в 32-битное значение 
используя инструкция \TT{CWD}}
\EN{The \Tint{}-typed 123 constant is first converted according to its sign into a 32-bit value using the \TT{CWD} instruction}.


\subsection{\Example{} \#5}
\label{win16_near_far_pointers}

\lstinputlisting[style=customc]{\CURPATH/ex5.c}

\lstinputlisting[style=customasmx86]{\CURPATH/ex5.lst}

\myindex{Intel!8086!\RU{Модель памяти}\EN{Memory model}}
\RU{Здесь мы можем увидеть разницу между указателями}
\EN{Here we see a difference between the so-called} \q{near} \RU{и указателями}\EN{pointers and the} \q{far} 
\RU{еще один ужасный артефакт сегментированной памяти 16-битного 8086}
\EN{pointers: another weird artifact of segmented memory in 16-bit 8086}.

\RU{Читайте больше об этом}\EN{You can read more about it here}: \myref{8086_memory_model}.

\RU{Указатели }\q{near} \RU{(\q{близкие}) это те которые указывают в пределах текущего сегмента}
\EN{pointers are those which point within the current data segment}.
\RU{Поэтому}\EN{Hence, the}\RU{, функция} \TT{string\_compare()} \RU{берет на вход только 2 16-битных
значения и работает с данными расположенными в сегменте, на который указывает \TT{DS}}\EN{function takes only
two 16-bit pointers, and accesses the data from the  segment that \TT{DS} points to} 
(\EN{The }\RU{инструкция}\TT{mov al, [bx]} \RU{на самом деле работает как}\EN{instruction actually works like} 
\TT{mov al, ds:[bx]} --- \TT{DS} \RU{используется здесь неявно}\EN{is implicit here}).

\RU{Указатели }\q{far} \RU{(далекие) могут указывать на данные в другом сегменте памяти}%
\EN{pointers are those which may point to data in another memory segment}.\\
\RU{Поэтому}\EN{Hence} \TT{string\_compare\_far()} \RU{берет на вход 16-битную пару как указатель, загружает старшую
часть в сегментный регистр \TT{ES} и обращается к данным через него}
\EN{takes the 16-bit pair as a pointer, loads the high part of it in the \TT{ES} segment register and accesses
the data through it}\\
(\TT{mov al, es:[bx]}).
\RU{Указатели }\q{far} \RU{также используются в моем win16-примере касательно}%
\EN{pointers are also used in my} \\
\TT{MessageBox()}\EN{ win16 example}: \myref{win16_messagebox}. 
\RU{Действительно, ядро Windows должно знать, из какого сегмента данных читать текстовые строки, так что ему нужна
полная информация}\EN{Indeed, the Windows kernel is not aware which data segment to use when accessing text strings,
so it need the complete information}.

\RU{Причина этой разница в том, что компактная программа вполне может обойтись одним сегментом данных размером 64 килобайта,
так что старшую часть указателя передавать не нужна (ведь она одинаковая везде)}
\EN{The reason for this distinction is that a compact program may use just one 64kb data segment, so it doesn't need
to pass the high part of the address, which is always the same}.
\RU{Б\`{о}льшие программы могут использовать несколько сегментов данных размером 64 килобайта,
так что нужно указывать каждый раз, в каком сегменте расположены данные}
\EN{A bigger program may use several 64kb data segments, so it needs to specify the segment of the data each time}.

\RU{То же касается и сегментов кода}\EN{It's the same story for code segments}.
\RU{Компактная программа может расположиться в пределах одного 64kb-сегмента, тогда
функции в ней будут вызываться инструкцией \TT{CALL NEAR}, 
а возвращаться управление используя \TT{RETN}.}
\EN{A compact program may have all executable code within one 64kb-segment, 
then all functions in it will be called using the \TT{CALL NEAR} instruction, 
and the code flow will be returned using \TT{RETN}.}
\RU{Но если сегментов кода несколько, тогда и адрес вызываемой функции будет задаваться парой, 
вызываться она будет используя
\TT{CALL FAR}, а возвращаться управление используя \TT{RETF}.}
\EN{But if there are several code segments, 
then the address of the function is to be specified by a pair,
it is to be called using the
\TT{CALL FAR} instruction, and the code flow is to be returned using \TT{RETF}.}

\RU{Это то что задается в компиляторе указывая}\EN{This is what is set in the compiler by specifying} \q{memory model}.

\RU{Компиляторы под MS-DOS и Win16 имели разные библиотеки под разные модели памяти: они отличались типами указателей для
кода и данных}\EN{The compilers targeting MS-DOS and Win16 have specific libraries for each memory model: they differ
by pointer types for code and data}.


\subsection{\Example{} \#6}

\lstinputlisting[style=customc]{\CURPATH/ex6.c}

\lstinputlisting[style=customasmx86]{\CURPATH/ex6.lst}

\myindex{\CStandardLibrary!time()}
\myindex{\CStandardLibrary!localtime()}
Время в формате UNIX это 32-битное значение, так что оно возвращается в паре регистров \TT{DX:AX} и сохраняется
в двух локальных 16-битных переменных.
Потом указатель на эту пару передается в функцию
\TT{localtime()}.
Функция \TT{localtime()} имеет структуру \TT{struct tm} расположенную у себя
где-то внутри, так что только указатель на нее возвращается. 
Кстати, это также означает, что функцию нельзя вызывать еще раз, пока её результаты не были использованы.

Для функций \TT{time()} и \TT{localtime()} используется
Watcom-соглашение о вызовах: первые четыре аргумента передаются через регистры
\TT{AX}, \TT{DX}, \TT{BX} и \TT{CX}, а остальные аргументы через стек.
Функции, использующие это соглашение, маркируется символом подчеркивания в конце имени.

Для вызова функции \TT{sprintf()} используется обычное соглашение \emph{cdecl} (\myref{cdecl}) вместо 
\TT{PASCAL} или Watcom, так что аргументы передаются привычным образом.

\subsubsection{Глобальные переменные}

Это тот же пример, только переменные теперь глобальные:

\lstinputlisting[style=customc]{\CURPATH/ex6_global.c}

\lstinputlisting[style=customasmx86]{\CURPATH/ex6_global.lst}

\TT{t} не будет использоваться, но компилятор создал код, записывающий в эту переменную.

Потому что он не уверен, может быть это значение будет прочитано где-то в другом модуле.



