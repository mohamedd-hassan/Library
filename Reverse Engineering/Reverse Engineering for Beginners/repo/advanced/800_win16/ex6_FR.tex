\subsection{\Example{} \#6}

\lstinputlisting[style=customc]{\CURPATH/ex6.c}

\lstinputlisting[style=customasmx86]{\CURPATH/ex6.lst}

\myindex{\CStandardLibrary!time()}
\myindex{\CStandardLibrary!localtime()}

Le temps UNIX est une valeur 32-bit, donc il est renvoyé dans la paire de registres
\TT{DX:AX} et est stocké dans deux variables locales 16-bit.
Puis, un pointeur sur la paire est passé à la fonction \TT{localtime()}.
La fonction \TT{localtime()} a une structure \TT{struct tm} allouée quelque part
dans les entrailles de la bibliothèque C, donc seul un pointeur est renvoyé.

À propos, ceci implique aussi que la fonction ne peut pas être appelée tant que le
résultat n'a pas été utilisé.

Pour les fonctions \TT{time()} et \TT{localtime()}, une convention d'appel Watcom
est utilisée ici:
les quatre premiers arguments sont passés dans les registres \TT{AX}, \TT{DX}, \TT{BX}
et \TT{CX}, et le reste des arguments par la pile.

Les fonctions utilisant cette convention sont aussi marquées par un souligné à la
fin de leur nom.

\TT{sprintf()} n'utilise pas la convention d'appel \TT{PASCAL}, ni la Watcom,\\ % varioref bug
donc les arguments sont passés de la manière \emph{cdecl} normale (\myref{cdecl}).

\subsubsection{Variables globales}

Ceci est le même exemple, mais cette fois les variables sont globales:

\lstinputlisting[style=customc]{\CURPATH/ex6_global.c}

\lstinputlisting[style=customasmx86]{\CURPATH/ex6_global.lst}

\TT{t} ne va pas être utilisée, mais le compilateur a généré le code qui stocke
la valeur.

Car il n'est pas sûr, peut-être que la valeur sera utilisée dans un autre module.
