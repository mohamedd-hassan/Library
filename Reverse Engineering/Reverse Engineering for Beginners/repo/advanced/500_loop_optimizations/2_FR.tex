\subsection{Autre optimisation de boucle}

Si vous traitez tous les éléments d'un tableau qui est situé dans la mémoire globale,
le compilateur peut l'optimiser.
Par exemple, calculons la somme de tous les éléments du tableau de 128 \emph{int}:

\begin{lstlisting}[style=customc]
#include <stdio.h>

int a[128];

int sum_of_a()
{
	int rt=0;
	
	for (int i=0; i<128; i++)
		rt=rt+a[i];

	return rt;
};

int main()
{
	// initialize
	for (int i=0; i<128; i++)
		a[i]=i;
	
	// calculate the sum
	printf ("%d\n", sum_of_a());
};
\end{lstlisting}

GCC 5.3.1 (x86) \Optimizing peut produire ceci (\IDA):

\lstinputlisting[style=customasmx86]{advanced/500_loop_optimizations/tmp1.lst}

Qu'est-ce que c'est que \TT{\_\_libc\_start\_main@@GLIBC\_2\_0} en \TT{0x080484C5}?
Ceci est un label situé juste après la fin du tableau \TT{a[]}.
Cette fonction peut être récrite comme ceci:

\begin{lstlisting}[style=customc]
int sum_of_a_v2()
{
	int *tmp=a;
	int rt=0;
	
	do
	{
		rt=rt+(*tmp);
		tmp++;
	}
	while (tmp<(a+128));

	return rt;
};
\end{lstlisting}

La première version a le compteur \emph{i}, et l'adresse de chaque élément du tableau
est calculée à chaque itération.
La seconde version est plus optimisée: le pointeur sur chaque élément du tableau
est toujours prêt et est déplacé de 4 octets à chaque itération.
Comment vérifier si la boucle est terminée?
Il suffit de comparer le pointeur avec l'adresse juste après la fin du tableau, qui
est dans notre cas l'adresse de la fonction \TT{\_\_libc\_start\_main()} importée de la Glibc 2.0.
Parfois ce genre de code est perturbant, et ceci est une astuce d'optimisation très
répandue, c'est pourquoi j'ai mis cet exemple.

Ma seconde version est très proche de ce que fait GCC, et lorsque je la compile,
le code est presque le même que dans la première version, mais les deux première
instructions sont échangées:

\lstinputlisting[style=customasmx86]{advanced/500_loop_optimizations/tmp2.lst}

Inutile de dire que cette optimisation n'est possible que si le compilateur peut
calculer l'adresse de la fin du tableau pendant la compilation.
Ceci se produit si le tableau est global et sa taille fixée.

Toutefois, si l'adresse du tableau est inconnue lors de la compilation, mais que la
taille est fixée, l'adresse du label juste après la fin du tableau peut être calculée.
% FIXME <!-- \ref{} to example -->

