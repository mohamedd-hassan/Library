\mysection{Ajustement de chaînes}
\newcommand{\CRLF}{\ac{CR}/\ac{LF}}

Un traitement de chaîne très courant est la suppression de certains caractères au
début et/où à la fin.

Dans cet exemple, nous allons travailler avec une fonction qui supprime tous les
caractères newline (\CRLF{}) à la fin de la chaîne entrée:

\lstinputlisting[style=customc]{\CURPATH/strtrim_FR.c}

L'argument en entrée est toujours renvoyé en sortie, ceci est pratique lorsque vous
voulez chaîner les fonctions de traitement de chaîne, comme c'est fait ici dans la
fonction \main.

\myindex{\CLanguageElements!Short-circuit}
La seconde partie de for() (\TT{str\_len>0 \&\& (c=s[str\_len-1])}) est appelé le
\q{short-circuit} en \CCpp et est très pratique \InSqBrackets{\CNotes 1.3.8}.

Les compilateurs \CCpp garantissent une séquence d'évaluation de gauche à droite.

Donc, si la première clause est fausse après l'évaluation, la seconde n'est pas évaluée.

% subsections
\subsubsection{MSVC: x64}

\myindex{x86-64}

Puisque nous travaillons ici avec des variables typées \Tint{}, qui sont toujours
32-bit en x86-64, nous voyons comment la partie 32-bit des registres (préfixés
avec \TT{E-}) est également utilisée ici. % TODO clarify
Lorsque l'on travaille avec des ponteurs, toutefois, les parties 64-bit des registres
sont utilisées, préfixés avec \TT{R-}.

\lstinputlisting[caption=MSVC 2012 x64,style=customasmx86]{patterns/04_scanf/3_checking_retval/ex3_MSVC_x64_FR.asm}


\myparagraph{ARM64}

\mysubparagraph{GCC \Optimizing (Linaro) 4.9}

\lstinputlisting[style=customasmARM]{patterns/10_strings/1_strlen/ARM/ARM64_GCC_O3_FR.lst}

L'algorithme est le même que dans \myref{strlen_MSVC_Ox}:
trouver un octet à zéro, calculer la différence antre les pointeurs et décrémenter
le résultat de 1.\TT{size\_t}
Quelques commentaires ont été ajouté par l'auteur de ce livre.

La seule différence notable est que cet exemple est un peu faux: \\
\TT{my\_strlen()} renvoie une valeur \Tint 32-bit, tandis qu'elle devrait renvoyer
un type \TT{size\_t} ou un autre type 64-bit.

La raison est que, théoriquement, \TT{strlen()} peut-être appelée pour un énorme
bloc de mémoire qui dépasse 4GB, donc elle doit être capable de renvoyer une valeur
64-bit sur une plate-forme 64-bit.

À cause de cette erreur, la dernière instruction \SUB opère sur la partie 32-bit
du registre, tandis que la pénultième instruction \SUB travaille sur un registre
64-bit complet (elle calcule la différence entre les pointeurs).

C'est une erreur de l'auteur, il est mieux de la laisser ainsi, comme un exemple
de ce à quoi ressemble le code dans un tel cas.

\mysubparagraph{GCC \NonOptimizing (Linaro) 4.9}

\lstinputlisting[style=customasmARM]{patterns/10_strings/1_strlen/ARM/ARM64_GCC_O0_FR.lst}

C'est plus verbeux.
Les variables sont beaucoup manipulées vers et depuis la mémoire (pile locale).
Il y a la même erreur ici: l'opération de décrémentation se produit sur la partie
32-bit du registre.

\subsubsection{ARM: \OptimizingKeilVI (\ARMMode)}
\myindex{\CLanguageElements!switch}

\lstinputlisting[style=customasmARM]{patterns/08_switch/1_few/few_ARM_ARM_O3.asm}

A nouveau, en investiguant ce code, nous ne pouvons pas dire si il y avait un switch()
dans le code source d'origine ou juste un ensemble de déclarations if().

\myindex{ARM!\Instructions!ADRcc}

En tout cas, nous voyons ici des instructions conditionnelles (comme \ADREQ (\emph{Equal}))
qui ne sont exécutées que si $R0=0$, et qui chargent ensuite l'adresse de la chaîne
\emph{<<zero\textbackslash{}n>>} dans \Reg{0}.
\myindex{ARM!\Instructions!BEQ}
L'instruction suivante \ac{BEQ} redirige le flux d'exécution en \TT{loc\_170}, si $R0=0$.

Le lecteur attentif peut se demander si \ac{BEQ} s'exécute correctement puisque \ADREQ
a déjà mis une autre valeur dans le registre \Reg{0}.

Oui, elle s'exécutera correctement, car \ac{BEQ} vérifie les flags mis par l'instruction
\CMP et \ADREQ ne modifie aucun flag.

Les instructions restantes nous sont déjà familières.
Il y a seulement un appel à \printf, à la fin, et nous avons déjà examiné cette
astuce ici~(\myref{ARM_B_to_printf}).
A la fin, il y a trois chemins vers \printf{}.

\myindex{ARM!\Instructions!ADRcc}
\myindex{ARM!\Instructions!CMP}
La dernière instruction, \TT{CMP R0, \#2}, est nécessaire pour vérifier si $a=2$.

Si ce n'est pas vrai, alors \ADRNE charge un pointeur sur la chaîne \emph{<<something unknown \textbackslash{}n>>}
dans \Reg{0}, puisque $a$ a déjà été comparée pour savoir s'elle est égale
à 0 ou 1, et nous sommes sûrs que la variable $a$ n'est pas égale à l'un de
ces nombres, à ce point.
Et si $R0=2$, un pointeur sur la chaîne \emph{<<two\textbackslash{}n>>} sera chargé
par \ADREQ dans \Reg{0}.

\subsubsection{ARM: \OptimizingKeilVI (\ThumbMode)}

\lstinputlisting[style=customasmARM]{patterns/08_switch/1_few/few_ARM_thumb_O3.asm}

% FIXME а каким можно? к каким нельзя? \myref{} ->

Comme il y déjà été dit, il n'est pas possible d'ajouter un prédicat conditionnel
à la plupart des instructions en mode Thumb, donc ce dernier est quelque peu similaire
au code \ac{CISC}-style x86, facilement compréhensible.

\subsubsection{ARM64: GCC (Linaro) 4.9 \NonOptimizing}

\lstinputlisting[style=customasmARM]{patterns/08_switch/1_few/ARM64_GCC_O0_FR.lst}

Le type de la valeur d'entrée est \Tint, par conséquent le registre \RegW{0} est
utilisé pour garder la valeur au lieu du registre complet \RegX{0}.

Les pointeurs de chaîne sont passés à \puts en utilisant la paire d'instructions
\INS{ADRP}/\INS{ADD} comme expliqué dans l'exemple \q{\HelloWorldSectionName}:~\myref{pointers_ADRP_and_ADD}.

\subsubsection{ARM64: GCC (Linaro) 4.9 \Optimizing}

\lstinputlisting[style=customasmARM]{patterns/08_switch/1_few/ARM64_GCC_O3_FR.lst}

Ce morceau de code est mieux optimisé.
L'instruction \TT{CBZ} (\emph{Compare and Branch on Zero} comparer et sauter si zéro)
effectue un saut si \RegW{0} vaut zéro.
Il y a alors un saut direct à \puts au lieu de l'appeler, comme cela a été expliqué
avant:~\myref{JMP_instead_of_RET}.

\subsubsection{MIPS}

\lstinputlisting[caption=GCC 4.4.5 \Optimizing (IDA),style=customasmMIPS]{patterns/08_switch/1_few/MIPS_O3_IDA_FR.lst}

\myindex{MIPS!\Instructions!JR}

La fonction se termine toujours en appelant \puts, donc nous voyons un saut à \puts
(\INS{JR}: \q{Jump Register}) au lieu de \q{jump and link}.
Nous avons parlé de ceci avant: \myref{JMP_instead_of_RET}.

\myindex{MIPS!Load delay slot}
Nous voyons aussi souvent l'instruction \INS{NOP} après \INS{LW}.
Ceci est le slot de délai de chargement (\q{load delay slot}): un autre slot de
délai (\emph{delay slot}) en MIPS.
\myindex{MIPS!\Instructions!LW}

Une instruction suivant \INS{LW} peut s'exécuter pendant que \INS{LW} charge une
valeur depuis la mémoire.

Toutefois, l'instruction suivante ne doit pas utiliser le résultat de \INS{LW}.

Les CPU MIPS modernes ont la capacité d'attendre si l'instruction suivante utilise
le résultat de \INS{LW}, donc ceci est un peu démodé, mais GCC ajoute toujours
des NOPs pour les anciens CPU MIPS.
En général, ça peut être ignoré.


