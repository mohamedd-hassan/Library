\mysection{Strings trimming}
\newcommand{\CRLF}{\ac{CR}/\ac{LF}}

A very common string processing task is to remove some characters at the start and/or at the end.

In this example, we are going to work with a function which removes all newline characters 
(\CRLF{}) from the end of the input string:

\lstinputlisting[style=customc]{\CURPATH/strtrim_EN.c}

The input argument is always returned on exit, this is convenient when you want to chain 
string processing functions, like it has done here in the \main function.

\myindex{\CLanguageElements!Short-circuit}
The second part of for() (\TT{str\_len>0 \&\& (c=s[str\_len-1])}) is the so called \q{short-circuit} 
in \CCpp and is very convenient \InSqBrackets{\CNotes 1.3.8}.

The \CCpp compilers guarantee an evaluation sequence from left to right.

So if the first clause is false after evaluation, the second one is never to be evaluated.

% subsections
\subsection{x64: \Optimizing MSVC 2013}

\lstinputlisting[caption=\Optimizing MSVC 2013 x64,style=customasmx86]{\CURPATH/MSVC2013_x64_Ox_EN.asm}

First, MSVC inlined the \strlen{} function code, because it concluded this 
is to be faster than the usual \strlen{} work + the cost of calling it and returning from it.
This is called inlining: \myref{inline_code}.

\myindex{x86!\Instructions!OR}
\myindex{\CStandardLibrary!strlen()}
\label{using_OR_instead_of_MOV}
The first instruction of the inlined \strlen{} is\\
\TT{OR RAX, 0xFFFFFFFFFFFFFFFF}. 

MSVC often uses \TT{OR} instead of \TT{MOV RAX, 0xFFFFFFFFFFFFFFFF}, because resulting opcode is shorter.

And of course, it is equivalent: all bits are set, and a number with all bits set is $-1$ 
in two's complement arithmetic.

Why would the $-1$ number be used in \strlen{}, one might ask.
Due to optimizations, of course.
Here is the code that MSVC generated:

\lstinputlisting[caption=Inlined \strlen{} by MSVC 2013 x64,style=customasmx86]{\CURPATH/strlen_MSVC_EN.asm}

Try to write shorter if you want to initialize the counter at 0!
OK, let' try:

\lstinputlisting[caption=Our version of \strlen{},style=customasmx86]{\CURPATH/my_strlen_EN.asm}

We failed. We have to use additional \INS{JMP} instruction!

So what the MSVC 2013 compiler did is to move the \TT{INC} instruction to the place before 
the actual character loading.

If the first character is 0, that's OK, \RAX is 0 at this moment, 
so the resulting string length is 0.

The rest in this function seems easy to understand.

\subsection{x64: \NonOptimizing GCC 4.9.1}

\lstinputlisting[style=customasmx86]{\CURPATH/GCC491_x64_O0_EN.asm}

Comments are added by the author of the book.

After the execution of \strlen{}, the control is passed to the L2 label, 
and there two clauses are checked, one after another.

\myindex{\CLanguageElements!Short-circuit}
The second will never be checked, if the first one (\emph{str\_len==0}) is false 
(this is \q{short-circuit}).

Now let's see this function in short form:

\begin{itemize}
\item First for() part (call to \strlen{})
\item goto L2
\item L5: for() body. goto exit, if needed
\item for() third part (decrement of str\_len)
\item L2: 
for() second part: check first clause, then second. goto loop body begin or exit.
\item L4: // exit
\item return s
\end{itemize}

\subsection{x64: \Optimizing GCC 4.9.1}
\label{string_trim_GCC_x64_O3}

\lstinputlisting[style=customasmx86]{\CURPATH/GCC491_x64_O3_EN.asm}

Now this is more complex.

The code before the loop's body start is executed only once, but it has the \CRLF{} 
characters check too!
What is this code duplication for?

The common way to implement the main loop is probably this:

\begin{itemize}
\item (loop start) check for 
\CRLF{} characters, make decisions
\item store zero character
\end{itemize}

But GCC has decided to reverse these two steps. 

Of course, \emph{store zero character} cannot be first step, so another check is needed:

\begin{itemize}
\item workout first character. match it to \CRLF{}, exit if character is not \CRLF{}

\item (loop begin) store zero character

\item check for \CRLF{} characters, make decisions
\end{itemize}

Now the main loop is very short, which is good for latest \ac{CPU}s.

The code doesn't use the str\_len variable, but str\_len-1.
So this is more like an index in a buffer.

Apparently, GCC notices that the str\_len-1 statement is used twice.

So it's better to allocate a variable which always holds a value that's smaller than 
the current string length by one, 
and decrement it (this is the same effect as decrementing the str\_len variable).

\subsubsection{ARM64}

\myparagraph{\Optimizing GCC (Linaro) 4.9}

\myindex{Fused multiply–add}
\myindex{ARM!\Instructions!MADD}
Everything here is simple.
\TT{MADD} is just an instruction doing fused multiply/add (similar to the \TT{MLA} we already saw).
All 3 arguments are passed in the 32-bit parts of X-registers.
Indeed, the argument types are 32-bit \emph{int}'s.
The result is returned in \TT{W0}.

\lstinputlisting[caption=\Optimizing GCC (Linaro) 4.9,style=customasmARM]{patterns/05_passing_arguments/ARM/ARM64_O3_EN.s}

Let's also extend all data types to 64-bit \TT{uint64\_t} and test:

\lstinputlisting[style=customc]{patterns/05_passing_arguments/ex64.c}

\begin{lstlisting}[style=customasmARM]
f:
	madd	x0, x0, x1, x2
	ret
main:
	mov	x1, 13396
	adrp	x0, .LC8
	stp	x29, x30, [sp, -16]!
	movk	x1, 0x27d0, lsl 16
	add	x0, x0, :lo12:.LC8
	movk	x1, 0x122, lsl 32
	add	x29, sp, 0
	movk	x1, 0x58be, lsl 48
	bl	printf
	mov	w0, 0
	ldp	x29, x30, [sp], 16
	ret

.LC8:
	.string	"%lld\n"
\end{lstlisting}

The \ttf{} function is the same, only the whole 64-bit X-registers are now used.
Long 64-bit values are loaded into the registers by parts, this is also described here: \myref{ARM_big_constants_loading}.

\myparagraph{\NonOptimizing GCC (Linaro) 4.9}

The non-optimizing compiler is more redundant:

\begin{lstlisting}[style=customasmARM]
f:
	sub	sp, sp, #16
	str	w0, [sp,12]
	str	w1, [sp,8]
	str	w2, [sp,4]
	ldr	w1, [sp,12]
	ldr	w0, [sp,8]
	mul	w1, w1, w0
	ldr	w0, [sp,4]
	add	w0, w1, w0
	add	sp, sp, 16
	ret
\end{lstlisting}

The code saves its input arguments in the local stack, 
in case someone (or something) in this function needs using the \TT{W0...W2} 
registers. This prevents overwriting the original
function arguments, which may be needed again in the future.

This is called \emph{Register Save Area.} \ARMPCS.
The callee, however, is not obliged to save them.
This is somewhat similar to \q{Shadow Space}: \myref{shadow_space}.

Why did the optimizing GCC 4.9 drop this argument saving code?
Because it did some additional optimizing work and concluded
that the function arguments will not be needed in the future 
and also that the registers \TT{W0...W2} will not be used.

\myindex{ARM!\Instructions!MUL}
\myindex{ARM!\Instructions!ADD}

We also see a \TT{MUL}/\TT{ADD} instruction pair instead of single a \TT{MADD}.

\subsection{ARM: \OptimizingKeilVI (\ARMMode)}

And again, the compiler took advantage of ARM mode's conditional instructions, 
so the code is much more compact.

\lstinputlisting[caption=\OptimizingKeilVI (\ARMMode),style=customasmARM]{\CURPATH/Keil_ARM_O3_EN.s}

\subsection{ARM: \OptimizingKeilVI (\ThumbMode)}
\myindex{\CompilerAnomaly}
\label{Keil_anomaly}

There are less conditional instructions in Thumb mode, so the code is simpler.

But there are is really weird thing with the 0x20 and 0x1F offsets (lines 22 and 23).
Why did the Keil compiler do so?
Honestly, it's hard to say.

It has to be a quirk of Keil's optimization process.
Nevertheless, the code works correctly.

\lstinputlisting[caption=\OptimizingKeilVI (\ThumbMode),numbers=left,style=customasmARM]{\CURPATH/Keil_thumb_O3_EN.s}


\scanf returns the result of its work in register \$V0. It is checked at address 0x004006E4
by comparing the values in \$V0 with \$V1 (1 has been stored in \$V1 earlier, at 0x004006DC).
\INS{BEQ} stands for \q{Branch Equal}.
If the two values are equal (i.e., success), the execution jumps to address 0x0040070C.



