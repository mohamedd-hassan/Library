\subsubsection{Множественное наследование}

Множественное наследование --- это создание класса наследующего поля и методы от двух или более классов.


Снова напишем простой пример:

\lstinputlisting[style=customc]{\CURPATH/classes/classes3_multiple.cpp}

Снова скомпилируем в MSVC 2008 с опциями \Ox и \Obzero и посмотрим код методов \TT{box::dump()},\\
\TT{solid\_object::dump()} и \TT{solid\_box::dump()}:


\lstinputlisting[caption=\Optimizing MSVC 2008 /Ob0,style=customasmx86]{\CURPATH/classes/classes3_1.asm}

\lstinputlisting[caption=\Optimizing MSVC 2008 /Ob0,style=customasmx86]{\CURPATH/classes/classes3_2.asm}

\lstinputlisting[caption=\Optimizing MSVC 2008 /Ob0,style=customasmx86]{\CURPATH/classes/classes3_3.asm}

Выходит, имеем такую разметку в памяти для всех трех классов:


класс \emph{box}:

\begin{center}
\begin{tabular}{ | l | l | }
\hline
  \tableheader{} \\
\hline
  +0x0 & width \\
\hline
  +0x4 & height \\
\hline
  +0x8 & depth \\
\hline
\end{tabular}
\end{center}

класс \emph{solid\_object}:

\begin{center}
\begin{tabular}{ | l | l | }
\hline
  \tableheader{} \\
\hline
  +0x0 & density \\
\hline
\end{tabular}
\end{center}

Можно сказать, что разметка класса \emph{solid\_box} \emph{объединённая}:


Класс \emph{solid\_box}:

\begin{center}
\begin{tabular}{ | l | l | }
\hline
  \tableheader{} \\
\hline
  +0x0 & width \\
\hline
  +0x4 & height \\
\hline
  +0x8 & depth \\
\hline
  +0xC & density \\
\hline
\end{tabular}
\end{center}

Код методов \TT{box::get\_volume()} и \TT{solid\_object::get\_density()} тривиален:


\lstinputlisting[caption=\Optimizing MSVC 2008 /Ob0,style=customasmx86]{\CURPATH/classes/classes3_4.asm}

\lstinputlisting[caption=\Optimizing MSVC 2008 /Ob0,style=customasmx86]{\CURPATH/classes/classes3_5.asm}

А вот код метода \TT{solid\_box::get\_weight()} куда интереснее:


\lstinputlisting[caption=\Optimizing MSVC 2008 /Ob0,style=customasmx86]{\CURPATH/classes/classes3_6.asm}

\TT{get\_weight()} просто вызывает два метода, но для \TT{get\_volume()} он передает просто указатель на \TT{this}, а для \TT{get\_density()}, он передает указатель на \TT{this} сдвинутый на 12 байт (либо \TT{0xC} байт), а там, 
в разметке класса \TT{solid\_box}, как раз начинаются поля класса \TT{solid\_object}.


Так, метод \TT{solid\_object::get\_density()} будет полагать что работает с обычным классом \\
\TT{solid\_object}, а метод \TT{box::get\_volume()} будет работать только со своими тремя полями, полагая, 
что работает с обычным экземпляром класса \TT{box}.


Таким образом, можно сказать, что экземпляр класса-наследника нескольких классов представляет в памяти просто 
\emph{объединённый} класс, содержащий все унаследованные поля. А каждый унаследованный метод вызывается с передачей
ему указателя на соответствующую часть структуры.


