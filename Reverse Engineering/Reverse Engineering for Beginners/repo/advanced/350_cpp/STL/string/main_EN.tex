\subsubsection{std::string}
\myindex{\Cpp!STL!std::string}
\label{std_string}

\myparagraph{Internals}

Many string libraries \InSqBrackets{\CNotes 2.2} implement a structure that contains a pointer to a string buffer,
a variable that always contains the current string length 
(which is very convenient for many functions: \InSqBrackets{\CNotes 2.2.1}) and
a variable containing the current buffer size.

The string in the buffer is usually terminated with zero, in order to be able to pass a pointer to the buffer
into the functions that take  usual C \ac{ASCIIZ} strings.

It is not specified in the \Cpp standard how std::string has to be implemented,
however, it is usually implemented as explained above.

The \Cpp string is not a class (as QString in Qt, for instance) but a template (basic\_string),
this is made in order to support various character types: at least \Tchar and \emph{wchar\_t}.

So, std::string is a class with \Tchar as its base type.

And std::wstring is a class with \emph{wchar\_t} as its base type.

\mysubparagraph{MSVC}

The MSVC implementation may store the buffer in place instead of using a pointer to a buffer 
(if the string is shorter than 16 symbols).

This implies that a short string is to occupy at least $16 + 4 + 4 = 24$ 
bytes in 32-bit environment or at least $16 + 8 + 8 = 32$ 

bytes in 64-bit one, and if the string is longer than 16 characters, we also have to add the length of the string itself.

\lstinputlisting[caption=example for MSVC,style=customc]{\CURPATH/STL/string/MSVC_EN.cpp}

Almost everything is clear from the source code.

A couple of notes:

If the string is shorter than 16 symbols, 
a buffer for the string is not to be allocated in the \gls{heap}.

This is convenient because
in practice, a lot of strings are short indeed.

Looks like that Microsoft's developers chose 16 characters as a good balance.

One very important thing here can be seen at the end of main(): we're not using the c\_str() method, nevertheless,
if we compile and run this code, both strings will appear in the console!

This is why it works.

In the first case the string is shorter than 16 characters and the buffer with the string is located in the
beginning of the std::string object (it can be treated as a structure).
printf() treats the pointer as a pointer to the null-terminated 
array of characters, hence it works.

Printing the second string (longer than 16 characters) is even more dangerous: it is a typical programmer's mistake
(or typo) to forget to write c\_str().

This works because at the moment a pointer to buffer is located at the start of structure.

This may stay unnoticed for a long time, until a longer string appears there at some time, 
then the process will crash.

\mysubparagraph{GCC}

GCC's implementation of this structure has one more variable---reference count.

One interesting fact is that in GCC, a pointer to an instance of std::string instance points not to
the beginning of the structure, but to the buffer pointer.
In \emph{libstdc++-v3\textbackslash{}include\textbackslash{}bits\textbackslash{}basic\_string.h}
we can read that it was done for more convenient debugging:

\begin{lstlisting}
   *  The reason you want _M_data pointing to the character %array and
   *  not the _Rep is so that the debugger can see the string
   *  contents. (Probably we should add a non-inline member to get
   *  the _Rep for the debugger to use, so users can check the actual
   *  string length.)
\end{lstlisting}

\href{http://gcc.gnu.org/onlinedocs/libstdc++/libstdc++-html-USERS-4.4/a01068.html}{basic\_string.h source code}

We consider this in our example:

\lstinputlisting[caption=example for GCC,style=customc]{\CURPATH/STL/string/GCC_EN.cpp}

A trickery has to be used to imitate the mistake we already have seen above because GCC
has stronger type checking, nevertheless, printf() works here without c\_str() as well.

\myparagraph{A more advanced example}

\lstinputlisting[style=customc]{\CURPATH/STL/string/3.cpp}

\lstinputlisting[caption=MSVC 2012,style=customasmx86]{\CURPATH/STL/string/3_MSVC_EN.asm}

The compiler does not construct strings statically: it would not be possible anyway if the buffer needs to be located
in the \gls{heap}.

Instead, the \ac{ASCIIZ} strings are stored in the data segment, and later, at runtime,
with the help of the \q{assign} method, the s1 and s2 strings are constructed.
And with the help of \TT{operator+}, the s3 string is constructed.

Please note that 
there is no call to the c\_str() method, because its code is tiny enough so the compiler
inlined it right there: if the string is shorter than 16 characters, a pointer to buffer is left
in \EAX, otherwise the address of the string buffer located in the \gls{heap} is fetched.

Next, we see calls to the 3 destructors, they are called if
the string is longer than 16 characters: then the buffers in the \gls{heap} have to be freed.
Otherwise, since all three std::string objects
are stored in the stack, they are freed automatically, when the function ends.

As a consequence, processing short strings is faster, because of less \gls{heap} accesses.

GCC code is even simpler (because the GCC way, as we saw above, is to not store
shorter strings right in the structure):

% TODO1 comment each function meaning
\lstinputlisting[caption=GCC 4.8.1,style=customasmx86]{\CURPATH/STL/string/3_GCC_EN.s}

It can be seen that it's not a pointer to the object that is passed to destructors, but rather an address 12 bytes (or 3 words)
before, i.e., a pointer to the real start of the structure.

\myparagraph{std::string as a global variable}
\label{sec:std_string_as_global_variable}

Experienced \Cpp programmers knows that global variables of \ac{STL} types can be defined without problems.

Yes, indeed:

\lstinputlisting[style=customc]{\CURPATH/STL/string/5.cpp}

But how and where \TT{std::string} constructor will be called?

In fact, this variable is to be initialized even before \main start.

\lstinputlisting[caption=MSVC 2012: here is how a global variable is constructed and also its destructor is registered,style=customasmx86]{\CURPATH/STL/string/5_MSVC_p2.asm}

\lstinputlisting[caption=MSVC 2012: here a global variable is used in \main,style=customasmx86]{\CURPATH/STL/string/5_MSVC_p1.asm}

\lstinputlisting[caption=MSVC 2012: this destructor function is called before exit,style=customasmx86]{\CURPATH/STL/string/5_MSVC_p3.asm}

\myindex{\CStandardLibrary!atexit()}

In fact, a special function with all constructors of global variables is called from \ac{CRT}, before
main().

More than that: with the help of atexit() another function is registered, 
which contain calls to all destructors of such global variables.

GCC works likewise:

\lstinputlisting[caption=GCC 4.8.1,style=customasmx86]{\CURPATH/STL/string/5_GCC.s}

But it does not create a separate function for this, 
each destructor is passed to atexit(), one by one.

% TODO а если глобальная STL-переменная в другом модуле? надо проверить.

