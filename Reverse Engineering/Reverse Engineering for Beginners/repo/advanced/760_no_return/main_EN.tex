\mysection{The case of forgotten return}
\label{ForgottenReturn}

Let's revisit the ``attempt to use the result of a function returning \Tvoid'' part: \myref{UseResultOfVoidFunc}.

This is a bug I once hit.

And this is also yet another demonstration, how \CCpp{} places return value into \EAX/\RAX register.

In the piece of code like that, I forgot to add \TT{return}:

\lstinputlisting[style=customc]{\CURPATH/1_EN.c}

Non-optimizing GCC 5.4 silently compiles this with no warnings.
And the code works!
Let's see, why:

\lstinputlisting[style=customasmx86,caption=\NonOptimizing GCC 5.4]{\CURPATH/O0_no_return_works_EN.lst}

If I add \TT{return rt;}, only one instruction is added at the end, which is redundant:

\lstinputlisting[style=customasmx86,caption=\NonOptimizing GCC 5.4]{\CURPATH/O0_return_works_EN.lst}

Bugs like that are very dangerous, sometimes they appear, sometimes hide.
It's like Heisenbug.

Now I'm trying optimizing GCC:

\lstinputlisting[style=customasmx86,caption=\Optimizing GCC 5.4]{\CURPATH/O3_no_return_crash_EN.lst}

Compiler deducing that nothing returns from the function, so it optimizes the whole function away.
And it assumes, that is returns 0 by default. The zero is then used as an address to a structure in main()..
Of course, this code crashes.

GCC in C++ mode silent about it as well.

Let's try non-optimizing MSVC 2015 x86.
It warns about the problem:

\begin{lstlisting}
c:\tmp\3.c(19) : warning C4716: 'create_color': must return a value                                                               
\end{lstlisting}

And generates crashing code:

\lstinputlisting[style=customasmx86,caption=\NonOptimizing MSVC 2015 x86]{\CURPATH/MSVC_x86_crash_EN.lst}

Now optimizing MSVC 2015 x86 generates crashing code as well, but for the different reason:

\lstinputlisting[style=customasmx86,caption=\Optimizing MSVC 2015 x86]{\CURPATH/MSVC_Ox_x86_crash_EN.lst}

However, non-optimizing MSVC 2015 x64 generates working code:

\lstinputlisting[style=customasmx86,caption=\NonOptimizing MSVC 2015 x64]{\CURPATH/MSVC_x64_works_EN.lst}

Optimizing MSVC 2015 x64 also inlines the function, as in case of x86, and the resulting code also crashes.

\myhrule{}

% FIXME URL to octothorpe
This is a real piece of code from my \emph{octothorpe} library, that worked and all tests passed.
It was so, without \verb|return| for quite a time...

\begin{lstlisting}
uint32_t LPHM_u32_hash(void *key)
{
        jenkins_one_at_a_time_hash_u32((uint32_t)key);
}
\end{lstlisting}

\myhrule{}

The moral of the story: warnings are very important, use \TT{-Wall}, etc, etc...
When \TT{return} statement is absent, compiler can just silently do nothing at that point.

Such a bug left unnoticed can ruin a day.

Also, \emph{shotgun debugging}
is bad, because again, such a bug can left unnoticed (``everything works now, so be it, let's leave it as is'').

See also: discussion at Hacker News\footnote{\url{https://news.ycombinator.com/item?id=18671609}}
and archived blog post\footnote{\url{https://web.archive.org/web/20190317231721/https://yurichev.com/blog/no_return/}}.

