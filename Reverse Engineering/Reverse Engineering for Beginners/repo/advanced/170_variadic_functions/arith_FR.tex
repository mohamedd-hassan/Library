\subsection{Calcul de la moyenne arithmétique}

Imaginons que nous voulions calculer la \glslink{arithmetic mean}{moyenne arithmétique},
et que pour une raison quelconque, nous voulions passer toutes les valeurs comme
arguments de la fonction.

Mais il est impossible d'obtenir le nombre d'arguments dans une fonction variadique
en \CCpp, donc indiquons la valeur $-1$ comme terminateur.

\subsubsection{Utilisation de la macro va\_arg}

Il y a le fichier d'entête standard stdarg.h qui défini des macros pour prendre en
compte de tels arguments.

Les fonctions \printf et \scanf l'utilisent aussi.

\lstinputlisting[style=customc]{\CURPATH/arith_FR.c}

Le premier argument doit être traité comme un argument normal.
\myindex{\CStandardLibrary!va\_arg}

Tous les autres arguments sont chargés en utilisant la macro \TT{va\_arg} et ensuite
ajoutés.

Qu'y a-t-il à l'intérieur?

\myparagraph{Convention d'appel \emph{cdecl}}

\lstinputlisting[caption=MSVC 6.0 \Optimizing,style=customasmx86]{\CURPATH/arith_MSVC60_Ox_FR.asm}

Les arguments, comme on le voit, sont passés à \main un par un.

Le premier argument est poussé sur la pile locale en premier.

La valeur terminale ($-1$) est poussée plus tard.

La fonction \TT{arith\_mean()} prend la valeur du premier argument et le stocke dans
la variable $sum$.

Puis, elle met dans le registre \EDX l'adresse du second argument, prend sa valeur,
l'ajoute à $sum$, et fait cela dans une boucle infinie, jusqu'à ce que $-1$ soit
trouvé.

Lorsqu'il est rencontré, la somme est divisée par le nombre de valeurs (en excluant $-1$)
et le \gls{quotient} est renvoyé.

Donc, autrement dit, la fonction traite le morceau de pile comme un tableau de valeurs
entières d'une longueur infinie.

Maintenant nous pouvons comprendre pourquoi la convention d'appel \emph{cdecl} nous
force à pousser le premier argument au moins sur la pile.

Car sinon, il ne serait pas possible de trouver le premier argument, ou, pour les
fonctions du genre de printf, il ne serait pas possible de trouver l'adresse de la
chaîne de format.

\myparagraph{Convention d'appel basée sur les registres}
\label{variadic_arith_registers}

Le lecteur attentif pourrait demander, qu'en est-il de la convention d'appel où les
tous premiers arguments sont passés dans des registres?
Regardons:

\lstinputlisting[caption=MSVC 2012 x64 \Optimizing,style=customasmx86]{\CURPATH/arith_MSVC_2012_Ox_x64_FR.asm}

Nous voyons que les 4 premiers arguments sont passés dans des registres, et les deux
autres---par la pile.

La fonction \TT{arith\_mean()} place d'abord ces 4 arguments dans le \emph{Shadow Space}
puis traite le \emph{Shadow Space} et la pile derrière comme s'il s'agissait d'un tableau
continu!

Qu'en est-il de GCC? Les choses sont légèrement plus maladroites ici, car maintenant
la fonction est divisée en deux parties:
la première partie sauve les registres dans la \q{zone rouge}, traite cet espace,
et la seconde partie traite la pile:

\lstinputlisting[caption=GCC 4.9.1 x64 \Optimizing,style=customasmx86]{\CURPATH/arith_GCC491_x64_O3_FR.s}

À propos, un usage similaire du \emph{Shadow Space} est aussi considéré ici: \myref{pointer_to_argument}.

\subsubsection{Utilisation du pointeur sur le premier argument de la fonction}

L'exemple peut être récrit sans la macro \TT{va\_arg}:

\lstinputlisting[style=customc]{\CURPATH/arith2.c}

Autrement dit, si l'argument mis est un tableau de mots (32-bit ou 64-bit), nous devons
juste énumérer les éléments du tableau en commençant par le premier.
