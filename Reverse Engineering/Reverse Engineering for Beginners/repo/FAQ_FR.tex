\subsection*{mini-FAQ}

\par Q: Est-ce que ce livre est plus simple/facile que les autres?
\par R: Non, c'est à peu près le même niveau que les autres livres sur ce sujet.

\par Q: J'ai trop peur de commencer à lire ce livre, il fait plus de 1000 pages.
"...for Beginners" dans le nom sonne un peu sarcastique.
\par R: Toutes sortes de listings constituent le gros de ce livre.
Le livre est en effet pour les débutants, il manque (encore) beaucoup de choses.

\par Q: Quels sont les pré-requis nécessaires avant de lire ce livre ?
\par R: Une compréhension de base du C/C++ serait l'idéal.

\par Q: Dois-je apprendre x86/x64/ARM et MIPS en même temps ? N'est-ce pas un peu trop ?
\par R: Je pense que les débutants peuvent seulement lire les parties x86/x64, tout en passant/feuilletant celles ARM/MIPS.

\par Q: Puis-je acheter une version papier du livre en russe / anglais ?
\par R: Malheureusement non, aucune maison d'édition n'a été intéressée pour publier une version en russe ou en anglais du livre jusqu'à présent.
Cependant, vous pouvez demander à votre imprimerie préférée de l'imprimer et de le relier.
\url{https://yurichev.com/news/20200222_printed_RE4B/}.

\par Q: Y a-il une version ePub/Mobi ?
\par R: Le livre dépend majoritairement de TeX/LaTeX, il n'est donc pas évident de le convertir en version ePub/Mobi.

\par Q: Pourquoi devrait-on apprendre l'assembleur de nos jours ?
\par R: A moins d'être un développeur d'\ac{OS}, vous n'aurez probablement pas besoin d'écrire en assembleur\textemdash{}les derniers compilateurs (ceux de notre décennie) sont meilleurs que les êtres humains en terme d'optimisation. \footnote{Un très bon article à ce sujet : \InSqBrackets{\AgnerFog}}.

De plus, les derniers \ac{CPU}s sont des appareils complexes et la connaissance de l'assembleur n'aide pas vraiment à comprendre leurs mécanismes internes.

Cela dit, il existe au moins deux domaines dans lesquels une bonne connaissance de l'assembleur peut être utile : 
Tout d'abord, pour de la recherche en sécurité ou sur des malwares. C'est également un bon moyen de comprendre un code compilé lorsqu'on le debug.
Ce livre est donc destiné à ceux qui veulent comprendre l'assembleur plutôt que d'écrire en assembleur, ce qui explique pourquoi il y a de nombreux exemples de résultats issus de compilateurs dans ce livre. 

\par Q: J'ai cliqué sur un lien dans le document PDF, comment puis-je retourner en arrière ?
\par R: Dans Adobe Acrobat Reader, appuyez sur Alt + Flèche gauche. Dans Evince, appuyez sur le bouton ``<''.

\par Q: Puis-je imprimer ce livre / l'utiliser pour de l'enseignement ?
\par R: Bien sûr ! C'est la raison pour laquelle le livre est sous licence Creative Commons (CC BY-SA 4.0).

\par Q: Pourquoi ce livre est-il gratuit ? Vous avez fait du bon boulot. C'est suspect, comme nombre de choses gratuites.
\par R: D'après ma propre expérience, les auteurs d'ouvrages techniques font cela pour l'auto-publicité. Il n'est pas possible de se faire beaucoup d'argent d'une telle manière.

\par Q: Comment trouver du travail dans le domaine de la rétro-ingénierie ?
\par R: Il existe des sujets d'embauche qui apparaissent de temps en temps sur Reddit, dédiés à la rétro-ingénierie (cf. reverse engineering ou RE)\FNURLREDDIT{}.
Jetez un \oe{}il ici.

Un sujet d'embauche quelque peu lié peut être trouvé dans le subreddit \q{netsec}.

\par Q: J'ai une question...
\par R: Envoyez-la moi par email (\EMAILS).

