\mysection{Определение \ac{ISA}}
\label{ISA_detect}

Часто, вы можете иметь дело с бинарным файлом для неизвестной \ac{ISA}.
Вероятно, простейший способ определить \ac{ISA} это пробовать разные в IDA, objdump или другом дизассемблере.

Чтобы этого достичь, нужно понимать разницу между некорректно дизассемблированным кодом, и корректно дизассемблированным.

% subsection:
\renewcommand{\CURPATH}{digging_into_code/incorrect_disassembly}
\subsection{Обменять входные значения друг с другом}

Вот так:

\lstinputlisting[style=customc]{patterns/061_pointers/swap/5_RU.c}

Как видим, байты загружаются в младшие 8-битные части регистров \TT{ECX} и \TT{EBX} используя \INS{MOVZX}
(так что старшие части регистров очищаются), затем байты записываются назад в другом порядке.

\lstinputlisting[style=customasmx86,caption=Optimizing GCC 5.4]{patterns/061_pointers/swap/5_GCC_O3_x86.s}

Адреса обоих байтов берутся из аргументов и во время исполнения ф-ции находятся в регистрах \TT{EDX} и \TT{EAX}.

Так что исопльзуем указатели --- вероятно, без них нет способа решить эту задачу лучше.



\subsection{Корректно дизассемблированный код}
\label{correctly_disasmed_code}

Каждая \ac{ISA} имеет десяток самых используемых инструкций, остальные используются куда реже.

Интересно знать тот факт, что в x86, инструкции вызовов ф-ций (\PUSH/\CALL/\ADD) и \MOV
это наиболее часто исполняющиеся инструкции в коде почти во всем ПО что мы используем.
Другими словами, \ac{CPU} очень занят передачей информации между уровнями абстракции, или, можно сказать, очень занят
переключением между этими уровнями.
Вне зависимости от \ac{ISA}.
Это цена расслоения программ на разные уровни абстракций (чтобы человеку было легче с ними управляться).

