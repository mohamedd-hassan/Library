\mysection{Strings}
\label{sec:digging_strings}

\mysection{x86}

\subsection{Terminologie}

Geläufig für 16-Bit (8086/80286), 32-Bit (80386, etc.), 64-Bit.

\myindex{IEEE 754}
\myindex{MS-DOS}
\begin{description}
	\item[Byte] 8-Bit.
		Die DB Assembler-Direktive wird zum Definieren von Variablen und Arrays genutzt.
		Bytes werden in dem 8-Bit-Teil der folgenden Register übergeben:
		\TT{AL/BL/CL/DL/AH/BH/CH/DH/SIL/DIL/R*L}.
	\item[Wort] 16-Bit.
		DW Assembler-Direktive \dittoclosing.
		Bytes werden in dem 16-Bit-Teil der folgenden Register übergeben:
			\TT{AX/BX/CX/DX/SI/DI/R*W}.
	\item[Doppelwort] (\q{dword}) 32-Bit.
		DD Assembler-Direktive \dittoclosing.
		Doppelwörter werden in Registern (x86) oder dem 32-Bit-Teil der Register (x64) übergeben.
		In 16-Bit-Code werden Doppelwörter in 16-Bit-Registerpaaren übergeben.
	\item[zwei Doppelwörter] (\q{qword}) 64-Bit.
		DQ Assembler-Direktive \dittoclosing.
		In 32-Bit-Umgebungen werden diese in 32-Bit-Registerpaaren übergeben.
	\item[tbyte] (10 Byte) 80-Bit oder 10 Bytes (für IEEE 754 FPU Register).
	\item[paragraph] (16 Byte) --- Bezeichnung war in MS-DOS Umgebungen gebräuchlich.
\end{description}

\myindex{Windows!API}

Datentypen der selben Breite (BYTE, WORD, DWORD) entsprechen auch denen in der Windows \ac{API}.

% TODO German Translation (DSiekmeier)
%\input{appendix/x86/registers} % subsection
%\input{appendix/x86/instructions} % subsection
\subsection{npad}
\label{sec:npad}

\RU{Это макрос в ассемблере, для выравнивания некоторой метки по некоторой границе.}
\EN{It is an assembly language macro for aligning labels on a specific boundary.}
\DE{Dies ist ein Assembler-Makro um Labels an bestimmten Grenzen auszurichten.}
\FR{C'est une macro du langage d'assemblage pour aligner les labels sur une limite
spécifique.}

\RU{Это нужно для тех \emph{нагруженных} меток, куда чаще всего передается управление, например, 
начало тела цикла. 
Для того чтобы процессор мог эффективнее вытягивать данные или код из памяти, через шину с памятью, 
кэширование, итд.}
\EN{That's often needed for the busy labels to where the control flow is often passed, e.g., loop body starts.
So the CPU can load the data or code from the memory effectively, through the memory bus, cache lines, etc.}
\DE{Dies ist oft nützlich Labels, die oft Ziel einer Kotrollstruktur sind, wie Schleifenköpfe.
Somit kann die CPU Daten oder Code sehr effizient vom Speicher durch den Bus, den Cache, usw. laden.}
\FR{C'est souvent nécessaire pour des labels très utilisés, comme par exemple le
début d'un corps de boucle. Ainsi, le CPU peut charger les données ou le code depuis
la mémoire efficacement, à travers le bus mémoire, les caches, etc.}

\RU{Взято из}\EN{Taken from}\DE{Entnommen von}\FR{Pris de} \TT{listing.inc} (MSVC):

\myindex{x86!\Instructions!NOP}
\RU{Это, кстати, любопытный пример различных вариантов \NOP{}-ов. 
Все эти инструкции не дают никакого эффекта, но отличаются разной длиной.}
\EN{By the way, it is a curious example of the different \NOP variations.
All these instructions have no effects whatsoever, but have a different size.}
\DE{Dies ist übrigens ein Beispiel für die unterschiedlichen \NOP-Variationen.
Keine dieser Anweisungen hat eine Auswirkung, aber alle haben eine unterschiedliche Größe.}
\FR{À propos, c'est un exemple curieux des différentes variations de \NOP. Toutes
ces instructions n'ont pas d'effet, mais ont une taille différente.}

\RU{Цель в том, чтобы была только одна инструкция, а не набор NOP-ов, 
считается что так лучше для производительности CPU.}
\EN{Having a single idle instruction instead of couple of NOP-s,
is accepted to be better for CPU performance.}
\DE{Eine einzelne Idle-Anweisung anstatt mehrerer NOPs hat positive Auswirkungen
auf die CPU-Performance.}
\FR{Avoir une seule instruction sans effet au lieu de plusieurs est accepté comme
étant meilleur pour la performance du CPU.}

\begin{lstlisting}[style=customasmx86]
;; LISTING.INC
;;
;; This file contains assembler macros and is included by the files created
;; with the -FA compiler switch to be assembled by MASM (Microsoft Macro
;; Assembler).
;;
;; Copyright (c) 1993-2003, Microsoft Corporation. All rights reserved.

;; non destructive nops
npad macro size
if size eq 1
  nop
else
 if size eq 2
   mov edi, edi
 else
  if size eq 3
    ; lea ecx, [ecx+00]
    DB 8DH, 49H, 00H
  else
   if size eq 4
     ; lea esp, [esp+00]
     DB 8DH, 64H, 24H, 00H
   else
    if size eq 5
      add eax, DWORD PTR 0
    else
     if size eq 6
       ; lea ebx, [ebx+00000000]
       DB 8DH, 9BH, 00H, 00H, 00H, 00H
     else
      if size eq 7
	; lea esp, [esp+00000000]
	DB 8DH, 0A4H, 24H, 00H, 00H, 00H, 00H 
      else
       if size eq 8
        ; jmp .+8; .npad 6
	DB 0EBH, 06H, 8DH, 9BH, 00H, 00H, 00H, 00H
       else
        if size eq 9
         ; jmp .+9; .npad 7
         DB 0EBH, 07H, 8DH, 0A4H, 24H, 00H, 00H, 00H, 00H
        else
         if size eq 10
          ; jmp .+A; .npad 7; .npad 1
          DB 0EBH, 08H, 8DH, 0A4H, 24H, 00H, 00H, 00H, 00H, 90H
         else
          if size eq 11
           ; jmp .+B; .npad 7; .npad 2
           DB 0EBH, 09H, 8DH, 0A4H, 24H, 00H, 00H, 00H, 00H, 8BH, 0FFH
          else
           if size eq 12
            ; jmp .+C; .npad 7; .npad 3
            DB 0EBH, 0AH, 8DH, 0A4H, 24H, 00H, 00H, 00H, 00H, 8DH, 49H, 00H
           else
            if size eq 13
             ; jmp .+D; .npad 7; .npad 4
             DB 0EBH, 0BH, 8DH, 0A4H, 24H, 00H, 00H, 00H, 00H, 8DH, 64H, 24H, 00H
            else
             if size eq 14
              ; jmp .+E; .npad 7; .npad 5
              DB 0EBH, 0CH, 8DH, 0A4H, 24H, 00H, 00H, 00H, 00H, 05H, 00H, 00H, 00H, 00H
             else
              if size eq 15
               ; jmp .+F; .npad 7; .npad 6
               DB 0EBH, 0DH, 8DH, 0A4H, 24H, 00H, 00H, 00H, 00H, 8DH, 9BH, 00H, 00H, 00H, 00H
              else
	       %out error: unsupported npad size
               .err
              endif
             endif
            endif
           endif
          endif
         endif
        endif
       endif
      endif
     endif
    endif
   endif
  endif
 endif
endif
endm
\end{lstlisting}
 % subsection


\subsection{Strings in Bin\"ar finden}

\epigraph{Actually, the best form of Unix documentation is frequently running the
\textbf{strings} command over a program’s object code. Using \textbf{strings}, you can get
a complete list of the program’s hard-coded file name, environment variables,
undocumented options, obscure error messages, and so forth.}{The Unix-Haters Handbook}

\myindex{UNIX!strings}
Das Standard UNIX \emph{strings} Utility ist ein quick-n-dirty Weg um alle Strings in der 
Datei an zu schauen. Zum Beispiel, in der OpenSSH 7.2 sshd executable Datei gibt es einige Strings:

\lstinputlisting{digging_into_code/sshd_strings.txt}

Dort kann man Optionen, Fehler Meldungen, Datei Pfade, importierte dynamische Module, Funktionen und einige andere komische 
Strings (keys?) sehen. Es gibt auch nicht druckbare Zeichen---x86 Code enth\"alt chunks von druckbaren ASCII Zeichen, bis zu ca 8 Zeichen. % <-- bessere formulierung?

Sicher, OpenSSH ist ein open-source Programm.
Aber sich die lesbaren Strings eines unbekannten Programms an zuschauen ist meist der erste Schritt bei 
der Analyse. 
\myindex{UNIX!grep}

\emph{grep} kann genauso benutzt werden.

\myindex{Hiew}
\myindex{Sysinternals}
Hiew hat die gleichen F\"ahigkeiten (Alt-F6), genau wie der Sysinternals ProcessMonitor.

\subsection{Error/debug Narchichten}

Debugging Messages sind auch sehr n\"utzlich, wenn vorhanden.
Auf gewisse weise, melden die debug Narichten was gerade
im Programm vorgeht. Oft schreiben diese \printf-\"ahnlichen Funktionen, in
log-Dateien oder sie schreiben nirgends hin aber die calls zu den printf-\"ahnlichen Funktionen sind noch vorhanden, 
weil der build kein Debug build aber ein \emph{release} ist. % <-- nochmal \"uber formulierung nachdenken
\myindex{\oracle}

Wenn lokale oder globale Variablen in Debug messages geschrieben werden, kann das auch 
hilfreich sein da man so an die Variablen Namen kommt.
Zum Beispiel, eine solche Funktion in \oracle ist \TT{ksdwrt()}.

Textstrings mit Aussage sind auch Hilfreich.
Der \IDA disassembler zeigt welche Funktion und von welchem Punkt aus ein spezifischer String benutzt wird.
Manchmal passieren lustige Dinge dabei\footnote{\href{http://blog.yurichev.com/node/32}{blog.yurichev.com}}.

Fehlermeldungen helfen uns genauso.
In \oracle, werden Fehler von einer Gruppe von Funktionen gemeldet.
\"Uber das Thema kann man mehr hier erfahren: \href{http://blog.yurichev.com/node/43}{blog.yurichev.com}.

\myindex{Error messages}

Es ist M\"oglich heraus zu finden welche Funktionen Fehler melden und unter welchen Bedingungen.


\"Ubrigens, das ist f\"ur Kopierschutztsysteme oft der Grund kryptische Fehlermeldungen oder einfach nur 
Fehlernummer aus zu geben. Niemand ist gl\"ucklich dar\"uber wenn der Softwarecracker den Kopierschutz besser
versteht nur weil dieser durch eine Fehlermeldung ausgel\"ost wurde.

Ein Beispiel von verschl\"usselten Fehlermeldungen gibt es hier: \myref{examples_SCO}.

\subsection{Verd\"achtige magic strings}

Manche Magic Strings die in Hintert\"uren benutzt werden sehen schon ziemlich verd\"achtig aus.

Zum Beispiel, es gab eine Hintert\"ur im TP-Link WR740 Home Router\footnote{\url{http://sekurak.pl/tp-link-httptftp-backdoor/}}.
Die Hintert\"ur konnte aktiviert werden wenn man folgende URL aufrief:
\url{http://192.168.0.1/userRpmNatDebugRpm26525557/start_art.html}.\\

Tats\"achlich, kann man den Magic String \q{userRpmNatDebugRpm26525557} in der Firmware finden.

Der String war nicht googlebar bis die Information \"offentlich \"uber die Hintert\"ur \"offentlich verbreitet wurde.


Man w\"urde solche Informationen nat\"urlich auch nicht in irgendeinem \ac{RFC} finden.


Man w\"urde auch keinen Algorithmus finden der solch seltsame Byte Sequenzen benutzt.


Und es sieht auch nicht nach einer Fehler- order Debugnaricht aus.


Also es ist immer eine gute Idee so seltsamen Dinge genauer zu betrachten.

\myindex{base64}

Manchmal, sind solche Strings auch mit base64 codiert.

Es ist also immer eine gute Idee diese Stings zu Decodieren und sie visuell zu durchsuchen, ein Blick
kann schon gen\"ugen.

\myindex{Security through obscurity}
Pr\"aziser gesagt, diese Methode Hintert\"uren zu verstecken nennt man \q{security through obscurity}.
