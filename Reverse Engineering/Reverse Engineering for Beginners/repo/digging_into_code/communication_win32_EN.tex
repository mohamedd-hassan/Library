\mysection{Communication with the outer world (win32)}

Sometimes it's enough to observe some function's inputs and outputs in order to understand what it does.
That way you can save time.

Files and registry access: 
for the very basic analysis, Process Monitor\footnote{\url{http://technet.microsoft.com/en-us/sysinternals/bb896645.aspx}}
utility from SysInternals can help.

For the basic analysis of network accesses, Wireshark\footnote{\url{http://www.wireshark.org/}} can be useful.

But then you will have to look inside anyway. \\
\\
The first thing to look for is which functions from the \ac{OS}'s \ac{API}s and standard libraries are used.

If the program is divided into a main executable file and a group of DLL files, sometimes the names of the functions in these DLLs can help.

If we are interested in exactly what can lead to a call to \TT{MessageBox()} with specific text, 
we can try to find this text in the data segment, find the references to it and find the points
from which the control may be passed to the \TT{MessageBox()} call we're interested in.

\myindex{\CStandardLibrary!rand()}
If we are talking about a video game and we're interested in which events are more or less random in it,
we may try to find the \rand function or its replacements (like the Mersenne twister algorithm) and find the places
from which those functions are called, and more importantly, how are the results used.
% BUG in varioref: http://tex.stackexchange.com/questions/104261/varioref-vref-or-vpageref-at-page-boundary-may-loop
One example: \myref{chap:color_lines}. 

But if it is not a game, and \rand is still used, it is also interesting to know why.
There are cases of unexpected \rand usage in data compression algorithms (for encryption imitation):
\href{http://blog.yurichev.com/node/44}{blog.yurichev.com}.

\subsection{Often used functions in the Windows API}

These functions may be among the imported.
It is worth to note that not every function might be used in the code that was written by the programmer.
A lot of functions might be called from library functions and \ac{CRT} code.

Some functions may have the \GTT{-A} suffix for the ASCII version and \GTT{-W} for the Unicode version.

\begin{itemize}

\item
Registry access (advapi32.dll): 
RegEnumKeyEx, RegEnumValue, RegGetValue, RegOpenKeyEx, RegQueryValueEx.

\item
Access to text .ini-files (kernel32.dll): 
GetPrivateProfileString.

\item
Dialog boxes (user32.dll): 
MessageBox, MessageBoxEx, CreateDialog, SetDlgItemText, GetDlgItemText.

\item
Resources access (\myref{PEresources}): (user32.dll): LoadMenu.

\item
TCP/IP networking (ws2\_32.dll):
WSARecv, WSASend.

\item
File access (kernel32.dll):
CreateFile, ReadFile, ReadFileEx, WriteFile, WriteFileEx.

\item
High-level access to the Internet (wininet.dll): WinHttpOpen.

\item
Checking the digital signature of an executable file (wintrust.dll):
WinVerifyTrust.

\item
The standard MSVC library (if it's linked dynamically) (msvcr*.dll):
assert, itoa, ltoa, open, printf, read, strcmp, atol, atoi, fopen, fread, fwrite, memcmp, rand,
strlen, strstr, strchr.

\end{itemize}

\subsection{Extending trial period}

Registry access functions are frequent targets for those who try to crack trial period of some software, which may save
installation date/time into registry.

Another popular target are GetLocalTime() and GetSystemTime() functions:
a trial software, at each startup, must check current date/time somehow anyway.

\subsection{Removing nag dialog box}

A popular way to find out what causing popping nag dialog box is intercepting MessageBox(), 
CreateDialog() and CreateWindow() functions.

\subsection{tracer: Intercepting all functions in specific module}
\myindex{tracer}

\myindex{x86!\Instructions!INT3}
There are INT3 breakpoints in the \tracer, that are triggered only once, however, they can be set for all functions
in a specific DLL.

\begin{lstlisting}
--one-time-INT3-bp:somedll.dll!.*
\end{lstlisting}

Or, let's set INT3 breakpoints on all functions with the \TT{xml} prefix in their name:

\begin{lstlisting}
--one-time-INT3-bp:somedll.dll!xml.*
\end{lstlisting}

On the other side of the coin, such breakpoints are triggered only once.
Tracer will show the call of a function, if it happens, but only once.
Another drawback---it is impossible to see the function's arguments.

Nevertheless, this feature is very useful when you know that the program uses a DLL,
but you do not know which functions are actually used.
And there are a lot of functions. 

\par
\myindex{Cygwin}
For example, let's see, what does the uptime utility from cygwin use:

\begin{lstlisting}
tracer -l:uptime.exe --one-time-INT3-bp:cygwin1.dll!.*
\end{lstlisting}

Thus we may see all that cygwin1.dll library functions that were called at least once, and where from:

\lstinputlisting{digging_into_code/uptime_cygwin.txt}

