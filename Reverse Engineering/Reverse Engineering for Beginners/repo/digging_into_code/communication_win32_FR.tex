\mysection{Communication avec le monde extérieur (win32)}

Parfois, il est suffisant d'observer les entrées/sorties d'une fonction pour comprendre
ce qu'elle fait.
Ainsi, vous pouvez gagner du temps.

Accès aux fichiers et au registre:
pour les analyses très basiques, l'utilitaire, Process Monitor\footnote{\url{http://technet.microsoft.com/en-us/sysinternals/bb896645.aspx}}
de SysInternals peut aider.

Pour l'analyse basique des accès au réseau, Wireshark\footnote{\url{http://www.wireshark.org/}}
peut être utile.

Mais vous devrez de toutes façons regarder à l'intérieur, \\
\\
Les premières choses à chercher sont les fonctions des \ac{API}s de l'\ac{OS} et
des bibliothèques standards qui sont utilisées.

Si le programme est divisé en un fichier exécutable et un groupe de fichiers DLL,
parfois le nom des fonctions dans ces DLLs peut aider.

Si nous sommes intéressés par exactement ce qui peut conduire à appeler \TT{MessageBox()}
avec un texte spécifique, nous pouvons essayer de trouver ce texte dans le segment
de données, trouver sa référence et trouver les points depuis lesquels le contrôle
peut être passé à l'appel à \TT{MessageBox()} qui nous intéresse.

\myindex{\CStandardLibrary!rand()}
Si nous parlons d'un jeu vidéo et que nous sommes intéressés par les évènements qui
y sont plus ou moins aléatoires, nous pouvons essayer de trouver la fonction \rand
ou sa remplaçante (comme l'algorithme du twister de Mersenne) et trouver les points
depuis lesquels ces fonctions sont appelées, et plus important, comment les résultats
sont utilisés.
% BUG in varioref: http://tex.stackexchange.com/questions/104261/varioref-vref-or-vpageref-at-page-boundary-may-loop
Un exemple: \myref{chap:color_lines}.

Mais si ce n'est pas un jeu, et que \rand est toujours utilisé, il est intéressant
de savoir pourquoi.
Il a y des cas d'utilisation inattendu de \rand dans des algorithmes de compression
de données (pour une imitation du chiffrement):
\href{http://blog.yurichev.com/node/44}{blog.yurichev.com}.

\subsection{Fonctions souvent utilisées dans l'API Windows}

Ces fonctions peuvent être parmi les fonctions importées.
Il est utile de noter que toutes les fonctions ne sont pas forcément utilisées dans
du code écrit par le programmeur.
Beaucoup de fonctions peuvent être appelées depuis des fonctions de bibliothèque
et du code \ac{CRT}.

Certaines fonctions peuvent avoir le suffixe \GTT{-A} pour la version ASCII et \GTT{-W}
pour la version Unicode.

\begin{itemize}

\item
Accès au registre (advapi32.dll): 
RegEnumKeyEx, RegEnumValue, RegGetValue, RegOpenKeyEx, RegQueryValueEx.

\item
Accès au text des fichiers .ini (kernel32.dll):
GetPrivateProfileString.

\item
Boites de dialogue (user32.dll):
MessageBox, MessageBoxEx, CreateDialog, SetDlgItemText, GetDlgItemText.

\item
Accès aux resources (\myref{PEresources}): (user32.dll): LoadMenu.

\item
Réseau TCP/IP (ws2\_32.dll):
WSARecv, WSASend.

\item
Accès fichier (kernel32.dll):
CreateFile, ReadFile, ReadFileEx, WriteFile, WriteFileEx.

\item
Accès haut niveau à Internet (wininet.dll):
WinHttpOpen.

\item
Vérifier la signature digitale d'uin fichier exécutable (wintrust.dll):
WinVerifyTrust.

\item
La bibliothèque MSVC standard (si elle est liée dynamiquement) (msvcr*.dll):
assert, itoa, ltoa, open, printf, read, strcmp, atol, atoi, fopen, fread, fwrite, memcmp, rand,
strlen, strstr, strchr.

\end{itemize}

\subsection{Étendre la période d'essai}

Les fonctions d'accès au registre sont des cibles fréquentes pour ceux qui veulent
essayer de craquer des logiciels avec période d'essai, qui peuvent sauvegarder la
date et l'heure dans un registre.

Des autres cibles courantes sont les fonctions GetLocalTime() et GetSystemTime():
un logiciel avec période d'essai, à chaque démarrage, doit de toutes façons vérifier
la date et l'heure d'une certaine façon.

\subsection{Supprimer la boite de dialogue nag}

Une manière répandue de trouver ce qui cause l'apparition de la boite de dialogue
nag est d'intercepter les fonctions MessageBox(), CreateDialog() et CreateWindow().

\subsection{tracer: Intercepter toutes les fonctions dans un module spécifique}
\myindex{tracer}

\myindex{x86!\Instructions!INT3}
Il y a un point d'arrêt INT3 dans \tracer, qui peut être déclenché seulement une
fois, toutefois, il peut être mis pour toutes les fonctions dans une DLL spécifique.

\begin{lstlisting}
--one-time-INT3-bp:somedll.dll!.*
\end{lstlisting}

Ou, mettons un point d'arrêt INT3 sur toutes les fonctions avec le préfixe \TT{xml}
dans leur nom:

\begin{lstlisting}
--one-time-INT3-bp:somedll.dll!xml.*
\end{lstlisting}

Le revers de la médaille est que de tels points d'arrêt ne sont déclenchés qu'une fois.
Tracer montrera l'appel à une fonction, s'il se produit, mais seulement une fois.
Un autre inconvénient---il est impossible de voir les arguments de la fonction.

Néanmoins, cette fonctionnalité est très utile lorsque vous avez qu'un programme
utilise une DLL, mais que vous ne savez pas quelles fonctions sont effectivement
utilisées.
Et il y a beaucoup de fonctions.

\par
\myindex{Cygwin}
Par exemple, regardons ce qu'utilise l'utilitaire uptime de Cygwin:

\begin{lstlisting}
tracer -l:uptime.exe --one-time-INT3-bp:cygwin1.dll!.*
\end{lstlisting}

Ainsi nous pouvons voir quelles sont les fonctions de la bibliothèque cygwin1.dll
qui sont appelées au moins une fois, et depuis où:

\lstinputlisting{digging_into_code/uptime_cygwin.txt}

