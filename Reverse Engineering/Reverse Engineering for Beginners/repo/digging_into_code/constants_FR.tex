\mysection{Constantes}

Les humains, programmeurs inclus, utilisent souvent des nombres ronds, comme 10, 100,
1000, dans la vie courante comme dans le code.

Le rétro ingénieur pratiquant connaît en général bien leur représentation décimale:
10=0xA, 100=0x64, 1000=0x3E8, 10000=0x2710.

Les constantes \TT{0xAAAAAAAA} (0b10101010101010101010101010101010) et \\
\TT{0x55555555} (0b01010101010101010101010101010101)  sont aussi répandues---elles
sont composées d'alternance de bits.

Cela peut aider à distinguer un signal d'un signal dans lequel tous les bits sont
à 1 (0b1111 \dots) ou à 0 (0b0000 \dots).
Par exemple, la constante \TT{0x55AA} est utilisée au moins dans le secteur de boot,
\ac{MBR}, et dans la \ac{ROM} de cartes d'extention de compatible IBM.

Certains algorithmes, particulièrement ceux de chiffrement, utilisent des constantes
distinctes, qui sont faciles à trouver dans le code en utilisant \IDA.

\myindex{MD5}

Par exemple, l'algorithme MD5 initialise ses propres
variables internes comme ceci:

\begin{verbatim}
var int h0 := 0x67452301
var int h1 := 0xEFCDAB89
var int h2 := 0x98BADCFE
var int h3 := 0x10325476
\end{verbatim}

Si vous trouvez ces quatre constantes utilisées à la suite dans du code, il est
très probable que cette fonction soit relatives à MD5.

\par Un autre exemple sont les algorithmes CRC16/CRC32, ces algorithmes de calcul
utilisent souvent des tables pré-calculées comme celle-ci:

\begin{lstlisting}[caption=linux/lib/crc16.c,style=customc]
/** CRC table for the CRC-16. The poly is 0x8005 (x^16 + x^15 + x^2 + 1) */
u16 const crc16_table[256] = {
	0x0000, 0xC0C1, 0xC181, 0x0140, 0xC301, 0x03C0, 0x0280, 0xC241,
	0xC601, 0x06C0, 0x0780, 0xC741, 0x0500, 0xC5C1, 0xC481, 0x0440,
	0xCC01, 0x0CC0, 0x0D80, 0xCD41, 0x0F00, 0xCFC1, 0xCE81, 0x0E40,
	...
\end{lstlisting}

Voir aussi la table pré-calculée pour CRC32: \myref{sec:CRC32}.

Dans les algorithmes CRC sans table, des polynômes bien connus sont utilisés, par
exemple 0xEDB88320 pour CRC32.

\subsection{Nombres magiques}
\label{magic_numbers}

De nombreux formats de fichier définissent un entête standard où un \emph{nombre(s) magique}
est utilisé, unique ou même plusieurs.

\myindex{MS-DOS}

Par exemple, tous les exécutables Win32 et MS-DOS débutent par ces deux caractères \q{MZ}\footnote{\href{http://en.wikipedia.org/wiki/DOS_MZ_executable}{Wikipédia}}.

\myindex{MIDI}

Au début d'un fichier MIDI, la signature \q{MThd} doit être présente.
Si nous avons un programme qui utilise des fichiers MIDI pour quelque chose, il
est très probable qu'il doit vérifier la validité du fichier en testant au moins
les 4 premiers octets.

Ça peut être fait comme ceci:
(\emph{buf} pointe sur le début du fichier chargé en mémoire)

\begin{lstlisting}[style=customasmx86]
cmp [buf], 0x6468544D ; "MThd"
jnz _error_not_a_MIDI_file
\end{lstlisting}

\myindex{\CStandardLibrary!memcmp()}
\myindex{x86!\Instructions!CMPSB}

\dots ou en appelant une fonction pour comparer des blocs de mémoire comme \TT{memcmp()}
ou tout autre code équivalent jusqu'à une instruction \TT{CMPSB} (\myref{REPE_CMPSx}).

Lorsque vous trouvez un tel point, vous pouvez déjà dire que le chargement du fichier
MIDI commence, ainsi, vous pouvez voir l'endroit où se trouve le buffer avec le contenu
du fichier MIDI, ce qui est utilisé dans le buffer et comment.

\subsubsection{Dates}

\myindex{UFS2}
\myindex{FreeBSD}
\myindex{HASP}

Souvent, on peut rencontrer des nombres comme \TT{0x19870116}, qui ressemble clairement
à une date (année 1987, 1er mois (janvier), 16ème jour).
Ça peut être la date de naissance de quelqu'un (un programmeur, une de ses relations,
un enfant), ou une autre date importante.
La date peut aussi être écrite dans l'ordre inverse, comme \TT{0x16011987}.
Les dates au format américain sont aussi courante, comme \TT{0x01161987}.

Un exemple célèbre est \TT{0x19540119} (nombre magique utilisé dans la structure
du super-bloc UFS2), qui est la date de naissance de Marshall Kirk McKusick, éminent
contributeur FreeBSD.

\myindex{Stuxnet}
Stuxnet utilise le nombre ``19790509'' (pas comme un nombre 32-bit, mais comme une
chaîne, toutefois), et ça a conduit à spéculer que le malware était relié à Israël%
\footnote{C'est la date d'exécution de Habib Elghanian, juif persan.}.

Aussi, des nombres comme ceux-ci sont très répandus dans dans le chiffrement niveau
amateur, par exemple, extrait de la \emph{fonction secrète} des entrailles du dongle
HASP3\footnote{\url{https://web.archive.org/web/20160311231616/http://www.woodmann.com/fravia/bayu3.htm}}:

\begin{lstlisting}[style=customc]
void xor_pwd(void)
{
	int i;

	pwd^=0x09071966;
	for(i=0;i<8;i++)
	{
		al_buf[i]= pwd & 7; pwd = pwd >> 3;
	}
};

void emulate_func2(unsigned short seed)
{
	int i, j;
	for(i=0;i<8;i++)
	{
		ch[i] = 0;

		for(j=0;j<8;j++)
		{
			seed *= 0x1989;
			seed += 5;
			ch[i] |= (tab[(seed>>9)&0x3f]) << (7-j);
		}
	}
}
\end{lstlisting}

\subsubsection{DHCP}

Ceci s'applique aussi aux protocoles réseaux.
Par exemple, les paquets réseau du protocole DHCP contiennent un soi-disant \emph{nombre
magique}: \TT{0x63538263}.
Tout code qui génère des paquets DHCP doit contenir quelque part cette constante
à insérer dans les paquets.
Si nous la trouvons dans du code, nous pouvons trouver ce qui s'y passe, et pas seulement ça.
Tout programme qui peut recevoir des paquet DHCP doit vérifier le \emph{cookie magique},
et le comparer à cette constante.

Par exemple, prenons le fichier dhcpcore.dll de Windows 7 x64 et cherchons cette constante.
Et nous la trouvons, deux fois:
Il semble que la constante soit utilisée dans deux fonctions avec des noms parlants\\
\TT{DhcpExtractOptionsForValidation()} et \TT{DhcpExtractFullOptions()}:

\begin{lstlisting}[caption=dhcpcore.dll (Windows 7 x64),style=customasmx86]
.rdata:000007FF6483CBE8 dword_7FF6483CBE8 dd 63538263h          ; DATA XREF: DhcpExtractOptionsForValidation+79
.rdata:000007FF6483CBEC dword_7FF6483CBEC dd 63538263h          ; DATA XREF: DhcpExtractFullOptions+97
\end{lstlisting}

Et ici sont les endroits où ces constantes sont accédées:

\begin{lstlisting}[caption=dhcpcore.dll (Windows 7 x64),style=customasmx86]
.text:000007FF6480875F  mov     eax, [rsi]
.text:000007FF64808761  cmp     eax, cs:dword_7FF6483CBE8
.text:000007FF64808767  jnz     loc_7FF64817179
\end{lstlisting}

Et:

\begin{lstlisting}[caption=dhcpcore.dll (Windows 7 x64),style=customasmx86]
.text:000007FF648082C7  mov     eax, [r12]
.text:000007FF648082CB  cmp     eax, cs:dword_7FF6483CBEC
.text:000007FF648082D1  jnz     loc_7FF648173AF
\end{lstlisting}

\subsection{Constantes spécifiques}

Parfois, il y a une constante spécifique pour un certain type de code.
Par exemple, je me suis plongé une fois dans du code, où le nombre 12 était rencontré
anormalement souvent.
La taille de nombreux tableaux était 12 ou un multiple de 12 (24, etc.).
Il s'est avéré que ce code prenait des fichiers audio de 12 canaux en entrée et les
traitait.

Et vice versa: par exemple, si un programme fonctionne avec des champs de texte qui
ont une longueur de 120 octets, il doit y avoir une constante 120 ou 119 quelque
part dans le code.
Si UTF-16 est utilisé, alors $2 \cdot 120$.
Si le code fonctionne avec des paquets réseau de taille fixe, c'est une bonne idée
de chercher cette constante dans le code.

C'est aussi vrai pour le chiffrement amateur (clefs de licence, etc.).
Si le bloc chiffré a une taille de $n$ octets, vous pouvez essayer de trouver des
occurrences de ce nombre à travers le code.
Aussi, si vous voyez un morceau de code qui est répété $n$ fois dans une boucle durant
l'exécution, ceci peut être une routine de chiffrement/déchiffrement.

\subsection{Chercher des constantes}

C'est facile dans \IDA: Alt-B or Alt-I.
\myindex{binary grep}
Et pour chercher une constante dans un grand nombre de fichiers, ou pour chercher
dans des fichiers non exécutables, il y a un petit utilitaire appelé \emph{binary grep}\footnote{\BGREPURL}.

