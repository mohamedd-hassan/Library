\subsection{Exemple \#3: MS-DOS}
\label{dongle_16bit_dos}

\myindex{MS-DOS}
Un autre très vieux logiciel pour MS-DOS de 1995, lui aussi développé par une société
disparue depuis longtemps.

\myindex{Intel!8086}
\myindex{Intel!80286}

À l' ère pré-DOS extenders, presque tous les logiciels pour MS-DOS s'appuyaient sur
sur des CPUs 8086 ou 80286, donc la code était massivement 16-bit.

Le code 16-bit est presque le même que celui déjà vu dans le livre, mais tous les
registres sont 16-bit et il y a moins d'instructions disponibles.

\label{IN_example}
\label{OUT_example}
\myindex{x86!\Instructions!IN}
\myindex{x86!\Instructions!OUT}

L'environnement MS-DOS n'avait pas de système de drivers, et n'importe quel programme
pouvait s'adresser au matériel via les ports, donc vous pouvez voir ici les instructions
\TT{OUT}/\TT{IN}, qui sont présentes dans la plupart des drivers de nos jours (il
est impossible d'accéder directement aux ports en \glslink{user mode}{mode utilisateur}
sur tous les \ac{OS}es modernes).

Compte tenu de ceci, le programme MS-DOS qui fonctionne avec un dongle doit accéder
le port imprimante LPT directement.

Donc nous devons simplement chercher des telles instructions. Et oui, elles y sont:

\lstinputlisting[style=customasmx86]{examples/dongles/3/1.lst}

(J'ai donné tous les noms de label dans cet exemple).

\TT{out\_port()} est référencé dans une seule fonction:

\lstinputlisting[style=customasmx86]{examples/dongles/3/2.lst}

Ceci est un \q{hashing} dongle Sentinel Pro, comme dans l'exemple précédent.
C'est remarquable car des chaînes de texte sont passées ici, aussi, et des valeurs 16-bit
sont renvoyées, puis comparées avec d'autres.

Donc, voici comment le Sentinel Pro est accédé via les ports.

L'adresse du port de sortie est en général 0x378, i.e., le port imprimante, où les
données pour les vieilles imprimantes de l'ère pré-USB étaient passées.

Le port est uni-directionnel, car lorsqu'il a été développé, personne n'imaginait
que quelqu'un aurait besoin de transférer de l'information depuis l'imprimante
\footnote{Si nous considérons seulement Centronics. Le standard IEEE 1284 suivant
permet le transfert d'information depuis l'imprimante.}.

Le seul moyen d'obtenir des informations de l'imprimante est le registre d'état sur
le port 0x379, qui contient des bits tels que \q{paper out}, \q{ack}, \q{busy}---ainsi
l'imprimante peut signaler si elle est prête ou non et si elle a du papier.

Donc, le dongle renvoie de l'information dans l'un de ces bits, un bit à chaque itération.

\TT{\_in\_port\_2} contient l'adresse du mot d'état (0x379) et \TT{\_in\_port\_1} contient
le registre de contrôle d'adresse (0x37A).

Il semble que le dongle renvoie de l'information via le flag \q{busy} en \TT{seg030:00B9}:
chaque bit est stocké dans le registre \TT{DI}, qui est renvoyé à la fin de la fonction.

Que signifie tous ces octets envoyés sur le port de sortie?
Difficile à dire. Peut-être des commandes pour le dongle.

Mais d'une manière générale, il n'est pas nécessaire de savoir: il est facile de
résoudre notre tâche sans le savoir.

Voici la routine de vérification du dongle:

\lstinputlisting[style=customasmx86]{examples/dongles/3/3.lst}

Puisque la routine peut être appelée très fréquemment, e.g., avant l'exécution de
chaque fonctionnalité importante du logiciel, et accéder au ongle est en général
lent (à cause du port de l'imprimante et aussi du \ac{MCU} lent du dongle), ils ont
probablement ajouté un moyen d'éviter le test du dongle, en vérifiant l'heure courante
dans la fonction \TT{biostime()}.

\myindex{\CStandardLibrary!rand()}
La fonction \TT{get\_rand()} utilise la fonction C standard:

\lstinputlisting[style=customasmx86]{examples/dongles/3/4.lst}

Donc la chaîne de texte est choisi au hasard, passé au dongle, et ensuite le résultat
du hachage est comparé à la valeur correcte.

Les chaînes de texte semblent être construites aléatoirement aussi, lors du développement
du logiciel.

Et voici comment la fonction principale de vérification du dongle est appelée:

\lstinputlisting[style=customasmx86]{examples/dongles/3/5.lst}

Il est facile de contourner le dongle, il suffit de forcer la fonction \TT{check\_dongle()}
à renvoyer toujours 0.

Par exemple, en insérant du code à son début:

\begin{lstlisting}[style=customasmx86]
mov ax,0
retf
\end{lstlisting}

\myindex{\CStandardLibrary!strcpy()}

Le lecteur attentif peut se rappeler que la fonction C \TT{strcpy()} prend en général
deux pointeurs dans ses arguments, mais nous voyons que 4 valeurs sont passées:

\lstinputlisting[style=customasmx86]{examples/dongles/3/tmp1.lst}

Ceci est relatif au modèle de mémoire de MS-DOS. Vous pouvez en lire plus à ce sujet ici:
\myref{8086_memory_model}.

Donc, comme vous pouvez le voir, \TT{strcpy()} et toute autre fonction qui prend
un/des pointeur(s) en argument travaille avec des paires 16-bit.

Retournons à notre exemple.
\TT{DS} est actuellement l'adresse du segment de données dans l'exécutable, où la
chaîne de texte est stockée.

\myindex{x86!\Instructions!LES}

Dans la fonction \TT{sent\_pro()}, chaque octet de la chaîne est chargé en\\
\TT{seg030:00EF}: l'instruction \TT{LES} charge simultanément la paire ES:BX depuis
l'argument transmis.

Le \MOV en \TT{seg030:00F5} charge l'octet depuis la mémoire sur laquelle pointe
la paire ES:BX.

% TODO rewrite
%
%At \TT{seg030:00F2} only a second 16-bit part of address is \glslink{increment}{incremented}.
%
%This implies that the string passed to the function cannot be located on the boundary between two data segments.
