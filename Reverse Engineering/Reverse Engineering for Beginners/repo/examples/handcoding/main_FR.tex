\mysection{Code assembleur écrit à la main}

\subsection{ Fichier test EICAR}
\label{subsec:EICAR}

\myindex{MS-DOS}
\myindex{EICAR}
Ce fichier .COM est destiné à tester les logiciels anti-virus, il est possible de
le lancer sous MS-DOS et il affiche cette chaîne: \q{EICAR-STANDARD-ANTIVIRUS-TEST-FILE!}
\footnote{\href{https://en.wikipedia.org/wiki/EICAR_test_file}{Wikipédia}}.
% FIXME1 \myref{} -> about .COM files

Une de ses propriété importante est qu'il est entièrement composé de symboles ASCII
affichables, qui, de fait, permet de le créer dans n'importe quel éditeur de texte:

\begin{lstlisting}
X5O!P%@AP[4\PZX54(P^)7CC)7}$EICAR-STANDARD-ANTIVIRUS-TEST-FILE!$H+H*
\end{lstlisting}

Décompilons-le:

\lstinputlisting[style=customasmx86]{examples/handcoding/EICAR_FR.lst}

J'ai ajouté des commentaires à propos des registres et de la pile après chaque instruction.

En gros, toutes ces instructions sont là seulement pour exécuter ce code:

\begin{lstlisting}[style=customasmx86]
B4 09     MOV AH, 9
BA 1C 01  MOV DX, 11Ch
CD 21     INT 21h
CD 20     INT 20h
\end{lstlisting}

\myindex{x86!\Instructions!INT}
\TT{INT 21h} avec la 9ème fonction (passée dans \TT{AH}) affiche simplement une chaîne,
dont l'adresse est passée dans \TT{DS:DX}.
À propos, la chaîne doit être terminée par le signe '\$'.
Apparemment, c'est hérité de \gls{CP/M} et cette fonction a été laissée dans DOS
pour la compatibilité.
\TT{INT 20h} renvoie au DOS.

Mais comme on peut le voir, l'opcode de cette instruction n'est pas strictement
affichable.
Donc, la partie principale du fichier EICAR est:

\begin{itemize}
\item prépare les valeurs du registre dont nous avons besoin (AH et DX);
\item prépare les opcodes INT 21 et INT 20 en mémoire;
\item exécute INT 21 et INT 20.
\end{itemize}

\myindex{Shellcode}

À propos, cette technique est largement utilisée dans la construction de shellcode,
lorsque l'on doit passer le code x86 sous la forme d'une chaîne.

Voici aussi une liste de toutes les instructions x86 qui ont des opcodes affichables:
\myref{printable_x86_opcodes}.
