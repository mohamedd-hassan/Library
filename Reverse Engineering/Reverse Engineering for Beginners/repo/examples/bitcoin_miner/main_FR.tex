\mysection{Overclocker le mineur de Bitcoin Cointerra}
\myindex{Bitcoin}
\myindex{BeagleBone}

Il y avait le mineur de Bitcoin Cointerra, ressemblant à ceci:

\begin{figure}[H]
\centering
\myincludegraphics{examples/bitcoin_miner/board.jpg}
\caption{Carte}
\end{figure}

Et il y avait aussi (peut-être leaké) l'utilitaire\footnote{Peut être téléchargé ici: \url{\RepoURL/examples/bitcoin_miner/files/cointool-overclock}}
qui peut définir la fréquence d'horloge pour la carte.
Il fonctionne sur une carte additionnelle BeagleBone Linux ARM (petite carte en bas
de l'image).

Et on m'avait demandé une fois s'il est possible de modifier cet utilitaire pour voir
quelles sont les fréquences qui peuvent être définies, et celles qui ne peuvent pas
l'être.
Et est-il possible de l'ajuster?

L'utilitaire doit être exécuté comme cela: \TT{./cointool-overclock 0 0 900}, où 900
est la fréquence en MHz.
Si la fréquence est trop grande, l'utilitaire affiche \q{Error with arguments} et
se termine.

Ceci est le morceau de code autour de la référence à la chaîne de texte \q{Error with arguments}:

\begin{lstlisting}[style=customasmARM]

...

.text:0000ABC4         STR      R3, [R11,#var_28]
.text:0000ABC8         MOV      R3, #optind
.text:0000ABD0         LDR      R3, [R3]
.text:0000ABD4         ADD      R3, R3, #1
.text:0000ABD8         MOV      R3, R3,LSL#2
.text:0000ABDC         LDR      R2, [R11,#argv]
.text:0000ABE0         ADD      R3, R2, R3
.text:0000ABE4         LDR      R3, [R3]
.text:0000ABE8         MOV      R0, R3  ; nptr
.text:0000ABEC         MOV      R1, #0  ; endptr
.text:0000ABF0         MOV      R2, #0  ; base
.text:0000ABF4         BL       strtoll
.text:0000ABF8         MOV      R2, R0
.text:0000ABFC         MOV      R3, R1
.text:0000AC00         MOV      R3, R2
.text:0000AC04         STR      R3, [R11,#var_2C]
.text:0000AC08         MOV      R3, #optind
.text:0000AC10         LDR      R3, [R3]
.text:0000AC14         ADD      R3, R3, #2
.text:0000AC18         MOV      R3, R3,LSL#2
.text:0000AC1C         LDR      R2, [R11,#argv]
.text:0000AC20         ADD      R3, R2, R3
.text:0000AC24         LDR      R3, [R3]
.text:0000AC28         MOV      R0, R3  ; nptr
.text:0000AC2C         MOV      R1, #0  ; endptr
.text:0000AC30         MOV      R2, #0  ; base
.text:0000AC34         BL       strtoll
.text:0000AC38         MOV      R2, R0
.text:0000AC3C         MOV      R3, R1
.text:0000AC40         MOV      R3, R2
.text:0000AC44         STR      R3, [R11,#third_argument]
.text:0000AC48         LDR      R3, [R11,#var_28]
.text:0000AC4C         CMP      R3, #0
.text:0000AC50         BLT      errors_with_arguments
.text:0000AC54         LDR      R3, [R11,#var_28]
.text:0000AC58         CMP      R3, #1
.text:0000AC5C         BGT      errors_with_arguments
.text:0000AC60         LDR      R3, [R11,#var_2C]
.text:0000AC64         CMP      R3, #0
.text:0000AC68         BLT      errors_with_arguments
.text:0000AC6C         LDR      R3, [R11,#var_2C]
.text:0000AC70         CMP      R3, #3
.text:0000AC74         BGT      errors_with_arguments
.text:0000AC78         LDR      R3, [R11,#third_argument]
.text:0000AC7C         CMP      R3, #0x31
.text:0000AC80         BLE      errors_with_arguments
.text:0000AC84         LDR      R2, [R11,#third_argument]
.text:0000AC88         MOV      R3, #950
.text:0000AC8C         CMP      R2, R3
.text:0000AC90         BGT      errors_with_arguments
.text:0000AC94         LDR      R2, [R11,#third_argument]
.text:0000AC98         MOV      R3, #0x51EB851F
.text:0000ACA0         SMULL    R1, R3, R3, R2
.text:0000ACA4         MOV      R1, R3,ASR#4
.text:0000ACA8         MOV      R3, R2,ASR#31
.text:0000ACAC         RSB      R3, R3, R1
.text:0000ACB0         MOV      R1, #50
.text:0000ACB4         MUL      R3, R1, R3
.text:0000ACB8         RSB      R3, R3, R2
.text:0000ACBC         CMP      R3, #0
.text:0000ACC0         BEQ      loc_ACEC
.text:0000ACC4
.text:0000ACC4 errors_with_arguments
.text:0000ACC4                                         
.text:0000ACC4         LDR      R3, [R11,#argv]
.text:0000ACC8         LDR      R3, [R3]
.text:0000ACCC         MOV      R0, R3  ; path
.text:0000ACD0         BL       __xpg_basename
.text:0000ACD4         MOV      R3, R0
.text:0000ACD8         MOV      R0, #aSErrorWithArgu ; format
.text:0000ACE0         MOV      R1, R3
.text:0000ACE4         BL       printf
.text:0000ACE8         B        loc_ADD4
.text:0000ACEC ; ------------------------------------------------------------
.text:0000ACEC
.text:0000ACEC loc_ACEC                 ; CODE XREF: main+66C
.text:0000ACEC         LDR      R2, [R11,#third_argument]
.text:0000ACF0         MOV      R3, #499
.text:0000ACF4         CMP      R2, R3
.text:0000ACF8         BGT      loc_AD08
.text:0000ACFC         MOV      R3, #0x64
.text:0000AD00         STR      R3, [R11,#unk_constant]
.text:0000AD04         B        jump_to_write_power
.text:0000AD08 ; ------------------------------------------------------------
.text:0000AD08
.text:0000AD08 loc_AD08                 ; CODE XREF: main+6A4
.text:0000AD08         LDR      R2, [R11,#third_argument]
.text:0000AD0C         MOV      R3, #799
.text:0000AD10         CMP      R2, R3
.text:0000AD14         BGT      loc_AD24
.text:0000AD18         MOV      R3, #0x5F
.text:0000AD1C         STR      R3, [R11,#unk_constant]
.text:0000AD20         B        jump_to_write_power
.text:0000AD24 ; ------------------------------------------------------------
.text:0000AD24
.text:0000AD24 loc_AD24                 ; CODE XREF: main+6C0
.text:0000AD24         LDR      R2, [R11,#third_argument]
.text:0000AD28         MOV      R3, #899
.text:0000AD2C         CMP      R2, R3
.text:0000AD30         BGT      loc_AD40
.text:0000AD34         MOV      R3, #0x5A
.text:0000AD38         STR      R3, [R11,#unk_constant]
.text:0000AD3C         B        jump_to_write_power
.text:0000AD40 ; ------------------------------------------------------------
.text:0000AD40
.text:0000AD40 loc_AD40                 ; CODE XREF: main+6DC
.text:0000AD40         LDR      R2, [R11,#third_argument]
.text:0000AD44         MOV      R3, #999
.text:0000AD48         CMP      R2, R3
.text:0000AD4C         BGT      loc_AD5C
.text:0000AD50         MOV      R3, #0x55
.text:0000AD54         STR      R3, [R11,#unk_constant]
.text:0000AD58         B        jump_to_write_power
.text:0000AD5C ; ------------------------------------------------------------
.text:0000AD5C
.text:0000AD5C loc_AD5C                 ; CODE XREF: main+6F8
.text:0000AD5C         LDR      R2, [R11,#third_argument]
.text:0000AD60         MOV      R3, #1099
.text:0000AD64         CMP      R2, R3
.text:0000AD68         BGT      jump_to_write_power
.text:0000AD6C         MOV      R3, #0x50
.text:0000AD70         STR      R3, [R11,#unk_constant]
.text:0000AD74
.text:0000AD74 jump_to_write_power                     ; CODE XREF: main+6B0
.text:0000AD74                                         ; main+6CC ...
.text:0000AD74         LDR      R3, [R11,#var_28]
.text:0000AD78         UXTB     R1, R3
.text:0000AD7C         LDR      R3, [R11,#var_2C]
.text:0000AD80         UXTB     R2, R3
.text:0000AD84         LDR      R3, [R11,#unk_constant]
.text:0000AD88         UXTB     R3, R3
.text:0000AD8C         LDR      R0, [R11,#third_argument]
.text:0000AD90         UXTH     R0, R0
.text:0000AD94         STR      R0, [SP,#0x44+var_44]
.text:0000AD98         LDR      R0, [R11,#var_24]
.text:0000AD9C         BL       write_power
.text:0000ADA0         LDR      R0, [R11,#var_24]
.text:0000ADA4         MOV      R1, #0x5A
.text:0000ADA8         BL       read_loop
.text:0000ADAC         B        loc_ADD4

...

.rodata:0000B378 aSErrorWithArgu DCB "%s: Error with arguments",0xA,0 ; DATA XREF: main+684

...

\end{lstlisting}

Les noms de fonctions étaient présents dans les informations de débogage du binaire
original, comme \TT{write\_power}, \TT{read\_loop}.
Mais j'ai nommé les labels à l'intérieur des fonctions.

\myindex{UNIX!getopt}
\myindex{strtoll()}
Le nom \TT{optind} semble familier. Il provient de la bibliothèque *NIX \emph{getopt}
qui sert à traiter les arguments de la ligne de commande---bien, c'est exactement
ce qui se passe dans ce code.
Ensuite, le 3ème argument (où la valeur de la fréquence est passée) est converti
d'une chaîne vers un nombre en utilisant un appel à la fonction \emph{strtoll()}.

La valeur est ensuite encore comparée par rapport à diverses constantes.
En 0xACEC, elle est testée, si elle est inférieure ou égale à 499, et si c'est le
cas, 0x64 est passé à la fonction \TT{write\_power()} (qui envoie une commande par
USB en utilisant \TT{send\_msg()}).
Si elle est plus grande que 499, un saut en 0xAD08 se produit.

En 0xAD08 on teste si elle est inférieure ou égale à 799. En cas de succès 0x5F est
alors passé à la fonction \TT{write\_power()}.

Il y a d'autres tests: par rapport à 899 en 0xAD24, à 0x999 en 0xAD40 et enfin,
à 1099 en 0xAD5C.
Si la fréquence est inférieure ou égale à 1099, 0x50 est passé (en 0xAD6C) à la fonction
\TT{write\_power()}.
Et il y a une sorte de bug.
Si la valeur est encore plus grande que 1099, la valeur elle-même est passée à la
fonction \TT{write\_power()}.
Oh, ce n'est pas un bug, car nous ne pouvons pas arriver là: la valeur est d'abord
comparée à 950 en 0xAC88, et si elle est plus grande, un message d'erreur est
affiché et l'utilitaire s'arrête.

Maintenant, la table des fréquences en MHz et la valeur passée à la fonction \TT{write\_power()}:

\begin{center}
\begin{longtable}{ | l | l | l | }
\hline
\HeaderColor MHz & \HeaderColor héxadecimal & \HeaderColor décimal \\
\hline
499MHz & 0x64 & 100 \\
\hline
799MHz & 0x5f & 95 \\
\hline
899MHz & 0x5a & 90 \\
\hline
999MHz & 0x55 & 85 \\
\hline
1099MHz & 0x50 & 80 \\
\hline
\end{longtable}
\end{center}

Il semble que la valeur passée à la carte décroît lorsque la fréquence croît.

Maintenant, nous voyons que la valeur de 950MHz est codée en dur, au moins dans cet
utilitaire. Pouvons-nous le truquer?

Retournons à ce morceau de code:

\begin{lstlisting}[style=customasmARM]
.text:0000AC84      LDR     R2, [R11,#third_argument]
.text:0000AC88      MOV     R3, #950
.text:0000AC8C      CMP     R2, R3
.text:0000AC90      BGT     errors_with_arguments ; j'ai modifié ici en 00 00 00 00
\end{lstlisting}

Nous devons désactiver l'instruction de branchement \INS{BGT} en 0xAC90. Et ceci est
du ARM en mode ARM, car, comme on le voit, toutes les adresses augmentent par 4,
i.e, chaque instruction a une taille de 4 octets.
L'instruction \TT{NOP} (no operation) en mode ARM est juste quatre octets à zéro:
\TT{00 00 00 00}.
Donc en écrivant quatre octets à zéro à l'adresse 0xAC90 (ou à l'offset 0x2C90 dans
le fichier), nous pouvons désactiver le test.

Maintenant, il est possible de définir la fréquence jusqu'à 1050MHz. Et même plus,
mais, à cause du bug, si la valeur en entrée est plus grande que 1099, la valeur
\emph{telle quelle} en MHz sera passée à la carte, ce qui est incorrect.

Je ne suis  pas allé plus loin, mais si je devais, j'essayerai de diminuer la valeur
qui est passée à la fonction \TT{write\_power()}.

Maintenant, le morceau de code effrayant que j'ai passé en premier:

\lstinputlisting[style=customasmARM]{examples/bitcoin_miner/tmp1.lst}

La division via la multiplication est utilisée ici, et la constante est 0x51EB851F.
Je me suis écrit un petit calculateur pour programmeur\footnote{\ProgCalcURL}.
Et il est capable de calculer le modulo inverse.

\begin{lstlisting}
modinv32(0x51EB851F)
Warning, result is not integer: 3.125000
(unsigned) dec: 3 hex: 0x3 bin: 11
\end{lstlisting}

Cela signifie que l'instruction \INS{SMULL} en 0xACA0 divise le 3ème argument par 3.125.
En fait, tout ce que la fonction \TT{modinv32()} de mon calculateur fait est ceci:

\[
\frac{1}{\frac{input}{2^{32}}} = \frac{2^{32}}{input}
\]

Ensuite il y a es décalages additionnels et maintenant nous voyons que le 3ème argument
est simplement divisé par 50.
Et ensuite il à nouveau multiplié par 50.
Pourquoi?
Ceci est un simple test, pour savoir si la valeur entrée est divisible par 50.
Si la valeur de cette expression est non nulle, $x$ n'est pas divisible par 50:

\[
x-((\frac{x}{50}) \cdot 50)
\]

Ceci est en fait une manière simple de calculer le reste de la division.

Et alors, si le reste est non nul, un message d'erreur est affiché.
Donc cet utilitaire prend des fréquences comme 850, 900, 950, 1000, etc., mais pas 855 ou 911.

C'est ça! Si vous faites quelque chose comme ça, soyez avertis que vous pouvez endommager
votre carte, tout comme en cas d'overclocking d'autres éléments comme les \ac{CPU}s,
\ac{GPU}s, etc.
Si vous avez une carte Cointerra, faites ceci à votre propre risque!

