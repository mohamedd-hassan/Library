\subsection*{mini-FAQ}

% TBT
%\par Q: Is this book simpler/easier than others?
%\par A: No, it is at about the same level as other books of this subject.
% TBT
%\par Q: I'm too frightened to start reading this book, there are more than 1000 pages.
%\par A: All sorts of listings are the bulk of the book.

\par Q: Adakah pra-syarat dalam membaca buku ini?
\par A: Penguasaan dasar dari  C/C++ diperlukan.

\par Q: Bisakah saya membeli dalam cetak keras Russian/English / bentuk Buku?
\par A: Sayang sekali tidak, Sejauh ini tidak ada penerbit yang tertarik untuk menerbitkan versi Inggris / Rusia.
Sementara itu, kamu bisa bertanya pada percetakan favorit anda untuk mencetaknya.

\par Q: apakah tersedia dalam versi epub/mobi ?
\par A: Buku ini bergantung pada TeX/LaTeX-specific hacks, jadi dalam konversinya ke HTML (epub/mobi terdiri atas HTML)
tidak akan mudah.

\par Q: mengapa kita harus belajar bahasa Assembly hari ini?
\par A: kecuali kamu seorang pengembang \ac{OS} , kamu mungkin tidak perlu koding dalam assembly\textemdash{} compiler terbaru (2010) lebih baik melakukan optimisasi dari manusia \footnote{teks yang bagus untuk topik ini : \InSqBrackets{\AgnerFog}}.

ada juga, \ac{CPU}s Terbaru merupakan perangkat yang sangat komplex dan pengetahuan Assembly tidak sungguh membantu untuk mengerti dalamanya.

Seperti dibilang, terdapat 2 bidang dimana pengertian assembly bisa sangat membantu : 
Pertama dan sering, Keamanan/riset Malware. ini juga merupaka jalan yang bagus untuk mendapatkan pemahaman yang baik dari kode yang telah kamu kompil dengan kata lain ialah debugging.
oleh sebab itu buku ini ditujukan untuk siapa yang ingin mengerti tentang bahasa Assembly daripada 
untuk koding, Itulah sebabnya mengapa ada banyak contoh output kompiler yang terkandung di dalamnya.

\par Q: Saya menglik sebuah link dalam dokumen, bagaimana saya kembali?
\par A: Dalam Adobe Acrobat Reader pencet Alt+LeftArrow pada keyboard. Klik tombol ``<'' .

\par Q:Bolehkah saya mencetak buku / untuk bahan pengajaraan ?
\par A: Tentu saja! itu sebabnya buku saya dibawah lisensi Creative Commons license (CC BY-SA 4.0).

\par Q: Kenapa buku ini gratis ? Kamu telah melakukan kerja yang bagus. Apakah ini mencurigakan, seperti hal gratis kebanyakan.
\par A: Menurut pengalaman saya, Secara Tehnis penulis mengunakan literatur sebagai besar untuk periklan sendiri . Tidak mungkin untuk mendapatkan uang yang layak dari perkerjaan ini.

\par Q: Bagaimana saya mendapatkan pekerjaan dalam bidang rekayasa mundur?
\par A: Disana ada thread tentang lowongan pekerjaan yang selalu muncul dari waktu ke waktu di Reddit , Dikhususkan ke RE\FNURLREDDIT{}.
Coba untuk singgah.

Sebuah thread perekrutan agak terkait dapat ditemukan di \q{netsec} subreddit.

\par Q: Saya punya pertanyaan ...
\par A: Kirim ke saya melalui E-mail (\EMAILS).
