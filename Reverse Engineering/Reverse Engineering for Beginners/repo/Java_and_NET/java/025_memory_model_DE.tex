\subsection{\ac{JVM}-Speichermodell}
x86 und andere low-level Umgebungen nutzen den Stack um Funktionsargumente zu übergeben
und lokale Variablen zu speichern.

\ac{JVM} behandelt dies ein wenig anders.

\begin{itemize}
	\item Local variable array (\ac{LVA}).
	Dieses wird als Speicher für Funktionsargumente und lokale Variablen genutzt.
	
	Anweisungen wie \INS{iload\_0} laden Werte vom LVA.
	
	\INS{istore} legt Werte in den Speicher des LVA. 
	Als erstes werden die Funktionsargumente gespeichert, welche mit dem Index 0 oder 1 
	(Falls das nullte Argument der \emph{this} Pointer ist) beginnen.
	
	Danach werden die lokalen Variablen alloziert.
	
	Jeder Eintrag besitzt eine Größe von 32-bit.
	
	Dadurch benötigen bestimmte Datentypen wie \emph{long} und \emph{double} zwei Einträge.
	
	
	\item Operand stack (or just \q{stack}).
	Dieser wird für Berechnungen und die Übergabe von Argumenten genutzt, während andere Funktionen aufgerufen werden.
	
	Anders als in low-level Umgebungen wie x86, ist es nicht möglich Zugriffe auf den Stack zu machen ohne dabei Anweisungen zu nutzen, welche explizit Werte vom Stack nehmen oder auf den Stack legen.
	
	\item Heap. Dieser wird als Speicher für Objekte und Arrays genutzt.
	
\end{itemize}

Diese 3 genannten Bereiche sind voneinander isoliert.