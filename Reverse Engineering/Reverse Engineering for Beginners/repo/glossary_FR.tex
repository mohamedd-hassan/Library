\newglossaryentry{tail call}
{
  name={tail call},
  description={C'est lorsque le compilateur (ou l'interpréteur) transforme la récursion (ce qui est possible: \emph{tail recursion})
  en une itération pour l'efficacité: \href{http://en.wikipedia.org/wiki/Tail_call}{wikipedia}}
}

\newglossaryentry{endianness}
{
  name=endianness,
  description={Ordre des octets: \myref{sec:endianness}}
}

\newglossaryentry{caller}
{
  name=caller,
  description={Une fonction en appelant une autre}
}

\newglossaryentry{callee}
{
  name=callee,
  description={Une fonction appelée par une autre}
}

\newglossaryentry{debuggee}
{
  name=debuggee,
  description={Un programme en train d'être débogué}
}

\newglossaryentry{leaf function}
{
  name=leaf function,
  description={Une fonction qui n'appelle pas d'autre fonction}
}

\newglossaryentry{link register}
{
  name=link register,
  description={(RISC) Un registre où l'adresse de retour est en général stockée. Ceci permet
  d'appeler une fonction leaf sans utiliser la pile, i.e, plus rapidemment}
}

\newglossaryentry{anti-pattern}
{
  name=anti-pattern,
  description={En général considéré comme une mauvaise pratique}
}

\newglossaryentry{stack pointer}
{
  name=pointeur de pile,
  description={Un registre qui pointe dans la pile}
}

\newglossaryentry{decrement}
{
  name=décrémenter,
  description={Décrémenter de 1}
}

\newglossaryentry{increment}
{
  name=incrémenter,
  description={Incrémenter de 1}
}

\newglossaryentry{loop unwinding}
{
  name=loop unwinding,
  description={C'est lorsqu'un compilateur, au lieu de générer du code pour une boucle de
  $n$ itérations, génère juste $n$ copies du corps de la boucle, afin de supprimer
  les instructions pour la gestion de la boucle}
}

\newglossaryentry{register allocator}
{
  name=register allocator,
  description={La partie du compilateur qui assigne des registes du CPU aux variables locales}
}

\newglossaryentry{quotient}
{
  name=quotient,
  description={Résultat de la division}
}

\newglossaryentry{product}
{
  name=produit,
  description={Résultat d'une multiplication}
}

\newglossaryentry{NOP}
{
  name=NOP,
  description={\q{no operation}, instruction ne faisant rien}
}

\newglossaryentry{POKE}
{
  name=POKE,
  description={instruction du langage BASIC pour écrire un octet à une adresse spécifique}
}

\newglossaryentry{keygenme}
{
  name=keygenme,
  description={Un programme qui imite la protection des logiciels pour lesquels on a besoin d'un générateur de clef/licence}
} % TODO clarify: A software which generate key/license value to bypass sotfware protection?

\newglossaryentry{dongle}
{
  name=dongle,
  description={Un dongle est un petit périphérique se connectant sur un port d'imprimante LPT (par le passé) ou USB.
  Sa fonction est similaire au tokens de sécurité, il a de la mémoire et, parfois, un algorithme secret de (crypto-)hachage.}
}

\newglossaryentry{thunk function}
{
  name=thunk function,
  description={Minuscule fonction qui a un seul rôle: appeler une autre fonction}
}

\newglossaryentry{user mode}
{
  name=user mode,
  description={Un mode CPU restreint dans lequel le code de toutes les applications est exécuté. cf. \gls{kernel mode}}
}

\newglossaryentry{kernel mode}
{
  name=kernel mode,
  description={Un mode CPU sans restriction dans lequel le noyau de l'OS et les drivers sont exécutés. cf. \gls{user mode}}
}

\newglossaryentry{Windows NT}
{
  name=Windows NT,
  description={Windows NT, 2000, XP, Vista, 7, 8, 10}
}

\newglossaryentry{atomic operation}
{
  name=atomic operation,
  description={
  \q{$\alpha{}\tau{}o\mu{}o\varsigma{}$}
  %\q{atomic}
  signifie \q{indivisible} en grec, donc il est garantie qu'une opération atomique ne sera pas interrompue par d'autres threads}
}

% to be proofreaded (begin)
\newglossaryentry{NaN}
{
  name=NaN,
  description={pas un nombre: un cas particulier pour les nombres à virgule flottante, indiquant généralement une erreur}
}

\newglossaryentry{NEON}
{
  name=NEON,
  description={\ac{AKA} \q{Advanced SIMD} --- \ac{SIMD} de ARM}
}

\newglossaryentry{reverse engineering}
{
  name=reverse engineering,
  description={action d'examiner et de comprendre comment quelque chose fonctionne, parfois dans le but de le reproduire}
}

\newglossaryentry{compiler intrinsic}
{
  name=compiler intrinsic,
  description={Une foncion spécifique d'un compilateur, qui n'est pas une fonction usuelle de bibliothèque.
    Le compilateur génère du code machine spécifique au lieu d'un appel à celui-ci.
    Souvent il s'agit d'une pseudo-fonction pour une instruction \ac{CPU} spécifique. Lire plus:
    (\myref{sec:compiler_intrinsic})}
}

\newglossaryentry{heap}
{
  name={tas},
  description={Généralement c'est un gros bout de mémoire fournit par l'\ac{OS} et utilisé par
  les applications pour le diviser comme elles le souhaitent. malloc()/free() fonctionnent en utilisant le tas}
}

\newglossaryentry{name mangling}
{
  name=name mangling,
  description={utilisé au moins en \Cpp, où le compilateur doit encoder le nom de la classe, la méthode et le type des arguments dans une chaîne,
  qui devient le nom interne de la fonction. Vous pouvez en lire plus à ce propos ici: \myref{namemangling}}
}

\newglossaryentry{xoring}
{
  name=xoring,
  description={souvent utilisé en anglais, qui signifie appliquer l'opération \ac{XOR}}
}

\newglossaryentry{security cookie}
{
  name=security cookie,
  description={Une valeur aléatoire, différente à chaque exécution. Vous pouvez en lire plus à ce propos ici: \myref{subsec:BO_protection}}
}

\newglossaryentry{tracer}
{
  name=tracer,
  description={Mon propre outil de debugging. Vous pouvez en lire plus à son propos ici: \myref{tracer}}
}

\newglossaryentry{GiB}
{
  name=GiB,
  description={Gibioctet: $2^{30}$ or 1024 mebioctets ou 1073741824 octets}
}

\newglossaryentry{CP/M}
{
  name=CP/M,
  description={Control Program for Microcomputers:
  un \ac{OS} de disque très basique utilisé avant MS-DOS}
}

\newglossaryentry{stack frame}
{
  name=stack frame,
  description={Une partie de la pile qui contient des informations spécifiques à la fonction courante:
  variables locales, arguments de la fonction, \ac{RA}, etc.}
}

\newglossaryentry{jump offset}
{
  name=jump offset,
  description={une partie de l'opcode de l'instruction JMP ou Jcc, qui doit être ajoutée à l'adresse de l'instruction suivante,
  et c'est ainsi que le nouveau \ac{PC} est calculé. Peut être négatif}
}

\newglossaryentry{integral type}
{
  name={type de donnée intégral},
  description={nombre usuel, mais pas un réel. peut être utilisé pour passer des variables de type booléen et des énumérations}
}

\newglossaryentry{real number}
{
  name={nombre réel},
  description={nombre qui peut contenir un point. comme \Tfloat et \Tdouble en \CCpp}
}

\newglossaryentry{PDB}
{
  name=PDB,
  description={(Win32) Fichier contenant des informations de débogage, en général seulement les noms des fonctions,
  mais aussi parfois les arguments des fonctions et le nom des variables locales}
}

\newglossaryentry{NTAPI}
{
  name=NTAPI,
  description={\ac{API} disponible seulement dans la série de Windows NT. Très peu documentée par Microsoft}
}

\newglossaryentry{stdout}
{
  name=stdout,
  description={standard output, sortie standard}
}

\newglossaryentry{word}
{
  name=word,
  description={Dans les ordinateurs plus vieux que les PCs, la taille de la mémoire était
  souvent mesurée en mots plutôt qu'en octet}
}

\newglossaryentry{arithmetic mean}
{
  name={moyenne arithmétique},
  description={la somme de toutes les valeurs, divisé par leur nombre}
}
\newglossaryentry{padding}
{
  name=padding,
  description={\emph{Padding} en anglais signifie rembourrer un oreiller, un matelas, etc. avec quelque chose afin de lui donner la forme désirée.
  En informatique, padding signifie ajouter des octets à un bloc, afin qu'il ait une certaine taille, comme $2^n$ octets.}
}

\newglossaryentry{basic block}
{
  name=basic block,
  description={
	un groupe d'instructions qui n'a pas d'instruction de saut/branchement,
	et donc n'as pas de saut de l'intérieur du bloc vers l'extérieur.
	Dans \IDA il ressemble à une liste d'instructions sans ligne vide}
}
