\mysection{Endianness}
\label{sec:endianness}

Endianness es una forma de representar valores en la memoria.

\subsection{Big-endian}

El valor \TT{0x12345678} es representado en la memoria como:

\begin{center}
\begin{tabular}{ | l | l | }
\hline
\HeaderColor direcci\'on en memoria & \HeaderColor valor del byte \\
\hline
+0 & 0x12 \\
\hline
+1 & 0x34 \\
\hline
+2 & 0x56 \\
\hline
+3 & 0x78 \\
\hline
\end{tabular}
\end{center}

Entre los CPU big-endian se encuentran Motorola 68k, IBM POWER.

\subsection{Little-endian}

El valor \TT{0x12345678} es representado en la memoria como:

\begin{center}
\begin{tabular}{ | l | l | }
\hline
\HeaderColor Direcci\'on en memoria & \HeaderColor valor del byte \\
\hline
+0 & 0x78 \\
\hline
+1 & 0x56 \\
\hline
+2 & 0x34 \\
\hline
+3 & 0x12 \\
\hline
\end{tabular}
\end{center}

Entre los CPU little-endian tenemos el Intel x86.
% TBT
%One important example of little-endian using in this book is:
%\myref{LargeInteger}.

\subsection{\Example}

Tomemos el MIPS big-endian para Linux instalado y listo en QEMU
\footnote{Disponible para descargar aqu\'i: \url{https://people.debian.org/~aurel32/qemu/mips/}}.

Y compilemos este ejemplo sencillo:

\begin{lstlisting}[style=customc]
#include <stdio.h>

int main()
{
	int v;

	v=123;

	printf ("%02X %02X %02X %02X\n", 
		*(char*)&v,
		*(((char*)&v)+1),
		*(((char*)&v)+2),
		*(((char*)&v)+3));
};
\end{lstlisting}

Despu\'es de correrlo obtenemos:

\begin{lstlisting}
root@debian-mips:~# ./a.out 
00 00 00 7B
\end{lstlisting}

Eso es todo.
0x7B es 123 en decimal.
En las arquitecturas little-endian, 7B es el primer byte (puedes checarlo en x86 o x86-64),
pero aqu\'i es el \'ultimo, porque el byte m\'as alto va primero.

Por eso hay distribuciones separadas de Linux para MIPS
(\q{mips} (big-endian) \ESph{} \q{mipsel} (little-endian)).
Es imposible que un binario compilado para un endianness trabaje en un SO con diferente endianness.

En este libro hay otro ejemplo de big-endianness en MIPS: \myref{MIPS_structure_big_endian}.

\subsection{Bi-endian}

CPUs que puede cambiar entre endianness son ARM, PowerPC, SPARC, MIPS, \ac{IA64}, etc.

\subsection{Convirtiendo datos}

\myindex{x86!\Instructions!BSWAP}
La instrucci\'on \TT{BSWAP} puede ser utilizada para conversiones.

\myindex{TCP/IP}
Los paquetes de datos en redes TCP/IP utilizan convenciones big-endian, por eso un programa corriendo en una arquitectura
little-endian tiene que convertir los valores.

Las funciones \TT{htonl()} \ESph{} \TT{htons()} son usadas generalmente.

En TCP/IP, big-endian tambi\'en es llamado \q{orden de bytes de red},
mientras que el orden de bytes en la computadora se conoce como \q{orden de bytes de host}.
\q{orden de bytes de host} es little-endian en Intel x86 y otras arquitecturas little-endian,
pero es big-endian en IBM POWER, asi que \TT{htonl()} y \TT{htons()} no cambian el orden de los bytes
en \'esta \'ultima.

