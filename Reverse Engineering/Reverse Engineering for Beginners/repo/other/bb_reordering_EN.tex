\mysection{Basic blocks reordering}

% TODO __builtin_expect in GCC?

\subsection{Profile-guided optimization}
\label{PGO}

\myindex{\oracle}
\myindex{Intel C++}

This optimization method can move some \gls{basic block}s to another section of the executable binary file.

Obviously, there are parts of a function which are executed more frequently (e.g., loop bodies)
and less often (e.g., error reporting code, exception handlers).

The compiler adds instrumentation code into the executable, then the developer runs it with
a lot of tests to collect statistics.

Then the compiler, with the help of the statistics gathered,
prepares final the executable file with all infrequently executed code moved into another section.

As a result, all frequently executed function code is compacted, and that is very important
for execution speed and cache usage.

An example from \oracle code, which was compiled with Intel C++:

\lstinputlisting[caption=orageneric11.dll (win32),style=customasmx86]{other/orageneric.lst}

The distance of addresses between these two code fragments is almost 9 MB.

All infrequently executed code was placed at the end of the code section of the DLL file,
among all function parts.

This part of the function was marked by the Intel C++ compiler with the \TT{VInfreq} prefix.

Here we see that a part of the function that writes to a log file (presumably in case of error or warning
or something like that) which was probably not executed very often when Oracle's developers gathered 
statistics (if it was executed at all).

The writing to log basic block eventually returns the control flow to the \q{hot} part of the function.

Another \q{infrequent} part is the \gls{basic block} returning error code 27050.

In Linux ELF files, all infrequently executed code is moved by Intel C++ into the separate 
\TT{text.unlikely} section, leaving all \q{hot} code in the \TT{text.hot} section.

From a reverse engineer's perspective, this information may help to split the function
into its core and error handling parts.
