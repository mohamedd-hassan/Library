\section{Nutzen von IMUL anstatt MUL}
\label{IMUL_over_MUL}

\myindex{x86!\Instructions!MUL}
\myindex{x86!\Instructions!IMUL}
Beispiele wie \lstref{unsigned_multiply_C} in denen zwei vorzeichenlose Werte miteinander multipliziert
werden, werden zu \lstref{unsigned_multiply_lst} kompiliert, so dass \IMUL statt \MUL genutzt wird.

Dies ist eine wichtige Eigenschaft der \MUL- und \IMUL-Anweisung.
Zunächst produzieren beide einen 64-Bit-Wert wenn zwei 32-Bit-Werte miteinander multipliziert werden,
oder einen 128-Bit-Wert wenn zwei 64-Bit-Werte miteinander multipliziert werden (größtes mögliches \gls{product}
in 32-Bit-Umgebungen ist \\
\GTT{0xffffffff*0xffffffff=0xfffffffe00000001}).
Der \CCpp-Standard kennt keine Möglichkeit auf die höherwertige Hälfte eines Ergebnisses zuzugreifen
und ein \gls{product} hat immer die gleiche Größe wie die Faktoren. % TODO \gls{}?
Beide Anweisungen \MUL und \IMUL arbeiten auf die gleiche Weise wenn die höherwertige Hälfte ignoriert wird.
Das heißt die niederwertigere Hälfte ist die gleiche.
Dies ist eine wichtige Eigenschaft der Repräsentation von vorzeichenbehafteten Zahlen im \q{Zweierkomplements}.

Somit kann der \CCpp-Compiler jede dieser Anweisungen nutzen.

Die \IMUL-Anweisung ist jedoch vielseitiger als \MUL weil sie jedes Register als Quelle akzeptiert,
während \MUL einen der Faktoren in den Registern \AX, \EAX oder \RAX erwartet.
Des weiteren sichert \MUL das Ergebnis in dem \GTT{EDX:EAX} Paar in einer 32-Bit-Umgebung oder
in \GTT{RDX:RAX} in einer 64-Bit-Umgebung. Die Anweisung berechnet also immer das gesamte Ergebnis.
Im Gegensatz dazu ist es möglich beim Nutzen von \IMUL statt eines Paares von Zielregistern ein
einzelnes Register anzugeben. Die \ac{CPU} wird dann lediglich den niederwertigen Teil berechnen,
was zu einer höheren Geschwindigkeit führt [siehe Torborn Granlund, \emph{Instruction latencies and throughput for AMD and Intel x86 processors}\footnote{\url{http://yurichev.com/mirrors/x86-timing.pdf}]}).

Aus diesen Gründen ist es möglich, dass ein \CCpp-Compilers öfter \IMUL-Anweisungen als \MUL nutzt.

\myindex{Compiler intrinsic}
Trotzdem ist es möglich mit intrinsischen Funktionen (Intrinsics) des Compilers vorzeichenlose Multiplikationen durchzuführen
und das \emph{volle} Ergebnis zu erhalten.
Dies wird manchmal \emph{erweiterte Multiplikation} genannt.
MSVC hat Intrinsics zu diesem Zweck die \emph{\_\_emul}\footnote{\url{https://msdn.microsoft.com/en-us/library/d2s81xt0(v=vs.80).aspx}}
und \emph{\_umul128}\footnote{\url{https://msdn.microsoft.com/library/3dayytw9%28v=vs.100%29.aspx}} genannt werden.
GCC stellt einen \emph{\_\_int128}-Datentyp zur Verfügung und 64-Bit-Faktoren werden zuerst auf 128-Bit erweitert,
Anschließend wird das \gls{product} in einem anderen \emph{\_\_int128} gesichert.
Das Ergebnis ist um 64-Bit nach rechts geshiftet um die höherwertigen Hälfte des Ergebnisses
zu erhalten\footnote{Example: \url{http://stackoverflow.com/a/13187798}}.

\subsection{MulDiv()-Funktion in Windows}
\myindex{Windows!Win32!MulDiv()}

Windows hat eine MulDiv()-Funktion
\footnote{\url{https://msdn.microsoft.com/en-us/library/windows/desktop/aa383718(v=vs.85).aspx}},
welche die Multiplikation und Division vereint und zwei 32-Bit-Integer in einen temporären 64-Bit-Wert
speichert. Anschließend findet eine Division durch eine dritte 32-Bit-Integerzahl statt.
Dies ist einfacher als zwei Compiler-Intrinsics zu nutzen, weswegen die Microsoft-Entwickler diese
spezielle Funktion dafür einführten.
Gemessen an der Häufigkeit der Nutzung ist dies eine populäre Funktion.
