\mysection{Uma Função Vazia}
\label{empty_func}

A função mais simples possível é uma vazia (sem corpo):

\lstinputlisting[label=lst:empty_func,caption=\CCpp Code,style=customc]{patterns/00_empty/1.c}

Vamos compilá-la!

\subsection{x86}

O que ambos compiladores GCC e MSVC produzem na plataforma x86:

\lstinputlisting[caption=\Optimizing GCC/MSVC (\assemblyOutput),style=customasmx86]{patterns/00_empty/1.s}

\myindex{x86!\Instructions!RET}
Há apenas uma instrução: \RET, que retorna a instrução para o \gls{caller}.

\subsection{Funções vazias, na prática}

Desconsiderando o fato que elas parecem sem utilidade, são bastante frequentes em código de baixo nível.
Primeiramente, são bastante populares em funções do modo ''debug'', como esta:

\lstinputlisting[caption=\CCpp code,style=customc]{patterns/00_empty/dbg_print_EN.c}

Numa compilação ''release'', a constante \TT{\_DEBUG} não é definida,
portando a função \TT{dbg\_print()}, ainda será chamada durante a execução,
mas estará vazia.

De forma similar, um método popular de proteção de software é criar uma compilação completa e outra ''demo''. Uma versão ''demo'' pode não conter importantes funções, como neste exemplo:

\lstinputlisting[caption=\CCpp code,style=customc]{patterns/00_empty/demo_EN.c}

A função \TT{save\_file()} poderia ser chamada quando o usuário clica \TT{File->Save} no menu do programa. 

A versão ''demo'' poderia ser distribuida com este item desabilitado (salvar), e mesmo que um ''cracker'' abilitasse-o, apenas uma função vazia e sem nenhum código útil seria chamada.

O software IDA destaca este tipo de funções com nomes do tipo \TT{nullsub\_00}, \TT{nullsub\_01}, etc.

