\mysection{Pusta funkcja}
\label{empty_func}

Najprostszą istniejącą funkcją jest funkcja, która nic nie robi:

\lstinputlisting[label=lst:empty_func,caption=Kod w \CCpp,style=customc]{patterns/00_empty/1.c}

Skompilujmy ją!

\subsection{x86}

Dla x86 i MSVC, i GCC generują ten sam kod:

\lstinputlisting[caption=\Optimizing GCC/MSVC (\assemblyOutput),style=customasmx86]{patterns/00_empty/1.s}

\myindex{x86!\Instructions!RET}
Mamy tu tylko jedną instrukcję: \RET, która jest instrukcją powrotu do \glslink{caller}{funkcji wywołującej}.

\subsection{ARM}

\lstinputlisting[caption=\OptimizingKeilVI (\ARMMode) (\assemblyOutput),style=customasmARM]{patterns/00_empty/1_Keil_ARM_O3.s}

Adres powrotu (\ac{RA}) w ARM zapisywany jest nie na stosie, a w rejestrze \ac{LR}.
Instrukcja \INS{BX LR}, wykonując skok pod ten adres, zwraca sterowanie do funkcji wywołującej.

\subsection{MIPS}

Są dwie konwencje nazewnictwa rejestrów w architekturze MIPS:
używająca numeru rejestru (od \$0 do \$31) lub nazwy umownej (\$V0, \$A0, itd.).

Wyjście asemblera w GCC pokazuje numery rejestrów

\lstinputlisting[caption=\Optimizing GCC 4.4.5 (\assemblyOutput),style=customasmMIPS]{patterns/00_empty/MIPS.s}

\dots a \IDA --- nazwy umowne:

\lstinputlisting[caption=\Optimizing GCC 4.4.5 (IDA),style=customasmMIPS]{patterns/00_empty/MIPS_IDA.lst}

\myindex{MIPS!\Instructions!J}

Pierwsza instrukcja jest instrukcją skoku (J lub JR),
która zwraca sterowanie do \glslink{caller}{funkcji wywołującej}, skacząc pod adres w rejestrze \$31 (\$RA).

Jest to rejestr odpowiadający \ac{LR} w ARM.

Druga instrukcja to \ac{NOP}, która nic nie robi.
Na razie możemy ją zignorować.

\subsubsection{Jeszcze małe co nieco o konwencji nazewnictwa w MIPS}
Nazwy rejestrów i instrukcji w MIPS tradycyjnie są zapisywane małymi literami,
lecz my będziemy je zapisywać dużymi, dlatego że nazwy instrukcji i rejestrów innych
\ac{ISA} w tej książce są zapisywane dużymi.

\subsection{Puste funkcje w praktyce}

Mimo, że puste funkcje są bezużyteczne, są one dość często spotykane w niskopoziomowym kodzie.

Po pierwsze, często spotykamy funkcje zapisujące szczegółowe informacje do logów, na przykład:

\lstinputlisting[caption=Kod w \CCpp,style=customc]{patterns/00_empty/dbg_print_PL.c}

Przy kompilacji wersji programu przeznaczonej do wdrożenia, \TT{\_DEBUG} jest niezdefiniowane,
więc funkcja \TT{dbg\_print()}, mimo, że jest wywoływana, będzie pusta.

Po drugie, popularnym sposobem na ochronę oprogramowania jest kompilacja kilku wersji: pierwsza dla legalnych konsumentów, druga - demonstracyjna.
Demonstracyjna wersja może nie zawierać jakiejś ważnej funkcjonalności, na przykład:

\lstinputlisting[caption=Kod w \CCpp,style=customc]{patterns/00_empty/demo_PL.c}

Funkcja \TT{save\_file()} może być wywołana, kiedy użytkownik klika w menu \TT{File->Save}.
Wersja demo może zawierać wyłączony przycisk menu, ale nawet jeśli cracker go włączy,
to zostanie wywoływana jedynie pusta funkcja, w której nie ma użytecznego kodu.

IDA oznacza takie funkcje jako \TT{nullsub\_00}, \TT{nullsub\_01}, itd.

