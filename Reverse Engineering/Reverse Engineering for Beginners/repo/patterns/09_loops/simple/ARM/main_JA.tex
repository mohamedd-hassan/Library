\subsubsection{ARM}

\myparagraph{\NonOptimizingKeilVI (\ARMMode)}

\lstinputlisting[label=Keil_number_sign,style=customasmARM]{patterns/09_loops/simple/ARM/Keil_ARM_O0.asm}

ループカウンタ $i$ は\Reg{4}レジスタに格納されます。
\INS{MOV R4, \#2}命令は $i$ をちょうど初期化します。 
\INS{MOV R0, R4}、および\INS{BL printing\_function}命令は、
\ttf 関数の引数を準備する最初の命令と2番目の関数を呼び出すループの本体を構成します。 
\myindex{ARM!\Instructions!ADD}
\INS{ADD R4, R4, \#1}命令は、各繰り返しで $i$ 変数に1を加算するだけです。 
\INS{CMP R4, \#0xA}は $i$ と\TT{0xA}(10)を比較します。 
次の命令\INS{BLT}(\emph{Branch Less Than})は、 $i$ が10未満の場合にジャンプします。
それ以外の場合は、\Reg{0}に0が書き込まれます(関数が0を返すため)。
そして関数の実行が終了します。

\myparagraph{\OptimizingKeilVI (\ThumbMode)}

\lstinputlisting[style=customasmARM]{patterns/09_loops/simple/ARM/Keil_thumb_O3.asm}

実質的に同じです。

\myparagraph{\OptimizingXcodeIV (\ThumbTwoMode)}
\label{ARM_unrolled_loops}

\lstinputlisting[style=customasmARM]{patterns/09_loops/simple/ARM/xcode_thumb_O3.asm}

実際、これは私の \ttf 関数にありました:

\begin{lstlisting}[style=customc]
void printing_function(int i)
{
    printf ("%d\n", i);
};
\end{lstlisting}

\myindex{Unrolled loop}
\myindex{Inline code}
だから、LLVMはループを\emph{展開}しただけでなく、
私の非常に単純な関数 \ttf を\emph{インライン}化し、
呼び出すのではなく本体を8回挿入しました。

これは、関数が(私の例のように)とても簡単で、(ここのように)あまり呼び出されないときに可能です。

\myparagraph{ARM64: \Optimizing GCC 4.9.1}

\lstinputlisting[caption=\Optimizing GCC 4.9.1,style=customasmARM]{patterns/09_loops/simple/ARM/ARM64_GCC491_O3_JA.s}

\myparagraph{ARM64: \NonOptimizing GCC 4.9.1}

\lstinputlisting[caption=\NonOptimizing GCC 4.9.1 -fno-inline,style=customasmARM]{patterns/09_loops/simple/ARM/ARM64_GCC491_O0_JA.s}
