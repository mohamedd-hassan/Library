\subsubsection{x86}

\myindex{x86!\Instructions!LOOP}
Es gibt einen speziellen \LOOP Befehl im x86 Befehlssatz, der den Wert des
Registers \ECX prüft und falls dieser ungleich 0 ist, dekrementiert und danach
die den control flow wieder an das Label des \LOOP Operanden übergibt.
Vermutlich ist dieser Befehl nicht allzu geläufig und es gibt keine modernen
Compiler, welche ihn automatisch erzeugen. Wenn man also diesen Befehl irgendwo
im Code entdeckt, dann ist es äußerst wahrscheinlich, dass es sich um ein
handgeschriebenes Stück Assemblercode handelt.

\par In \CCpp werden Schleifen normalerweise mittels \TT{for()}-, \TT{while()}- oder
\TT{do/while()}-Ausdrücken erzeugt.

Starten wir mit \TT{for()}.

\myindex{\CLanguageElements!for}

Dieser Ausdruck definiert eine Schleifeninitialisierung (setzt den Zähler auf
einen Startwert), definiert eine Schleifenbedingung (ist der Zähler größer als
ein Grenzwert?), legt fest, was in jedem Durchlauf
(\glslink{increment}{Inkrement}/\glslink{decrement}{Dekrement}) geschieht und umschließt einen
Schleifenkörper.

\lstinputlisting[style=customc]{patterns/09_loops/simple/loops_1_DE.c}

Der erzeugte Code besteht ebenfalls aus vier Teilen.

Beginnen wir mit einem einfachen Beispiel:

\lstinputlisting[label=loops_src,style=customc]{patterns/09_loops/simple/loops_2.c}

Ergebnis (MSVC 2010):

\lstinputlisting[caption=MSVC 2010,style=customasmx86]{patterns/09_loops/simple/1_MSVC_DE.asm}

Hier gibt es nichts Besonderes zu sehen.

GCC 4.4.1 erzeugt einen fast identischen Code mit nur einen kleinen Unterschied:

\lstinputlisting[caption=GCC 4.4.1,style=customasmx86]{patterns/09_loops/simple/1_GCC_DE.asm}

Schauen wir uns nun an, was wir erhalten, wenn wir die Optimierung aktivieren
(\TT{\Ox}):

\lstinputlisting[caption=\Optimizing MSVC,style=customasmx86]{patterns/09_loops/simple/1_MSVC_Ox.asm}

Was hier passiert ist, dass der Speicherplatz für die $i$ Variable nicht mehr
auf dem lokalen Stack bereitgestellt wird, sondern das extra ein Register, \ESI,
hierfür verwendet wird. Dies ist bei derartig kleinen Funktionen möglich, wenn
nicht zu viele lokalen Variablen existieren.

Wichtig ist, dass die \ttf Funktion den Wert im Register \ESI nicht verändern
darf. Unser Compiler ist sich dieser Sache hier sicher. 
Und falls der Compiler entscheidet, das \ESI auch innerhalb der Funktion \ttf zu
verwenden, würde der Wert des Registers im Funktionsprolog gesichert und im
Funktionsepilog wiederhergestellt werden; fast genauso wie im folgenden Listing.
Man beachte das \TT{PUSH ESI/POP ESI} bei Funktionsbeginn und -ende. 
 
Probieren wir aus, was GCC 4.4.1 mit maximaler Optimierung (\Othree option)
liefert:

\lstinputlisting[caption=\Optimizing GCC 4.4.1,style=customasmx86]{patterns/09_loops/simple/1_GCC_O3.asm}

\myindex{Loop unwinding}

Aha, GCC hat unsere Schleife unrolled (d.h. ausgerollt).

\Gls{loop unwinding} hat Vorteile in Fällen, in denen es nicht viele
Schleifendurchläufe gibt und Ausführungszeit durch das Weglassen der
Befehle für die Kontrollstrukturen der Schleife gewonnen werden kann. 
Andererseits ist der erzeugte Code natürlich deutlich länger.

Große Schleifen zu \textit{unrollen} ist heutzutage nicht empfehlenswert, denn
größere Funktionen erfordern einen größeren Cache-Fußabdruck.%
%
\footnote{Ein hervorragender Artikel zum Thema: \DrepperMemory.
Für weitere Empfehlungen von Intel zum Unrolling siehe hier: 
\InSqBrackets{\IntelOptimization 3.4.1.7}.}.

Gut, nun wollen wir den Höchstwert der Variable $i$ auf 100 setzen und
kompilieren erneut. GCC liefert:

\lstinputlisting[caption=GCC,style=customasmx86]{patterns/09_loops/simple/2_GCC_DE.asm}

Das Ergebnis ist sehr ähnlich dem, das MSVC 2010 mit Optimierung (\Ox) erzeugt,
mit der Ausnahme, dass das \EBX Register für die Variable $i$ verwendet wird.

GCC ist sicher, dass das Register innerhalb der \ttf Funktion nicht verändert
wird und sollte dies doch der Fall sein, dass es im Funktionsprolog gesichert
und im Funktionsepilog wiederhergestellt werden wird, genau wie hier in der
\main Funktion.

\clearpage
\subsubsection{x86: \olly}
\myindex{\olly}

Wir kompilieren unser Beispiel in MSVC 2010 mit den Optionen \Ox und \Obzero und
laden es in \olly.

Es scheint, dass \olly in der Lage ist, einfache Schleifen zu erkennen und in
eckigen Klammern darzustellen, um die Übersichtlichkeit zu erhöhen:

\begin{figure}[H]
\centering
\myincludegraphics{patterns/09_loops/simple/olly1.png}
\caption{\olly: \main Einstieg}
\label{fig:loops_olly_1}
\end{figure}

Verfolgen mit (F8~--- \stepover) zeigt \ESI 
\glslink{increment}{incrementing}.
Hier ist zum Beispiel, $ESI=i=6$:

\begin{figure}[H]
\centering
\myincludegraphics{patterns/09_loops/simple/olly2.png}
\caption{\olly: Schleifenkörper wird gerade ausgeführt für $i=6$}
\label{fig:loops_olly_2}
\end{figure}

9 ist der letzte Wert in der Schleife
Deshalb triggert \JL nach der \glslink{increment}{inkrement}-Anweisung nicht und die Funktion
wird beendet:

\begin{figure}[H]
\centering
\myincludegraphics{patterns/09_loops/simple/olly3.png}
\caption{\olly: $ESI=10$, Ende der Schleife}
\label{fig:loops_olly_3}
\end{figure}

\subsubsection{x86: tracer}
\myindex{tracer}

Wie wir bemerken ist es nicht sonderlich komfortabel, Werte im Debugger manuell
nachzuverfolgen. Aus diesem Grund probieren wir \tracer aus.

Wir öffnen das kompilierte Beispiel in \IDA, finden die Adresse mit dem Befehl
\INS{PUSH ESI} (das einzige Argument an \ttf übergebend), welche hier
\TT{0x401026} ist und aktivieren den \tracer:

\begin{lstlisting}
tracer.exe -l:loops_2.exe bpx=loops_2.exe!0x00401026
\end{lstlisting}

\TT{BPX} setzt einen Breakpoint an der Adresse und der Tracer zeigt uns den
momentanen Status der Register an. In \TT{tracer.log} sehen wir das Folgende:

\lstinputlisting{patterns/09_loops/simple/tracer.log}

Wir sehen wie der Wert des \ESI Registers sich schrittweise von 2 zu 9
verändert. 

Mehr noch, der \tracer kann alle Registerwerte für alle Adressen innerhalb der
Funktion zusammensammeln. Dies wird hier mit \emph{trace} (dt. Nachverfolgung)
bezeichnet. Jeder Befehl wird verfolgt, alle interessanten Registerwerte werden
aufgezeichnet.

Danach die ein \IDA .idc-script erzeugt, das Kommentare hinzufügt. Wir haben
also herausgefunden, dass die Adresse der \main Funktion \TT{0x00401020} ist und
wir führen nun das Folgende aus:

\begin{lstlisting}
tracer.exe -l:loops_2.exe bpf=loops_2.exe!0x00401020,trace:cc
\end{lstlisting}

\TT{BPF} setzt einen Breakpoint auf eine Funktion.

Als Ergebnis erahlten wir die Skripte \TT{loops\_2.exe.idc} und
\TT{loops\_2.exe\_clear.idc}.

\clearpage
Wir laden \TT{loops\_2.exe.idc} in \IDA und erhalten:

\begin{figure}[H]
\centering
\myincludegraphics{patterns/09_loops/simple/IDA_tracer_cc.png}
\caption{\IDA mit geladenem .idc-script}
\label{fig:loops_IDA_tracer}
\end{figure}

Wir sehen, dass der Wert von \ESI zu Beginn der Schleife zwischen 2 und 9 und
nach dem Inkrement zwischen 3 und 0xA (10) liegt. Wir sehen auch, dass die
Funktion \main mit dem Rückgabewert 0 in \EAX terminiert.

\tracer erzeugt ebenfalls die Datei \TT{loops\_2.exe.txt}, welche Informationen
darüber enthält, welcher Befehl wie oft ausgeführt wurde, sowie zugehörige
Registerwerte:

\lstinputlisting[caption=loops\_2.exe.txt]{patterns/09_loops/simple/loops_2.exe.txt}
\myindex{\GrepUsage}
An dieser Stelle können wir grep verwenden.

