\subsubsection{MIPS}
\label{MIPS_structure_big_endian}

\lstinputlisting[caption=\Optimizing GCC 4.4.5 (IDA),numbers=left,style=customasmMIPS]{patterns/15_structs/4_packing/MIPS_O3_IDA_RU.lst}

Поля структуры приходят в регистрах \$A0..\$A3 и затем перетасовываются в регистры \$A1..\$A3 для \printf,
в то время как 4-е поле (из \$A3) передается через локальный стек используя \INS{SW}.

Но здесь есть две инструкции SRA (\q{Shift Word Right Arithmetic}), которые готовят поля типа \Tchar.

Почему?
По умолчанию, MIPS это big-endian архитектура \myref{sec:endianness}, и Debian Linux в котором мы работаем, также big-endian.

Так что когда один байт расположен в 32-битном элементе структуры, он занимает биты 31..24.

И когда переменную типа \Tchar нужно расширить до 32-битного значения, она должна быть сдвинута вправо
на 24 бита.

\Tchar это знаковый тип, так что здесь нужно использовать арифметический сдвиг вместо логического.

