\subsection{\StructurePackingSectionName}
\label{structure_packing}

Достаточно немаловажный момент, это упаковка полей в структурах.

Возьмем простой пример:

\lstinputlisting[style=customc]{patterns/15_structs/4_packing/packing.c}

Как видно, мы имеем два поля \Tchar (занимающий один байт) и еще два ~--- \Tint (по 4 байта).

% subsections:
\subsubsection{x86}

\myparagraph{\NonOptimizing MSVC}

Имеем в итоге (MSVC 2010):

\lstinputlisting[caption=MSVC 2010,style=customasmx86]{patterns/14_bitfields/2_set_reset/set_reset_msvc.asm}

\myindex{x86!\Instructions!OR}
Инструкция \OR здесь устанавливает в регистре один бит, игнорируя остальные биты-единицы.

\myindex{x86!\Instructions!AND}
А \AND сбрасывает некий бит. Можно также сказать, что \AND здесь копирует все биты, кроме одного. 
Действительно, во втором операнде \AND выставлены в единицу те биты, которые нужно сохранить, 
кроме одного, копировать который мы не хотим (и который 0 в битовой маске).
Так проще понять и запомнить.

\clearpage
\mysubparagraph{\olly}

Попробуем этот пример в \olly.
Сначала, посмотрим на двоичное представление используемых нами констант:

\TT{0x200} (0b0000000000000000000{\color{red}1}000000000) (т.е. 10-й бит (считая с первого)).

Инвертированное \TT{0x200} это \TT{0xFFFFFDFF}\\
(0b1111111111111111111{\color{red}0}111111111).

\TT{0x4000} (0b00000000000000{\color{red}1}00000000000000) (т.е. 15-й бит).

Входное значение это: \TT{0x12340678} \\
(0b10010001101000000011001111000).
Видим, как оно загрузилось:

\begin{figure}[H]
\centering
\myincludegraphics{patterns/14_bitfields/2_set_reset/olly1.png}
\caption{\olly: значение загружено в \ECX}
\label{fig:set_reset_olly1}
\end{figure}

\clearpage
\OR исполнилась:

\begin{figure}[H]
\centering
\myincludegraphics{patterns/14_bitfields/2_set_reset/olly2.png}
\caption{\olly: \OR сработал}
\label{fig:set_reset_olly2}
\end{figure}

15-й бит выставлен: \TT{0x1234{\color{red}4}678}\\ 
(0b10010001101000{\color{red}1}00011001111000).

\clearpage
Значение перезагружается снова (потому что использовался режим компилятора без оптимизации): 

\begin{figure}[H]
\centering
\myincludegraphics{patterns/14_bitfields/2_set_reset/olly3.png}
\caption{\olly: значение перезагрузилось в \EDX}
\label{fig:set_reset_olly3}
\end{figure}

\clearpage
\AND исполнилась:

\begin{figure}[H]
\centering
\myincludegraphics{patterns/14_bitfields/2_set_reset/olly4.png}
\caption{\olly: \AND сработал}
\label{fig:set_reset_olly4}
\end{figure}

10-й бит очищен (или, иным языком, оставлены все биты кроме 10-го) и итоговое значение это \\
\TT{0x12344{\color{red}4}78} (0b1001000110100010001{\color{red}0}001111000).

\myparagraph{\Optimizing MSVC}

Если скомпилировать в MSVC с оптимизацией (\Ox), то код еще короче:

\lstinputlisting[caption=\Optimizing MSVC,style=customasmx86]{patterns/14_bitfields/2_set_reset/set_reset_msvc_Ox.asm}

\myparagraph{\NonOptimizing GCC}

Попробуем GCC 4.4.1 без оптимизации:

\lstinputlisting[caption=\NonOptimizing GCC,style=customasmx86]{patterns/14_bitfields/2_set_reset/set_reset_gcc.asm}

Также избыточный код, хотя короче, чем у MSVC без оптимизации.

Попробуем теперь GCC с оптимизацией \Othree:

\myparagraph{\Optimizing GCC}

\lstinputlisting[caption=\Optimizing GCC,style=customasmx86]{patterns/14_bitfields/2_set_reset/set_reset_gcc_O3.asm}

Уже короче. Важно отметить, что через регистр \AH компилятор работает с частью регистра \EAX. 
Это его часть от 8-го до 15-го бита включительно.

\RegTableOne{RAX}{EAX}{AX}{AH}{AL}

\myindex{Intel!8086}
\myindex{Intel!80386}
N.B. В 16-битном процессоре 8086 аккумулятор имел название \AX 
и состоял из двух 8-битных половин~--- \AL (младшая часть) и \AH (старшая). 
В 80386 регистры были расширены до 32-бит, 
аккумулятор стал называться \EAX, но в целях совместимости, к его \emph{более старым} частям всё ещё можно 
обращаться как к \AX/\AH/\AL.

Из-за того, что все x86 процессоры~--- наследники 16-битного 8086, эти \emph{старые} 16-битные опкоды короче 
нежели более новые 32-битные. 
Поэтому инструкция \INS{or ah, 40h} занимает только 3 байта. 
Было бы логичнее сгенерировать здесь \INS{or eax, 04000h}, но это уже 5 байт, или даже 6 
(если регистр в первом операнде не \EAX).

\myparagraph{\Optimizing GCC и regparm}

Если мы скомпилируем этот же пример не только с включенной оптимизацией \Othree, 
но ещё и с опцией \TT{regparm=3}, о которой я писал немного выше, то получится ещё короче:

\lstinputlisting[caption=\Optimizing GCC,style=customasmx86]{patterns/14_bitfields/2_set_reset/set_reset_gcc_O3_regparm3.asm}

\myindex{Inline code}
Действительно~--- первый аргумент уже загружен в \EAX, и прямо здесь можно начинать с ним работать. 
Интересно, что и пролог функции (\INS{push ebp / mov ebp,esp}) и эпилог (\INS{pop ebp}) 
функции можно смело выкинуть за ненадобностью, 
но возможно GCC ещё не так хорош для подобных оптимизаций по размеру кода. 
Впрочем, в реальной жизни подобные короткие функции лучше всего автоматически делать в виде 
\emph{inline-функций} (\myref{inline_code}).


\subsubsection{ARM + \OptimizingXcodeIV (\ARMMode)}

\lstinputlisting[caption=\OptimizingXcodeIV (\ARMMode),label=ARM_leaf_example4,style=customasmARM]{patterns/14_bitfields/4_popcnt/ARM_Xcode_O3_RU.lst}

\myindex{ARM!\Instructions!TST}
\TST это то же что и \TEST в x86.

\myindex{ARM!Optional operators!LSL}
\myindex{ARM!Optional operators!LSR}
\myindex{ARM!Optional operators!ASR}
\myindex{ARM!Optional operators!ROR}
\myindex{ARM!Optional operators!RRX}
\myindex{ARM!\Instructions!MOV}
\myindex{ARM!\Instructions!TST}
\myindex{ARM!\Instructions!CMP}
\myindex{ARM!\Instructions!ADD}
\myindex{ARM!\Instructions!SUB}
\myindex{ARM!\Instructions!RSB}
Как уже было указано~(\myref{shifts_in_ARM_mode}),
в режиме ARM нет отдельной инструкции для сдвигов.

Однако, модификаторами 
LSL (\emph{Logical Shift Left}), 
LSR (\emph{Logical Shift Right}), 
ASR (\emph{Arithmetic Shift Right}), 
ROR (\emph{Rotate Right}) и
RRX (\emph{Rotate Right with Extend}) можно дополнять некоторые инструкции, такие как \MOV, \TST,
\CMP, \ADD, \SUB, \RSB\footnote{\DataProcessingInstructionsFootNote}.

Эти модификаторы указывают, как сдвигать второй операнд, и на сколько.

\myindex{ARM!\Instructions!TST}
\myindex{ARM!Optional operators!LSL}
Таким образом, инструкция  \TT{\q{TST R1, R2,LSL R3}} здесь работает как $R1 \land (R2 \ll R3)$.

\subsubsection{ARM + \OptimizingXcodeIV (\ThumbTwoMode)}

\myindex{ARM!\Instructions!LSL.W}
\myindex{ARM!\Instructions!LSL}
Почти такое же, только здесь применяется пара инструкций \INS{LSL.W}/\TST вместо одной \TST,
ведь в режиме Thumb нельзя добавлять модификатор \LSL прямо в \TST.

\begin{lstlisting}[label=ARM_leaf_example5,style=customasmARM]
                MOV             R1, R0
                MOVS            R0, #0
                MOV.W           R9, #1
                MOVS            R3, #0
loc_2F7A
                LSL.W           R2, R9, R3
                TST             R2, R1
                ADD.W           R3, R3, #1
                IT NE
                ADDNE           R0, #1
                CMP             R3, #32
                BNE             loc_2F7A
                BX              LR
\end{lstlisting}

\subsubsection{ARM64 + \Optimizing GCC 4.9}

Возьмем 64-битный пример, который уже был здесь использован: \myref{popcnt_x64_example}.

\lstinputlisting[caption=\Optimizing GCC (Linaro) 4.8,style=customasmARM]{patterns/14_bitfields/4_popcnt/ARM64_GCC_O3_RU.s}
Результат очень похож на тот, что GCC сгенерировал для x64: \myref{shifts64_GCC_O3}.

\myindex{ARM!\Instructions!CSEL}
Инструкция \CSEL это \q{Conditional SELect} (выбор при условии). 
Она просто выбирает одну из переменных, в зависимости от флагов выставленных \TST и копирует значение в регистр \RegW{2}, содержащий переменную \q{rt}.

\subsubsection{ARM64 + \NonOptimizing GCC 4.9}

И снова будем использовать 64-битный пример, который мы использовали ранее: \myref{popcnt_x64_example}.
Код более многословный, как обычно.

\lstinputlisting[caption=\NonOptimizing GCC (Linaro) 4.8,style=customasmARM]{patterns/14_bitfields/4_popcnt/ARM64_GCC_O0_RU.s}


\subsection{MIPS}

\lstinputlisting[caption=\Optimizing GCC 4.4.5,style=customasmMIPS]{patterns/05_passing_arguments/MIPS_O3_IDA_RU.lst}

Первые 4 аргумента функции передаются в четырех регистрах с префиксами A-.

\myindex{MIPS!\Instructions!MULT}
В MIPS есть два специальных регистра: HI и LO, которые выставляются в 64-битный результат умножения
во время исполнения инструкции \TT{MULT}.

\myindex{MIPS!\Instructions!MFLO}
\myindex{MIPS!\Instructions!MFHI}
К регистрам можно обращаться только используя инструкции \TT{MFLO} и \TT{MFHI}.
Здесь \TT{MFLO} берет младшую часть результата умножения и записывает в \$V0.
Так что старшая 32-битная часть результата игнорируется (содержимое регистра HI не используется).
Действительно, мы ведь работаем с 32-битным типом \Tint.


\myindex{MIPS!\Instructions!ADDU}
И наконец, \TT{ADDU} (\q{Add Unsigned}~--- добавить беззнаковое) прибавляет значение третьего аргумента к результату.

\myindex{MIPS!\Instructions!ADD}
\myindex{MIPS!\Instructions!ADDU}
\myindex{Ada}
\myindex{Integer overflow}
В MIPS есть две разных инструкции сложения: \TT{ADD} и \TT{ADDU}.
На самом деле, дело не в знаковых числах, а в исключениях: \TT{ADD} может вызвать исключение
во время переполнения. Это иногда полезно\footnote{\url{http://blog.regehr.org/archives/1154}} и поддерживается,
например, в \ac{PL} Ada.

\TT{ADDU} не вызывает исключения во время переполнения.
А так как \CCpp не поддерживает всё это, мы видим здесь \TT{ADDU} вместо \TT{ADD}.

32-битный результат оставляется в \$V0.

\myindex{MIPS!\Instructions!JAL}
\myindex{MIPS!\Instructions!JALR}
В \main есть новая для нас инструкция: \TT{JAL} (\q{Jump and Link}). 
Разница между \INS{JAL} и \INS{JALR} в том, что относительное смещение кодируется в первой инструкции,
а \INS{JALR} переходит по абсолютному адресу, записанному в регистр (\q{Jump and Link Register}).

Обе функции \ttf и \main расположены в одном объектном файле, так что относительный адрес \ttf известен и фиксирован.



\subsubsection{Еще кое-что}

Передача структуры как аргумент функции (вместо передачи указателя на структуру) это то же
что и передача всех полей структуры по одному.

Если поля в структуре пакуются по умолчанию, то функцию f() можно переписать так:

\begin{lstlisting}[style=customc]
void f(char a, int b, char c, int d)
{
    printf ("a=%d; b=%d; c=%d; d=%d\n", a, b, c, d);
};
\end{lstlisting}

И в итоге будет такой же код.
