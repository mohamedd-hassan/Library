\subsubsection{x86}

このようにコンパイルされます。

\lstinputlisting[caption=MSVC 2012 /GS- /Ob0,label=src:struct_packing_4,numbers=left,style=customasmx86]{patterns/15_structs/4_packing/packing_JA.asm}

構造全体を渡しますが、実際には、構造体は
一時的な領域にコピーされて、(スタック内の領域は10行目に割り当てられ、
次に4つのフィールドはすべて1つずつ、12行目から19行目にコピーされます)
そのポインタ(アドレス)が渡されます。

\ttf{} 関数が構造体を変更するかどうかわからないため、
構造体がコピーされます。 
それが変更された場合、 \main の構造体はそのままでいなければなりません。

私たちは \CCpp ポインタを使うことができました。結果のコードはほぼ同じですが、
コピーは行いません。

次に見るように、各フィールドのアドレスは4バイトの境界に揃えられています。 
だからこそ、各 \Tchar が( \Tint のように)4バイトを占めるのです。なぜでしょうか? 
CPUが整列したアドレスでメモリにアクセスし、メモリからデータをキャッシュする方が簡単であるためです。

しかし、あまり経済的ではありません。

オプション(\TT{/Zp1})(nバイト境界で構造体をパックする \emph{/Zp[n]})で
コンパイルしてみましょう。

\lstinputlisting[caption=MSVC 2012 /GS- /Zp1,label=src:struct_packing_1,numbers=left,style=customasmx86]{patterns/15_structs/4_packing/packing_msvc_Zp1_JA.asm}

Now the structure takes only 10 bytes and each \Tchar value takes 1 byte. What does it give to us?
Size economy. And as drawback~---the CPU accessing these fields slower than it could.

構造体は10バイトしかなく、各 \Tchar 値は1バイト必要です。それは私たちに何を与えるのですか?
サイズ経済。そして欠点として、CPUはこれらのフィールドにアクセスするのが遅くなります。

\label{short_struct_copying_using_MOV}

構造体も \main にコピーされます。フィールド単位ではなく、3つの \MOV ペアを使用して直接10バイトをコピーします。
なぜ4ではないのでしょうか?

コンパイラは、3つの \MOV ペアを使用して10バイトをコピーする方が、2つの32ビットワードと
4つの \MOV ペアを使用して2バイトをコピーするよりも優れていると判断しました。

ちなみに、\TT{memcpy()}関数を呼び出す代わりに \MOV を使用するようなコピーの実装は、
\TT{memcpy()}の呼び出しよりも速いため、広く使用されています。
\myref{copying_short_blocks}

簡単に推測できるように、構造体が多くのソースファイルとオブジェクトファイルで使用されている場合、
構造体パッキングについてはすべて同じ規則でコンパイルする必要があります。

各構造体フィールドの配置方法を設定するMSVC \TT{/Zp}オプションの他に、
\TT{\#pragma pack}コンパイラオプションもあります。このオプションはソースコード内で直接定義できます。 
MSVC\FNURLMSDNZP と GCC\FNURLGCCPC{} の両方で利用できます。

16ビットのフィールドで構成される\TT{SYSTEMTIME}構造体に戻りましょう。
私たちのコンパイラは、1バイト境界でパックすることをどうやって知っていますか?

\TT{WinNT.h}ファイルはこれを持っています:

\begin{lstlisting}[caption=WinNT.h,style=customc]
#include "pshpack1.h"
\end{lstlisting}

そしてこれを。

\begin{lstlisting}[caption=WinNT.h,style=customc]
#include "pshpack4.h"                   // 4バイトパッキングがデフォルト
\end{lstlisting}

PshPack1.h ファイルはこのようになっています。

\lstinputlisting[caption=PshPack1.h,style=customc]{patterns/15_structs/4_packing/tmp1.c}

コンパイラは \TT{\#pragma pack} の後で定義される構造体をパックする方法を知らせます。

\clearpage
\myparagraph{x86 + MSVC + \olly}
\myindex{\olly}
\myindex{x86!\Registers!\Flags}

\olly でこの例を実行すると、フラグがどのように設定されているかを見ることができます。 
符号なしの数値で動作する\TT{f\_unsigned()}から始めましょう。

\CMP はここで3回実行されますが、同じ引数についてはフラグは毎回同じです。

最初の比較の結果は、

\begin{figure}[H]
\centering
\myincludegraphics{patterns/07_jcc/simple/olly_unsigned1.png}
\caption{\olly: \TT{f\_unsigned()}: 最初の条件付きジャンプ}
\label{fig:jcc_olly_unsigned_1}
\end{figure}

従って、フラグは、C=1、P=1、A=1、Z=0、S=1、T=0、D=0、O=0です。

これらは \olly では1文字の略号で命名されています。

\olly は、(\JBE)ジャンプがトリガーされることを示唆しています。 
実際に、インテルのマニュアル(\myref{x86_manuals})を調べると、
CF=1またはZF=1の場合、JBEが起動することがわかります。 
条件はここに当てはまるので、ジャンプが開始されます。

\clearpage
次の条件付きジャンプは、

\begin{figure}[H]
\centering
\myincludegraphics{patterns/07_jcc/simple/olly_unsigned2.png}
\caption{\olly: \TT{f\_unsigned()}: 2番目の条件付きジャンプ}
\label{fig:jcc_olly_unsigned_2}
\end{figure}

\olly は、 \JNZ がトリガーされることを示唆しています。 
実際、ZF=0(ゼロフラグ)の場合、JNZが起動します。

\clearpage
3番目の条件付きジャンプは、 \JNB です。

\begin{figure}[H]
\centering
\myincludegraphics{patterns/07_jcc/simple/olly_unsigned3.png}
\caption{\olly: \TT{f\_unsigned()}: 3番目の条件付きジャンプ}
\label{fig:jcc_olly_unsigned_3}
\end{figure}

インテルのマニュアル(\myref{x86_manuals})では、CF=0(キャリーフラグ)の場合に \JNB が起動することがわかります。 
今回は当てはまらないので、3番目の \printf が実行されます。

\clearpage
次に、\olly で、符号付きの値で動作する\TT{f\_signed()}関数を見てみましょう。 
フラグは、C=1、P=1、A=1、Z=0、S=1、T=0、D=0、O=0と同様に設定されます。 
最初の条件付きジャンプ \JLE が起動されます。

インテルマニュアル(166ページの7.1.4)では、ZF = 1またはSFxOFの場合にこの命令がトリガされることがわかりました。 SFxOF私たちの場合は、ジャンプがトリガするように。

\begin{figure}[H]
\centering
\myincludegraphics{patterns/07_jcc/simple/olly_signed1.png}
\caption{\olly: \TT{f\_signed()}: 最初の条件付きジャンプ}
\label{fig:jcc_olly_signed_1}
\end{figure}

インテルマニュアル(\myref{x86_manuals})では、ZF=1または SF$\neq$OF の場合にこの命令が起動されることがわかりました。 
私たちの場合では SF$\neq$OF が、ジャンプが起動されます。

\clearpage
2番目の \JNZ 条件付きジャンプはZF=0の場合(ゼロ・フラグ)に起動します。

\begin{figure}[H]
\centering
\myincludegraphics{patterns/07_jcc/simple/olly_signed2.png}
\caption{\olly: \TT{f\_signed()}: 2番目の条件付きジャンプ}
\label{fig:jcc_olly_signed_2}
\end{figure}

\clearpage
第3の条件付きジャンプ \JGE は、SF=OFの場合にのみ実行されるため、起動しません。今回は、当てはまりません。

\begin{figure}[H]
\centering
\myincludegraphics{patterns/07_jcc/simple/olly_signed3.png}
\caption{\olly: \TT{f\_signed()}: 3番目の条件付きジャンプ}
\label{fig:jcc_olly_signed_3}
\end{figure}

