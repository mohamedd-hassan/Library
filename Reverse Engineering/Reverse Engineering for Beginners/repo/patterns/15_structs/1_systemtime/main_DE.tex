\subsection{MSVC: SYSTEMTIME Beispiel}
\label{sec:SYSTEMTIME}

\newcommand{\FNSYSTEMTIME}{\footnote{\href{http://msdn.microsoft.com/en-us/library/ms724950(VS.85).aspx}{MSDN: SYSTEMTIME structure}}}
Betrachten wir das SYSTEMTIME\FNSYSTEMTIME{} struct in win32, das die Systemzeit beschreibt.

Das struct ist folgendermaßen definiert:

\begin{lstlisting}[caption=WinBase.h,style=customc]
typedef struct _SYSTEMTIME {
  WORD wYear;
  WORD wMonth;
  WORD wDayOfWeek;
  WORD wDay;
  WORD wHour;
  WORD wMinute;
  WORD wSecond;
  WORD wMilliseconds;
} SYSTEMTIME, *PSYSTEMTIME;
\end{lstlisting}
Schreiben wir eine C-Funktion, um die aktuelle Zeit auszugeben:

\lstinputlisting[style=customc]{patterns/15_structs/1_systemtime/systemtime.c}

Wir erhalten das Folgende (MSVC 2010):

\lstinputlisting[caption=MSVC 2010 /GS-,style=customasmx86]{patterns/15_structs/1_systemtime/systemtime.asm}
Für dieses struct werden 16 Byte im lokalen Stack reserviert~---das entspricht genau \TT{sizeof(WORD)*8} (es gibt 8
WORD Variablen in diesem struct).

\newcommand{\FNMSDNGST}{\footnote{\href{http://msdn.microsoft.com/en-us/library/ms724390(VS.85).aspx}{MSDN: GetSystemTime function}}}
Man beachte, dass dieses struct mit dem \TT{wYear} Feld beginnt.
Man kann sagen, dass ein Pointer auf das SYSTEMTIME struct an die Funktion \TT{GetSystemTime()}\FNSYSTEMTIME übergeben
wird, aber man könnte auch sagen, dass ein Pointer auf das Feld \TT{wYear} übergeben wird, denn dabei handelt es sich um
dasselbe!
\TT{GetSystemTime()} schreibt das aktuelle Jahr in den WORD Pointer, verschiebt um 2 Byte, schreibt den aktuellen Monat,
usw. usf. 

\clearpage
\myparagraph{\olly + standardmäßig gepackte Felder}
\myindex{\olly}
Betrachten wir unser Beispiel (in dem die Felder standardmäßig auf 4 Byte angeordnet werden) in \olly:

\begin{figure}[H]
\centering
\myincludegraphics{patterns/15_structs/4_packing/olly_packing_4.png}
\caption{\olly: vor der Ausführung von \printf}
\label{fig:packing_olly_4}
\end{figure}
Wir sehen unsere 4 Felder im Datenfenster.

Wir fragen uns aber, woher die Zufallsbytes (0x30, 0x37, 0x01) stammen, die neben dem ersten ($a$) und dritten ($c$)
Feld liegen.

Betrachten wir unser Listing \myref{src:struct_packing_4}, erkennen wir, dass das erste und dritte Feld vom Typ \Tchar
ist, und daher nur ein Byte geschrieben wird, nämlich 1 bzw. 3 (Zeilen 6 und 8).

Die übrigen 3 Byte des 32-Bit-Wortes werden im Speicher nicht verändert!
Deshalb befinden sich hier zufällige Reste.

\myindex{x86!\Instructions!MOVSX}
Diese Reste beeinflussen den Output von \printf in keinster Weise, da die Werte für die Funktion mithilfe von \MOVSX
vorbereitet werden, der Bytes und nicht Worte als Argumente hat: 
\lstref{src:struct_packing_4} (Zeilen 34 und 38).
Der vorzeichenerweiternde Befehl \MOVSX wird hier übrigens verwendet, da \Tchar standardmäßig in MSVC und GCC
vorzeichenbehaftet ist.
Würde hier der Datentyp \TT{unsigned char} oder \TT{uint8\_t} verwendet, würde der Befehl \MOVZX stattdessen verwendet.

\clearpage
\myparagraph{\olly + Felder auf 1 Byte Grenzen angeordnet}
\myindex{\olly}
Hier sind die Dinge viel klarer ersichtlich: 4 Felder benötigen 16 Byte und die Werte werden nebeneinander gespeichert.

\begin{figure}[H]
\centering
\myincludegraphics{patterns/15_structs/4_packing/olly_packing_1.png}
\caption{\olly: Vor der Ausführung von \printf}
\label{fig:packing_olly_1}
\end{figure}


\subsubsection{Ein struct durch ein Array ersetzen}
Die Tatsache, dass die Felder eines structs Variablen sind, die nebeneinander angeordnet sind, kann leicht durch
folgendes Beispiel belegt werden. Wir erinnern uns an die Beschreibung des \TT{SYSTEMTIME} structs und schreiben unser
Beispiel wie folgt um: 

\lstinputlisting[style=customc]{patterns/15_structs/1_systemtime/systemtime2.c}
Der Compiler meckert ein wenig:

\begin{lstlisting}
systemtime2.c(7) : warning C4133: 'function' : incompatible types - from 'WORD [8]' to 'LPSYSTEMTIME'
\end{lstlisting}

Trotzdem erzeugt er den folgenden Code:

\lstinputlisting[caption=\NonOptimizing MSVC 2010,style=customasmx86]{patterns/15_structs/1_systemtime/systemtime2.asm}
Dieses Programm funktioniert genau wie das erste!

Sehr interessant ist, dass dieser Assemblercode nicht vom entsprechenden Code des ersten Beispiels unterschieden werden
kann.

Beim bloßen Ansehen des Codes kann man also nicht feststellen, ob ein struct oder ein Array deklariert wurde.

Trotzdem würde man letzteres nicht annehmen, da es sehr ungebräuchlich ist.

DIe Felder des structs können durch den Entwickler ausgetauscht oder verändert werden, etc.

Wir untersuchen dieses Beispiel nicht in \olly, da es mit dem Beispiel mit dem struct identisch ist.

