\subsection{MSVC: Пример SYSTEMTIME}
\label{sec:SYSTEMTIME}

\newcommand{\FNSYSTEMTIME}{\footnote{\href{http://msdn.microsoft.com/en-us/library/ms724950(VS.85).aspx}{MSDN: SYSTEMTIME structure}}}

Возьмем, к примеру, структуру SYSTEMTIME\FNSYSTEMTIME{} из win32 описывающую время.

Она объявлена так:

\begin{lstlisting}[caption=WinBase.h,style=customc]
typedef struct _SYSTEMTIME {
  WORD wYear;
  WORD wMonth;
  WORD wDayOfWeek;
  WORD wDay;
  WORD wHour;
  WORD wMinute;
  WORD wSecond;
  WORD wMilliseconds;
} SYSTEMTIME, *PSYSTEMTIME;
\end{lstlisting}

Пишем на Си функцию для получения текущего системного времени:

\lstinputlisting[style=customc]{patterns/15_structs/1_systemtime/systemtime.c}

Что в итоге (MSVC 2010):

\lstinputlisting[caption=MSVC 2010 /GS-,style=customasmx86]{patterns/15_structs/1_systemtime/systemtime.asm}

Под структуру в стеке выделено 16 байт ~--- именно столько будет \TT{sizeof(WORD)*8}
(в структуре 8 переменных с типом WORD).

\newcommand{\FNMSDNGST}{\footnote{\href{http://msdn.microsoft.com/en-us/library/ms724390(VS.85).aspx}{MSDN: GetSystemTime function}}}

Обратите внимание на тот факт, что структура начинается с поля \TT{wYear}. 
Можно сказать, что в качестве аргумента для \TT{GetSystemTime()}\FNMSDNGST передается указатель на структуру 
SYSTEMTIME, но можно также сказать, что передается указатель на поле \TT{wYear}, 
что одно и тоже! 
\TT{GetSystemTime()} пишет текущий год в тот WORD на который указывает переданный указатель, 
затем сдвигается на 2 байта вправо, пишет текущий месяц, итд, итд.

\clearpage
\myparagraph{\olly + упаковка полей по умолчанию}
\myindex{\olly}

Попробуем в \olly наш пример, где поля выровнены по умолчанию (4 байта):

\begin{figure}[H]
\centering
\myincludegraphics{patterns/15_structs/4_packing/olly_packing_4.png}
\caption{\olly: Перед исполнением \printf}
\label{fig:packing_olly_4}
\end{figure}

В окне данных видим наши четыре поля.
Вот только, откуда взялись случайные байты (0x30, 0x37, 0x01) рядом с первым (a) и третьим (c) полем?

Если вернетесь к листингу \myref{src:struct_packing_4}, то увидите, что первое и третье поле имеет
тип \Tchar, а следовательно, туда записывается только один байт, 1 и 3 соответственно (строки 6 и 8).

Остальные три байта 32-битного слова не будут модифицироваться в памяти!

А, следовательно, там остается случайный мусор.
\myindex{x86!\Instructions!MOVSX}
Этот мусор никак не будет влиять на работу \printf,
потому что значения для нее готовятся при помощи инструкции \MOVSX, которая загружает
из памяти байты а не слова: 
\lstref{src:struct_packing_4} (строки 34 и 38).

Кстати, здесь используется именно \MOVSX (расширяющая знак), потому что тип 
\Tchar --- знаковый по умолчанию в MSVC и GCC.

Если бы здесь был тип \TT{unsigned char} или \TT{uint8\_t}, 
то здесь была бы инструкция \MOVZX.

\clearpage
\myparagraph{\olly + упаковка полей по границе в 1 байт}
\myindex{\olly}

Здесь всё куда понятнее: 4 поля занимают 10 байт и значения сложены в памяти друг к другу

\begin{figure}[H]
\centering
\myincludegraphics{patterns/15_structs/4_packing/olly_packing_1.png}
\caption{\olly: Перед исполнением \printf}
\label{fig:packing_olly_1}
\end{figure}


\subsubsection{Замена структуры массивом}

Тот факт, что поля структуры --- это просто переменные расположенные рядом, легко проиллюстрировать следующим образом.%

Глядя на описание структуры \TT{SYSTEMTIME}, можно переписать этот простой пример так:%

\lstinputlisting[style=customc]{patterns/15_structs/1_systemtime/systemtime2.c}

Компилятор немного ворчит:

\begin{lstlisting}
systemtime2.c(7) : warning C4133: 'function' : incompatible types - from 'WORD [8]' to 'LPSYSTEMTIME'
\end{lstlisting}

Тем не менее, выдает такой код:

\lstinputlisting[caption=\NonOptimizing MSVC 2010,style=customasmx86]{patterns/15_structs/1_systemtime/systemtime2.asm}

И это работает так же!

Любопытно что результат на ассемблере неотличим от предыдущего.
Таким образом, глядя на этот код, 
никогда нельзя сказать с уверенностью, была ли там объявлена структура, либо просто набор переменных.

Тем не менее, никто в здравом уме делать так не будет.

Потому что это неудобно. 
К тому же, иногда, поля в структуре могут меняться разработчиками, переставляться местами, итд.

С \olly этот пример изучать не будем, потому что он будет точно такой же, как и в случае со структурой.

