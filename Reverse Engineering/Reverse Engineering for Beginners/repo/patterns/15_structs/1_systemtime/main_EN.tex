\subsection{MSVC: SYSTEMTIME example}
\label{sec:SYSTEMTIME}

\newcommand{\FNSYSTEMTIME}{\footnote{\href{http://msdn.microsoft.com/en-us/library/ms724950(VS.85).aspx}{MSDN: SYSTEMTIME structure}}}

Let's take the SYSTEMTIME\FNSYSTEMTIME{} win32 structure that describes time.

This is how it's defined:

\begin{lstlisting}[caption=WinBase.h,style=customc]
typedef struct _SYSTEMTIME {
  WORD wYear;
  WORD wMonth;
  WORD wDayOfWeek;
  WORD wDay;
  WORD wHour;
  WORD wMinute;
  WORD wSecond;
  WORD wMilliseconds;
} SYSTEMTIME, *PSYSTEMTIME;
\end{lstlisting}

Let's write a C function to get the current time:

\lstinputlisting[style=customc]{patterns/15_structs/1_systemtime/systemtime.c}

We get (MSVC 2010):

\lstinputlisting[caption=MSVC 2010 /GS-,style=customasmx86]{patterns/15_structs/1_systemtime/systemtime.asm}

16 bytes are allocated for this structure in the local stack~---that is exactly \TT{sizeof(WORD)*8}
(there are 8 WORD variables in the structure).

\newcommand{\FNMSDNGST}{\footnote{\href{http://msdn.microsoft.com/en-us/library/ms724390(VS.85).aspx}{MSDN: GetSystemTime function}}}

Pay attention to the fact that the structure begins with the \TT{wYear} field.
It can be said that a pointer to the SYSTEMTIME structure is passed to the \TT{GetSystemTime()}\FNSYSTEMTIME,
but it is also can be said that a pointer to the \TT{wYear} field is passed, and that is the same!
\TT{GetSystemTime()} writes the current year to the WORD pointer pointing to, then shifts 2 bytes
ahead, writes current month, etc., etc.

\clearpage
\subsubsection{MSVC + \olly}
\myindex{\olly}

Let's load our example into \olly and set a breakpoint on \comp.
We can see how the values are compared at the first \comp call:

\begin{figure}[H]
\centering
\myincludegraphics{patterns/18_pointers_to_functions/olly1.png}
\caption{\olly: first call of \comp}
\label{fig:qsort_olly1}
\end{figure}

\olly shows the compared values in the window under the code window, for convenience.
We can also see that the \ac{SP} points to \ac{RA}, where the \qsort function is (located in \TT{MSVCR100.DLL}).

\clearpage
By tracing (F8) until the \TT{RETN} instruction and pressing F8 one more time, we return to the \qsort function:

\begin{figure}[H]
\centering
\myincludegraphics{patterns/18_pointers_to_functions/olly2.png}
\caption{\olly: the code in \qsort right after \comp call}
\label{fig:qsort_olly2}
\end{figure}

That has been a call to the comparison function.

\clearpage
Here is also a screenshot of the moment of the second call of \comp{}---now values that have to be compared are different:

\begin{figure}[H]
\centering
\myincludegraphics{patterns/18_pointers_to_functions/olly3.png}
\caption{\olly: second call of \comp}
\label{fig:qsort_olly3}
\end{figure}


\subsubsection{Replacing the structure with array}

The fact that the structure fields are just variables located side-by-side, can be easily demonstrated by doing the following.
Keeping in mind the \TT{SYSTEMTIME} structure description, it's possible to rewrite this simple example like this:

\lstinputlisting[style=customc]{patterns/15_structs/1_systemtime/systemtime2.c}

The compiler grumbles a bit:

\begin{lstlisting}
systemtime2.c(7) : warning C4133: 'function' : incompatible types - from 'WORD [8]' to 'LPSYSTEMTIME'
\end{lstlisting}

But nevertheless, it produces this code:

\lstinputlisting[caption=\NonOptimizing MSVC 2010,style=customasmx86]{patterns/15_structs/1_systemtime/systemtime2.asm}

And it works just as the same!

It is very interesting that the
result in assembly form cannot be distinguished from the result of the previous compilation.

So by looking at this code, one cannot say for sure if there was a structure declared, or an array. 

Nevertheless, no sane person would do it, as it is not convenient. 

Also the structure fields may be changed by developers, swapped, etc.

We will not study this example in \olly, because it will be just the same as in the case with the structure.

