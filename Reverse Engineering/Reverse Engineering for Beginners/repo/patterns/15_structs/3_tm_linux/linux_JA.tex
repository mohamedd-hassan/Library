\subsubsection{Linux}

Linuxの\TT{time.h}の\TT{tm}構造体を例にとりましょう。

\lstinputlisting[style=customc]{patterns/15_structs/3_tm_linux/GCC_tm.c}

GCC 4.4.1でコンパイルしましょう。

\lstinputlisting[caption=GCC 4.4.1,style=customasmx86]{patterns/15_structs/3_tm_linux/GCC_tm_JA.asm}

どういうわけか、 \IDA はローカルスタックにローカル変数を書き込みませんでした。
しかし、経験を積んだリバース・エンジニアなので :-) この単純な例では情報なしで実行できるかもしれません。

\myindex{x86!\Instructions!LEA}

\TT{lea edx, [eax+76Ch]}にも注意してください。この命令は\TT{0x76C} (1900)を \EAX の値に追加するだけですが、
フラグは変更しません。 \LEA{}~(\myref{sec:LEA})に関する関連セクションも参照してください。

\myparagraph{GDB}

例をGDBにロードしてみましょう。
\footnote{デモの目的で、\emph{date}の結果が少し修正されます。 
もちろん、同じ秒で、GDBをすばやく実行することはできません。}

\lstinputlisting[caption=GDB]{patterns/15_structs/3_tm_linux/GCC_tm_GDB.txt}

私たちは簡単にスタック内の構造を見つけることができます。
まず、\emph{time.h}でどのように定義されているのかを見てみましょう。

\begin{lstlisting}[caption=time.h, label=struct_tm,style=customc]
struct tm
{
  int	tm_sec;
  int	tm_min;
  int	tm_hour;
  int	tm_mday;
  int	tm_mon;
  int	tm_year;
  int	tm_wday;
  int	tm_yday;
  int	tm_isdst;
};
\end{lstlisting}

ここでは、SYSTEMTIMEのWORDの代わりに32ビット \Tint 
が使用されていることに注意してください。 
したがって、各フィールドは32ビットを占有します。

スタック内の構造体のフィールドは次のとおりです。

\lstinputlisting{patterns/15_structs/3_tm_linux/stack.txt}

またはテーブルとして。

\begin{center}
\begin{tabular}{ | l | l | l | }
\hline
\headercolor{} 16進数 & 
\headercolor{} 10進数 & 
\headercolor{} フィールド名 \\
\hline
0x00000025 & 37 	& tm\_sec \\
\hline
0x0000000a & 10 	& tm\_min \\
\hline
0x00000012 & 18 	& tm\_hour \\	
\hline
0x00000002 & 2 		& tm\_mday \\	
\hline
0x00000005 & 5 		& tm\_mon \\	
\hline
0x00000072 & 114 	& tm\_year \\
\hline
0x00000001 & 1 		& tm\_wday \\	
\hline
0x00000098 & 152 	& tm\_yday \\	
\hline
0x00000001 & 1 		& tm\_isdst \\
\hline
\end{tabular}
\end{center}

SYSTEMTIMEと同じように(\myref{sec:SYSTEMTIME})

tm\_wday, tm\_yday, tm\_isdstなど、使用されていない他のフィールドも使用できます。
