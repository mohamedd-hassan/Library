\subsubsection{\WorkingWithFloatAsWithStructSubSubSectionName}
\label{sec:floatasstruct}

Comme nous l'avons expliqué dans la section traitant de la FPU~(\myref{sec:FPU}), les types \Tfloat et 
\Tdouble sont constitués d'un \emph{signe}, d'un \emph{significande} (ou \emph{fraction}) et d'un \emph{exposant}.
Mais serions nous capable de travailler avec chacun de ces champs indépendamment? Essayons avec un \Tfloat.

\bigskip
% a hack used here! http://tex.stackexchange.com/questions/73524/bytefield-package
\begin{center}
\begin{bytefield}{32}
	\bitheader[endianness=big]{0,22,23,30,31} \\
	\bitbox{1}{S} &
	\bitbox{8}{%
		\RU{экспонента}%
		\EN{exponent}%
		\ES{exponente}%
		\PTBRph{}%
		\DEph{}\PLph{}%
		\IT{esponente}%
		\FR{exposant}%
		\JA{指数}
	} &
	\bitbox{23}{%
		\RU{мантисса}%
		\EN{mantissa or fraction}%
		\ES{mantisa o fracci\'on}%
		\PTBRph{}%
		\DEph{}
		\PLph{}%
		\IT{mantissa}%
		\FR{mantisse ou fraction}%
		\JA{対数または比}
	}
\end{bytefield}
\end{center}

\begin{center}
( S --- %
	\RU{знак}%
	\EN{sign}%
	\ES{signo}%
	\PTBRph{}%
	\DEph{}\PLph{}%
	\IT{segno}%
	\FR{signe}%
	\JA{記号}
)
\end{center}


\lstinputlisting[style=customc]{patterns/15_structs/6_bitfields/float/float_FR.c}

La structure \TT{float\_as\_struct} occupe le même espace qu'un \Tfloat, soit 4 octets ou 32 bits.

Nous positionnons maintenant le signe pour qu'il soit négatif puis en ajoutant à la valeur de l'exposant, 
ce qui fait que nous multiplions le nombre par \TT{$2^2$}, soit 4.

Compilons notre exemple avec MSVC 2008, sans optimisation:

\lstinputlisting[caption=MSVC 2008 \NonOptimizing,style=customasmx86]{patterns/15_structs/6_bitfields/float/float_msvc_FR.asm}

Si nous avions compilé avec le flag \Ox il n'y aurait pas d'appel à la fonction \TT{memcpy()}, et la variable 
\TT{f} serait utilisée directement. Mais la compréhension est facilitée lorsque l'on s'intéresse à la version 
non optimisée.

A quoi cela ressemblerait si nous utilisions l'option \Othree avec le compilateur GCC 4.4.1 ?

\lstinputlisting[caption=GCC 4.4.1 \Optimizing,style=customasmx86]{patterns/15_structs/6_bitfields/float/float_gcc_O3_FR.asm}

La fonction \ttf est à peu près compréhensible. Par contre ce qui est intéressant c'est que GCC a été 
capable de calculer le résultat de \TT{f(1.234)} durant la compilation malgré tous les triturages des champs 
de la structure et a directement préparé l'argument passé à \printf{} durant la compilation!
