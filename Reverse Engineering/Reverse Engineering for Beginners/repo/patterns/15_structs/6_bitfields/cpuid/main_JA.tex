\subsubsection{CPUIDの例}

\CCpp 言語では、各構造体フィールドの正確なビット数を定義できます。 
メモリ空間を節約する必要がある場合に非常に便利です。 
たとえば、 \Tbool 変数には1ビットで十分です。 
しかし、スピードが重要なら合理的ではありません。
% FIXME!
% another use of this is to parse binary protocols/packets, for example
% the definition of struct iphdr in include/linux/ip.h

\newcommand{\FNCPUID}{\footnote{\href{http://en.wikipedia.org/wiki/CPUID}{wikipedia}}}

\myindex{x86!\Instructions!CPUID}
\label{cpuid}

\CPUID\FNCPUID 命令の例を考えてみましょう。
この命令は、現在のCPUとその機能に関する情報を返します。

命令が実行される前に \EAX が1に設定されている場合、
\CPUID は \EAX レジスタに情報がパックされて返ります。

\begin{center}
\begin{tabular}{ | l | l | }
\hline
3:0 (4 bits)& ステッピング \\
7:4 (4 bits) & モデル \\
11:8 (4 bits) & ファミリーy \\
13:12 (2 bits) & プロセッサタイプ \\
19:16 (4 bits) & 拡張モデル \\
27:20 (8 bits) & 拡張ファミリー \\
\hline
\end{tabular}
\end{center}

\newcommand{\FNGCCAS}{\footnote{\href{http://www.ibiblio.org/gferg/ldp/GCC-Inline-Assembly-HOWTO.html}
{GCCアセンブラ内部の詳細}}}

MSVC 2010には \CPUID マクロがありますが、GCC 4.4.1にはありません。 
ですから組み込みアセンブラ \FNGCCAS の助けを借りてGCCのためにこの機能を自分自身で作ってみましょう。

\lstinputlisting[style=customc]{patterns/15_structs/6_bitfields/cpuid/CPUID.c}

\CPUID が \EAX/\EBX/\ECX/\EDX を満たすと、これらのレジスタは\TT{b[]}配列に書き込まれます。 
次に、\TT{CPUID\_1\_EAX}構造体へのポインタを持ち、それを\TT{b[]}配列から \EAX の値に向けます。

つまり、32ビットの \Tint 値を構造体として扱います。 
次に、構造体から特定のビットを読み込みます。

\myparagraph{MSVC}

\Ox オプションを付けてMSVC 2008でコンパイルしてみましょう。

\lstinputlisting[caption=\Optimizing MSVC 2008,style=customasmx86]{patterns/15_structs/6_bitfields/cpuid/CPUID_msvc_Ox.asm}

\myindex{x86!\Instructions!SHR}

\TT{SHR}命令は、 \EAX 内の値を、\emph{スキップ}しなければならないビット数だけシフトします。
例えば、\emph{右側の}ビットを無視します。

\myindex{x86!\Instructions!AND}

\AND 命令は、\emph{左側の}不要ビットをクリアします。言い換えれば、
必要な \EAX レジスタのビットだけを残します。

\clearpage
\myparagraph{x86 + MSVC + \olly}
\myindex{\olly}
\myindex{x86!\Registers!\Flags}

\olly でこの例を実行すると、フラグがどのように設定されているかを見ることができます。 
符号なしの数値で動作する\TT{f\_unsigned()}から始めましょう。

\CMP はここで3回実行されますが、同じ引数についてはフラグは毎回同じです。

最初の比較の結果は、

\begin{figure}[H]
\centering
\myincludegraphics{patterns/07_jcc/simple/olly_unsigned1.png}
\caption{\olly: \TT{f\_unsigned()}: 最初の条件付きジャンプ}
\label{fig:jcc_olly_unsigned_1}
\end{figure}

従って、フラグは、C=1、P=1、A=1、Z=0、S=1、T=0、D=0、O=0です。

これらは \olly では1文字の略号で命名されています。

\olly は、(\JBE)ジャンプがトリガーされることを示唆しています。 
実際に、インテルのマニュアル(\myref{x86_manuals})を調べると、
CF=1またはZF=1の場合、JBEが起動することがわかります。 
条件はここに当てはまるので、ジャンプが開始されます。

\clearpage
次の条件付きジャンプは、

\begin{figure}[H]
\centering
\myincludegraphics{patterns/07_jcc/simple/olly_unsigned2.png}
\caption{\olly: \TT{f\_unsigned()}: 2番目の条件付きジャンプ}
\label{fig:jcc_olly_unsigned_2}
\end{figure}

\olly は、 \JNZ がトリガーされることを示唆しています。 
実際、ZF=0(ゼロフラグ)の場合、JNZが起動します。

\clearpage
3番目の条件付きジャンプは、 \JNB です。

\begin{figure}[H]
\centering
\myincludegraphics{patterns/07_jcc/simple/olly_unsigned3.png}
\caption{\olly: \TT{f\_unsigned()}: 3番目の条件付きジャンプ}
\label{fig:jcc_olly_unsigned_3}
\end{figure}

インテルのマニュアル(\myref{x86_manuals})では、CF=0(キャリーフラグ)の場合に \JNB が起動することがわかります。 
今回は当てはまらないので、3番目の \printf が実行されます。

\clearpage
次に、\olly で、符号付きの値で動作する\TT{f\_signed()}関数を見てみましょう。 
フラグは、C=1、P=1、A=1、Z=0、S=1、T=0、D=0、O=0と同様に設定されます。 
最初の条件付きジャンプ \JLE が起動されます。

インテルマニュアル(166ページの7.1.4)では、ZF = 1またはSFxOFの場合にこの命令がトリガされることがわかりました。 SFxOF私たちの場合は、ジャンプがトリガするように。

\begin{figure}[H]
\centering
\myincludegraphics{patterns/07_jcc/simple/olly_signed1.png}
\caption{\olly: \TT{f\_signed()}: 最初の条件付きジャンプ}
\label{fig:jcc_olly_signed_1}
\end{figure}

インテルマニュアル(\myref{x86_manuals})では、ZF=1または SF$\neq$OF の場合にこの命令が起動されることがわかりました。 
私たちの場合では SF$\neq$OF が、ジャンプが起動されます。

\clearpage
2番目の \JNZ 条件付きジャンプはZF=0の場合(ゼロ・フラグ)に起動します。

\begin{figure}[H]
\centering
\myincludegraphics{patterns/07_jcc/simple/olly_signed2.png}
\caption{\olly: \TT{f\_signed()}: 2番目の条件付きジャンプ}
\label{fig:jcc_olly_signed_2}
\end{figure}

\clearpage
第3の条件付きジャンプ \JGE は、SF=OFの場合にのみ実行されるため、起動しません。今回は、当てはまりません。

\begin{figure}[H]
\centering
\myincludegraphics{patterns/07_jcc/simple/olly_signed3.png}
\caption{\olly: \TT{f\_signed()}: 3番目の条件付きジャンプ}
\label{fig:jcc_olly_signed_3}
\end{figure}


\myparagraph{GCC}

\Othree オプション付きのGCC 4.4.1を試してみましょう。

\lstinputlisting[caption=\Optimizing GCC 4.4.1,style=customasmx86]{patterns/15_structs/6_bitfields/cpuid/CPUID_gcc_O3.asm}

ほとんど同じです。 
唯一注目すべきは、GCCは、 \printf の各呼び出しの前に個別に計算するのではなく、
\TT{extended\_model\_id}と\TT{extended\_family\_id}の計算を
どういうわけか1つのブロックに組み合わせることです。
