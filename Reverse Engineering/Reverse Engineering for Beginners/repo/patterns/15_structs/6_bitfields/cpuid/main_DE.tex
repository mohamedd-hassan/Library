\subsubsection{CPUID Beispiel}
Die Sprache \CCpp erlaubt die Definition der exakten Anzahl von Bits für jedes Feld in einem struct.
Das ist sehr nützlich, wenn man Speicherplatz sparen muss.
Zum Beispiel genügt ein Bit für eine \Tbool Variable.
Natürlich ist dieses Vorgehen nicht angebracht, wenn die Geschwindigkeit wichtig ist.

% FIXME!
% another use of this is to parse binary protocols/packets, for example
% the definition of struct iphdr in include/linux/ip.h

\newcommand{\FNCPUID}{\footnote{\href{http://en.wikipedia.org/wiki/CPUID}{wikipedia}}}

\myindex{x86!\Instructions!CPUID}
\label{cpuid}
Betrachten wir das Beispiel des \CPUID\FNCPUID Befehls.
Dieser Befehl liefert Informationen über die aktuelle CPU und ihre Eigenschaften.

Wenn \EAX vor der Ausführung des Befehls auf 1 gesetzt ist, liefert \CPUID diese Informationen gepackt in das \EAX
Register zurück:

\begin{center}
\begin{tabular}{ | l | l | }
\hline
3:0 (4 bits)& Schrittweite \\
7:4 (4 bits) & Modell \\
11:8 (4 bits) & Familie \\
13:12 (2 bits) & Prozessortype \\
19:16 (4 bits) & Erweitertes Modell \\
27:20 (8 bits) & Erweiterte Familie \\
\hline
\end{tabular}
\end{center}

\newcommand{\FNGCCAS}{\footnote{\href{http://www.ibiblio.org/gferg/ldp/GCC-Inline-Assembly-HOWTO.html}
{Mehr zum internen GCC Assembler}}}
MSVC 2010 verfügt über ein \CPUID Makro, aber GCC 4.4.1 nicht.
Erstellen wir also für uns eine solche Funktion in GCC, indem wir den built-in Assembler\FNGCCAS verwenden.

\lstinputlisting[style=customc]{patterns/15_structs/6_bitfields/cpuid/CPUID.c}
Nachdem \CPUID die Register \EAX/\EBX/\ECX/\EDX befüllt hat, werden deren Inhalte in das Array \TT{b[]} geschrieben.
Danach haben wir einen Pointer auf das \TT{CPUID\_1\_EAX} struct und zeigen auf den Wert in \EAX aus dem Array \TT{b[]}.

Mit anderen Worten: wir behandeln einen 32-Bit \Tint wie ein struct.
Danach lesen wir spezifische Bits aus dem struct.

\myparagraph{MSVC}
Kompilieren wir das Beispiel in MSVC 2008 mit der Option \Ox:

\lstinputlisting[caption=\Optimizing MSVC 2008,style=customasmx86]{patterns/15_structs/6_bitfields/cpuid/CPUID_msvc_Ox.asm}

\myindex{x86!\Instructions!SHR}
Der Befehl \TT{SHR} verschiebt den Wert in \EAX um die Anzahl der Bits die überprungen werden müssen, d.h. wir
ignorieren einige Bits am rechten Rand.

\myindex{x86!\Instructions!AND}
Der Befehl \AND löscht die nicht benötigten Bits am linken Rand bzw. belässt nur die Bits in \EAX, die wir auch
benötigen.

\clearpage
\myparagraph{\olly + standardmäßig gepackte Felder}
\myindex{\olly}
Betrachten wir unser Beispiel (in dem die Felder standardmäßig auf 4 Byte angeordnet werden) in \olly:

\begin{figure}[H]
\centering
\myincludegraphics{patterns/15_structs/4_packing/olly_packing_4.png}
\caption{\olly: vor der Ausführung von \printf}
\label{fig:packing_olly_4}
\end{figure}
Wir sehen unsere 4 Felder im Datenfenster.

Wir fragen uns aber, woher die Zufallsbytes (0x30, 0x37, 0x01) stammen, die neben dem ersten ($a$) und dritten ($c$)
Feld liegen.

Betrachten wir unser Listing \myref{src:struct_packing_4}, erkennen wir, dass das erste und dritte Feld vom Typ \Tchar
ist, und daher nur ein Byte geschrieben wird, nämlich 1 bzw. 3 (Zeilen 6 und 8).

Die übrigen 3 Byte des 32-Bit-Wortes werden im Speicher nicht verändert!
Deshalb befinden sich hier zufällige Reste.

\myindex{x86!\Instructions!MOVSX}
Diese Reste beeinflussen den Output von \printf in keinster Weise, da die Werte für die Funktion mithilfe von \MOVSX
vorbereitet werden, der Bytes und nicht Worte als Argumente hat: 
\lstref{src:struct_packing_4} (Zeilen 34 und 38).
Der vorzeichenerweiternde Befehl \MOVSX wird hier übrigens verwendet, da \Tchar standardmäßig in MSVC und GCC
vorzeichenbehaftet ist.
Würde hier der Datentyp \TT{unsigned char} oder \TT{uint8\_t} verwendet, würde der Befehl \MOVZX stattdessen verwendet.

\clearpage
\myparagraph{\olly + Felder auf 1 Byte Grenzen angeordnet}
\myindex{\olly}
Hier sind die Dinge viel klarer ersichtlich: 4 Felder benötigen 16 Byte und die Werte werden nebeneinander gespeichert.

\begin{figure}[H]
\centering
\myincludegraphics{patterns/15_structs/4_packing/olly_packing_1.png}
\caption{\olly: Vor der Ausführung von \printf}
\label{fig:packing_olly_1}
\end{figure}


\myparagraph{GCC}
Versuchen wir es mit GCC 4.4.1 mit der Option \Othree

\lstinputlisting[caption=\Optimizing GCC 4.4.1,style=customasmx86]{patterns/15_structs/6_bitfields/cpuid/CPUID_gcc_O3.asm}
Fast das gleiche. Das einzig Bemerkenswerte ist, dass GCC die Berechnung von \TT{extended\_model\_id} und
\TT{extended\_family\_id} in einem Block kombiniert, anstatt sie vor jedem Aufruf von \printf getrennt zu berechnen.
