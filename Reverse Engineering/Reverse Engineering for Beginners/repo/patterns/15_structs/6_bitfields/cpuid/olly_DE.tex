\clearpage
\myparagraph{MSVC + \olly}
\myindex{\olly}

Laden wir unser Beispiel in \olly 
und schauen welche Werte sich nach der Ausführung von CPUID in den Register \EAX/\EBX/\ECX und \EDX befinden:

\begin{figure}[H]
\centering
\myincludegraphics{patterns/15_structs/6_bitfields/cpuid/olly.png}
\caption{\olly: Nach Ausführung von CPUID}
\label{fig:cpuid_olly_1}
\end{figure}

EAX enthält \TT{0x000206A7} (meine \ac{CPU} ist ein Intel Xeon E3-1220).\\
Dies entspricht $0b0000 0000 0000 0010 0000 0110 1010 0111$ in Binärdarstellung.

Die Bits werden wie folgt durch die Felder aufgeteilt:

\begin{center}
\begin{tabular}{ | l | l | l | }
\hline
\headercolor{} Feld &
\headercolor{} in binär &
\headercolor{} in dezimal \\
\hline
reserved2		& 0000 & 0 \\
\hline
extended\_family\_id	& 00000000 & 0 \\
\hline
extended\_model\_id	& 0010 & 2 \\
\hline
reserved1		& 00 & 0 \\
\hline
processor\_id		& 00 & 0 \\
\hline
family\_id		& 0110 & 6 \\
\hline
model			& 1010 & 10 \\
\hline
stepping		& 0111 & 7 \\
\hline
\end{tabular}
\end{center}

\lstinputlisting[caption=Console output]{patterns/15_structs/6_bitfields/cpuid/console.txt}
