\subsection{32ビット値から64ビット値への変換}
\label{subsec:sign_extending_32_to_64}

\lstinputlisting[style=customc]{patterns/185_64bit_in_32_env/conversion/4.c}

\subsubsection{x86}

\lstinputlisting[caption=\Optimizing MSVC 2012,style=customasmx86]{patterns/185_64bit_in_32_env/conversion/MSVC2012_Ox.asm}

ここでも、32ビットの符号付き値を64ビットの符号付き値に拡張する必要があります。 
符号なしの値は単純に変換されます:上位部分のすべてのビットは0に設定する必要があります。
ただし、符号付きデータ型には適していません:符号は結果の数値の上位部分にコピーする必要があります。
\myindex{x86!\Instructions!CDQ}

\INS{CDQ}命令はここでそれを行います。\EAX{}でその入力値を取り、それを64ビットに拡張しそして
\EDX{}:\EAX{}レジスタペアに残します。
つまり、\INS{CDQ}は(EAXの最上位ビットを取得することによって)\EAX{}から番号記号を取得し、
それに応じて\EDX{}の32ビットすべてを0または1に設定します。
その動作は、\MOVSX{}命令とやや似ています。

\subsubsection{ARM}

\lstinputlisting[caption=\OptimizingKeilVI (\ARMMode),style=customasmARM]{patterns/185_64bit_in_32_env/conversion/Keil_ARM_O3.s}

ARM用Keilは異なります。入力値を算術的に右に31ビットシフトします。 
知っての通り、符号ビットは\ac{MSB}で、算術シフトは符号ビットを\q{出現した}ビットにコピーします。 
したがって、\q{ASR r1,r0,\#31}の後、入力値が負の場合は\Reg{1}に0xFFFFFFFFが含まれ、それ以外の場合は0が含まれます。 
\Reg{1}には、結果の64ビット値の上位部分が含まれています。 
つまり、このコードは\Reg{0} の入力値から結果の64ビット値の上位32ビット部分のすべてのビットに\ac{MSB}(符号ビット)をコピーするだけです。

\subsubsection{MIPS}

MIPS向けのGCCは、KeilがARMモードで行ったのと同じことを行います。

\lstinputlisting[caption=\Optimizing GCC 4.4.5 (IDA),style=customasmMIPS]{patterns/185_64bit_in_32_env/conversion/MIPS_O3_IDA.lst}
