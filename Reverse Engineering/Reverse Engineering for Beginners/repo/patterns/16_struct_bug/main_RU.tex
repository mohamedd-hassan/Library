\mysection{Классическая ошибка с \emph{struct}}

Вот классическая ошибка с \emph{struct}.

Простое определение:

\lstinputlisting[style=customc]{patterns/16_struct_bug/old.h}

И файлы на Си:

\lstinputlisting[style=customc]{patterns/16_struct_bug/setter.c}

\lstinputlisting[style=customc]{patterns/16_struct_bug/printer.c}

Пока всё хорошо.

Теперь вы добавляете третье поле в структуру, где-то между двумя полями:

\lstinputlisting[style=customc]{patterns/16_struct_bug/new.h}

И наверное вы модифицируете ф-цию \verb|setter()|, но забываете за \verb|printer()|:

\lstinputlisting[style=customc]{patterns/16_struct_bug/setter_new.c}

Компилируете ваш проект, но файл на Си где находится ф-ция \verb|printer()| не перекомпилируется.
потому что ваша \ac{IDE} или система сборки не знает о том, что этот модуль зависим от определения структуры \emph{test}.
Может быть, потому что забыли добавить \verb|#include <new.h>|.
А может потому что заголовочный файл \verb|new.h| включен в \verb|printer.c| через какой-то другой заголовочный файл.
Объектный файл остается неизменным (\ac{IDE} думает, что его не нужно пересобирать),
в то время как ф-ция \verb|setter()| уже новой версии.
Эти два объектных файла (старый и новый) в итоге компонуются в исполняемый файл.

Потом вы его запускаете, и \verb|setter()| записывает 3 поля по смещениям +0, +4 и +8.
Хотяr, ф-ция \verb|printer()| знает только о двух полях, и во время печати читает их по смещениям +0 и +4.

Это приводит к очень запутанным и неприятным ошибкам.
Причина в том, что \ac{IDE} или система сборки или Makefile не знают, что оба файла на Си (или модули) зависят от заголовочного
файла с определением \emph{test}.
Популярное решение это пересобрать проект полностью, заново.

Это так же справедливо и для классов в Си++, так как они работают как структуры: \myref{CppClasses}.

Это болезнь Си/Си++, и одна из причин критики, да.
Множество новых \ac{PL} имеют л\'{у}чшую поддержку модулей и интерфейсов.
Но не забывайте о том, когда создавался компилятор Си: в 1970-ых, на старых компьютерах PDP.
Так что создателями Си всё было упрощено до предела.

