\subsubsection{x86}

\myparagraph{x86 + MSVC}

La funzione \TT{f\_signed()} appare così:

\lstinputlisting[caption=\NonOptimizing MSVC 2010,style=customasmx86]{patterns/07_jcc/simple/signed_MSVC.asm}

\myindex{x86!\Instructions!JLE}

La prima istruzione, \JLE, sta per \emph{Jump if Less or Equal} (\emph{salta se è minore o uguale}). 
In altre parole, se il secondo operando è
maggiore o uguale al primo, il flusso di controllo sarà pasato all'indirizzo o alla label specificata nell'istruzione. 
Se questa condizione non è soddisfatta, poiché il secondo operando è più piccolo del primo, il flusso non viene alterato e la prima \printf sarà eseguita.

\myindex{x86!\Instructions!JNE}
Il secondo controllo è \JNE: \emph{Jump if Not Equal}.
Il flusso non cambia se i due operandi sono uguali.

\myindex{x86!\Instructions!JGE}
Il terzo controllo è \JGE: \emph{Jump if Greater or Equal}---salta se il primo operando è maggiore del secondo, o se sono uguali.
Quindi, se tutti i tre salti condizionali vengono innescati, nessuna delle chiamate a \printf sarà eseguita.
Ciò è chiaramente impossibile, almeno senza un intervento speciale.
Diamo ora un'occhiata alla funzione \TT{f\_unsigned()}.
La funzione \TT{f\_unsigned()} è uguale a \TT{f\_signed()}, con l'eccezione che le istruzioni \JBE e \JAE
sono utilizzate al posto di \JLE e \JGE:

\lstinputlisting[caption=GCC,style=customasmx86]{patterns/07_jcc/simple/unsigned_MSVC.asm}

\myindex{x86!\Instructions!JBE}
\myindex{x86!\Instructions!JAE}

Come già detto, le istruzioni di salto (branch instructions) sono diverse:
\JBE---\emph{Jump if Below or Equal} e \JAE---\emph{Jump if Above or Equal}.
Queste istruzioni (\INS{JA}/\JAE/\JB/\JBE) differiscono da \JG/\JGE/\JL/\JLE in quanto operano con numeri senza segno (unsigned).

\myindex{x86!\Instructions!JA}
\myindex{x86!\Instructions!JB}
\myindex{x86!\Instructions!JG}
\myindex{x86!\Instructions!JL}
\myindex{Signed numbers}

Questo è il motivo per cui se vediamo usare \JG/\JL al posto di \INS{JA}/\JB, o viceversa,
possiamo essere quasi certi che le variabili sono rispettivamente di tipo signed o unsigned.
Di seguito è riportata anche la funzione \main, dove non c'è niente di nuovo:

\lstinputlisting[caption=\main,style=customasmx86]{patterns/07_jcc/simple/main_MSVC.asm}

\clearpage
\myparagraph{\Optimizing MSVC + \olly}
\myindex{\olly}

Testiamo questo esempio (ottimizzato) in \olly.  Questa è la prima iterazione:

\begin{figure}[H]
\centering
\myincludegraphics{patterns/10_strings/1_strlen/olly1.png}
\caption{\olly: inizio prima iterazione}
\label{fig:strlen_olly_1}
\end{figure}

Notiamo che \olly ha trovato un ciclo e per convenienza, ha \emph{avvolto} le sue istruzioni dentro le parentesi.
Cliccando con il tasto destro su \EAX e scegliendo 
\q{Follow in Dump}, la finestra della memoria scorrerà fino al punto giusto.
Possiamo vedere la stringa \q{hello!} in memoria.
C'è almeno
uno zero byte dopo la stringa e poi spazzatura casuale.

Se \olly vede un registro contenente un indirizzo valido, che punta ad una stringa,  
lo mostra come stringa.

\clearpage
Premiamo F8 (\stepover) un paio di volte, per arrivare all' inizio del corpo del ciclo:

\begin{figure}[H]
\centering
\myincludegraphics{patterns/10_strings/1_strlen/olly2.png}
\caption{\olly: inizio seconda iterazione}
\label{fig:strlen_olly_2}
\end{figure}

Notiamo che \EAX contiene l'indirizzo del secondo carattere nella stringa.

\clearpage

Dobbiamo premere F8 un numero di volte sufficente per uscire dal ciclo:

\begin{figure}[H]
\centering
\myincludegraphics{patterns/10_strings/1_strlen/olly3.png}
\caption{\olly: differenza di puntatori da calcolare}
\label{fig:strlen_olly_3}
\end{figure}

Notiamo che ora \EAX contiene l'indirizzo dello zero byte che si trova subito dopo la stringa più 1 (perché INC EAX è stato eseguito indipendentemente dal fatto che siamo usciti o meno dal ciclo).
Nel frattempo, \EDX non è cambiato,
quindi sta ancora puntando all' inizio della stringa.

La differenza tra questi due indirizzi verrà calcolata ora.

\clearpage
L' istruzione \SUB è stata appena eseguita:

\begin{figure}[H]
\centering
\myincludegraphics{patterns/10_strings/1_strlen/olly4.png}
\caption{\olly: \EAX  sta venendo decrementato}
\label{fig:strlen_olly_4}
\end{figure}

La differenza tra i puntatori, in questo momento, si trova nel registro \EAX---7.
In realtà, la lunghezza della stringa \q{hello!} è 6, 
ma con lo zero byte incluso---7.
Ma \TT{strlen()} deve ritornare il numero di caratteri nella stringa, diversi da zero.
Quindi viene eseguito un decremento, dopodichè la funziona ritorna.


\clearpage
\myparagraph{x86 + MSVC + Hiew}
\myindex{Hiew}

Possiamo provare ad applicare una patch all' eseguibile in maniera tale che la funzione \TT{f\_unsigned()} stampi sempre \q{a==b}, 
a prescindere dai valori in input.

\begin{figure}[H]
\centering
\myincludegraphics{patterns/07_jcc/simple/hiew_unsigned1.png}
\caption{Hiew: funzione \TT{f\_unsigned()}}
\label{fig:jcc_hiew_1}
\end{figure}

Essenzialmente, per ottenere il risultato desiderato, dobbiamo:
\begin{itemize}
\item forzare il primo jump in modo che sia sempre seguito;
\item forzare il secondo jump a non essere mai seguito;
\item forzare il terzo jump ad essere sempre seguito.
\end{itemize}

Possiamo così diriggere il flusso di esecuzione in modo tale da farlo sempre passare attraverso la seconda \printf, dando in output \q{a==b}.

Devono essere corrette (patchate) tre istruzioni (o byte):

\begin{itemize}
\item Il primo jump diventa \JMP, ma il \gls{jump offset} resta invariato.

\item 
Il secondo jump potrebbe essere innescato in alcune occasioni, ma in ogni caso salterebbe alla prossima istruzione, poiché settiamo il \gls{jump offset} a 0.

In queste istruzioni il \gls{jump offset} viene sommato all'indirizzo della prossima istruzione.
Quindi se l'offset è 0, il jump trasferirà il controllo all'istruzione successiva,

\item 
Possiamo sostituire il terzo jump con \JMP allo stesso modo del primo, in modo che sia sempre innescato.

\end{itemize}

\clearpage
Ecco il codice modificato:

\begin{figure}[H]
\centering
\myincludegraphics{patterns/07_jcc/simple/hiew_unsigned2.png}
\caption{Hiew: funzione \TT{f\_unsigned()} modificata}
\label{fig:jcc_hiew_2}
\end{figure}

Se ci dimentichiamo di cambiare uno di questi jump, potrebbero essere eseguite diverse chiamate a \printf, ma noi vogliamo eseguirne solo una.

\myparagraph{\NonOptimizing GCC}

\myindex{puts() instead of printf()}
\NonOptimizing GCC 4.4.1 
produce pressoché lo stesso codice, ma usa \puts~(\myref{puts}) invece di \printf.

\myparagraph{\Optimizing GCC}

Un lettore attento potrebbe domandare: perchè eseguire \CMP più volte se i flag hanno gli stessi valori dopo ogni esecuzione?

Forse MSVC con ottimizzazioni non è in grado di applicare questa ottimizzazione, al contrario di GCC 4.8.1:

\lstinputlisting[caption=GCC 4.8.1 f\_signed(),style=customasmx86]{patterns/07_jcc/simple/GCC_O3_signed.asm}

% should be here instead of 'switch' section?
Notiamo anche l'uso di \TT{JMP puts} al posto di \TT{CALL puts / RETN}.
Questo trucco sarà spiegato più avanti: \myref{JMP_instead_of_RET}.

Questo tipo di codice x86 è piuttosto raro.
MSVC 2012 apparentemente non è in grado di generarne di simile.
Dall'altro lato, i programmatori assembly sanno perfettamente che le istruzioni \TT{Jcc} possono essere disposte in fila.

Se vedete codice con una disposizione simile, è molto probabile che sia stato scritto a mano.

La funzione \TT{f\_unsigned()} non è esteticamente corta allo stesso modo:

\lstinputlisting[caption=GCC 4.8.1 f\_unsigned(),style=customasmx86]{patterns/07_jcc/simple/GCC_O3_unsigned_IT.asm}

Ciò nonostante, ci sono due istruzioni \TT{CMP} invece di tre.
Gli algoritmi di ottimizzazione di GCC 4.8.1 probabilmente non sono ancora perfetti. 
