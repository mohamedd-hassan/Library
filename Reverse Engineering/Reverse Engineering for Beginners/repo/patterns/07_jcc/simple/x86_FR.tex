\subsubsection{x86}

\myparagraph{x86 + MSVC}

Voici à quoi ressemble la fonction  \TT{f\_signed()}:

\lstinputlisting[caption=MSVC 2010 \NonOptimizing,style=customasmx86]{patterns/07_jcc/simple/signed_MSVC.asm}

\myindex{x86!\Instructions!JLE}

La première instruction, \JLE, représente \emph{Jump if Less or Equal} (saut si inférieur ou égal).
En d'autres mots, si le deuxième opérande est plus grand ou égal au premier,
le flux d'exécution est passé à l'adresse ou au label spécifié dans l'instruction.
Si la condition ne déclenche pas le saut, car le second opérande est plus petit que
le premier, le flux d'exécution ne sera pas altéré et le premier \printf sera
exécuté.
\myindex{x86!\Instructions!JNE}
Le second test est \JNE: \emph{Jump if Not Equal} (saut si non égal).
Le flux d'exécution ne changera pas si les opérandes sont égaux.

\myindex{x86!\Instructions!JGE}
Le troisième test est \JGE: \emph{Jump if Greater or Equal}---saute si le premier
opérande est supérieur ou égal au deuxième.
Donc, si les trois sauts conditionnels sont effectués, aucun des appels à \printf
ne sera exécuté.
Ceci est impossible sans intervention spéciale.
Regardons maintenant la fonction \TT{f\_unsigned()}.
La fonction \TT{f\_unsigned()} est la même que \TT{f\_signed()}, à la différence
que les instructions \JBE et \JAE sont utilisées à la place de \JLE et \JGE, comme
suit:

\lstinputlisting[caption=GCC,style=customasmx86]{patterns/07_jcc/simple/unsigned_MSVC.asm}

\myindex{x86!\Instructions!JBE}
\myindex{x86!\Instructions!JAE}

Comme déjà mentionné, les instructions de branchement sont différentes:
\JBE---\emph{Jump if Below or Equal} (saut si inférieur ou égal) et \JAE---\emph{Jump if Above or Equal}
(saut si supérieur ou égal).
Ces instructions (\INS{JA}/\JAE/\JB/\JBE) diffèrent de \JG/\JGE/\JL/\JLE par le fait qu'elles
travaillent avec des nombres non signés.

\myindex{x86!\Instructions!JA}
\myindex{x86!\Instructions!JB}
\myindex{x86!\Instructions!JG}
\myindex{x86!\Instructions!JL}
\myindex{Signed numbers}

C'est pourquoi si nous voyons que \JG/\JL sont utilisés à la place de \INS{JA}/\JB ou
vice-versa, nous pouvons être presque sûr que les variables sont signées ou non
signées, respectivement.
Voici la fonction \main, où presque rien n'est nouveau pour nous:

\lstinputlisting[caption=\main,style=customasmx86]{patterns/07_jcc/simple/main_MSVC.asm}

\clearpage
\subsubsection{MSVC: x86 + \olly}

Essayons de hacker notre programme dans \olly, pour le forcer à penser que \scanf
fonctionne toujours sans erreur.
Lorsque l'adresse d'une variable locale est passée à \scanf, la variable contient
initiallement toujours des restes de données aléatoires, dans ce cas \TT{0x6E494714}:

\begin{figure}[H]
\centering
\myincludegraphics{patterns/04_scanf/3_checking_retval/olly_1.png}
\caption{\olly: passer l'adresse de la variable à \scanf}
\label{fig:scanf_ex3_olly_1}
\end{figure}

\clearpage
Lorsque \scanf s'exécute dans la console, entrons quelque chose qui n'est pas du
tout un nombre, comme \q{asdasd}.
\scanf termine avec 0 dans \EAX, ce qui indique qu'une erreur s'est produite.

Nous pouvons vérifier la variable locale dans le pile et noter qu'elle n'a pas changé.
En effet, qu'aurait écrit \scanf ici?
Elle n'a simplement rien fait à part renvoyer zéro.

Essayons de \q{hacker} notre programme.
Clique-droit sur \EAX,
parmi les options il y a \q{Set to 1} (mettre à 1).
C'est ce dont nous avons besoin.

Nous avons maintenant 1 dans \EAX, donc la vérification suivante va s'exécuter comme
souhaiter et \printf va afficher la valeur de la variable dans la pile.

Lorsque nous lançons le programme (F9) nous pouvons voir ceci dans la fenêtre
de la console:

\lstinputlisting[caption=fenêtre console]{patterns/04_scanf/3_checking_retval/console.txt}

En effet, 1850296084 est la représentation en décimal du nombre dans la pile (\TT{0x6E494714})!


\clearpage
\myparagraph{x86 + MSVC + Hiew}
\myindex{Hiew}

Nous pouvons essayer de patcher l'exécutable afin que la fonction \TT{f\_unsigned()}
affiche toujours \q{a==b}, quelque soient les valeurs en entrée.
Voici à quoi ça ressemble dans Hiew:

\begin{figure}[H]
\centering
\myincludegraphics{patterns/07_jcc/simple/hiew_unsigned1.png}
\caption{Hiew: fonction \TT{f\_unsigned()}}
\label{fig:jcc_hiew_1}
\end{figure}

Essentiellement, nous devons accomplir ces trois choses:
\begin{itemize}
\item forcer le premier saut à toujours être effectué;
\item forcer le second saut à n'être jamais effectué;
\item forcer le troisième saut à toujours être effectué.
\end{itemize}

Nous devons donc diriger le déroulement du code pour toujours effectuer le second \printf,
et afficher \q{a==b}.

Trois instructions (ou octets) doivent être modifiées:

\begin{itemize}
\item Le premier saut devient un \JMP, mais l'\glslink{jump offset}{offset} reste
le même.

\item
Le second saut peut être parfois suivi, mais dans chaque cas il sautera à l'instruction
suivante, car nous avons mis l'\glslink{jump offset}{offset} à 0.

Dans cette instruction, l'\glslink{jump offset}{offset} est ajouté à l'adresse
de l'instruction suivante. Donc si l'offset est 0, le saut va transférer l'exécution
à l'instruction suivante.

\item
Le troisième saut est remplacé par \JMP comme nous l'avons fait pour le premier,
il sera donc toujours effectué.

\end{itemize}

\clearpage
Voici le code modifié:

\begin{figure}[H]
\centering
\myincludegraphics{patterns/07_jcc/simple/hiew_unsigned2.png}
\caption{Hiew: modifions la fonction \TT{f\_unsigned()}}
\label{fig:jcc_hiew_2}
\end{figure}

Si nous oublions de modifier une de ces sauts conditionnels, plusieurs appels à \printf
pourraient être faits, alors que nous voulons qu'un seul soit exécuté.

\myparagraph{GCC \NonOptimizing}

\myindex{puts() instead of printf()}
GCC 4.4.1 \NonOptimizing produit presque le même code, mais avec \puts~(\myref{puts})
à la place de \printf.

\myparagraph{GCC \Optimizing}

Le lecteur attentif pourrait demander pourquoi exécuter \CMP plusieurs fois, si
les flags ont les mêmes valeurs après l'exécution ?

Peut-être que l'optimiseur de de MSVC ne peut pas faire cela, mais celui de GCC
4.8.1 peut aller plus loin:

\lstinputlisting[caption=GCC 4.8.1 f\_signed(),style=customasmx86]{patterns/07_jcc/simple/GCC_O3_signed.asm}

% should be here instead of 'switch' section?
Nous voyons ici \TT{JMP puts} au lieu de \TT{CALL puts / RETN}.

Ce genre de truc sera expliqué plus loin: \myref{JMP_instead_of_RET}.

Ce genre de code x86 est plutôt rare.
Il semble que MSVC 2012 ne puisse pas générer un tel code.
D'un autre côté, les programmeurs en langage d'assemblage sont parfaitement conscients
du fait que les instructions \TT{Jcc} peuvent être empilées.

Donc si vous voyez ce genre d'empilement, il est très probable que le code a été
écrit à la main.

La fonction \TT{f\_unsigned()} n'est pas si esthétiquement courte:

\lstinputlisting[caption=GCC 4.8.1 f\_unsigned(),style=customasmx86]{patterns/07_jcc/simple/GCC_O3_unsigned_FR.asm}

Néanmoins, il y a deux instructions \TT{CMP} au lieu de trois.

Donc les algorithmes d'optimisation de GCC 4.8.1 ne sont probablement pas encore parfaits.

