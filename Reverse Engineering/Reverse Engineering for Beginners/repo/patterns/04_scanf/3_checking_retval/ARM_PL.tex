\subsubsection{ARM}

\myparagraph{ARM: \OptimizingKeilVI (\ThumbMode)}

\lstinputlisting[caption=\OptimizingKeilVI (\ThumbMode),style=customasmARM]{patterns/04_scanf/3_checking_retval/ex3_ARM_Keil_thumb_O3.asm}

\myindex{ARM!\Instructions!CMP}
\myindex{ARM!\Instructions!BEQ}

Nowymi instrukcjami są \CMP i \ac{BEQ}.

\CMP jest równoważna instrukcji o takiej samej nazwie z x86. Odejmuje ona jeden argument od drugiego (bez zapisywania wyniku) i ustawia odpowiednie flagi procesora.
% TODO: в мануале ARM $op1 + NOT(op2) + 1$ вместо вычитания

\myindex{ARM!\Registers!Z}
\myindex{x86!\Instructions!JZ}
\ac{BEQ} skacze pod inny adres, jeśli wartość flagi Z jest równa 1. Taka sytuacja w naszym przykładzie zajdzie, jeśli wartość rejestru i liczby z operandów instrukcji \CMP będą sobie równe (tym samym ich różnica będzie równa 0).
Tak samo zachowuje się instrukcja \JZ na x86.

Cała reszta kodu jest bardzo prosta: przepływ sterowania dzieli się na dwa rozgałęzienia, które spotykają się w punkcie, gdzie 0 jest zapisywane do rejestru \Reg{0} jako wartość zwracana. Następnie funkcja się kończy.

\myparagraph{ARM64}

\lstinputlisting[caption=\NonOptimizing GCC 4.9.1 ARM64,numbers=left,style=customasmARM]{patterns/04_scanf/3_checking_retval/ARM64_GCC491_O0_PL.s}

\myindex{ARM!\Instructions!CMP}
\myindex{ARM!\Instructions!Bcc}
Przepływ sterowania w tym przypadku rozgałęzia się dzięki parze instrukcji \INS{CMP}/\INS{BNE} (Branch if Not Equal).

