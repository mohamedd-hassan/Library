\subsubsection{MSVC: x86}

\lstinputlisting[style=customasmx86]{patterns/04_scanf/2_global/ex2_MSVC.asm}

Nesse caso, a variável \TT{x} é definida no segmento \TT{\_DATA} e nenhuma memória é alocada na pilha local.
Ela é acessada diretamente, não através da pilha.
Variáveis globais não inicialiadas não ocupam espaço no arquivo executável 
(realmente, ninguém precisa alocar espaço para uma variável inicialmente valendo zero), 
mas quando alguém acessa o endereço delas, o sistema operacional vai alocar um bloco contendo somente zeros nele.
\footnote{\ac{TBT}: That is how a \ac{VM} behaves}.

Agora vamos definir um valor para a variável:

% TODO translate
\lstinputlisting[style=customc]{patterns/04_scanf/2_global/default_value_EN.c}

Nós temos:

\begin{lstlisting}[style=customasmx86]
_DATA	SEGMENT
_x	DD	0aH

...
\end{lstlisting}

Aqui nós vemos um valor \TT{0xA} do tipo DWORD (DD significa DWORD = 32 bits) para essa variável.

Se você abrir o .exe compilado no \IDA, você pode ver a variável \emph{x} colocada no começo do segmento \TT{\_DATA},
e depois disso você pode ver as strings.

Se você abrir o .exe compilado no exemplo anterior no \IDA, onde o valor de x não foi declarado, você poderá ver algo assim:

\lstinputlisting[caption=\IDA,style=customasmx86]{patterns/04_scanf/2_global/IDA.lst}

\label{BSSClearedByCStd}
\TT{\_x} está marcada com \TT{?} juntamente com o resto das variáveis que não precisam ser inicializadas.
Isso implica que após carregar o .exe para a memória, um espaço para todas essas variáveis será alocado e preenchido com zeros \InSqBrackets{\CNineNineStd 6.7.8p10}.
Mas no arquivo .exe essas variáveis não inicializadas não ocupam nenhum espaço.
Isso é conveniente para arrays grandes, por exemplo.

\EN{\clearpage
\subsubsection{MSVC + \olly}
\myindex{\olly}

Let's load our example into \olly and set a breakpoint on \comp.
We can see how the values are compared at the first \comp call:

\begin{figure}[H]
\centering
\myincludegraphics{patterns/18_pointers_to_functions/olly1.png}
\caption{\olly: first call of \comp}
\label{fig:qsort_olly1}
\end{figure}

\olly shows the compared values in the window under the code window, for convenience.
We can also see that the \ac{SP} points to \ac{RA}, where the \qsort function is (located in \TT{MSVCR100.DLL}).

\clearpage
By tracing (F8) until the \TT{RETN} instruction and pressing F8 one more time, we return to the \qsort function:

\begin{figure}[H]
\centering
\myincludegraphics{patterns/18_pointers_to_functions/olly2.png}
\caption{\olly: the code in \qsort right after \comp call}
\label{fig:qsort_olly2}
\end{figure}

That has been a call to the comparison function.

\clearpage
Here is also a screenshot of the moment of the second call of \comp{}---now values that have to be compared are different:

\begin{figure}[H]
\centering
\myincludegraphics{patterns/18_pointers_to_functions/olly3.png}
\caption{\olly: second call of \comp}
\label{fig:qsort_olly3}
\end{figure}
}
\RU{\clearpage
\myparagraph{\olly + упаковка полей по умолчанию}
\myindex{\olly}

Попробуем в \olly наш пример, где поля выровнены по умолчанию (4 байта):

\begin{figure}[H]
\centering
\myincludegraphics{patterns/15_structs/4_packing/olly_packing_4.png}
\caption{\olly: Перед исполнением \printf}
\label{fig:packing_olly_4}
\end{figure}

В окне данных видим наши четыре поля.
Вот только, откуда взялись случайные байты (0x30, 0x37, 0x01) рядом с первым (a) и третьим (c) полем?

Если вернетесь к листингу \myref{src:struct_packing_4}, то увидите, что первое и третье поле имеет
тип \Tchar, а следовательно, туда записывается только один байт, 1 и 3 соответственно (строки 6 и 8).

Остальные три байта 32-битного слова не будут модифицироваться в памяти!

А, следовательно, там остается случайный мусор.
\myindex{x86!\Instructions!MOVSX}
Этот мусор никак не будет влиять на работу \printf,
потому что значения для нее готовятся при помощи инструкции \MOVSX, которая загружает
из памяти байты а не слова: 
\lstref{src:struct_packing_4} (строки 34 и 38).

Кстати, здесь используется именно \MOVSX (расширяющая знак), потому что тип 
\Tchar --- знаковый по умолчанию в MSVC и GCC.

Если бы здесь был тип \TT{unsigned char} или \TT{uint8\_t}, 
то здесь была бы инструкция \MOVZX.

\clearpage
\myparagraph{\olly + упаковка полей по границе в 1 байт}
\myindex{\olly}

Здесь всё куда понятнее: 4 поля занимают 10 байт и значения сложены в памяти друг к другу

\begin{figure}[H]
\centering
\myincludegraphics{patterns/15_structs/4_packing/olly_packing_1.png}
\caption{\olly: Перед исполнением \printf}
\label{fig:packing_olly_1}
\end{figure}
}
\IT{\clearpage
\myparagraph{\Optimizing MSVC + \olly}
\myindex{\olly}

Testiamo questo esempio (ottimizzato) in \olly.  Questa è la prima iterazione:

\begin{figure}[H]
\centering
\myincludegraphics{patterns/10_strings/1_strlen/olly1.png}
\caption{\olly: inizio prima iterazione}
\label{fig:strlen_olly_1}
\end{figure}

Notiamo che \olly ha trovato un ciclo e per convenienza, ha \emph{avvolto} le sue istruzioni dentro le parentesi.
Cliccando con il tasto destro su \EAX e scegliendo 
\q{Follow in Dump}, la finestra della memoria scorrerà fino al punto giusto.
Possiamo vedere la stringa \q{hello!} in memoria.
C'è almeno
uno zero byte dopo la stringa e poi spazzatura casuale.

Se \olly vede un registro contenente un indirizzo valido, che punta ad una stringa,  
lo mostra come stringa.

\clearpage
Premiamo F8 (\stepover) un paio di volte, per arrivare all' inizio del corpo del ciclo:

\begin{figure}[H]
\centering
\myincludegraphics{patterns/10_strings/1_strlen/olly2.png}
\caption{\olly: inizio seconda iterazione}
\label{fig:strlen_olly_2}
\end{figure}

Notiamo che \EAX contiene l'indirizzo del secondo carattere nella stringa.

\clearpage

Dobbiamo premere F8 un numero di volte sufficente per uscire dal ciclo:

\begin{figure}[H]
\centering
\myincludegraphics{patterns/10_strings/1_strlen/olly3.png}
\caption{\olly: differenza di puntatori da calcolare}
\label{fig:strlen_olly_3}
\end{figure}

Notiamo che ora \EAX contiene l'indirizzo dello zero byte che si trova subito dopo la stringa più 1 (perché INC EAX è stato eseguito indipendentemente dal fatto che siamo usciti o meno dal ciclo).
Nel frattempo, \EDX non è cambiato,
quindi sta ancora puntando all' inizio della stringa.

La differenza tra questi due indirizzi verrà calcolata ora.

\clearpage
L' istruzione \SUB è stata appena eseguita:

\begin{figure}[H]
\centering
\myincludegraphics{patterns/10_strings/1_strlen/olly4.png}
\caption{\olly: \EAX  sta venendo decrementato}
\label{fig:strlen_olly_4}
\end{figure}

La differenza tra i puntatori, in questo momento, si trova nel registro \EAX---7.
In realtà, la lunghezza della stringa \q{hello!} è 6, 
ma con lo zero byte incluso---7.
Ma \TT{strlen()} deve ritornare il numero di caratteri nella stringa, diversi da zero.
Quindi viene eseguito un decremento, dopodichè la funziona ritorna.
}
\DE{\clearpage
\myparagraph{\olly + standardmäßig gepackte Felder}
\myindex{\olly}
Betrachten wir unser Beispiel (in dem die Felder standardmäßig auf 4 Byte angeordnet werden) in \olly:

\begin{figure}[H]
\centering
\myincludegraphics{patterns/15_structs/4_packing/olly_packing_4.png}
\caption{\olly: vor der Ausführung von \printf}
\label{fig:packing_olly_4}
\end{figure}
Wir sehen unsere 4 Felder im Datenfenster.

Wir fragen uns aber, woher die Zufallsbytes (0x30, 0x37, 0x01) stammen, die neben dem ersten ($a$) und dritten ($c$)
Feld liegen.

Betrachten wir unser Listing \myref{src:struct_packing_4}, erkennen wir, dass das erste und dritte Feld vom Typ \Tchar
ist, und daher nur ein Byte geschrieben wird, nämlich 1 bzw. 3 (Zeilen 6 und 8).

Die übrigen 3 Byte des 32-Bit-Wortes werden im Speicher nicht verändert!
Deshalb befinden sich hier zufällige Reste.

\myindex{x86!\Instructions!MOVSX}
Diese Reste beeinflussen den Output von \printf in keinster Weise, da die Werte für die Funktion mithilfe von \MOVSX
vorbereitet werden, der Bytes und nicht Worte als Argumente hat: 
\lstref{src:struct_packing_4} (Zeilen 34 und 38).
Der vorzeichenerweiternde Befehl \MOVSX wird hier übrigens verwendet, da \Tchar standardmäßig in MSVC und GCC
vorzeichenbehaftet ist.
Würde hier der Datentyp \TT{unsigned char} oder \TT{uint8\_t} verwendet, würde der Befehl \MOVZX stattdessen verwendet.

\clearpage
\myparagraph{\olly + Felder auf 1 Byte Grenzen angeordnet}
\myindex{\olly}
Hier sind die Dinge viel klarer ersichtlich: 4 Felder benötigen 16 Byte und die Werte werden nebeneinander gespeichert.

\begin{figure}[H]
\centering
\myincludegraphics{patterns/15_structs/4_packing/olly_packing_1.png}
\caption{\olly: Vor der Ausführung von \printf}
\label{fig:packing_olly_1}
\end{figure}
}
\FR{\clearpage
\subsubsection{MSVC: x86 + \olly}

Essayons de hacker notre programme dans \olly, pour le forcer à penser que \scanf
fonctionne toujours sans erreur.
Lorsque l'adresse d'une variable locale est passée à \scanf, la variable contient
initiallement toujours des restes de données aléatoires, dans ce cas \TT{0x6E494714}:

\begin{figure}[H]
\centering
\myincludegraphics{patterns/04_scanf/3_checking_retval/olly_1.png}
\caption{\olly: passer l'adresse de la variable à \scanf}
\label{fig:scanf_ex3_olly_1}
\end{figure}

\clearpage
Lorsque \scanf s'exécute dans la console, entrons quelque chose qui n'est pas du
tout un nombre, comme \q{asdasd}.
\scanf termine avec 0 dans \EAX, ce qui indique qu'une erreur s'est produite.

Nous pouvons vérifier la variable locale dans le pile et noter qu'elle n'a pas changé.
En effet, qu'aurait écrit \scanf ici?
Elle n'a simplement rien fait à part renvoyer zéro.

Essayons de \q{hacker} notre programme.
Clique-droit sur \EAX,
parmi les options il y a \q{Set to 1} (mettre à 1).
C'est ce dont nous avons besoin.

Nous avons maintenant 1 dans \EAX, donc la vérification suivante va s'exécuter comme
souhaiter et \printf va afficher la valeur de la variable dans la pile.

Lorsque nous lançons le programme (F9) nous pouvons voir ceci dans la fenêtre
de la console:

\lstinputlisting[caption=fenêtre console]{patterns/04_scanf/3_checking_retval/console.txt}

En effet, 1850296084 est la représentation en décimal du nombre dans la pile (\TT{0x6E494714})!
}
\JA{\clearpage
\myparagraph{x86 + MSVC + \olly}
\myindex{\olly}
\myindex{x86!\Registers!\Flags}

\olly でこの例を実行すると、フラグがどのように設定されているかを見ることができます。 
符号なしの数値で動作する\TT{f\_unsigned()}から始めましょう。

\CMP はここで3回実行されますが、同じ引数についてはフラグは毎回同じです。

最初の比較の結果は、

\begin{figure}[H]
\centering
\myincludegraphics{patterns/07_jcc/simple/olly_unsigned1.png}
\caption{\olly: \TT{f\_unsigned()}: 最初の条件付きジャンプ}
\label{fig:jcc_olly_unsigned_1}
\end{figure}

従って、フラグは、C=1、P=1、A=1、Z=0、S=1、T=0、D=0、O=0です。

これらは \olly では1文字の略号で命名されています。

\olly は、(\JBE)ジャンプがトリガーされることを示唆しています。 
実際に、インテルのマニュアル(\myref{x86_manuals})を調べると、
CF=1またはZF=1の場合、JBEが起動することがわかります。 
条件はここに当てはまるので、ジャンプが開始されます。

\clearpage
次の条件付きジャンプは、

\begin{figure}[H]
\centering
\myincludegraphics{patterns/07_jcc/simple/olly_unsigned2.png}
\caption{\olly: \TT{f\_unsigned()}: 2番目の条件付きジャンプ}
\label{fig:jcc_olly_unsigned_2}
\end{figure}

\olly は、 \JNZ がトリガーされることを示唆しています。 
実際、ZF=0(ゼロフラグ)の場合、JNZが起動します。

\clearpage
3番目の条件付きジャンプは、 \JNB です。

\begin{figure}[H]
\centering
\myincludegraphics{patterns/07_jcc/simple/olly_unsigned3.png}
\caption{\olly: \TT{f\_unsigned()}: 3番目の条件付きジャンプ}
\label{fig:jcc_olly_unsigned_3}
\end{figure}

インテルのマニュアル(\myref{x86_manuals})では、CF=0(キャリーフラグ)の場合に \JNB が起動することがわかります。 
今回は当てはまらないので、3番目の \printf が実行されます。

\clearpage
次に、\olly で、符号付きの値で動作する\TT{f\_signed()}関数を見てみましょう。 
フラグは、C=1、P=1、A=1、Z=0、S=1、T=0、D=0、O=0と同様に設定されます。 
最初の条件付きジャンプ \JLE が起動されます。

インテルマニュアル(166ページの7.1.4)では、ZF = 1またはSFxOFの場合にこの命令がトリガされることがわかりました。 SFxOF私たちの場合は、ジャンプがトリガするように。

\begin{figure}[H]
\centering
\myincludegraphics{patterns/07_jcc/simple/olly_signed1.png}
\caption{\olly: \TT{f\_signed()}: 最初の条件付きジャンプ}
\label{fig:jcc_olly_signed_1}
\end{figure}

インテルマニュアル(\myref{x86_manuals})では、ZF=1または SF$\neq$OF の場合にこの命令が起動されることがわかりました。 
私たちの場合では SF$\neq$OF が、ジャンプが起動されます。

\clearpage
2番目の \JNZ 条件付きジャンプはZF=0の場合(ゼロ・フラグ)に起動します。

\begin{figure}[H]
\centering
\myincludegraphics{patterns/07_jcc/simple/olly_signed2.png}
\caption{\olly: \TT{f\_signed()}: 2番目の条件付きジャンプ}
\label{fig:jcc_olly_signed_2}
\end{figure}

\clearpage
第3の条件付きジャンプ \JGE は、SF=OFの場合にのみ実行されるため、起動しません。今回は、当てはまりません。

\begin{figure}[H]
\centering
\myincludegraphics{patterns/07_jcc/simple/olly_signed3.png}
\caption{\olly: \TT{f\_signed()}: 3番目の条件付きジャンプ}
\label{fig:jcc_olly_signed_3}
\end{figure}
}
\PL{\clearpage
\subsubsection{MSVC: x86 + \olly}

Spróbujmy zhackować nasz program w \olly, zmuszając go, by uznał, że funkcja \scanf wykonała się bez błędów.
Kiedy adres zmiennej lokalnej jest przekazywany do \scanf,
zmienna początkowo zawiera przypadkową wartość, w tym wypadku \TT{0x6E494714}:

\begin{figure}[H]
\centering
\myincludegraphics{patterns/04_scanf/3_checking_retval/olly_1.png}
\caption{\olly: przekazywanie adresu zmiennej do \scanf}
\label{fig:scanf_ex3_olly_1}
\end{figure}

\clearpage
Kiedy wykonywana jest funkcja \scanf , w konsoli wpiszmy coś, co z pewnością nie jest liczbą, na przykład \q{asdasd}.
\scanf kończy działanie z 0 w \EAX, co wskazuje na wystąpienie błędu.

Możemy sprawdzić wartość zmiennej lokalnej na stosie i zauważyć, że się ona nie zmieniła.
W rzeczy samej, dlaczego funkcja \scanf miałaby cokolwiek tam zapisać?
Jej wykonanie nie spowodowało nic, poza zwróceniem zera.

Spróbujmy \q{zhackować} nasz program.
Kliknij prawym przyciskiem na \EAX,
wśród opcji znajduje się \q{Set to 1} (\emph{ustaw na 1}).
To jest to, czego szukamy.

Mamy teraz 1 w \EAX, a więc kolejne sprawdzenie powinno się wykonać zgodnie z oczekiwaniami i
\printf powinna wyświetlić wartość zmiennej ze stosu.

Po wznowieniu wykonania programu (F9) widzimy następujący efekt w oknie konsoli:

\lstinputlisting[caption=console window]{patterns/04_scanf/3_checking_retval/console.txt}

1850296084 to postać dziesiętna liczby, którą widzieliśmy na stosie (\TT{0x6E494714})!
}


\subsubsection{GCC: x86}

\PTBRph{}

\subsubsection{MSVC: x64}

% TODO translate
\lstinputlisting[caption=MSVC 2012 x64,style=customasmx86]{patterns/04_scanf/2_global/ex2_MSVC_x64_EN.asm}

O código é quase o mesmo que no x86.
Por favor, perceba que o endereço da variável x é passado para \TT{scanf()} usando uma instrução \LEA,
enquanto os valores das variáveis são passadas para o segundo \printf usando uma instrução \MOV.
\TT{DWORD PTR} é uma parte da linguagem assembly (sem relação com o código de máquina),
indicando que o tamanho da informação da variável é de 32-bits e que a instrução \MOV tem de ser codificada de acordo.

