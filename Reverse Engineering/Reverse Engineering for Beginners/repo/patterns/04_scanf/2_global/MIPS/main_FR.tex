\subsubsection{MIPS}

\myparagraph{Variable globale non initialisée}

Donc maintenant, la variable $x$ est globale.
Compilons en un exécutable plutôt qu'un fichier objet et chargeons-le dans \IDA.
IDA affiche la variable $x$ dans la section ELF \TT{.sbss} (vous vous rappelez du
\q{Pointeur Global}? \myref{MIPS_GP}),
puisque cette variable n'est pas initialisée au début.

\lstinputlisting[caption=GCC 4.4.5 \Optimizing (IDA),style=customasmMIPS]{patterns/04_scanf/2_global/MIPS/O3_IDA_FR.lst}

IDA réduit le volume des informations, donc nous allons générer un listing avec
objdump et le commenter:

\lstinputlisting[caption=GCC 4.4.5 \Optimizing (objdump),numbers=left,style=customasmMIPS]{patterns/04_scanf/2_global/MIPS/O3_objdump_FR.txt}

Nous voyons maintenant que l'adresse de la variable $x$ est lue depuis un buffer
de 64KiB en utilisant GP et en lui ajoutant un offset négatif (ligne 18).
Plus que ça, les adresses des trois fonctions externes qui sont utilisées dans
notre exemple (\puts, \scanf, \printf), sont aussi lues depuis le buffer de données
globale en utilisant GP (lignes 9, 16 et 26).
GP pointe sur le milieu du buffer, et de tels offsets suggèrent que les adresses
des trois fonctions, et aussi l'adresse de la variable $x$, sont toutes stockées
quelque part au début du buffer.
Cela fait du sens, car notre exemple est minuscule.

\myindex{MIPS!\Pseudoinstructions!MOVE}
\myindex{MIPS!\Pseudoinstructions!NOP}

Une autre chose qui mérite d'être mentionnée est que la fonction se termine avec
deux \ac{NOP}s (\TT{MOVE \$AT,\$AT} --- une instruction sans effet), afin d'aligner
le début de la fonction suivante sur un bloc de 16-octet.

\myparagraph{Variable globale initialisée}

Modifions notre exemple en affectant une valeur par défaut à la variable $x$:

\lstinputlisting[style=customc]{patterns/04_scanf/2_global/default_value_FR.c}

Maintenant IDA montre que la variable $x$ se trouve dans la section .data:

\lstinputlisting[caption=GCC 4.4.5 \Optimizing (IDA),style=customasmMIPS]{patterns/04_scanf/2_global/MIPS/O3_IDA_init_FR.lst}

Pourquoi pas .sdata? Peut-être que cela dépend d'une option de GCC?

Néanmoins, maintenant $x$ est dans .data, qui une zone mémoire générale, et nous
pouvons regarder comment y travailler avec des variables.

\myindex{MIPS!\Instructions!LUI}
\myindex{MIPS!\Instructions!ADDIU}

L'adresse de la variable doit être formée en utilisant une paire d'instructions.

Dans notre cas, ce sont \INS{LUI} (\q{Load Upper Immediate}) et \INS{ADDIU} (\q{Add Immediate Unsigned Word}).

Voici le listing d'objdump pour y regarder de plus près:

\lstinputlisting[caption=GCC 4.4.5 \Optimizing (objdump),style=customasmMIPS]{patterns/04_scanf/2_global/MIPS/O3_objdump_init_FR.txt}

\myindex{MIPS!\Instructions!LUI}
\myindex{MIPS!\Instructions!ADDIU}
\myindex{MIPS!\Instructions!LW}

Nous voyons que l'adresse est formée en utilisant \INS{LUI} et \INS{ADDIU}, mais
la partie haute de l'adresse est toujours dans le registre \$S0, et il est possible
d'encoder l'offset en une instruction \INS{LW} (\q{Load Word}), donc une seule instruction
\INS{LW} est suffisante pour charger une valeur de la variable et la passer à \printf.

Les registres contenant des données temporaires sont préfixés avec T-, mais ici
nous en voyons aussi qui sont préfixés par S-, leur contenu doit être doit être
sauvegardé quelque part avant de les utiliser dans d'autres fonctions.
% FIXME:
% This needs to be clarified a bit, e.g. "the registers need to be preserved if a function is called and it wants to use them

C'est pourquoi la valeur de \$S0 a été mise à l'adresse 0x4006cc et utilisée
de nouveau à l'adresse 0x4006e8, après l'appel de \scanf.
La fonction \scanf ne change pas cette valeur.

% TODO non-optimized example?
