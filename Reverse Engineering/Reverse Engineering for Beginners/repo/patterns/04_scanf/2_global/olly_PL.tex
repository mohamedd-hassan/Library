\clearpage
\subsubsection{MSVC: x86 + \olly}
\myindex{\olly}

Teraz jest nawet prościej:

\begin{figure}[H]
\centering
\myincludegraphics{patterns/04_scanf/2_global/ex2_olly_1.png}
\caption{\olly: po wykonaniu \scanf}
\label{fig:scanf_ex2_olly_1}
\end{figure}

Zmienna jest umieszczona w segmencie danych.
Po wykonaniu instrukcji \PUSH (odłożenie adresu $x$),
adres pojawia się w oknie stosu. Kliknij prawym przyciskiem na ten wiersz i wybierz \q{Follow in dump}.
Zmienna pojawi się w oknie pamięci, po lewej stroniej.
Po wprowadzeniu 123 w konsoli,
\TT{0x7B} pojawi się w oknie pamięci (patrz miejsca zaznaczane na zrzucie ekranu).

Ale dlaczego pierwszy bajt to \TT{7B}?
Przecież logicznie rzecz biorąc, powinno być \TT{00 00 00 7B}!
Przyczyną jest tzw. \glslink{endianness}{kolejność bajtów (ang. \emph{endianess})}, a x86 używa kolejności od najmniej znaczącego bajtu (tzw. \emph{cienkokońcowość}, ang. \emph{little-endian}).
Więcej o tym przeczytasz tutaj:: \myref{sec:endianness}.
Wracając do przykładu, wartość 32-bitowa jest ładowana z adresu w pamięci do \EAX a następnie przekazywana do funkcji \printf.

$x$ znajduje się pod adresem \TT{0x00C53394}.

\clearpage
W \olly można sprawdzić mapę pamięci procesu (Alt-M).
Widzimy, że ten adres znajduje się w segmencie PE \TT{.data} programu.:

\label{olly_memory_map_example}
\begin{figure}[H]
\centering
\myincludegraphics{patterns/04_scanf/2_global/ex2_olly_2.png}
\caption{\olly: mapa pamięci procesu}
\label{fig:scanf_ex2_olly_2}
\end{figure}

