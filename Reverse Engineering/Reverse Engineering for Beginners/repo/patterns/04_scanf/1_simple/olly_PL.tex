\clearpage
\subsubsection{MSVC + \olly}
\myindex{\olly}

Otwórzmy przykład w \olly.
Po załadowaniu wciskamy kilka razy F8, aż dotrzemy do naszego pliku wykonywalnego, zamiast \TT{ntdll.dll}.
Scrollujemy na górę, aż pojawi się funkcja \main.

Kliknij na pierwszą instrukcję (\TT{PUSH EBP}) i naciśnij F2 (\emph{ustaw breakpoint}), a następnie F9 (\emph{Run}).
Zatrzymamy się na początku funkcji main.

Przejdźmy do miejsca, w którym wyliczany jest adres zmiennej $x$:

\begin{figure}[H]
\centering
\myincludegraphics{patterns/04_scanf/1_simple/ex1_olly_1.png}
\caption{\olly: wyliczanie adresu zmiennej lokalnej}
\label{fig:scanf_ex1_olly_1}
\end{figure}

Kliknij prawym przyciskiem na rejestr \EAX w oknie z rejestrami i wybierz \q{Follow in stack}.

Adres z \EAX pojawi się w oknie z widokiem stosu.
Czerwona strzałka pokazuje na zmienną lokalną na stosie.
W tej chwili są tam śmieci ~--- (\TT{0x6E494714}).
Za pomocą instrukcji \PUSH adres tego elementu na stosie również trafi na stos, jako kolejny element.
Wciskając F8, przejdźmy za wywołanie funkcji \scanf. W trakcie wykonywania funkcji musimy podać jakiś wejściowy ciąg znaków w oknie konsoli, np. \q{123}.

\lstinputlisting{patterns/04_scanf/1_simple/console.txt}

\clearpage
Funkcja \scanf zakończyła swoje wykonanie.

\begin{figure}[H]
\centering
\myincludegraphics{patterns/04_scanf/1_simple/ex1_olly_3.png}
\caption{\olly: stan po zakończeniu funkcji \scanf}
\label{fig:scanf_ex1_olly_3}
\end{figure}

Funkcja \scanf zwróciła 1 w \EAX, co oznacza, że wczytała jedną wartość.
Jeśli ponownie spojrzymy na element na stosie odpowiadający zmiennej lokalnej, zobaczymy, że ma on teraz wartość \TT{0x7B} (123).

\clearpage

Później wartość zostanie skopiowana ze stosu do rejestru \ECX i przekazana do funkcji \printf:

\begin{figure}[H]
\centering
\myincludegraphics{patterns/04_scanf/1_simple/ex1_olly_4.png}
\caption{\olly: przygotowanie argumentu funkcji \printf}
\label{fig:scanf_ex1_olly_4}
\end{figure}
