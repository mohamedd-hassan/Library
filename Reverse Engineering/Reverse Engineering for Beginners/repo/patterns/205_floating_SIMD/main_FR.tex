% FIXME1 divide this file into separate ones...
\mysection{Travailler avec des nombres à virgule flottante en utilisant SIMD}

\label{floating_SIMD}
\myindex{IEEE 754}
\myindex{SIMD}
\myindex{SSE}
\myindex{SSE2}

Bien sûr. le \ac{FPU} est resté dans les processeurs compatible x86 lorsque les extensions
\ac{SIMD} ont été ajoutées.

L'extension \ac{SIMD} (SSE2) offre un moyen facile de travailler avec des nombres
à virgule flottante.

Le format des nombres reste le même (IEEE 754).

\myindex{x86-64}
Donc, les compilateurs modernes (incluant ceux générant pour x86-64) utilisent les
instructions \ac{SIMD} au lieu de celles pour FPU.

On peut dire que c'est une bonne nouvelle, car il est plus facile de travailler avec
elles.

Nous allons ré-utiliser les exemples de la section FPU ici: \myref{sec:FPU}.

\subsection{Simple exemple}

\lstinputlisting[style=customc]{patterns/12_FPU/1_simple/simple.c}

\subsubsection{x64}

\lstinputlisting[caption=MSVC 2012 x64 \Optimizing,style=customasmx86]{patterns/205_floating_SIMD/simple_MSVC_2012_x64_Ox.asm}

Les valeurs en virgule flottante entrées sont passées dans les registres \XMM{0}-\XMM{3},
tout le reste---via la pile
\footnote{\href{http://msdn.microsoft.com/en-us/library/zthk2dkh.aspx}{MSDN: Parameter Passing}}.

$a$ est passé dans \XMM{0}, $b$---via \XMM{1}.

Les registres XMM font 128-bit (comme nous le savons depuis la section à propos de
\ac{SIMD}: \myref{SIMD_x86}), mais les valeurs \Tdouble font 64-bit, donc seulement
la moitié basse du registre est utilisée.

\myindex{x86!\Instructions!DIVSD}
\TT{DIVSD} est une instruction SSE qui signifie \q{Divide Scalar Double-Precision
Floating-Point Values} (Diviser des nombres flottants double-précision), elle divise
une valeur de type \Tdouble par une autre, stockées dans la moitié basse des opérandes.

Les constantes sont encodées par le compilateur au format IEEE 754.

\myindex{x86!\Instructions!MULSD}
\myindex{x86!\Instructions!ADDSD}
\TT{MULSD} et \TT{ADDSD} fonctionnent de même, mais font la multiplication et l'addition.

Le résultat de l'exécution de la fonction de type \Tdouble est laissé dans le registre
\XMM{0}.\\
\\
C'est ainsi que travaille MSVC sans optimisation:

\lstinputlisting[caption=MSVC 2012 x64,style=customasmx86]{patterns/205_floating_SIMD/simple_MSVC_2012_x64.asm}

\myindex{Shadow space}
Légèrement redondant.
Les arguments en entrée sont sauvés dans le \q{shadow space} (\myref{shadow_space}),
mais seule leur moitié inférieure, i.e., seulement la valeur 64-bit de type \Tdouble.
GCC produit le même code.

\subsubsection{x86}

Compilons cet exemple pour x86. Bien qu'il compile pour x86, MSVC 2012 utilise des
instructions SSE2:

\lstinputlisting[caption=MSVC 2012 x86 \NonOptimizing,style=customasmx86]{patterns/205_floating_SIMD/simple_MSVC_2012_x86.asm}

\lstinputlisting[caption=MSVC 2012 x86 \Optimizing,style=customasmx86]{patterns/205_floating_SIMD/simple_MSVC_2012_x86_Ox.asm}

C'est presque le même code, toutefois, il y a quelques différences relatives aux
conventions d'appel:
1) les arguments ne sont pas passés dans des registres XMM, mais par la pile, comme
dans les exemples FPU (\myref{sec:FPU});
2) le résultat de la fonction est renvoyé dans \ST{0} --- afin de faire cela, il
est copié (à travers la variable locale \TT{tv}) depuis un des registres XMM dans
\ST{0}.

\clearpage
Essayons l'exemple optimisé dans \olly:

\begin{figure}[H]
\centering
\myincludegraphics{patterns/205_floating_SIMD/simple_olly1.png}
\caption{\olly: \TT{MOVSD} charge la valeur de $a$ dans \XMM{1}}
\label{fig:FPU_SIMD_simple_olly1}
\end{figure}

\clearpage
\begin{figure}[H]
\centering
\myincludegraphics{patterns/205_floating_SIMD/simple_olly2.png}
\caption{\olly: \TT{DIVSD} a calculé le \gls{quotient} 
et l'a stocké dans \XMM{1}}
\label{fig:FPU_SIMD_simple_olly2}
\end{figure}

\clearpage
\begin{figure}[H]
\centering
\myincludegraphics{patterns/205_floating_SIMD/simple_olly3.png}
\caption{\olly: \TT{MULSD} a calculé le \glslink{product}{produit} et l'a stocké
dans \XMM{0}}
\label{fig:FPU_SIMD_simple_olly3}
\end{figure}

\clearpage
\begin{figure}[H]
\centering
\myincludegraphics{patterns/205_floating_SIMD/simple_olly4.png}
\caption{\olly: \TT{ADDSD} ajoute la valeur dans \XMM{0} à celle dans \XMM{1}}
\label{fig:FPU_SIMD_simple_olly4}
\end{figure}

\clearpage
\begin{figure}[H]
\centering
\myincludegraphics{patterns/205_floating_SIMD/simple_olly5.png}
\caption{\olly: \FLD laisse le résultat de la fonction dans \ST{0}}
\label{fig:FPU_SIMD_simple_olly5}
\end{figure}

Nous voyons qu'\olly montre les registres XMM comme des paires de nombres \Tdouble,
mais seule la partie \emph{basse} est utilisée.

Apparemment, \olly les montre dans ce format car les instructions SSE2 (suffixées
avec \TT{-SD}) sont exécutées actuellement.

Mais bien sûr, il est possible de changer le format du registre et de voir le contenu
comme 4 nombres \Tfloat{} ou juste comme 16 octets.

\clearpage
\subsection{Passer des nombres à virgule flottante via les arguments}

\lstinputlisting[style=customc]{patterns/12_FPU/2_passing_floats/pow.c}

Ils sont passés dans la moitié basse des registres \XMM{0}-\XMM{3}.

\lstinputlisting[caption=MSVC 2012 x64 \Optimizing,style=customasmx86]{patterns/205_floating_SIMD/pow_MSVC_2012_x64_Ox.asm}

\myindex{x86!\Instructions!MOVSD}
\myindex{x86!\Instructions!MOVSDX}
Il n'y a pas d'instruction \TT{MOVSDX} dans les manuels Intel et AMD (\myref{x86_manuals}),
elle y est appelée \TT{MOVSD}.
Donc il y a deux instructions qui partagent le même nom en x86 (à propos de l'autre
lire: \myref{REP_MOVSx}).
Apparemment, les développeurs de Microsoft voulaient arrêter cette pagaille, donc
ils l'ont renommée \TT{MOVSDX}.
Elle charge simplement une valeur dans la moitié inférieure d'un registre XMM.

\TT{pow()} prends ses arguments de \XMM{0} et \XMM{1}, et renvoie le résultat dans
\XMM{0}.
Il est ensuite déplacé dans \RDX pour \printf.
Pourquoi?
Peut-être parce que \printf{}---est une fonction avec un nombre variable d'arguments?

\lstinputlisting[caption=GCC 4.4.6 x64 \Optimizing,style=customasmx86]{patterns/205_floating_SIMD/pow_GCC446_x64_O3_FR.s}

GCC génère une sortie plus claire.
La valeur pour \printf est passée dans \XMM{0}.
À propos, il y a un cas lorsque 1 est écrit dans \EAX pour \printf ---ceci implique
qu'un argument sera passé dans des registres vectoriels, comme le requiert le standard
\SysVABI.

\subsection{Exemple de comparaison}

\lstinputlisting[style=customc]{patterns/12_FPU/3_comparison/d_max.c}

\subsubsection{x64}

\lstinputlisting[caption=MSVC 2012 x64 \Optimizing,style=customasmx86]{patterns/205_floating_SIMD/d_max_MSVC_2012_x64_Ox.asm}

MSVC \Optimizing génère un code très facile à comprendre.

\myindex{x86!\Instructions!COMISD}
\TT{COMISD} is \q{Compare Scalar Ordered Double-Precision Floating-Point Values and
Set EFLAGS} (comparer des valeurs double précision en virgule flottante scalaire
ordrées et mettre les EFLAGS). Pratiquement, c'est ce qu'elle fait.\\
\\
MSVC \NonOptimizing génère plus de code redondant, mais il n'est toujours pas très
difficile à comprendre:

\lstinputlisting[caption=MSVC 2012 x64,style=customasmx86]{patterns/205_floating_SIMD/d_max_MSVC_2012_x64.asm}

\myindex{x86!\Instructions!MAXSD}
Toutefois, GCC 4.4.6 effectue plus d'optimisations et utilise l'instruction \TT{MAXSD}
(\q{Return Maximum Scalar Double-Precision Floating-Point Value}) qui choisit la
valeur maximum!

\lstinputlisting[caption=GCC 4.4.6 x64 \Optimizing,style=customasmx86]{patterns/205_floating_SIMD/d_max_GCC446_x64_O3.s}

\clearpage
\subsubsection{x86}

Compilons cet exemple dans MSVC 2012 avec l'optimisation activée:

\lstinputlisting[caption=MSVC 2012 x86 \Optimizing,style=customasmx86]{patterns/205_floating_SIMD/d_max_MSVC_2012_x86_Ox.asm}

Presque la même chose, mais les valeurs de $a$ et $b$ sont prises depuis la pile
et le résultat de la fonction est laissé dans \ST{0}.

Si nous chargeons cet exemple dans \olly, nous pouvons voir comment l'instruction
\TT{COMISD} compare les valeurs et met/efface les flags \CF et \PF:

\begin{figure}[H]
\centering
\myincludegraphics{patterns/205_floating_SIMD/d_max_olly.png}
\caption{\olly: \TT{COMISD} a changé les flags \CF et \PF}
\label{fig:FPU_SIMD_d_max_olly}
\end{figure}

\subsection{Calcul de l'epsilon de la machine: x64 et SIMD}
\label{machine_epsilon_x64_and_SIMD}

Revoyons l'exemple \q{calcul de l'epsilon de la machine} pour \Tdouble\ \lstref{machine_epsilon_double_c}.

Maintenant nous compilons pour x64:

\lstinputlisting[caption=MSVC 2012 x64 \Optimizing,style=customasmx86]{patterns/205_floating_SIMD/epsilon_double_MSVC_2012_x64_Ox.asm}

Il n'y a pas moyen d'ajouter 1 à une valeur dans un registre XMM 128-bit, donc il
doit être placé en mémoire.

Il y a toutefois l'instruction \INS{ADDSD} (\emph{Add Scalar Double-Precision Floating-Point
Values} ajouter des valeurs scalaires à virgule flottante double-précision), qui
peut ajouter une valeur dans la moitié 64-bit basse d'un registre XMM en ignorant
celle du haut, mais MSVC 2012 n'est probablement pas encore assez bon \footnote{À
titre d'exercice, vous pouvez retravailler ce code pour éliminer l'usage de la pile
locale}.

Néanmoins, la valeur est ensuite rechargée dans un registre XMM et la soustraction
est effectuée.
\INS{SUBSD} est \q{Subtract Scalar Double-Precision Floating-Point Values} (soustraire
des valeurs en virgule flottante double-précision), i.e., elle opère sur la partie
64-bit basse d'un registre XMM 128-bit.
Le résultat est renvoyé dans le registre XMM0.

\subsection{Exemple \#2: SCO OpenServer}

\label{examples_SCO}
\myindex{SCO OpenServer}
Un ancien logiciel pour SCO OpenServer de 1997 développé par une société qui a disparue
depuis longtemps.

Il y a un driver de dongle special à installer dans le système, qui contient les
chaînes de texte suivantes:
\q{Copyright 1989, Rainbow Technologies, Inc., Irvine, CA}
et
\q{Sentinel Integrated Driver Ver. 3.0 }.

Après l'installation du driver dans SCO OpenServer, ces fichiers apparaissent dans
l'arborescence /dev:

\begin{lstlisting}
/dev/rbsl8
/dev/rbsl9
/dev/rbsl10
\end{lstlisting}

Le programme renvoie une erreur lorsque le dongle n'est pas connecté, mais le message
d'erreur n'est pas trouvé dans les exécutables.

\myindex{COFF}

Grâce à \ac{IDA}, il est facile de charger l'exécutable COFF utilisé dans SCO OpenServer.

Essayons de trouver la chaîne \q{rbsl} et en effet, elle se trouve dans ce morceau
de code:

\lstinputlisting[style=customasmx86]{examples/dongles/2/1.lst}

Oui, en effet, le programme doit communiquer d'une façon ou d'une autre avec le driver.

\myindex{thunk-functions}
Le seul endroit où la fonction \TT{SSQC()} est appelée est dans la \glslink{thunk
 function}{fonction thunk}:

\lstinputlisting[style=customasmx86]{examples/dongles/2/2.lst}

SSQ() peut être appelé depuis au moins 2 fonctions.

L'une d'entre elles est:

\lstinputlisting[style=customasmx86]{examples/dongles/2/check1_EN.lst}

\q{\TT{3C}} et \q{\TT{3E}} semblent familiers: il y avait un dongle Sentinel Pro de
Rainbow sans mémoire, fournissant seulement une fonction de crypto-hachage secrète.

Vous pouvez lire une courte description de la fonction de hachage dont il s'agit
ici: \myref{hash_func}.

Mais retournons au programme.

Donc le programme peut seulement tester si un dongle est connecté ou s'il est absent.

Aucune autre information ne peut être écrite dans un tel dongle, puisqu'il n'a pas
de mémoire.
Les codes sur deux caractères sont des commandes (nous pouvons voir comment les commandes
sont traitées dans la fonction \TT{SSQC()}) et toutes les autres chaînes sont hachées
dans le dongle, transformées en un nombre 16-bit.
L'algorithme était secret, donc il n'était pas possible d'écrire un driver de remplacement
ou de refaire un dongle matériel qui l'émulerait parfaitement.

Toutefois, il est toujours possible d'intercepter tous les accès au dongle et de
trouver les constantes auxquelles les résultats de la fonction de hachage sont comparées.

Mais nous devons dire qu'il est possible de construire un schéma de logiciel de protection
de copie robuste basé sur une fonction secrète de hachage cryptographique: il suffit
qu'elle chiffre/déchiffre les fichiers de données utilisés par votre logiciel.

Mais retournons au code:

Les codes 51/52/53 sont utilisés pour choisir le port imprimante LPT.
3x/4x sont utilisés pour le choix de la \q{famille} (c'est ainsi que les dongles
Sentinel Pro sont différenciés les uns des autres: plus d'un dongle peut être connecté
sur un port LPT).

La seule chaîne passée à la fonction qui ne fasse pas 2 caractères est "0123456789".

Ensuite, le résultat est comparé à l'ensemble des résultats valides.

Si il est correct, 0xC ou 0xB est écrit dans la variable globale \TT{ctl\_model}.%

Une autre chaîne de texte qui est passée est
"PRESS ANY KEY TO CONTINUE: ", mais le résultat n'est pas testé.
Difficile de dire pourquoi, probablement une erreur\footnote{C'est un sentiment
étrange de trouver un bug dans un logiciel aussi ancien.}.

Voyons où la valeur de la variable globale \TT{ctl\_model} est utilisée.

Un tel endroit est:

\lstinputlisting[style=customasmx86]{examples/dongles/2/4.lst}

Si c'est 0, un message d'erreur chiffré est passé à une routine de déchiffrement
et affiché.

\myindex{x86!\Instructions!XOR}

La routine de déchiffrement de la chaîne semble être un simple \glslink{xoring}{xor}:

\lstinputlisting[style=customasmx86]{examples/dongles/2/err_warn.lst}

C'est pourquoi nous étions incapable de trouver le message d'erreur dans les fichiers
exécutable, car ils sont chiffrés (ce qui est une pratique courante).

Un autre appel à la fonction de hachage \TT{SSQ()} lui passe la chaîne \q{offln}
et le résultat est comparé avec \TT{0xFE81} et \TT{0x12A9}.

Si ils ne correspondent pas, ça se comporte comme une sorte de fonction \TT{timer()}
(peut-être en attente qu'un dongle mal connecté soit reconnecté et re-testé?) et ensuite
déchiffre un autre message d'erreur à afficher.

\lstinputlisting[style=customasmx86]{examples/dongles/2/check2_EN.lst}

Passer outre le dongle est assez facile: il suffit de patcher tous les sauts après
les instructions \CMP pertinentes.

Une autre option est d'écrire notre propre driver SCO OpenServer, contenant une table
de questions et de réponses, toutes celles qui sont présentent dans le programme.

\subsubsection{Déchiffrer les messages d'erreur}

À propos, nous pouvons aussi essayer de déchiffrer tous les messages d'erreurs.
L'algorithme qui se trouve dans la fonction \TT{err\_warn()} est très simple, en effet:

\lstinputlisting[caption=Decryption function,style=customasmx86]{examples/dongles/2/decrypting_FR.lst}

Comme on le voit, non seulement la chaîne est transmise à la fonction de déchiffrement
mais aussi la clef:

\lstinputlisting[style=customasmx86]{examples/dongles/2/tmp1_EN.asm}

L'algorithme est un simple \glslink{xoring}{xor}: chaque octet est xoré avec la clef, mais
la clef est incrémentée de 3 après le traitement de chaque octet.

Nous pouvons écrire un petit script Python pour vérifier notre hypothèse:

\lstinputlisting[caption=Python 3.x]{examples/dongles/2/decr1.py}

Et il affiche: \q{check security device connection}.
Donc oui, ceci est le message déchiffré.

Il y a d'autres messages chiffrés, avec leur clef correspondante.
Mais inutile de dire qu'il est possible de les déchiffrer sans leur clef.
Premièrement, nous voyons que le clef est en fait un octet.
C'est parce que l'instruction principale de déchiffrement (\XOR) fonctionne au niveau
de l'octet.
La clef se trouve dans le registre \ESI, mais seulement une partie de \ESI d'un octet
est utilisée.
Ainsi, une clef pourrait être plus grande que 255, mais sa valeur est toujours arrondie.

En conséquence, nous pouvons simplement essayer de brute-forcer, en essayant toutes
les clefs possible dans l'intervalle 0..255.
Nous allons aussi écarter les messages comportants des caractères non-imprimable.

\lstinputlisting[caption=Python 3.x]{examples/dongles/2/decr2.py}

Et nous obtenons:

\lstinputlisting[caption=Results]{examples/dongles/2/decr2_result.txt}

Ici il y a un peu de déchet, mais nous pouvons rapidement trouver les messages en
anglais.

À propos, puisque l'algorithme est un simple chiffrement xor, la même fonction peut
être utilisée pour chiffrer les messages.
Si besoin, nous pouvons chiffrer nos propres messages, et patcher le programme en les insérant.


\subsection{Résumé}

Seule la moitié basse des registres XMM est utilisée dans tous les exemples ici,
pour stocker un nombre au format IEEE 754.

Pratiquement, toutes les instructions préfixées par \TT{-SD} (\q{Scalar Double-Precision})---sont
des instructions travaillant avec des nombres à virgule flottante au format IEEE
754, stockés dans la moitié 64-bit basse d'un registre XMM.

Et c'est plus facile que dans le FPU, sans doute parce que les extensions SIMD ont
évolué dans un chemin moins chaotique que celles FPU dans le passé.
Le modèle de pile de registre n'est pas utilisé.

\myindex{x86!\Instructions!ADDSS}
\myindex{x86!\Instructions!MOVSS}
\myindex{x86!\Instructions!COMISS}
% TODO1: do this!
Si vous voulez, essayez de remplacer \Tdouble avec \Tfloat

% FIXME1 ... but their -SS versions
dans ces exemples, la même instruction sera utilisée, mais préfixée avec \TT{-SS}
(\q{Scalar Single-Precision} scalaire simple-précision), par exemple, \TT{MOVSS},
\TT{COMISS}, \TT{ADDSS}, etc.

\q{Scalaire} 
implique que le registre SIMD contienne seulement une valeur au lieu de plusieurs.

Les instructions travaillant avec plusieurs valeurs dans un registre simultanément
ont \q{Packed} dans leur nom.

Inutile de dire que les instructions SSE2 travaillent avec des nombres 64-bit au
format IEEE 754 (\Tdouble), alors que la représentation interne des nombres à virgule
flottante dans le FPU est sur 80-bit.

C'est pourquoi la FPU produit moins d'erreur d'arrondi et par conséquent, le FPU
peut donner des résultats de calcul plus précis.
