\myparagraph{MSVC 2010 \Optimizing}

\lstinputlisting[caption=MSVC 2010 \Optimizing,style=customasmx86]{patterns/12_FPU/3_comparison/x86/MSVC_Ox/MSVC_FR.asm}

\myindex{x86!\Instructions!FCOM}

\FCOM diffère de \FCOMP dans le sens où il compare seulement les deux valeurs, et
ne change pas la pile du FPU.
Contrairement à l'exemple précédent, ici les opérandes sont dans l'ordre inverse,
c'est pourquoi le résultat de la comparaison dans \CThreeBits est différent.

\begin{itemize}
\item si $a>b$ dans notre exemple, alors les bits \CThreeBits sont mis comme suit: 0, 0, 0.
\item si $b>a$, alors les bits sont: 0, 0, 1.
\item si $a=b$, alors les bits sont: 1, 0, 0.
\end{itemize}
% TODO: table?

L'instruction \INS{test ah, 65} laisse seulement deux bits~---\Cthree et \Czero.
Les deux seront à zéro si $a>b$: dans ce cas le saut \JNE ne sera pas effectué.
Puis \INS{FSTP ST(1)} suit~---cette instruction copie la valeur de \ST{0} dans l'opérande
et supprime une valeur de la pile du FPU.
En d'autres mots, l'instruction copie \ST{0} (où la valeur de \GTT{\_a} se trouve)
dans \ST{1}.
Après cela, deux copies de {\_a} sont sur le sommet de la pile.
Puis, une valeur est supprimée.
Après cela, \ST{0} contient {\_a} et la fonction se termine.

Le saut conditionnel \JNE est effectué dans deux cas: si $b>a$ ou $a=b$.
\ST{0} est copié dans \ST{0}, c'est comme une opération sans effet (\ac{NOP}), puis
une valeur est supprimée de la pile et le sommet de la pile (\ST{0}) contient la
valeur qui était avant dans \ST{1} (qui est {\_b}).
Puis la fonction se termine.
La raison pour laquelle cette instruction est utilisée ici est sans doute que le
\ac{FPU} n'a pas d'autre instruction pour prendre une valeur sur la pile et la
supprimer.

\clearpage
\subsubsection{MSVC: x86 + \olly}

Essayons de hacker notre programme dans \olly, pour le forcer à penser que \scanf
fonctionne toujours sans erreur.
Lorsque l'adresse d'une variable locale est passée à \scanf, la variable contient
initiallement toujours des restes de données aléatoires, dans ce cas \TT{0x6E494714}:

\begin{figure}[H]
\centering
\myincludegraphics{patterns/04_scanf/3_checking_retval/olly_1.png}
\caption{\olly: passer l'adresse de la variable à \scanf}
\label{fig:scanf_ex3_olly_1}
\end{figure}

\clearpage
Lorsque \scanf s'exécute dans la console, entrons quelque chose qui n'est pas du
tout un nombre, comme \q{asdasd}.
\scanf termine avec 0 dans \EAX, ce qui indique qu'une erreur s'est produite.

Nous pouvons vérifier la variable locale dans le pile et noter qu'elle n'a pas changé.
En effet, qu'aurait écrit \scanf ici?
Elle n'a simplement rien fait à part renvoyer zéro.

Essayons de \q{hacker} notre programme.
Clique-droit sur \EAX,
parmi les options il y a \q{Set to 1} (mettre à 1).
C'est ce dont nous avons besoin.

Nous avons maintenant 1 dans \EAX, donc la vérification suivante va s'exécuter comme
souhaiter et \printf va afficher la valeur de la variable dans la pile.

Lorsque nous lançons le programme (F9) nous pouvons voir ceci dans la fenêtre
de la console:

\lstinputlisting[caption=fenêtre console]{patterns/04_scanf/3_checking_retval/console.txt}

En effet, 1850296084 est la représentation en décimal du nombre dans la pile (\TT{0x6E494714})!

