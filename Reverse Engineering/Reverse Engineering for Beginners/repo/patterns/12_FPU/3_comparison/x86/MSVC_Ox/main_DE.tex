\myparagraph{\Optimizing MSVC 2010}

\lstinputlisting[caption=\Optimizing MSVC 2010,style=customasmx86]{patterns/12_FPU/3_comparison/x86/MSVC_Ox/MSVC_DE.asm}

\myindex{x86!\Instructions!FCOM}
\FCOM unterscheidet sich von \FCOMP in dem Sinne, dass es nur die Werte vergleicht ohne den FPU Stack zu verändern. 
Anders als im vorangehenden Beispiel liegen die Operanden hier in umgekehrter Reihenfolge um vor. Dies ist der Grund
warum das Ergebnis dieses Vergleichs bezüglich der \CThreeBits unterschiedlich ist:

\begin{itemize}
\item Falls $a>b$ in unserem Beispiel, dann werden die \CThreeBits Bits wie folgt gesetzt: 0, 0, 0.
\item Falls $b>a$, dann ist das Bitmuster: 0, 0, 1.
\item Falls $a=b$, dann ist das Bitmuster: 1, 0, 0.
\end{itemize}
% TODO: table?
Der Befehl \INS{test ah, 65} setzt zwei Bits~---\Cthree und \Czero. 
Beide werden auf 0 gesetzt, falls $a>b$: in diesem Fall wird der \JNE Sprungbefehl nicht ausgeführt.
Dann folgt \INS{FSTP ST(1)}~---dieser Befehl kopiert den Wert von \ST{0} in den Operanden und holt einen Wert vom FPU
Stack.
Mit anderen Worten, der Befehl kopiert \ST{0} (in dem sich gerade der Wert von \GTT{\_a} befindet) nach \ST{1}. 
Anschließend befinden sich zwei Kopien von {\_a} oben auf dem Stack.
Nun wird ein Wert wieder vom Stack geholt. Schließlich enthält \ST{0} den Wert {\_a} und die Funktion beendet sich.

Der bedingte Sprung \JNE wird in zwei Fällen ausgeführt: wenn $b>a$ oder wenn $a=b$.
\ST{0} wird nach \ST{0} kopiert; dabei handelt es sich um eine Operation ohne Wirkung (\ac{NOP}), dann wird ein Wert
 vom Stack geholt und in \ST{0} steht dann was sich vorher in \ST{1} befunden hat, nämlich {\_b}. Danach beendet sich
 die Funktion.
Der Grund dafür, dass dieser Befehl hier erzeugt wird ist wahrscheinlich, dass die \ac{FPU} über keinen anderen Befehl
verfügt um einen Wert vom Stack zu holen und anschließend zu entsorgen.

\clearpage
\myparagraph{\olly + standardmäßig gepackte Felder}
\myindex{\olly}
Betrachten wir unser Beispiel (in dem die Felder standardmäßig auf 4 Byte angeordnet werden) in \olly:

\begin{figure}[H]
\centering
\myincludegraphics{patterns/15_structs/4_packing/olly_packing_4.png}
\caption{\olly: vor der Ausführung von \printf}
\label{fig:packing_olly_4}
\end{figure}
Wir sehen unsere 4 Felder im Datenfenster.

Wir fragen uns aber, woher die Zufallsbytes (0x30, 0x37, 0x01) stammen, die neben dem ersten ($a$) und dritten ($c$)
Feld liegen.

Betrachten wir unser Listing \myref{src:struct_packing_4}, erkennen wir, dass das erste und dritte Feld vom Typ \Tchar
ist, und daher nur ein Byte geschrieben wird, nämlich 1 bzw. 3 (Zeilen 6 und 8).

Die übrigen 3 Byte des 32-Bit-Wortes werden im Speicher nicht verändert!
Deshalb befinden sich hier zufällige Reste.

\myindex{x86!\Instructions!MOVSX}
Diese Reste beeinflussen den Output von \printf in keinster Weise, da die Werte für die Funktion mithilfe von \MOVSX
vorbereitet werden, der Bytes und nicht Worte als Argumente hat: 
\lstref{src:struct_packing_4} (Zeilen 34 und 38).
Der vorzeichenerweiternde Befehl \MOVSX wird hier übrigens verwendet, da \Tchar standardmäßig in MSVC und GCC
vorzeichenbehaftet ist.
Würde hier der Datentyp \TT{unsigned char} oder \TT{uint8\_t} verwendet, würde der Befehl \MOVZX stattdessen verwendet.

\clearpage
\myparagraph{\olly + Felder auf 1 Byte Grenzen angeordnet}
\myindex{\olly}
Hier sind die Dinge viel klarer ersichtlich: 4 Felder benötigen 16 Byte und die Werte werden nebeneinander gespeichert.

\begin{figure}[H]
\centering
\myincludegraphics{patterns/15_structs/4_packing/olly_packing_1.png}
\caption{\olly: Vor der Ausführung von \printf}
\label{fig:packing_olly_1}
\end{figure}

