\myparagraph{\Optimizing MSVC 2010}

\lstinputlisting[caption=\Optimizing MSVC 2010,style=customasmx86]{patterns/12_FPU/3_comparison/x86/MSVC_Ox/MSVC_RU.asm}

\myindex{x86!\Instructions!FCOM}
\FCOM отличается от \FCOMP тем, что просто сравнивает значения и оставляет стек в том же состоянии. 
В отличие от предыдущего примера, операнды здесь в обратном порядке. 
Поэтому и результат сравнения в \CThreeBits будет отличаться:

\begin{itemize}
\item Если $a>b$, то биты \CThreeBits должны быть выставлены так: 0, 0, 0.
\item Если $b>a$, то биты будут выставлены так: 0, 0, 1.
\item Если $a=b$, то биты будут выставлены так: 1, 0, 0.
\end{itemize}
% TODO: table?

Инструкция \INS{test ah, 65} как бы оставляет только два бита~--- \Cthree и \Czero. 
Они оба будут нулями, если $a>b$: в таком случае переход \JNE не сработает. 
\myindex{ARM!\Instructions!FSTP}
Далее имеется инструкция \INS{FSTP ST(1)}~--- эта инструкция копирует 
значение \ST{0} в указанный операнд и выдергивает одно значение из стека. В данном случае, 
она копирует \ST{0} 
(где сейчас лежит~\GTT{\_a})~в~\ST{1}. 
После этого на вершине стека два раза лежит~\GTT{\_a}. Затем одно значение выдергивается. 
После этого в \ST{0} остается~\GTT{\_a} и функция завершается.

Условный переход \JNE сработает в двух других случаях: если $b>a$ или $a=b$. 
\ST{0} скопируется в \ST{0} (как бы холостая операция). 
Затем одно значение из стека вылетит и на вершине стека останется то, что 
до этого лежало в \ST{1} (то~есть~\GTT{\_b}). И функция завершится. 
Эта инструкция используется здесь видимо потому что в FPU 
нет другой инструкции, которая просто выдергивает 
значение из стека и выбрасывает его.

\clearpage
\myparagraph{\olly + упаковка полей по умолчанию}
\myindex{\olly}

Попробуем в \olly наш пример, где поля выровнены по умолчанию (4 байта):

\begin{figure}[H]
\centering
\myincludegraphics{patterns/15_structs/4_packing/olly_packing_4.png}
\caption{\olly: Перед исполнением \printf}
\label{fig:packing_olly_4}
\end{figure}

В окне данных видим наши четыре поля.
Вот только, откуда взялись случайные байты (0x30, 0x37, 0x01) рядом с первым (a) и третьим (c) полем?

Если вернетесь к листингу \myref{src:struct_packing_4}, то увидите, что первое и третье поле имеет
тип \Tchar, а следовательно, туда записывается только один байт, 1 и 3 соответственно (строки 6 и 8).

Остальные три байта 32-битного слова не будут модифицироваться в памяти!

А, следовательно, там остается случайный мусор.
\myindex{x86!\Instructions!MOVSX}
Этот мусор никак не будет влиять на работу \printf,
потому что значения для нее готовятся при помощи инструкции \MOVSX, которая загружает
из памяти байты а не слова: 
\lstref{src:struct_packing_4} (строки 34 и 38).

Кстати, здесь используется именно \MOVSX (расширяющая знак), потому что тип 
\Tchar --- знаковый по умолчанию в MSVC и GCC.

Если бы здесь был тип \TT{unsigned char} или \TT{uint8\_t}, 
то здесь была бы инструкция \MOVZX.

\clearpage
\myparagraph{\olly + упаковка полей по границе в 1 байт}
\myindex{\olly}

Здесь всё куда понятнее: 4 поля занимают 10 байт и значения сложены в памяти друг к другу

\begin{figure}[H]
\centering
\myincludegraphics{patterns/15_structs/4_packing/olly_packing_1.png}
\caption{\olly: Перед исполнением \printf}
\label{fig:packing_olly_1}
\end{figure}

