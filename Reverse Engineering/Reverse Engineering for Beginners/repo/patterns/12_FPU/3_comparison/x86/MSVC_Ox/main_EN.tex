\myparagraph{\Optimizing MSVC 2010}

\lstinputlisting[caption=\Optimizing MSVC 2010,style=customasmx86]{patterns/12_FPU/3_comparison/x86/MSVC_Ox/MSVC_EN.asm}

\myindex{x86!\Instructions!FCOM}

\FCOM differs from \FCOMP in the sense that it just compares the values and doesn't change the FPU stack. 
Unlike the previous example, here the operands are in reverse order, 
which is why the result of the comparison in \CThreeBits is different:

\begin{itemize}
\item If $a>b$ in our example, then \CThreeBits bits are to be set as: 0, 0, 0.
\item If $b>a$, then the bits are: 0, 0, 1.
\item If $a=b$, then the bits are: 1, 0, 0.
\end{itemize}
% TODO: table?

The \INS{test ah, 65} instruction leaves just two bits~---\Cthree and \Czero. 
Both will be zero if $a>b$: in that case the \JNE jump will not be triggered. 
Then \INS{FSTP ST(1)} follows~---this instruction copies the value from \ST{0} to the operand and 
pops one value from the FPU stack.
In other words, the instruction copies \ST{0} (where the value of \GTT{\_a} is now) into \ST{1}.
After that, two copies of {\_a} are at the top of the stack. 
Then, one value is popped.
After that, \ST{0} contains {\_a} and the function is finishes.

The conditional jump \JNE is triggering in two cases: if $b>a$ or $a=b$. 
\ST{0} is copied into \ST{0}, it is just like an idle (\ac{NOP}) operation, then one value 
is popped from the stack and the top of the stack (\ST{0}) is contain what has been in \ST{1} before 
(that is {\_b}). 
Then the function finishes. 
The reason this instruction is used here probably is because the \ac{FPU} 
has no other instruction to pop a value from the stack and discard it.

\clearpage
\subsubsection{MSVC + \olly}
\myindex{\olly}

Let's load our example into \olly and set a breakpoint on \comp.
We can see how the values are compared at the first \comp call:

\begin{figure}[H]
\centering
\myincludegraphics{patterns/18_pointers_to_functions/olly1.png}
\caption{\olly: first call of \comp}
\label{fig:qsort_olly1}
\end{figure}

\olly shows the compared values in the window under the code window, for convenience.
We can also see that the \ac{SP} points to \ac{RA}, where the \qsort function is (located in \TT{MSVCR100.DLL}).

\clearpage
By tracing (F8) until the \TT{RETN} instruction and pressing F8 one more time, we return to the \qsort function:

\begin{figure}[H]
\centering
\myincludegraphics{patterns/18_pointers_to_functions/olly2.png}
\caption{\olly: the code in \qsort right after \comp call}
\label{fig:qsort_olly2}
\end{figure}

That has been a call to the comparison function.

\clearpage
Here is also a screenshot of the moment of the second call of \comp{}---now values that have to be compared are different:

\begin{figure}[H]
\centering
\myincludegraphics{patterns/18_pointers_to_functions/olly3.png}
\caption{\olly: second call of \comp}
\label{fig:qsort_olly3}
\end{figure}

