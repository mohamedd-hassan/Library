\myparagraph{\NonOptimizing MSVC}

MSVC 2010は以下のコードを生成します。

\lstinputlisting[caption=\NonOptimizing MSVC 2010,style=customasmx86]{patterns/12_FPU/3_comparison/x86/MSVC/MSVC_JA.asm}

\myindex{x86!\Instructions!FLD}

\FLD は\GTT{\_b}を\ST{0}にロードします。

\label{Czero_etc}
\newcommand{\Czero}{\GTT{C0}\xspace}
\newcommand{\Ctwo}{\GTT{C2}\xspace}
\newcommand{\Cthree}{\GTT{C3}\xspace}
\newcommand{\CThreeBits}{\Cthree/\Ctwo/\Czero}

\myindex{x86!\Instructions!FCOMP}

\FCOMP は\ST{0}の値と\GTT{\_a}の値を比較し、
それに応じてFPUステータスワードレジスタの \CThreeBits ビットを設定します。
これは、FPUの現在の状態を反映する16ビットのレジスタです。

ビットがセットされると、 \FCOMP 命令はスタックから1つの変数もポップします。
これは、値を比較してスタックを同じ状態にしておく \FCOM とは区別されます。

残念ながら、インテルP6 
\footnote{インテルP6はPentium Pro、Pentium IIなどです。}
より前のCPUには、 \CThreeBits ビットをチェックする条件付きジャンプ命令はありません。
おそらく、それは歴史の問題です。(思い起こしてみてください:FPUは過去に別のチップでした)

インテルP6で始まる最新のCPUは、\FCOMI/\FCOMIP/\FUCOMI/\FUCOMIP 命令を持っていて、
同じことをしますが、 \ZF/\PF/\CF CPUフラグを変更します。

\myindex{x86!\Instructions!FNSTSW}

\FNSTSW 命令は状態レジスタであるFPUを \AX にコピーします。 
\CThreeBits ビットは14/10/8の位置に配置され、
\AX レジスタの同じ位置にあり、 \AX{}~---\AH{} の上位部分に配置されます。

\begin{itemize}
\item この例では $b>a$ の場合、 \CThreeBits ビットは0,0,0と設定します。
\item $a>b$ の場合、ビットは0,0,1です。
\item $a=b$ の場合、ビットは1,0,0です。
\item 結果が順序付けられていない場合(エラーの場合)、セットされたビットは1,1,1,1です。
\end{itemize}
% TODO: table here?

これは、 \CThreeBits ビットが \AX レジスタにどのように配置されるかを示しています。

\begin{center}
\begin{bytefield}[endianness=big,bitwidth=0.03\linewidth]{16}
\bitheader{14,10,9,8} \\
\bitbox{1}{} & 
\bitbox{1}{C3} & 
\bitbox{3}{} & 
\bitbox{1}{C2} & 
\bitbox{1}{C1} & 
\bitbox{1}{C0} &
\bitbox{8}{}
\end{bytefield}
\end{center}


これは、 \CThreeBits ビットが \AH レジスタにどのように配置されるかを示しています。

\begin{center}
\begin{bytefield}[endianness=big,bitwidth=0.03\linewidth]{8}
\bitheader{6,2,1,0} \\
\bitbox{1}{} & 
\bitbox{1}{C3} & 
\bitbox{3}{} & 
\bitbox{1}{C2} & 
\bitbox{1}{C1} & 
\bitbox{1}{C0}
\end{bytefield}
\end{center}


\INS{test ah, 5}\footnote{5=101b}の実行後、
\Czero と \Ctwo ビット(0と2の位置)のみが考慮され、他のビットはすべて無視されます。

\label{parity_flag}
\myindex{x86!\Registers!\Flags!Parity flag}

さて、\emph{パリティーフラグ}と注目すべきもう1つの歴史的基礎についてお話しましょう。

このフラグは、最後の計算結果の1の数が偶数の場合は1に設定され、奇数の場合は0に設定されます。

Wikipedia\footnote{\WikipediaParityFlag}を見てみましょう:

\begin{framed}
\begin{quotation}
パリティフラグをテストする一般的な理由の1つに、無関係なFPUフラグをチェックすることがあります。 FPUには4つの条件フラグ
(C0~C3)がありますが、直接テストすることはできず、最初にフラグレジスタにコピーする必要があります。 
これが起こると、C0はキャリーフラグに、C2はパリティフラグに、C3はゼロフラグに置かれます。 
C2フラグは、例えば比較できない浮動小数点値(NaNまたはサポートされていない形式)がFUCOM命令と比較されます。
\end{quotation}
\end{framed}

Wikipediaで述べられているように、パリティフラグはFPUコードで使用されることがあります。

\myindex{x86!\Instructions!JP}

\Czero と \Ctwo の両方が0に設定されている場合、 \PF フラグは1に設定されます。その場合、
後続の \JP (\emph{jump if PF==1})が実行されます。 
いろいろな場合の \CThreeBits の値を思い出すと、
条件ジャンプ \JP は、 $b>a$ または $a=b$ の場合に実行されます。
(\INS{test ah, 5}命令によってクリアされているので、 \Cthree ビットはここでは考慮されていません)

それ以降はすべて簡単です。 
条件付きジャンプが実行された場合、
\FLD は\ST{0}の\GTT{\_b}の値をロードし、
実行されていなければ\GTT{\_a}の値をロードします。

\mysubparagraph{\Ctwo? のチェックは?}

\Ctwo フラグはエラー(\gls{NaN}など)の場合に設定されますが、コードではチェックされません。

プログラマがFPUエラーを気にする場合は、チェックを追加する必要があります。

\clearpage
\myparagraph{x86 + MSVC + \olly}
\myindex{\olly}
\myindex{x86!\Registers!\Flags}

\olly でこの例を実行すると、フラグがどのように設定されているかを見ることができます。 
符号なしの数値で動作する\TT{f\_unsigned()}から始めましょう。

\CMP はここで3回実行されますが、同じ引数についてはフラグは毎回同じです。

最初の比較の結果は、

\begin{figure}[H]
\centering
\myincludegraphics{patterns/07_jcc/simple/olly_unsigned1.png}
\caption{\olly: \TT{f\_unsigned()}: 最初の条件付きジャンプ}
\label{fig:jcc_olly_unsigned_1}
\end{figure}

従って、フラグは、C=1、P=1、A=1、Z=0、S=1、T=0、D=0、O=0です。

これらは \olly では1文字の略号で命名されています。

\olly は、(\JBE)ジャンプがトリガーされることを示唆しています。 
実際に、インテルのマニュアル(\myref{x86_manuals})を調べると、
CF=1またはZF=1の場合、JBEが起動することがわかります。 
条件はここに当てはまるので、ジャンプが開始されます。

\clearpage
次の条件付きジャンプは、

\begin{figure}[H]
\centering
\myincludegraphics{patterns/07_jcc/simple/olly_unsigned2.png}
\caption{\olly: \TT{f\_unsigned()}: 2番目の条件付きジャンプ}
\label{fig:jcc_olly_unsigned_2}
\end{figure}

\olly は、 \JNZ がトリガーされることを示唆しています。 
実際、ZF=0(ゼロフラグ)の場合、JNZが起動します。

\clearpage
3番目の条件付きジャンプは、 \JNB です。

\begin{figure}[H]
\centering
\myincludegraphics{patterns/07_jcc/simple/olly_unsigned3.png}
\caption{\olly: \TT{f\_unsigned()}: 3番目の条件付きジャンプ}
\label{fig:jcc_olly_unsigned_3}
\end{figure}

インテルのマニュアル(\myref{x86_manuals})では、CF=0(キャリーフラグ)の場合に \JNB が起動することがわかります。 
今回は当てはまらないので、3番目の \printf が実行されます。

\clearpage
次に、\olly で、符号付きの値で動作する\TT{f\_signed()}関数を見てみましょう。 
フラグは、C=1、P=1、A=1、Z=0、S=1、T=0、D=0、O=0と同様に設定されます。 
最初の条件付きジャンプ \JLE が起動されます。

インテルマニュアル(166ページの7.1.4)では、ZF = 1またはSFxOFの場合にこの命令がトリガされることがわかりました。 SFxOF私たちの場合は、ジャンプがトリガするように。

\begin{figure}[H]
\centering
\myincludegraphics{patterns/07_jcc/simple/olly_signed1.png}
\caption{\olly: \TT{f\_signed()}: 最初の条件付きジャンプ}
\label{fig:jcc_olly_signed_1}
\end{figure}

インテルマニュアル(\myref{x86_manuals})では、ZF=1または SF$\neq$OF の場合にこの命令が起動されることがわかりました。 
私たちの場合では SF$\neq$OF が、ジャンプが起動されます。

\clearpage
2番目の \JNZ 条件付きジャンプはZF=0の場合(ゼロ・フラグ)に起動します。

\begin{figure}[H]
\centering
\myincludegraphics{patterns/07_jcc/simple/olly_signed2.png}
\caption{\olly: \TT{f\_signed()}: 2番目の条件付きジャンプ}
\label{fig:jcc_olly_signed_2}
\end{figure}

\clearpage
第3の条件付きジャンプ \JGE は、SF=OFの場合にのみ実行されるため、起動しません。今回は、当てはまりません。

\begin{figure}[H]
\centering
\myincludegraphics{patterns/07_jcc/simple/olly_signed3.png}
\caption{\olly: \TT{f\_signed()}: 3番目の条件付きジャンプ}
\label{fig:jcc_olly_signed_3}
\end{figure}

