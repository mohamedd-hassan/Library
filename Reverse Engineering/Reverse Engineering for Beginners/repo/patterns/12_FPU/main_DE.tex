\mysection{\FPUChapterName}
\label{sec:FPU}

Die \ac{FPU} is ein Gerät innerhalb der \ac{CPU}, welche speziell für den Umgang
mit Fließkommazahlen ausgelegt ist. 

In der Vergangenheit wurde die \ac{FPU} auch als \q{Koprozessor} bezeichnet und
sie befindet sich neben der \ac{CPU}. 

\subsection{IEEE 754}

Eine Zahl besteht gemäß IEEE 754 Format aus einem \emph{Vorzeichne}, einer
\emph{Mantisse} (auch \emph{Bruch} genannt) und einem \emph{Exponenten}. 

\subsection{x86}

Es lohnt sich einen Blick auf die Stackmaschine zu werfen oder die
Grundlagen der Sprache Forth zu erlernen, bevor man die \ac{FPU} in
x86 genauer untersucht. 

\myindex{Intel!80486}
\myindex{Intel!FPU}
Es ist interessant zu wissen, dass sich der Koprozessor in der Vergangenheit
(vor dem 80486 Chip) auf einem separaten Chip befand und nicht immer auf dem
Mainboard vorinstalliert war. Es war möglich, diesen separat zu kaufen und zu
installieren\footnote{John Carmack z.B. verwendete Fixpunktarithmetikwerte 
seinem Videospiel Doom, gespeichert in 32-bit \ac{GPR} Registern (16 Bit
für den Hauptteil und weitere 16 Bit für den gebrochenen Teil), sodass Doom
auf 32-Bit-Computern ohne FPU, d.h., 80386 und 80486 SX, lauffähig war.}.

Seit der 80486 DX CPU ist die \ac{FPU} in die \ac{CPU} integriert.

\myindex{x86!\Instructions!FWAIT}
Der Befehl \INS{FWAIT} erinnert uns an diese Tatsache--er lässt die \ac{CPU} in
einen Wartezustand wechseln, in dem sie verbleibt, bis die \ac{FPU} ihre Arbeit
beendet hat.

Ein anderes Überbleibsel ist die Tatsache, dass die Opcodes der \ac{FPU} Befehle
mit sogenannten \q{escape}-Opcodes (\GTT{D8..DF}, d.h. Opcodes, die an einen
separaten Koprozessor übergeben werden) beginnen.

\myindex{IEEE 754}
\label{FPU_is_stack}
Die FPU besitzt einen Stack, auf dem 8 80-bit-Register Platz finden, wobei jedes
Register eine Zahl im IEEE 754 Format aufnehmen kann.

Es gibt \ST{0}..\ST{7}. Verkürzend stellen \IDA und \olly \ST{0}, welches in
manchen (Hand-)büchern als \q{Stack Top} dargestellt wird, als \GTT{ST} dar.

\subsection{ARM, MIPS, x86/x64 SIMD}
In ARM und MIPS ist die FPU kein Stack, sondern ein Registersatz.
% TBT

Das gleiche Paradigma wird in den SIMD-Erweiterungen von x86/x64 CPUs verwendet.

\subsection{\CCpp}

\myindex{float}
\myindex{double}
Die Standardsparche \CCpp bietet zumindest two unterschiedliche
Fließkommezahlentypen, \Tfloat (\emph{einfache Genauigkeit}, 32 Bit)
\footnote{Fließkommazahlen mit einfacher Genauigkeit werden auch im Abschnitt
\emph{\WorkingWithFloatAsWithStructSubSubSectionName}~(\myref{sec:floatasstruct})
behandelt} und \Tdouble (\emph{doppelte Genauigkeit}, 64 Bit).

In \InSqBrackets{\TAOCPvolII 246} können wir nachlesen, dass \emph{einfache
Genauigkeit} bedeutet, dass die Fließkommazahl in ein einzelnes (32-Bit)-Wort
hineinpasst, \emph{doppelte Genauigkeit} bedeutet dagegen, dass die Zahl in zwei
Worten (64 Bit) abgelegt werden kann.

\myindex{long double}
GCC unterstützt im Gegensatz zu MSVC ebenfalls den \emph{long double} Typ
(\emph{erweiterte Genauigkeit} 80 Bit).

Der \Tfloat Typ benötigt dieselbe Anzahl an Bits wie der \Tint Typ in
32-Bit-Umgebungen, aber die Darstellung der Zahlen ist komplett verschieden
voneinander.

\EN{\mysection{Task manager practical joke (Windows Vista)}
\myindex{Windows!Windows Vista}

Let's see if it's possible to hack Task Manager slightly so it would detect more \ac{CPU} cores.

\myindex{Windows!NTAPI}

Let us first think, how does the Task Manager know the number of cores?

There is the \TT{GetSystemInfo()} win32 function present in win32 userspace which can tell us this.
But it's not imported in \TT{taskmgr.exe}.

There is, however, another one in \gls{NTAPI}, \TT{NtQuerySystemInformation()}, 
which is used in \TT{taskmgr.exe} in several places.

To get the number of cores, one has to call this function with the \TT{SystemBasicInformation} constant
as a first argument (which is zero
\footnote{\href{http://msdn.microsoft.com/en-us/library/windows/desktop/ms724509(v=vs.85).aspx}{MSDN}}).

The second argument has to point to the buffer which is getting all the information.

So we have to find all calls to the \\
\TT{NtQuerySystemInformation(0, ?, ?, ?)} function.
Let's open \TT{taskmgr.exe} in IDA. 
\myindex{Windows!PDB}

What is always good about Microsoft executables is that IDA can download the corresponding \gls{PDB} 
file for this executable and show all function names.

It is visible that Task Manager is written in \Cpp and some of the function names and classes are really 
speaking for themselves.
There are classes CAdapter, CNetPage, CPerfPage, CProcInfo, CProcPage, CSvcPage, 
CTaskPage, CUserPage.

Apparently, each class corresponds to each tab in Task Manager.

Let's visit each call and add comment with the value which is passed as the first function argument.
We will write \q{not zero} at some places, because the value there was clearly not zero, 
but something really different (more about this in the second part of this chapter).

And we are looking for zero passed as argument, after all.

\begin{figure}[H]
\centering
\myincludegraphics{examples/taskmgr/IDA_xrefs.png}
\caption{IDA: cross references to NtQuerySystemInformation()}
\end{figure}

Yes, the names are really speaking for themselves.

When we closely investigate each place where\\
\TT{NtQuerySystemInformation(0, ?, ?, ?)} is called,
we quickly find what we need in the \TT{InitPerfInfo()} function:

\lstinputlisting[caption=taskmgr.exe (Windows Vista),style=customasmx86]{examples/taskmgr/taskmgr.lst}

\TT{g\_cProcessors} is a global variable, and this name has been assigned by 
IDA according to the \gls{PDB} loaded from Microsoft's symbol server.

The byte is taken from \TT{var\_C20}. 
And \TT{var\_C58} is passed to\\
\TT{NtQuerySystemInformation()} 
as a pointer to the receiving buffer.
The difference between 0xC20 and 0xC58 is 0x38 (56).

Let's take a look at format of the return structure, which we can find in MSDN:

\begin{lstlisting}[style=customc]
typedef struct _SYSTEM_BASIC_INFORMATION {
    BYTE Reserved1[24];
    PVOID Reserved2[4];
    CCHAR NumberOfProcessors;
} SYSTEM_BASIC_INFORMATION;
\end{lstlisting}

This is a x64 system, so each PVOID takes 8 bytes.

All \emph{reserved} fields in the structure take $24+4*8=56$ bytes.

Oh yes, this implies that \TT{var\_C20} is the local stack is exactly the
\TT{NumberOfProcessors} field of the \TT{SYSTEM\_BASIC\_INFORMATION} structure.

Let's check our guess.
Copy \TT{taskmgr.exe} from \TT{C:\textbackslash{}Windows\textbackslash{}System32} 
to some other folder 
(so the \emph{Windows Resource Protection} 
will not try to restore the patched \TT{taskmgr.exe}).

Let's open it in Hiew and find the place:

\begin{figure}[H]
\centering
\myincludegraphics{examples/taskmgr/hiew2.png}
\caption{Hiew: find the place to be patched}
\end{figure}

Let's replace the \TT{MOVZX} instruction with ours.
Let's pretend we've got 64 CPU cores.

Add one additional \ac{NOP} (because our instruction is shorter than the original one):

\begin{figure}[H]
\centering
\myincludegraphics{examples/taskmgr/hiew1.png}
\caption{Hiew: patch it}
\end{figure}

And it works!
Of course, the data in the graphs is not correct.

At times, Task Manager even shows an overall CPU load of more than 100\%.

\begin{figure}[H]
\centering
\myincludegraphics{examples/taskmgr/taskmgr_64cpu_crop.png}
\caption{Fooled Windows Task Manager}
\end{figure}

The biggest number Task Manager does not crash with is 64.

Apparently, Task Manager in Windows Vista was not tested on computers with a large number of cores.

So there are probably some static data structure(s) inside it limited to 64 cores.

\subsection{Using LEA to load values}
\label{TaskMgr_LEA}

Sometimes, \TT{LEA} is used in \TT{taskmgr.exe} instead of \TT{MOV} to set the first argument of \\
\TT{NtQuerySystemInformation()}:

\lstinputlisting[caption=taskmgr.exe (Windows Vista),style=customasmx86]{examples/taskmgr/taskmgr2.lst}

\myindex{x86!\Instructions!LEA}

Perhaps \ac{MSVC} did so because machine code of \INS{LEA} is shorter than \INS{MOV REG, 5} (would be 5 instead of 4).

\INS{LEA} with offset in $-128..127$ range (offset will occupy 1 byte in opcode) with 32-bit registers is even shorter (for lack of REX prefix)---3 bytes.

Another example of such thing is: \myref{using_MOV_and_pack_of_LEA_to_load_values}.
}%
\RU{\subsection{Обменять входные значения друг с другом}

Вот так:

\lstinputlisting[style=customc]{patterns/061_pointers/swap/5_RU.c}

Как видим, байты загружаются в младшие 8-битные части регистров \TT{ECX} и \TT{EBX} используя \INS{MOVZX}
(так что старшие части регистров очищаются), затем байты записываются назад в другом порядке.

\lstinputlisting[style=customasmx86,caption=Optimizing GCC 5.4]{patterns/061_pointers/swap/5_GCC_O3_x86.s}

Адреса обоих байтов берутся из аргументов и во время исполнения ф-ции находятся в регистрах \TT{EDX} и \TT{EAX}.

Так что исопльзуем указатели --- вероятно, без них нет способа решить эту задачу лучше.

}%
\FR{\subsection{Exemple \#2: SCO OpenServer}

\label{examples_SCO}
\myindex{SCO OpenServer}
Un ancien logiciel pour SCO OpenServer de 1997 développé par une société qui a disparue
depuis longtemps.

Il y a un driver de dongle special à installer dans le système, qui contient les
chaînes de texte suivantes:
\q{Copyright 1989, Rainbow Technologies, Inc., Irvine, CA}
et
\q{Sentinel Integrated Driver Ver. 3.0 }.

Après l'installation du driver dans SCO OpenServer, ces fichiers apparaissent dans
l'arborescence /dev:

\begin{lstlisting}
/dev/rbsl8
/dev/rbsl9
/dev/rbsl10
\end{lstlisting}

Le programme renvoie une erreur lorsque le dongle n'est pas connecté, mais le message
d'erreur n'est pas trouvé dans les exécutables.

\myindex{COFF}

Grâce à \ac{IDA}, il est facile de charger l'exécutable COFF utilisé dans SCO OpenServer.

Essayons de trouver la chaîne \q{rbsl} et en effet, elle se trouve dans ce morceau
de code:

\lstinputlisting[style=customasmx86]{examples/dongles/2/1.lst}

Oui, en effet, le programme doit communiquer d'une façon ou d'une autre avec le driver.

\myindex{thunk-functions}
Le seul endroit où la fonction \TT{SSQC()} est appelée est dans la \glslink{thunk
 function}{fonction thunk}:

\lstinputlisting[style=customasmx86]{examples/dongles/2/2.lst}

SSQ() peut être appelé depuis au moins 2 fonctions.

L'une d'entre elles est:

\lstinputlisting[style=customasmx86]{examples/dongles/2/check1_EN.lst}

\q{\TT{3C}} et \q{\TT{3E}} semblent familiers: il y avait un dongle Sentinel Pro de
Rainbow sans mémoire, fournissant seulement une fonction de crypto-hachage secrète.

Vous pouvez lire une courte description de la fonction de hachage dont il s'agit
ici: \myref{hash_func}.

Mais retournons au programme.

Donc le programme peut seulement tester si un dongle est connecté ou s'il est absent.

Aucune autre information ne peut être écrite dans un tel dongle, puisqu'il n'a pas
de mémoire.
Les codes sur deux caractères sont des commandes (nous pouvons voir comment les commandes
sont traitées dans la fonction \TT{SSQC()}) et toutes les autres chaînes sont hachées
dans le dongle, transformées en un nombre 16-bit.
L'algorithme était secret, donc il n'était pas possible d'écrire un driver de remplacement
ou de refaire un dongle matériel qui l'émulerait parfaitement.

Toutefois, il est toujours possible d'intercepter tous les accès au dongle et de
trouver les constantes auxquelles les résultats de la fonction de hachage sont comparées.

Mais nous devons dire qu'il est possible de construire un schéma de logiciel de protection
de copie robuste basé sur une fonction secrète de hachage cryptographique: il suffit
qu'elle chiffre/déchiffre les fichiers de données utilisés par votre logiciel.

Mais retournons au code:

Les codes 51/52/53 sont utilisés pour choisir le port imprimante LPT.
3x/4x sont utilisés pour le choix de la \q{famille} (c'est ainsi que les dongles
Sentinel Pro sont différenciés les uns des autres: plus d'un dongle peut être connecté
sur un port LPT).

La seule chaîne passée à la fonction qui ne fasse pas 2 caractères est "0123456789".

Ensuite, le résultat est comparé à l'ensemble des résultats valides.

Si il est correct, 0xC ou 0xB est écrit dans la variable globale \TT{ctl\_model}.%

Une autre chaîne de texte qui est passée est
"PRESS ANY KEY TO CONTINUE: ", mais le résultat n'est pas testé.
Difficile de dire pourquoi, probablement une erreur\footnote{C'est un sentiment
étrange de trouver un bug dans un logiciel aussi ancien.}.

Voyons où la valeur de la variable globale \TT{ctl\_model} est utilisée.

Un tel endroit est:

\lstinputlisting[style=customasmx86]{examples/dongles/2/4.lst}

Si c'est 0, un message d'erreur chiffré est passé à une routine de déchiffrement
et affiché.

\myindex{x86!\Instructions!XOR}

La routine de déchiffrement de la chaîne semble être un simple \glslink{xoring}{xor}:

\lstinputlisting[style=customasmx86]{examples/dongles/2/err_warn.lst}

C'est pourquoi nous étions incapable de trouver le message d'erreur dans les fichiers
exécutable, car ils sont chiffrés (ce qui est une pratique courante).

Un autre appel à la fonction de hachage \TT{SSQ()} lui passe la chaîne \q{offln}
et le résultat est comparé avec \TT{0xFE81} et \TT{0x12A9}.

Si ils ne correspondent pas, ça se comporte comme une sorte de fonction \TT{timer()}
(peut-être en attente qu'un dongle mal connecté soit reconnecté et re-testé?) et ensuite
déchiffre un autre message d'erreur à afficher.

\lstinputlisting[style=customasmx86]{examples/dongles/2/check2_EN.lst}

Passer outre le dongle est assez facile: il suffit de patcher tous les sauts après
les instructions \CMP pertinentes.

Une autre option est d'écrire notre propre driver SCO OpenServer, contenant une table
de questions et de réponses, toutes celles qui sont présentent dans le programme.

\subsubsection{Déchiffrer les messages d'erreur}

À propos, nous pouvons aussi essayer de déchiffrer tous les messages d'erreurs.
L'algorithme qui se trouve dans la fonction \TT{err\_warn()} est très simple, en effet:

\lstinputlisting[caption=Decryption function,style=customasmx86]{examples/dongles/2/decrypting_FR.lst}

Comme on le voit, non seulement la chaîne est transmise à la fonction de déchiffrement
mais aussi la clef:

\lstinputlisting[style=customasmx86]{examples/dongles/2/tmp1_EN.asm}

L'algorithme est un simple \glslink{xoring}{xor}: chaque octet est xoré avec la clef, mais
la clef est incrémentée de 3 après le traitement de chaque octet.

Nous pouvons écrire un petit script Python pour vérifier notre hypothèse:

\lstinputlisting[caption=Python 3.x]{examples/dongles/2/decr1.py}

Et il affiche: \q{check security device connection}.
Donc oui, ceci est le message déchiffré.

Il y a d'autres messages chiffrés, avec leur clef correspondante.
Mais inutile de dire qu'il est possible de les déchiffrer sans leur clef.
Premièrement, nous voyons que le clef est en fait un octet.
C'est parce que l'instruction principale de déchiffrement (\XOR) fonctionne au niveau
de l'octet.
La clef se trouve dans le registre \ESI, mais seulement une partie de \ESI d'un octet
est utilisée.
Ainsi, une clef pourrait être plus grande que 255, mais sa valeur est toujours arrondie.

En conséquence, nous pouvons simplement essayer de brute-forcer, en essayant toutes
les clefs possible dans l'intervalle 0..255.
Nous allons aussi écarter les messages comportants des caractères non-imprimable.

\lstinputlisting[caption=Python 3.x]{examples/dongles/2/decr2.py}

Et nous obtenons:

\lstinputlisting[caption=Results]{examples/dongles/2/decr2_result.txt}

Ici il y a un peu de déchet, mais nous pouvons rapidement trouver les messages en
anglais.

À propos, puisque l'algorithme est un simple chiffrement xor, la même fonction peut
être utilisée pour chiffrer les messages.
Si besoin, nous pouvons chiffrer nos propres messages, et patcher le programme en les insérant.
}


\EN{\mysection{Task manager practical joke (Windows Vista)}
\myindex{Windows!Windows Vista}

Let's see if it's possible to hack Task Manager slightly so it would detect more \ac{CPU} cores.

\myindex{Windows!NTAPI}

Let us first think, how does the Task Manager know the number of cores?

There is the \TT{GetSystemInfo()} win32 function present in win32 userspace which can tell us this.
But it's not imported in \TT{taskmgr.exe}.

There is, however, another one in \gls{NTAPI}, \TT{NtQuerySystemInformation()}, 
which is used in \TT{taskmgr.exe} in several places.

To get the number of cores, one has to call this function with the \TT{SystemBasicInformation} constant
as a first argument (which is zero
\footnote{\href{http://msdn.microsoft.com/en-us/library/windows/desktop/ms724509(v=vs.85).aspx}{MSDN}}).

The second argument has to point to the buffer which is getting all the information.

So we have to find all calls to the \\
\TT{NtQuerySystemInformation(0, ?, ?, ?)} function.
Let's open \TT{taskmgr.exe} in IDA. 
\myindex{Windows!PDB}

What is always good about Microsoft executables is that IDA can download the corresponding \gls{PDB} 
file for this executable and show all function names.

It is visible that Task Manager is written in \Cpp and some of the function names and classes are really 
speaking for themselves.
There are classes CAdapter, CNetPage, CPerfPage, CProcInfo, CProcPage, CSvcPage, 
CTaskPage, CUserPage.

Apparently, each class corresponds to each tab in Task Manager.

Let's visit each call and add comment with the value which is passed as the first function argument.
We will write \q{not zero} at some places, because the value there was clearly not zero, 
but something really different (more about this in the second part of this chapter).

And we are looking for zero passed as argument, after all.

\begin{figure}[H]
\centering
\myincludegraphics{examples/taskmgr/IDA_xrefs.png}
\caption{IDA: cross references to NtQuerySystemInformation()}
\end{figure}

Yes, the names are really speaking for themselves.

When we closely investigate each place where\\
\TT{NtQuerySystemInformation(0, ?, ?, ?)} is called,
we quickly find what we need in the \TT{InitPerfInfo()} function:

\lstinputlisting[caption=taskmgr.exe (Windows Vista),style=customasmx86]{examples/taskmgr/taskmgr.lst}

\TT{g\_cProcessors} is a global variable, and this name has been assigned by 
IDA according to the \gls{PDB} loaded from Microsoft's symbol server.

The byte is taken from \TT{var\_C20}. 
And \TT{var\_C58} is passed to\\
\TT{NtQuerySystemInformation()} 
as a pointer to the receiving buffer.
The difference between 0xC20 and 0xC58 is 0x38 (56).

Let's take a look at format of the return structure, which we can find in MSDN:

\begin{lstlisting}[style=customc]
typedef struct _SYSTEM_BASIC_INFORMATION {
    BYTE Reserved1[24];
    PVOID Reserved2[4];
    CCHAR NumberOfProcessors;
} SYSTEM_BASIC_INFORMATION;
\end{lstlisting}

This is a x64 system, so each PVOID takes 8 bytes.

All \emph{reserved} fields in the structure take $24+4*8=56$ bytes.

Oh yes, this implies that \TT{var\_C20} is the local stack is exactly the
\TT{NumberOfProcessors} field of the \TT{SYSTEM\_BASIC\_INFORMATION} structure.

Let's check our guess.
Copy \TT{taskmgr.exe} from \TT{C:\textbackslash{}Windows\textbackslash{}System32} 
to some other folder 
(so the \emph{Windows Resource Protection} 
will not try to restore the patched \TT{taskmgr.exe}).

Let's open it in Hiew and find the place:

\begin{figure}[H]
\centering
\myincludegraphics{examples/taskmgr/hiew2.png}
\caption{Hiew: find the place to be patched}
\end{figure}

Let's replace the \TT{MOVZX} instruction with ours.
Let's pretend we've got 64 CPU cores.

Add one additional \ac{NOP} (because our instruction is shorter than the original one):

\begin{figure}[H]
\centering
\myincludegraphics{examples/taskmgr/hiew1.png}
\caption{Hiew: patch it}
\end{figure}

And it works!
Of course, the data in the graphs is not correct.

At times, Task Manager even shows an overall CPU load of more than 100\%.

\begin{figure}[H]
\centering
\myincludegraphics{examples/taskmgr/taskmgr_64cpu_crop.png}
\caption{Fooled Windows Task Manager}
\end{figure}

The biggest number Task Manager does not crash with is 64.

Apparently, Task Manager in Windows Vista was not tested on computers with a large number of cores.

So there are probably some static data structure(s) inside it limited to 64 cores.

\subsection{Using LEA to load values}
\label{TaskMgr_LEA}

Sometimes, \TT{LEA} is used in \TT{taskmgr.exe} instead of \TT{MOV} to set the first argument of \\
\TT{NtQuerySystemInformation()}:

\lstinputlisting[caption=taskmgr.exe (Windows Vista),style=customasmx86]{examples/taskmgr/taskmgr2.lst}

\myindex{x86!\Instructions!LEA}

Perhaps \ac{MSVC} did so because machine code of \INS{LEA} is shorter than \INS{MOV REG, 5} (would be 5 instead of 4).

\INS{LEA} with offset in $-128..127$ range (offset will occupy 1 byte in opcode) with 32-bit registers is even shorter (for lack of REX prefix)---3 bytes.

Another example of such thing is: \myref{using_MOV_and_pack_of_LEA_to_load_values}.
}%
\RU{\subsection{Обменять входные значения друг с другом}

Вот так:

\lstinputlisting[style=customc]{patterns/061_pointers/swap/5_RU.c}

Как видим, байты загружаются в младшие 8-битные части регистров \TT{ECX} и \TT{EBX} используя \INS{MOVZX}
(так что старшие части регистров очищаются), затем байты записываются назад в другом порядке.

\lstinputlisting[style=customasmx86,caption=Optimizing GCC 5.4]{patterns/061_pointers/swap/5_GCC_O3_x86.s}

Адреса обоих байтов берутся из аргументов и во время исполнения ф-ции находятся в регистрах \TT{EDX} и \TT{EAX}.

Так что исопльзуем указатели --- вероятно, без них нет способа решить эту задачу лучше.

}%
\FR{\subsection{Exemple \#2: SCO OpenServer}

\label{examples_SCO}
\myindex{SCO OpenServer}
Un ancien logiciel pour SCO OpenServer de 1997 développé par une société qui a disparue
depuis longtemps.

Il y a un driver de dongle special à installer dans le système, qui contient les
chaînes de texte suivantes:
\q{Copyright 1989, Rainbow Technologies, Inc., Irvine, CA}
et
\q{Sentinel Integrated Driver Ver. 3.0 }.

Après l'installation du driver dans SCO OpenServer, ces fichiers apparaissent dans
l'arborescence /dev:

\begin{lstlisting}
/dev/rbsl8
/dev/rbsl9
/dev/rbsl10
\end{lstlisting}

Le programme renvoie une erreur lorsque le dongle n'est pas connecté, mais le message
d'erreur n'est pas trouvé dans les exécutables.

\myindex{COFF}

Grâce à \ac{IDA}, il est facile de charger l'exécutable COFF utilisé dans SCO OpenServer.

Essayons de trouver la chaîne \q{rbsl} et en effet, elle se trouve dans ce morceau
de code:

\lstinputlisting[style=customasmx86]{examples/dongles/2/1.lst}

Oui, en effet, le programme doit communiquer d'une façon ou d'une autre avec le driver.

\myindex{thunk-functions}
Le seul endroit où la fonction \TT{SSQC()} est appelée est dans la \glslink{thunk
 function}{fonction thunk}:

\lstinputlisting[style=customasmx86]{examples/dongles/2/2.lst}

SSQ() peut être appelé depuis au moins 2 fonctions.

L'une d'entre elles est:

\lstinputlisting[style=customasmx86]{examples/dongles/2/check1_EN.lst}

\q{\TT{3C}} et \q{\TT{3E}} semblent familiers: il y avait un dongle Sentinel Pro de
Rainbow sans mémoire, fournissant seulement une fonction de crypto-hachage secrète.

Vous pouvez lire une courte description de la fonction de hachage dont il s'agit
ici: \myref{hash_func}.

Mais retournons au programme.

Donc le programme peut seulement tester si un dongle est connecté ou s'il est absent.

Aucune autre information ne peut être écrite dans un tel dongle, puisqu'il n'a pas
de mémoire.
Les codes sur deux caractères sont des commandes (nous pouvons voir comment les commandes
sont traitées dans la fonction \TT{SSQC()}) et toutes les autres chaînes sont hachées
dans le dongle, transformées en un nombre 16-bit.
L'algorithme était secret, donc il n'était pas possible d'écrire un driver de remplacement
ou de refaire un dongle matériel qui l'émulerait parfaitement.

Toutefois, il est toujours possible d'intercepter tous les accès au dongle et de
trouver les constantes auxquelles les résultats de la fonction de hachage sont comparées.

Mais nous devons dire qu'il est possible de construire un schéma de logiciel de protection
de copie robuste basé sur une fonction secrète de hachage cryptographique: il suffit
qu'elle chiffre/déchiffre les fichiers de données utilisés par votre logiciel.

Mais retournons au code:

Les codes 51/52/53 sont utilisés pour choisir le port imprimante LPT.
3x/4x sont utilisés pour le choix de la \q{famille} (c'est ainsi que les dongles
Sentinel Pro sont différenciés les uns des autres: plus d'un dongle peut être connecté
sur un port LPT).

La seule chaîne passée à la fonction qui ne fasse pas 2 caractères est "0123456789".

Ensuite, le résultat est comparé à l'ensemble des résultats valides.

Si il est correct, 0xC ou 0xB est écrit dans la variable globale \TT{ctl\_model}.%

Une autre chaîne de texte qui est passée est
"PRESS ANY KEY TO CONTINUE: ", mais le résultat n'est pas testé.
Difficile de dire pourquoi, probablement une erreur\footnote{C'est un sentiment
étrange de trouver un bug dans un logiciel aussi ancien.}.

Voyons où la valeur de la variable globale \TT{ctl\_model} est utilisée.

Un tel endroit est:

\lstinputlisting[style=customasmx86]{examples/dongles/2/4.lst}

Si c'est 0, un message d'erreur chiffré est passé à une routine de déchiffrement
et affiché.

\myindex{x86!\Instructions!XOR}

La routine de déchiffrement de la chaîne semble être un simple \glslink{xoring}{xor}:

\lstinputlisting[style=customasmx86]{examples/dongles/2/err_warn.lst}

C'est pourquoi nous étions incapable de trouver le message d'erreur dans les fichiers
exécutable, car ils sont chiffrés (ce qui est une pratique courante).

Un autre appel à la fonction de hachage \TT{SSQ()} lui passe la chaîne \q{offln}
et le résultat est comparé avec \TT{0xFE81} et \TT{0x12A9}.

Si ils ne correspondent pas, ça se comporte comme une sorte de fonction \TT{timer()}
(peut-être en attente qu'un dongle mal connecté soit reconnecté et re-testé?) et ensuite
déchiffre un autre message d'erreur à afficher.

\lstinputlisting[style=customasmx86]{examples/dongles/2/check2_EN.lst}

Passer outre le dongle est assez facile: il suffit de patcher tous les sauts après
les instructions \CMP pertinentes.

Une autre option est d'écrire notre propre driver SCO OpenServer, contenant une table
de questions et de réponses, toutes celles qui sont présentent dans le programme.

\subsubsection{Déchiffrer les messages d'erreur}

À propos, nous pouvons aussi essayer de déchiffrer tous les messages d'erreurs.
L'algorithme qui se trouve dans la fonction \TT{err\_warn()} est très simple, en effet:

\lstinputlisting[caption=Decryption function,style=customasmx86]{examples/dongles/2/decrypting_FR.lst}

Comme on le voit, non seulement la chaîne est transmise à la fonction de déchiffrement
mais aussi la clef:

\lstinputlisting[style=customasmx86]{examples/dongles/2/tmp1_EN.asm}

L'algorithme est un simple \glslink{xoring}{xor}: chaque octet est xoré avec la clef, mais
la clef est incrémentée de 3 après le traitement de chaque octet.

Nous pouvons écrire un petit script Python pour vérifier notre hypothèse:

\lstinputlisting[caption=Python 3.x]{examples/dongles/2/decr1.py}

Et il affiche: \q{check security device connection}.
Donc oui, ceci est le message déchiffré.

Il y a d'autres messages chiffrés, avec leur clef correspondante.
Mais inutile de dire qu'il est possible de les déchiffrer sans leur clef.
Premièrement, nous voyons que le clef est en fait un octet.
C'est parce que l'instruction principale de déchiffrement (\XOR) fonctionne au niveau
de l'octet.
La clef se trouve dans le registre \ESI, mais seulement une partie de \ESI d'un octet
est utilisée.
Ainsi, une clef pourrait être plus grande que 255, mais sa valeur est toujours arrondie.

En conséquence, nous pouvons simplement essayer de brute-forcer, en essayant toutes
les clefs possible dans l'intervalle 0..255.
Nous allons aussi écarter les messages comportants des caractères non-imprimable.

\lstinputlisting[caption=Python 3.x]{examples/dongles/2/decr2.py}

Et nous obtenons:

\lstinputlisting[caption=Results]{examples/dongles/2/decr2_result.txt}

Ici il y a un peu de déchet, mais nous pouvons rapidement trouver les messages en
anglais.

À propos, puisque l'algorithme est un simple chiffrement xor, la même fonction peut
être utilisée pour chiffrer les messages.
Si besoin, nous pouvons chiffrer nos propres messages, et patcher le programme en les insérant.
}


\EN{\mysection{Task manager practical joke (Windows Vista)}
\myindex{Windows!Windows Vista}

Let's see if it's possible to hack Task Manager slightly so it would detect more \ac{CPU} cores.

\myindex{Windows!NTAPI}

Let us first think, how does the Task Manager know the number of cores?

There is the \TT{GetSystemInfo()} win32 function present in win32 userspace which can tell us this.
But it's not imported in \TT{taskmgr.exe}.

There is, however, another one in \gls{NTAPI}, \TT{NtQuerySystemInformation()}, 
which is used in \TT{taskmgr.exe} in several places.

To get the number of cores, one has to call this function with the \TT{SystemBasicInformation} constant
as a first argument (which is zero
\footnote{\href{http://msdn.microsoft.com/en-us/library/windows/desktop/ms724509(v=vs.85).aspx}{MSDN}}).

The second argument has to point to the buffer which is getting all the information.

So we have to find all calls to the \\
\TT{NtQuerySystemInformation(0, ?, ?, ?)} function.
Let's open \TT{taskmgr.exe} in IDA. 
\myindex{Windows!PDB}

What is always good about Microsoft executables is that IDA can download the corresponding \gls{PDB} 
file for this executable and show all function names.

It is visible that Task Manager is written in \Cpp and some of the function names and classes are really 
speaking for themselves.
There are classes CAdapter, CNetPage, CPerfPage, CProcInfo, CProcPage, CSvcPage, 
CTaskPage, CUserPage.

Apparently, each class corresponds to each tab in Task Manager.

Let's visit each call and add comment with the value which is passed as the first function argument.
We will write \q{not zero} at some places, because the value there was clearly not zero, 
but something really different (more about this in the second part of this chapter).

And we are looking for zero passed as argument, after all.

\begin{figure}[H]
\centering
\myincludegraphics{examples/taskmgr/IDA_xrefs.png}
\caption{IDA: cross references to NtQuerySystemInformation()}
\end{figure}

Yes, the names are really speaking for themselves.

When we closely investigate each place where\\
\TT{NtQuerySystemInformation(0, ?, ?, ?)} is called,
we quickly find what we need in the \TT{InitPerfInfo()} function:

\lstinputlisting[caption=taskmgr.exe (Windows Vista),style=customasmx86]{examples/taskmgr/taskmgr.lst}

\TT{g\_cProcessors} is a global variable, and this name has been assigned by 
IDA according to the \gls{PDB} loaded from Microsoft's symbol server.

The byte is taken from \TT{var\_C20}. 
And \TT{var\_C58} is passed to\\
\TT{NtQuerySystemInformation()} 
as a pointer to the receiving buffer.
The difference between 0xC20 and 0xC58 is 0x38 (56).

Let's take a look at format of the return structure, which we can find in MSDN:

\begin{lstlisting}[style=customc]
typedef struct _SYSTEM_BASIC_INFORMATION {
    BYTE Reserved1[24];
    PVOID Reserved2[4];
    CCHAR NumberOfProcessors;
} SYSTEM_BASIC_INFORMATION;
\end{lstlisting}

This is a x64 system, so each PVOID takes 8 bytes.

All \emph{reserved} fields in the structure take $24+4*8=56$ bytes.

Oh yes, this implies that \TT{var\_C20} is the local stack is exactly the
\TT{NumberOfProcessors} field of the \TT{SYSTEM\_BASIC\_INFORMATION} structure.

Let's check our guess.
Copy \TT{taskmgr.exe} from \TT{C:\textbackslash{}Windows\textbackslash{}System32} 
to some other folder 
(so the \emph{Windows Resource Protection} 
will not try to restore the patched \TT{taskmgr.exe}).

Let's open it in Hiew and find the place:

\begin{figure}[H]
\centering
\myincludegraphics{examples/taskmgr/hiew2.png}
\caption{Hiew: find the place to be patched}
\end{figure}

Let's replace the \TT{MOVZX} instruction with ours.
Let's pretend we've got 64 CPU cores.

Add one additional \ac{NOP} (because our instruction is shorter than the original one):

\begin{figure}[H]
\centering
\myincludegraphics{examples/taskmgr/hiew1.png}
\caption{Hiew: patch it}
\end{figure}

And it works!
Of course, the data in the graphs is not correct.

At times, Task Manager even shows an overall CPU load of more than 100\%.

\begin{figure}[H]
\centering
\myincludegraphics{examples/taskmgr/taskmgr_64cpu_crop.png}
\caption{Fooled Windows Task Manager}
\end{figure}

The biggest number Task Manager does not crash with is 64.

Apparently, Task Manager in Windows Vista was not tested on computers with a large number of cores.

So there are probably some static data structure(s) inside it limited to 64 cores.

\subsection{Using LEA to load values}
\label{TaskMgr_LEA}

Sometimes, \TT{LEA} is used in \TT{taskmgr.exe} instead of \TT{MOV} to set the first argument of \\
\TT{NtQuerySystemInformation()}:

\lstinputlisting[caption=taskmgr.exe (Windows Vista),style=customasmx86]{examples/taskmgr/taskmgr2.lst}

\myindex{x86!\Instructions!LEA}

Perhaps \ac{MSVC} did so because machine code of \INS{LEA} is shorter than \INS{MOV REG, 5} (would be 5 instead of 4).

\INS{LEA} with offset in $-128..127$ range (offset will occupy 1 byte in opcode) with 32-bit registers is even shorter (for lack of REX prefix)---3 bytes.

Another example of such thing is: \myref{using_MOV_and_pack_of_LEA_to_load_values}.
}%
\RU{\subsection{Обменять входные значения друг с другом}

Вот так:

\lstinputlisting[style=customc]{patterns/061_pointers/swap/5_RU.c}

Как видим, байты загружаются в младшие 8-битные части регистров \TT{ECX} и \TT{EBX} используя \INS{MOVZX}
(так что старшие части регистров очищаются), затем байты записываются назад в другом порядке.

\lstinputlisting[style=customasmx86,caption=Optimizing GCC 5.4]{patterns/061_pointers/swap/5_GCC_O3_x86.s}

Адреса обоих байтов берутся из аргументов и во время исполнения ф-ции находятся в регистрах \TT{EDX} и \TT{EAX}.

Так что исопльзуем указатели --- вероятно, без них нет способа решить эту задачу лучше.

}%
\FR{\subsection{Exemple \#2: SCO OpenServer}

\label{examples_SCO}
\myindex{SCO OpenServer}
Un ancien logiciel pour SCO OpenServer de 1997 développé par une société qui a disparue
depuis longtemps.

Il y a un driver de dongle special à installer dans le système, qui contient les
chaînes de texte suivantes:
\q{Copyright 1989, Rainbow Technologies, Inc., Irvine, CA}
et
\q{Sentinel Integrated Driver Ver. 3.0 }.

Après l'installation du driver dans SCO OpenServer, ces fichiers apparaissent dans
l'arborescence /dev:

\begin{lstlisting}
/dev/rbsl8
/dev/rbsl9
/dev/rbsl10
\end{lstlisting}

Le programme renvoie une erreur lorsque le dongle n'est pas connecté, mais le message
d'erreur n'est pas trouvé dans les exécutables.

\myindex{COFF}

Grâce à \ac{IDA}, il est facile de charger l'exécutable COFF utilisé dans SCO OpenServer.

Essayons de trouver la chaîne \q{rbsl} et en effet, elle se trouve dans ce morceau
de code:

\lstinputlisting[style=customasmx86]{examples/dongles/2/1.lst}

Oui, en effet, le programme doit communiquer d'une façon ou d'une autre avec le driver.

\myindex{thunk-functions}
Le seul endroit où la fonction \TT{SSQC()} est appelée est dans la \glslink{thunk
 function}{fonction thunk}:

\lstinputlisting[style=customasmx86]{examples/dongles/2/2.lst}

SSQ() peut être appelé depuis au moins 2 fonctions.

L'une d'entre elles est:

\lstinputlisting[style=customasmx86]{examples/dongles/2/check1_EN.lst}

\q{\TT{3C}} et \q{\TT{3E}} semblent familiers: il y avait un dongle Sentinel Pro de
Rainbow sans mémoire, fournissant seulement une fonction de crypto-hachage secrète.

Vous pouvez lire une courte description de la fonction de hachage dont il s'agit
ici: \myref{hash_func}.

Mais retournons au programme.

Donc le programme peut seulement tester si un dongle est connecté ou s'il est absent.

Aucune autre information ne peut être écrite dans un tel dongle, puisqu'il n'a pas
de mémoire.
Les codes sur deux caractères sont des commandes (nous pouvons voir comment les commandes
sont traitées dans la fonction \TT{SSQC()}) et toutes les autres chaînes sont hachées
dans le dongle, transformées en un nombre 16-bit.
L'algorithme était secret, donc il n'était pas possible d'écrire un driver de remplacement
ou de refaire un dongle matériel qui l'émulerait parfaitement.

Toutefois, il est toujours possible d'intercepter tous les accès au dongle et de
trouver les constantes auxquelles les résultats de la fonction de hachage sont comparées.

Mais nous devons dire qu'il est possible de construire un schéma de logiciel de protection
de copie robuste basé sur une fonction secrète de hachage cryptographique: il suffit
qu'elle chiffre/déchiffre les fichiers de données utilisés par votre logiciel.

Mais retournons au code:

Les codes 51/52/53 sont utilisés pour choisir le port imprimante LPT.
3x/4x sont utilisés pour le choix de la \q{famille} (c'est ainsi que les dongles
Sentinel Pro sont différenciés les uns des autres: plus d'un dongle peut être connecté
sur un port LPT).

La seule chaîne passée à la fonction qui ne fasse pas 2 caractères est "0123456789".

Ensuite, le résultat est comparé à l'ensemble des résultats valides.

Si il est correct, 0xC ou 0xB est écrit dans la variable globale \TT{ctl\_model}.%

Une autre chaîne de texte qui est passée est
"PRESS ANY KEY TO CONTINUE: ", mais le résultat n'est pas testé.
Difficile de dire pourquoi, probablement une erreur\footnote{C'est un sentiment
étrange de trouver un bug dans un logiciel aussi ancien.}.

Voyons où la valeur de la variable globale \TT{ctl\_model} est utilisée.

Un tel endroit est:

\lstinputlisting[style=customasmx86]{examples/dongles/2/4.lst}

Si c'est 0, un message d'erreur chiffré est passé à une routine de déchiffrement
et affiché.

\myindex{x86!\Instructions!XOR}

La routine de déchiffrement de la chaîne semble être un simple \glslink{xoring}{xor}:

\lstinputlisting[style=customasmx86]{examples/dongles/2/err_warn.lst}

C'est pourquoi nous étions incapable de trouver le message d'erreur dans les fichiers
exécutable, car ils sont chiffrés (ce qui est une pratique courante).

Un autre appel à la fonction de hachage \TT{SSQ()} lui passe la chaîne \q{offln}
et le résultat est comparé avec \TT{0xFE81} et \TT{0x12A9}.

Si ils ne correspondent pas, ça se comporte comme une sorte de fonction \TT{timer()}
(peut-être en attente qu'un dongle mal connecté soit reconnecté et re-testé?) et ensuite
déchiffre un autre message d'erreur à afficher.

\lstinputlisting[style=customasmx86]{examples/dongles/2/check2_EN.lst}

Passer outre le dongle est assez facile: il suffit de patcher tous les sauts après
les instructions \CMP pertinentes.

Une autre option est d'écrire notre propre driver SCO OpenServer, contenant une table
de questions et de réponses, toutes celles qui sont présentent dans le programme.

\subsubsection{Déchiffrer les messages d'erreur}

À propos, nous pouvons aussi essayer de déchiffrer tous les messages d'erreurs.
L'algorithme qui se trouve dans la fonction \TT{err\_warn()} est très simple, en effet:

\lstinputlisting[caption=Decryption function,style=customasmx86]{examples/dongles/2/decrypting_FR.lst}

Comme on le voit, non seulement la chaîne est transmise à la fonction de déchiffrement
mais aussi la clef:

\lstinputlisting[style=customasmx86]{examples/dongles/2/tmp1_EN.asm}

L'algorithme est un simple \glslink{xoring}{xor}: chaque octet est xoré avec la clef, mais
la clef est incrémentée de 3 après le traitement de chaque octet.

Nous pouvons écrire un petit script Python pour vérifier notre hypothèse:

\lstinputlisting[caption=Python 3.x]{examples/dongles/2/decr1.py}

Et il affiche: \q{check security device connection}.
Donc oui, ceci est le message déchiffré.

Il y a d'autres messages chiffrés, avec leur clef correspondante.
Mais inutile de dire qu'il est possible de les déchiffrer sans leur clef.
Premièrement, nous voyons que le clef est en fait un octet.
C'est parce que l'instruction principale de déchiffrement (\XOR) fonctionne au niveau
de l'octet.
La clef se trouve dans le registre \ESI, mais seulement une partie de \ESI d'un octet
est utilisée.
Ainsi, une clef pourrait être plus grande que 255, mais sa valeur est toujours arrondie.

En conséquence, nous pouvons simplement essayer de brute-forcer, en essayant toutes
les clefs possible dans l'intervalle 0..255.
Nous allons aussi écarter les messages comportants des caractères non-imprimable.

\lstinputlisting[caption=Python 3.x]{examples/dongles/2/decr2.py}

Et nous obtenons:

\lstinputlisting[caption=Results]{examples/dongles/2/decr2_result.txt}

Ici il y a un peu de déchet, mais nous pouvons rapidement trouver les messages en
anglais.

À propos, puisque l'algorithme est un simple chiffrement xor, la même fonction peut
être utilisée pour chiffrer les messages.
Si besoin, nous pouvons chiffrer nos propres messages, et patcher le programme en les insérant.
}



\subsection{Einige Konstanten}
Es ist leicht, in Wikipedia die Darstellungen einiger Konstanten nach dem IEEE
754 Standard nachzulesen. Es ist interessant zu wissen, dass 0.0 nach IEEE 754
als 32 Nullbits (in einfacher Genauigkeit) oder 64 Nullbits (in doppelter
Genauigkeit) dargestellt wird.
Wenn also eine Fließkommavariable im Register oder Speicher auf 0.0 gesetzt
werden soll, kann der Befehl \MOV oder \emph{XOR reg, reg} verwendet werden.
\myindex{\CStandardLibrary!memset()}
Dies ist geeignet für Strukturen, in denen viele Variable unterschiedlichster
Datentypen vorhanden sind. Mit der gewöhnlichen memset() Funktion kann man alle
Integervariablen auf 0, alle Booleschen Variablen auf \emph{false}, alle Pointer
auf NULL und alle Fließkommavariablen (beliebiger Genauigkeit) auf 0.0 setzen.

\subsection{Kopieren}
Man könnte fälschlicherweise annehmen, dass die Befehle \INS{FLD}/\INS{FST}
verwendet werden müssen, um IEEE 754 Werte zu laden oder zu speichern (also
auch zu kopieren). Nichtsdestotrotz kann dies einfacher durch den Befehl
\INS{MOV} erreicht werden, welcher, logischerweise, Werte bitweise kopiert.

\subsection{Stack, Taschenrechner und umgekehrte polnische Notation}

\myindex{Reverse Polish notation}
Jetzt können wir einsehen, warum manche alten Taschenrechner die umgekehrte
polnische Notation verwenden.

Für die Addition von 12 und 34 muss beispielsweise zuerst 12, dann 34 und dann
das \q{plus}-Zeichen eingegeben werden. 

Dies liegt daran, dass alte Taschenrechner als einfache Stackmaschinen
implementiert waren und es auf diese Weise wesentlich einfacher war, mit
komplexen geklammerten Ausdrücke umzugehen.

% TBT
% Such a calculator still present in many Unix distributions: \emph{dc}.

\subsection{x64}
Mehr dazu wie Fließkommazahlen in x86-64 verarbeitet werden unter:
\myref{floating_SIMD}

% sections
\subsection{\Exercises}

\begin{itemize}
	\item \url{http://challenges.re/67}
	\item \url{http://challenges.re/68}
	\item \url{http://challenges.re/69}
	\item \url{http://challenges.re/70}
\end{itemize}


