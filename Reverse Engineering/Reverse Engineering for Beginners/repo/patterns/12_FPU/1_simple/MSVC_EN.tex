\myparagraph{MSVC}

Compile it in MSVC 2010:

\lstinputlisting[caption=MSVC 2010: \ttf{},style=customasmx86]{patterns/12_FPU/1_simple/MSVC_EN.asm}

\FLD takes 8 bytes from stack and loads the number into the \ST{0} register, automatically converting 
it into the internal 80-bit format (\emph{extended precision}).

\myindex{x86!\Instructions!FDIV}

\FDIV divides the value in \ST{0} by the number stored at address \\
\GTT{\_\_real@40091eb851eb851f}~---the value 3.14 is encoded there. 
The assembly syntax doesn't support floating point numbers, so 
what we see here is the hexadecimal representation of 3.14 in 64-bit IEEE 754 format.

After the execution of \FDIV \ST{0} holds the \gls{quotient}.

\myindex{x86!\Instructions!FDIVP}

By the way, there is also the \FDIVP instruction, which divides \ST{1} by \ST{0}, 
popping both these values from stack and then pushing the result. 
If you know the Forth language,
you can quickly understand that this is a stack machine.

The subsequent \FLD instruction pushes the value of $b$ into the stack.

After that, the quotient is placed in \ST{1}, and \ST{0} has the value of $b$.

\myindex{x86!\Instructions!FMUL}

The next \FMUL instruction does multiplication: $b$ from \ST{0} is multiplied by value at\\
\GTT{\_\_real@4010666666666666} (the number 4.1 is there) and leaves the result in the \ST{0} register.

\myindex{x86!\Instructions!FADDP}

The last \FADDP instruction adds the two values at top of stack, storing the result in \ST{1} 
and then popping the value of \ST{0}, thereby leaving the result at the top of the stack, in \ST{0}.

The function must return its result in the \ST{0} register, 
so there are no any other instructions except the function epilogue after \FADDP.

\clearpage
\subsubsection{MSVC + \olly}
\myindex{\olly}

Let's load our example into \olly and set a breakpoint on \comp.
We can see how the values are compared at the first \comp call:

\begin{figure}[H]
\centering
\myincludegraphics{patterns/18_pointers_to_functions/olly1.png}
\caption{\olly: first call of \comp}
\label{fig:qsort_olly1}
\end{figure}

\olly shows the compared values in the window under the code window, for convenience.
We can also see that the \ac{SP} points to \ac{RA}, where the \qsort function is (located in \TT{MSVCR100.DLL}).

\clearpage
By tracing (F8) until the \TT{RETN} instruction and pressing F8 one more time, we return to the \qsort function:

\begin{figure}[H]
\centering
\myincludegraphics{patterns/18_pointers_to_functions/olly2.png}
\caption{\olly: the code in \qsort right after \comp call}
\label{fig:qsort_olly2}
\end{figure}

That has been a call to the comparison function.

\clearpage
Here is also a screenshot of the moment of the second call of \comp{}---now values that have to be compared are different:

\begin{figure}[H]
\centering
\myincludegraphics{patterns/18_pointers_to_functions/olly3.png}
\caption{\olly: second call of \comp}
\label{fig:qsort_olly3}
\end{figure}

