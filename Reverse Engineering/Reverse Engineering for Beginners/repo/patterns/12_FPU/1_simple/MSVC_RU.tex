\myparagraph{MSVC}

Компилируем в MSVC 2010:

\lstinputlisting[caption=MSVC 2010: \ttf{},style=customasmx86]{patterns/12_FPU/1_simple/MSVC_RU.asm}

\FLD берет 8 байт из стека и загружает их в регистр \ST{0}, автоматически конвертируя во внутренний 
80-битный формат (\emph{extended precision}).

\myindex{x86!\Instructions!FDIV}
\FDIV делит содержимое регистра \ST{0} на число, лежащее по адресу \\
\GTT{\_\_real@40091eb851eb851f}~--- 
там закодировано значение 3,14. Синтаксис ассемблера не поддерживает подобные числа, 
поэтому мы там видим шестнадцатеричное представление числа 3,14 в формате IEEE 754.

После выполнения \FDIV в \ST{0} остается \glslink{quotient}{частное}.

\myindex{x86!\Instructions!FDIVP}
Кстати, есть ещё инструкция \FDIVP, которая делит \ST{1} на \ST{0}, 
выталкивает эти числа из стека и заталкивает результат. 
Если вы знаете язык Forth, то это как раз оно и есть~--- стековая машина.

Следующая \FLD заталкивает в стек значение $b$.

После этого в \ST{1} перемещается результат деления, а в \ST{0} теперь $b$.

\myindex{x86!\Instructions!FMUL}
Следующий \FMUL умножает $b$ из \ST{0} на значение \\
\GTT{\_\_real@4010666666666666} --- там лежит число 4,1~--- и оставляет результат в \ST{0}.

\myindex{x86!\Instructions!FADDP}
Самая последняя инструкция \FADDP складывает два значения из вершины стека 
в \ST{1} и затем выталкивает значение, лежащее в \ST{0}. 
Таким образом результат сложения остается на вершине стека в \ST{0}.

Функция должна вернуть результат в \ST{0}, так что больше ничего здесь не производится, 
кроме эпилога функции.

\clearpage
\myparagraph{\olly + упаковка полей по умолчанию}
\myindex{\olly}

Попробуем в \olly наш пример, где поля выровнены по умолчанию (4 байта):

\begin{figure}[H]
\centering
\myincludegraphics{patterns/15_structs/4_packing/olly_packing_4.png}
\caption{\olly: Перед исполнением \printf}
\label{fig:packing_olly_4}
\end{figure}

В окне данных видим наши четыре поля.
Вот только, откуда взялись случайные байты (0x30, 0x37, 0x01) рядом с первым (a) и третьим (c) полем?

Если вернетесь к листингу \myref{src:struct_packing_4}, то увидите, что первое и третье поле имеет
тип \Tchar, а следовательно, туда записывается только один байт, 1 и 3 соответственно (строки 6 и 8).

Остальные три байта 32-битного слова не будут модифицироваться в памяти!

А, следовательно, там остается случайный мусор.
\myindex{x86!\Instructions!MOVSX}
Этот мусор никак не будет влиять на работу \printf,
потому что значения для нее готовятся при помощи инструкции \MOVSX, которая загружает
из памяти байты а не слова: 
\lstref{src:struct_packing_4} (строки 34 и 38).

Кстати, здесь используется именно \MOVSX (расширяющая знак), потому что тип 
\Tchar --- знаковый по умолчанию в MSVC и GCC.

Если бы здесь был тип \TT{unsigned char} или \TT{uint8\_t}, 
то здесь была бы инструкция \MOVZX.

\clearpage
\myparagraph{\olly + упаковка полей по границе в 1 байт}
\myindex{\olly}

Здесь всё куда понятнее: 4 поля занимают 10 байт и значения сложены в памяти друг к другу

\begin{figure}[H]
\centering
\myincludegraphics{patterns/15_structs/4_packing/olly_packing_1.png}
\caption{\olly: Перед исполнением \printf}
\label{fig:packing_olly_1}
\end{figure}

