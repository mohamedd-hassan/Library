\mysection[Линейный конгруэнтный генератор]{Линейный конгруэнтный генератор как генератор псевдослучайных чисел}
\myindex{\CStandardLibrary!rand()}
\label{LCG_simple}

Линейный конгруэнтный генератор, пожалуй, самый простой способ генерировать псевдослучайные числа.

Он не в почете в наше время\footnote{Вихрь Мерсенна куда лучше}, но он настолько прост
(только одно умножение, одно сложение и одна операция \q{И}),
что мы можем использовать его в качестве примера.

\lstinputlisting[style=customc]{patterns/145_LCG/rand_RU.c}

Здесь две функции: одна используется для инициализации внутреннего состояния, а вторая
вызывается собственно для генерации псевдослучайных чисел.

Мы видим, что в алгоритме применяются две константы.
Они взяты из
[William H. Press and Saul A. Teukolsky and William T. Vetterling and Brian P. Flannery, \emph{Numerical Recipes}, (2007)].
Определим их используя выражение \CCpp \TT{\#define}. Это макрос.

Разница между макросом в \CCpp и константой в том, что все макросы заменяются на значения препроцессором
\CCpp и они не занимают места в памяти как переменные.

А константы, напротив, это переменные только для чтения.

Можно взять указатель (или адрес) переменной-константы, но это невозможно сделать с макросом.

Последняя операция \q{И} нужна, потому что согласно стандарту Си \TT{my\_rand()} должна возвращать значение в пределах
0..32767.

Если вы хотите получать 32-битные псевдослучайные значения, просто уберите последнюю операцию \q{И}.

\subsection{x86}

\lstinputlisting[caption=\Optimizing MSVC 2013,style=customasmx86]{patterns/145_LCG/rand_MSVC_2013_x86_Ox.asm}

Вот мы это и видим: обе константы встроены в код.

Память для них не выделяется.
Функция \TT{my\_srand()} просто копирует входное значение во внутреннюю переменную \TT{rand\_state}.

\TT{my\_rand()} берет её, вычисляет следующее состояние \TT{rand\_state}, 
обрезает его и оставляет в регистре EAX.

Неоптимизированная версия побольше:

\lstinputlisting[caption=\NonOptimizing MSVC 2013,style=customasmx86]{patterns/145_LCG/rand_MSVC_2013_x86.asm}

\subsection{x64}

Версия для x64 почти такая же, и использует 32-битные регистры вместо 64-битных
(потому что мы работаем здесь с переменными типа \Tint).

Но функция \TT{my\_srand()} берет входной аргумент из регистра \ECX, а не из стека:

\lstinputlisting[caption=\Optimizing MSVC 2013 x64,style=customasmx86]{patterns/145_LCG/rand_MSVC_2013_x64_Ox_RU.asm}

GCC делает почти такой же код.

\subsection{32-bit ARM}

\lstinputlisting[caption=\OptimizingKeilVI (\ARMMode),style=customasmARM]{patterns/145_LCG/rand.s_Keil_ARM_O3_RU.s}

В ARM инструкцию невозможно встроить 32-битную константу, так что Keil-у приходится размещать их отдельно и дополнительно загружать.
Вот еще что интересно: константу 0x7FFF также нельзя встроить.
Поэтому Keil сдвигает \TT{rand\_state} влево на 17 бит и затем сдвигает вправо на 17 бит.
Это аналогично \CCpp{}-выражению $(rand\_state \ll 17) \gg 17$.
Выглядит как бессмысленная операция, но тем не менее, что она делает это очищает старшие 17 бит, оставляя младшие 15 бит нетронутыми, и это наша цель, в конце концов. \\
\\
\Optimizing Keil для режима Thumb делает почти такой же код.

\subsection{MIPS}

\lstinputlisting[caption=\Optimizing GCC 4.4.5,style=customasmMIPS]{patterns/05_passing_arguments/MIPS_O3_IDA_RU.lst}

Первые 4 аргумента функции передаются в четырех регистрах с префиксами A-.

\myindex{MIPS!\Instructions!MULT}
В MIPS есть два специальных регистра: HI и LO, которые выставляются в 64-битный результат умножения
во время исполнения инструкции \TT{MULT}.

\myindex{MIPS!\Instructions!MFLO}
\myindex{MIPS!\Instructions!MFHI}
К регистрам можно обращаться только используя инструкции \TT{MFLO} и \TT{MFHI}.
Здесь \TT{MFLO} берет младшую часть результата умножения и записывает в \$V0.
Так что старшая 32-битная часть результата игнорируется (содержимое регистра HI не используется).
Действительно, мы ведь работаем с 32-битным типом \Tint.


\myindex{MIPS!\Instructions!ADDU}
И наконец, \TT{ADDU} (\q{Add Unsigned}~--- добавить беззнаковое) прибавляет значение третьего аргумента к результату.

\myindex{MIPS!\Instructions!ADD}
\myindex{MIPS!\Instructions!ADDU}
\myindex{Ada}
\myindex{Integer overflow}
В MIPS есть две разных инструкции сложения: \TT{ADD} и \TT{ADDU}.
На самом деле, дело не в знаковых числах, а в исключениях: \TT{ADD} может вызвать исключение
во время переполнения. Это иногда полезно\footnote{\url{http://blog.regehr.org/archives/1154}} и поддерживается,
например, в \ac{PL} Ada.

\TT{ADDU} не вызывает исключения во время переполнения.
А так как \CCpp не поддерживает всё это, мы видим здесь \TT{ADDU} вместо \TT{ADD}.

32-битный результат оставляется в \$V0.

\myindex{MIPS!\Instructions!JAL}
\myindex{MIPS!\Instructions!JALR}
В \main есть новая для нас инструкция: \TT{JAL} (\q{Jump and Link}). 
Разница между \INS{JAL} и \INS{JALR} в том, что относительное смещение кодируется в первой инструкции,
а \INS{JALR} переходит по абсолютному адресу, записанному в регистр (\q{Jump and Link Register}).

Обе функции \ttf и \main расположены в одном объектном файле, так что относительный адрес \ttf известен и фиксирован.



\subsection{Версия этого примера для многопоточной среды}

Версия примера для многопоточной среды будет рассмотрена позже: \myref{LCG_TLS}.

