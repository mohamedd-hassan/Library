\mysection{Cabeçalhos e rodapés de funções}
\label{sec:prologepilog}
\myindex{\PTBRph{}} % Function epilogue
\myindex{\PTBRph{}} % Function prologue

Um cabeçalho de uma função é uma sequência de instruções no começo da função. Ele geralmente se parece com algo como o código a seguir:

\begin{lstlisting}[style=customasmx86]
    push    ebp
    mov     ebp, esp
    sub     esp, X
\end{lstlisting}

O que essas instruções fazem: salvam o valor no registrador \EBP, mudam o valor de \EBP para o valor em \ESP e então aloca espaço na pilha para variáveis locais.

O valor em \EBP continua o mesmo depois do período de execução da função e é para ser usado para variáveis locais e acessos de argumentos.
Para o mesmo propósito pode ser usado o \ESP, mas considerando que ele muda com o tempo, essa abordagem não é muito conveniente.

O rodapé da função libera o espaço alocado na pilha, retorna o valor do registrador \EBP de volta ao seu estado inicial e retorna o controle de volta para a chamada da função:

\begin{lstlisting}[style=customasmx86]
    mov    esp, ebp
    pop    ebp
    ret    0
\end{lstlisting}

Os cabeçalhos e rodapés das funções geralmente são detectados na desassemblagem para a delimitação das funções.

\subsection{\Recursion}

\ac{TBT}

\myindex{\Recursion}
\ac{TBT}

Epilogues and prologues can negatively affect the recursion performance.

More about recursion in this book: \myref{Recursion_and_tail_call}.

