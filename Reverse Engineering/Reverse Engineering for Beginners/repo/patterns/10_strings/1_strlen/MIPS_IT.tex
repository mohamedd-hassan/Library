\subsubsection{MIPS}

\lstinputlisting[caption=\Optimizing GCC 4.4.5 (IDA),style=customasmMIPS]{patterns/10_strings/1_strlen/MIPS_O3_IDA_IT.lst}

\myindex{MIPS!\Instructions!NOR}
\myindex{MIPS!\Pseudoinstructions!NOT}

MIPS non ha una istruzione \NOT, ma ha \NOR che è l'operazione \TT{OR~+~NOT}.


Questa operazione è ampiamente utilizzata nell' elettronica digitale\footnote{NOR è chiamata \q{porta universale}}.
\myindex{Apollo Guidance Computer}
Per esempio, l' Apollo Guidance Computer utilizzato nel programma Apollo, 
fu costruito solamente utilizzando 5600 porte NOR :
[Jens Eickhoff, \emph{Onboard Computers, Onboard Software and Satellite Operations: An Introduction}, (2011)].
Ma l' elemento NOR non è molto popolare nella programmazione.

Quindi, l' operazione NOT qui è implementata come \TT{NOR~DST,~\$ZERO,~SRC}.

% to be synced with Eng
Dai fondamenti sappiamo che l' inversione di tutti i bit di un numero con segno, equivale 
a cambiargli il segno e sottrarre 1 dal risultato.

Quindi quello che \NOT fa in questo caso, è prendere il valore di $str$ e trasformarlo in $-str-1$.
L' operazione di addizione che segue prepara il risusltato.
