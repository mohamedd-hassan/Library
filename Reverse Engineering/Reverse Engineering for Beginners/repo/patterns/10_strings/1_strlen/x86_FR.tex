\subsubsection{x86}

\myparagraph{MSVC \NonOptimizing}

Compilons:

\lstinputlisting[style=customasmx86]{patterns/10_strings/1_strlen/10_1_msvc_FR.asm}

\myindex{x86!\Instructions!MOVSX}
\myindex{x86!\Instructions!TEST}

Nous avons ici deux nouvelles instructions: \MOVSX et \TEST.

\label{MOVSX}

La première---\MOVSX---prend un octet depuis une adresse en mémoire et stocke la
valeur dans un registre 32-bit. 
\MOVSX signifie \emph{MOV with Sign-Extend} (déplacement avec extension de signe).
\MOVSX met le reste des bits, du 8ème au 31ème, à 1 si l'octet source est \emph{négatif}
ou à 0 si il est \emph{positif}.

Et voici pourquoi.

Par défaut, le type \Tchar est signé dans MSVC et GCC. Si nous avons deux valeurs
dont l'une d'elle est un \Tchar et l'autre un \Tint, (\Tint est signé aussi), et
si la première valeur contient -2 (codé en \TT{0xFE}) et que nous copions simplement
cet octet dans le conteneur \Tint, cela fait \TT{0x000000FE}, et ceci, pour le type
\Tint représente 254, mais pas -2. Dans un entier signé, -2 est codé en \TT{0xFFFFFFFE}.
Donc, si nous devons transférer \TT{0xFE} depuis une variable de type \Tchar vers
une de type \Tint, nous devons identifier son signe et l'étendre. C'est ce qu'effectue
\MOVSX.

Il est difficile de dire si le compilateur doit stocker une variable \Tchar dans
\EDX, il pourrait simplement utiliser une partie 8-bit du registre (par exemple \DL).
Apparemment, l'\glslink{register allocator}{allocateur de registre} fonctionne comme
ça.

\myindex{ARM!\Instructions!TEST}

Ensuite nous voyons \TT{TEST EDX, EDX}.
Vous pouvez en lire plus à propos de l'instruction \TEST dans la section concernant
les champs de bit~(\myref{sec:bitfields}).
Ici cette instruction teste simplement si la valeur dans \EDX est égale à 0.

\myparagraph{GCC \NonOptimizing}

Essayons GCC 4.4.1:

\lstinputlisting[style=customasmx86]{patterns/10_strings/1_strlen/10_3_gcc.asm}

\label{movzx}
\myindex{x86!\Instructions!MOVZX}

Le résultat est presque le même qu'avec MSVC, mais ici nous voyons \MOVZX au lieu
de \MOVSX.
\MOVZX signifie \emph{MOV with Zero-Extend} (déplacement avec extension à 0).
Cette instruction copie une valeur 8-bit ou 16-bit dans un registre 32-bit et met
les bits restant à 0.
En fait, cette instructions n'est pratique que pour nous permettre de remplacer cette
paire d'instructions:\\
\TT{xor eax, eax / mov al, [...]}.

D'un autre côté, il est évident que le compilateur pourrait produire ce code: \\
\TT{mov al, byte ptr [eax] / test al, al}---c'est presque le même, toutefois, les
bits les plus haut du registre \EAX vont contenir des valeurs aléatoires.
Mais, admettons que c'est un inconvénient du compilateur----il ne peut pas produire
du code plus compréhensible. À strictement parler, le compilateur n'est pas du tout
obligé de générer du code compréhensible par les humains.

\myindex{x86!\Instructions!SETcc}

La nouvelle instruction suivante est \SETNZ.
Ici, si \AL ne contient pas zéro, \TT{test al, al} met le flag \ZF à 0, mais \SETNZ,
si \TT{ZF==0} (\emph{NZ} signifie \emph{not zero}, non zéro) met \AL à 1. En langage
naturel, \emph{si \AL n'est pas zéro, sauter en loc\_80483F0}. Le compilateur génère
du code redondant, mais n'oublions pas qu'il n'est pas en mode optimisation.

\myparagraph{MSVC \Optimizing}
\label{strlen_MSVC_Ox}

Maintenant, compilons tout cela avec MSVC 2012, avec le flag d'optimisation (\Ox):

\lstinputlisting[caption=MSVC 2012 \Optimizing /Ob0,style=customasmx86]{patterns/10_strings/1_strlen/10_2_FR.asm}

C'est plus simple maintenant. Inutile de préciser que le compilateur ne peut utiliser
les registres aussi efficacement que dans une petite fonction, avec peu de variables
locales.

\myindex{x86!\Instructions!INC}
\myindex{x86!\Instructions!DEC}
\INC/\DEC---sont des instructions de \glslink{increment}{incrémentation}/\glslink{decrement}{décrémentation},
en d'autres mots: ajouter ou soustraire 1 d'une/à une variable.

\clearpage
\subsubsection{MSVC: x86 + \olly}

Essayons de hacker notre programme dans \olly, pour le forcer à penser que \scanf
fonctionne toujours sans erreur.
Lorsque l'adresse d'une variable locale est passée à \scanf, la variable contient
initiallement toujours des restes de données aléatoires, dans ce cas \TT{0x6E494714}:

\begin{figure}[H]
\centering
\myincludegraphics{patterns/04_scanf/3_checking_retval/olly_1.png}
\caption{\olly: passer l'adresse de la variable à \scanf}
\label{fig:scanf_ex3_olly_1}
\end{figure}

\clearpage
Lorsque \scanf s'exécute dans la console, entrons quelque chose qui n'est pas du
tout un nombre, comme \q{asdasd}.
\scanf termine avec 0 dans \EAX, ce qui indique qu'une erreur s'est produite.

Nous pouvons vérifier la variable locale dans le pile et noter qu'elle n'a pas changé.
En effet, qu'aurait écrit \scanf ici?
Elle n'a simplement rien fait à part renvoyer zéro.

Essayons de \q{hacker} notre programme.
Clique-droit sur \EAX,
parmi les options il y a \q{Set to 1} (mettre à 1).
C'est ce dont nous avons besoin.

Nous avons maintenant 1 dans \EAX, donc la vérification suivante va s'exécuter comme
souhaiter et \printf va afficher la valeur de la variable dans la pile.

Lorsque nous lançons le programme (F9) nous pouvons voir ceci dans la fenêtre
de la console:

\lstinputlisting[caption=fenêtre console]{patterns/04_scanf/3_checking_retval/console.txt}

En effet, 1850296084 est la représentation en décimal du nombre dans la pile (\TT{0x6E494714})!


\myparagraph{GCC \Optimizing}

Regardons ce que génère GCC 4.4.1 avec l'option d'optimisation \Othree:

\lstinputlisting[style=customasmx86]{patterns/10_strings/1_strlen/10_3_gcc_O3.asm}

\TT{mov dl, byte ptr [eax]}.
Ici GCC génère presque le même code que MSVC, à l'exception de la présence de \MOVZX.
Toutefois, ici, \MOVZX pourrait être remplacé par\\
\TT{mov dl, byte ptr [eax]}.

Peut-être est-il plus simple pour le générateur de code de GCC se se \emph{rappeler}
que le registre 32-bit \EDX est alloué entièrement pour une variable \Tchar et il
est sûr que les bits en partie haute ne contiennent pas de bruit indéfini.

\label{strlen_NOT_ADD}
\myindex{x86!\Instructions!NOT}
\myindex{x86!\Instructions!XOR}

Après cela, nous voyons une nouvelle instruction---\NOT. Cette instruction inverse
tout les bits de l'opérande. \\
Elle peut être vu comme un synonyme de l'instruction \TT{XOR ECX, 0ffffffffh}.
\NOT et l'instruction suivante \ADD calcule la différence entre les pointeurs et
soustrait 1, d'une façon différente.
Au début, \ECX, où le pointeur sur \emph{str} est stocké, est inversé et 1 en est soustrait.

En d'autres mots, à la fin de la fonction juste après le corps de la boucle, ces opérations
sont exécutées:

\begin{lstlisting}[style=customc]
ecx=str;
eax=eos;
ecx=(-ecx)-1;
eax=eax+ecx
return eax
\end{lstlisting}

\dots~et ceci est effectivement équivalent à:

\begin{lstlisting}[style=customc]
ecx=str;
eax=eos;
eax=eax-ecx;
eax=eax-1;
return eax
\end{lstlisting}

Pourquoi est-ce que GCC décide que cela est mieux? Difficile à deviner.
Mais peut-être que les deux variantes sont également efficaces.
