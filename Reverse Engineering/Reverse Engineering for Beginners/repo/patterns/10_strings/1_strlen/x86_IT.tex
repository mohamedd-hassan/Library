\subsubsection{x86}

\myparagraph{\NonOptimizing MSVC}

Compiliamolo:

\lstinputlisting[style=customasmx86]{patterns/10_strings/1_strlen/10_1_msvc_IT.asm}

\myindex{x86!\Instructions!MOVSX}
\myindex{x86!\Instructions!TEST}

Qui, abbiamo due nuove istruzioni: \MOVSX e \TEST.

\label{MOVSX}

La prima---\MOVSX---prende un byte da un indirizzo in memoria e salva il valore in un registro a 32-bit. 
\MOVSX sta per \emph{MOV with Sign-Extend}. 
\MOVSX imposta i restanti bit, dal 8 al 31, 
a 1 se il byte sorgente è \emph{negativo} o a 0 se \emph{positivo}.

Vediamo il perchè.

Di default, il tipo \Tchar è con segno in MSVC e GCC. Se abbiamo due valori, uno dei quali è \Tchar 
mentre l' altro è \Tint, (anche \Tint è con segno), se il primo valore contiene -2 (codificato come \TT{0xFE}) 
e copiamo il byte nel contenitore \Tint, sarebbe \TT{0x000000FE} e ciò 
dal punto di vista di un \Tint con segno è 254, non -2. Negli interi con segno, -2 è codificato come \TT{0xFFFFFFFE}. 
Quindi se vogliamo trasferire \TT{0xFE} da una variabile di tipo \Tchar a \Tint, 
dobbiamo identficare il suo segno estenderlo. Questo è ciò che fa \MOVSX.

E difficile dire se il compilatore necessiti di salvare una varibile \Tchar in \EDX, potrebbe prendere solo la parte a 8-bit di un registro 
(per esempio \DL). Apparentemente, il \gls{register allocator} del compilatore funziona così.

\myindex{ARM!\Instructions!TEST}

Dopodichè vediamo \TT{TEST EDX, EDX}. 
Maggiori dettagli riguardo all' istruzione \TEST nella sezione dei campi di bit~(\myref{sec:bitfields}).
In questo caso, questa istruzione controlla solamente se il valore in \EDX è pari a 0.

\myparagraph{\NonOptimizing GCC}

Proviamo GCC 4.4.1:

\lstinputlisting[style=customasmx86]{patterns/10_strings/1_strlen/10_3_gcc.asm}

\label{movzx}
\myindex{x86!\Instructions!MOVZX}

Il risultato è quasi lo stesso di MSVC, ma qui possiamo notare \MOVZX al posto di \MOVSX. 
\MOVZX sta per \emph{MOV with Zero-Extend}. 
Questa istruzione copia un valore a 8 o 16 bit in un registro a 32-bit e imposta i restanti bit a 0. 
Infatti, questa istruzione è conveniente solo perchè ci permette di rimpiazzare questa coppia di istruzioni:\\
\TT{xor eax, eax / mov al, [...]}.

D' altronde, è ovvio che il compilatore possa produrre questo codice:\\
\TT{mov al, byte ptr [eax] / test al, al}---è quasi lo stesso, tuttavia, 
i bit più alti del registro \EAX conterranno rumore casuale. 
Ma supponiamo sia un ostacolo del compilatore---non potrebbe più produrre codice leggibile. 
Parlando francamente, il compilatore non è obbligato ad emettere codice del tutto comprensibile (agli umani).

\myindex{x86!\Instructions!SETcc}

La prossima nuova istruzione è \SETNZ. 
In questo caso, se \AL non contiene zero, \TT{test al, al} 
imposta la flag \ZF a 0, ma \SETNZ, se \TT{ZF==0} (\emph{NZ} sta per \emph{not zero}) imposta \AL a 1.
Parlando con un liguaggio naturale, \emph{se \AL non è zero, salta a loc\_80483F0}. 
Il compilatore emette molto codice rindondante, ma non dimentichiamo che le ottimizzazioni sono spente.

\myparagraph{\Optimizing MSVC}
\label{strlen_MSVC_Ox}

Ora compiliamo il tutto in MSVC 2012, con le ottimizzazioni attivate (\Ox):

\lstinputlisting[caption=\Optimizing MSVC 2012 /Ob0,style=customasmx86]{patterns/10_strings/1_strlen/10_2_IT.asm}

Ora è tutto più semplice.
Inutile dire che il compilatore può usare i registri con tale efficienza 
solo in piccole funzioni con poche variabili locali.

\myindex{x86!\Instructions!INC}
\myindex{x86!\Instructions!DEC}
\INC/\DEC---sono le istruzioni \gls{increment}/\gls{decrement}, in altre parole: aggiunge o sottrae 1 a/da una variable.

\clearpage
\myparagraph{\Optimizing MSVC + \olly}
\myindex{\olly}

Testiamo questo esempio (ottimizzato) in \olly.  Questa è la prima iterazione:

\begin{figure}[H]
\centering
\myincludegraphics{patterns/10_strings/1_strlen/olly1.png}
\caption{\olly: inizio prima iterazione}
\label{fig:strlen_olly_1}
\end{figure}

Notiamo che \olly ha trovato un ciclo e per convenienza, ha \emph{avvolto} le sue istruzioni dentro le parentesi.
Cliccando con il tasto destro su \EAX e scegliendo 
\q{Follow in Dump}, la finestra della memoria scorrerà fino al punto giusto.
Possiamo vedere la stringa \q{hello!} in memoria.
C'è almeno
uno zero byte dopo la stringa e poi spazzatura casuale.

Se \olly vede un registro contenente un indirizzo valido, che punta ad una stringa,  
lo mostra come stringa.

\clearpage
Premiamo F8 (\stepover) un paio di volte, per arrivare all' inizio del corpo del ciclo:

\begin{figure}[H]
\centering
\myincludegraphics{patterns/10_strings/1_strlen/olly2.png}
\caption{\olly: inizio seconda iterazione}
\label{fig:strlen_olly_2}
\end{figure}

Notiamo che \EAX contiene l'indirizzo del secondo carattere nella stringa.

\clearpage

Dobbiamo premere F8 un numero di volte sufficente per uscire dal ciclo:

\begin{figure}[H]
\centering
\myincludegraphics{patterns/10_strings/1_strlen/olly3.png}
\caption{\olly: differenza di puntatori da calcolare}
\label{fig:strlen_olly_3}
\end{figure}

Notiamo che ora \EAX contiene l'indirizzo dello zero byte che si trova subito dopo la stringa più 1 (perché INC EAX è stato eseguito indipendentemente dal fatto che siamo usciti o meno dal ciclo).
Nel frattempo, \EDX non è cambiato,
quindi sta ancora puntando all' inizio della stringa.

La differenza tra questi due indirizzi verrà calcolata ora.

\clearpage
L' istruzione \SUB è stata appena eseguita:

\begin{figure}[H]
\centering
\myincludegraphics{patterns/10_strings/1_strlen/olly4.png}
\caption{\olly: \EAX  sta venendo decrementato}
\label{fig:strlen_olly_4}
\end{figure}

La differenza tra i puntatori, in questo momento, si trova nel registro \EAX---7.
In realtà, la lunghezza della stringa \q{hello!} è 6, 
ma con lo zero byte incluso---7.
Ma \TT{strlen()} deve ritornare il numero di caratteri nella stringa, diversi da zero.
Quindi viene eseguito un decremento, dopodichè la funziona ritorna.


\myparagraph{\Optimizing GCC}

Vediamo GCC 4.4.1 con le ottimizzazioni attivate (\Othree key):

\lstinputlisting[style=customasmx86]{patterns/10_strings/1_strlen/10_3_gcc_O3.asm}
 
Qui GCC è quasi lo stesso di MSVC, eccetto per la presenza di \MOVZX.
Tuttavia, in questo caso \MOVZX può essere rimpiazzato con\\
\TT{mov dl, byte ptr [eax]}.

Forse è più semplice per il generatore di codice di GCC \emph{ricordare} 
che l' intero registro \EDX a 32-bit 
è stato allocato per una variabile \Tchar e quindi è sicuro che i bit più alti 
non contengono rumore in nessun momento.

\label{strlen_NOT_ADD}
\myindex{x86!\Instructions!NOT}
\myindex{x86!\Instructions!XOR}

Dopodichè vediamo una nuova istruzione---\NOT. Questa istruzione inverte tutti i bit nell' operando. \\
Possiamo dire che è sinonimo dell' istruzione \TT{XOR ECX, 0ffffffffh}. 
\NOT e il seguente \ADD calcolano la differenza di puntatori e sottraggono 1, solamente in maniera diversa. 
All' inizio \ECX, dove è salvato il puntatore a \emph{str}, viene invertito e gli viene sottratto 1.

In altre parole, alla fine della funzione, appena prima del corpo del ciclo, vengono eseguite queste istruzioni:

\begin{lstlisting}[style=customc]
ecx=str;
eax=eos;
ecx=(-ecx)-1; 
eax=eax+ecx
return eax
\end{lstlisting}

\dots~che effettivamente è equivalente a:

\begin{lstlisting}[style=customc]
ecx=str;
eax=eos;
eax=eax-ecx;
eax=eax-1;
return eax
\end{lstlisting}

Perchè GCC ha deciso che è meglio così? Difficile da dire.
Ma forse entrambe le varianti hanno efficenza equivalente.
