\clearpage
\subsubsection{MSVC + \olly}
\myindex{\olly}

Chargeons notre exemple dans \olly et mettons un point d'arrêt sur \comp.
Nous voyons comment les valeurs sont comparées lors du premier appel de \comp:

\begin{figure}[H]
\centering
\myincludegraphics{patterns/18_pointers_to_functions/olly1.png}
\caption{\olly: premier appel de \comp}
\label{fig:qsort_olly1}
\end{figure}

\olly montre les valeurs comparées dans la fenêtre sous celle du code, par commodité.
Nous voyons que \ac{SP} pointe sur \ac{RA}, où se trouve la fonction \qsort (dans
\TT{MSVCR100.DLL}).

\clearpage
En traçant (F8) jusqu'à l'instruction \TT{RETN} et appuyant sur F8 une fois de plus,
nous retournons à la fonction \qsort:

\begin{figure}[H]
\centering
\myincludegraphics{patterns/18_pointers_to_functions/olly2.png}
\caption{\olly: le code dans \qsort juste après l'appel de \comp}
\label{fig:qsort_olly2}
\end{figure}

Ça a été un appel à la fonction de comparaison.

\clearpage
Voici aussi une copie d'écran au moment du second appel à\comp{}---maintenant les
valeurs qui doivent être comparées sont différentes:

\begin{figure}[H]
\centering
\myincludegraphics{patterns/18_pointers_to_functions/olly3.png}
\caption{\olly: second appel de \comp}
\label{fig:qsort_olly3}
\end{figure}
