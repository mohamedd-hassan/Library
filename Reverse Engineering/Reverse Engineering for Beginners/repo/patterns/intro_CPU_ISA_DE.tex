\subsection{Eine kurze Einführung in die CPU}

Die \ac{CPU} ist die Komponente, die den Maschinencode ausführt aus dem ein Programm besteht.

\textbf{Ein kurzes Glossar:}

\begin{description}
\item[Befehl]: Ein einfaches \ac{CPU}-Kommando.
Die einfachsten Beispiele hierfür sind: Verschieben von Daten zwischen Registern, Arbeiten mit Speicher,
einfache arithmetische Operationen.
In der Regel hat jede \ac{CPU} ihre eigene Befehlssatz-Architektur (\ac{ISA}).

\item[Maschinencode]: Code den die \ac{CPU} direkt verarbeitet. 
Jeder Befehl ist in der Regel durch ein paar Byte kodiert.
\item[Assembler-Sprache]: Mnemonics und einige Makro-ähnliche Erweiterungen um das Leben der Programmierer zu erleichtern.
\item[CPU-Register]: Jede \ac{CPU} hat eine feste Anzahl von Mehrzweck-Registern (\ac{GPR}).
$\approx 8$ bei x86, $\approx 16$ bei x86-64, $\approx 16$ bei ARM.
Die einfachste Möglichkeit um Register zu verstehen, ist sie als typlose, temporäre Variable zu betrachten.
Stellen Sie sich vor Sie würden mit einer Hochsprache arbeiten und könnten nur acht 32-Bit (oder 64-Bit) Variablen
nutzen. Dennoch ist damit eine Menge möglich!
\end{description}

% TODO1 add about linker: "компоновщик" и "редактор связей" в русскоязычной лит-ре

% A note on the experiments in this area (like the LISP machines http://en.wikipedia.org/wiki/Lisp_machine
% might be useful
Man kann sich jetzt fragen, warum die Unterscheidung zwischen Maschinencode und einer Hochsprache notwendig ist.
Die Antwort liegt einfach in der Tatsache, dass Menschen und \ac{CPU}s nicht gleich sind---%
Es ist sehr viel einfacher für Menschen eine Programmiersprache wie \CCpp, Java, Python, usw. zu lesen. Für eine \ac{CPU} jedoch
ist es einfacher eine geringere Abstraktion zu verarbeiten.

Vielleicht wäre es möglich eine \ac{CPU} zu entwickeln die direkt eine Hochsprache ausführen kann,
diese wäre aber sehr viel komplexer als es heute der Fall ist.
In ähnlicher Weise ist es für Menschen äußerst unkomfortabel in einer Assembler-Sprache zu programmieren.
Diese ist sehr hardwarenah und kaum zu realisieren ohne schwer zu findende Fehler zu machen.
Das Programm, welches die Hochsprache in Assembler konvertiert nennt sich \emph{Compiler} oder Übersetzer.

\myindex{ARM!\ARMMode}%
\myindex{ARM!\ThumbMode}%
\myindex{ARM!\ThumbTwoMode}%

\subsubsection{Einige Anmerkungen zu den verschiedenen \ac{ISA}s}
Die x86 \ac{ISA} hatte immer Opcodes variabler Länge. Als die 64-Bit-Ära aufkam beeinflussten deren Erweiterungen die Befehlssatz-Architektur
nicht sehr stark. Tatsächlich enthält die x86-Architektur immer noch viele Befehle, die zunächst in den 16-bit 8086 CPU implementiert
wurden und noch immer in aktuellen CPU enthalten sind.
ARM ist eine \ac{RISC} \ac{CPU} mit einer konstanten Opcode-Länge, die in der Vergangenheit einige Vorteile aufwies.
In den Anfängen waren alle ARM-Befehle in vier Byte kodiert%
\footnote{
Befehle mit fester Länge sind ürigens so praktisch weil die nächste (oder vorherige) 
Befehlsadresse ohne großen Aufwand berechnet werden kann. Diese Eigenschaft wird im Kapitel switch() diskutiert.}.
Dies ist nun als \q{ARM-Mode} bekannt.
Irgendwann dachte ARM, dass dieser Vorgehensweise nicht so sparsam war wie zuerst angenommen.
Tatsächlich, benötigen die meisten Befehle in der Praxis\footnote{zum Beispiel MOV/PUSH/CALL/Jcc} weniger Platz für die Kodierung.
Darum wurde ein Befehlssatz, genannt Thumb, eingeführt, in dem jeder Befehl in zwei Byte kodiert wird.
Dies ist nun als \q{Thumb-Mode} bekannt.
Trotzdem können nicht \emph{alle} ARM-Befehle in nur zwei Byte kodiert werden, was den Thumb Befehlssatz in gewisser Weise einschränkt.
Erwähnenswert ist, dass Code der für ARM- und Thumb-Mode kompiliert wurde in einem einzelnen Programm vermischt sein kann.
Die ARM-Erfinder beschlossen den Thumb-Mode zu erweitern, was zu Thumb-2 führte, der in ARMv7 auftaucht.
Thumb-2 nutzt immer noch 2-Byte-Befehle, enthält aber zusätzlich auch einige 4-Byte-Befehle.
Ein verbreitetes Missverständnis ist, dass Thumb-2 eine Mischung aus und Thumb und ARM ist. Dies ist falsch.
Stattdessen wurde Thumb-2 erweitert um alle Prozessor-Eigenschaften zu unterstützen, so dass er mit dem ARM-Mode konkurrieren kann---ein
Ziel, dass klar erreicht wurde, da der Großteil der \idevices mit dem Thumb-2-Befehlssatz kompiliert werden (zugegebenermaßen, auch durch
die Tatsache, dass Xcode dies als Standard-Einstellung so macht).
Später erschienen die 64-bit ARM. Diese \ac{ISA} hat 4-Byte-Opcodes und benötigt den traditionellen Thumb-Mode nicht mehr.
Die 64-Bit-Anforderungen beeinflussten die \ac{ISA}, was uns nun zu drei ARM-Befehlssätzen führt: ARM-Mode, Thumb-Mode (inklusive Thumb-2) und ARM64.
Diese Befehlssätze überschneiden sich teilweise, dennoch sind sie eher eigenständig als Variationen eines einzelnen Befehlssatz.
Aus diesem Grund wird versucht in diesem Buch von allen drei Varianten Code-Beispiele zu zeigen.
\myindex{PowerPC}%
\myindex{MIPS}%
\myindex{Alpha AXP}%
Es gibt übrigens noch eine Menge anderer \ac{RISC} \ac{ISA}s mit einer konstanten Opcode-Länge von 32 Bit, wie MIPS, PowerPC und Alpha AXP.
