\subsection{Rückgabe Adresse der Funktion speichern}

\myparagraph{x86}

\myindex{x86!\Instructions!CALL}
Wenn man eine Funktion mit der \CALL Instruktion aufruft, wird die Adresse direkt nach der
\CALL Instruktion auf dem Stack gespeichert und der unbedingte jump wird ausgeführt.

\myindex{x86!\Instructions!PUSH}
\myindex{x86!\Instructions!JMP}
Die \CALL Instruktion ist äquivalent zu dem \INS{PUSH address\_after\_call / JMP operand} Instruktions paar.

\myindex{x86!\Instructions!RET}
\myindex{x86!\Instructions!POP}
\RET ruft die Rückkehr Adresse vom Stack ab und springt zu dieser~---was äquivalent zu einem \TT{POP tmp / JMP tmp} Instruktions
paar ist.

\myindex{\Stack!\MLStackOverflow}
\myindex{\Recursion}

Den Stack zum überlaufen zu bringen ist recht einfach, einfach eine 
endlos rekursive Funktion Aufrufen:


\begin{lstlisting}[style=customc]
void f()
{
	f();
};
\end{lstlisting}


MSVC 2008 hat eine Erkennung für das Problem:


\begin{lstlisting}
c:\tmp6>cl ss.cpp /Fass.asm
Microsoft (R) 32-bit C/C++ Optimizing Compiler Version 15.00.21022.08 for 80x86
Copyright (C) Microsoft Corporation.  All rights reserved.

ss.cpp
c:\tmp6\ss.cpp(4) : warning C4717: 'f' : recursive on all control paths, function will cause runtime stack overflow
\end{lstlisting}

\dots aber der Compiler erzeugt den Code trotzdem:

\lstinputlisting[style=customasmx86]{patterns/02_stack/1.asm}

\dots Auch wenn wir die Compiler Optimierungen einschalten (\TT{/0x} Option) wird der optimierte Code nicht
den Stack zum überlaufen bringen. Stattdessen wird der Code \emph{korrekt}\footnote{Ironie hier} ausgeführt: 

\lstinputlisting[style=customasmx86]{patterns/02_stack/2.asm}

GCC 4.4.1 generiert vergleichbaren Code in beiden Fällen, jedoch ohne über das Overflow Problem zu warnen.

\myparagraph{ARM}

\myindex{ARM!\Registers!Link Register}

ARM Programme benutzen den Stack um Rücksprung Adressen zu speichern, aber anders.
Wie bereits erwähnt in \q{\HelloWorldSectionName}~(\myref{sec:hw_ARM}),
wird der \ac{RA} Wert im \ac{LR} (\gls{link register}) gespeichert.
Wenn nun eine andere Funktion aufgerufen werden muss und auf das \ac{LR} Register 
zu greift, muss der aktuelle Wert im Register irgendwo gespeichert werden.

\myindex{Funktion Prologe}
Normal wird der Wert im Funktion Prolog gespeichert.

\myindex{ARM!\Instructions!PUSH}
\myindex{ARM!\Instructions!POP}

Oft sieht man Instruktionen wie z.B \INS{PUSH {R4-R7,LR}} zusammen mit dieser Instruktion im 
Epilog \INS{POP {R4-R7,PC}}---Somit werden Werte die in den Funktionen benötigt werden auf dem 
Stack gespeichert, inklusive \ac{LR}.

\myindex{ARM!Leaf Funktion}
Wenn eine Funktion nie eine andere Funktion aufruft, nennt man das in der \ac{RISC} Terminologie eine
\emph{\glslink{leaf function}{leaf Funktion}}\footnote{\href{http://infocenter.arm.com/help/index.jsp?topic=/com.arm.doc.faqs/ka13785.html}{infocenter.arm.com/help/index.jsp?topic=/com.arm.doc.faqs/ka13785.html}}.  % <-- attention could be a compilier bug
Als Konsequenz ergibt sich, das leaf Funktionen nicht das \ac{LR} Register speichern (da sie es nicht modifizieren).
Wenn solche Funktionen klein sind und nur eine geringe Anzahl an Registern benutzt, ist es möglich das der Stack
gar nicht benutzt wird. Es ist also möglich leaf Funktionen zu benutzen ohne den Stack zurück zu greifen, die Ausführung
ist hier schneller als auf älteren x86 Maschinen weil kein externer RAM für den Stack benutzt wird 
\footnote{Bis vor einer weile war es sehr teuer auf PDP-11 und VAX Maschinen die CALL Instruktion zu benutzen; bis zu 50\%
der Rechenzeit wurde allein für diese Instruktion verschwendet, man hat dabei festgestellt das eine große Anzahl an kleinen
Funktionen zu haben ein \gls{anti-pattern} \InSqBrackets{\TAOUP Chapter 4, Part II}.} ist.
Diese Eigenschaft kann nützlich sein wenn der Speicher für den Stack noch nicht alloziert oder verfügbar ist.

Ein paar Beispiele für leaf Funktionen:

\myref{ARM_leaf_example1}, \myref{ARM_leaf_example2}, 
\myref{ARM_leaf_example3}, \myref{ARM_leaf_example4}, \myref{ARM_leaf_example5},
\myref{ARM_leaf_example6}, \myref{ARM_leaf_example7}, \myref{ARM_leaf_example10}.

