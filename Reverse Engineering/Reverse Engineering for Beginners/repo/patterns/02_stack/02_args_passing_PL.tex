\subsubsection{Przekazywanie argumentów funkcji}

Najpopularniejszy sposób na przekazywanie parametrów funkcji w x86 to \q{cdecl}:

\begin{lstlisting}[style=customasmx86]
push arg3
push arg2
push arg1
call f
add esp, 12 ; 4*3=12
\end{lstlisting}

Wywoływana funkcja pobiera swoje argumentu za pomocą wskaźnik stosu.

Stos, przed wykonaniem pierwszej instrukcji z \ttf{}, wygląda następująco:

\begin{center}
\begin{tabular}{ | l | l | }
\hline
ESP & adres powrotu \\
\hline
ESP+4 & \argument \#1, \MarkedInIDAAs{} \TT{arg\_0} \\
\hline
ESP+8 & \argument \#2, \MarkedInIDAAs{} \TT{arg\_4} \\
\hline
ESP+0xC & \argument \#3, \MarkedInIDAAs{} \TT{arg\_8} \\
\hline
\dots & \dots \\
\hline
\end{tabular}
\end{center}

Opis innych konwencji wywoływania funkcji znajduje się tutaj: ~(\myref{sec:callingconventions}).

\par Nawiasem mówiąc, funkcja wywoływana nie posiada informacji o liczbie przekazywanych do niej argumentów.
Funkcja \printf, która może mieć zmienną liczbę argumentów, ustala ich liczbę za pomocą  specyfikatorów formatu (rozpoczynających się od znaku \%).

Jeśli napiszemy:

\begin{lstlisting}
printf("%d %d %d", 1234);
\end{lstlisting}

\printf wypisze 1234, a następnie jeszcze dwie losowe\footnote{Tak na prawdę nie są one losowe, patrz: \myref{noise_in_stack}} liczby, który przypadkowo znalazły się na stosie obok.

\label{main_arguments}
\par
Dlatego nie ma znaczenia jak zapiszemy funkcję \main{}:\\
jak \main{}, \TT{main(int argc, char *argv[])}\\
lub \TT{main(int argc, char *argv[], char *envp[])}.

W rzeczywistości kod z \ac{CRT} wywołuje \main mniej więcej tak:
	
\begin{lstlisting}[style=customasmx86]
push envp
push argv
push argc
call main
...
\end{lstlisting}

Jeśli zadeklarujesz \main bez argumentów, one i tak będą na stosie, lecz nie zostaną wykorzystane.
Jeśli zadeklarujesz \main jako \TT{main(int argc, char *argv[])},
to będziesz mógł skorzystać z pierwszych dwóch argumentów, a trzeci będzie dla funkcji \q{niewidoczny}.
Co więcej, można nawet zadeklarować \TT{main(int argc)} i to również zadziała.

Inny, podobny, przykład: \ref{cdecl_DLL}.

\myparagraph{Alternatywne sposoby na przekazywanie argumentów}

Warto zauważyć, że nic nie zmusza programisty do przekazywanie argumentów przez stos. Nie ma takiego wymagania, można to robić to zupełnie inaczej, nie korzystając ze stosu.

Dość popularnym sposobem wśród początkujących jest przekazywanie argumentów przez zmienne globalne, na przykład:

\lstinputlisting[caption=Kod w asemblerze,style=customasmx86]{patterns/02_stack/global_args.asm}

Ta metoda posiada oczywistą wadę: funkcja \emph{do\_something()} nie może wywołać sama siebie przez rekurencję (lub za pomocą innej funkcji), gdyż musiałaby nadpisać własne argumenty.
To samo dotyczy zmiennych lokalnych, gdyby przechowywać je w zmiennych globalnych, to funkcja nie będzie mogła wywołać sama siebie.
Co więcej, nie jest to bezpieczne w środowisku wielowątkowym\footnote{W poprawnej implementacji
każdy wątek miałby własny stos lokalny, ze swoimi argumentami/zmiennymi.}.
Przechowywania danych na stosie wszystko upraszcza ---
stos może przechować tyle argumentów funkcji/zmiennych,
na ile pozwoli jego rozmiar.

W \InSqBrackets{\TAOCPvolI{}, 189} można przeczytać o jeszcze dziwniejszych metodach przekazywania argumentów, szczególnie wygodnych na IBM System/360.

\myindex{MS-DOS}
\myindex{x86!\Instructions!INT}

W MS-DOS istniała metoda przekazywania argumentów przez rejestry, na przykład ten fragment kodu na wiekowym 16-bitowym MS-DOS
wypisze \q{Hello, world!}:

\begin{lstlisting}[style=customasmx86]
mov  dx, msg      ; adres wiadomości
mov  ah, 9        ; 9 oznacza funkcję "wypisz łańcuch znaków"
int  21h          ; wywołanie systemowe ("syscall") DOS

mov  ah, 4ch      ; funkcja "zakończ program"
int  21h          ; wywołanie systemowe ("syscall") DOS

msg  db 'Hello, World!\$'
\end{lstlisting}

\myindex{fastcall}
Jest to całkiem podobne do metody \myref{fastcall}.
Przypomina to również sposoby korzystania z wywołań systemowych (ang. syscall) na systemach Linuks (\myref{linux_syscall}) i Windows.

\myindex{x86!\Flags!CF}
Jeżeli funkcja w MS-DOS zwraca wartość typu boolean (jeden bit, zwykle oznaczający wystąpienie błędu), to często wykorzystywana jest flaga \TT{CF}.

Na przykład:

\begin{lstlisting}[style=customasmx86]
mov ah, 3ch       ; "3c" oznacza "stwórz plik
lea dx, filename
mov cl, 1
int 21h
jc  error
mov file_handle, ax
...
error:
...
\end{lstlisting}

W razie wystąpienia błędu flaga \TT{CF} zostaje ustawiona i instrukcja \TT{JC} skacze do miejsca oznaczonego etykietą error. W przeciwnym razie uchwyt (ang. \emph{file handle}) stworzonego pliku zwracany jest przez rejestr \TT{AX}.

Ta metoda wciąż jest wykorzystywana przez programistów asemblera.
W kodach źródłowych Windows Research Kernel (który jest bardzo podobny do Windows 2003) możemy znaleźć coś takiego\\
(plik \emph{base/ntos/ke/i386/cpu.asm}):

\begin{lstlisting}[style=customasmx86]
        public  Get386Stepping
Get386Stepping  proc

        call    MultiplyTest            ; Perform multiplication test
        jnc     short G3s00             ; if nc, muttest is ok
        mov     ax, 0
        ret
G3s00:
        call    Check386B0              ; Check for B0 stepping
        jnc     short G3s05             ; if nc, it's B1/later
        mov     ax, 100h                ; It is B0/earlier stepping
        ret

G3s05:
        call    Check386D1              ; Check for D1 stepping
        jc      short G3s10             ; if c, it is NOT D1
        mov     ax, 301h                ; It is D1/later stepping
        ret

G3s10:
        mov     ax, 101h                ; assume it is B1 stepping
        ret

	...

MultiplyTest    proc

        xor     cx,cx                   ; 64K times is a nice round number
mlt00:  push    cx
        call    Multiply                ; does this chip's multiply work?
        pop     cx
        jc      short mltx              ; if c, No, exit
        loop    mlt00                   ; if nc, YEs, loop to try again
        clc
mltx:
        ret

MultiplyTest    endp
\end{lstlisting}



