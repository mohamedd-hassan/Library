\mysection{\Stack}
\label{sec:stack}
\myindex{\Stack}

Lo stack è una delle strutture dati più importanti in informatica
\footnote{\href{http://en.wikipedia.org/wiki/Call_stack}{wikipedia.org/wiki/Call\_stack}}.
\ac{AKA} \ac{LIFO}.

Tecnicamente, è soltanto un blocco di memoria nella memoria di un processo insieme al registro \ESP o \RSP in x86 o x64, o il registro \ac{SP} in ARM, come puntatore all'interno di quel blocco.

\myindex{ARM!\Instructions!PUSH}
\myindex{ARM!\Instructions!POP}
\myindex{x86!\Instructions!PUSH}
\myindex{x86!\Instructions!POP}
Le istruzioni di accesso allo stack più usate sono \PUSH e \POP (sia in x86 che in ARM Thumb-mode).
\PUSH sottrae da \ESP/\RSP/\ac{SP} 4 in modalità 32-bit (oppure 8 in modalità 64-bit) e scrive successivamente il contenuto del suo unico operando nell'indirizzo di memoria puntato da \ESP/\RSP/\ac{SP}.

\POP è l'operazione inversa: recupera il dato dalla memoria a cui punta \ac{SP}, lo carica nell'operando dell'istruzione (di solito un registro)
e successivamente aggiunge 4 (o 8) allo \gls{stack pointer}.

A seguito dell'allocazione dello stack, lo \gls{stack pointer} punta alla base (fondo) dello stack.
\PUSH decrementa lo \gls{stack pointer} e \POP lo incrementa.
La base dello stack è in realtà all'inizio del blocco di memoria allocato per lo stack. Sembra strano, ma è così.

ARM supporta sia stack decrescenti che crescenti.

\myindex{ARM!\Instructions!STMFD}
\myindex{ARM!\Instructions!LDMFD}
\myindex{ARM!\Instructions!STMED}
\myindex{ARM!\Instructions!LDMED}
\myindex{ARM!\Instructions!STMFA}
\myindex{ARM!\Instructions!LDMFA}
\myindex{ARM!\Instructions!STMEA}
\myindex{ARM!\Instructions!LDMEA}

Ad esempio le istruzioni \ac{STMFD}/\ac{LDMFD}, \ac{STMED}/\ac{LDMED} sono fatte per operare con uno stack decrescente (che cresce verso il basso, inizia con un indirizzo alto e prosegue verso il basso).
Le istruzioni \ac{STMFA}/\ac{LDMFA}, \ac{STMEA}/\ac{LDMEA} sono fatte per operare con uno stack crescente (che cresce verso l'alto, da un indirizzo basso verso uno più alto).

% It might be worth mentioning that STMED and STMEA write first,
% and then move the pointer,
% and that LDMED and LDMEA move the pointer first, and then read.
% In other words, ARM not only lets the stack grow in a non-standard direction,
% but also in a non-standard order.
% Maybe this can be in the glossary, which would explain why E stands for "empty".

\subsection{Perchè lo stack cresce al contrario?}
\label{stack_grow_backwards}

Intuitivamente potremmo pensare che lo stack cresca verso l'alto, ovvero verso indirizzi più alti, come qualunque altra struttura dati.

La ragione per cui lo stack cresce verso il basso è probabilmente di natura storica.
Quando i computer erano talmente grandi da occupare un'intera stanza, era facile dividere la memoria in due parti, una per lo
\gls{heap} e l'altra per lo stack.
Ovviamente non era possibile sapere a priori quanto sarebbero stati grandi lo stack e lo \gls{heap} durante l'esecuzione di un programma,
e questa soluzione era la più semplice.

\begin{center}
	\begin{tikzpicture}
	\tikzstyle{every path}=[thick]

	\node [rectangle,draw,minimum width=6cm, minimum height=2cm] (memory) {};
	\node [] [right=0.2cm of memory.west] (heap) {\MLHeap};
	\node [] [left=0.2cm of memory.east] (stack) {\MLStack};

	\node [] (center1) [right=2cm of memory.west] {};
	\node [] (center2) [left=2cm of memory.east] {};

	\draw [->] (heap) -- (center1);
	\draw [->] (stack) -- (center2);

	\node [] [above left=1.1cm and 0.2cm of heap] (t1) {\MLStartOfHeap};
	\node [] [above right=1.1cm and 0.2cm of stack] (t2) {\MLStartOfStack};

	\draw [->] (t1) -- (memory.west);
	\draw [->] (t2) -- (memory.east);

	\end{tikzpicture}
\end{center}


In \RitchieThompsonUNIX possiamo leggere:

\begin{framed}
\begin{quotation}
Il nucleo utente di una immagine è diviso in tre segmenti logici.
Il segmento text del programma inizia in posizione 0 nel virtual address space.
Durante l'esecuzione questo segmento viene protetto da scrittura, ed una sua singola copia viene condivisa tra i processi che eseguono lo stesso programma.
Al primo limite di 8K byte sopra il segmento text del programma, nel virtual address space comincia un segmento dati scrivibile, non condiviso, le cui dimensioni possono essere estese da una chiamata di sistema.A partire dall'indirizzo più alto nel virtual address space c'è lo stack segment, che automaticammente cresce verso il basso al variare dello stack pointer hardware.
\end{quotation}
\end{framed}

Questo ricorda molto come alcuni studenti utilizzino lo stesso quaderno per prendere appunti di due diverse materie:
gli appunti per la prima materia sono scritti normalmente, e quelli della seconda materia sono scritti a partire dalla fine del quaderno, capovolgendolo.
Le note si potrebbero "incontrare" da qualche parte in mezzo al quaderno, nel caso in cui non ci sia abbastanza spazio libero.

% I think if we want to expand on this analogy,
% one might remember that the line number increases as as you go down a page.
% So when you decrease the address when pushing to the stack, visually,
% the stack does grow upwards.
% Of course, the problem is that in most human languages,
% just as with computers,
% we write downwards, so this direction is what makes buffer overflows so messy.

\subsection{Per cosa viene usato lo stack?}

% subsections
\subsubsection{Salvare l'indirizzo di ritorno della funzione}

\myparagraph{x86}

\myindex{x86!\Instructions!CALL}
Quando si chiama una funzione con l'istruzione \CALL, l'indirizzo del punto esattamente dopo la \CALL viene salvato nello stack, e successivamente
viene eseguito un jump non condizionale all'indirizzo dell'operando di \CALL.

\myindex{x86!\Instructions!PUSH}
\myindex{x86!\Instructions!JMP}
L'istruzione \CALL è equivalente alla coppia di istruzioni \INS{PUSH indirizzo\_dopo\_call / JMP operando}.

\myindex{x86!\Instructions!RET}
\myindex{x86!\Instructions!POP}
\RET preleva un valore dallo stack ed effettua un jump ad esso~--- ciò equivale alla coppia di istruzioni \TT{POP tmp / JMP tmp}.

\myindex{\Stack!\MLStackOverflow}
\myindex{\Recursion}

Riempire lo stack fino allo straripamento è semplicissimo. Basta ricorrere alla ricorsione eterna:

\begin{lstlisting}[style=customc]
void f()
{
	f();
};
\end{lstlisting}

MSVC 2008 riporta il problema:

\begin{lstlisting}
c:\tmp6>cl ss.cpp /Fass.asm
Microsoft (R) 32-bit C/C++ Optimizing Compiler Version 15.00.21022.08 for 80x86
Copyright (C) Microsoft Corporation.  All rights reserved.

ss.cpp
c:\tmp6\ss.cpp(4) : warning C4717: 'f' : recursive on all control paths, function will cause runtime stack overflow
\end{lstlisting}

\dots ma genera in ogni caso il codice correttamente:

\lstinputlisting[style=customasmx86]{patterns/02_stack/1.asm}

\dots Se attiviamo le ottimizzazioni del compilatore (\TT{\Ox} option) il codice ottimizzato non causerà overflow dello stack
e funzionerà invece \emph{correttamente}\footnote{sarcasmo, si fa per dire}:

\lstinputlisting[style=customasmx86]{patterns/02_stack/2.asm}

GCC 4.4.1 genera codice simile in antrambi i casi, senza avvertire del problema.

\myparagraph{ARM}

\myindex{ARM!\Registers!Link Register}
Anche i programmi ARM usano lo stack per salvare gli indirizzi di ritorno, ma lo fanno in maniera diversa.
Come detto in \q{\HelloWorldSectionName}~(\myref{sec:hw_ARM}),
il \ac{RA} viene salvato nel \ac{LR} (\gls{link register}).
Se si presenta comunque la necessità di chiamare un'altra funzione ed usare il registro \ac{LR} ancora una volta,
il suo valore deve essere salvato.
\myindex{Function prologue}
Solitamente questo valore viene salvato nel preambolo della funzione.

\myindex{ARM!\Instructions!PUSH}
\myindex{ARM!\Instructions!POP}
Spesso vediamo istruzioni come \INS{PUSH {R4-R7,LR}} insieme ad istruzioni nell'epilogo come
\INS{POP {R4-R7,PC}}---perciò i valori dei registri che saranno usati nella funzione vengono salvati nello stack, incluso \ac{LR}.

\myindex{ARM!Leaf function}
Ciononostante, se una funzione non chiama al suo interno nessun'altra funzione, in terminologia \ac{RISC} è detta
\emph{\gls{leaf function}}, o funzione foglia.\footnote{\href{http://infocenter.arm.com/help/index.jsp?topic=/com.arm.doc.faqs/ka13785.html}{infocenter.arm.com/help/index.jsp?topic=/com.arm.doc.faqs/ka13785.html}}.
Di conseguenza, le leaf functions non salvano il registro \ac{LR} register (perchè difatti non lo modificano).
Se una simile funzione è molto breve e usa un piccolo numero di registri, potrebbe non usare del tutto lo stack.
E' quindi possible chiamare le leaf functions senza usare lo stack, cosa che può essere più veloce rispetto alle vecchie macchine x86 perchè la RAM esterna non viene usata per lo stack
\footnote{Tempo fa, su PDP-11 e VAX, l'istruzione CALL (usata per chiamare altre funzioni) era costosa; poteva richiedere fino al 50\%
del tempo di esecuzione, ed era quindi consuetudine pensare che avere un grande numero di piccole funzioni fosse un \gls{anti-pattern} \InSqBrackets{\TAOUP Chapter 4, Part II}.}.
Lo stesso principio può tornare utile quando la memoria per lo stack non è stata ancora allocata o non è disponibile.

Alcuni esempi di funzioni foglia:
\myref{ARM_leaf_example1}, \myref{ARM_leaf_example2},
\myref{ARM_leaf_example3}, \myref{ARM_leaf_example4}, \myref{ARM_leaf_example5},
\myref{ARM_leaf_example6}, \myref{ARM_leaf_example7}, \myref{ARM_leaf_example10}.

\subsubsection{Passaggio di argomenti alle funzioni}

Il modo più diffuso per passare parametri in x86 è detto \q{cdecl}:

\begin{lstlisting}[style=customasmx86]
push arg3
push arg2
push arg1
call f
add esp, 12 ; 4*3=12
\end{lstlisting}

La funzioni chiamate, \Gls{callee}, ricevono i propri argomenti tramite lo stack pointer.

Quindi è così che i valori degli argomenti sono posizionati nello stack prima dell'esecuzione della prima istruzione della funzione \ttf{}:

\begin{center}
\begin{tabular}{ | l | l | }
\hline
ESP & return address \\
\hline
ESP+4 & \argument \#1, \MarkedInIDAAs{} \TT{arg\_0} \\
\hline
ESP+8 & \argument \#2, \MarkedInIDAAs{} \TT{arg\_4} \\
\hline
ESP+0xC & \argument \#3, \MarkedInIDAAs{} \TT{arg\_8} \\
\hline
\dots & \dots \\
\hline
\end{tabular}
\end{center}

Per ulteriori informazioni su altri tipi di convenzioni di chiamata (calling conventions), fare riferimento alla sezione~(\myref{sec:callingconventions}).

\par
A proposito, la funzione \gls{callee}{chiamata} non possiede alcuna informazione su quanti argomenti sono stati passati.
Le funzioni C con un numero variabile di argomenti (come \printf) determinano il loro numero attraverso specificatori di formato stringa (che iniziano con il simbolo \%).
% to be sync: C functions with a variable number of arguments (like \printf) can determine their number using format string specifiers (which begin with the \% symbol).

Se scriviamo qualcosa come:

\begin{lstlisting}
printf("%d %d %d", 1234);
\end{lstlisting}

\printf scriverà 1234, e successivamente due numeri casuali\footnote{Non casuali in senso stretto, ma piuttosto non predicibili: \myref{noise_in_stack}}, che si trovavano lì vicino nello stack.

\label{main_arguments}
\par
Per questo motivo non è molto importante come dichiariamo la funzione \main: come \main, \\
\TT{main(int argc, char *argv[])} oppure \TT{main(int argc, char *argv[], char *envp[])}.

Infatti, il codice \ac{CRT} sta chiamando \main circa in questo modo:

\begin{lstlisting}[style=customasmx86]
push envp
push argv
push argc
call main
...
\end{lstlisting}

Se dichiari \main come \main senza argomenti, questi sono, in ogni caso, ancora presenti nello stack, ma non vengono utilizzati.
Se dichiari \main come \TT{main(int argc, char *argv[])},
sarai in grado di utilizzare i primi due argomenti, ed il terzo rimarrà \q{invisibile} per la tua funzione.
In più, è possibile dichiarare \TT{main(int argc)}, e continuerà a funzionare.

% TBT Another related example: \ref{cdecl_DLL}.

\myparagraph{Metodi alternativi per passare argomenti}

Vale la pena notare che non c'è nulla che obbliga il programmatore a passare gli argomenti attraverso lo stack. Non è un requisito necessario.
Si potrebbe implementare un qualunque altro metodo anche senza usare per niente lo stack.

Un metodo abbastanza popolare tra chi inizia a programmare in linguaggio assembly language è di passare argomenti attraverso variabili globali, in questo modo:

\lstinputlisting[caption=Assembly code,style=customasmx86]{patterns/02_stack/global_args.asm}

Tuttavia questo metodo ha un limite evidente: la funzione \emph{do\_something()} non può richiamare sè stessa in modo ricorsivo (o attraverso un'altra funzione),
perchè deve cancellare i suoi stessi argomenti.
Lo stesso accade con le variabili locali: se le tieni in variabili globali, la funzione non può chiamare se stessa.
Inoltre questo non sarebbe thread-safe
\footnote{Implementato correttamente, ciascun thread avrebbe il suo proprio stack con i suoi argomenti/variabili.}.
Il metodo di memorizzare queste informazioni nello stack rende il tutto più semplice---può mantenere quanti argomenti di funzione e/o valori,
quanto spazio è disponibile.

\InSqBrackets{\TAOCPvolI{}, 189} menziona alcuni schemi ancora più strani e particolarmente convenienti su IBM System/360.

\myindex{MS-DOS}
\myindex{x86!\Instructions!INT}

MS-DOS utilizzava un modo per passare tutti gli argomenti di funzione via registri, ad esempio, in questo pezzo
di codice per MS-DOS a 16 bit scrive ``Hello, world!'':

\begin{lstlisting}[style=customasmx86]
mov  dx, msg      ; indirizzo del messaggio
mov  ah, 9        ; 9 indica la funzione "print string"
int  21h          ; "syscall" (chiamata di sistema) DOS

mov  ah, 4ch      ; funzione "termina il programma"
int  21h          ; "syscall" DOS

msg  db 'Hello, World!\$'
\end{lstlisting}

\myindex{fastcall}
Questo è abbastanza simile al metodo \myref{fastcall}.
Ed è inoltre molto simile alle chiamate syscalls in Linux (\myref{linux_syscall}) e Windows.

\myindex{x86!\Flags!CF}
Se una funzione MS-DOS restituisce un valore di tipo boolean (cioè, un singolo bit, di solito per indicare uno stato di errore),
il flag \TT{CF} era spesso utilizzato.

Ad esempio:

\begin{lstlisting}[style=customasmx86]
mov ah, 3ch       ; crea file
lea dx, filename
mov cl, 1
int 21h
jc  error
mov file_handle, ax
...
error:
...
\end{lstlisting}

In caso di errore, il flag \TT{CF} viene innalzato. Altrimenti, l'handle ad un nuovo file creato viene restituito attraverso \TT{AX}.

Questo metodo viene ancora utilizzato dai programmatori assembly.
Nel codice sorgente del Windows Research Kernel (che è abbastanza simile a Windows 2003) possiamo trovare qualcosa tipo:
(file \emph{base/ntos/ke/i386/cpu.asm}):

\begin{lstlisting}[style=customasmx86]
        public  Get386Stepping
Get386Stepping  proc

        call    MultiplyTest            ; Esegue test di moltiplicazione
        jnc     short G3s00             ; se nc, muttest è ok
        mov     ax, 0
        ret
G3s00:
        call    Check386B0              ; Verifica B0 stepping
        jnc     short G3s05             ; se nc, è B1/later
        mov     ax, 100h                ; è B0/earlier stepping
        ret

G3s05:
        call    Check386D1              ; Verifica D1 stepping
        jc      short G3s10             ; se c, non è D1
        mov     ax, 301h                ; è D1/later stepping
        ret

G3s10:
        mov     ax, 101h                ; suppone che sia B1 stepping
        ret

	...

MultiplyTest    proc

        xor     cx,cx                   ; 64K volte è un bel numero tondo
mlt00:  push    cx
        call    Multiply                ; la moltiplicazione funziona in questo chip?
        pop     cx
        jc      short mltx              ; se c, No, esci
        loop    mlt00                   ; se nc, Si, cicla per riprovare
        clc
mltx:
        ret

MultiplyTest    endp
\end{lstlisting}

\subsubsection{Memorizzazione di variabili locali}

Una funzione può allocare spazio nello stack per le sue variabili locali, semplicemente decrementando
lo \gls{stack pointer} verso il basso dello stack.

% I think here, "stack bottom" means the lowest address in the stack space,
% but the reader might also think it means towards the top of the stack space,
% like in a pop, so you might change "towards the stack bottom" to
% "towards the lowest address of the stack", or just take it out,
% since "decreasing" also suggests that.

Pertanto l'operazione risulta molto veloce, a prescinedere dal numero di variabili locali definite.
Anche in questo caso utilizzare lo stack per memorizzare variabili locali non è un requisito necessario.
Si possono memorizzare le variabili locali dove si vuole,
ma tradizionalmente si fa in questo modo.

\EN{\mysection{Task manager practical joke (Windows Vista)}
\myindex{Windows!Windows Vista}

Let's see if it's possible to hack Task Manager slightly so it would detect more \ac{CPU} cores.

\myindex{Windows!NTAPI}

Let us first think, how does the Task Manager know the number of cores?

There is the \TT{GetSystemInfo()} win32 function present in win32 userspace which can tell us this.
But it's not imported in \TT{taskmgr.exe}.

There is, however, another one in \gls{NTAPI}, \TT{NtQuerySystemInformation()}, 
which is used in \TT{taskmgr.exe} in several places.

To get the number of cores, one has to call this function with the \TT{SystemBasicInformation} constant
as a first argument (which is zero
\footnote{\href{http://msdn.microsoft.com/en-us/library/windows/desktop/ms724509(v=vs.85).aspx}{MSDN}}).

The second argument has to point to the buffer which is getting all the information.

So we have to find all calls to the \\
\TT{NtQuerySystemInformation(0, ?, ?, ?)} function.
Let's open \TT{taskmgr.exe} in IDA. 
\myindex{Windows!PDB}

What is always good about Microsoft executables is that IDA can download the corresponding \gls{PDB} 
file for this executable and show all function names.

It is visible that Task Manager is written in \Cpp and some of the function names and classes are really 
speaking for themselves.
There are classes CAdapter, CNetPage, CPerfPage, CProcInfo, CProcPage, CSvcPage, 
CTaskPage, CUserPage.

Apparently, each class corresponds to each tab in Task Manager.

Let's visit each call and add comment with the value which is passed as the first function argument.
We will write \q{not zero} at some places, because the value there was clearly not zero, 
but something really different (more about this in the second part of this chapter).

And we are looking for zero passed as argument, after all.

\begin{figure}[H]
\centering
\myincludegraphics{examples/taskmgr/IDA_xrefs.png}
\caption{IDA: cross references to NtQuerySystemInformation()}
\end{figure}

Yes, the names are really speaking for themselves.

When we closely investigate each place where\\
\TT{NtQuerySystemInformation(0, ?, ?, ?)} is called,
we quickly find what we need in the \TT{InitPerfInfo()} function:

\lstinputlisting[caption=taskmgr.exe (Windows Vista),style=customasmx86]{examples/taskmgr/taskmgr.lst}

\TT{g\_cProcessors} is a global variable, and this name has been assigned by 
IDA according to the \gls{PDB} loaded from Microsoft's symbol server.

The byte is taken from \TT{var\_C20}. 
And \TT{var\_C58} is passed to\\
\TT{NtQuerySystemInformation()} 
as a pointer to the receiving buffer.
The difference between 0xC20 and 0xC58 is 0x38 (56).

Let's take a look at format of the return structure, which we can find in MSDN:

\begin{lstlisting}[style=customc]
typedef struct _SYSTEM_BASIC_INFORMATION {
    BYTE Reserved1[24];
    PVOID Reserved2[4];
    CCHAR NumberOfProcessors;
} SYSTEM_BASIC_INFORMATION;
\end{lstlisting}

This is a x64 system, so each PVOID takes 8 bytes.

All \emph{reserved} fields in the structure take $24+4*8=56$ bytes.

Oh yes, this implies that \TT{var\_C20} is the local stack is exactly the
\TT{NumberOfProcessors} field of the \TT{SYSTEM\_BASIC\_INFORMATION} structure.

Let's check our guess.
Copy \TT{taskmgr.exe} from \TT{C:\textbackslash{}Windows\textbackslash{}System32} 
to some other folder 
(so the \emph{Windows Resource Protection} 
will not try to restore the patched \TT{taskmgr.exe}).

Let's open it in Hiew and find the place:

\begin{figure}[H]
\centering
\myincludegraphics{examples/taskmgr/hiew2.png}
\caption{Hiew: find the place to be patched}
\end{figure}

Let's replace the \TT{MOVZX} instruction with ours.
Let's pretend we've got 64 CPU cores.

Add one additional \ac{NOP} (because our instruction is shorter than the original one):

\begin{figure}[H]
\centering
\myincludegraphics{examples/taskmgr/hiew1.png}
\caption{Hiew: patch it}
\end{figure}

And it works!
Of course, the data in the graphs is not correct.

At times, Task Manager even shows an overall CPU load of more than 100\%.

\begin{figure}[H]
\centering
\myincludegraphics{examples/taskmgr/taskmgr_64cpu_crop.png}
\caption{Fooled Windows Task Manager}
\end{figure}

The biggest number Task Manager does not crash with is 64.

Apparently, Task Manager in Windows Vista was not tested on computers with a large number of cores.

So there are probably some static data structure(s) inside it limited to 64 cores.

\subsection{Using LEA to load values}
\label{TaskMgr_LEA}

Sometimes, \TT{LEA} is used in \TT{taskmgr.exe} instead of \TT{MOV} to set the first argument of \\
\TT{NtQuerySystemInformation()}:

\lstinputlisting[caption=taskmgr.exe (Windows Vista),style=customasmx86]{examples/taskmgr/taskmgr2.lst}

\myindex{x86!\Instructions!LEA}

Perhaps \ac{MSVC} did so because machine code of \INS{LEA} is shorter than \INS{MOV REG, 5} (would be 5 instead of 4).

\INS{LEA} with offset in $-128..127$ range (offset will occupy 1 byte in opcode) with 32-bit registers is even shorter (for lack of REX prefix)---3 bytes.

Another example of such thing is: \myref{using_MOV_and_pack_of_LEA_to_load_values}.
}%
\RU{\subsection{Обменять входные значения друг с другом}

Вот так:

\lstinputlisting[style=customc]{patterns/061_pointers/swap/5_RU.c}

Как видим, байты загружаются в младшие 8-битные части регистров \TT{ECX} и \TT{EBX} используя \INS{MOVZX}
(так что старшие части регистров очищаются), затем байты записываются назад в другом порядке.

\lstinputlisting[style=customasmx86,caption=Optimizing GCC 5.4]{patterns/061_pointers/swap/5_GCC_O3_x86.s}

Адреса обоих байтов берутся из аргументов и во время исполнения ф-ции находятся в регистрах \TT{EDX} и \TT{EAX}.

Так что исопльзуем указатели --- вероятно, без них нет способа решить эту задачу лучше.

}%
\FR{\subsection{Exemple \#2: SCO OpenServer}

\label{examples_SCO}
\myindex{SCO OpenServer}
Un ancien logiciel pour SCO OpenServer de 1997 développé par une société qui a disparue
depuis longtemps.

Il y a un driver de dongle special à installer dans le système, qui contient les
chaînes de texte suivantes:
\q{Copyright 1989, Rainbow Technologies, Inc., Irvine, CA}
et
\q{Sentinel Integrated Driver Ver. 3.0 }.

Après l'installation du driver dans SCO OpenServer, ces fichiers apparaissent dans
l'arborescence /dev:

\begin{lstlisting}
/dev/rbsl8
/dev/rbsl9
/dev/rbsl10
\end{lstlisting}

Le programme renvoie une erreur lorsque le dongle n'est pas connecté, mais le message
d'erreur n'est pas trouvé dans les exécutables.

\myindex{COFF}

Grâce à \ac{IDA}, il est facile de charger l'exécutable COFF utilisé dans SCO OpenServer.

Essayons de trouver la chaîne \q{rbsl} et en effet, elle se trouve dans ce morceau
de code:

\lstinputlisting[style=customasmx86]{examples/dongles/2/1.lst}

Oui, en effet, le programme doit communiquer d'une façon ou d'une autre avec le driver.

\myindex{thunk-functions}
Le seul endroit où la fonction \TT{SSQC()} est appelée est dans la \glslink{thunk
 function}{fonction thunk}:

\lstinputlisting[style=customasmx86]{examples/dongles/2/2.lst}

SSQ() peut être appelé depuis au moins 2 fonctions.

L'une d'entre elles est:

\lstinputlisting[style=customasmx86]{examples/dongles/2/check1_EN.lst}

\q{\TT{3C}} et \q{\TT{3E}} semblent familiers: il y avait un dongle Sentinel Pro de
Rainbow sans mémoire, fournissant seulement une fonction de crypto-hachage secrète.

Vous pouvez lire une courte description de la fonction de hachage dont il s'agit
ici: \myref{hash_func}.

Mais retournons au programme.

Donc le programme peut seulement tester si un dongle est connecté ou s'il est absent.

Aucune autre information ne peut être écrite dans un tel dongle, puisqu'il n'a pas
de mémoire.
Les codes sur deux caractères sont des commandes (nous pouvons voir comment les commandes
sont traitées dans la fonction \TT{SSQC()}) et toutes les autres chaînes sont hachées
dans le dongle, transformées en un nombre 16-bit.
L'algorithme était secret, donc il n'était pas possible d'écrire un driver de remplacement
ou de refaire un dongle matériel qui l'émulerait parfaitement.

Toutefois, il est toujours possible d'intercepter tous les accès au dongle et de
trouver les constantes auxquelles les résultats de la fonction de hachage sont comparées.

Mais nous devons dire qu'il est possible de construire un schéma de logiciel de protection
de copie robuste basé sur une fonction secrète de hachage cryptographique: il suffit
qu'elle chiffre/déchiffre les fichiers de données utilisés par votre logiciel.

Mais retournons au code:

Les codes 51/52/53 sont utilisés pour choisir le port imprimante LPT.
3x/4x sont utilisés pour le choix de la \q{famille} (c'est ainsi que les dongles
Sentinel Pro sont différenciés les uns des autres: plus d'un dongle peut être connecté
sur un port LPT).

La seule chaîne passée à la fonction qui ne fasse pas 2 caractères est "0123456789".

Ensuite, le résultat est comparé à l'ensemble des résultats valides.

Si il est correct, 0xC ou 0xB est écrit dans la variable globale \TT{ctl\_model}.%

Une autre chaîne de texte qui est passée est
"PRESS ANY KEY TO CONTINUE: ", mais le résultat n'est pas testé.
Difficile de dire pourquoi, probablement une erreur\footnote{C'est un sentiment
étrange de trouver un bug dans un logiciel aussi ancien.}.

Voyons où la valeur de la variable globale \TT{ctl\_model} est utilisée.

Un tel endroit est:

\lstinputlisting[style=customasmx86]{examples/dongles/2/4.lst}

Si c'est 0, un message d'erreur chiffré est passé à une routine de déchiffrement
et affiché.

\myindex{x86!\Instructions!XOR}

La routine de déchiffrement de la chaîne semble être un simple \glslink{xoring}{xor}:

\lstinputlisting[style=customasmx86]{examples/dongles/2/err_warn.lst}

C'est pourquoi nous étions incapable de trouver le message d'erreur dans les fichiers
exécutable, car ils sont chiffrés (ce qui est une pratique courante).

Un autre appel à la fonction de hachage \TT{SSQ()} lui passe la chaîne \q{offln}
et le résultat est comparé avec \TT{0xFE81} et \TT{0x12A9}.

Si ils ne correspondent pas, ça se comporte comme une sorte de fonction \TT{timer()}
(peut-être en attente qu'un dongle mal connecté soit reconnecté et re-testé?) et ensuite
déchiffre un autre message d'erreur à afficher.

\lstinputlisting[style=customasmx86]{examples/dongles/2/check2_EN.lst}

Passer outre le dongle est assez facile: il suffit de patcher tous les sauts après
les instructions \CMP pertinentes.

Une autre option est d'écrire notre propre driver SCO OpenServer, contenant une table
de questions et de réponses, toutes celles qui sont présentent dans le programme.

\subsubsection{Déchiffrer les messages d'erreur}

À propos, nous pouvons aussi essayer de déchiffrer tous les messages d'erreurs.
L'algorithme qui se trouve dans la fonction \TT{err\_warn()} est très simple, en effet:

\lstinputlisting[caption=Decryption function,style=customasmx86]{examples/dongles/2/decrypting_FR.lst}

Comme on le voit, non seulement la chaîne est transmise à la fonction de déchiffrement
mais aussi la clef:

\lstinputlisting[style=customasmx86]{examples/dongles/2/tmp1_EN.asm}

L'algorithme est un simple \glslink{xoring}{xor}: chaque octet est xoré avec la clef, mais
la clef est incrémentée de 3 après le traitement de chaque octet.

Nous pouvons écrire un petit script Python pour vérifier notre hypothèse:

\lstinputlisting[caption=Python 3.x]{examples/dongles/2/decr1.py}

Et il affiche: \q{check security device connection}.
Donc oui, ceci est le message déchiffré.

Il y a d'autres messages chiffrés, avec leur clef correspondante.
Mais inutile de dire qu'il est possible de les déchiffrer sans leur clef.
Premièrement, nous voyons que le clef est en fait un octet.
C'est parce que l'instruction principale de déchiffrement (\XOR) fonctionne au niveau
de l'octet.
La clef se trouve dans le registre \ESI, mais seulement une partie de \ESI d'un octet
est utilisée.
Ainsi, une clef pourrait être plus grande que 255, mais sa valeur est toujours arrondie.

En conséquence, nous pouvons simplement essayer de brute-forcer, en essayant toutes
les clefs possible dans l'intervalle 0..255.
Nous allons aussi écarter les messages comportants des caractères non-imprimable.

\lstinputlisting[caption=Python 3.x]{examples/dongles/2/decr2.py}

Et nous obtenons:

\lstinputlisting[caption=Results]{examples/dongles/2/decr2_result.txt}

Ici il y a un peu de déchet, mais nous pouvons rapidement trouver les messages en
anglais.

À propos, puisque l'algorithme est un simple chiffrement xor, la même fonction peut
être utilisée pour chiffrer les messages.
Si besoin, nous pouvons chiffrer nos propres messages, et patcher le programme en les insérant.
}


\subsubsection{(Windows) SEH}
\myindex{Windows!Structured Exception Handling}

\ifdefined\RUSSIAN
В стеке хранятся записи \ac{SEH} для функции (если они присутствуют).
Читайте больше о нем здесь: (\myref{sec:SEH}).
\fi % RUSSIAN

\ifdefined\ENGLISH
\ac{SEH} records are also stored on the stack (if they are present).
Read more about it: (\myref{sec:SEH}).
\fi % ENGLISH

\ifdefined\BRAZILIAN
\ac{SEH} também são guardados na pilha (se estiverem presentes).
\PTBRph{}: (\myref{sec:SEH}).
\fi % BRAZILIAN

\ifdefined\ITALIAN
I record \ac{SEH}, se presenti, sono anch'essi memorizzati nello stack.
Maggiori informazioni qui: (\myref{sec:SEH}).
\fi % ITALIAN

\ifdefined\FRENCH
Les enregistrements \ac{SEH} sont aussi stockés dans la pile (s'ils sont présents).
Lire à ce propos: (\myref{sec:SEH}).
\fi % FRENCH


\ifdefined\POLISH
Na stosie są przechowywane wpisy \ac{SEH} dla funkcji (jeśli są one obecne).
Więcej o tym tutaj: (\myref{sec:SEH}).
\fi % POLISH

\ifdefined\JAPANESE
\ac{SEH}レコードはスタックにも格納されます(存在する場合)。
それについてもっと読む:(\myref{sec:SEH})
\fi % JAPANESE

\ifdefined\ENGLISH
\subsubsection{Buffer overflow protection}

More about it here~(\myref{subsec:bufferoverflow}).
\fi

\ifdefined\RUSSIAN
\subsubsection{Защита от переполнений буфера}

Здесь больше об этом~(\myref{subsec:bufferoverflow}).
\fi

\ifdefined\BRAZILIAN
\subsubsection{Proteção contra estouro de buffer}

Mais sobre aqui~(\myref{subsec:bufferoverflow}).
\fi

\ifdefined\ITALIAN
\subsubsection{Protezione contro buffer overflow}

Maggiori informazioni qui~(\myref{subsec:bufferoverflow}).
\fi

\ifdefined\FRENCH
\subsubsection{Protection contre les débordements de tampon}

Lire à ce propos~(\myref{subsec:bufferoverflow}).
\fi


\ifdefined\POLISH
\subsubsection{Ochrona przed przepełnieniem bufora}

Więcej o tym tutaj~(\myref{subsec:bufferoverflow}).
\fi

\ifdefined\JAPANESE
\subsubsection{バッファオーバーフロー保護}

詳細はこちら~(\myref{subsec:bufferoverflow})
\fi


\subsubsection{Deallocazione automatica dei dati nello stack}

Probabilmente la ragione per cui si memorizzano nello stack le variabili locali e i record SEH deriva dal fatto che questi dati vengono "liberati" automaticamente all'uscita dalla funzione,
usando soltanto un'istruzione per correggere lo stack pointer (spesso è \ADD).
Si può dire che anche gli argomenti delle funzioni sono deallocati automaticamente alla fine della funzione.
Invece, qualunque altra cosa memorizzata nello \emph{heap} deve essere deallocata esplicitamente.

% subsections
\subsection{Una tipico layout dello stack}

Una disposizione tipica dello stack in un ambiente a 32-bit all'inizio di una funzione,
prima dell'esecuzione della sua prima istruzione, appare così:

\begin{center}
\begin{tabular}{ | l | l | }
\hline
\dots & \dots \\
\hline
ESP-0xC & \localVariable \#2, \MarkedInIDAAs{} \TT{var\_8} \\
\hline
ESP-8 & \localVariable \#1, \MarkedInIDAAs{} \TT{var\_4} \\
\hline
ESP-4 & \savedValueOf \EBP \\
\hline
ESP & \ReturnAddress \\
\hline
ESP+4 & \argument \#1, \MarkedInIDAAs{} \TT{arg\_0} \\
\hline
ESP+8 & \argument \#2, \MarkedInIDAAs{} \TT{arg\_4} \\
\hline
ESP+0xC & \argument \#3, \MarkedInIDAAs{} \TT{arg\_8} \\
\hline
\dots & \dots \\
\hline
\end{tabular}
\end{center}



% I think this only applies to RISC architectures
% that don't have a POP instruction that only lets you read one value
% (ie. ARM and MIPS).
% In x86, the return address is saved before entering the function,
% and the function does not have the chance to save the frame pointer.
% Also, you should mention that this is how the stack looks like
% right after the function prologue,
% which some readers might think is the first instruction,
% but is needed to save the frame pointer.

\EN{\mysection{Task manager practical joke (Windows Vista)}
\myindex{Windows!Windows Vista}

Let's see if it's possible to hack Task Manager slightly so it would detect more \ac{CPU} cores.

\myindex{Windows!NTAPI}

Let us first think, how does the Task Manager know the number of cores?

There is the \TT{GetSystemInfo()} win32 function present in win32 userspace which can tell us this.
But it's not imported in \TT{taskmgr.exe}.

There is, however, another one in \gls{NTAPI}, \TT{NtQuerySystemInformation()}, 
which is used in \TT{taskmgr.exe} in several places.

To get the number of cores, one has to call this function with the \TT{SystemBasicInformation} constant
as a first argument (which is zero
\footnote{\href{http://msdn.microsoft.com/en-us/library/windows/desktop/ms724509(v=vs.85).aspx}{MSDN}}).

The second argument has to point to the buffer which is getting all the information.

So we have to find all calls to the \\
\TT{NtQuerySystemInformation(0, ?, ?, ?)} function.
Let's open \TT{taskmgr.exe} in IDA. 
\myindex{Windows!PDB}

What is always good about Microsoft executables is that IDA can download the corresponding \gls{PDB} 
file for this executable and show all function names.

It is visible that Task Manager is written in \Cpp and some of the function names and classes are really 
speaking for themselves.
There are classes CAdapter, CNetPage, CPerfPage, CProcInfo, CProcPage, CSvcPage, 
CTaskPage, CUserPage.

Apparently, each class corresponds to each tab in Task Manager.

Let's visit each call and add comment with the value which is passed as the first function argument.
We will write \q{not zero} at some places, because the value there was clearly not zero, 
but something really different (more about this in the second part of this chapter).

And we are looking for zero passed as argument, after all.

\begin{figure}[H]
\centering
\myincludegraphics{examples/taskmgr/IDA_xrefs.png}
\caption{IDA: cross references to NtQuerySystemInformation()}
\end{figure}

Yes, the names are really speaking for themselves.

When we closely investigate each place where\\
\TT{NtQuerySystemInformation(0, ?, ?, ?)} is called,
we quickly find what we need in the \TT{InitPerfInfo()} function:

\lstinputlisting[caption=taskmgr.exe (Windows Vista),style=customasmx86]{examples/taskmgr/taskmgr.lst}

\TT{g\_cProcessors} is a global variable, and this name has been assigned by 
IDA according to the \gls{PDB} loaded from Microsoft's symbol server.

The byte is taken from \TT{var\_C20}. 
And \TT{var\_C58} is passed to\\
\TT{NtQuerySystemInformation()} 
as a pointer to the receiving buffer.
The difference between 0xC20 and 0xC58 is 0x38 (56).

Let's take a look at format of the return structure, which we can find in MSDN:

\begin{lstlisting}[style=customc]
typedef struct _SYSTEM_BASIC_INFORMATION {
    BYTE Reserved1[24];
    PVOID Reserved2[4];
    CCHAR NumberOfProcessors;
} SYSTEM_BASIC_INFORMATION;
\end{lstlisting}

This is a x64 system, so each PVOID takes 8 bytes.

All \emph{reserved} fields in the structure take $24+4*8=56$ bytes.

Oh yes, this implies that \TT{var\_C20} is the local stack is exactly the
\TT{NumberOfProcessors} field of the \TT{SYSTEM\_BASIC\_INFORMATION} structure.

Let's check our guess.
Copy \TT{taskmgr.exe} from \TT{C:\textbackslash{}Windows\textbackslash{}System32} 
to some other folder 
(so the \emph{Windows Resource Protection} 
will not try to restore the patched \TT{taskmgr.exe}).

Let's open it in Hiew and find the place:

\begin{figure}[H]
\centering
\myincludegraphics{examples/taskmgr/hiew2.png}
\caption{Hiew: find the place to be patched}
\end{figure}

Let's replace the \TT{MOVZX} instruction with ours.
Let's pretend we've got 64 CPU cores.

Add one additional \ac{NOP} (because our instruction is shorter than the original one):

\begin{figure}[H]
\centering
\myincludegraphics{examples/taskmgr/hiew1.png}
\caption{Hiew: patch it}
\end{figure}

And it works!
Of course, the data in the graphs is not correct.

At times, Task Manager even shows an overall CPU load of more than 100\%.

\begin{figure}[H]
\centering
\myincludegraphics{examples/taskmgr/taskmgr_64cpu_crop.png}
\caption{Fooled Windows Task Manager}
\end{figure}

The biggest number Task Manager does not crash with is 64.

Apparently, Task Manager in Windows Vista was not tested on computers with a large number of cores.

So there are probably some static data structure(s) inside it limited to 64 cores.

\subsection{Using LEA to load values}
\label{TaskMgr_LEA}

Sometimes, \TT{LEA} is used in \TT{taskmgr.exe} instead of \TT{MOV} to set the first argument of \\
\TT{NtQuerySystemInformation()}:

\lstinputlisting[caption=taskmgr.exe (Windows Vista),style=customasmx86]{examples/taskmgr/taskmgr2.lst}

\myindex{x86!\Instructions!LEA}

Perhaps \ac{MSVC} did so because machine code of \INS{LEA} is shorter than \INS{MOV REG, 5} (would be 5 instead of 4).

\INS{LEA} with offset in $-128..127$ range (offset will occupy 1 byte in opcode) with 32-bit registers is even shorter (for lack of REX prefix)---3 bytes.

Another example of such thing is: \myref{using_MOV_and_pack_of_LEA_to_load_values}.
}%
\RU{\subsection{Обменять входные значения друг с другом}

Вот так:

\lstinputlisting[style=customc]{patterns/061_pointers/swap/5_RU.c}

Как видим, байты загружаются в младшие 8-битные части регистров \TT{ECX} и \TT{EBX} используя \INS{MOVZX}
(так что старшие части регистров очищаются), затем байты записываются назад в другом порядке.

\lstinputlisting[style=customasmx86,caption=Optimizing GCC 5.4]{patterns/061_pointers/swap/5_GCC_O3_x86.s}

Адреса обоих байтов берутся из аргументов и во время исполнения ф-ции находятся в регистрах \TT{EDX} и \TT{EAX}.

Так что исопльзуем указатели --- вероятно, без них нет способа решить эту задачу лучше.

}%
\FR{\subsection{Exemple \#2: SCO OpenServer}

\label{examples_SCO}
\myindex{SCO OpenServer}
Un ancien logiciel pour SCO OpenServer de 1997 développé par une société qui a disparue
depuis longtemps.

Il y a un driver de dongle special à installer dans le système, qui contient les
chaînes de texte suivantes:
\q{Copyright 1989, Rainbow Technologies, Inc., Irvine, CA}
et
\q{Sentinel Integrated Driver Ver. 3.0 }.

Après l'installation du driver dans SCO OpenServer, ces fichiers apparaissent dans
l'arborescence /dev:

\begin{lstlisting}
/dev/rbsl8
/dev/rbsl9
/dev/rbsl10
\end{lstlisting}

Le programme renvoie une erreur lorsque le dongle n'est pas connecté, mais le message
d'erreur n'est pas trouvé dans les exécutables.

\myindex{COFF}

Grâce à \ac{IDA}, il est facile de charger l'exécutable COFF utilisé dans SCO OpenServer.

Essayons de trouver la chaîne \q{rbsl} et en effet, elle se trouve dans ce morceau
de code:

\lstinputlisting[style=customasmx86]{examples/dongles/2/1.lst}

Oui, en effet, le programme doit communiquer d'une façon ou d'une autre avec le driver.

\myindex{thunk-functions}
Le seul endroit où la fonction \TT{SSQC()} est appelée est dans la \glslink{thunk
 function}{fonction thunk}:

\lstinputlisting[style=customasmx86]{examples/dongles/2/2.lst}

SSQ() peut être appelé depuis au moins 2 fonctions.

L'une d'entre elles est:

\lstinputlisting[style=customasmx86]{examples/dongles/2/check1_EN.lst}

\q{\TT{3C}} et \q{\TT{3E}} semblent familiers: il y avait un dongle Sentinel Pro de
Rainbow sans mémoire, fournissant seulement une fonction de crypto-hachage secrète.

Vous pouvez lire une courte description de la fonction de hachage dont il s'agit
ici: \myref{hash_func}.

Mais retournons au programme.

Donc le programme peut seulement tester si un dongle est connecté ou s'il est absent.

Aucune autre information ne peut être écrite dans un tel dongle, puisqu'il n'a pas
de mémoire.
Les codes sur deux caractères sont des commandes (nous pouvons voir comment les commandes
sont traitées dans la fonction \TT{SSQC()}) et toutes les autres chaînes sont hachées
dans le dongle, transformées en un nombre 16-bit.
L'algorithme était secret, donc il n'était pas possible d'écrire un driver de remplacement
ou de refaire un dongle matériel qui l'émulerait parfaitement.

Toutefois, il est toujours possible d'intercepter tous les accès au dongle et de
trouver les constantes auxquelles les résultats de la fonction de hachage sont comparées.

Mais nous devons dire qu'il est possible de construire un schéma de logiciel de protection
de copie robuste basé sur une fonction secrète de hachage cryptographique: il suffit
qu'elle chiffre/déchiffre les fichiers de données utilisés par votre logiciel.

Mais retournons au code:

Les codes 51/52/53 sont utilisés pour choisir le port imprimante LPT.
3x/4x sont utilisés pour le choix de la \q{famille} (c'est ainsi que les dongles
Sentinel Pro sont différenciés les uns des autres: plus d'un dongle peut être connecté
sur un port LPT).

La seule chaîne passée à la fonction qui ne fasse pas 2 caractères est "0123456789".

Ensuite, le résultat est comparé à l'ensemble des résultats valides.

Si il est correct, 0xC ou 0xB est écrit dans la variable globale \TT{ctl\_model}.%

Une autre chaîne de texte qui est passée est
"PRESS ANY KEY TO CONTINUE: ", mais le résultat n'est pas testé.
Difficile de dire pourquoi, probablement une erreur\footnote{C'est un sentiment
étrange de trouver un bug dans un logiciel aussi ancien.}.

Voyons où la valeur de la variable globale \TT{ctl\_model} est utilisée.

Un tel endroit est:

\lstinputlisting[style=customasmx86]{examples/dongles/2/4.lst}

Si c'est 0, un message d'erreur chiffré est passé à une routine de déchiffrement
et affiché.

\myindex{x86!\Instructions!XOR}

La routine de déchiffrement de la chaîne semble être un simple \glslink{xoring}{xor}:

\lstinputlisting[style=customasmx86]{examples/dongles/2/err_warn.lst}

C'est pourquoi nous étions incapable de trouver le message d'erreur dans les fichiers
exécutable, car ils sont chiffrés (ce qui est une pratique courante).

Un autre appel à la fonction de hachage \TT{SSQ()} lui passe la chaîne \q{offln}
et le résultat est comparé avec \TT{0xFE81} et \TT{0x12A9}.

Si ils ne correspondent pas, ça se comporte comme une sorte de fonction \TT{timer()}
(peut-être en attente qu'un dongle mal connecté soit reconnecté et re-testé?) et ensuite
déchiffre un autre message d'erreur à afficher.

\lstinputlisting[style=customasmx86]{examples/dongles/2/check2_EN.lst}

Passer outre le dongle est assez facile: il suffit de patcher tous les sauts après
les instructions \CMP pertinentes.

Une autre option est d'écrire notre propre driver SCO OpenServer, contenant une table
de questions et de réponses, toutes celles qui sont présentent dans le programme.

\subsubsection{Déchiffrer les messages d'erreur}

À propos, nous pouvons aussi essayer de déchiffrer tous les messages d'erreurs.
L'algorithme qui se trouve dans la fonction \TT{err\_warn()} est très simple, en effet:

\lstinputlisting[caption=Decryption function,style=customasmx86]{examples/dongles/2/decrypting_FR.lst}

Comme on le voit, non seulement la chaîne est transmise à la fonction de déchiffrement
mais aussi la clef:

\lstinputlisting[style=customasmx86]{examples/dongles/2/tmp1_EN.asm}

L'algorithme est un simple \glslink{xoring}{xor}: chaque octet est xoré avec la clef, mais
la clef est incrémentée de 3 après le traitement de chaque octet.

Nous pouvons écrire un petit script Python pour vérifier notre hypothèse:

\lstinputlisting[caption=Python 3.x]{examples/dongles/2/decr1.py}

Et il affiche: \q{check security device connection}.
Donc oui, ceci est le message déchiffré.

Il y a d'autres messages chiffrés, avec leur clef correspondante.
Mais inutile de dire qu'il est possible de les déchiffrer sans leur clef.
Premièrement, nous voyons que le clef est en fait un octet.
C'est parce que l'instruction principale de déchiffrement (\XOR) fonctionne au niveau
de l'octet.
La clef se trouve dans le registre \ESI, mais seulement une partie de \ESI d'un octet
est utilisée.
Ainsi, une clef pourrait être plus grande que 255, mais sa valeur est toujours arrondie.

En conséquence, nous pouvons simplement essayer de brute-forcer, en essayant toutes
les clefs possible dans l'intervalle 0..255.
Nous allons aussi écarter les messages comportants des caractères non-imprimable.

\lstinputlisting[caption=Python 3.x]{examples/dongles/2/decr2.py}

Et nous obtenons:

\lstinputlisting[caption=Results]{examples/dongles/2/decr2_result.txt}

Ici il y a un peu de déchet, mais nous pouvons rapidement trouver les messages en
anglais.

À propos, puisque l'algorithme est un simple chiffrement xor, la même fonction peut
être utilisée pour chiffrer les messages.
Si besoin, nous pouvons chiffrer nos propres messages, et patcher le programme en les insérant.
}


\subsection{\Exercises}

\begin{itemize}
	\item \url{http://challenges.re/67}
	\item \url{http://challenges.re/68}
	\item \url{http://challenges.re/69}
	\item \url{http://challenges.re/70}
\end{itemize}



