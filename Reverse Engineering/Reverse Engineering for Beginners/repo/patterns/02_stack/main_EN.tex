\mysection{\Stack}
\label{sec:stack}
\myindex{\Stack}

The stack is one of the most fundamental data structures in computer science
\footnote{\href{http://en.wikipedia.org/wiki/Call_stack}{wikipedia.org/wiki/Call\_stack}}.
\ac{AKA} \ac{LIFO}.

Technically, it is just a block of memory in process memory along with the \ESP or \RSP register in x86 or x64, or the \ac{SP} register in ARM, as a pointer within that block.

\myindex{ARM!\Instructions!PUSH}
\myindex{ARM!\Instructions!POP}
\myindex{x86!\Instructions!PUSH}
\myindex{x86!\Instructions!POP}
The most frequently used stack access instructions are \PUSH and \POP (in both x86 and ARM Thumb-mode). 
\PUSH subtracts from \ESP/\RSP/\ac{SP} 4 in 32-bit mode (or 8 in 64-bit mode) and then writes the contents of its sole operand to the memory address pointed by \ESP/\RSP/\ac{SP}.

\POP is the reverse operation: retrieve the data from the memory location that \ac{SP} points to, 
load it into the instruction operand (often a register) and then add 4 (or 8) to the \gls{stack pointer}.

After stack allocation, the \gls{stack pointer} points at the bottom of the stack.
\PUSH decreases the \gls{stack pointer} and \POP increases it.
The bottom of the stack is actually at the beginning of the memory allocated for the stack block. It seems strange, but that's the way it is.

ARM supports both descending and ascending stacks.

\myindex{ARM!\Instructions!STMFD}
\myindex{ARM!\Instructions!LDMFD}
\myindex{ARM!\Instructions!STMED}
\myindex{ARM!\Instructions!LDMED}
\myindex{ARM!\Instructions!STMFA}
\myindex{ARM!\Instructions!LDMFA}
\myindex{ARM!\Instructions!STMEA}
\myindex{ARM!\Instructions!LDMEA}

For example the \ac{STMFD}/\ac{LDMFD}, \ac{STMED}/\ac{LDMED} instructions are intended to deal with a descending stack (grows downwards, starting with a high address and progressing to a lower one).
The \ac{STMFA}/\ac{LDMFA}, \ac{STMEA}/\ac{LDMEA} instructions are intended to deal with an ascending stack (grows upwards, starting from a low address and progressing to a higher one).

% It might be worth mentioning that STMED and STMEA write first,
% and then move the pointer,
% and that LDMED and LDMEA move the pointer first, and then read.
% In other words, ARM not only lets the stack grow in a non-standard direction,
% but also in a non-standard order.
% Maybe this can be in the glossary, which would explain why E stands for "empty".

\subsection{Why does the stack grow backwards?}
\label{stack_grow_backwards}

Intuitively, we might think that the stack grows upwards, i.e. towards higher addresses, like any other data structure.

The reason that the stack grows backward is probably historical.
When the computers were big and occupied a whole room, it was easy to divide memory into two parts, one for the \gls{heap} and one for the stack.
Of course, it was unknown how big the \gls{heap} and the stack would be during program execution, so this solution was the simplest possible.

\begin{center}
	\begin{tikzpicture}
	\tikzstyle{every path}=[thick]

	\node [rectangle,draw,minimum width=6cm, minimum height=2cm] (memory) {};
	\node [] [right=0.2cm of memory.west] (heap) {\MLHeap};
	\node [] [left=0.2cm of memory.east] (stack) {\MLStack};

	\node [] (center1) [right=2cm of memory.west] {};
	\node [] (center2) [left=2cm of memory.east] {};

	\draw [->] (heap) -- (center1);
	\draw [->] (stack) -- (center2);

	\node [] [above left=1.1cm and 0.2cm of heap] (t1) {\MLStartOfHeap};
	\node [] [above right=1.1cm and 0.2cm of stack] (t2) {\MLStartOfStack};

	\draw [->] (t1) -- (memory.west);
	\draw [->] (t2) -- (memory.east);

	\end{tikzpicture}
\end{center}


In \RitchieThompsonUNIX we can read:

\begin{framed}
\begin{quotation}
The user-core part of an image is divided into three logical segments. The program text segment begins at location 0 in the virtual address space. During execution, this segment is write-protected and a single copy of it is shared among all processes executing the same program. At the first 8K byte boundary above the program text segment in the virtual address space begins a nonshared, writable data segment, the size of which may be extended by a system call. Starting at the highest address in the virtual address space is a stack segment, which automatically grows downward as the hardware's stack pointer fluctuates.
\end{quotation}
\end{framed}

This reminds us how some students write two lecture notes using only one notebook:
notes for the first lecture are written as usual, 
and notes for the second one are written from the end of notebook, by flipping it.
Notes may meet each other somewhere in between, in case of lack of free space.

% I think if we want to expand on this analogy,
% one might remember that the line number increases as as you go down a page.
% So when you decrease the address when pushing to the stack, visually,
% the stack does grow upwards.
% Of course, the problem is that in most human languages,
% just as with computers,
% we write downwards, so this direction is what makes buffer overflows so messy.

\subsection{What is the stack used for?}

% subsections
\subsubsection{Save the function's return address}

\myparagraph{x86}

\myindex{x86!\Instructions!CALL}
When calling another function with a \CALL instruction, the address of the point exactly after the \CALL instruction is saved 
to the stack and then an unconditional jump to the address in the \CALL operand is executed.

\myindex{x86!\Instructions!PUSH}
\myindex{x86!\Instructions!JMP}
The \CALL instruction is equivalent to a\\
\INS{PUSH address\_after\_call / JMP operand} instruction pair.

\myindex{x86!\Instructions!RET}
\myindex{x86!\Instructions!POP}
\RET fetches a value from the stack and jumps to it~---that is equivalent to a \TT{POP tmp / JMP tmp} instruction pair.

\myindex{\Stack!\MLStackOverflow}
\myindex{\Recursion}
Overflowing the stack is straightforward. Just run eternal recursion:

\begin{lstlisting}[style=customc]
void f()
{
	f();
};
\end{lstlisting}

MSVC 2008 reports the problem:

\begin{lstlisting}
c:\tmp6>cl ss.cpp /Fass.asm
Microsoft (R) 32-bit C/C++ Optimizing Compiler Version 15.00.21022.08 for 80x86
Copyright (C) Microsoft Corporation.  All rights reserved.

ss.cpp
c:\tmp6\ss.cpp(4) : warning C4717: 'f' : recursive on all control paths, function will cause runtime stack overflow
\end{lstlisting}

\dots but generates the right code anyway:

\lstinputlisting[style=customasmx86]{patterns/02_stack/1.asm}

\dots Also if we turn on the compiler optimization (\TT{\Ox} option) the optimized code will not overflow the stack 
and will work \emph{correctly}\footnote{irony here} instead:

\lstinputlisting[style=customasmx86]{patterns/02_stack/2.asm}

GCC 4.4.1 generates similar code in both cases without, however,  issuing any warning about the problem.

\myparagraph{ARM}

\myindex{ARM!\Registers!Link Register}
ARM programs also use the stack for saving return addresses, but differently.
As mentioned in \q{\HelloWorldSectionName}~(\myref{sec:hw_ARM}),
the \ac{RA} is saved to the \ac{LR} (\gls{link register}).
If one needs, however, to call another function and use the \ac{LR} register
one more time, its value has to be saved.
\myindex{Function prologue}
Usually it is saved in the function prologue.

\myindex{ARM!\Instructions!PUSH}
\myindex{ARM!\Instructions!POP}
Often, we see instructions like \INS{PUSH {R4-R7,LR}} along with this instruction in epilogue
\INS{POP {R4-R7,PC}}---thus register values to be used in the function are saved in the stack, including \ac{LR}.

\myindex{ARM!Leaf function}
Nevertheless, if a function never calls any other function, in \ac{RISC} terminology it is called a
\emph{\gls{leaf function}}\footnote{\href{http://infocenter.arm.com/help/index.jsp?topic=/com.arm.doc.faqs/ka13785.html}{infocenter.arm.com/help/index.jsp?topic=/com.arm.doc.faqs/ka13785.html}}. 
As a consequence, leaf functions do not save the \ac{LR} register (because they don't modify it).
If such function is small and uses a small number of registers, it may not use the stack at all.
Thus, it is possible to call leaf functions without using the stack,
which can be faster than on older x86 machines because external RAM is not used for the stack
\footnote{Some time ago, on PDP-11 and VAX, the CALL instruction (calling other functions) was expensive; up to 50\%
of execution time might be spent on it, so it was considered that having a big number of small functions is an \gls{anti-pattern} \InSqBrackets{\TAOUP Chapter 4, Part II}.}.
This can be also useful for situations when memory for the stack is not yet allocated or not available.

Some examples of leaf functions:
\myref{ARM_leaf_example1}, \myref{ARM_leaf_example2}, 
\myref{ARM_leaf_example3}, \myref{ARM_leaf_example4}, \myref{ARM_leaf_example5},
\myref{ARM_leaf_example6}, \myref{ARM_leaf_example7}, \myref{ARM_leaf_example10}.


\subsubsection{Passing function arguments}

The most popular way to pass parameters in x86 is called \q{cdecl}:

\begin{lstlisting}[style=customasmx86]
push arg3
push arg2
push arg1
call f
add esp, 12 ; 4*3=12
\end{lstlisting}

\Gls{callee} functions get their arguments via the stack pointer.

Therefore, this is how the argument values are located in the stack before the execution of the \ttf{} function's very first instruction:

\begin{center}
\begin{tabular}{ | l | l | }
\hline
ESP & return address \\
\hline
ESP+4 & \argument \#1, \MarkedInIDAAs{} \TT{arg\_0} \\
\hline
ESP+8 & \argument \#2, \MarkedInIDAAs{} \TT{arg\_4} \\
\hline
ESP+0xC & \argument \#3, \MarkedInIDAAs{} \TT{arg\_8} \\
\hline
\dots & \dots \\
\hline
\end{tabular}
\end{center}

For more information on other calling conventions see also section~(\myref{sec:callingconventions}).

\par
By the way, the \gls{callee} function does not have any information about how many arguments were passed.
C functions with a variable number of arguments (like \printf) can determine their number using format string specifiers (which begin with the \% symbol).

If we write something like:

\begin{lstlisting}
printf("%d %d %d", 1234);
\end{lstlisting}

\printf will print 1234, and then two random numbers\footnote{Not random in strict sense, but rather unpredictable: \myref{noise_in_stack}}, which were lying next to it in the stack.

\label{main_arguments}
\par
That's why it is not very important how we declare the \main function: as \main, \\
\TT{main(int argc, char *argv[])} or \TT{main(int argc, char *argv[], char *envp[])}.

In fact, the \ac{CRT}-code is calling \main roughly as:
	
\begin{lstlisting}[style=customasmx86]
push envp
push argv
push argc
call main
...
\end{lstlisting}

If you declare \main as \main without arguments, they are, nevertheless, still present in the stack, but are not used.
If you declare \main as  \TT{main(int argc, char *argv[])},
you will be able to use first two arguments, and the third will remain \q{invisible} for your function.
Even more, it is possible to declare \TT{main(int argc)}, and it will work.

Another related example: \ref{cdecl_DLL}.

\myparagraph{Alternative ways of passing arguments}

It is worth noting that nothing obliges programmers to pass arguments through the stack. It is not a requirement.
One could implement any other method without using the stack at all.

A somewhat popular way among assembly language newbies is to pass arguments via global variables, like:

\lstinputlisting[caption=Assembly code,style=customasmx86]{patterns/02_stack/global_args.asm}

But this method has obvious drawback: \emph{do\_something()} function cannot call itself recursively (or via another function),
because it has to zap its own arguments.
The same story with local variables: if you hold them in global variables, the function couldn't call itself.
And this is also not thread-safe
\footnote{Correctly implemented, each thread would have its own stack with its own arguments/variables.}.
A method to store such information in stack makes this easier---it can hold as many function arguments and/or values,
as much space it has.

\InSqBrackets{\TAOCPvolI{}, 189} mentions even weirder schemes particularly convenient on IBM System/360.

\myindex{MS-DOS}
\myindex{x86!\Instructions!INT}

MS-DOS had a way of passing all function arguments via registers, for example, this is piece of
code for ancient 16-bit MS-DOS prints ``Hello, world!'':

\begin{lstlisting}[style=customasmx86]
mov  dx, msg      ; address of message
mov  ah, 9        ; 9 means "print string" function
int  21h          ; DOS "syscall"

mov  ah, 4ch      ; "terminate program" function
int  21h          ; DOS "syscall"

msg  db 'Hello, World!\$'
\end{lstlisting}

\myindex{fastcall}
This is quite similar to \myref{fastcall} method.
And also it's very similar to calling syscalls in Linux (\myref{linux_syscall}) and Windows.

\myindex{x86!\Flags!CF}
If a MS-DOS function is going to return a boolean value (i.e., single bit, usually indicating error state),
\TT{CF} flag was often used.

For example:

\begin{lstlisting}[style=customasmx86]
mov ah, 3ch       ; create file
lea dx, filename
mov cl, 1
int 21h
jc  error
mov file_handle, ax
...
error:
...
\end{lstlisting}

In case of error, \TT{CF} flag is raised. Otherwise, handle of newly created file is returned via \TT{AX}.

This method is still used by assembly language programmers.
In Windows Research Kernel source code (which is quite similar to Windows 2003) we can find something like this
(file \emph{base/ntos/ke/i386/cpu.asm}):

\begin{lstlisting}[style=customasmx86]
        public  Get386Stepping
Get386Stepping  proc

        call    MultiplyTest            ; Perform multiplication test
        jnc     short G3s00             ; if nc, muttest is ok
        mov     ax, 0
        ret
G3s00:
        call    Check386B0              ; Check for B0 stepping
        jnc     short G3s05             ; if nc, it's B1/later
        mov     ax, 100h                ; It is B0/earlier stepping
        ret

G3s05:
        call    Check386D1              ; Check for D1 stepping
        jc      short G3s10             ; if c, it is NOT D1
        mov     ax, 301h                ; It is D1/later stepping
        ret

G3s10:
        mov     ax, 101h                ; assume it is B1 stepping
        ret

	...

MultiplyTest    proc

        xor     cx,cx                   ; 64K times is a nice round number
mlt00:  push    cx
        call    Multiply                ; does this chip's multiply work?
        pop     cx
        jc      short mltx              ; if c, No, exit
        loop    mlt00                   ; if nc, YEs, loop to try again
        clc
mltx:
        ret

MultiplyTest    endp
\end{lstlisting}


\subsubsection{Local variable storage}

A function could allocate space in the stack for its local variables just by decreasing 
the \gls{stack pointer} towards the stack bottom.

% I think here, "stack bottom" means the lowest address in the stack space,
% but the reader might also think it means towards the top of the stack space,
% like in a pop, so you might change "towards the stack bottom" to
% "towards the lowest address of the stack", or just take it out,
% since "decreasing" also suggests that.

Hence, it's very fast, no matter how many local variables are defined.
It is also not a requirement to store local variables in the stack.
You could store local variables wherever you like, 
but traditionally this is how it's done.


\EN{\mysection{Task manager practical joke (Windows Vista)}
\myindex{Windows!Windows Vista}

Let's see if it's possible to hack Task Manager slightly so it would detect more \ac{CPU} cores.

\myindex{Windows!NTAPI}

Let us first think, how does the Task Manager know the number of cores?

There is the \TT{GetSystemInfo()} win32 function present in win32 userspace which can tell us this.
But it's not imported in \TT{taskmgr.exe}.

There is, however, another one in \gls{NTAPI}, \TT{NtQuerySystemInformation()}, 
which is used in \TT{taskmgr.exe} in several places.

To get the number of cores, one has to call this function with the \TT{SystemBasicInformation} constant
as a first argument (which is zero
\footnote{\href{http://msdn.microsoft.com/en-us/library/windows/desktop/ms724509(v=vs.85).aspx}{MSDN}}).

The second argument has to point to the buffer which is getting all the information.

So we have to find all calls to the \\
\TT{NtQuerySystemInformation(0, ?, ?, ?)} function.
Let's open \TT{taskmgr.exe} in IDA. 
\myindex{Windows!PDB}

What is always good about Microsoft executables is that IDA can download the corresponding \gls{PDB} 
file for this executable and show all function names.

It is visible that Task Manager is written in \Cpp and some of the function names and classes are really 
speaking for themselves.
There are classes CAdapter, CNetPage, CPerfPage, CProcInfo, CProcPage, CSvcPage, 
CTaskPage, CUserPage.

Apparently, each class corresponds to each tab in Task Manager.

Let's visit each call and add comment with the value which is passed as the first function argument.
We will write \q{not zero} at some places, because the value there was clearly not zero, 
but something really different (more about this in the second part of this chapter).

And we are looking for zero passed as argument, after all.

\begin{figure}[H]
\centering
\myincludegraphics{examples/taskmgr/IDA_xrefs.png}
\caption{IDA: cross references to NtQuerySystemInformation()}
\end{figure}

Yes, the names are really speaking for themselves.

When we closely investigate each place where\\
\TT{NtQuerySystemInformation(0, ?, ?, ?)} is called,
we quickly find what we need in the \TT{InitPerfInfo()} function:

\lstinputlisting[caption=taskmgr.exe (Windows Vista),style=customasmx86]{examples/taskmgr/taskmgr.lst}

\TT{g\_cProcessors} is a global variable, and this name has been assigned by 
IDA according to the \gls{PDB} loaded from Microsoft's symbol server.

The byte is taken from \TT{var\_C20}. 
And \TT{var\_C58} is passed to\\
\TT{NtQuerySystemInformation()} 
as a pointer to the receiving buffer.
The difference between 0xC20 and 0xC58 is 0x38 (56).

Let's take a look at format of the return structure, which we can find in MSDN:

\begin{lstlisting}[style=customc]
typedef struct _SYSTEM_BASIC_INFORMATION {
    BYTE Reserved1[24];
    PVOID Reserved2[4];
    CCHAR NumberOfProcessors;
} SYSTEM_BASIC_INFORMATION;
\end{lstlisting}

This is a x64 system, so each PVOID takes 8 bytes.

All \emph{reserved} fields in the structure take $24+4*8=56$ bytes.

Oh yes, this implies that \TT{var\_C20} is the local stack is exactly the
\TT{NumberOfProcessors} field of the \TT{SYSTEM\_BASIC\_INFORMATION} structure.

Let's check our guess.
Copy \TT{taskmgr.exe} from \TT{C:\textbackslash{}Windows\textbackslash{}System32} 
to some other folder 
(so the \emph{Windows Resource Protection} 
will not try to restore the patched \TT{taskmgr.exe}).

Let's open it in Hiew and find the place:

\begin{figure}[H]
\centering
\myincludegraphics{examples/taskmgr/hiew2.png}
\caption{Hiew: find the place to be patched}
\end{figure}

Let's replace the \TT{MOVZX} instruction with ours.
Let's pretend we've got 64 CPU cores.

Add one additional \ac{NOP} (because our instruction is shorter than the original one):

\begin{figure}[H]
\centering
\myincludegraphics{examples/taskmgr/hiew1.png}
\caption{Hiew: patch it}
\end{figure}

And it works!
Of course, the data in the graphs is not correct.

At times, Task Manager even shows an overall CPU load of more than 100\%.

\begin{figure}[H]
\centering
\myincludegraphics{examples/taskmgr/taskmgr_64cpu_crop.png}
\caption{Fooled Windows Task Manager}
\end{figure}

The biggest number Task Manager does not crash with is 64.

Apparently, Task Manager in Windows Vista was not tested on computers with a large number of cores.

So there are probably some static data structure(s) inside it limited to 64 cores.

\subsection{Using LEA to load values}
\label{TaskMgr_LEA}

Sometimes, \TT{LEA} is used in \TT{taskmgr.exe} instead of \TT{MOV} to set the first argument of \\
\TT{NtQuerySystemInformation()}:

\lstinputlisting[caption=taskmgr.exe (Windows Vista),style=customasmx86]{examples/taskmgr/taskmgr2.lst}

\myindex{x86!\Instructions!LEA}

Perhaps \ac{MSVC} did so because machine code of \INS{LEA} is shorter than \INS{MOV REG, 5} (would be 5 instead of 4).

\INS{LEA} with offset in $-128..127$ range (offset will occupy 1 byte in opcode) with 32-bit registers is even shorter (for lack of REX prefix)---3 bytes.

Another example of such thing is: \myref{using_MOV_and_pack_of_LEA_to_load_values}.
}%
\RU{\subsection{Обменять входные значения друг с другом}

Вот так:

\lstinputlisting[style=customc]{patterns/061_pointers/swap/5_RU.c}

Как видим, байты загружаются в младшие 8-битные части регистров \TT{ECX} и \TT{EBX} используя \INS{MOVZX}
(так что старшие части регистров очищаются), затем байты записываются назад в другом порядке.

\lstinputlisting[style=customasmx86,caption=Optimizing GCC 5.4]{patterns/061_pointers/swap/5_GCC_O3_x86.s}

Адреса обоих байтов берутся из аргументов и во время исполнения ф-ции находятся в регистрах \TT{EDX} и \TT{EAX}.

Так что исопльзуем указатели --- вероятно, без них нет способа решить эту задачу лучше.

}%
\FR{\subsection{Exemple \#2: SCO OpenServer}

\label{examples_SCO}
\myindex{SCO OpenServer}
Un ancien logiciel pour SCO OpenServer de 1997 développé par une société qui a disparue
depuis longtemps.

Il y a un driver de dongle special à installer dans le système, qui contient les
chaînes de texte suivantes:
\q{Copyright 1989, Rainbow Technologies, Inc., Irvine, CA}
et
\q{Sentinel Integrated Driver Ver. 3.0 }.

Après l'installation du driver dans SCO OpenServer, ces fichiers apparaissent dans
l'arborescence /dev:

\begin{lstlisting}
/dev/rbsl8
/dev/rbsl9
/dev/rbsl10
\end{lstlisting}

Le programme renvoie une erreur lorsque le dongle n'est pas connecté, mais le message
d'erreur n'est pas trouvé dans les exécutables.

\myindex{COFF}

Grâce à \ac{IDA}, il est facile de charger l'exécutable COFF utilisé dans SCO OpenServer.

Essayons de trouver la chaîne \q{rbsl} et en effet, elle se trouve dans ce morceau
de code:

\lstinputlisting[style=customasmx86]{examples/dongles/2/1.lst}

Oui, en effet, le programme doit communiquer d'une façon ou d'une autre avec le driver.

\myindex{thunk-functions}
Le seul endroit où la fonction \TT{SSQC()} est appelée est dans la \glslink{thunk
 function}{fonction thunk}:

\lstinputlisting[style=customasmx86]{examples/dongles/2/2.lst}

SSQ() peut être appelé depuis au moins 2 fonctions.

L'une d'entre elles est:

\lstinputlisting[style=customasmx86]{examples/dongles/2/check1_EN.lst}

\q{\TT{3C}} et \q{\TT{3E}} semblent familiers: il y avait un dongle Sentinel Pro de
Rainbow sans mémoire, fournissant seulement une fonction de crypto-hachage secrète.

Vous pouvez lire une courte description de la fonction de hachage dont il s'agit
ici: \myref{hash_func}.

Mais retournons au programme.

Donc le programme peut seulement tester si un dongle est connecté ou s'il est absent.

Aucune autre information ne peut être écrite dans un tel dongle, puisqu'il n'a pas
de mémoire.
Les codes sur deux caractères sont des commandes (nous pouvons voir comment les commandes
sont traitées dans la fonction \TT{SSQC()}) et toutes les autres chaînes sont hachées
dans le dongle, transformées en un nombre 16-bit.
L'algorithme était secret, donc il n'était pas possible d'écrire un driver de remplacement
ou de refaire un dongle matériel qui l'émulerait parfaitement.

Toutefois, il est toujours possible d'intercepter tous les accès au dongle et de
trouver les constantes auxquelles les résultats de la fonction de hachage sont comparées.

Mais nous devons dire qu'il est possible de construire un schéma de logiciel de protection
de copie robuste basé sur une fonction secrète de hachage cryptographique: il suffit
qu'elle chiffre/déchiffre les fichiers de données utilisés par votre logiciel.

Mais retournons au code:

Les codes 51/52/53 sont utilisés pour choisir le port imprimante LPT.
3x/4x sont utilisés pour le choix de la \q{famille} (c'est ainsi que les dongles
Sentinel Pro sont différenciés les uns des autres: plus d'un dongle peut être connecté
sur un port LPT).

La seule chaîne passée à la fonction qui ne fasse pas 2 caractères est "0123456789".

Ensuite, le résultat est comparé à l'ensemble des résultats valides.

Si il est correct, 0xC ou 0xB est écrit dans la variable globale \TT{ctl\_model}.%

Une autre chaîne de texte qui est passée est
"PRESS ANY KEY TO CONTINUE: ", mais le résultat n'est pas testé.
Difficile de dire pourquoi, probablement une erreur\footnote{C'est un sentiment
étrange de trouver un bug dans un logiciel aussi ancien.}.

Voyons où la valeur de la variable globale \TT{ctl\_model} est utilisée.

Un tel endroit est:

\lstinputlisting[style=customasmx86]{examples/dongles/2/4.lst}

Si c'est 0, un message d'erreur chiffré est passé à une routine de déchiffrement
et affiché.

\myindex{x86!\Instructions!XOR}

La routine de déchiffrement de la chaîne semble être un simple \glslink{xoring}{xor}:

\lstinputlisting[style=customasmx86]{examples/dongles/2/err_warn.lst}

C'est pourquoi nous étions incapable de trouver le message d'erreur dans les fichiers
exécutable, car ils sont chiffrés (ce qui est une pratique courante).

Un autre appel à la fonction de hachage \TT{SSQ()} lui passe la chaîne \q{offln}
et le résultat est comparé avec \TT{0xFE81} et \TT{0x12A9}.

Si ils ne correspondent pas, ça se comporte comme une sorte de fonction \TT{timer()}
(peut-être en attente qu'un dongle mal connecté soit reconnecté et re-testé?) et ensuite
déchiffre un autre message d'erreur à afficher.

\lstinputlisting[style=customasmx86]{examples/dongles/2/check2_EN.lst}

Passer outre le dongle est assez facile: il suffit de patcher tous les sauts après
les instructions \CMP pertinentes.

Une autre option est d'écrire notre propre driver SCO OpenServer, contenant une table
de questions et de réponses, toutes celles qui sont présentent dans le programme.

\subsubsection{Déchiffrer les messages d'erreur}

À propos, nous pouvons aussi essayer de déchiffrer tous les messages d'erreurs.
L'algorithme qui se trouve dans la fonction \TT{err\_warn()} est très simple, en effet:

\lstinputlisting[caption=Decryption function,style=customasmx86]{examples/dongles/2/decrypting_FR.lst}

Comme on le voit, non seulement la chaîne est transmise à la fonction de déchiffrement
mais aussi la clef:

\lstinputlisting[style=customasmx86]{examples/dongles/2/tmp1_EN.asm}

L'algorithme est un simple \glslink{xoring}{xor}: chaque octet est xoré avec la clef, mais
la clef est incrémentée de 3 après le traitement de chaque octet.

Nous pouvons écrire un petit script Python pour vérifier notre hypothèse:

\lstinputlisting[caption=Python 3.x]{examples/dongles/2/decr1.py}

Et il affiche: \q{check security device connection}.
Donc oui, ceci est le message déchiffré.

Il y a d'autres messages chiffrés, avec leur clef correspondante.
Mais inutile de dire qu'il est possible de les déchiffrer sans leur clef.
Premièrement, nous voyons que le clef est en fait un octet.
C'est parce que l'instruction principale de déchiffrement (\XOR) fonctionne au niveau
de l'octet.
La clef se trouve dans le registre \ESI, mais seulement une partie de \ESI d'un octet
est utilisée.
Ainsi, une clef pourrait être plus grande que 255, mais sa valeur est toujours arrondie.

En conséquence, nous pouvons simplement essayer de brute-forcer, en essayant toutes
les clefs possible dans l'intervalle 0..255.
Nous allons aussi écarter les messages comportants des caractères non-imprimable.

\lstinputlisting[caption=Python 3.x]{examples/dongles/2/decr2.py}

Et nous obtenons:

\lstinputlisting[caption=Results]{examples/dongles/2/decr2_result.txt}

Ici il y a un peu de déchet, mais nous pouvons rapidement trouver les messages en
anglais.

À propos, puisque l'algorithme est un simple chiffrement xor, la même fonction peut
être utilisée pour chiffrer les messages.
Si besoin, nous pouvons chiffrer nos propres messages, et patcher le programme en les insérant.
}


\subsubsection{(Windows) SEH}
\myindex{Windows!Structured Exception Handling}

\ifdefined\RUSSIAN
В стеке хранятся записи \ac{SEH} для функции (если они присутствуют).
Читайте больше о нем здесь: (\myref{sec:SEH}).
\fi % RUSSIAN

\ifdefined\ENGLISH
\ac{SEH} records are also stored on the stack (if they are present).
Read more about it: (\myref{sec:SEH}).
\fi % ENGLISH

\ifdefined\BRAZILIAN
\ac{SEH} também são guardados na pilha (se estiverem presentes).
\PTBRph{}: (\myref{sec:SEH}).
\fi % BRAZILIAN

\ifdefined\ITALIAN
I record \ac{SEH}, se presenti, sono anch'essi memorizzati nello stack.
Maggiori informazioni qui: (\myref{sec:SEH}).
\fi % ITALIAN

\ifdefined\FRENCH
Les enregistrements \ac{SEH} sont aussi stockés dans la pile (s'ils sont présents).
Lire à ce propos: (\myref{sec:SEH}).
\fi % FRENCH


\ifdefined\POLISH
Na stosie są przechowywane wpisy \ac{SEH} dla funkcji (jeśli są one obecne).
Więcej o tym tutaj: (\myref{sec:SEH}).
\fi % POLISH

\ifdefined\JAPANESE
\ac{SEH}レコードはスタックにも格納されます(存在する場合)。
それについてもっと読む:(\myref{sec:SEH})
\fi % JAPANESE

\ifdefined\ENGLISH
\subsubsection{Buffer overflow protection}

More about it here~(\myref{subsec:bufferoverflow}).
\fi

\ifdefined\RUSSIAN
\subsubsection{Защита от переполнений буфера}

Здесь больше об этом~(\myref{subsec:bufferoverflow}).
\fi

\ifdefined\BRAZILIAN
\subsubsection{Proteção contra estouro de buffer}

Mais sobre aqui~(\myref{subsec:bufferoverflow}).
\fi

\ifdefined\ITALIAN
\subsubsection{Protezione contro buffer overflow}

Maggiori informazioni qui~(\myref{subsec:bufferoverflow}).
\fi

\ifdefined\FRENCH
\subsubsection{Protection contre les débordements de tampon}

Lire à ce propos~(\myref{subsec:bufferoverflow}).
\fi


\ifdefined\POLISH
\subsubsection{Ochrona przed przepełnieniem bufora}

Więcej o tym tutaj~(\myref{subsec:bufferoverflow}).
\fi

\ifdefined\JAPANESE
\subsubsection{バッファオーバーフロー保護}

詳細はこちら~(\myref{subsec:bufferoverflow})
\fi


\subsubsection{Automatic deallocation of data in stack}

Perhaps the reason for storing local variables and SEH records in the stack is that they are freed automatically upon function exit,
using just one instruction to correct the stack pointer (it is often \ADD).
Function arguments, as we could say, are also deallocated automatically at the end of function.
In contrast, everything stored in the \emph{heap} must be deallocated explicitly.

% subsections
\subsection{A typical stack layout}

A typical stack layout in a 32-bit environment at the start of a function, 
before the first instruction execution looks like this:

\begin{center}
\begin{tabular}{ | l | l | }
\hline
\dots & \dots \\
\hline
ESP-0xC & \localVariable \#2, \MarkedInIDAAs{} \TT{var\_8} \\
\hline
ESP-8 & \localVariable \#1, \MarkedInIDAAs{} \TT{var\_4} \\
\hline
ESP-4 & \savedValueOf \EBP \\
\hline
ESP & \ReturnAddress \\
\hline
ESP+4 & \argument \#1, \MarkedInIDAAs{} \TT{arg\_0} \\
\hline
ESP+8 & \argument \#2, \MarkedInIDAAs{} \TT{arg\_4} \\
\hline
ESP+0xC & \argument \#3, \MarkedInIDAAs{} \TT{arg\_8} \\
\hline
\dots & \dots \\
\hline
\end{tabular}
\end{center}



% I think this only applies to RISC architectures
% that don't have a POP instruction that only lets you read one value
% (ie. ARM and MIPS).
% In x86, the return address is saved before entering the function,
% and the function does not have the chance to save the frame pointer.
% Also, you should mention that this is how the stack looks like
% right after the function prologue,
% which some readers might think is the first instruction,
% but is needed to save the frame pointer.

\EN{\mysection{Task manager practical joke (Windows Vista)}
\myindex{Windows!Windows Vista}

Let's see if it's possible to hack Task Manager slightly so it would detect more \ac{CPU} cores.

\myindex{Windows!NTAPI}

Let us first think, how does the Task Manager know the number of cores?

There is the \TT{GetSystemInfo()} win32 function present in win32 userspace which can tell us this.
But it's not imported in \TT{taskmgr.exe}.

There is, however, another one in \gls{NTAPI}, \TT{NtQuerySystemInformation()}, 
which is used in \TT{taskmgr.exe} in several places.

To get the number of cores, one has to call this function with the \TT{SystemBasicInformation} constant
as a first argument (which is zero
\footnote{\href{http://msdn.microsoft.com/en-us/library/windows/desktop/ms724509(v=vs.85).aspx}{MSDN}}).

The second argument has to point to the buffer which is getting all the information.

So we have to find all calls to the \\
\TT{NtQuerySystemInformation(0, ?, ?, ?)} function.
Let's open \TT{taskmgr.exe} in IDA. 
\myindex{Windows!PDB}

What is always good about Microsoft executables is that IDA can download the corresponding \gls{PDB} 
file for this executable and show all function names.

It is visible that Task Manager is written in \Cpp and some of the function names and classes are really 
speaking for themselves.
There are classes CAdapter, CNetPage, CPerfPage, CProcInfo, CProcPage, CSvcPage, 
CTaskPage, CUserPage.

Apparently, each class corresponds to each tab in Task Manager.

Let's visit each call and add comment with the value which is passed as the first function argument.
We will write \q{not zero} at some places, because the value there was clearly not zero, 
but something really different (more about this in the second part of this chapter).

And we are looking for zero passed as argument, after all.

\begin{figure}[H]
\centering
\myincludegraphics{examples/taskmgr/IDA_xrefs.png}
\caption{IDA: cross references to NtQuerySystemInformation()}
\end{figure}

Yes, the names are really speaking for themselves.

When we closely investigate each place where\\
\TT{NtQuerySystemInformation(0, ?, ?, ?)} is called,
we quickly find what we need in the \TT{InitPerfInfo()} function:

\lstinputlisting[caption=taskmgr.exe (Windows Vista),style=customasmx86]{examples/taskmgr/taskmgr.lst}

\TT{g\_cProcessors} is a global variable, and this name has been assigned by 
IDA according to the \gls{PDB} loaded from Microsoft's symbol server.

The byte is taken from \TT{var\_C20}. 
And \TT{var\_C58} is passed to\\
\TT{NtQuerySystemInformation()} 
as a pointer to the receiving buffer.
The difference between 0xC20 and 0xC58 is 0x38 (56).

Let's take a look at format of the return structure, which we can find in MSDN:

\begin{lstlisting}[style=customc]
typedef struct _SYSTEM_BASIC_INFORMATION {
    BYTE Reserved1[24];
    PVOID Reserved2[4];
    CCHAR NumberOfProcessors;
} SYSTEM_BASIC_INFORMATION;
\end{lstlisting}

This is a x64 system, so each PVOID takes 8 bytes.

All \emph{reserved} fields in the structure take $24+4*8=56$ bytes.

Oh yes, this implies that \TT{var\_C20} is the local stack is exactly the
\TT{NumberOfProcessors} field of the \TT{SYSTEM\_BASIC\_INFORMATION} structure.

Let's check our guess.
Copy \TT{taskmgr.exe} from \TT{C:\textbackslash{}Windows\textbackslash{}System32} 
to some other folder 
(so the \emph{Windows Resource Protection} 
will not try to restore the patched \TT{taskmgr.exe}).

Let's open it in Hiew and find the place:

\begin{figure}[H]
\centering
\myincludegraphics{examples/taskmgr/hiew2.png}
\caption{Hiew: find the place to be patched}
\end{figure}

Let's replace the \TT{MOVZX} instruction with ours.
Let's pretend we've got 64 CPU cores.

Add one additional \ac{NOP} (because our instruction is shorter than the original one):

\begin{figure}[H]
\centering
\myincludegraphics{examples/taskmgr/hiew1.png}
\caption{Hiew: patch it}
\end{figure}

And it works!
Of course, the data in the graphs is not correct.

At times, Task Manager even shows an overall CPU load of more than 100\%.

\begin{figure}[H]
\centering
\myincludegraphics{examples/taskmgr/taskmgr_64cpu_crop.png}
\caption{Fooled Windows Task Manager}
\end{figure}

The biggest number Task Manager does not crash with is 64.

Apparently, Task Manager in Windows Vista was not tested on computers with a large number of cores.

So there are probably some static data structure(s) inside it limited to 64 cores.

\subsection{Using LEA to load values}
\label{TaskMgr_LEA}

Sometimes, \TT{LEA} is used in \TT{taskmgr.exe} instead of \TT{MOV} to set the first argument of \\
\TT{NtQuerySystemInformation()}:

\lstinputlisting[caption=taskmgr.exe (Windows Vista),style=customasmx86]{examples/taskmgr/taskmgr2.lst}

\myindex{x86!\Instructions!LEA}

Perhaps \ac{MSVC} did so because machine code of \INS{LEA} is shorter than \INS{MOV REG, 5} (would be 5 instead of 4).

\INS{LEA} with offset in $-128..127$ range (offset will occupy 1 byte in opcode) with 32-bit registers is even shorter (for lack of REX prefix)---3 bytes.

Another example of such thing is: \myref{using_MOV_and_pack_of_LEA_to_load_values}.
}%
\RU{\subsection{Обменять входные значения друг с другом}

Вот так:

\lstinputlisting[style=customc]{patterns/061_pointers/swap/5_RU.c}

Как видим, байты загружаются в младшие 8-битные части регистров \TT{ECX} и \TT{EBX} используя \INS{MOVZX}
(так что старшие части регистров очищаются), затем байты записываются назад в другом порядке.

\lstinputlisting[style=customasmx86,caption=Optimizing GCC 5.4]{patterns/061_pointers/swap/5_GCC_O3_x86.s}

Адреса обоих байтов берутся из аргументов и во время исполнения ф-ции находятся в регистрах \TT{EDX} и \TT{EAX}.

Так что исопльзуем указатели --- вероятно, без них нет способа решить эту задачу лучше.

}%
\FR{\subsection{Exemple \#2: SCO OpenServer}

\label{examples_SCO}
\myindex{SCO OpenServer}
Un ancien logiciel pour SCO OpenServer de 1997 développé par une société qui a disparue
depuis longtemps.

Il y a un driver de dongle special à installer dans le système, qui contient les
chaînes de texte suivantes:
\q{Copyright 1989, Rainbow Technologies, Inc., Irvine, CA}
et
\q{Sentinel Integrated Driver Ver. 3.0 }.

Après l'installation du driver dans SCO OpenServer, ces fichiers apparaissent dans
l'arborescence /dev:

\begin{lstlisting}
/dev/rbsl8
/dev/rbsl9
/dev/rbsl10
\end{lstlisting}

Le programme renvoie une erreur lorsque le dongle n'est pas connecté, mais le message
d'erreur n'est pas trouvé dans les exécutables.

\myindex{COFF}

Grâce à \ac{IDA}, il est facile de charger l'exécutable COFF utilisé dans SCO OpenServer.

Essayons de trouver la chaîne \q{rbsl} et en effet, elle se trouve dans ce morceau
de code:

\lstinputlisting[style=customasmx86]{examples/dongles/2/1.lst}

Oui, en effet, le programme doit communiquer d'une façon ou d'une autre avec le driver.

\myindex{thunk-functions}
Le seul endroit où la fonction \TT{SSQC()} est appelée est dans la \glslink{thunk
 function}{fonction thunk}:

\lstinputlisting[style=customasmx86]{examples/dongles/2/2.lst}

SSQ() peut être appelé depuis au moins 2 fonctions.

L'une d'entre elles est:

\lstinputlisting[style=customasmx86]{examples/dongles/2/check1_EN.lst}

\q{\TT{3C}} et \q{\TT{3E}} semblent familiers: il y avait un dongle Sentinel Pro de
Rainbow sans mémoire, fournissant seulement une fonction de crypto-hachage secrète.

Vous pouvez lire une courte description de la fonction de hachage dont il s'agit
ici: \myref{hash_func}.

Mais retournons au programme.

Donc le programme peut seulement tester si un dongle est connecté ou s'il est absent.

Aucune autre information ne peut être écrite dans un tel dongle, puisqu'il n'a pas
de mémoire.
Les codes sur deux caractères sont des commandes (nous pouvons voir comment les commandes
sont traitées dans la fonction \TT{SSQC()}) et toutes les autres chaînes sont hachées
dans le dongle, transformées en un nombre 16-bit.
L'algorithme était secret, donc il n'était pas possible d'écrire un driver de remplacement
ou de refaire un dongle matériel qui l'émulerait parfaitement.

Toutefois, il est toujours possible d'intercepter tous les accès au dongle et de
trouver les constantes auxquelles les résultats de la fonction de hachage sont comparées.

Mais nous devons dire qu'il est possible de construire un schéma de logiciel de protection
de copie robuste basé sur une fonction secrète de hachage cryptographique: il suffit
qu'elle chiffre/déchiffre les fichiers de données utilisés par votre logiciel.

Mais retournons au code:

Les codes 51/52/53 sont utilisés pour choisir le port imprimante LPT.
3x/4x sont utilisés pour le choix de la \q{famille} (c'est ainsi que les dongles
Sentinel Pro sont différenciés les uns des autres: plus d'un dongle peut être connecté
sur un port LPT).

La seule chaîne passée à la fonction qui ne fasse pas 2 caractères est "0123456789".

Ensuite, le résultat est comparé à l'ensemble des résultats valides.

Si il est correct, 0xC ou 0xB est écrit dans la variable globale \TT{ctl\_model}.%

Une autre chaîne de texte qui est passée est
"PRESS ANY KEY TO CONTINUE: ", mais le résultat n'est pas testé.
Difficile de dire pourquoi, probablement une erreur\footnote{C'est un sentiment
étrange de trouver un bug dans un logiciel aussi ancien.}.

Voyons où la valeur de la variable globale \TT{ctl\_model} est utilisée.

Un tel endroit est:

\lstinputlisting[style=customasmx86]{examples/dongles/2/4.lst}

Si c'est 0, un message d'erreur chiffré est passé à une routine de déchiffrement
et affiché.

\myindex{x86!\Instructions!XOR}

La routine de déchiffrement de la chaîne semble être un simple \glslink{xoring}{xor}:

\lstinputlisting[style=customasmx86]{examples/dongles/2/err_warn.lst}

C'est pourquoi nous étions incapable de trouver le message d'erreur dans les fichiers
exécutable, car ils sont chiffrés (ce qui est une pratique courante).

Un autre appel à la fonction de hachage \TT{SSQ()} lui passe la chaîne \q{offln}
et le résultat est comparé avec \TT{0xFE81} et \TT{0x12A9}.

Si ils ne correspondent pas, ça se comporte comme une sorte de fonction \TT{timer()}
(peut-être en attente qu'un dongle mal connecté soit reconnecté et re-testé?) et ensuite
déchiffre un autre message d'erreur à afficher.

\lstinputlisting[style=customasmx86]{examples/dongles/2/check2_EN.lst}

Passer outre le dongle est assez facile: il suffit de patcher tous les sauts après
les instructions \CMP pertinentes.

Une autre option est d'écrire notre propre driver SCO OpenServer, contenant une table
de questions et de réponses, toutes celles qui sont présentent dans le programme.

\subsubsection{Déchiffrer les messages d'erreur}

À propos, nous pouvons aussi essayer de déchiffrer tous les messages d'erreurs.
L'algorithme qui se trouve dans la fonction \TT{err\_warn()} est très simple, en effet:

\lstinputlisting[caption=Decryption function,style=customasmx86]{examples/dongles/2/decrypting_FR.lst}

Comme on le voit, non seulement la chaîne est transmise à la fonction de déchiffrement
mais aussi la clef:

\lstinputlisting[style=customasmx86]{examples/dongles/2/tmp1_EN.asm}

L'algorithme est un simple \glslink{xoring}{xor}: chaque octet est xoré avec la clef, mais
la clef est incrémentée de 3 après le traitement de chaque octet.

Nous pouvons écrire un petit script Python pour vérifier notre hypothèse:

\lstinputlisting[caption=Python 3.x]{examples/dongles/2/decr1.py}

Et il affiche: \q{check security device connection}.
Donc oui, ceci est le message déchiffré.

Il y a d'autres messages chiffrés, avec leur clef correspondante.
Mais inutile de dire qu'il est possible de les déchiffrer sans leur clef.
Premièrement, nous voyons que le clef est en fait un octet.
C'est parce que l'instruction principale de déchiffrement (\XOR) fonctionne au niveau
de l'octet.
La clef se trouve dans le registre \ESI, mais seulement une partie de \ESI d'un octet
est utilisée.
Ainsi, une clef pourrait être plus grande que 255, mais sa valeur est toujours arrondie.

En conséquence, nous pouvons simplement essayer de brute-forcer, en essayant toutes
les clefs possible dans l'intervalle 0..255.
Nous allons aussi écarter les messages comportants des caractères non-imprimable.

\lstinputlisting[caption=Python 3.x]{examples/dongles/2/decr2.py}

Et nous obtenons:

\lstinputlisting[caption=Results]{examples/dongles/2/decr2_result.txt}

Ici il y a un peu de déchet, mais nous pouvons rapidement trouver les messages en
anglais.

À propos, puisque l'algorithme est un simple chiffrement xor, la même fonction peut
être utilisée pour chiffrer les messages.
Si besoin, nous pouvons chiffrer nos propres messages, et patcher le programme en les insérant.
}


\subsection{\Exercises}

\begin{itemize}
	\item \url{http://challenges.re/67}
	\item \url{http://challenges.re/68}
	\item \url{http://challenges.re/69}
	\item \url{http://challenges.re/70}
\end{itemize}


