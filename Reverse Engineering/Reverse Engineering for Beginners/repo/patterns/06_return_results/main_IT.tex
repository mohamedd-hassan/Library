% TODO resync with EN version
\mysection{Ulteriori considerazioni sulla restituzione dei risultati}

\myindex{x86!\Registers!EAX}

In x86, il risultato dell'esecuzione dei una funzione è generalmente restituito
\footnote{\Seealso: MSDN: Return Values (C++): \href{http://msdn.microsoft.com/en-us/library/7572ztz4.aspx}{MSDN}}
nel registro \EAX. 
Se il tipo del risultato è un byte o un \Tchar, viene utilizzata la parte bassa del registro \EAX (\AL). 
Se una funzione restituisce un numero di tipo \Tfloat, viene invece utilizzato il registro FPU \ST{0}.
\myindex{ARM!\Registers!R0}
In ARM, il risultato è solitamente restituito nel registro \Reg{0}.

\subsection{Tentativo di utilizzare il risultato di una funzione che resituisce \Tvoid}
\label{UseResultOfVoidFunc}

Che succederebbe se la funzione main dichiarasse il valore di ritorno di tipo \Tvoid invece di \Tint?
Il cosiddetto startup-code chiama \main più o meno così:

\begin{lstlisting}[style=customasmx86]
push envp
push argv
push argc
call main
push eax
call exit
\end{lstlisting}

In altre parole:

\begin{lstlisting}[style=customc]
exit(main(argc,argv,envp));
\end{lstlisting}

Se dichiariamo \main come \Tvoid, non viene esplicitamente restituito nulla  (usando lo statement \emph{return}),
e quindi qualche valore casuale, che si trova memorizzato nel registro \EAX alla fine di \main, diventa argomento della funzione exit().
Molto probabilmente si tratterà di un valore casuale, residuo dell'esecuzione della nostra funzione, quindi l'exit code del programma 
è pseudo-casuale.
\par
Illustriamo meglio questo fatto. 
La funzione \main ha ora un valore di ritorno di tipo \Tvoid:

\begin{lstlisting}[style=customc]
#include <stdio.h>

void main()
{
	printf ("Hello, world!\n");
};
\end{lstlisting}

Compiliamo il programma su Linux.

\myindex{puts() instead of printf()}
GCC 4.8.1 ha sostituito \printf con \puts 
(abbiamo già visto questo caso: \myref{puts}), e va del tutto bene, poichè \puts restituisce il numero di caratteri stampati proprio come \printf.
Notiamo che \EAX non viene azzerato prima della fine di \main.

Ciò implica che il valore di \EAX alla fine di \main conterrà il valore lasciato lì da \puts.

\begin{lstlisting}[caption=GCC 4.8.1,style=customasmx86]
.LC0:
	.string	"Hello, world!"
main:
	push	ebp
	mov	ebp, esp
	and	esp, -16
	sub	esp, 16
	mov	DWORD PTR [esp], OFFSET FLAT:.LC0
	call	puts
	leave
	ret
\end{lstlisting}

\myindex{bash}

Scriviamo uno script bash che mostra l'exit status:

\begin{lstlisting}[caption=tst.sh]
#!/bin/sh
./hello_world
echo $?
\end{lstlisting}

Eseguiamolo:

\begin{lstlisting}
$ tst.sh 
Hello, world!
14
\end{lstlisting}

14 è il numero di caratteri stampati. Il numero dei caratteri stampati \textit{scivola} da \printf{} attraverso \TT{EAX}/\TT{RAX} nell' \q{exit code}.

Un altro esempio nel libro: \myref{ForgottenReturn}.

\myindex{Hex-Rays}
Comunque, quando decompiliamo C++ in Hex-Rays, spesso possiamo trovare una funzione che termina con il distruttore di qualche classe:

\begin{lstlisting}[style=customasmx86]
...

call    ??1CString@@QAE@XZ ; CString::~CString(void)
mov     ecx, [esp+30h+var_C]
pop     edi
pop     ebx
mov     large fs:0, ecx
add     esp, 28h
retn
\end{lstlisting}

Dallo standard C++, i distruttori non ritornano nulla, ma quando Hex-Rays non lo capisce e pensa che sia il distruttore che la funzione ritornino \Tint, possiamo avere un output simile a questo:

\begin{lstlisting}[style=customc] 
...

        return CString::~CString(&Str);
}
\end{lstlisting}

\subsection{Che succede se il risultato della funzione non viene usato?}

\printf restituisce il numero di caratteri mandati in output con successo, ma il risultato di questa funzione
è usato molto raramente.

E' possibile anche chiamare una funzione la cui essenza risiede nel restituire un valore e non usarlo del tutto:

\begin{lstlisting}[style=customc]
int f()
{
    // salta i primi 3 valori casuali:
    rand();
    rand();
    rand();
    // e usa il 4°:
    return rand();
};
\end{lstlisting}

Il risultato della funzione rand() è lasciato in \EAX in tutti e quattro i casi.
Nei primi 3 però il valore in \EAX non viene usato.

\subsection{Restituire una struttura}

\myindex{\CLanguageElements!return}

Torniamo al fatto che il valore di ritorno è lasciato nel registro \EAX.
Questo è il motivo per cui i vecchi compilatori C non possono creare funzioni in grado di restituire qualcosa che non entri perfettamente in un 
registro (solitamente un \Tint). Se lo si vuole fare, è necessario restituire l'informazione attraverso puntatori passati come argomenti alla funzione.

Quindi, generalmente, se una funzione deve restituire più valori, ne restituisce (realmente) soltanto uno, ed il resto---tramite puntatori.

Oggi è possibile restituire anche un'intera struttura, ma non è ancora una pratica molto diffusa.
Se una funzione deve restituire una struttura grande, il chiamante (\gls{caller}) deve allocarla e passare come primo argomento della funzione un puntatore alla struttura, il tutto in modo trasparente per il programmatore.
E' pressochè la stessa cosa di passare un puntatore manualmente come primo argomento, ma il compilatore "nasconde" questo passaggio.

Un piccolo esempio:

\lstinputlisting[style=customc]{patterns/06_return_results/6_1.c}

\dots otteniamo (MSVC 2010 \Ox):

\lstinputlisting[style=customasmx86]{patterns/06_return_results/6_1.asm}

Il nome della macro per il passaggio interno del puntatore alla struttura è in questo caso \GTT{\$T3853}.

\myindex{\CLanguageElements!C99}
Questo stesso esempio può essere riscritto utilizzando l'estensione del linguaggio C99:

\lstinputlisting[style=customc]{patterns/06_return_results/6_1_C99.c}

\lstinputlisting[caption=GCC 4.8.1,style=customasmx86]{patterns/06_return_results/6_1_C99.asm}

Come possiamo vedere, la funzione chiamata non fa altro che riempire i campi della struttura allocata dalla funzione chiamante,
come se un puntatore alla struttura fosse stato passato.
Pertanto non ci sono neanche impatti negativi sulla performance.
