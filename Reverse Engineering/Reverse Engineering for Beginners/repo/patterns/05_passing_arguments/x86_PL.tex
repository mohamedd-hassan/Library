\subsection{x86}

\subsubsection{MSVC}

Poniżej wynik kompilacji (MSVC 2010 Express):

\lstinputlisting[label=src:passing_arguments_ex_MSVC_cdecl,caption=MSVC 2010 Express,style=customasmx86]{patterns/05_passing_arguments/msvc_EN.asm}

\myindex{x86!\Registers!EBP}

Na listingu widać, jak funkcja \main odkłada na stos 3 liczby i wywołuje \TT{f(int,int,int).}

Dostęp do argumentów \ttf uzyskuje za pomocą makr, jak np.:\\
\TT{\_a\$ = 8},
podobnie jak do zmiennych lokalnych, ale z dodatnim przesunięciem.
Niejako adresujemy pamięć \emph{poza stosem}, gdyż stos rośnie w dół, a my dodajemy wartość dodatnią  \TT{\_a\$} do rejestru \EBP (wskaźnik ramki stosu).

\myindex{x86!\Instructions!IMUL}
\myindex{x86!\Instructions!ADD}

Następnie wartość $a$ jest zapisywana do \EAX. Po wykonaniu instrukcji \IMUL, wartość w \EAX
jest iloczynem wartości z \EAX i wartości wskazywanej przez przesunięcie \TT{\_b}.

Kolejno wykonywana jest instrukcja \ADD, która dodaje wartość pokazywaną przez przesunięcie \TT{\_c} do \EAX.

Wartość w \EAX już nie musi być nigdzie zapisywana, gdyż jest to wynik funkcji, a w tej konwencji wywoływania jest on zwracany przez rejestr \EAX.
Po powrocie \glslink{caller}{funkcja wywołująca} pobiera wartość z \EAX i używa jako argumentu do \printf.

\clearpage
\subsubsection{MSVC: x86 + \olly}

Spróbujmy zhackować nasz program w \olly, zmuszając go, by uznał, że funkcja \scanf wykonała się bez błędów.
Kiedy adres zmiennej lokalnej jest przekazywany do \scanf,
zmienna początkowo zawiera przypadkową wartość, w tym wypadku \TT{0x6E494714}:

\begin{figure}[H]
\centering
\myincludegraphics{patterns/04_scanf/3_checking_retval/olly_1.png}
\caption{\olly: przekazywanie adresu zmiennej do \scanf}
\label{fig:scanf_ex3_olly_1}
\end{figure}

\clearpage
Kiedy wykonywana jest funkcja \scanf , w konsoli wpiszmy coś, co z pewnością nie jest liczbą, na przykład \q{asdasd}.
\scanf kończy działanie z 0 w \EAX, co wskazuje na wystąpienie błędu.

Możemy sprawdzić wartość zmiennej lokalnej na stosie i zauważyć, że się ona nie zmieniła.
W rzeczy samej, dlaczego funkcja \scanf miałaby cokolwiek tam zapisać?
Jej wykonanie nie spowodowało nic, poza zwróceniem zera.

Spróbujmy \q{zhackować} nasz program.
Kliknij prawym przyciskiem na \EAX,
wśród opcji znajduje się \q{Set to 1} (\emph{ustaw na 1}).
To jest to, czego szukamy.

Mamy teraz 1 w \EAX, a więc kolejne sprawdzenie powinno się wykonać zgodnie z oczekiwaniami i
\printf powinna wyświetlić wartość zmiennej ze stosu.

Po wznowieniu wykonania programu (F9) widzimy następujący efekt w oknie konsoli:

\lstinputlisting[caption=console window]{patterns/04_scanf/3_checking_retval/console.txt}

1850296084 to postać dziesiętna liczby, którą widzieliśmy na stosie (\TT{0x6E494714})!


\subsubsection{GCC}

Skompilujmy ten sam przykład w GCC 4.4.1 i podejrzyjmy rezultat w programie \IDA:

\lstinputlisting[caption=GCC 4.4.1,style=customasmx86]{patterns/05_passing_arguments/gcc_PL.asm}

Efekt jest taki sam, z małymi różnicami, które już omówiliśmy wcześniej.

\glslink{stack pointer}{Wskaźnik stosu} nie jest przywracany po dwóch wywołaniach funkcji (f oraz \printf),
ponieważ przedostatnia instrukcja \TT{LEAVE} (\myref{x86_ins:LEAVE}) zajmie się tym na końcu.
