\subsection{MIPS}

\lstinputlisting[caption=\Optimizing GCC 4.4.5,style=customasmMIPS]{patterns/05_passing_arguments/MIPS_O3_IDA_IT.lst}

I primi quattro argomenti della funzione sono passati in quattro registri con prefisso A-.

\myindex{MIPS!\Instructions!MULT}

Ci sono due registri speciali in MIPS: HI e LO che durante l'esecuzione dell' istruzione \TT{MULT}, vengono riempiti con il risultato su 64-bit della moltiplicazione.
\myindex{MIPS!\Instructions!MFLO}
\myindex{MIPS!\Instructions!MFHI}

Questi registri sono accessibili solamente usando le istruzioni \TT{MFLO} e \TT{MFHI}.
In questo caso \TT{MFLO} prende la parte bassa del risultato della moltiplicazione e la salva in \$V0.
Di conseguenza i 32 bit della parte alta del risultato della moltiplicazione sono scartati (il contenuto del registro HI non viene usato).
Infatti: noi qui lavoriamo con tipi di dati \Tint a 32 bit.

\myindex{MIPS!\Instructions!ADDU}

Infine, \TT{ADDU} (\q{Add Unsigned}) sommano il valore del terzo argomento al risultato.

\myindex{MIPS!\Instructions!ADD}
\myindex{MIPS!\Instructions!ADDU}
\myindex{Ada}
\myindex{Integer overflow}

Ci sono due tipi di istruzione addizione in MIPS: \TT{ADD} and \TT{ADDU}.
La differnza non è legata al segno, ma alle eccezioni. \TT{ADD} può sollevare un' eccezione in caso di overflow, che di solito è utile\footnote{\url{http://blog.regehr.org/archives/1154}} e supportato in Ada \ac{PL}, per esempio.
\TT{ADDU} non solleva eccezioni in caso di overlflow.

Siccome \CCpp non lo supporta, nei nostri esempi vediamo \TT{ADDU} anzichè \TT{ADD}.

Il risultato a 32 bit viene lasciato in \$V0.

\myindex{MIPS!\Instructions!JAL}
\myindex{MIPS!\Instructions!JALR}

C'è una nuova istruzione per noi nel \main: \TT{JAL} (\q{Jump and Link}). 

La differenza tra \INS{JAL} e \INS{JALR} è che l'offset relativo viene codificato nella prima istruzione, 
mentre \INS{JALR} salta all'indirizzo assoluto salvato in un registro (\q{Jump and Link Register}).

Entrambe le funzioni \ttf e \main si trovano nello stesso file oggetto, quindi l'indirizzo relativo di \ttf è conosciuto e fissato.
