\subsection{x86}

\subsubsection{MSVC}

Das ist das Ergebnis nach dem kompilieren (MSVC 2010 Express):

\lstinputlisting[label=src:passing_arguments_ex_MSVC_cdecl,caption=MSVC 2010 Express,style=customasmx86]{patterns/05_passing_arguments/msvc_DE.asm}

\myindex{x86!\Registers!EBP}

Was wir hier sehen ist das die \main Funktion drei Zahlen auf den Stack schiebt und \TT{f(int,int,int).} aufruft

Der Argument zugriff innerhalb von \ttf wird organisiert mit der Hilfe von Makros wie zum Beispiel:\\
\TT{\_a\$ = 8}, 
auf die gleiche weise wie Lokale Variablen allerdings mit positiven Offsets (adressiert mit \emph{plus}).

Also adressieren wir die \emph{äussere} Seite des \glslink{stack frame}{Stack frame} indem wir \TT{\_a\$} Makros zum Wert des \EBP Registers addieren  

\myindex{x86!\Instructions!IMUL}
\myindex{x86!\Instructions!ADD}

Dann wird der Wert von $a$ in \EAX gespeichert. Nachdem die \IMUL Instruktion ausgeführt wurde, ist
der Wert in \EAX ein Produkt des Wertes aus \EAX und dem Inhalt von \TT{\_b}.

Nun addiert \ADD den Wert in \TT{\_c} auf \EAX

Der Wert in \EAX muss nicht verschoben werden: Der Wert von \EAX befindet sich schon wo er sein muss

Beim zurück kehren zur \gls{caller} Funktion, wird der Wert aus \EAX genommen und als Argument 
für den \printf Aufruf benutzt.


\clearpage
\myparagraph{\olly + standardmäßig gepackte Felder}
\myindex{\olly}
Betrachten wir unser Beispiel (in dem die Felder standardmäßig auf 4 Byte angeordnet werden) in \olly:

\begin{figure}[H]
\centering
\myincludegraphics{patterns/15_structs/4_packing/olly_packing_4.png}
\caption{\olly: vor der Ausführung von \printf}
\label{fig:packing_olly_4}
\end{figure}
Wir sehen unsere 4 Felder im Datenfenster.

Wir fragen uns aber, woher die Zufallsbytes (0x30, 0x37, 0x01) stammen, die neben dem ersten ($a$) und dritten ($c$)
Feld liegen.

Betrachten wir unser Listing \myref{src:struct_packing_4}, erkennen wir, dass das erste und dritte Feld vom Typ \Tchar
ist, und daher nur ein Byte geschrieben wird, nämlich 1 bzw. 3 (Zeilen 6 und 8).

Die übrigen 3 Byte des 32-Bit-Wortes werden im Speicher nicht verändert!
Deshalb befinden sich hier zufällige Reste.

\myindex{x86!\Instructions!MOVSX}
Diese Reste beeinflussen den Output von \printf in keinster Weise, da die Werte für die Funktion mithilfe von \MOVSX
vorbereitet werden, der Bytes und nicht Worte als Argumente hat: 
\lstref{src:struct_packing_4} (Zeilen 34 und 38).
Der vorzeichenerweiternde Befehl \MOVSX wird hier übrigens verwendet, da \Tchar standardmäßig in MSVC und GCC
vorzeichenbehaftet ist.
Würde hier der Datentyp \TT{unsigned char} oder \TT{uint8\_t} verwendet, würde der Befehl \MOVZX stattdessen verwendet.

\clearpage
\myparagraph{\olly + Felder auf 1 Byte Grenzen angeordnet}
\myindex{\olly}
Hier sind die Dinge viel klarer ersichtlich: 4 Felder benötigen 16 Byte und die Werte werden nebeneinander gespeichert.

\begin{figure}[H]
\centering
\myincludegraphics{patterns/15_structs/4_packing/olly_packing_1.png}
\caption{\olly: Vor der Ausführung von \printf}
\label{fig:packing_olly_1}
\end{figure}


\subsubsection{GCC}


Lasst uns das gleiche in GCC kompilieren und die Ergebnisse in \IDA betrachten:

\lstinputlisting[caption=GCC 4.4.1,style=customasmx86]{patterns/05_passing_arguments/gcc_DE.asm}

Das Ergebnis ist fast das gleiche aber mit kleineren Unterschieden die wir bereits früher
besprochen haben.

Der \gls{stack pointer} wird nicht zurück gesetzt nach den beiden Funktion aufrufen (f und printf),
weil die vorletzte \TT{LEAVE} Instruktion (\myref{x86_ins:LEAVE}) sich um das zurück setzen kümmert.
