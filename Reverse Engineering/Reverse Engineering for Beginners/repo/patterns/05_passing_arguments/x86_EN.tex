\subsection{x86}

\subsubsection{MSVC}

Here is what we get after compilation (MSVC 2010 Express):

\lstinputlisting[label=src:passing_arguments_ex_MSVC_cdecl,caption=MSVC 2010 Express,style=customasmx86]{patterns/05_passing_arguments/msvc_EN.asm}

\myindex{x86!\Registers!EBP}

What we see is that the \main function pushes 3 numbers onto the stack and calls \TT{f(int,int,int).} 

Argument access inside \ttf is organized with the help of macros like:\\
\TT{\_a\$ = 8}, 
in the same way as local variables, but with positive offsets (addressed with \emph{plus}).
So, we are addressing the \emph{outer} side of the \gls{stack frame} by adding the \TT{\_a\$} macro to the value in the \EBP register.

\myindex{x86!\Instructions!IMUL}
\myindex{x86!\Instructions!ADD}

Then the value of $a$ is stored into \EAX. After \IMUL instruction execution, the value in \EAX is 
a \gls{product} of the value in \EAX and the content of \TT{\_b}.

After that, \ADD adds the value in \TT{\_c} to \EAX.

The value in \EAX does not need to be moved: it is already where it must be.
On returning to \gls{caller}, it takes the \EAX value and uses it as an argument to \printf.

\clearpage
\subsubsection{MSVC + \olly}
\myindex{\olly}

Let's load our example into \olly and set a breakpoint on \comp.
We can see how the values are compared at the first \comp call:

\begin{figure}[H]
\centering
\myincludegraphics{patterns/18_pointers_to_functions/olly1.png}
\caption{\olly: first call of \comp}
\label{fig:qsort_olly1}
\end{figure}

\olly shows the compared values in the window under the code window, for convenience.
We can also see that the \ac{SP} points to \ac{RA}, where the \qsort function is (located in \TT{MSVCR100.DLL}).

\clearpage
By tracing (F8) until the \TT{RETN} instruction and pressing F8 one more time, we return to the \qsort function:

\begin{figure}[H]
\centering
\myincludegraphics{patterns/18_pointers_to_functions/olly2.png}
\caption{\olly: the code in \qsort right after \comp call}
\label{fig:qsort_olly2}
\end{figure}

That has been a call to the comparison function.

\clearpage
Here is also a screenshot of the moment of the second call of \comp{}---now values that have to be compared are different:

\begin{figure}[H]
\centering
\myincludegraphics{patterns/18_pointers_to_functions/olly3.png}
\caption{\olly: second call of \comp}
\label{fig:qsort_olly3}
\end{figure}


\subsubsection{GCC}

Let's compile the same in GCC 4.4.1 and see the results in \IDA:

\lstinputlisting[caption=GCC 4.4.1,style=customasmx86]{patterns/05_passing_arguments/gcc_EN.asm}

The result is almost the same with some minor differences discussed earlier.

The \gls{stack pointer} is not set back after the two function calls(f and printf), 
because the penultimate \TT{LEAVE} (\myref{x86_ins:LEAVE}) 
instruction takes care of this at the end.
