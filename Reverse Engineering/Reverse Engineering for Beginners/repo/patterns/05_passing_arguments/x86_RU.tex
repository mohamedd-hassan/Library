\subsection{x86}

\subsubsection{MSVC}

Рассмотрим пример, скомпилированный в (MSVC 2010 Express):

\lstinputlisting[label=src:passing_arguments_ex_MSVC_cdecl,caption=MSVC 2010 Express,style=customasmx86]{patterns/05_passing_arguments/msvc_RU.asm}

\myindex{x86!\Registers!EBP}
Итак, здесь видно: в функции \main заталкиваются три числа в стек и вызывается функция \\
\TT{f(int,int,int)}.
 
Внутри \ttf доступ к аргументам, также как и к локальным переменным, происходит через макросы: 
\TT{\_a\$ = 8}, но разница в том, что эти смещения со знаком \emph{плюс}, 
таким образом если прибавить макрос \TT{\_a\$} к указателю на \EBP, то адресуется \emph{внешняя} 
часть \glslink{stack frame}{фрейма} стека относительно \EBP.

\myindex{x86!\Instructions!IMUL}
\myindex{x86!\Instructions!ADD}
Далее всё более-менее просто: значение $a$ помещается в \EAX. 
Далее \EAX умножается при помощи инструкции \IMUL на то, что лежит в \TT{\_b}, 
и в \EAX остается \glslink{product}{произведение} этих двух значений.

Далее к регистру \EAX прибавляется то, что лежит в \TT{\_c}.

Значение из \EAX никуда не нужно перекладывать, оно уже лежит где надо. 
Возвращаем управление вызывающей функции~--- она возьмет значение из \EAX и отправит его в \printf.

\clearpage
\myparagraph{\olly + упаковка полей по умолчанию}
\myindex{\olly}

Попробуем в \olly наш пример, где поля выровнены по умолчанию (4 байта):

\begin{figure}[H]
\centering
\myincludegraphics{patterns/15_structs/4_packing/olly_packing_4.png}
\caption{\olly: Перед исполнением \printf}
\label{fig:packing_olly_4}
\end{figure}

В окне данных видим наши четыре поля.
Вот только, откуда взялись случайные байты (0x30, 0x37, 0x01) рядом с первым (a) и третьим (c) полем?

Если вернетесь к листингу \myref{src:struct_packing_4}, то увидите, что первое и третье поле имеет
тип \Tchar, а следовательно, туда записывается только один байт, 1 и 3 соответственно (строки 6 и 8).

Остальные три байта 32-битного слова не будут модифицироваться в памяти!

А, следовательно, там остается случайный мусор.
\myindex{x86!\Instructions!MOVSX}
Этот мусор никак не будет влиять на работу \printf,
потому что значения для нее готовятся при помощи инструкции \MOVSX, которая загружает
из памяти байты а не слова: 
\lstref{src:struct_packing_4} (строки 34 и 38).

Кстати, здесь используется именно \MOVSX (расширяющая знак), потому что тип 
\Tchar --- знаковый по умолчанию в MSVC и GCC.

Если бы здесь был тип \TT{unsigned char} или \TT{uint8\_t}, 
то здесь была бы инструкция \MOVZX.

\clearpage
\myparagraph{\olly + упаковка полей по границе в 1 байт}
\myindex{\olly}

Здесь всё куда понятнее: 4 поля занимают 10 байт и значения сложены в памяти друг к другу

\begin{figure}[H]
\centering
\myincludegraphics{patterns/15_structs/4_packing/olly_packing_1.png}
\caption{\olly: Перед исполнением \printf}
\label{fig:packing_olly_1}
\end{figure}


\subsubsection{GCC}

Скомпилируем то же в GCC 4.4.1 и посмотрим результат в \IDA:

\lstinputlisting[caption=GCC 4.4.1,style=customasmx86]{patterns/05_passing_arguments/gcc_RU.asm}

Практически то же самое, если не считать мелких отличий описанных ранее.

После вызова обоих функций \glslink{stack pointer}{указатель стека} не возвращается назад, 
потому что предпоследняя инструкция \TT{LEAVE} (\myref{x86_ins:LEAVE}) делает это за один раз, в конце исполнения.

