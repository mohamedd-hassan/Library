\subsection{x86}

\subsubsection{MSVC}

Voici ce que l'on obtient après compilation (MSVC 2010 Express) :

\lstinputlisting[label=src:passing_arguments_ex_MSVC_cdecl,caption=MSVC 2010 Express,style=customasmx86]{patterns/05_passing_arguments/msvc_FR.asm}

\myindex{x86!\Registers!EBP}

Ce que l'on voit, c'est que la fonction \main pousse 3 nombres sur la pile et appelle
\TT{f(int,int,int)}.

L'accès aux arguments à l'intérieur de \ttf est organisé à l'aide de macros
comme:\\
\TT{\_a\$ = 8},
de la même façon que pour les variables locales, mais avec des offsets positifs
(accédés avec \emph{plus}).
Donc, nous accédons à la partie \emph{hors} de la \glslink{stack frame}{structure locale de pile}
en ajoutant la macro \TT{\_a\$} à la valeur du registre \EBP.

\myindex{x86!\Instructions!IMUL}
\myindex{x86!\Instructions!ADD}

Ensuite, la valeur de $a$ est stockée dans \EAX. Après l'exécution de l'instruction
\IMUL, la valeur de \EAX est le \glslink{product}{produit} de la valeur de \EAX
et du contenu de \TT{\_b}.

Après cela, \ADD ajoute la valeur dans \TT{\_c} à \EAX.

La valeur dans \EAX n'a pas besoin d'être déplacée/copiée : elle est déjà là
où elle doit être.
Lors du retour dans la fonction \glslink{caller}{appelante}, elle prend la valeur dans
\EAX et l'utilise comme argument pour \printf.

\clearpage
\subsubsection{MSVC: x86 + \olly}

Essayons de hacker notre programme dans \olly, pour le forcer à penser que \scanf
fonctionne toujours sans erreur.
Lorsque l'adresse d'une variable locale est passée à \scanf, la variable contient
initiallement toujours des restes de données aléatoires, dans ce cas \TT{0x6E494714}:

\begin{figure}[H]
\centering
\myincludegraphics{patterns/04_scanf/3_checking_retval/olly_1.png}
\caption{\olly: passer l'adresse de la variable à \scanf}
\label{fig:scanf_ex3_olly_1}
\end{figure}

\clearpage
Lorsque \scanf s'exécute dans la console, entrons quelque chose qui n'est pas du
tout un nombre, comme \q{asdasd}.
\scanf termine avec 0 dans \EAX, ce qui indique qu'une erreur s'est produite.

Nous pouvons vérifier la variable locale dans le pile et noter qu'elle n'a pas changé.
En effet, qu'aurait écrit \scanf ici?
Elle n'a simplement rien fait à part renvoyer zéro.

Essayons de \q{hacker} notre programme.
Clique-droit sur \EAX,
parmi les options il y a \q{Set to 1} (mettre à 1).
C'est ce dont nous avons besoin.

Nous avons maintenant 1 dans \EAX, donc la vérification suivante va s'exécuter comme
souhaiter et \printf va afficher la valeur de la variable dans la pile.

Lorsque nous lançons le programme (F9) nous pouvons voir ceci dans la fenêtre
de la console:

\lstinputlisting[caption=fenêtre console]{patterns/04_scanf/3_checking_retval/console.txt}

En effet, 1850296084 est la représentation en décimal du nombre dans la pile (\TT{0x6E494714})!


\subsubsection{GCC}

Compilons le même code avec GCC 4.4.1 et regardons le résultat dans \IDA :

\lstinputlisting[caption=GCC 4.4.1,style=customasmx86]{patterns/05_passing_arguments/gcc_FR.asm}

Le résultat est presque le même, avec quelques différences mineures discutées
précédemment.

Le \glslink{stack pointer}{pointeur de pile} n'est pas remis après les deux appels
de fonction (f et printf), car la pénultième instruction \TT{LEAVE} (\myref{x86_ins:LEAVE})
s'en occupe à la fin.
