\subsubsection{\NonOptimizingKeilVI (\ARMMode)}

\lstinputlisting[style=customasmARM]{patterns/05_passing_arguments/ARM/tmp1.asm}

La fonction \main appelle simplement deux autres fonctions, avec trois valeurs passées
à la première~---(\ttf).

Comme il y déjà été écrit, en ARM les 4 premières valeurs sont en général
passées par les 4 premiers registres (\Reg{0}-\Reg{3}).

La fonction \ttf, comme il semble, utilise les 3 premiers registres (\Reg{0}-\Reg{2}) comme arguments. 

\myindex{ARM!\Instructions!MLA}
L'instruction \TT{MLA} (\emph{Multiply Accumulate}) multiplie ses deux premiers opérandes
(\Reg{3} et \Reg{1}), additionne le troisième opérande (\Reg{2}) au produit et stocke
le résultat dans le registre zéro (\Reg{0}), par lequel, d'après le standard, les
fonctions retournent leur résultat.

\myindex{Fused multiply–add}
La multiplication et l'addition en une fois (\emph{Fused multiply–add})
est une instruction très utile. À propos, il n'y avait pas une telle instruction en
x86 avant les instructions FMA apparues en SIMD
\footnote{\href{https://en.wikipedia.org/wiki/FMA_instruction_set}{Wikipédia}}.

La toute première instruction \TT{MOV R3, R0}, est, apparemment, redondante (car
une seule instruction \TT{MLA} pourrait être utilisée à la place ici).
Le compilateur ne l'a pas optimisé, puisqu'il n'y a pas l'option d'optimisation.

\myindex{ARM!Mode switching}
\myindex{ARM!\Instructions!BX}

L'instruction \TT{BX} rend le contrôle à l'adresse stockée dans le registre \ac{LR}
et, si nécessaire, change le mode du processeur de Thumb à ARM et vice versa.
Ceci peut être nécessaire puisque, comme on peut le voir, la fonction \ttf n'est
pas au courant depuis quel sorte de code elle peut être appelée, ARM ou Thumb.
Ainsi, si elle est appelée depuis du code Thumb, \TT{BX} ne va pas seulement retourner
le contrôle à la fonction appelante, mais également changer le mode du processeur
à Thumb.
Ou ne pas changer si la fonction a été appelée depuis du code ARM \InSqBrackets{\ARMSevenRef A2.3.2}.
% look for "BXWritePC()" in manual
