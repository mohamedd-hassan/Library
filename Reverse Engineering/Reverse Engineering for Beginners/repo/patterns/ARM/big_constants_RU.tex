\subsection{Загрузка констант в регистр}
\label{ARM_big_constants}

\subsubsection{32-битный ARM}
\label{ARM_big_constants_loading}

Как мы уже знаем, все инструкции имеют длину в 4 байта в режиме ARM и 2 байта в режиме Thumb.

Как в таком случае записать в регистр 32-битное число, если его невозможно закодировать
внутри одной инструкции?

Попробуем:

\begin{lstlisting}[style=customc]
unsigned int f()
{
	return 0x12345678;
};
\end{lstlisting}

\begin{lstlisting}[caption=GCC 4.6.3 -O3 \ARMMode,style=customasmARM]
f:
        ldr     r0, .L2
        bx      lr
.L2:
        .word   305419896 ; 0x12345678
\end{lstlisting}

Т.е., значение \TT{0x12345678} просто записано в памяти отдельно и загружается, если нужно.

Но можно обойтись и без дополнительного обращения к памяти.

\begin{lstlisting}[caption=GCC 4.6.3 -O3 -march{=}armv7-a (\ARMMode),style=customasmARM]
movw    r0, #22136      ; 0x5678
movt    r0, #4660       ; 0x1234
bx      lr
\end{lstlisting}

Видно, что число загружается в регистр по частям, в начале младшая часть 
(при помощи инструкции \INS{MOVW}), затем старшая (при помощи \INS{MOVT}).

Следовательно, нужно 2 инструкции в режиме ARM, чтобы записать 32-битное число в регистр.

Это не так уж и страшно, потому что в реальном коде не так уж и много констант (кроме 0 и 1).

Значит ли это, что это исполняется медленнее чем одна инструкция, как две инструкции?

Вряд ли, наверняка современные процессоры ARM наверняка умеют распознавать такие 
последовательности и исполнять их быстро.

А \IDA легко распознает подобные паттерны в коде и дизассемблирует эту функцию как:

\begin{lstlisting}[style=customasmARM]
MOV    R0, 0x12345678
BX     LR
\end{lstlisting}

\subsubsection{ARM64}

\begin{lstlisting}[style=customc]
uint64_t f()
{
	return 0x12345678ABCDEF01;
};
\end{lstlisting}

\begin{lstlisting}[caption=GCC 4.9.1 -O3,style=customasmARM]
mov	x0, 61185   ; 0xef01
movk	x0, 0xabcd, lsl 16
movk	x0, 0x5678, lsl 32
movk	x0, 0x1234, lsl 48
ret
\end{lstlisting}

\myindex{ARM!\Instructions!MOVK}
\TT{MOVK} означает \q{MOV Keep}, т.е. она записывает 16-битное значение в регистр, не трогая
при этом остальные биты.
\myindex{ARM!Optional operators!LSL}
Суффикс \TT{LSL} сдвигает значение в каждом случае влево на 16, 32 и 48 бит. Сдвиг происходит
перед загрузкой.
Таким образом, нужно 4 инструкции, чтобы записать в регистр 64-битное значение.

\myparagraph{Записать числа с плавающей точкой в регистр}

Некоторые числа можно записывать в D-регистр при помощи только одной инструкции.

Например:

\begin{lstlisting}[style=customc]
double a()
{
	return 1.5;
};
\end{lstlisting}

\begin{lstlisting}[caption=GCC 4.9.1 -O3 + objdump,style=customasmARM]
0000000000000000 <a>:
   0:   1e6f1000        fmov    d0, #1.500000000000000000e+000
   4:   d65f03c0        ret
\end{lstlisting}

Число $1.5$ действительно было закодировано в 32-битной инструкции.

Но как?
\myindex{ARM!\Instructions!FMOV}
В ARM64, инструкцию \TT{FMOV} есть 8 бит для кодирования некоторых чисел с плавающей запятой.

В \ARMSixFourRefURL алгоритм называется \TT{VFPExpandImm()}.

\myindex{minifloat}
Это также называется \emph{minifloat}\footnote{\href{http://en.wikipedia.org/wiki/Minifloat}{wikipedia}}.
Мы можем попробовать разные: $30.0$ и $31.0$ компилятору удается закодировать, а $32.0$ уже нет, для него
приходится выделять 8 байт в памяти и записать его там в формате IEEE 754:

\begin{lstlisting}[style=customc]
double a()
{
	return 32;
};
\end{lstlisting}

\begin{lstlisting}[caption=GCC 4.9.1 -O3,style=customasmARM]
a:
	ldr	d0, .LC0
	ret
.LC0:
	.word	0
	.word	1077936128
\end{lstlisting}
