\subsubsection{x86}

\myparagraph{\NonOptimizing MSVC}

Wir erhalten (MSVC 2010):

\lstinputlisting[caption=MSVC 2010,style=customasmx86]{patterns/08_switch/2_lot/lot_msvc_DE.asm}

\myindex{jumptable}
Was wir hier sehen ist eine Ansammlung von Aufrufen von \printf mit diversen Argumenten.
Alle haben nicht Adressen im Speicher des Prozesses, sondern auch interne symbolische Labels, die ihnen vom Compiler
zugewiesen werden.
Alle diese Labels werden auch in der internen Tabelle \TT{\$LN11@f} aufgeführt. 

Zu Beginn der Funktion wird der Control Flow an das Label \TT{\$LN1@f} abgegeben, wenn $a$ größer ist als 4. An diesem
Label wird \printf mit dem Argument \TT{'something unknown'} aufgerufen.

Wenn aber der Wert von $a$ kleiner gleich 4 ist, dann wird dieser mit 4 multipliziert und mit der Tabellenadresse
\TT{\$LN11@f} addiert. Auf diese Weise wird die Adresse innerhalb der Tabelle konstruiert und zeigt genau auf das
gewünschte Element. Nehmen wir zum Beispiel an, dass $a$ gleich 2 ist. $2\cdot 4=8$ (alle Tabellenelemente sind
Adressen in einem 32-Bit-Prozess und haben daher eine Breite von 4 Bytes).
Die Adresse an der Stelle \TT{\$LN11@f} + 8 ist das Tabellenelement, an dem das Label \TT{\$LN4@f} gespeichert ist.
\JMP holt die Adresse \TT{\$LN4@f} aus der Tabelle und springt dorthin.

Diese Tabelle wird manchmal \emph{Jumptable} oder \emph{Verzweigungstabelle} genannt\footnote{Die ganze Methode wurde
in früheren Versionen von Fortran \emph{berechnetes GOTO} genannt:
\href{http://en.wikipedia.org/wiki/Branch_table}{wikipedia}.
Heutzutage zwar nicht mehr relevant, aber welch ein Ausdruck!}.

Dann wird das zugehörige \printf mit dem Argument \TT{'two'} aufgerufen.\\
Der Befehl TT{jmp DWORD PTR \$LN11@f[ecx*4]} bedeutet dabei \emph{springe zum an dieser Stelle gespeicherten
DWORD}\TT{\$LN11@f + ecx * 4}.

\TT{npad} (\myref{sec:npad}) ist ein Assemblermakro, dass das nächste Label so angeordnet, dass es an einer 4 Byte
(oder 16 Bit) Adressgrenze gespeichert wird. Das ist für den Prozessor sehr praktisch, da er die 32-Bit-Werte aus dem
Speicher durch den Speicherbus, den Cache, etc. in effektiverer Weise laden kann.

\clearpage
\myparagraph{\olly + standardmäßig gepackte Felder}
\myindex{\olly}
Betrachten wir unser Beispiel (in dem die Felder standardmäßig auf 4 Byte angeordnet werden) in \olly:

\begin{figure}[H]
\centering
\myincludegraphics{patterns/15_structs/4_packing/olly_packing_4.png}
\caption{\olly: vor der Ausführung von \printf}
\label{fig:packing_olly_4}
\end{figure}
Wir sehen unsere 4 Felder im Datenfenster.

Wir fragen uns aber, woher die Zufallsbytes (0x30, 0x37, 0x01) stammen, die neben dem ersten ($a$) und dritten ($c$)
Feld liegen.

Betrachten wir unser Listing \myref{src:struct_packing_4}, erkennen wir, dass das erste und dritte Feld vom Typ \Tchar
ist, und daher nur ein Byte geschrieben wird, nämlich 1 bzw. 3 (Zeilen 6 und 8).

Die übrigen 3 Byte des 32-Bit-Wortes werden im Speicher nicht verändert!
Deshalb befinden sich hier zufällige Reste.

\myindex{x86!\Instructions!MOVSX}
Diese Reste beeinflussen den Output von \printf in keinster Weise, da die Werte für die Funktion mithilfe von \MOVSX
vorbereitet werden, der Bytes und nicht Worte als Argumente hat: 
\lstref{src:struct_packing_4} (Zeilen 34 und 38).
Der vorzeichenerweiternde Befehl \MOVSX wird hier übrigens verwendet, da \Tchar standardmäßig in MSVC und GCC
vorzeichenbehaftet ist.
Würde hier der Datentyp \TT{unsigned char} oder \TT{uint8\_t} verwendet, würde der Befehl \MOVZX stattdessen verwendet.

\clearpage
\myparagraph{\olly + Felder auf 1 Byte Grenzen angeordnet}
\myindex{\olly}
Hier sind die Dinge viel klarer ersichtlich: 4 Felder benötigen 16 Byte und die Werte werden nebeneinander gespeichert.

\begin{figure}[H]
\centering
\myincludegraphics{patterns/15_structs/4_packing/olly_packing_1.png}
\caption{\olly: Vor der Ausführung von \printf}
\label{fig:packing_olly_1}
\end{figure}


\myparagraph{\NonOptimizing GCC}
\label{switch_lot_GCC}

Schauen wir was GCC 4.4.1 erzeugt:

\lstinputlisting[caption=GCC 4.4.1,style=customasmx86]{patterns/08_switch/2_lot/lot_gcc.asm}

\myindex{x86!\Registers!JMP}
Es ist bis auf eine Nuance das gleiche: das Argument \TT{arg\_0} wird mit 4 multipliziert durch eine Verschiebung von 2
Bits nach links (dies entspricht einer Multiplikation mit 4)~(\myref{SHR}).
Dann wird die Adresse des Labels vom \TT{off\_804855C} genommen, die in \EAX gespeichert wird, und dann wird mit
\TT{JMP EAX} der eigentliche Sprung durchgeführt.


