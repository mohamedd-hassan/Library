\subsubsection{x86}

\myparagraph{MSVC \NonOptimizing}

Nous obtenons (MSVC 2010):

\lstinputlisting[caption=MSVC 2010,style=customasmx86]{patterns/08_switch/2_lot/lot_msvc_FR.asm}

\myindex{jumptable}

Ce que nous voyons ici est un ensemble d'appels à \printf avec des arguments variés.
Ils ont tous, non seulement des adresses dans la mémoire du processus, mais aussi
des labels symboliques internes assignés par le compilateur.
Tous ces labels ont aussi mentionnés dans la table interne \TT{\$LN11@f}.

Au début de la fonctions, si $a$ est supérieur à 4, l'exécution est passée au
labal \TT{\$LN1@f}, où \printf est appelé avec l'argument \TT{'something unknown'}.

Mais si la valeur de $a$ est inférieure ou égale à 4, elle est alors multipliée
par 4 et ajoutée à l'adresse de la table \TT{\$LN11@f}. C'est ainsi qu'une adresse
à l'intérieur de la table est construite, pointant exactement sur l'élément dont
nous avons besoin. Par exemple, supposons que $a$ soit égale à 2. $2*4 = 8$ (tous
les éléments de la table sont adressés dans un processus 32-bit, c'est pourquoi les
éléments ont une taille de 4 octets).
L'adresse de la table \TT{\$LN11@f} + 8 est celle de l'élément de la table où
le label \TT{\$LN4@f} est stocké.
\JMP prend l'adresse de \TT{\$LN4@f} dans la table et y saute.

Cette table est quelquefois appelée \emph{jumptable} (table de saut) ou \emph{branch table}
(table de branchement)\footnote{L'ensemble de la méthode était appelé \emph{computed
GOTO} (GOTO calculés) dans les premières versions de ForTran:
\href{http://en.wikipedia.org/wiki/Branch_table}{Wikipédia}.
Pas très pertinent de nos jours, mais quel terme!}.

Le \printf correspondant est appelé avec l'argument \TT{'two'}.\\
Littéralement, l'instruction \TT{jmp DWORD PTR \$LN11@f[ecx*4]} signifie
\emph{sauter au DWORD qui est stocké à l'adresse} \TT{\$LN11@f + ecx * 4}.

\TT{npad} (\myref{sec:npad}) est une macro du langage d'assemblage qui aligne le
label suivant de telle sorte qu'il soit stocké à une adresse alignée sur une limite
de 4 octets (ou 16 octets).
C'est très adapté pour le processeur puisqu'il est capable d'aller chercher des
valeurs 32-bit dans la mémoire à travers le bus mémoire, la mémoire cache, etc.,
de façons beaucoup plus efficace si c'est aligné.

\clearpage
\subsubsection{MSVC: x86 + \olly}

Essayons de hacker notre programme dans \olly, pour le forcer à penser que \scanf
fonctionne toujours sans erreur.
Lorsque l'adresse d'une variable locale est passée à \scanf, la variable contient
initiallement toujours des restes de données aléatoires, dans ce cas \TT{0x6E494714}:

\begin{figure}[H]
\centering
\myincludegraphics{patterns/04_scanf/3_checking_retval/olly_1.png}
\caption{\olly: passer l'adresse de la variable à \scanf}
\label{fig:scanf_ex3_olly_1}
\end{figure}

\clearpage
Lorsque \scanf s'exécute dans la console, entrons quelque chose qui n'est pas du
tout un nombre, comme \q{asdasd}.
\scanf termine avec 0 dans \EAX, ce qui indique qu'une erreur s'est produite.

Nous pouvons vérifier la variable locale dans le pile et noter qu'elle n'a pas changé.
En effet, qu'aurait écrit \scanf ici?
Elle n'a simplement rien fait à part renvoyer zéro.

Essayons de \q{hacker} notre programme.
Clique-droit sur \EAX,
parmi les options il y a \q{Set to 1} (mettre à 1).
C'est ce dont nous avons besoin.

Nous avons maintenant 1 dans \EAX, donc la vérification suivante va s'exécuter comme
souhaiter et \printf va afficher la valeur de la variable dans la pile.

Lorsque nous lançons le programme (F9) nous pouvons voir ceci dans la fenêtre
de la console:

\lstinputlisting[caption=fenêtre console]{patterns/04_scanf/3_checking_retval/console.txt}

En effet, 1850296084 est la représentation en décimal du nombre dans la pile (\TT{0x6E494714})!


\myparagraph{GCC \NonOptimizing}
\label{switch_lot_GCC}

Voyons ce que GCC 4.4.1 génère:

\lstinputlisting[caption=GCC 4.4.1,style=customasmx86]{patterns/08_switch/2_lot/lot_gcc.asm}

\myindex{x86!\Registers!JMP}

C'est presque la même chose, avec une petite nuance: l'argument \TT{arg\_0} est multiplié
par 4 en décalant de 2 bits vers la gauche (c'est presque comme multiplier par 4)~(\myref{SHR}).
Ensuite l'adresse du label est prise depuis le tableau \TT{off\_804855C}, stockée
dans \EAX, et ensuite \TT{JMP EAX} effectue le saut réel.

