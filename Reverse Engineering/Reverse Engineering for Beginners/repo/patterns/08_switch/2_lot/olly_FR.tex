\clearpage
\mysubparagraph{\olly}
\myindex{\olly}

Essayons cet exemple dans \olly.
La valeur d'entrée de la fonction (2) est chargée dans \EAX:

\begin{figure}[H]
\centering
\myincludegraphics{patterns/08_switch/2_lot/olly1.png}
\caption{\olly: la valeur d'entrée de la fonction est chargée dans \EAX}
\label{fig:switch_lot_olly1}
\end{figure}

\clearpage
La valeur entrée est testée, est-elle plus grande que 4?
Si non, le saut par \q{défaut} n'est pas pris:
\begin{figure}[H]
\centering
\myincludegraphics{patterns/08_switch/2_lot/olly2.png}
\caption{\olly: 2 n'est pas plus grand que 4: le saut n'est pas pris}
\label{fig:switch_lot_olly2}
\end{figure}

\clearpage
Ici, nous voyons une table des sauts:

\begin{figure}[H]
\centering
\myincludegraphics{patterns/08_switch/2_lot/olly3.png}
\caption{\olly: calcul de l'adresse de destination en utilisant la table des sauts}
\label{fig:switch_lot_olly3}
\end{figure}

Ici, nous avons cliqué \q{Follow in Dump} $\rightarrow$ \q{Address constant}, donc
nous voyons maintenant la \emph{jumptable} dans la fenêtre des données.
Il y a 5 valeurs 32-bit\footnote{Elles sont soulignées par \olly car ce sont aussi
des FIXUPs: \myref{subsec:relocs}, nous y reviendrons plus tard}.
\ECX contient maintenant 2, donc le troisième élément (peut être indexé par 2\footnote{À
propos des index de tableaux, lire aussi: \myref{arrays_at_one}}) de la table va être
utilisé.
Il est également possible de cliquer sur \q{Follow in Dump} $\rightarrow$ \q{Memory
address} et \olly va montrer l'élément adressé par l'instruction \JMP.
Il s'agit de \TT{0x010B103A}.

\clearpage
Après le saut, nous sommes en \TT{0x010B103A}: le code qui affiche \q{two} va être
exécuté:

\begin{figure}[H]
\centering
\myincludegraphics{patterns/08_switch/2_lot/olly4.png}
\caption{\olly: maintenant nous sommes au \emph{cas:} label}
\label{fig:switch_lot_olly4}
\end{figure}
