\subsubsection{x86}

\myparagraph{\NonOptimizing MSVC}

Con MSVC 2010 otteniamo:

\lstinputlisting[caption=MSVC 2010,style=customasmx86]{patterns/08_switch/2_lot/lot_msvc_IT.asm}

\myindex{jumptable}

Vediamo una serie di chiamate a \printf con vari argomenti. Hanno tutte non solo indirizzi nella memoria del processo, ma anche etichette
simboliche assegnate dal compilatore. Queste label sono anche menzionate nalle tabella interna \TT{\$LN11@f}.

All'inizio della funzione, se $a$ è maggiore di 4, il controllo del flusso è passato alla label 
\TT{\$LN1@f}, dove viene chiamata \printf con argomento \TT{'something unknown'}.

Se invece il valore di $a$ è minore o uguale a 4, viene moltiplicato per 4 e sommato all'indirizzo della tabella \TT{\$LN11@f}. 
In questo modo vengono costruiti gli indirizzi della tabella, facendo puntare esattamente all'elemento giusto per ogni caso,

Poniamo ad esempio che $a$ sia uguale a 2. $2*4 = 8$ (tutti gli elementi della tabella sono indirizzi in un processo a 32-bit, perciò tutti gli elementi sono larghi 4 byte).
L'indirizzo della tabella \TT{\$LN11@f} + 8 corrisponde all'elemento della tabella in cui è memorizzata la label \TT{\$LN4@f}.
L'istruzione \JMP recupera quindi l'indirizzo di \TT{\$LN4@f} dalla tabella e salta.

Questa tabella è talvolta detta \emph{jumptable} o \emph{branch table}\footnote{L'intero metodo una volta era noto come 
\emph{computed GOTO} nelle prime versioni di Fortran:
\href{http://en.wikipedia.org/wiki/Branch_table}{wikipedia}.
Non è molto rilevante oggigiorno, ma che termine!}.

Successivamente la corrispondente \printf viene chiamata con argomento \TT{'two'}.\\
Letteralmente, l'istruzione \TT{jmp DWORD PTR \$LN11@f[ecx*4]} corrisponde a 
\emph{salta alla DWORD che è memorizzata all'indirizzo} \TT{\$LN11@f + ecx * 4}.

\TT{npad} (\myref{sec:npad}) è una macro del linguaggio assembly che allinea la prossima label in modo tale che sia memorizzata 
ad un indirizzo allineato a 4 byte (or a 16 byte).

Ciò è molto utile in termini di performance poiché così il processore è in grado di recuperare valori a 32-bit dalla memoria attraverso il memory bus, 
la cache, etc., in maniera più efficiente se è allineata.

\clearpage
\myparagraph{\Optimizing MSVC + \olly}
\myindex{\olly}

Testiamo questo esempio (ottimizzato) in \olly.  Questa è la prima iterazione:

\begin{figure}[H]
\centering
\myincludegraphics{patterns/10_strings/1_strlen/olly1.png}
\caption{\olly: inizio prima iterazione}
\label{fig:strlen_olly_1}
\end{figure}

Notiamo che \olly ha trovato un ciclo e per convenienza, ha \emph{avvolto} le sue istruzioni dentro le parentesi.
Cliccando con il tasto destro su \EAX e scegliendo 
\q{Follow in Dump}, la finestra della memoria scorrerà fino al punto giusto.
Possiamo vedere la stringa \q{hello!} in memoria.
C'è almeno
uno zero byte dopo la stringa e poi spazzatura casuale.

Se \olly vede un registro contenente un indirizzo valido, che punta ad una stringa,  
lo mostra come stringa.

\clearpage
Premiamo F8 (\stepover) un paio di volte, per arrivare all' inizio del corpo del ciclo:

\begin{figure}[H]
\centering
\myincludegraphics{patterns/10_strings/1_strlen/olly2.png}
\caption{\olly: inizio seconda iterazione}
\label{fig:strlen_olly_2}
\end{figure}

Notiamo che \EAX contiene l'indirizzo del secondo carattere nella stringa.

\clearpage

Dobbiamo premere F8 un numero di volte sufficente per uscire dal ciclo:

\begin{figure}[H]
\centering
\myincludegraphics{patterns/10_strings/1_strlen/olly3.png}
\caption{\olly: differenza di puntatori da calcolare}
\label{fig:strlen_olly_3}
\end{figure}

Notiamo che ora \EAX contiene l'indirizzo dello zero byte che si trova subito dopo la stringa più 1 (perché INC EAX è stato eseguito indipendentemente dal fatto che siamo usciti o meno dal ciclo).
Nel frattempo, \EDX non è cambiato,
quindi sta ancora puntando all' inizio della stringa.

La differenza tra questi due indirizzi verrà calcolata ora.

\clearpage
L' istruzione \SUB è stata appena eseguita:

\begin{figure}[H]
\centering
\myincludegraphics{patterns/10_strings/1_strlen/olly4.png}
\caption{\olly: \EAX  sta venendo decrementato}
\label{fig:strlen_olly_4}
\end{figure}

La differenza tra i puntatori, in questo momento, si trova nel registro \EAX---7.
In realtà, la lunghezza della stringa \q{hello!} è 6, 
ma con lo zero byte incluso---7.
Ma \TT{strlen()} deve ritornare il numero di caratteri nella stringa, diversi da zero.
Quindi viene eseguito un decremento, dopodichè la funziona ritorna.


\myparagraph{\NonOptimizing GCC}
\label{switch_lot_GCC}

Vediamo il codice generato da GCC 4.4.1:

\lstinputlisting[caption=GCC 4.4.1,style=customasmx86]{patterns/08_switch/2_lot/lot_gcc.asm}

\myindex{x86!\Registers!JMP}

E' pressoché identico, con una leggera variazione: l'argomento \TT{arg\_0} è moltiplicato per 4
effettuando uno shift a sinistra di 2 bit (quasi identico alla moltiplicazione per 4)~(\myref{SHR}).
Successivamente l'indirizzo della label è preso dall'array \TT{off\_804855C}, memorizzato in 
\EAX, e infine \TT{JMP EAX} effettua il salto.

