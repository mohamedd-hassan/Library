\subsubsection{x86}

\myparagraph{\NonOptimizing MSVC}

We get (MSVC 2010):

\lstinputlisting[caption=MSVC 2010,style=customasmx86]{patterns/08_switch/2_lot/lot_msvc_EN.asm}

\myindex{jumptable}

What we see here is a set of \printf calls with various arguments. 
All they have not only addresses in the memory of the process, but also internal symbolic labels assigned 
by the compiler. 
All these labels are also mentioned in the \TT{\$LN11@f} internal table.

At the function start, if $a$ is greater than 4, control flow is passed to label 
\TT{\$LN1@f}, where \printf with argument \TT{'something unknown'} is called.

But if the value of $a$ is less or equals to 4, then it gets multiplied by 4 and added with the \TT{\$LN11@f} 
table address. That is how an address inside the table is constructed, pointing exactly to the 
element we need. For example, let's say $a$ is equal to 2. $2*4 = 8$ (all table elements 
are addresses in a 32-bit process and that is why all elements are 4 bytes wide). 
The address of the \TT{\$LN11@f} table + 8 is the table element where the \TT{\$LN4@f} label is stored.
\JMP fetches the \TT{\$LN4@f} address from the table and jumps to it.

This table is sometimes called \emph{jumptable} or \emph{branch table}\footnote{The whole method was once called 
\emph{computed GOTO} in early versions of Fortran:
\href{http://en.wikipedia.org/wiki/Branch_table}{wikipedia}.
Not quite relevant these days, but what a term!}.

Then the corresponding \printf is called with argument \TT{'two'}.\\
Literally, the \TT{jmp DWORD PTR \$LN11@f[ecx*4]} instruction implies
\emph{jump to the DWORD that is stored at address} \TT{\$LN11@f + ecx * 4}.

\TT{npad} (\myref{sec:npad}) is an assembly language macro that align the next label so that it will be stored at an address aligned on a 4 bytes
(or 16 bytes) boundary.
This is very suitable for the processor since it is able to fetch 32-bit values from memory through the memory bus,
cache memory, etc., in a more effective way if it is aligned.

\clearpage
\subsubsection{MSVC + \olly}
\myindex{\olly}

Let's load our example into \olly and set a breakpoint on \comp.
We can see how the values are compared at the first \comp call:

\begin{figure}[H]
\centering
\myincludegraphics{patterns/18_pointers_to_functions/olly1.png}
\caption{\olly: first call of \comp}
\label{fig:qsort_olly1}
\end{figure}

\olly shows the compared values in the window under the code window, for convenience.
We can also see that the \ac{SP} points to \ac{RA}, where the \qsort function is (located in \TT{MSVCR100.DLL}).

\clearpage
By tracing (F8) until the \TT{RETN} instruction and pressing F8 one more time, we return to the \qsort function:

\begin{figure}[H]
\centering
\myincludegraphics{patterns/18_pointers_to_functions/olly2.png}
\caption{\olly: the code in \qsort right after \comp call}
\label{fig:qsort_olly2}
\end{figure}

That has been a call to the comparison function.

\clearpage
Here is also a screenshot of the moment of the second call of \comp{}---now values that have to be compared are different:

\begin{figure}[H]
\centering
\myincludegraphics{patterns/18_pointers_to_functions/olly3.png}
\caption{\olly: second call of \comp}
\label{fig:qsort_olly3}
\end{figure}


\myparagraph{\NonOptimizing GCC}
\label{switch_lot_GCC}

Let's see what GCC 4.4.1 generates:

\lstinputlisting[caption=GCC 4.4.1,style=customasmx86]{patterns/08_switch/2_lot/lot_gcc.asm}

\myindex{x86!\Registers!JMP}

It is almost the same, with a little nuance: argument \TT{arg\_0} is multiplied by 4 by
shifting it to left by 2 bits (it is almost the same as multiplication by 4)~(\myref{SHR}).
Then the address of the label is taken from the \TT{off\_804855C} array, stored in 
\EAX, and then \TT{JMP EAX} does the actual jump.

