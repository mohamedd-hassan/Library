\subsection{フォールスルー}

\TT{switch()}演算子の別のポピュラーな使い方は\q{フォールスルー}です。
単純なサンプルがあります。\footnote{\url{https://github.com/azonalon/prgraas/blob/master/prog1lib/lecture_examples/is_whitespace.c}からコピーペースト}:

\lstinputlisting[numbers=left,style=customc]{patterns/08_switch/4_fallthrough/fallthrough1.c}

やや難しいものをLinuxカーネル\footnote{\url{https://github.com/torvalds/linux/blob/master/drivers/media/dvb-frontends/lgdt3306a.c}からコピーペースト}:

\lstinputlisting[numbers=left,style=customc]{patterns/08_switch/4_fallthrough/fallthrough2.c}

\lstinputlisting[caption=Optimizing GCC 5.4.0 x86,numbers=left,style=customasmx86]{patterns/08_switch/4_fallthrough/fallthrough2.s}

関数の入力に3250という数字がある場合、\TT{.L5}ラベルを得ることができます。
しかし、我々は反対側からこのラベルに行くことができます:
\printf 呼び出しと\TT{.L5}ラベルの間にはジャンプがないことがわかります。

\emph{switch()}文がバグの原因となることが理解できます。
\emph{break}を1つ忘れるとはあなたの\emph{switch()}文を\emph{フォールスルー}に変換し、1つのブロックの代わりにいくつかのブロックが実行されます。
