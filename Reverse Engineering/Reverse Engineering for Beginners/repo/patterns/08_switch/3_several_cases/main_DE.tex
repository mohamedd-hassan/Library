\subsection{Wenn es mehrere \emph{case} Ausdrücke in einem Block gibt}
Hier ist eine weit verbreitete Konstruktion: mehrere \emph{case} Ausdrücke für einen einzigen Block:

\lstinputlisting[style=customc]{patterns/08_switch/3_several_cases/several_cases.c}
Es ist zu verschwenderisch einen Block für jeden möglichen Fall zu erzeugen, sodass normalerweise ein Block und eine Art
Dispatcher erzeugt werden.

\subsubsection{MSVC}

\lstinputlisting[caption=\Optimizing MSVC 2010,numbers=left,style=customasmx86]{patterns/08_switch/3_several_cases/several_cases_MSVC_2010_Ox_DE.asm}
Wir sehen hier zwei Tabellen: die erste Tabelle (\TT{\$LN10@f}) ist eine Indextabelle und die zweite (\TT{\$LN11@f}) ist
ein Array von Pointern auf Blöcke.

Zuerst wird der Eingabewert als Index in der Indextabelle verwendet (Zeile 13).

Hier ist eine kurze Legende für die Werte in der Tabelle:
0 ist der erste \emph{case} Block (für die Werte 1, 2, 7, 10),
1 ist der zweite (für die Werte 3, 4, 5),
2 ist der dritte (für die Werte 8, 9, 21),
3 ist der vierte (für die Werte 22),
4 ist der Defaultblock.
Hier erhalten wir einen Index für die zweite Tabelle aus Pointern und springen zu einem solchen (Zeile 14).

Bemerkenswert ist auch, dass es keinen Case für den Eingabewert 0 gibt. 

Aus diesem Grund haben wir den \DEC Befehl in Zeile 10 und die Tabelle beginnt bei $a=1$, da kein Tabellenelement für
$a=0$ angelegt werden muss.

Dies ist ein weitverbreitetes Muster.

Warum ist dieses Vorgehen so ökonomisch?
Warum ist es nicht möglich wie vorher in (\myref{switch_lot_GCC}) vorzugehen mit nur einer Tabelle aus Blockpointern?
Der Grund hierfür ist, dass die Elemente in der Indextabelle nur 8 Bit groß sind und alles deshalb deutlich kompakter
ist.

\subsubsection{GCC}

GCC erledigt seinen Job unter Verwendung von nur einer Pointertabelle wie bereits hier besprochen
(\myref{switch_lot_GCC}).

\subsubsection{ARM64: \Optimizing GCC 4.9.1}
Es wird kein Code ausgeführt, wenn der Eingabewert 0 ist, weshalb GCC versucht, die Jumptable kleiner zu machen und erst
beim Eingabewert 1 zu beginnen.

GCC 4.9.1 für ARM64 verwendeten einen noch ausgefeilteren Trick.
Es ist möglich alle Offsets als 8 Bit Werte zu kodieren.

Erinnern wir uns, dass alle ARM64 Befehle eine Größe von 4 Bytes haben.

GCC verwendet den Umstand, dass alle Offsets in unserem Minimalbeispiel in der Nähe voneinander liegen. Daher kann die
Jumptable aus einzelnen Bytes bestehen.

\lstinputlisting[caption=\Optimizing GCC 4.9.1 ARM64,style=customasmARM]{patterns/08_switch/3_several_cases/ARM64_GCC491_O3_DE.s}
Kompilieren wir dieses Beispiel in eine Object-Datei und öffnen es ist \IDA. Hier ist die Jumptable:

\lstinputlisting[caption=jumptable in IDA,style=customasmARM]{patterns/08_switch/3_several_cases/ARM64_GCC491_O3_IDA.lst}

Im Fall von 1, 9 wird also mit 4 multipliziert und zur Adresse des Labels \TT{Lrtx4} addiert.

Im Fall von 22, wird 0 mit 4 multipliziert; mit dem Ergebnis 0. 

Direkt hinter dem \TT{Lrtx4} Label befindet sich das \TT{L7} Label, an dem sich der Code befindet, der \q{22} ausgibt.

Es gibt keine Jumptable im Codesegment; sie wird in einem getrennten .rodata Segment angelegt (es besteht keine
Notwendigkeit, die Tabelle im Codesegment anzulegen).

Hier befinden sich auch negative Bytes (0xF7), die für das Zurückspringen im Code verwendet werden, um den \q{Default}
String am Label \TT{.L2} auszugeben.

