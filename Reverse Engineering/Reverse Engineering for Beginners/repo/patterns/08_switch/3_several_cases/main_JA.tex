\subsection{あるブロックに複数の\emph{case}文があるとき}

よく用いられる構成があります:単一ブロックにいくつか\emph{case}ステートメントがあります:

\lstinputlisting[style=customc]{patterns/08_switch/3_several_cases/several_cases.c}

可能性のあるケースごとにブロックを生成するのは無駄です。
通常は、各ブロックに何らかのディスパッチャーを加えたものを生成します。

\subsubsection{MSVC}

\lstinputlisting[caption=\Optimizing MSVC 2010,numbers=left,style=customasmx86]{patterns/08_switch/3_several_cases/several_cases_MSVC_2010_Ox_JA.asm}

最初のテーブル(\TT{\$LN10@f})はインデックステーブルで、2番目のテーブル(\TT{\$LN11@f})はブロックへのポインタの配列です。

まず、入力値がインデックステーブルのインデックスとして使用されます(13行目)。

表の値の短い凡例は次のとおりです。
0は最初の\emph{case}ブロックです(値1,2,7,10の場合)。
1は2番目の値(値3,4,5)です。
2は3番目の値(値8,9,21)です。
3は4番目の値(値22)です。
4はデフォルトブロック用です。

コードポインタの2番目のテーブルのインデックスを取得し、それにジャンプします(14行目)。

注目すべき点は、入力値0の場合がないことです。

そのため、10行目の \DEC 命令が表示され、$a=1$にテーブル要素を割り当てる必要がないため、
$a=1$でテーブルが開始されます。

これはよく用いられるパターンです。

それでなぜこれが経済的なのでしょうか?
以前はブロックポインタで構成された1つのテーブルだけで、
それを作ることができないのはなぜですか?(\myref{switch_lot_GCC})
その理由は、インデックステーブルの要素が8ビットで、よりコンパクトなためです。

\subsubsection{GCC}

GCCはすでに述べた方法で(\myref{switch_lot_GCC})、ポインタのテーブルを1つだけ使用して仕事をしています。

\subsubsection{ARM64: \Optimizing GCC 4.9.1}

入力値が0の場合にトリガされるコードはないので、GCCはジャンプテーブルをよりコンパクトにしようとし、
入力値として1から開始します。

ARM64用のGCC 4.9.1は、より巧妙なトリックを使用します。
すべてのオフセットを8ビットのバイトとしてエンコードできます。

すべてのARM64命令のサイズが4バイトであることを思い出してみましょう。

GCCは、私の小さな例のすべてのオフセットがお互いに非常に近いという事実を利用しています。
ジャンプテーブルは1バイトで構成されています。

\lstinputlisting[caption=\Optimizing GCC 4.9.1 ARM64,style=customasmARM]{patterns/08_switch/3_several_cases/ARM64_GCC491_O3_JA.s}

この例をオブジェクトファイルにコンパイルし、 \IDA で開きましょう。 ここにジャンプテーブルがあります:

\lstinputlisting[caption=jumptable in IDA,style=customasmARM]{patterns/08_switch/3_several_cases/ARM64_GCC491_O3_IDA.lst}

したがって、1の場合、9は4で乗算され、\TT{Lrtx4}ラベルのアドレスに追加されます。

22の場合、0には4が掛けられ、結果は0になります。

\TT{Lrtx4}ラベルの直後に\TT{L7}ラベルがあります。このラベルでは、\q{22}を出力するコードを見つけることができます。

コードセグメントにはジャンプテーブルはありません。別の.rodataセクションに割り当てられています
(コードセクションに配置する特別な必要はありません)。

負のバイト(0xF7)もあり、\q{default}文字列(\TT{.L2})を出力するコードにジャンプするために使用されます。
