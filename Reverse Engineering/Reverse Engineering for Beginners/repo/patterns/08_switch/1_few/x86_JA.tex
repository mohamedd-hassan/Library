\subsubsection{x86}

\myparagraph{\NonOptimizing MSVC}

結果 (MSVC 2010):

\lstinputlisting[caption=MSVC 2010,style=customasmx86]{patterns/08_switch/1_few/few_msvc.asm}

実際、switch()でいくつかのcaseを持つ私たちの関数は、この構造に似ています。

\lstinputlisting[label=switch_few_ifelse,style=customc]{patterns/08_switch/1_few/few_analogue.c}

\myindex{\CLanguageElements!switch}
\myindex{\CLanguageElements!if}

いくつかのcaseでswitch()を使用する場合、ソースコード内の実際のswitch()か、
単にif文の組であるかどうかを確認することは不可能です。
\myindex{\SyntacticSugar}

これはswitch()が多段にネストされたif文との糖衣構文のようなものであることを意味します。

コンパイラが入力変数 $a$ を一時的なローカル変数\TT{tv64}に移動することを除いて、
生成されたコードには特に新しいことはありません。
\footnote{スタック内のローカル変数には接頭辞\TT{tv}が付きます。MSVCが内部変数として使用するために命名しています。}

これをGCC 4.4.1でコンパイルすると、最大限の最適化(\Othree option)を有効にしても
ほぼ同じ結果になります。

\myparagraph{\Optimizing MSVC}

% TODO separate various kinds of \TT
% idea: enclose command lines in a specific environment, like \cmdline{} 
% assembly instructions in \asm{} (now both \TT and \q{} are used),
% variables in,  like \var{}
% messages (string constants) in something else, like \strconst
% to separate them all. Now they all use \TT, which is not best
% \INS{} for all instructions including operands? --DY
では、MSVC(\Ox)の最適化を有効にしましょう:\TT{cl 1.c /Fa1.asm /Ox}

\label{JMP_instead_of_RET}
\lstinputlisting[caption=MSVC,style=customasmx86]{patterns/08_switch/1_few/few_msvc_Ox.asm}

ここで、汚いハックを見ることができます。

\myindex{x86!\Instructions!JZ}
\myindex{x86!\Instructions!JE}
\myindex{x86!\Instructions!SUB}

最初に、 $a$ の値を \EAX に置き、0を引きます。 EAXの値が0かどうかを確認するために行われますが、
そうであれば、 \ZF フラグがセットされます(例えば、0からの減算は0)
最初の条件ジャンプ \JE (\emph{Jump if Equal} またはあ同義語 \JZ~---\emph{Jump if Zero})は実行され、
制御フローは\TT{\$LN4@f}ラベルに渡されます。ここでは、 \TT{'zero'}メッセージが出力されます。
最初のジャンプが実行されない場合は、入力値から1が減算され、結果が0の場合、対応するジャンプが実行されます。

また、ジャンプが全く実行されない場合、制御フローは文字列引数\TT{'something unknown'}を \printf に渡します。

\label{jump_to_last_printf}
\myindex{\Stack}

次に、文字列ポインタが $a$ 変数に置かれ、 \printf が \CALL ではなく \JMP を介して呼び出されます。 
簡単に説明するとこうなります:
\gls{caller} は値をスタックにプッシュし、 \CALL 経由で関数を呼び出します。
\CALL 自体は戻りアドレス(\ac{RA})をスタックにプッシュし、関数アドレスへの無条件ジャンプを行います。
スタックポインタを移動させる命令が含まれていないため、任意の実行時点での関数は、次のスタックレイアウトを持ちます。

\begin{itemize}
\item\ESP---points to \ac{RA}
\item\TT{ESP+4}---points to the $a$ variable 
\end{itemize}

反対に、\printf をここで呼び出さなければならないときは、文字列を指し示す必要がある最初の\printf 引数を除いて、
全く同じスタックレイアウトが必要です。それが私たちのコードがすることです。

ファンクションの最初の引数を文字列のアドレスに置き換え、
関数 \ttf を直接呼び出しずに直接 \printf を呼び出すかのように、 \printf にジャンプします。
\printf は文字列を \gls{stdout} に出力し、 \RET 命令を実行します。スタックから\ac{RA}を取り出し、
制御フローは \ttf ではなく \ttf 関数の終りをバイパスして、 \ttf の \gls{caller} です。

\myindex{\CStandardLibrary!longjmp()}
\newcommand{\URLSJ}{\href{http://en.wikipedia.org/wiki/Setjmp.h}{wikipedia}}

% TODO \myref{}
\printf はすべての場合に  \ttf 関数の終わりで右に呼ばれるので、これはすべて可能です。
ある意味では、\TT{longjmp()}\footnote{\URLSJ}関数に似ています。
そしてもちろん、それはスピードのためにすべて行われます。

ARMコンパイラと同様のケースは、\q{\PrintfSeveralArgumentsSectionName}セクションに記載されています。
こちら:~(\myref{ARM_B_to_printf})

\clearpage
\myparagraph{x86 + MSVC + \olly}
\myindex{\olly}
\myindex{x86!\Registers!\Flags}

\olly でこの例を実行すると、フラグがどのように設定されているかを見ることができます。 
符号なしの数値で動作する\TT{f\_unsigned()}から始めましょう。

\CMP はここで3回実行されますが、同じ引数についてはフラグは毎回同じです。

最初の比較の結果は、

\begin{figure}[H]
\centering
\myincludegraphics{patterns/07_jcc/simple/olly_unsigned1.png}
\caption{\olly: \TT{f\_unsigned()}: 最初の条件付きジャンプ}
\label{fig:jcc_olly_unsigned_1}
\end{figure}

従って、フラグは、C=1、P=1、A=1、Z=0、S=1、T=0、D=0、O=0です。

これらは \olly では1文字の略号で命名されています。

\olly は、(\JBE)ジャンプがトリガーされることを示唆しています。 
実際に、インテルのマニュアル(\myref{x86_manuals})を調べると、
CF=1またはZF=1の場合、JBEが起動することがわかります。 
条件はここに当てはまるので、ジャンプが開始されます。

\clearpage
次の条件付きジャンプは、

\begin{figure}[H]
\centering
\myincludegraphics{patterns/07_jcc/simple/olly_unsigned2.png}
\caption{\olly: \TT{f\_unsigned()}: 2番目の条件付きジャンプ}
\label{fig:jcc_olly_unsigned_2}
\end{figure}

\olly は、 \JNZ がトリガーされることを示唆しています。 
実際、ZF=0(ゼロフラグ)の場合、JNZが起動します。

\clearpage
3番目の条件付きジャンプは、 \JNB です。

\begin{figure}[H]
\centering
\myincludegraphics{patterns/07_jcc/simple/olly_unsigned3.png}
\caption{\olly: \TT{f\_unsigned()}: 3番目の条件付きジャンプ}
\label{fig:jcc_olly_unsigned_3}
\end{figure}

インテルのマニュアル(\myref{x86_manuals})では、CF=0(キャリーフラグ)の場合に \JNB が起動することがわかります。 
今回は当てはまらないので、3番目の \printf が実行されます。

\clearpage
次に、\olly で、符号付きの値で動作する\TT{f\_signed()}関数を見てみましょう。 
フラグは、C=1、P=1、A=1、Z=0、S=1、T=0、D=0、O=0と同様に設定されます。 
最初の条件付きジャンプ \JLE が起動されます。

インテルマニュアル(166ページの7.1.4)では、ZF = 1またはSFxOFの場合にこの命令がトリガされることがわかりました。 SFxOF私たちの場合は、ジャンプがトリガするように。

\begin{figure}[H]
\centering
\myincludegraphics{patterns/07_jcc/simple/olly_signed1.png}
\caption{\olly: \TT{f\_signed()}: 最初の条件付きジャンプ}
\label{fig:jcc_olly_signed_1}
\end{figure}

インテルマニュアル(\myref{x86_manuals})では、ZF=1または SF$\neq$OF の場合にこの命令が起動されることがわかりました。 
私たちの場合では SF$\neq$OF が、ジャンプが起動されます。

\clearpage
2番目の \JNZ 条件付きジャンプはZF=0の場合(ゼロ・フラグ)に起動します。

\begin{figure}[H]
\centering
\myincludegraphics{patterns/07_jcc/simple/olly_signed2.png}
\caption{\olly: \TT{f\_signed()}: 2番目の条件付きジャンプ}
\label{fig:jcc_olly_signed_2}
\end{figure}

\clearpage
第3の条件付きジャンプ \JGE は、SF=OFの場合にのみ実行されるため、起動しません。今回は、当てはまりません。

\begin{figure}[H]
\centering
\myincludegraphics{patterns/07_jcc/simple/olly_signed3.png}
\caption{\olly: \TT{f\_signed()}: 3番目の条件付きジャンプ}
\label{fig:jcc_olly_signed_3}
\end{figure}


