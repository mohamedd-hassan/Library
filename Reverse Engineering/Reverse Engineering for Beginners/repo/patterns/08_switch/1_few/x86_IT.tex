\subsubsection{x86}

\myparagraph{\NonOptimizing MSVC}

Risultato (MSVC 2010):

\lstinputlisting[caption=MSVC 2010,style=customasmx86]{patterns/08_switch/1_few/few_msvc.asm}

La nostra funzione con pochi casi nello switch() è praticamente analogo a questa costruzione:

\lstinputlisting[label=switch_few_ifelse,style=customc]{patterns/08_switch/1_few/few_analogue.c}

\myindex{\CLanguageElements!switch}
\myindex{\CLanguageElements!if}

Nel caso di switch() con un piccolo numero di casi, è impossibile essere sicuri che nel sorgente originale ci fosse
veramente la funzione switch() o soltanto una serie di costrutti if().


\myindex{\SyntacticSugar}

Ciò vuol dire che switch() si comporta come "zucchero sintattico", equivalente ad un alto numero di if() annidati.

Dal nostro punto di vista non c'è niente di particolarmente nuovo nel codice generato,
con l'eccezione dello spostamento (da parte del compilatore) della variabile in input $a$ in una variabile locale temporanea \TT{tv64}
\footnote{Le variabili locali nello stack hanno il prefisso \TT{tv}--- MSVC nomina così le varibili locali per i suoi scopi}.

Se compiliamo questo codice con 4.4.1 otteniamo pressoché lo stesso risultato, anche con il maggior livello di ottimizzazione (\Othree option).

\myparagraph{\Optimizing MSVC}

% TODO separate various kinds of \TT
% idea: enclose command lines in a specific environment, like \cmdline{} 
% assembly instructions in \asm{} (now both \TT and \q{} are used),
% variables in,  like \var{}
% messages (string constants) in something else, like \strconst
% to separate them all. Now they all use \TT, which is not best
% \INS{} for all instructions including operands? --DY
Ora attiviamo l'ottimizzaione in MSVC (\Ox): \TT{cl 1.c /Fa1.asm /Ox}

\label{JMP_instead_of_RET}
\lstinputlisting[caption=MSVC,style=customasmx86]{patterns/08_switch/1_few/few_msvc_Ox.asm}

Qui possiamo vedere un po' di trucchetti.

\myindex{x86!\Instructions!JZ}
\myindex{x86!\Instructions!JE}
\myindex{x86!\Instructions!SUB}

Primo: il valore di $a$ è messo in \EAX, e gli viene sottratto 0. Sembra assurdo, ma è fatto per controllare se il valore 
in \EAX è 0. Se si, il flag \ZF viene settato (i.e. sottrarre 0 a 0 da 0)
e il primo jump condizionale \JE (\emph{Jump if Equal} o il sinonimo \JZ~---\emph{Jump if Zero}) è innescato e il controllo del flusso 
passato alla label \TT{\$LN4@f}, dove il messaggio \TT{'zero'} viene stampato. 
Se il primo jump non viene innescato, 1 viene sottratto dal valore in input e se ad un certo punto il risultato è 0, il jump corrispondente
viene innescato.

Se nessun jump viene innescato, il controllo del flusso passa a \printf con l'argomento \\
\TT{'something unknown'}.

\label{jump_to_last_printf}
\myindex{\Stack}

Secondo: notiamo qualcosa di inusuale per noi: un puntatore a stringa viene messo nella variabile $a$, e 
successivamente viene chiamata \printf non tramite \CALL, ma via \JMP. C'è una spiegazione semplice dietro ciò: 
il \gls{caller} mette un valore sullo stack e chiama la nostra funzione tramite \CALL. 
La stessa \CALL fa il push del return address (\ac{RA}) sullo stack e fa un salto non condizionale all'indirizzo della nostra funzione. 
La nostra funzione, in qualunque punto della sua esecuzione (poiché non contiene istruzioni che spostano lo stack pointer)
ha il seguente layout dello stack:

\begin{itemize}
\item\ESP---punta a \ac{RA}
\item\TT{ESP+4}---punta alla variabile $a$ 
\end{itemize}

Dall'altro lato, quando dobbiamo chiamare \printf qui abbiamo bisogno esattamente dello stesso layout dello stack,
eccetto per il primo argomento di \printf, che deve puntare alla stringa. 
E questo è ciò che fa il codice.

Sostituisce il primo argomento della funzione con l'indirizzo della stringa, e salta a \printf, come se non avessimo chiamato
la nostra funzione \ttf, ma direttamente \printf.
\printf stampa una stringa su \gls{stdout} e successivamente esegue l'istruzione \RET , che fa il POP del 
\ac{RA} dallo stack, e il controllo del flusso viene restituito non a \ttf ma al \gls{caller} di \ttf, 
bypassando la fine della funzione \ttf.

\myindex{\CStandardLibrary!longjmp()}
\newcommand{\URLSJ}{\href{http://en.wikipedia.org/wiki/Setjmp.h}{wikipedia}}

% TODO \myref{}
Tutto ciò è possibile perchè \printf è chiamata proprio alla fine della funzione \ttf in tutti i casi.
In qualche modo è simile alla funzione \TT{longjmp()}\footnote{\URLSJ} function.
E, ovviamente, tutto ciò viene fatto a favore della velocità di esecuzione.

Un simile caso con il compilatore ARM è descritto nella sezione \q{\PrintfSeveralArgumentsSectionName}, qui~(\myref{ARM_B_to_printf}).

\clearpage
\myparagraph{\Optimizing MSVC + \olly}
\myindex{\olly}

Testiamo questo esempio (ottimizzato) in \olly.  Questa è la prima iterazione:

\begin{figure}[H]
\centering
\myincludegraphics{patterns/10_strings/1_strlen/olly1.png}
\caption{\olly: inizio prima iterazione}
\label{fig:strlen_olly_1}
\end{figure}

Notiamo che \olly ha trovato un ciclo e per convenienza, ha \emph{avvolto} le sue istruzioni dentro le parentesi.
Cliccando con il tasto destro su \EAX e scegliendo 
\q{Follow in Dump}, la finestra della memoria scorrerà fino al punto giusto.
Possiamo vedere la stringa \q{hello!} in memoria.
C'è almeno
uno zero byte dopo la stringa e poi spazzatura casuale.

Se \olly vede un registro contenente un indirizzo valido, che punta ad una stringa,  
lo mostra come stringa.

\clearpage
Premiamo F8 (\stepover) un paio di volte, per arrivare all' inizio del corpo del ciclo:

\begin{figure}[H]
\centering
\myincludegraphics{patterns/10_strings/1_strlen/olly2.png}
\caption{\olly: inizio seconda iterazione}
\label{fig:strlen_olly_2}
\end{figure}

Notiamo che \EAX contiene l'indirizzo del secondo carattere nella stringa.

\clearpage

Dobbiamo premere F8 un numero di volte sufficente per uscire dal ciclo:

\begin{figure}[H]
\centering
\myincludegraphics{patterns/10_strings/1_strlen/olly3.png}
\caption{\olly: differenza di puntatori da calcolare}
\label{fig:strlen_olly_3}
\end{figure}

Notiamo che ora \EAX contiene l'indirizzo dello zero byte che si trova subito dopo la stringa più 1 (perché INC EAX è stato eseguito indipendentemente dal fatto che siamo usciti o meno dal ciclo).
Nel frattempo, \EDX non è cambiato,
quindi sta ancora puntando all' inizio della stringa.

La differenza tra questi due indirizzi verrà calcolata ora.

\clearpage
L' istruzione \SUB è stata appena eseguita:

\begin{figure}[H]
\centering
\myincludegraphics{patterns/10_strings/1_strlen/olly4.png}
\caption{\olly: \EAX  sta venendo decrementato}
\label{fig:strlen_olly_4}
\end{figure}

La differenza tra i puntatori, in questo momento, si trova nel registro \EAX---7.
In realtà, la lunghezza della stringa \q{hello!} è 6, 
ma con lo zero byte incluso---7.
Ma \TT{strlen()} deve ritornare il numero di caratteri nella stringa, diversi da zero.
Quindi viene eseguito un decremento, dopodichè la funziona ritorna.


