\subsubsection{ARM + \OptimizingKeilVI (\ARMMode)}

\begin{lstlisting}[caption=\OptimizingKeilVI (\ARMMode),style=customasmARM]
02 0C C0 E3          BIC     R0, R0, #0x200
01 09 80 E3          ORR     R0, R0, #0x4000
1E FF 2F E1          BX      LR
\end{lstlisting}

\myindex{ARM!\Instructions!BIC}
Der Befehl \INS{BIC} (\emph{BItwise bit Clear}) dient zum Löschen spezifischer
Bits. Er arbeitet wie ein \AND Befehl mit invertierten Operanden. Er entspricht
also einem \NOT+\AND Befehlspaar.

\myindex{ARM!\Instructions!ORR}
\INS{ORR} bedeutet \q{logical or} und ist analog zu \OR in x86.

So weit, so gut.

\subsubsection{ARM + \OptimizingKeilVI (\ThumbMode)}

\begin{lstlisting}[caption=\OptimizingKeilVI (\ThumbMode),style=customasmARM]
01 21 89 03          MOVS    R1, 0x4000
08 43                ORRS    R0, R1
49 11                ASRS    R1, R1, #5   ; erzeuge 0x200 und speichere in R1
88 43                BICS    R0, R1
70 47                BX      LR
\end{lstlisting}
Es scheint, dass Keil enschieden hat, dass der Code im Thumn mode, der
\TT{0x200} statt \TT{0x4000} verwendet, kompakter ist, als einer, der \TT{0x200}
in ein beliebiges Register schreibt.

% TODO1 пример, как компилятор при помощи сдвигов оптизирует такое: a=0x1000; b=0x2000; c=0x4000, etc

\myindex{ARM!\Instructions!ASRS}
Das ist der Grund dafür, dass dieser Wert mithilfe von \INS{ASRS} (\ASRdesc) als
$\TT{0x4000} \gg 5$ berechnet wird.

\subsubsection{ARM + \OptimizingXcodeIV (\ARMMode)}
\label{anomaly:LLVM}
\myindex{\CompilerAnomaly}

\begin{lstlisting}[caption=\OptimizingXcodeIV (\ARMMode),label=ARM_leaf_example3,style=customasmARM]
42 0C C0 E3          BIC             R0, R0, #0x4200
01 09 80 E3          ORR             R0, R0, #0x4000
1E FF 2F E1          BX              LR
\end{lstlisting}

Der Code, der von LLVM erzeugt wurde, könnte als Quellcode etwa wie folgt
ausgesehen haben:

\begin{lstlisting}[style=customc]
    REMOVE_BIT (rt, 0x4200);
    SET_BIT (rt, 0x4000);
\end{lstlisting}
Er macht genau das, was wir erwarten. Aber woher das \TT{0x4200}? Möglicherweise
handelt es sich um ein Artefakt des Optimierers von LLVM.

\footnote{Hier wurde der LLVM Build 2410.2.00 mit Apple Xcode 4.6.3 gebündelt}.
Möglicherweise also ein Fehler des Optimierers im Compiler; aber der erzeugte
Code funktioniert trotzdem korrekt.

Mehr zu Compiler-Anomalien hier:~(\myref{anomaly:Intel}).

\OptimizingXcodeIV im Thumb mode erzeugt identischen Code.

\subsubsection{ARM: Mehr zum Befehl \INS{BIC}}
\myindex{ARM!\Instructions!BIC}

Überarbeiten wir unser Beispiel ein wenig:

\begin{lstlisting}[style=customc]
int f(int a)
{
    int rt=a;

    REMOVE_BIT (rt, 0x1234);

    return rt;
};
\end{lstlisting}

Jetzt erzeugt der optimierende Keil 5.03 im ARM mode folgenden Code:

\begin{lstlisting}[style=customasmARM]
f PROC
        BIC      r0,r0,#0x1000
        BIC      r0,r0,#0x234
        BX       lr
        ENDP
\end{lstlisting}
Es gibt zwei \INS{BIC} Befehle, d.h. die Bits \TT{0x1234} werden in zwei
Durchgängen gelöscht.

Das liegt daran, dass es nicht möglich ist, \TT{0x1234} in einem einzigen
\INS{BIC} Befehl zu kodieren; deshalb müssen \TT{0x1000} und \TT{0x234} getrennt
werden.

\subsubsection{ARM64: \Optimizing GCC (Linaro) 4.9}

\Optimizing GCCcompiling für ARM64 kann den Befehl \AND anstelle von
\INS{BIC} verwenden:

\begin{lstlisting}[caption=\Optimizing GCC (Linaro) 4.9,style=customasmARM]
f:
	and	w0, w0, -513	; 0xFFFFFFFFFFFFFDFF
	orr	w0, w0, 16384	; 0x4000
	ret
\end{lstlisting}

\subsubsection{ARM64: \NonOptimizing GCC (Linaro) 4.9}

\NonOptimizing GCC erzeugt mehr redundanten Code; dieser funktiniert aber wie
die optimierte Variante:

\begin{lstlisting}[caption=\NonOptimizing GCC (Linaro) 4.9,style=customasmARM]
f:
	sub	sp, sp, #32
	str	w0, [sp,12]
	ldr	w0, [sp,12]
	str	w0, [sp,28]
	ldr	w0, [sp,28]
	orr	w0, w0, 16384	; 0x4000
	str	w0, [sp,28]
	ldr	w0, [sp,28]
	and	w0, w0, -513	; 0xFFFFFFFFFFFFFDFF
	str	w0, [sp,28]
	ldr	w0, [sp,28]
	add	sp, sp, 32
	ret
\end{lstlisting}
