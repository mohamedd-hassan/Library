\subsubsection{ARM + \OptimizingKeilVI (\ARMMode)}

\begin{lstlisting}[caption=\OptimizingKeilVI (\ARMMode),style=customasmARM]
02 0C C0 E3          BIC     R0, R0, #0x200
01 09 80 E3          ORR     R0, R0, #0x4000
1E FF 2F E1          BX      LR
\end{lstlisting}

\myindex{ARM!\Instructions!BIC}
\INS{BIC} (\emph{BItwise bit Clear})は特定のビットをクリアする命令です。
\AND 命令に似ていますが、反転したオペランドを使用します。
つまり、 \NOT+\AND 命令ペアに類似しています。

\myindex{ARM!\Instructions!ORR}
\INS{ORR} is \q{logical or}, analogous to \OR in x86.

\INS{ORR} は\q{論理OR}です。x86の \OR に類似しています。

ここまでは簡単です。

\subsubsection{ARM + \OptimizingKeilVI (\ThumbMode)}

\begin{lstlisting}[caption=\OptimizingKeilVI (\ThumbMode),style=customasmARM]
01 21 89 03          MOVS    R1, 0x4000
08 43                ORRS    R0, R1
49 11                ASRS    R1, R1, #5   ; 0x200を生成しR1に配置する
88 43                BICS    R0, R1
70 47                BX      LR
\end{lstlisting}

KeilはThumbモードのコードが\TT{0x4000}から\TT{0x200}になり、
\TT{0x200}を任意のレジスタに書き込むコードよりも
コンパクトであると判断したようです。
% TODO1 пример, как компилятор при помощи сдвигов оптизирует такое: a=0x1000; b=0x2000; c=0x4000, etc

\myindex{ARM!\Instructions!ASRS}

したがって、\INS{ASRS}(\ASRdesc)の助けを借りて、この値は$\TT{0x4000} \gg 5$として計算されます。

\subsubsection{ARM + \OptimizingXcodeIV (\ARMMode)}
\label{anomaly:LLVM}
\myindex{\CompilerAnomaly}

\begin{lstlisting}[caption=\OptimizingXcodeIV (\ARMMode),label=ARM_leaf_example3,style=customasmARM]
42 0C C0 E3          BIC             R0, R0, #0x4200
01 09 80 E3          ORR             R0, R0, #0x4000
1E FF 2F E1          BX              LR
\end{lstlisting}

LLVMが生成したコードは、次のようになるかもしれません。

\begin{lstlisting}[style=customc]
    REMOVE_BIT (rt, 0x4200);
    SET_BIT (rt, 0x4000);
\end{lstlisting}

そして、これはまさに必要としているものです。
しかしなぜ\TT{0x4200}なのでしょうか。
おそらく、LLVMのオプチマイザが生成した生成物でしょう。

\footnote{Apple Xcode 4.6.3にバンドルされたLLVM build 2410.2.00です}

コンパイラのオプチマイザのエラーかもしれませんが、生成されたコードはともあれ正しく動作します。

コンパイラのアノマリについての詳細はこちら~(\myref{anomaly:Intel})

Thumbモードでの \OptimizingXcodeIV は同じコードを生成します。

\subsubsection{ARM: \INS{BIC} 命令についての詳細}
\myindex{ARM!\Instructions!BIC}

例を少し改変してみましょう。

\begin{lstlisting}[style=customc]
int f(int a)
{
    int rt=a;

    REMOVE_BIT (rt, 0x1234);

    return rt;
};
\end{lstlisting}

ARMモードの最適化Keil 5.03 の結果は
以下のようになります。

\begin{lstlisting}[style=customasmARM]
f PROC
        BIC      r0,r0,#0x1000
        BIC      r0,r0,#0x234
        BX       lr
        ENDP
\end{lstlisting}

\INS{BIC}命令が2つあります。すなわち、ビット\TT{0x1234}は2パスでクリアされます。

なぜなら1つの\INS{BIC}命令では\TT{0x1234}をエンコードすることが不可能だからです。
しかし、\TT{0x1000} と \TT{0x234}をエンコードすることはできます。

\subsubsection{ARM64: \Optimizing GCC (Linaro) 4.9}

\Optimizing GCCコンパイラでARM64をコンパイルするなら\INS{BIC}の代わりに \AND 命令を使用できます。

\begin{lstlisting}[caption=\Optimizing GCC (Linaro) 4.9,style=customasmARM]
f:
	and	w0, w0, -513	; 0xFFFFFFFFFFFFFDFF
	orr	w0, w0, 16384	; 0x4000
	ret
\end{lstlisting}

\subsubsection{ARM64: \NonOptimizing GCC (Linaro) 4.9}

\NonOptimizing GCC はもっと冗長なコードを生成しますが、最適化されたように動作します。

\begin{lstlisting}[caption=\NonOptimizing GCC (Linaro) 4.9,style=customasmARM]
f:
	sub	sp, sp, #32
	str	w0, [sp,12]
	ldr	w0, [sp,12]
	str	w0, [sp,28]
	ldr	w0, [sp,28]
	orr	w0, w0, 16384	; 0x4000
	str	w0, [sp,28]
	ldr	w0, [sp,28]
	and	w0, w0, -513	; 0xFFFFFFFFFFFFFDFF
	str	w0, [sp,28]
	ldr	w0, [sp,28]
	add	sp, sp, 32
	ret
\end{lstlisting}
