\subsection{Compter les bits mis à 1}

Voici un exemple simple d'une fonction qui compte le nombre de bits mis à 1 dans
la valeur en entrée.

Cette opération est aussi appelée \q{population count}\footnote{les CPUs x86 modernes
(qui supportent SSE4) ont même une instruction POPCNT pour cela}.

\lstinputlisting[style=customc]{patterns/14_bitfields/4_popcnt/shifts.c}

Dans cette boucle, la variable d'itération $i$ prend les valeurs de 0 à 31, donc
la déclaration $1 \ll i$ prend les valeurs de 1 à \TT{0x80000000}.
Pour décrire cette opération en langage naturel, nous dirions \emph{décaler 1 par n bits à gauche}.
En d'autres mots, la déclaration $1 \ll i$ produit consécutivement toutes les positions
possible pour un bit dans un nombre de 32-bit.
Le bit libéré à droite est toujours à 0.

\label{2n_numbers_table}
Voici une table de tous les $1 \ll i$ possible
for $i=0 \ldots 31$:

%\small
\begin{center}
\begin{tabular}{ | l | l | l | l | }
\hline
\HeaderColor \CCpp expression & 
\HeaderColor Puissance de deux & 
\HeaderColor Forme décimale & 
\HeaderColor Forme hexadécimale \\
\hline
$1 \ll 0$ & $2^{0}$ & 1 & 1 \\
\hline
$1 \ll 1$ & $2^{1}$ & 2 & 2 \\
\hline
$1 \ll 2$ & $2^{2}$ & 4 & 4 \\
\hline
$1 \ll 3$ & $2^{3}$ & 8 & 8 \\
\hline
$1 \ll 4$ & $2^{4}$ & 16 & 0x10 \\
\hline
$1 \ll 5$ & $2^{5}$ & 32 & 0x20 \\
\hline
$1 \ll 6$ & $2^{6}$ & 64 & 0x40 \\
\hline
$1 \ll 7$ & $2^{7}$ & 128 & 0x80 \\
\hline
$1 \ll 8$ & $2^{8}$ & 256 & 0x100 \\
\hline
$1 \ll 9$ & $2^{9}$ & 512 & 0x200 \\
\hline
$1 \ll 10$ & $2^{10}$ & 1024 & 0x400 \\
\hline
$1 \ll 11$ & $2^{11}$ & 2048 & 0x800 \\
\hline
$1 \ll 12$ & $2^{12}$ & 4096 & 0x1000 \\
\hline
$1 \ll 13$ & $2^{13}$ & 8192 & 0x2000 \\
\hline
$1 \ll 14$ & $2^{14}$ & 16384 & 0x4000 \\
\hline
$1 \ll 15$ & $2^{15}$ & 32768 & 0x8000 \\
\hline
$1 \ll 16$ & $2^{16}$ & 65536 & 0x10000 \\
\hline
$1 \ll 17$ & $2^{17}$ & 131072 & 0x20000 \\
\hline
$1 \ll 18$ & $2^{18}$ & 262144 & 0x40000 \\
\hline
$1 \ll 19$ & $2^{19}$ & 524288 & 0x80000 \\
\hline
$1 \ll 20$ & $2^{20}$ & 1048576 & 0x100000 \\
\hline
$1 \ll 21$ & $2^{21}$ & 2097152 & 0x200000 \\
\hline
$1 \ll 22$ & $2^{22}$ & 4194304 & 0x400000 \\
\hline
$1 \ll 23$ & $2^{23}$ & 8388608 & 0x800000 \\
\hline
$1 \ll 24$ & $2^{24}$ & 16777216 & 0x1000000 \\
\hline
$1 \ll 25$ & $2^{25}$ & 33554432 & 0x2000000 \\
\hline
$1 \ll 26$ & $2^{26}$ & 67108864 & 0x4000000 \\
\hline
$1 \ll 27$ & $2^{27}$ & 134217728 & 0x8000000 \\
\hline
$1 \ll 28$ & $2^{28}$ & 268435456 & 0x10000000 \\
\hline
$1 \ll 29$ & $2^{29}$ & 536870912 & 0x20000000 \\
\hline
$1 \ll 30$ & $2^{30}$ & 1073741824 & 0x40000000 \\
\hline
$1 \ll 31$ & $2^{31}$ & 2147483648 & 0x80000000 \\
\hline
\end{tabular}
\end{center}
%\normalsize

Ces constantes (masques de bit) apparaissent très souvent le code et un rétro-ingénieur
pratiquant doit pouvoir les repérer rapidement.

Les nombres décimaux avant 65536 et les hexadécimaux sont faciles à mémoriser.
Tandis que les nombres décimaux après 65536 ne valent probablement pas la peine de
l'être.

Ces constantes sont utilisées très souvent pour mapper des flags sur des bits spécifiques.
Par exemple, voici un extrait de \TT{ssl\_private.h} du code source d'Apache 2.4.6:

\begin{lstlisting}[style=customc]
/**
 * Define the SSL options
 */
#define SSL_OPT_NONE           (0)
#define SSL_OPT_RELSET         (1<<0)
#define SSL_OPT_STDENVVARS     (1<<1)
#define SSL_OPT_EXPORTCERTDATA (1<<3)
#define SSL_OPT_FAKEBASICAUTH  (1<<4)
#define SSL_OPT_STRICTREQUIRE  (1<<5)
#define SSL_OPT_OPTRENEGOTIATE (1<<6)
#define SSL_OPT_LEGACYDNFORMAT (1<<7)
\end{lstlisting}

Revenons à notre exemple.

La macro \TT{IS\_SET} teste la présence d'un bit dans $a$.
\myindex{x86!\Instructions!AND}

La macro \TT{IS\_SET} est en fait l'opération logique AND (\emph{AND}) et elle renvoie
0 si le bit testé est absent (à 0), ou le masque de bit, si le bit est présent (à 1).
L'opérateur \emph{if()} en \CCpp exécute son code si l'expression n'est pas zéro,
cela peut même être 123456, c'est pourquoi il fonctionne toujours correctement.

% subsections
\subsubsection{MSVC: x86}

Voici ce que nous obtenons dans la sortie assembleur (MSVC 2010):

\lstinputlisting[style=customasmx86]{patterns/04_scanf/3_checking_retval/ex3_MSVC_x86.asm}

\myindex{x86!\Registers!EAX}
La fonction \glslink{caller}{appelante} (\main) a besoin du résultat de la fonction
\glslink{callee}{appelée}, donc la fonction \glslink{callee}{appelée} le renvoie
dans la registre \EAX.

\myindex{x86!\Instructions!CMP}
Nous le vérifions avec l'aide de l'instruction \TT{CMP EAX, 1} (\emph{CoMPare}).
En d'autres mots, nous comparons la valeur dans le registre \EAX avec 1.

\myindex{x86!\Instructions!JNE}
Une instruction de saut conditionnelle \JNE suit l'instruction \CMP. \JNE signifie
\emph{Jump if Not Equal} (saut si non égal).

Donc, si la valeur dans le registre \EAX n'est pas égale à 1, le \ac{CPU} va poursuivre
l'exécution à l'adresse mentionnée dans l'opérande \JNE, dans notre cas \TT{\$LN2@main}.
Passer le contrôle à cette adresse résulte en l'exécution par le \ac{CPU} de
\printf avec l'argument \TT{What you entered? Huh?}.
Mais si tout est bon, le saut conditionnel n'est pas pris, et un autre appel à \printf
est exécuté, avec deux arguments:\\
\TT{'You entered \%d...'} et la valeur de \TT{x}.

\myindex{x86!\Instructions!XOR}
\myindex{\CLanguageElements!return}
Puisque dans ce cas le second \printf n'a pas été exécuté, il y a un \JMP qui le précède (saut inconditionnel).
Il passe le contrôle au point après le second \printf et juste avant l'instruction \TT{XOR EAX, EAX}, qui implémente \TT{return 0}.

% FIXME internal \ref{} to x86 flags instead of wikipedia
\myindex{x86!\Registers!\Flags}
Donc, on peut dire que comparer une valeur avec une autre est \emph{usuellement} implémenté
par la paire d'instructions \CMP/\Jcc, où \emph{cc} est un \emph{code de condition}.
\CMP compare deux valeurs et met les flags\footnote{flags x86, voir aussi: \href{http://en.wikipedia.org/wiki/FLAGS_register_(computing)}{Wikipédia}.}
du processeur.
\Jcc vérifie ces flags et décide de passer LE Contrôle à l'adresse spécifiée ou non.

\myindex{x86!\Instructions!CMP}
\myindex{x86!\Instructions!SUB}
\myindex{x86!\Instructions!JNE}
\myindex{x86!\Registers!ZF}
\label{CMPandSUB} 
Cela peut sembler paradoxal, mais l'instruction \CMP est en fait un \SUB (soustraction).
Toutes les instructions arithmétiques mettent les flags du processeur, pas seulement \CMP.
Si nous comparons 1 et 1, $1-1$ donne 0 donc le flag \ZF va être mis (signifiant
que le dernier résultat est 0).
Dans aucune autre circonstance \ZF ne sera mis, sauf si les opérandes sont égaux.
\JNE vérifie seulement le flag \ZF et saute seulement si il n'est pas mis. \JNE
est un synonyme pour \JNZ (\emph{Jump if Not Zero} (saut si non zéro)).
L'assembleur génère le même opcode pour les instructions \JNE et \JNZ.
Donc, l'instruction \CMP peut être remplacée par une instruction \SUB et presque
tout ira bien, à la différence que \SUB altère la valeur du premier opérande.
\CMP est un \emph{SUB sans sauver le résultat, mais modifiant les flags}.

\subsubsection{MSVC: x86: IDA}

\myindex{IDA}
C'est le moment de lancer \IDA et d'essayer de faire quelque chose avec.
À propos, pour les débutants, c'est une bonne idée d'utiliser l'option \TT{/MD}
de MSVC, qui signifie que toutes les fonctions standards ne vont pas être liées
avec le fichier exécutable, mais vont à la place être importées depuis le fichier
\TT{MSVCR*.DLL}.
Ainsi il est plus facile de voir quelles fonctions standards sont utilisées et où.

En analysant du code dans \IDA, il est très utile de laisser des notes pour soi-même
(et les autres).
En la circonstance, analysons cet exemple, nous voyons que \TT{JNZ} sera déclenché
en cas d'erreur.
Donc il est possible de déplacer le curseur sur le label, de presser \q{n} et de
lui donner le nom \q{error}.
Créons un autre label---dans \q{exit}.
Voici mon résultat:

\lstinputlisting[style=customasmx86]{patterns/04_scanf/3_checking_retval/ex3.lst}

Maintenant, il est légèrement plus facile de comprendre le code.
Toutefois, ce n'est pas une bonne idée de commenter chaque instruction.

% FIXME draw button?
Vous pouvez aussi cacher (replier) des parties d'une fonction dans \IDA.
Pour faire cela, marquez le bloc, puis appuyez sur Ctrl-\q{--} sur le pavé numérique et
entrez le texte qui doit être affiché à la place.

Cachons deux blocs et donnons leurs un nom:

\lstinputlisting[style=customasmx86]{patterns/04_scanf/3_checking_retval/ex3_2.lst}

% FIXME draw button?
Pour étendre les parties de code précédemment cachées. utilisez Ctrl-\q{+} sur le
pavé numérique.

\clearpage
En appuyant sur \q{space}, nous voyons comment \IDA représente une fonction sous
forme de graphe:

\begin{figure}[H]
\centering
\myincludegraphics{patterns/04_scanf/3_checking_retval/IDA.png}
\caption{IDA en mode graphe}
\label{fig:ex3_IDA_1}
\end{figure}

Il y a deux flèches après chaque saut conditionnel: une verte et une rouge.
La flèche verte pointe vers le bloc qui sera exécuté si le saut est déclenché,
et la rouge sinon.

\clearpage
Il est possible de replier des nœuds dans ce mode et de leurs donner aussi un nom (\q{group nodes}).
Essayons avec 3 blocs:

\begin{figure}[H]
\centering
\myincludegraphics{patterns/04_scanf/3_checking_retval/IDA2.png}
\caption{IDA en mode graphe avec 3 nœuds repliés}
\label{fig:ex3_IDA_2}
\end{figure}

C'est très pratique.
On peut dire qu'une part importante du travail des rétro-ingénieurs (et de tout
autre chercheur également) est de réduire la quantité d'information avec laquelle
travailler.

\clearpage
\subsubsection{MSVC: x86 + \olly}

Essayons de hacker notre programme dans \olly, pour le forcer à penser que \scanf
fonctionne toujours sans erreur.
Lorsque l'adresse d'une variable locale est passée à \scanf, la variable contient
initiallement toujours des restes de données aléatoires, dans ce cas \TT{0x6E494714}:

\begin{figure}[H]
\centering
\myincludegraphics{patterns/04_scanf/3_checking_retval/olly_1.png}
\caption{\olly: passer l'adresse de la variable à \scanf}
\label{fig:scanf_ex3_olly_1}
\end{figure}

\clearpage
Lorsque \scanf s'exécute dans la console, entrons quelque chose qui n'est pas du
tout un nombre, comme \q{asdasd}.
\scanf termine avec 0 dans \EAX, ce qui indique qu'une erreur s'est produite.

Nous pouvons vérifier la variable locale dans le pile et noter qu'elle n'a pas changé.
En effet, qu'aurait écrit \scanf ici?
Elle n'a simplement rien fait à part renvoyer zéro.

Essayons de \q{hacker} notre programme.
Clique-droit sur \EAX,
parmi les options il y a \q{Set to 1} (mettre à 1).
C'est ce dont nous avons besoin.

Nous avons maintenant 1 dans \EAX, donc la vérification suivante va s'exécuter comme
souhaiter et \printf va afficher la valeur de la variable dans la pile.

Lorsque nous lançons le programme (F9) nous pouvons voir ceci dans la fenêtre
de la console:

\lstinputlisting[caption=fenêtre console]{patterns/04_scanf/3_checking_retval/console.txt}

En effet, 1850296084 est la représentation en décimal du nombre dans la pile (\TT{0x6E494714})!


\clearpage
\subsubsection{MSVC: x86 + Hiew}
\myindex{Hiew}

Cela peut également être utilisé comme un exemple simple de modification de fichier
exécutable.
Nous pouvons essayer de modifier l'exécutable de telle sorte que le programme va
toujours afficher notre entrée, quelle qui'elle soit.

En supposant que l'exécutable est compilé avec la bibliothèque externe \TT{MSVCR*.DLL}
(i.e., avec l'option \TT{/MD}) \footnote{c'est aussi appelé \q{dynamic linking}},
nous voyons la fonction \main au début de la section \TT{.text}.
Ouvrons l'exécutable dans Hiew et cherchons le début de la section \TT{.text} (Enter,
F8, F6, Enter, Enter).

Nous pouvons voir cela:

\begin{figure}[H]
\centering
\myincludegraphics{patterns/04_scanf/3_checking_retval/hiew_1.png}
\caption{Hiew: fonction \main}
\label{fig:scanf_ex3_hiew_1}
\end{figure}

Hiew trouve les chaîne \ac{ASCIIZ} et les affiche, comme il le fait avec le nom
des fonctions importées.

\clearpage
Déplacez le curseur à l'adresse \TT{.00401027} (où se trouve l'instruction \TT{JNZ},
que l'on doit sauter), appuyez sur F3, et ensuite tapez \q{9090} (qui signifie deux
\ac{NOP}s):

\begin{figure}[H]
\centering
\myincludegraphics{patterns/04_scanf/3_checking_retval/hiew_2.png}
\caption{Hiew: remplacement de \TT{JNZ} par deux \ac{NOP}s}
\label{fig:scanf_ex3_hiew_2}
\end{figure}

Appuyez sur F9 (update). Maintenant, l'exécutable est sauvé sur le disque. Il va
se comporter comme nous le voulions.

Deux \ac{NOP}s ne constitue probablement pas l'approche la plus esthétique.
Une autre façon de modifier cette instruction est d'écrire simplement 0 dans le
second octet de l'opcode ((\gls{jump offset}), donc ce \TT{JNZ} va toujours sauter
à l'instruction suivante.

Nous pouvons également faire le contraire: remplacer le premier octet avec \TT{EB}
sans modifier le second octet (\gls{jump offset}).
Nous obtiendrions un saut inconditionnel qui est toujours déclenché.
Dans ce cas le message d'erreur sera affiché à chaque fois, peu importe l'entrée.


\subsubsection{MSVC: x64}

\myindex{x86-64}

Puisque nous travaillons ici avec des variables typées \Tint{}, qui sont toujours
32-bit en x86-64, nous voyons comment la partie 32-bit des registres (préfixés
avec \TT{E-}) est également utilisée ici. % TODO clarify
Lorsque l'on travaille avec des ponteurs, toutefois, les parties 64-bit des registres
sont utilisées, préfixés avec \TT{R-}.

\lstinputlisting[caption=MSVC 2012 x64,style=customasmx86]{patterns/04_scanf/3_checking_retval/ex3_MSVC_x64_FR.asm}


\subsubsection{ARM: \OptimizingKeilVI (\ARMMode)}
\myindex{\CLanguageElements!switch}

\lstinputlisting[style=customasmARM]{patterns/08_switch/1_few/few_ARM_ARM_O3.asm}

A nouveau, en investiguant ce code, nous ne pouvons pas dire si il y avait un switch()
dans le code source d'origine ou juste un ensemble de déclarations if().

\myindex{ARM!\Instructions!ADRcc}

En tout cas, nous voyons ici des instructions conditionnelles (comme \ADREQ (\emph{Equal}))
qui ne sont exécutées que si $R0=0$, et qui chargent ensuite l'adresse de la chaîne
\emph{<<zero\textbackslash{}n>>} dans \Reg{0}.
\myindex{ARM!\Instructions!BEQ}
L'instruction suivante \ac{BEQ} redirige le flux d'exécution en \TT{loc\_170}, si $R0=0$.

Le lecteur attentif peut se demander si \ac{BEQ} s'exécute correctement puisque \ADREQ
a déjà mis une autre valeur dans le registre \Reg{0}.

Oui, elle s'exécutera correctement, car \ac{BEQ} vérifie les flags mis par l'instruction
\CMP et \ADREQ ne modifie aucun flag.

Les instructions restantes nous sont déjà familières.
Il y a seulement un appel à \printf, à la fin, et nous avons déjà examiné cette
astuce ici~(\myref{ARM_B_to_printf}).
A la fin, il y a trois chemins vers \printf{}.

\myindex{ARM!\Instructions!ADRcc}
\myindex{ARM!\Instructions!CMP}
La dernière instruction, \TT{CMP R0, \#2}, est nécessaire pour vérifier si $a=2$.

Si ce n'est pas vrai, alors \ADRNE charge un pointeur sur la chaîne \emph{<<something unknown \textbackslash{}n>>}
dans \Reg{0}, puisque $a$ a déjà été comparée pour savoir s'elle est égale
à 0 ou 1, et nous sommes sûrs que la variable $a$ n'est pas égale à l'un de
ces nombres, à ce point.
Et si $R0=2$, un pointeur sur la chaîne \emph{<<two\textbackslash{}n>>} sera chargé
par \ADREQ dans \Reg{0}.

\subsubsection{ARM: \OptimizingKeilVI (\ThumbMode)}

\lstinputlisting[style=customasmARM]{patterns/08_switch/1_few/few_ARM_thumb_O3.asm}

% FIXME а каким можно? к каким нельзя? \myref{} ->

Comme il y déjà été dit, il n'est pas possible d'ajouter un prédicat conditionnel
à la plupart des instructions en mode Thumb, donc ce dernier est quelque peu similaire
au code \ac{CISC}-style x86, facilement compréhensible.

\subsubsection{ARM64: GCC (Linaro) 4.9 \NonOptimizing}

\lstinputlisting[style=customasmARM]{patterns/08_switch/1_few/ARM64_GCC_O0_FR.lst}

Le type de la valeur d'entrée est \Tint, par conséquent le registre \RegW{0} est
utilisé pour garder la valeur au lieu du registre complet \RegX{0}.

Les pointeurs de chaîne sont passés à \puts en utilisant la paire d'instructions
\INS{ADRP}/\INS{ADD} comme expliqué dans l'exemple \q{\HelloWorldSectionName}:~\myref{pointers_ADRP_and_ADD}.

\subsubsection{ARM64: GCC (Linaro) 4.9 \Optimizing}

\lstinputlisting[style=customasmARM]{patterns/08_switch/1_few/ARM64_GCC_O3_FR.lst}

Ce morceau de code est mieux optimisé.
L'instruction \TT{CBZ} (\emph{Compare and Branch on Zero} comparer et sauter si zéro)
effectue un saut si \RegW{0} vaut zéro.
Il y a alors un saut direct à \puts au lieu de l'appeler, comme cela a été expliqué
avant:~\myref{JMP_instead_of_RET}.

\subsubsection{MIPS}

\lstinputlisting[caption=GCC 4.4.5 \Optimizing (IDA),style=customasmMIPS]{patterns/08_switch/1_few/MIPS_O3_IDA_FR.lst}

\myindex{MIPS!\Instructions!JR}

La fonction se termine toujours en appelant \puts, donc nous voyons un saut à \puts
(\INS{JR}: \q{Jump Register}) au lieu de \q{jump and link}.
Nous avons parlé de ceci avant: \myref{JMP_instead_of_RET}.

\myindex{MIPS!Load delay slot}
Nous voyons aussi souvent l'instruction \INS{NOP} après \INS{LW}.
Ceci est le slot de délai de chargement (\q{load delay slot}): un autre slot de
délai (\emph{delay slot}) en MIPS.
\myindex{MIPS!\Instructions!LW}

Une instruction suivant \INS{LW} peut s'exécuter pendant que \INS{LW} charge une
valeur depuis la mémoire.

Toutefois, l'instruction suivante ne doit pas utiliser le résultat de \INS{LW}.

Les CPU MIPS modernes ont la capacité d'attendre si l'instruction suivante utilise
le résultat de \INS{LW}, donc ceci est un peu démodé, mais GCC ajoute toujours
des NOPs pour les anciens CPU MIPS.
En général, ça peut être ignoré.

