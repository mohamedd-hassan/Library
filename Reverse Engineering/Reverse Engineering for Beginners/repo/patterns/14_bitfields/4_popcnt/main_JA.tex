\subsection{Counting bits set to 1}

入力値のビットの数を計算する関数の単純な例です。

この操作は\q{集団カウント}とも呼ばれます。\footnote{(SSE4をサポートする)モダンx86 CPUはこのためにPOPCNT命令を持っています}

\lstinputlisting[style=customc]{patterns/14_bitfields/4_popcnt/shifts.c}

このループでは、ループカウント値 $i$ は0から31を数えます。
$1 \ll i$ 文は1から\TT{0x80000000}まで数えます。
自然言語でこの操作を説明すると、\emph{1をnビット左シフトする}といえます。
言い換えると、$1 \ll i$ 文は結果として32ビット数のすべての可能なビット位置を生成します。
右側の解放されたビットは常にクリアされます。

\label{2n_numbers_table}
$i=0 \ldots 31$ で取りうるすべての値の表です。

%\small
\begin{center}
\begin{tabular}{ | l | l | l | l | }
\hline
\HeaderColor \CCpp 表現 & 
\HeaderColor 2のべき乗 & 
\HeaderColor 10進数形式 & 
\HeaderColor 16進数形式 \\
\hline
$1 \ll 0$ & $2^{0}$ & 1 & 1 \\
\hline
$1 \ll 1$ & $2^{1}$ & 2 & 2 \\
\hline
$1 \ll 2$ & $2^{2}$ & 4 & 4 \\
\hline
$1 \ll 3$ & $2^{3}$ & 8 & 8 \\
\hline
$1 \ll 4$ & $2^{4}$ & 16 & 0x10 \\
\hline
$1 \ll 5$ & $2^{5}$ & 32 & 0x20 \\
\hline
$1 \ll 6$ & $2^{6}$ & 64 & 0x40 \\
\hline
$1 \ll 7$ & $2^{7}$ & 128 & 0x80 \\
\hline
$1 \ll 8$ & $2^{8}$ & 256 & 0x100 \\
\hline
$1 \ll 9$ & $2^{9}$ & 512 & 0x200 \\
\hline
$1 \ll 10$ & $2^{10}$ & 1024 & 0x400 \\
\hline
$1 \ll 11$ & $2^{11}$ & 2048 & 0x800 \\
\hline
$1 \ll 12$ & $2^{12}$ & 4096 & 0x1000 \\
\hline
$1 \ll 13$ & $2^{13}$ & 8192 & 0x2000 \\
\hline
$1 \ll 14$ & $2^{14}$ & 16384 & 0x4000 \\
\hline
$1 \ll 15$ & $2^{15}$ & 32768 & 0x8000 \\
\hline
$1 \ll 16$ & $2^{16}$ & 65536 & 0x10000 \\
\hline
$1 \ll 17$ & $2^{17}$ & 131072 & 0x20000 \\
\hline
$1 \ll 18$ & $2^{18}$ & 262144 & 0x40000 \\
\hline
$1 \ll 19$ & $2^{19}$ & 524288 & 0x80000 \\
\hline
$1 \ll 20$ & $2^{20}$ & 1048576 & 0x100000 \\
\hline
$1 \ll 21$ & $2^{21}$ & 2097152 & 0x200000 \\
\hline
$1 \ll 22$ & $2^{22}$ & 4194304 & 0x400000 \\
\hline
$1 \ll 23$ & $2^{23}$ & 8388608 & 0x800000 \\
\hline
$1 \ll 24$ & $2^{24}$ & 16777216 & 0x1000000 \\
\hline
$1 \ll 25$ & $2^{25}$ & 33554432 & 0x2000000 \\
\hline
$1 \ll 26$ & $2^{26}$ & 67108864 & 0x4000000 \\
\hline
$1 \ll 27$ & $2^{27}$ & 134217728 & 0x8000000 \\
\hline
$1 \ll 28$ & $2^{28}$ & 268435456 & 0x10000000 \\
\hline
$1 \ll 29$ & $2^{29}$ & 536870912 & 0x20000000 \\
\hline
$1 \ll 30$ & $2^{30}$ & 1073741824 & 0x40000000 \\
\hline
$1 \ll 31$ & $2^{31}$ & 2147483648 & 0x80000000 \\
\hline
\end{tabular}
\end{center}
%\normalsize

このような定数(ビットマスク)はコード上、非常によく現れます。現役のリバースエンジニアは
これらを素早く見つけなければなりません。

65536以下の10進数と16進数は簡単に記憶できます。
65536を超える10進数はおそらく記憶する価値はないでしょう。

これらの定数は、フラグを特定のビットにマッピングするために非常によく使用されます。
たとえば、Apache 2.4.6のソースコードから\TT{ssl\_private.h}
を抜粋した例を次に示します。

\begin{lstlisting}[style=customc]
/**
 * Define the SSL options
 */
#define SSL_OPT_NONE           (0)
#define SSL_OPT_RELSET         (1<<0)
#define SSL_OPT_STDENVVARS     (1<<1)
#define SSL_OPT_EXPORTCERTDATA (1<<3)
#define SSL_OPT_FAKEBASICAUTH  (1<<4)
#define SSL_OPT_STRICTREQUIRE  (1<<5)
#define SSL_OPT_OPTRENEGOTIATE (1<<6)
#define SSL_OPT_LEGACYDNFORMAT (1<<7)
\end{lstlisting}

私たちの例に戻りましょう。

\TT{IS\_SET}マクロはビットの数を $a$ でチェックします。
\myindex{x86!\Instructions!AND}

\TT{IS\_SET}マクロは実際、論理AND演算(\emph{AND})で、
特定のビットがそこになければ0を返すか、
ビットが存在すれば、ビットをマスクします。
\CCpp の\emph{if()}演算子は、その式がゼロでない場合に実行しますが、
123456であっても正しく動作します。

% subsections
\subsubsection{x86}

\myparagraph{\NonOptimizing MSVC}

次の結果を得ます。(MSVC 2010)

\lstinputlisting[caption=MSVC 2010,style=customasmx86]{patterns/14_bitfields/2_set_reset/set_reset_msvc.asm}

\myindex{x86!\Instructions!OR}

\OR 命令は、他の1ビットを無視して1ビットをレジスタに設定します。

\myindex{x86!\Instructions!AND}

\AND は1ビットをリセットします。  \AND は1を除くすべてのビットをコピーするだけであると言えます。 
実際、2番目の \AND オペランドでは、保存する必要があるビットのみが設定され、
コピーしたくないビットは設定されません(ビットマスクでは0)。 
これは、ロジックを覚えるのが簡単な方法です。

\clearpage
\mysubparagraph{\olly}

\olly でこの例を試してみましょう。

まず、使用する定数のバイナリ形式を見てみましょう。

\TT{0x200} (0b0000000000000000000{\color{red}1}000000000) (すなわち、10番目のビット(1から数えて))

Inverted \TT{0x200} is \TT{0xFFFFFDFF} (0b1111111111111111111{\color{red}0}111111111).

\TT{0x4000} (0b00000000000000{\color{red}1}00000000000000) (すなわち、15番目のビット)

入力値は \TT{0x12340678} (0b10010001101000000011001111000)。
どのようにロードされるか見ていきます。

\begin{figure}[H]
\centering
\myincludegraphics{patterns/14_bitfields/2_set_reset/olly1.png}
\caption{\olly: 値が \ECX にロード}
\label{fig:set_reset_olly1}
\end{figure}

\clearpage
\OR が実行される。

\begin{figure}[H]
\centering
\myincludegraphics{patterns/14_bitfields/2_set_reset/olly2.png}
\caption{\olly: \OR が実行}
\label{fig:set_reset_olly2}
\end{figure}

15番目のビットがセット。 \TT{0x1234{\color{red}4}678} 
(0b10010001101000{\color{red}1}00011001111000).

\clearpage
値がリロードされる(コンパイラが最適化モードではないから)。

\begin{figure}[H]
\centering
\myincludegraphics{patterns/14_bitfields/2_set_reset/olly3.png}
\caption{\olly: 値が \EDX にリロード}
\label{fig:set_reset_olly3}
\end{figure}

\clearpage
\AND が実行される。

\begin{figure}[H]
\centering
\myincludegraphics{patterns/14_bitfields/2_set_reset/olly4.png}
\caption{\olly: \AND が実行}
\label{fig:set_reset_olly4}
\end{figure}

10番目のビットがクリアされる。(または、言い換えると、10番目を除いてすべてのビットが残りました)そして、最終的な値は
\TT{0x12344{\color{red}4}78} (0b1001000110100010001{\color{red}0}001111000)です。

\myparagraph{\Optimizing MSVC}

MSVCで最適化を有効に(\Ox)してコンパイルすると、コードはもっと短くなります。

\lstinputlisting[caption=\Optimizing MSVC,style=customasmx86]{patterns/14_bitfields/2_set_reset/set_reset_msvc_Ox.asm}

\myparagraph{\NonOptimizing GCC}

最適化なしのGCC 4.4.1を試してみましょう。

\lstinputlisting[caption=\NonOptimizing GCC,style=customasmx86]{patterns/14_bitfields/2_set_reset/set_reset_gcc.asm}

冗長なコードが見られますが、非最適化MSVC版より短くなります。

最適化 \Othree を有効にしてGCCを試してみましょう。

\myparagraph{\Optimizing GCC}

\lstinputlisting[caption=\Optimizing GCC,style=customasmx86]{patterns/14_bitfields/2_set_reset/set_reset_gcc_O3.asm}

短くなります。
より短いです。コンパイラが \AH レジスタを介して \EAX レジスタの部分で動作することは注目に値します。これは、8番目のビットから15番目のビットまでの \EAX レジスタの部分です。

\RegTableOne{RAX}{EAX}{AX}{AH}{AL}

\myindex{Intel!8086}
\myindex{Intel!80386}
注意:16ビットCPU 8086アキュムレータは \AX と命名され、8ビットの2つのレジスタで構成されていました。
\AL (下位バイト)および \AH (上位バイト)です。
80386ではほとんどすべてのレジスタが32ビットに拡張されて、アキュムレータの名前は \EAX でしたが、
互換性のために
\emph{古い部分}には \AX/\AH/\AL としてアクセスすることができます。

すべてのx86 CPUは16ビットの8086 CPUの後継バージョンなので、\emph{古い}16ビットのオペコードは
新しい32ビットのものよりも短くなります。 
だから、\INS{or ah, 40h}命令は3バイトしか占有しません。
ここでは\INS{or eax, 04000h}を発行する方が論理的ですが、
それは5または6バイトです。
(最初のオペランドのレジスタが \EAX でない場合)

\myparagraph{\Optimizing GCC and regparm}

\Othree 最適化フラグをオンにして\TT{regparm=3}に設定するとさらに短くなります。

\lstinputlisting[caption=\Optimizing GCC,style=customasmx86]{patterns/14_bitfields/2_set_reset/set_reset_gcc_O3_regparm3.asm}

\myindex{Inline code}

実際、最初の引数はすでに \EAX にロードされているので、インプレースで処理することは可能です。 
関数プロローグ(\INS{push ebp / mov ebp,esp})とエピローグ(\INS{pop ebp})は
ここでは簡単に省略することができますが、GCCはおそらくこのようなコードサイズの最適化を行うには不十分であることに注意してください。 
しかし、このような短い関数は\emph{インライン関数}より優れています。(\myref{inline_code})

\subsubsection{MSVC: x64}

\myindex{x86-64}

ここではx86-64の32ビットである \Tint 型変数について説明しているので、ここではレジスタの32ビット部分(\TT{E-}を前に付ける)も同様に使用されています。 
ただし、ポインタを使用している間は、64ビットのレジスタ部分が使用され、先頭に\TT{R-}が付きます。

\lstinputlisting[caption=MSVC 2012 x64,style=customasmx86]{patterns/04_scanf/3_checking_retval/ex3_MSVC_x64_JA.asm}


\subsubsection{ARM}

\myparagraph{\NonOptimizingKeilVI (\ARMMode)}

\lstinputlisting[style=customasmARM]{patterns/13_arrays/1_simple/simple_Keil_ARM_O0_JA.asm}

\Tint 型は32ビットのストレージを必要とします(または4バイト)。

20個の \Tint 変数を保存するには80バイト(\TT{0x50})が必要です。
だから、\INS{SUB SP, SP, \#0x50}のようになっています。

関数プロローグの命令はスタックにちょうどその分の空間を確保しています。

最初と次のループの両方で、ループイテレータ \var{i} は\Reg{4}レジスタに置かれています。

\myindex{ARM!Optional operators!LSL}

配列に書かれる数は $i*2$ として計算されます。これは1ビット左シフトすることと同じで、
\INS{MOV R0, R4,LSL\#1}命令がこれをしています。

\myindex{ARM!\Instructions!STR}
\INS{STR R0, [SP,R4,LSL\#2]}は\Reg{0}の内容を配列に書き込んでいます。

配列の要素へのポインタがどのように計算されるかを示しています。\ac{SP}は配列の先頭を示しています。\Reg{4}は $i$ です。

$i$ を2ビット左シフトすると、4倍することに等しいです。
(各配列の要素は4バイトです)そして配列の先頭アドレスに追加されます。

\myindex{ARM!\Instructions!LDR}

次のループは\INS{LDR R2, [SP,R4,LSL\#2]}命令の逆です。
配列から必要とする値をロードし、ポインタもまた同様に計算されます。

\myparagraph{\OptimizingKeilVI (\ThumbMode)}

\lstinputlisting[style=customasmARM]{patterns/13_arrays/1_simple/simple_Keil_thumb_O3_JA.asm}

Thumbコードも大変似ています。
\myindex{ARM!\Instructions!LSLS}

Thumbコードはビットシフト用の特別な命令を持っています(\TT{LSLS}のような)。
これは配列に書き込まれる値を計算し、また配列の各要素のアドレスも同様に計算します。

コンパイラはもう少し余分な空間をローカルスタックに確保します。しかし、最後の4バイトは使用されません。

\myparagraph{\NonOptimizing GCC 4.9.1 (ARM64)}

\lstinputlisting[caption=\NonOptimizing GCC 4.9.1 (ARM64),style=customasmARM]{patterns/13_arrays/1_simple/ARM64_GCC491_O0_JA.s}


\subsection{MIPS}

\lstinputlisting[caption=\Optimizing GCC 4.4.5 (IDA),style=customasmMIPS]{patterns/145_LCG/MIPS_O3_IDA_JA.lst}

おっと、ここでは1つの定数(0x3C6EF35Fまたは1013904223)しか表示されません。 
もう1つはどこでしょうか(1664525)?

1664525による乗算は、シフトと加算だけを使用して実行されるようです! 
この仮定を確認してみましょう:

\lstinputlisting[style=customc]{patterns/145_LCG/test.c}

\lstinputlisting[caption=\Optimizing GCC 4.4.5 (IDA),style=customasmMIPS]{patterns/145_LCG/test_O3_MIPS.lst}

本当に!

\subsubsection{MIPSの再配置}

また、メモリやストアから実際にメモリにロードする操作がどのように機能するかにも焦点を当てます。

ここのリストはIDAによって作成され、IDAはいくつかの詳細を隠しています。

objdumpを2回実行します:逆アセンブルされたリストと再配置リストを取得します。

\lstinputlisting[caption=\Optimizing GCC 4.4.5 (objdump)]{patterns/145_LCG/MIPS_O3_objdump.txt}

\TT{my\_srand()}関数の2つの再配置を考えてみましょう。

最初のアドレス0は\TT{R\_MIPS\_HI16}のタイプを持ち、
アドレス8の2番目のアドレスは\TT{R\_MIPS\_LO16}のタイプです。

つまり、.bssセグメントの先頭のアドレスは、0(アドレスの上位部分)および
8(アドレスの下位部分)のアドレスに書き込まれることを意味します。

\TT{rand\_state}変数は、.bssセグメントの先頭にあります。

したがって、命令 \LUI と \SW のオペランドにはゼロがあります。何もまだ存在しないからです。
コンパイラは何をそこに書き込んだらいいかわかりません。

リンカがこれを修正し、アドレスの上位部分が \LUI のオペランドに書き込まれ、
アドレスの下位部分が \SW のオペランドに書き込まれます。

\SW はアドレスの下位部分とレジスタ \$V0 にあるものを合計します(上位部分はそこにあります)。

これは my\_rand() 関数の場合と同じです。 R\_MIPS\_HI16 再配置は、リンカに.bssセグメントアドレスの上位部分を 
\LUI 命令に書き込むように指示します。

したがって、rand\_state 変数アドレスの上位部分はレジスタ \$V1 に存在します。

アドレス0x10にある \LW 命令は、上位部分と下位部分を合計し、rand\_state
変数の値を \$V0 にロードします。

アドレス0x54にある \SW 命令は、加算を再度行い、新しい値をrand\_stateグローバル変数に
格納します。

IDAは、ロード中に再配置を処理するため、これらの詳細は隠していますが、それらを念頭に置いておく必要があります。

% TODO add example of compiled binary, GDB example, etc...

