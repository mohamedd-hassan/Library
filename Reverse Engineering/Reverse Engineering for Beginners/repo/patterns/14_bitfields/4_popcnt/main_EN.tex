\subsection{Counting bits set to 1}

Here is a simple example of a function that calculates the number of bits set in the input value.

This operation is also called \q{population count}\footnote{modern x86 CPUs (supporting SSE4) even have a POPCNT instruction for it}.

\lstinputlisting[style=customc]{patterns/14_bitfields/4_popcnt/shifts.c}

In this loop, the iteration count value $i$ is counting from 0 to 31, 
so the $1 \ll i$ statement is counting from 1 to \TT{0x80000000}.
Describing this operation in natural language, we would say \emph{shift 1 by n bits left}.
In other words, $1 \ll i$ statement consequently produces all possible bit positions in a 32-bit number.
The freed bit at right is always cleared.

\label{2n_numbers_table}
Here is a table of all possible $1 \ll i$ 
for $i=0 \ldots 31$:

%\small
\begin{center}
\begin{tabular}{ | l | l | l | l | }
\hline
\HeaderColor \CCpp expression & 
\HeaderColor Power of two & 
\HeaderColor Decimal form & 
\HeaderColor Hexadecimal form \\
\hline
$1 \ll 0$ & $2^{0}$ & 1 & 1 \\
\hline
$1 \ll 1$ & $2^{1}$ & 2 & 2 \\
\hline
$1 \ll 2$ & $2^{2}$ & 4 & 4 \\
\hline
$1 \ll 3$ & $2^{3}$ & 8 & 8 \\
\hline
$1 \ll 4$ & $2^{4}$ & 16 & 0x10 \\
\hline
$1 \ll 5$ & $2^{5}$ & 32 & 0x20 \\
\hline
$1 \ll 6$ & $2^{6}$ & 64 & 0x40 \\
\hline
$1 \ll 7$ & $2^{7}$ & 128 & 0x80 \\
\hline
$1 \ll 8$ & $2^{8}$ & 256 & 0x100 \\
\hline
$1 \ll 9$ & $2^{9}$ & 512 & 0x200 \\
\hline
$1 \ll 10$ & $2^{10}$ & 1024 & 0x400 \\
\hline
$1 \ll 11$ & $2^{11}$ & 2048 & 0x800 \\
\hline
$1 \ll 12$ & $2^{12}$ & 4096 & 0x1000 \\
\hline
$1 \ll 13$ & $2^{13}$ & 8192 & 0x2000 \\
\hline
$1 \ll 14$ & $2^{14}$ & 16384 & 0x4000 \\
\hline
$1 \ll 15$ & $2^{15}$ & 32768 & 0x8000 \\
\hline
$1 \ll 16$ & $2^{16}$ & 65536 & 0x10000 \\
\hline
$1 \ll 17$ & $2^{17}$ & 131072 & 0x20000 \\
\hline
$1 \ll 18$ & $2^{18}$ & 262144 & 0x40000 \\
\hline
$1 \ll 19$ & $2^{19}$ & 524288 & 0x80000 \\
\hline
$1 \ll 20$ & $2^{20}$ & 1048576 & 0x100000 \\
\hline
$1 \ll 21$ & $2^{21}$ & 2097152 & 0x200000 \\
\hline
$1 \ll 22$ & $2^{22}$ & 4194304 & 0x400000 \\
\hline
$1 \ll 23$ & $2^{23}$ & 8388608 & 0x800000 \\
\hline
$1 \ll 24$ & $2^{24}$ & 16777216 & 0x1000000 \\
\hline
$1 \ll 25$ & $2^{25}$ & 33554432 & 0x2000000 \\
\hline
$1 \ll 26$ & $2^{26}$ & 67108864 & 0x4000000 \\
\hline
$1 \ll 27$ & $2^{27}$ & 134217728 & 0x8000000 \\
\hline
$1 \ll 28$ & $2^{28}$ & 268435456 & 0x10000000 \\
\hline
$1 \ll 29$ & $2^{29}$ & 536870912 & 0x20000000 \\
\hline
$1 \ll 30$ & $2^{30}$ & 1073741824 & 0x40000000 \\
\hline
$1 \ll 31$ & $2^{31}$ & 2147483648 & 0x80000000 \\
\hline
\end{tabular}
\end{center}
%\normalsize

These constant numbers (bit masks) very often appear in code and a practicing reverse engineer 
must be able to spot them quickly.

Decimal numbers below 65536 and hexadecimal ones are very easy to memorize.
While decimal numbers above 65536 are, probably, not worth memorizing.

These constants are very often used for mapping flags to specific bits.
For example, here is excerpt from \TT{ssl\_private.h} 
from Apache 2.4.6 source code:

\begin{lstlisting}[style=customc]
/**
 * Define the SSL options
 */
#define SSL_OPT_NONE           (0)
#define SSL_OPT_RELSET         (1<<0)
#define SSL_OPT_STDENVVARS     (1<<1)
#define SSL_OPT_EXPORTCERTDATA (1<<3)
#define SSL_OPT_FAKEBASICAUTH  (1<<4)
#define SSL_OPT_STRICTREQUIRE  (1<<5)
#define SSL_OPT_OPTRENEGOTIATE (1<<6)
#define SSL_OPT_LEGACYDNFORMAT (1<<7)
\end{lstlisting}

Let's get back to our example.

The \TT{IS\_SET} macro checks bit presence in $a$.
\myindex{x86!\Instructions!AND}

The \TT{IS\_SET} macro is in fact the logical AND operation (\emph{AND}) 
and it returns 0 if the specific bit is absent there,
or the bit mask, if the bit is present.
\emph{The if()} operator in \CCpp triggers if the expression in it is not zero, it might be even 123456, that is why
it always works correctly.

% subsections
\subsubsection{x86}

\myparagraph{x86 + MSVC}

Here is how the \TT{f\_signed()} function looks like:

\lstinputlisting[caption=\NonOptimizing MSVC 2010,style=customasmx86]{patterns/07_jcc/simple/signed_MSVC.asm}

\myindex{x86!\Instructions!JLE}

The first instruction, \JLE, stands for \emph{Jump if Less or Equal}. 
In other words, if the second operand is 
larger or equal to the first one, the control flow will be passed to the address or label specified in the instruction.
If this condition does not trigger because the second operand is smaller than the first one, the control flow would not be altered and the first \printf would be executed.
\myindex{x86!\Instructions!JNE}
The second check is \JNE: \emph{Jump if Not Equal}.
The control flow will not change if the operands are equal.

\myindex{x86!\Instructions!JGE}
The third check is \JGE: \emph{Jump if Greater or Equal}---jump if the first operand is larger than 
the second or if they are equal.
So, if all three conditional jumps are triggered, none of the \printf calls would be executed whatsoever. 
This is impossible without special intervention.
Now let's take a look at the \TT{f\_unsigned()} function.
The \TT{f\_unsigned()} function is the same as \TT{f\_signed()}, with the exception that the \JBE and \JAE instructions
are used instead of \JLE and \JGE, as follows:

\lstinputlisting[caption=GCC,style=customasmx86]{patterns/07_jcc/simple/unsigned_MSVC.asm}

\myindex{x86!\Instructions!JBE}
\myindex{x86!\Instructions!JAE}

As already mentioned, the branch instructions are different:
\JBE---\emph{Jump if Below or Equal} and \JAE---\emph{Jump if Above or Equal}.
These instructions (\INS{JA}/\JAE/\JB/\JBE) differ from \JG/\JGE/\JL/\JLE in the fact that they work with unsigned numbers.

\myindex{x86!\Instructions!JA}
\myindex{x86!\Instructions!JB}
\myindex{x86!\Instructions!JG}
\myindex{x86!\Instructions!JL}
\myindex{Signed numbers}

That is why if we see \JG/\JL in use instead of \INS{JA}/\JB or vice-versa, 
we can be almost sure that the variables are signed or unsigned, respectively.
Here is also the \main function, where there is nothing much new to us:

\lstinputlisting[caption=\main,style=customasmx86]{patterns/07_jcc/simple/main_MSVC.asm}

\clearpage
\subsubsection{MSVC + \olly}
\myindex{\olly}

Let's load our example into \olly and set a breakpoint on \comp.
We can see how the values are compared at the first \comp call:

\begin{figure}[H]
\centering
\myincludegraphics{patterns/18_pointers_to_functions/olly1.png}
\caption{\olly: first call of \comp}
\label{fig:qsort_olly1}
\end{figure}

\olly shows the compared values in the window under the code window, for convenience.
We can also see that the \ac{SP} points to \ac{RA}, where the \qsort function is (located in \TT{MSVCR100.DLL}).

\clearpage
By tracing (F8) until the \TT{RETN} instruction and pressing F8 one more time, we return to the \qsort function:

\begin{figure}[H]
\centering
\myincludegraphics{patterns/18_pointers_to_functions/olly2.png}
\caption{\olly: the code in \qsort right after \comp call}
\label{fig:qsort_olly2}
\end{figure}

That has been a call to the comparison function.

\clearpage
Here is also a screenshot of the moment of the second call of \comp{}---now values that have to be compared are different:

\begin{figure}[H]
\centering
\myincludegraphics{patterns/18_pointers_to_functions/olly3.png}
\caption{\olly: second call of \comp}
\label{fig:qsort_olly3}
\end{figure}


\clearpage
\myparagraph{x86 + MSVC + Hiew}
\myindex{Hiew}

We can try to patch the executable file in a way 
that the \TT{f\_unsigned()} function would always print \q{a==b}, 
no matter the input values.
Here is how it looks in Hiew:

\begin{figure}[H]
\centering
\myincludegraphics{patterns/07_jcc/simple/hiew_unsigned1.png}
\caption{Hiew: \TT{f\_unsigned()} function}
\label{fig:jcc_hiew_1}
\end{figure}

Essentially, we have to accomplish three tasks:
\begin{itemize}
\item force the first jump to always trigger;
\item force the second jump to never trigger;
\item force the third jump to always trigger.
\end{itemize}

Thus we can direct the code flow to always pass through the second \printf, and output \q{a==b}.

Three instructions (or bytes) has to be patched:

\begin{itemize}
\item The first jump becomes \JMP, but the \gls{jump offset} would remain the same.

\item 
The second jump might be triggered sometimes, but in any case it will jump to the next
instruction, because, we set the \gls{jump offset} to 0.

In these instructions the \gls{jump offset} is added to the address for the next instruction.
So if the offset is 0,
the jump will transfer the control to the next instruction.

\item 
The third jump we replace with \JMP just as we do with the first one, so it will always trigger.

\end{itemize}

\clearpage
Here is the modified code:

\begin{figure}[H]
\centering
\myincludegraphics{patterns/07_jcc/simple/hiew_unsigned2.png}
\caption{Hiew: let's modify the \TT{f\_unsigned()} function}
\label{fig:jcc_hiew_2}
\end{figure}

If we miss to change any of these jumps, then several \printf calls may execute, while we want to execute only one.

\myparagraph{\NonOptimizing GCC}

\myindex{puts() instead of printf()}
\NonOptimizing GCC 4.4.1 
produces almost the same code, but with \puts~(\myref{puts}) instead of \printf.

\myparagraph{\Optimizing GCC}

An observant reader may ask, why execute \CMP several times, 
if the flags has the same values after each execution?

Perhaps optimizing MSVC cannot do this, but optimizing GCC 4.8.1 can go deeper:

\lstinputlisting[caption=GCC 4.8.1 f\_signed(),style=customasmx86]{patterns/07_jcc/simple/GCC_O3_signed.asm}

% should be here instead of 'switch' section?
We also see \TT{JMP puts} here instead of \TT{CALL puts / RETN}.

This kind of trick will have explained later: \myref{JMP_instead_of_RET}.

This type of x86 code 
is somewhat rare.
MSVC 2012 as it seems, can't generate such code.
On the other hand, assembly language programmers are fully aware of the fact that \TT{Jcc} 
instructions can be stacked.

So if you see such stacking somewhere, it is highly probable that the code was hand-written.

The \TT{f\_unsigned()} function is not that 
\ae{}sthetically short:

\lstinputlisting[caption=GCC 4.8.1 f\_unsigned(),style=customasmx86]{patterns/07_jcc/simple/GCC_O3_unsigned_EN.asm}

Nevertheless, there are two \TT{CMP} instructions instead of three.

So optimization algorithms of GCC 4.8.1 are probably not perfect yet. 

\subsection{x64: \Optimizing MSVC 2013}

\lstinputlisting[caption=\Optimizing MSVC 2013 x64,style=customasmx86]{\CURPATH/MSVC2013_x64_Ox_EN.asm}

First, MSVC inlined the \strlen{} function code, because it concluded this 
is to be faster than the usual \strlen{} work + the cost of calling it and returning from it.
This is called inlining: \myref{inline_code}.

\myindex{x86!\Instructions!OR}
\myindex{\CStandardLibrary!strlen()}
\label{using_OR_instead_of_MOV}
The first instruction of the inlined \strlen{} is\\
\TT{OR RAX, 0xFFFFFFFFFFFFFFFF}. 

MSVC often uses \TT{OR} instead of \TT{MOV RAX, 0xFFFFFFFFFFFFFFFF}, because resulting opcode is shorter.

And of course, it is equivalent: all bits are set, and a number with all bits set is $-1$ 
in two's complement arithmetic.

Why would the $-1$ number be used in \strlen{}, one might ask.
Due to optimizations, of course.
Here is the code that MSVC generated:

\lstinputlisting[caption=Inlined \strlen{} by MSVC 2013 x64,style=customasmx86]{\CURPATH/strlen_MSVC_EN.asm}

Try to write shorter if you want to initialize the counter at 0!
OK, let' try:

\lstinputlisting[caption=Our version of \strlen{},style=customasmx86]{\CURPATH/my_strlen_EN.asm}

We failed. We have to use additional \INS{JMP} instruction!

So what the MSVC 2013 compiler did is to move the \TT{INC} instruction to the place before 
the actual character loading.

If the first character is 0, that's OK, \RAX is 0 at this moment, 
so the resulting string length is 0.

The rest in this function seems easy to understand.

\subsection{x64: \NonOptimizing GCC 4.9.1}

\lstinputlisting[style=customasmx86]{\CURPATH/GCC491_x64_O0_EN.asm}

Comments are added by the author of the book.

After the execution of \strlen{}, the control is passed to the L2 label, 
and there two clauses are checked, one after another.

\myindex{\CLanguageElements!Short-circuit}
The second will never be checked, if the first one (\emph{str\_len==0}) is false 
(this is \q{short-circuit}).

Now let's see this function in short form:

\begin{itemize}
\item First for() part (call to \strlen{})
\item goto L2
\item L5: for() body. goto exit, if needed
\item for() third part (decrement of str\_len)
\item L2: 
for() second part: check first clause, then second. goto loop body begin or exit.
\item L4: // exit
\item return s
\end{itemize}

\subsection{x64: \Optimizing GCC 4.9.1}
\label{string_trim_GCC_x64_O3}

\lstinputlisting[style=customasmx86]{\CURPATH/GCC491_x64_O3_EN.asm}

Now this is more complex.

The code before the loop's body start is executed only once, but it has the \CRLF{} 
characters check too!
What is this code duplication for?

The common way to implement the main loop is probably this:

\begin{itemize}
\item (loop start) check for 
\CRLF{} characters, make decisions
\item store zero character
\end{itemize}

But GCC has decided to reverse these two steps. 

Of course, \emph{store zero character} cannot be first step, so another check is needed:

\begin{itemize}
\item workout first character. match it to \CRLF{}, exit if character is not \CRLF{}

\item (loop begin) store zero character

\item check for \CRLF{} characters, make decisions
\end{itemize}

Now the main loop is very short, which is good for latest \ac{CPU}s.

The code doesn't use the str\_len variable, but str\_len-1.
So this is more like an index in a buffer.

Apparently, GCC notices that the str\_len-1 statement is used twice.

So it's better to allocate a variable which always holds a value that's smaller than 
the current string length by one, 
and decrement it (this is the same effect as decrementing the str\_len variable).

\subsection{ARM: \OptimizingKeilVI (\ARMMode)}

And again, the compiler took advantage of ARM mode's conditional instructions, 
so the code is much more compact.

\lstinputlisting[caption=\OptimizingKeilVI (\ARMMode),style=customasmARM]{\CURPATH/Keil_ARM_O3_EN.s}

\subsection{ARM: \OptimizingKeilVI (\ThumbMode)}
\myindex{\CompilerAnomaly}
\label{Keil_anomaly}

There are less conditional instructions in Thumb mode, so the code is simpler.

But there are is really weird thing with the 0x20 and 0x1F offsets (lines 22 and 23).
Why did the Keil compiler do so?
Honestly, it's hard to say.

It has to be a quirk of Keil's optimization process.
Nevertheless, the code works correctly.

\lstinputlisting[caption=\OptimizingKeilVI (\ThumbMode),numbers=left,style=customasmARM]{\CURPATH/Keil_thumb_O3_EN.s}


\scanf returns the result of its work in register \$V0. It is checked at address 0x004006E4
by comparing the values in \$V0 with \$V1 (1 has been stored in \$V1 earlier, at 0x004006DC).
\INS{BEQ} stands for \q{Branch Equal}.
If the two values are equal (i.e., success), the execution jumps to address 0x0040070C.


