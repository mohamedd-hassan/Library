\subsection{Gesetzte Bits zählen}
Hier ist ein einfaches Beispiel einer Funktion, die die Anzahl der gesetzten
Bits in einem Eingabewert zählt.

Diese Operation wird auch \q{population count}\footnote{moderne x86 CPUs
(die SSE4 unterstützen) haben zu diesem Zweck sogar einen eigenen POPCNT Befehl}
genannt.

\lstinputlisting[style=customc]{patterns/14_bitfields/4_popcnt/shifts.c}
In dieser Schleife wird der Wert von $i$ schrittweise von 0 bis 31 erhöht,
sodass der Ausdruck $1 \ll i$ von 1 bis \TT{0x80000000} zählt.
In natürlicher Sprache würden wir diese Operation als \emph{verschiebe 1 um n
Bits nach links} beschreiben.
Mit anderen Worten: Der Ausdruck $1 \ll i$ erzeugt alle möglichen Bitpositionen
in einer 32-Bit-Zahl.
Das freie Bit auf der rechten Seite wird jeweils gelöscht.

\label{2n_numbers_table}
Hier ist eine Tabelle mit allen Werten von $1 \ll i$ 
für $i=0 \ldots 31$:

%\small
\begin{center}
\begin{tabular}{ | l | l | l | l | }
\hline
\HeaderColor \CCpp Ausdruck & 
\HeaderColor Zweierpotenz & 
\HeaderColor Dezimalzahl & 
\HeaderColor Hexadezimalzahl \\
\hline
$1 \ll 0$ & $2^{0}$ & 1 & 1 \\
\hline
$1 \ll 1$ & $2^{1}$ & 2 & 2 \\
\hline
$1 \ll 2$ & $2^{2}$ & 4 & 4 \\
\hline
$1 \ll 3$ & $2^{3}$ & 8 & 8 \\
\hline
$1 \ll 4$ & $2^{4}$ & 16 & 0x10 \\
\hline
$1 \ll 5$ & $2^{5}$ & 32 & 0x20 \\
\hline
$1 \ll 6$ & $2^{6}$ & 64 & 0x40 \\
\hline
$1 \ll 7$ & $2^{7}$ & 128 & 0x80 \\
\hline
$1 \ll 8$ & $2^{8}$ & 256 & 0x100 \\
\hline
$1 \ll 9$ & $2^{9}$ & 512 & 0x200 \\
\hline
$1 \ll 10$ & $2^{10}$ & 1024 & 0x400 \\
\hline
$1 \ll 11$ & $2^{11}$ & 2048 & 0x800 \\
\hline
$1 \ll 12$ & $2^{12}$ & 4096 & 0x1000 \\
\hline
$1 \ll 13$ & $2^{13}$ & 8192 & 0x2000 \\
\hline
$1 \ll 14$ & $2^{14}$ & 16384 & 0x4000 \\
\hline
$1 \ll 15$ & $2^{15}$ & 32768 & 0x8000 \\
\hline
$1 \ll 16$ & $2^{16}$ & 65536 & 0x10000 \\
\hline
$1 \ll 17$ & $2^{17}$ & 131072 & 0x20000 \\
\hline
$1 \ll 18$ & $2^{18}$ & 262144 & 0x40000 \\
\hline
$1 \ll 19$ & $2^{19}$ & 524288 & 0x80000 \\
\hline
$1 \ll 20$ & $2^{20}$ & 1048576 & 0x100000 \\
\hline
$1 \ll 21$ & $2^{21}$ & 2097152 & 0x200000 \\
\hline
$1 \ll 22$ & $2^{22}$ & 4194304 & 0x400000 \\
\hline
$1 \ll 23$ & $2^{23}$ & 8388608 & 0x800000 \\
\hline
$1 \ll 24$ & $2^{24}$ & 16777216 & 0x1000000 \\
\hline
$1 \ll 25$ & $2^{25}$ & 33554432 & 0x2000000 \\
\hline
$1 \ll 26$ & $2^{26}$ & 67108864 & 0x4000000 \\
\hline
$1 \ll 27$ & $2^{27}$ & 134217728 & 0x8000000 \\
\hline
$1 \ll 28$ & $2^{28}$ & 268435456 & 0x10000000 \\
\hline
$1 \ll 29$ & $2^{29}$ & 536870912 & 0x20000000 \\
\hline
$1 \ll 30$ & $2^{30}$ & 1073741824 & 0x40000000 \\
\hline
$1 \ll 31$ & $2^{31}$ & 2147483648 & 0x80000000 \\
\hline
\end{tabular}
\end{center}
%\normalsize

Diese Konstanten (Bitmasken) tauchen im Code oft auf und ein Reverse Engineer
muss in der Lage sein, sie schnell und sicher zu erkennen.

% TBT
Es dazu jedoch nicht notwendig, die Dezimalzahlen (Zweierpotenzen) größer
65535 auswendig zu kennen. Die hexadezimalen Zahlen sind leicht zu merken.

Die Konstanten werden häufig verwendet um Flags einzelnen Bits zuzuordnen. 
Hier ist zum Beispiel ein Auszug aus \TT{ssl\_private.h} aus dem Quellcode von
Apache 2.4.6:

\begin{lstlisting}[style=customc]
/**
 * Define the SSL options
 */
#define SSL_OPT_NONE           (0)
#define SSL_OPT_RELSET         (1<<0)
#define SSL_OPT_STDENVVARS     (1<<1)
#define SSL_OPT_EXPORTCERTDATA (1<<3)
#define SSL_OPT_FAKEBASICAUTH  (1<<4)
#define SSL_OPT_STRICTREQUIRE  (1<<5)
#define SSL_OPT_OPTRENEGOTIATE (1<<6)
#define SSL_OPT_LEGACYDNFORMAT (1<<7)
\end{lstlisting}

Zurück zu unserem Beispiel.

Das Makro \TT{IS\_SET} prüft auf Anwesenheit von Bits in $a$.
\myindex{x86!\Instructions!AND}

Das Makro \TT{IS\_SET} entspricht dabei dem logischen (\emph{AND})
und gibt 0 zurück, wenn das entsprechende Bit nicht gesetzt ist, oder die
Bitmaske, wenn das Bit gesetzt ist.
Der Operator \emph{if()} wird in \CCpp ausgeführt, wenn der boolesche Ausdruck
nicht null ist (er könnte sogar 123456 sein), weshalb es meistens richtig
funktioniert.


% subsections
\subsubsection{x86}

\myparagraph{\NonOptimizing MSVC}

Kompilieren wir es:

\lstinputlisting[style=customasmx86]{patterns/10_strings/1_strlen/10_1_msvc_DE.asm}

\myindex{x86!\Instructions!MOVSX}
\myindex{x86!\Instructions!TEST}

Wir finden hier zwei neue Befehle: \MOVSX und \TEST.

\label{MOVSX}

Der erste --\MOVSX--nimmt ein Byte aus einer Speicheradresse und speichert den
Wert in einem 32-bit-Register.
\MOVSX steht für \emph{MOV with Sign-Extend}.
\MOVSX setzt die übrigen Bits vom 8. bis zum 31. auf 1, falls das Quellbyte
\emph{negativ} ist oder auf 0, falls es \emph{positiv} ist.

Und hier ist der Grund dafür.

Standardmäßig ist der \Tchar Datentyp in MSVC und GCC vorzeichenbehaftet
(signed). Wenn wir zwei Werte haben, einen \Tchar und einen \Tint, (\Tint ist
ebenfalls vorzeichenbehaftet) und der erste Wert enthält -2 (kodiert als
\TT{0xFE}) und wir kopieren dieses Byte in den \Tint Container, erhalten wir
\TT{0x000000FE} und dies entspricht als signed \Tint 254, aber nicht -2. Der
signed \Tint -2 wird als \TT{0xFFFFFFFE} dargestellt. Wenn wir also \TT{0xFE}
vom Datentyp \Tchar nach \Tint übertragen wollen, müssen wir das Vorzeichen
identifizieren und den Wert entsprechend erweitern. Genau dies tut der Befehl
\MOVSX.

Es ist schwer zu sagen, ob der Compiler tatsächlich eine \Tchar Variable in \EDX
speichern muss, er könnte auch einen 8-Bit-Registerteil (z.B. \DL) dafür
verwenden . Offenbar arbeitet der \gls{register allocator} des Compilers auf
diese Art.

\myindex{ARM!\Instructions!TEST}

Wir finden im Weiteren den Befehl \TT{TEST EDX, EDX}. 
Für mehr Informationen zum \TEST Befehl siehe auch den Abschnitt über
Bitfelder~(\myref{sec:bitfields}).
In unserem Fall überprüft der Befehl lediglich, ob der Wert im Register \EDX
gleich 0 ist.

\myparagraph{\NonOptimizing GCC}

Schauen wir uns GCC 4.4.1 an:

\lstinputlisting[style=customasmx86]{patterns/10_strings/1_strlen/10_3_gcc.asm}

\label{movzx}
\myindex{x86!\Instructions!MOVZX}

Das Ergebnis ist fast identisch mit dem von MSVC, aber hier finden wir \MOVZX
anstelle von \MOVSX. 
\MOVZX steht für \emph{MOV with Zero-Extend}. 
Dieser Befehl kopiert einen 8-Bit- oder 16-Bit-Wert in ein 32-Bit-Register und
setzt die übrigen Bits auf 0.
Tatsächlich findet dieser Befehl vor allem deshalb Anwendung, weil er es uns
erlaubt, folgendes Befehlspaar zu ersetzen:\\
\TT{xor eax, eax / mov al, [...]}.

Andererseits ist offensichtlich, dass der Compiler folgenden Code erzeugen kann:
\\
\TT{mov al, byte ptr [eax] / test al, al}--es ist fast das gleiche, aber die
oberen Bits des \EAX Registers enthalten hier Zufallswerte bzw.
sogenanntes Zufallsrauschen.
Aber bedenken wir den Nachteil des Compilers--er kann nicht leichter
verständlichen Code erzeugen. 
Genau genommen, ist der Compiler überhaupt nicht daran gebunden, (Menschen)
verständlichen Code zu erzeugen.

\myindex{x86!\Instructions!SETcc}

Der nächste neue Befehl für uns ist \SETNZ.
In diesem Fall setzt \TT{test al,al} das \ZF flag auf 0, falls \AL nicht 0
enthät, aber \SETNZ setzt \AL auf 1, falls \TT{ZF==0} (IT{NZ} steht für
\emph{non zero}).
In natürlicher Sprache, \emph{falls \AL ungleich 0, springe zu loc\_80483F0}. 
Der Compiler erzeugt leicht redundanten Code, aber bedenken wir, dass die
Optimierung hier deaktiviert ist.

\myparagraph{\Optimizing MSVC}
\label{strlen_MSVC_Ox}

Kompilieren wir nun alles in MSVC 2012 mit aktivierter Optimierung (\Ox):

\lstinputlisting[caption=\Optimizing MSVC 2012 /Ob0,style=customasmx86]{patterns/10_strings/1_strlen/10_2_DE.asm}

Jetzt ist alles einfacher.
Unnötig zu erwähnen, dass der Compiler Register mit solcher Effizienz nur in
kleinen Funktionen mit einigen wenigen lokalen Variablen verwenden kann.

\myindex{x86!\Instructions!INC}
\myindex{x86!\Instructions!DEC}
\INC/\DEC---sind \glslink{increment}{inkrement}/\glslink{decrement}{dekrement} Befehle; mit anderen Worten:
addiere oder subtrahiere 1 zu bzw. von einer Variable. 

\clearpage
\myparagraph{\olly + standardmäßig gepackte Felder}
\myindex{\olly}
Betrachten wir unser Beispiel (in dem die Felder standardmäßig auf 4 Byte angeordnet werden) in \olly:

\begin{figure}[H]
\centering
\myincludegraphics{patterns/15_structs/4_packing/olly_packing_4.png}
\caption{\olly: vor der Ausführung von \printf}
\label{fig:packing_olly_4}
\end{figure}
Wir sehen unsere 4 Felder im Datenfenster.

Wir fragen uns aber, woher die Zufallsbytes (0x30, 0x37, 0x01) stammen, die neben dem ersten ($a$) und dritten ($c$)
Feld liegen.

Betrachten wir unser Listing \myref{src:struct_packing_4}, erkennen wir, dass das erste und dritte Feld vom Typ \Tchar
ist, und daher nur ein Byte geschrieben wird, nämlich 1 bzw. 3 (Zeilen 6 und 8).

Die übrigen 3 Byte des 32-Bit-Wortes werden im Speicher nicht verändert!
Deshalb befinden sich hier zufällige Reste.

\myindex{x86!\Instructions!MOVSX}
Diese Reste beeinflussen den Output von \printf in keinster Weise, da die Werte für die Funktion mithilfe von \MOVSX
vorbereitet werden, der Bytes und nicht Worte als Argumente hat: 
\lstref{src:struct_packing_4} (Zeilen 34 und 38).
Der vorzeichenerweiternde Befehl \MOVSX wird hier übrigens verwendet, da \Tchar standardmäßig in MSVC und GCC
vorzeichenbehaftet ist.
Würde hier der Datentyp \TT{unsigned char} oder \TT{uint8\_t} verwendet, würde der Befehl \MOVZX stattdessen verwendet.

\clearpage
\myparagraph{\olly + Felder auf 1 Byte Grenzen angeordnet}
\myindex{\olly}
Hier sind die Dinge viel klarer ersichtlich: 4 Felder benötigen 16 Byte und die Werte werden nebeneinander gespeichert.

\begin{figure}[H]
\centering
\myincludegraphics{patterns/15_structs/4_packing/olly_packing_1.png}
\caption{\olly: Vor der Ausführung von \printf}
\label{fig:packing_olly_1}
\end{figure}


\myparagraph{\Optimizing GCC}

Schauen wir uns GCC 4.4.1 mit aktiverter Optimierung (\Othree key) an:

\lstinputlisting[style=customasmx86]{patterns/10_strings/1_strlen/10_3_gcc_O3.asm}
 
Hier erzeugt GCC fast identischen Code zu MSVC, außer dass hier ein \MOVZX
auftritt. 
In der Tat könnte \MOVZX hier durch \TT{mov dl, byte ptr [eax]} ersetzt werden.
 
Möglicherweise ist es einfacher für den GCC Code Generator sich daran zu
\emph{erinnern}, dass das gesamte 32-bit-\EDX Register für eine \Tchar Variable
reserviert ist und so sicherzustellen, dass die oberen Bits zu keinem Zeitpunkt
Zufallsrauschen enthalten.

\label{strlen_NOT_ADD}
\myindex{x86!\Instructions!NOT}
\myindex{x86!\Instructions!XOR}

Danach finden wir also einen neuen Befehl--\NOT. Dieser Befehl kippt alle Bits
in seinem Operanden.\\
Man kann sagen, dass es sich um ein Synonym zum Befehl \TT{XOR ECX, 0ffffffffh}
handelt. 
\NOT und das darauf folgende \ADD berechnen die Differenz im Pointer und
subtrahieren 1, nur auf eine andere Art und Weise. 
Zu Beginn wird \ECX, in dem der Pointer auf \emph{str} gespeichert ist, invertiert
und vom Ergebnis wird 1 abgezogen.

Mit anderen Worten, am Ende der Funktion, direkt nach dem Schleifenkörper,
werden die folgenden Befehle ausgeführt:

\begin{lstlisting}[style=customc]
ecx=str;
eax=eos;
ecx=(-ecx)-1; 
eax=eax+ecx
return eax
\end{lstlisting}

\dots~und das ist äquivalent zu:

\begin{lstlisting}[style=customc]
ecx=str;
eax=eos;
eax=eax-ecx;
eax=eax-1;
return eax
\end{lstlisting}

Warum GCC entschieden hat, dass das eine besser ist als das andere? Schwer zu
sagen.
Möglicherweise sind aber beide Variante gleichermaßen effizient.

\subsubsection{x64}
\label{subsec:popcnt}
Verändern wir das Beispiel ein wenig um es auf 64 Bit zu erweitern:

\lstinputlisting[label=popcnt_x64_example,style=customc]{patterns/14_bitfields/4_popcnt/shifts64.c}

\myparagraph{\NonOptimizing GCC 4.8.2}

So weit, so einfach.

\lstinputlisting[caption=\NonOptimizing GCC 4.8.2,style=customasmx86]{patterns/14_bitfields/4_popcnt/shifts64_GCC_O0_DE.s}

\myparagraph{\Optimizing GCC 4.8.2}

\lstinputlisting[caption=\Optimizing GCC 4.8.2,numbers=left,label=shifts64_GCC_O3,style=customasmx86]{patterns/14_bitfields/4_popcnt/shifts64_GCC_O3_DE.s}
Dieser Code ist kürzer, birgt aber eine Besonderheit.

In allen bisher betrachteten Beispieln haben wir den Wert von \q{rt} nach dem
Vergleich mit einem speziellen Bit erhöht, aber dieser Code erhöht \q{rt} vorher
(Zeile 6) und schreibt den neuen Wert in das Register \EDX.
Dadurch überträgt der Befehl \CMOVNE\footnote{Conditional MOVe if Not Equal}
(der ein Synonym für \CMOVNZ\footnote{Conditional MOVe if Not Zero} ist) den
neuen Wert von \q{rt} durch Verschieben des Wertes in \EDX (vorgeschlagener
Wert von \q{rt}) nach \EAX (\q{aktueller Wert von rt}). Der in \EAX befindliche
Wert wird schließlich zurückgegeben.

Deshalb wird die Erhöhung des Zählers in jedem Durchlauf der Schleife
durchgeführt, d.h. 64 mal, ohne dass eine Abhängigkeit vom Eingabewert
besteht.

Der Vorteil dieses Code ist, dass er nur einen bedingten Sprung enthält (am
Ende der Schleife) anstatt zwei Sprüngen (Überspringen des Erhöhens von \q{rt}
und Ende der Schleife). 
Der Code ist somit auf modernen CPUs mit Branch Pedictors möglicherweise
schneller:\myref{branch_predictors}.

\label{FATRET}
\myindex{x86!\Instructions!FATRET}
Der letzte Befehl hier ist \INS{REP RET} (Opcode \TT{F3 C3}), der von MSVC auch
\INS{FATRET} genannt wird.
Hierbei handelt es sich um eine optimierte Version von \RET, die von ARM
bevorzugt am Ende der Funktion verwendet wird, wenn \RET direkt nach einem
bedingten Sprung folg:.
\InSqBrackets{\AMDOptimization p.15}
\footnote{Mehr Informationen dazu: \url{http://repzret.org/p/repzret/}}.

\myparagraph{\Optimizing MSVC 2010}

\lstinputlisting[caption=\Optimizing MSVC 2010,style=customasmx86]{patterns/14_bitfields/4_popcnt/MSVC_2010_x64_Ox_DE.asm}

\myindex{x86!\Instructions!ROL}
Hier wird der Befehl \ROL anstelle von \SHL verwendet, welches einer
\q{Linksrotation} anstatt einer \q{Linksverschiebung} entspricht.
In diesem Beispiel entspricht \ROL einem \TT{SHL}.

Für mehr Informationen zu Rotationsbefehlen siehe: \myref{ROL_ROR}. 

\Reg{8} zählt hier von 64 auf 0 herunter. 
Dies entspricht dem invertierten $i$.

Hier ist eine Tabelle einiger Register während der Ausführung des Programms:

\begin{center}
\begin{tabular}{ | l | l | }
\hline
\HeaderColor RDX & \HeaderColor R8 \\
\hline
0x0000000000000001 & 64 \\
\hline
0x0000000000000002 & 63 \\
\hline
0x0000000000000004 & 62 \\
\hline
0x0000000000000008 & 61 \\
\hline
... & ... \\
\hline
0x4000000000000000 & 2 \\
\hline
0x8000000000000000 & 1 \\
\hline
\end{tabular}
\end{center}

\myindex{x86!\Instructions!FATRET}
Am Ende finden wir den Befehl \INS{FATRET}, der hier schon erklärt
wurde:\myref{FATRET}.

\myparagraph{\Optimizing MSVC 2012}

\lstinputlisting[caption=\Optimizing MSVC 2012,style=customasmx86]{patterns/14_bitfields/4_popcnt/MSVC_2012_x64_Ox_DE.asm}

\myindex{\CompilerAnomaly}
\label{MSVC2012_anomaly}
Der optimierende MSVC 2012 erzeugt fast den gleichen Code wie MSVC 2012,
generiert aber aus irgendeinem Grund zwei identischen Rümpfe für die Schleifen
und die Schleife zählt nun bis 32 anstatt 64.

Ehrlich gesagt, kann man nicht genau erklären warum. Es könnte sich um einen
Optimierungstrick handeln. Vielleicht ist es für den Rumpf der Schleife besser
ein wenig länger zu sein.

Trotzdem ist solcher Code relevant um zu zeigen, dass der Output des Compilers
manchmal sehr merkwürdig und unlogisch sein kann und dennoch tadellos
funktioniert.

\subsubsection{ARM: \OptimizingKeilVI (\ARMMode)}
\myindex{\CLanguageElements!switch}

\lstinputlisting[style=customasmARM]{patterns/08_switch/1_few/few_ARM_ARM_O3.asm}
Auch hier können wir bei Untersuchung des Code nicht sagen, ob im Quellcode ein switch() oder eine Folge von
if()-Ausdrücken vorliegt.


\myindex{ARM!\Instructions!ADRcc}
Wir finden hier Befehle mit Prädikaten wieder (wie \ADREQ (\emph{Equal})), welcher nur dann ausgeführt wird, wenn $R0=0$
und dann die Adresse des Strings IT{<<zero\textbackslash{}n>>} nach \Reg{0} lädt.

\myindex{ARM!\Instructions!BEQ}
Der folgende \ac{BEQ} Befehl übergibt den Control Flow an \TT{loc\_170}, falls $R0=0$.
Ein aufmerksamer Leser könnte sich fragen, ob \ac{BEQ} korrekt ausgelöst wird, da \ADREQ das \Reg{0} Register bereits
mit einem anderen Wert befüllt hat.
Es wird korrekt ausgelöst, da \ac{BEQ} die Flags, die vom \CMP Befehl gesetzt wurden, prüft und \ADREQ die Flags nicht
verändert.

Die übrigen Befehle kennen wir bereits.
Es gibt nur einen Aufruf von \printf am Ende und wir haben diesen Trick bereits hier
kennengelernt~(\myref{ARM_B_to_printf}). Am Ende gibt es drei Wege zur Ausführung von \printf.

\myindex{ARM!\Instructions!ADRcc}
\myindex{ARM!\Instructions!CMP}
Der letzte Befehl, \TT{CMP R0, \#2}, wird benötigt, um zu prüfen, ob $a=2$.
Wenn dies nicht der Fall ist, lädt \ADRNE einen Pointer auf den String \emph{<<something unknown \textbackslash{}n>>} nach
\Reg{0}, da $a$ bereits auf Gleichheit mit 0 oder 1 geprüft wurde und wir können sicher sein, dass die Variable $a$ an
dieser Stelle keinen dieser beiden Werte enthält.
Falls $R0=2$ ist, lädt \ADREQ einen Pointer auf den String \emph{<<two\textbackslash{}n>>} nach \Reg{0}. 

\subsubsection{ARM: \OptimizingKeilVI (\ThumbMode)}

\lstinputlisting[style=customasmARM]{patterns/08_switch/1_few/few_ARM_thumb_O3.asm}

% FIXME а каким можно? к каким нельзя? \myref{} ->
Wie bereits erwähnt ist es bei den meisten Befehlen im Thumb mode nicht möglich Prädikate für Bedingungen hinzuzufügen,
sodass der Thumb-Code hier dem leicht verständlichen x86 \ac{CISC}-style Code sehr ähnlich ist.

\subsubsection{ARM64: \NonOptimizing GCC (Linaro) 4.9}

\lstinputlisting[style=customasmARM]{patterns/08_switch/1_few/ARM64_GCC_O0_DE.lst}
Der Datentyp des Eingabewertes ist \Tint, deshalb wird das Register \RegW{0} anstatt des \RegX{0} Registers verwendet,
um ihn aufzunehmen.

Die Pointer auf die Strings werden an \puts mit einem \INS{ADRP}/\INS{ADD} Befehlspaar übergeben, genauso wie wir es im
\q{\HelloWorldSectionName} Beispiel gezeigt haben:~\myref{pointers_ADRP_and_ADD}.

\subsubsection{ARM64: \Optimizing GCC (Linaro) 4.9}

\lstinputlisting[style=customasmARM]{patterns/08_switch/1_few/ARM64_GCC_O3_DE.lst}
Ein besser optimiertes Stück Code. 
Der Befehl \TT{CBZ} (\emph{Compare and Branch on Zero}) springt, falls \RegW{0} gleich null ist.
Es gibt auch einen direkten Sprung zu \puts anstelle eines Aufrufs, so wie bereits hier
erklärt:~\myref{JMP_instead_of_RET}.


\subsubsection{MIPS}

\lstinputlisting[caption=\Optimizing GCC 4.4.5 (IDA),style=customasmMIPS]{patterns/12_FPU/2_passing_floats/MIPS_O3_IDA_DE.lst}
Und wieder sehen wir hier, dass der Befehl \INS{LUI} einen 32-Bit-Teil einer
\Tdouble Zahl nach \$V0 lädt.
Und wiederum ist es schwer nachzuvollziehen warum dies geschieht.

\myindex{MIPS!\Instructions!MFC1}
Der für uns neue Befehl an dieser Stelle ist \INS{MFC1}(\q{Move From Coprocessor
1}). Die Nummer des FPU-Koprozessors ist 1, daher die \q{1} im Namen des
Befehls. 
Dieser Befehl überträgt Werte aus den Registern des Koprozessors in die Register
der CPU (\ac{GPR}).
Auf diese Weise wird das Ergebnis von \TT{pow()} schließlich in die Register
\$A3 und \$A2 verschoben und \printf übernimmt einen 64-Bit-Wert von doppelter
Genauigkeit aus diesem Registerpaar.

