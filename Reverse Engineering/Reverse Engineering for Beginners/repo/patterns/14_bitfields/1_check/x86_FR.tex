\subsubsection{x86}
\myindex{Windows!Win32}

Exemple avec l'API win32:

\begin{lstlisting}[style=customc]
	HANDLE fh;

	fh=CreateFile ("file", GENERIC_WRITE | GENERIC_READ, FILE_SHARE_READ, NULL, OPEN_ALWAYS, FILE_ATTRIBUTE_NORMAL, NULL);
\end{lstlisting}

Nous obtenons (MSVC 2010):

\begin{lstlisting}[caption=MSVC 2010,style=customasmx86]
	push	0
	push	128		; 00000080H
	push	4
	push	0
	push	1
	push	-1073741824	; c0000000H
	push	OFFSET $SG78813
	call	DWORD PTR __imp__CreateFileA@28
	mov	DWORD PTR _fh$[ebp], eax
\end{lstlisting}

Regardons dans WinNT.h:

\begin{lstlisting}[caption=WinNT.h,style=customc]
#define GENERIC_READ                     (0x80000000L)
#define GENERIC_WRITE                    (0x40000000L)
#define GENERIC_EXECUTE                  (0x20000000L)
#define GENERIC_ALL                      (0x10000000L)
\end{lstlisting}

Tout est clair,
GENERIC\_READ | GENERIC\_WRITE = 0x80000000 | 0x40000000 = 0xC0000000,
et c'est la valeur utilisée comme second argument pour la fonction
\TT{CreateFile()}\footnote{\href{http://msdn.microsoft.com/en-us/library/aa363858(VS.85).aspx}{msdn.microsoft.com/en-us/library/aa363858(VS.85).aspx}}.

Comment \TT{CreateFile()} va tester ces flags?
\myindex{Windows!KERNEL32.DLL}

Si nous regardons dans KERNEL32.DLL de Windows XP SP3 x86, nous trouverons ce morceau
de code dans \TT{CreateFileW}:

\begin{lstlisting}[caption=KERNEL32.DLL (Windows XP SP3 x86),style=customasmx86]
.text:7C83D429     test    byte ptr [ebp+dwDesiredAccess+3], 40h
.text:7C83D42D     mov     [ebp+var_8], 1
.text:7C83D434     jz      short loc_7C83D417
.text:7C83D436     jmp     loc_7C810817
\end{lstlisting}

\myindex{x86!\Instructions!TEST}

Ici nous voyons l'instruction \TEST, toutefois elle n'utilise pas complètement le
second argument, mais seulement l'octet le plus significatif et le teste avec le
flag \TT{0x40} (ce qui implique le flag \TT{GENERIC\_WRITE} ici).
\myindex{x86!\Instructions!AND}

\TEST est essentiellement la même chose que \AND, mais sans sauver le résultat
(rappelez vous le cas de \CMP qui est la même chose que \SUB, mais sans sauver le
résultat~(\myref{CMPandSUB})).

La logique de ce bout de code est la suivante:

\begin{lstlisting}[style=customc]
if ((dwDesiredAccess&0x40000000) == 0) goto loc_7C83D417
\end{lstlisting}

\myindex{x86!\Instructions!AND}
\myindex{x86!\Registers!ZF}

Si l'instruction \AND laisse ce bit, le flag \ZF sera mis à zéro et le saut conditionnel
\JZ ne sera pas effectué.
Le saut conditionnel est effectué uniquement su la bit \TT{0x40000000} est absent
dans la variable \TT{dwDesiredAccess}~---auquel cas le résultat du \AND est 0, \ZF
est mis à 1 et le saut conditionnel est effectué.

Essayons avec GCC 4.4.1 et Linux:

\begin{lstlisting}[style=customc]
#include <stdio.h>
#include <fcntl.h>

void main()
{
	int handle;

	handle=open ("file", O_RDWR | O_CREAT);
};
\end{lstlisting}

Nous obtenons:

\lstinputlisting[caption=GCC 4.4.1,style=customasmx86]{patterns/14_bitfields/1_check/check.asm}

\myindex{Linux!libc.so.6}
\myindex{syscall}

Si nous regardons dans la fonction \TT{open()} de la bibliothèque \TT{libc.so.6},
c'est seulement un appel système:

\lstinputlisting[caption=open() (libc.so.6),style=customasmx86]{patterns/14_bitfields/1_check/tmp1.asm}

Donc, le champ de bits pour \TT{open()} est apparemment testé quelque part dans le
noyau Linux.

Bien sûr, il est facile de télécharger le code source de la Glibc et du noyau Linux,
mais nous voulons comprendre ce qui se passe sans cela.

Donc, à partir de Linux 2.6, lorsque l'appel système \TT{sys\_open} est appelé, le
contrôle passe finalement à \TT{do\_sys\_open}, et à partir de là---à la fonction
\TT{do\_filp\_open()} (elle est située ici \TT{fs/namei.c} dans l'arborescence des
sources du noyau).

\newcommand{\URLREGPARM}{\href{http://www.ohse.de/uwe/articles/gcc-attributes.html\#func-regparm}{ohse.de/uwe/articles/gcc-attributes.html\#func-regparm}}

\myindex{fastcall}
\label{regparm}
N.B. Outre le passage des arguments par la pile, il y a aussi une méthode consistant
à passer certains d'entre eux par des registres. Ceci est aussi appelé fastcall~(\myref{fastcall}).
Ceci fonctionne plus vite puisque le CPU ne doit pas faire d'accès à la pile en mémoire
pour lire la valeur des arguments.
GCC a l'option \emph{regparm}\footnote{\URLREGPARM}, avec laquelle il est possible
de définir le nombre d'arguments qui peuvent être passés par des registres.

\newcommand{\URLKERNELNEWB}{\href{http://kernelnewbies.org/Linux_2_6_20\#head-042c62f290834eb1fe0a1942bbf5bb9a4accbc8f}{kernelnewbies.org/Linux\_2\_6\_20\#head-042c62f290834eb1fe0a1942bbf5bb9a4accbc8f}}
\newcommand{\CALLINGHFILE}{arch/x86/include/asm/calling.h}

Le noyau Linux 2.6 est compilé avec l'option \TT{-mregparm=3}~\footnote{\URLKERNELNEWB}
\footnote{Voir aussi le fichier \TT{\CALLINGHFILE} dans l'arborescence du noyau}.

Cela signifie que les 3 premiers arguments sont passés par les registres \EAX, \EDX
et \ECX, et le reste via la pile.
Bien sûr, si le nombre d'arguments est moins que 3, seule une partie de ces registres
seront utilisés.

Donc, téléchargeons le noyau Linux 2.6.31, compilons-le dans Ubuntu: \TT{make vmlinux},
ouvrons-le dans \IDA, et cherchons la fonction \TT{do\_filp\_open()}. Au début, nous
voyons (les commentaires sont les miens):

\lstinputlisting[caption=do\_filp\_open() (noyau Linux kernel 2.6.31),style=customasmx86]{patterns/14_bitfields/1_check/check2_FR.asm}

GCC sauve les valeurs des 3 premiers arguments dans la pile locale.
Si cela n'était pas fait, le compilateur ne toucherait pas ces registres, et ça serait
un environnement trop étroit pour l'\glslink{register allocator}{allocateur de registres}
du compilateur.

Cherchons ce morceau de code:

\lstinputlisting[caption=do\_filp\_open() (noyau Linux 2.6.31),style=customasmx86]{patterns/14_bitfields/1_check/check3.asm}

\TT{0x40}---c'est ce à quoi est égale la macro \TT{O\_CREAT}.
Le bit \TT{0x40} de \TT{open\_flag} est testé, et si il est à 1, le saut de l'instruction
\JNZ suivante est effectué.
