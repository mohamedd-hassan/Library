\subsubsection{ARM}

\myindex{Linux}

Le bit \TT{O\_CREAT} est testé différemment dans le noyau Linux 3.8.0.

\begin{lstlisting}[caption=noyau Linux 3.8.0,style=customc]
struct file *do_filp_open(int dfd, struct filename *pathname,
		const struct open_flags *op)
{
...
	filp = path_openat(dfd, pathname, &nd, op, flags | LOOKUP_RCU);
...
}

static struct file *path_openat(int dfd, struct filename *pathname,
		struct nameidata *nd, const struct open_flags *op, int flags)
{
...
	error = do_last(nd, &path, file, op, &opened, pathname);
...
}


static int do_last(struct nameidata *nd, struct path *path,
		   struct file *file, const struct open_flags *op,
		   int *opened, struct filename *name)
{
...
	if (!(open_flag & O_CREAT)) {
    ...
		error = lookup_fast(nd, path, &inode);
    ...
	} else {
    ...
		error = complete_walk(nd);
        }
...
}
\end{lstlisting}

Voici à quoi ressemble le noyau compilé pour le mode ARM dans \IDA:

\lstinputlisting[caption=do\_last() dans vmlinux (IDA),style=customasmARM]{patterns/14_bitfields/1_check/IDA.lst}

\myindex{ARM!\Instructions!TST}
\myindex{x86!\Instructions!TEST}
\TT{TST} est analogue à l'instruction \TEST en x86.
Nous pouvons \q{pointer} visuellement ce morceau de code grâce au fait que la fonction
\TT{lookup\_fast()} doit être exécutée dans un cas et \TT{complete\_walk()} dans
l'autre.
Ceci correspond au code source de la fonction \TT{do\_last()}.
La macro \TT{O\_CREAT} vaut \TT{0x40} ici aussi.

