% TODO: add ROL/ROR
\subsection{\Conclusion{}}

\myindex{x86!\Instructions!SHR}
\myindex{x86!\Instructions!SHL}
\myindex{x86!\Instructions!SAR}

Semblables aux opérateurs de décalage de \CCpp \TT{$\ll$} et \TT{$\gg$}, les instructions
de décalage en x86 sont \SHR/\SHL (pour les valeurs non-signées) et \SAR/\SHL (pour
les valeurs signées).

\myindex{ARM!\Instructions!LSR}
\myindex{ARM!\Instructions!LSL}
\myindex{ARM!\Instructions!ASR}

Les instructions de décalages en ARM sont \LSR/\LSL (pour les valeurs non-signées)
et \ASR/\LSL (pour les valeurs signées).

Il est aussi possible d'ajouter un suffixe de décalage à certaines instructions (qui
sont appelées \q{data processing instructions/instructions de traitement de données}).
% FIXME: which instructions?

\subsubsection{Tester un bit spécifique (connu à l'étape de compilation)}

Tester si le bit 0b1000000 (0x40) est présent dans la valeur du registre:

\lstinputlisting[caption=\CCpp,style=customc]{patterns/14_bitfields/c_snippet0.c}

\lstinputlisting[caption=x86,style=customasmx86]{patterns/14_bitfields/TEST_JNZ_FR.lst}

\lstinputlisting[caption=x86,style=customasmx86]{patterns/14_bitfields/TEST_JZ_FR.lst}

\lstinputlisting[caption=ARM (\ARMMode),style=customasmARM]{patterns/14_bitfields/TST_BNE_FR.lst}

\myindex{x86!\Instructions!AND}
\myindex{x86!\Instructions!TEST}

Parfois, \AND est utilisé au lieu de \TEST, mais les flags qui sont mis sont les
même.

\subsubsection{Tester un bit spécifique (spécifié lors de l'exécution)}

Ceci est effectué en général par ce bout de code \CCpp (décaler la valeur de $n$
bits vers la droite, puis couper le plus petit bit):

\lstinputlisting[caption=\CCpp,style=customc]{patterns/14_bitfields/c_snippet1.c}

Ceci est en général implémenté en code x86 avec:

\begin{lstlisting}[caption=x86,style=customasmx86]
; REG=input\_value
; CL=n
SHR REG, CL
AND REG, 1
\end{lstlisting}

Ou (décaler 1 bit $n$ fois à gauche, isoler ce bit dans la valeur entrée et tester
si ce n'est pas zéro):

\lstinputlisting[caption=\CCpp,style=customc]{patterns/14_bitfields/c_snippet2.c}

Ceci est en général implémenté en code x86 avec:

\begin{lstlisting}[caption=x86,style=customasmx86]
; CL=n
MOV REG, 1
SHL REG, CL
AND input_value, REG
\end{lstlisting}

\subsubsection{Mettre à 1 un bit spécifique (connu à l'étape de compilation)}

\begin{lstlisting}[caption=\CCpp]
value=value|0x40;
\end{lstlisting}

\begin{lstlisting}[caption=x86,style=customasmx86]
OR REG, 40h
\end{lstlisting}

\begin{lstlisting}[caption=ARM (\ARMMode) and ARM64,style=customasmARM]
ORR R0, R0, #0x40
\end{lstlisting}

\subsubsection{Mettre à 1 un bit spécifique (spécifié lors de l'exécution)}

\lstinputlisting[caption=\CCpp,style=customc]{patterns/14_bitfields/c_snippet3.c}

Ceci est en général implémenté en code x86 avec:

\begin{lstlisting}[caption=x86,style=customasmx86]
; CL=n
MOV REG, 1
SHL REG, CL
OR input_value, REG
\end{lstlisting}

\subsubsection{Mettre à 0 un bit spécifique (connu à l'étape de compilation)}

Il suffit d'effectuer l'opération \AND sur la valeur inversée:

\begin{lstlisting}[caption=\CCpp,style=customc]
value=value&(~0x40);
\end{lstlisting}

\begin{lstlisting}[caption=x86,style=customasmx86]
AND REG, 0FFFFFFBFh
\end{lstlisting}

\begin{lstlisting}[caption=x64,style=customasmx86]
AND REG, 0FFFFFFFFFFFFFFBFh
\end{lstlisting}

Ceci laisse tous les bits qui sont à 1 inchangés excepté un.

\myindex{ARM!\Instructions!BIC}

ARM en mode ARM a l'instruction \BIC, qui fonctionne comme la paire d'instructions:
\NOT+\AND:

\begin{lstlisting}[caption=ARM (\ARMMode),style=customasmARM]
BIC R0, R0, #0x40
\end{lstlisting}

\subsubsection{
Mettre à 0 un bit spécifique (spécifié lors de l'exécution)}

\lstinputlisting[caption=\CCpp,style=customc]{patterns/14_bitfields/c_snippet4.c}

\begin{lstlisting}[caption=x86,style=customasmx86]
; CL=n
MOV REG, 1
SHL REG, CL
NOT REG
AND input_value, REG
\end{lstlisting}
