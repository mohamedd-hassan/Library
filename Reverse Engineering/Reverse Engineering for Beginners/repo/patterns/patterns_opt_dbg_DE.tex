\mysection{Die Methode}

Als der Autor dieses Buches zunächst C und später \Cpp lernte, schrieb er kleine Codestücke, kompilierte sie und
untersuchte anschließend den Assembler-Code des Compilers. Dies machte es sehr einfach zu verstehen, was in dem Code
den er geschrieben hatte passierte.
\footnote{Tatsächlich tut er dies immer noch wenn er einen bestimmten Code-Teil nicht versteht.} 
% TBT Recent example from 2019: \verb|p += p+(i&1)+2;| from the ``SAT0W'' SAT-solver by D.Knuth.}.
Er hat dies so oft getan, dass der Zusammenhang zwischen dem \CCpp-Code und dem was der Compiler daraus macht tief in
seinem Verstand verankert ist. % Not very elegant translation to be honest... 
So ist es einfach sich schnell einen Überblick über das Aussehen und die Funktion einer C-Quelle zu verschaffen. 
Vielleicht ist diese Vorgehensweise auch für andere hilfreich.

%There are a lot of examples for both x86/x64 and ARM.
%Those who already familiar with one of architectures, may freely skim over pages.

Übrigens gibt es eine hervorragende Webseite auf der dasselbe mit verschiedenen
Compilern getan werden kann, anstatt diese zu installieren.
Auch diese kann genutzt werden: \url{https://godbolt.org/}.

\section*{\Exercises}

Als der Autor dieses Buches Assembler erlernte, hat er oft kleine C-Funktionen kompiliert und anschließend nach und nach in
Assembler nachprogrammiert um den Code so klein wie möglich zu machen.
Dies ist in einer Anwendung in der Praxis heutzutage vielleicht nicht mehr sinnvoll, weil moderne Compiler in der Regel weitaus
besser optimieren können als ein Mensch. Dennoch ist es ein guter Weg um Assembler besser zu verstehen. Versuchen Sie ruhig
einmal einen Assembler-Quelltext aus dem Buch bei gleicher Funktionalität kürzer zu schreiben.
Vergessen Sie aber nicht Dinge die Sie programmiert haben zu testen.

% rewrote to show that debug\release and optimisations levels are orthogonal concepts.
\section*{Optimierungsstufen und Debug-Informationen}

Quellcode kann von verschiedenen Compilern mit verschiedenen Optimierungsstufen übersetzt werden.
Üblicherweise hat ein Compiler drei solcher Stufen, wobei Stufe Null eine deaktivierte Optimierung bedeutet.
Die Optimierung kann sich ebenso auf die Code-Größe als auch auf die Ausführgeschwindigkeit beziehen.
Ein Nicht-optimierender Compiler ist schneller und erstellt einfacher zu verstehenden (aber auch längeren) Code.
Demgegenüber ist ein optimierender Compiler langsamer und versucht Code zu erstellen, dessen Ausführgeschwindigkeit
höher, aber nicht notwendigerweise kompakter, ist.
Zusätzlich zu diesen Optimierungsmöglichkeiten kann ein Compiler Informationen in den Code einfügen, die das spätere
Debuggen vereinfachen.
Eine wichtige Eigenschaft des Debug-Codes ist, dass er gegebenenfalls Links zwischen den Zeilen des Quellcodes und
den entsprechenden Maschinen-Code-Adressen enthält.
Optimierende Compiler hingegen tendieren dazu Code zu erzeugen, der ganze Zeilen des Quellcodes wegoptimiert, die dann
dementsprechend im Maschinencode nicht auftauchen.
Ein Reverse Engineer kann beiden Varianten begegnen, einfach, weil einige Software-Entwickler die Optimierung des Compilers
nutzen und andere nicht. Aufgrund dieser Tatsache werden Sie in diesem Buch auch Beispielcode für optimierten und
nichtoptimierten Compiler-Code finden.
