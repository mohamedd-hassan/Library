\subsubsection{MSVC: x86-64}

\myindex{x86-64}
Przyjrzyjmy się wynikom kompilacji 64-bitowego MSVC:

\lstinputlisting[caption=MSVC 2012 x64,style=customasmx86]{patterns/01_helloworld/MSVC_x64.asm}

\myindex{fastcall}

W x86-64 wszystkie rejestry zostały rozszerzone do 64 bitów i ich nazwy zyskały prefiks \TT{R-}.
Żeby jak najrzadziej korzystać ze stosu (inaczej mówiąc, jak najmniej korzystać z pamięci cache i pamięci zewnętrznej), istnieje popularna metoda przekazywania argumentów funkcji przez rejestry (\emph{fastcall}) \myref{fastcall}.
Tzn. część argumentów funkcji jest przekazywana przez rejestry a część --- przez stos.
W Win64 pierwsze 4 argumenty funkcji są przekazywane przez rejestry \RCX, \RDX, \Reg{8} i \Reg{9}.
Widać to w powyższym przykładzie: wskaźnik na argument funkcji \printf (łańcuch znaków) teraz jest przekazywany nie przez stos, a przez rejestr \RCX.
Wskaźniki są teraz 64-bitowe, więc są przekazywane przez przez 64-bitowe rejestry (mające prefiks \TT{R-}).
Ale dla wstecznej kompatybilności można adresować również młodsze 32 bity rejestrów poprzez prefiks \TT{E-}.
W ten oto sposób wygląda rejestr \RAX/\EAX/\AX/\AL w x86-64:

\RegTableOne{RAX}{EAX}{AX}{AH}{AL}

Funkcja \main zwraca wartość typu \Tint, który w \CCpp, dla większej kompatybilności,
pozostał 32-bitowy. Właśnie dlatego na końcu funkcji \main zeruje się nie \RAX, a \EAX, czyli 32-bitową część rejestru.
Dodatkowo na stosie lokalnym jest zarezerwowanych 40 bajtów.
Jest to tzw. \q{shadow space}, który będzie omawiany później: \myref{shadow_space}.

