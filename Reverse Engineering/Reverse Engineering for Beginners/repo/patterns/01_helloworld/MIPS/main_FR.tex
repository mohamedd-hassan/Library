\subsection{MIPS}

\subsubsection{Un mot à propos du \q{pointeur global}}
\label{MIPS_GP}

\myindex{MIPS!\GlobalPointer}

Un concept MIPS important est le \q{pointeur global}.
Comme nous le savons déjà, chaque instruction MIPS a une taille de 32-bit, donc
il est impossible d'avoir une adresse 32-bit dans une instruction: il faut pour
cela utiliser une paire.
(comme le fait GCC dans notre exemple pour le chargement de l'adresse de la chaîne
de texte).
Il est possible, toutefois, de charger des données depuis une adresse dans l'interval
$register-32768...register+32767$ en utilisant une seule instruction (car un offset
signé de 16 bits peut être encodé dans une seule instruction).
Nous pouvons alors allouer un registre dans ce but et dédier un bloc de 64KiB
pour les données les plus utilisées.
Ce registre dédié est appelé un \q{pointeur global} et il pointe au milieu du
bloc de 64 KiB.
Ce bloc contient en général les variables globales et les adresses des fonctions
importées, comme \printf, car les développeurs de GCC ont décidé qu'obtenir
l'adresse d'une fonction devait se faire en une instruction au lieu de deux.
Dans un fichier ELF ce bloc de 64KiB se trouve en partie dans une section .sbss
(\q{small \ac{BSS}}) pour les données non initialisées et .sdata (\q{small data})
pour celles initialisées.
Cela implique que le programmeur peut choisir quelle donnée il/elle souhaite rendre
accessible rapidement et doit les stocker dans .sdata/.sbss.
Certains programmeurs old-school peuvent se souvenir du modèle de mémoire MS-DOS
\myref{8086_memory_model} ou des gestionnaires de mémoire MS-DOS comme XMS/EMS
où toute la mémoire était divisée en bloc de 64KiB.

\myindex{PowerPC}

Ce concept n'est pas restreint à MIPS. Au moins les PowerPC utilisent aussi cette
technique.

\subsubsection{GCC \Optimizing}

Considérons l'exemple suivant, qui illustre le concept de \q{pointeur global}.

\lstinputlisting[caption=GCC 4.4.5 \Optimizing (\assemblyOutput),numbers=left,style=customasmMIPS]{patterns/01_helloworld/MIPS/hw_O3_FR.s}

Comme on le voit, le registre \$GP est défini dans le prologue de la fonction
pour pointer au milieu de ce bloc.
Le registre \ac{RA} est sauvé sur la pile locale.
\puts est utilisé ici au lieu de \printf.
\myindex{MIPS!\Instructions!LW}
L'adresse de la fonction \puts est chargée dans \$25 en utilisant l'instruction \INS{LW} (\q{Load Word}).
\myindex{MIPS!\Instructions!LUI}
\myindex{MIPS!\Instructions!ADDIU}
Ensuite l'adresse de la chaîne de texte est chargée dans \$4 avec la paire
d'instructions \INS{LUI} ((\q{Load Upper Immediate}) et \INS{ADDIU}
(\q{Add Immediate Unsigned Word}).
\INS{LUI} défini les 16 bits de poids fort du registre (d'où le mot \q{upper}
dans le nom de l'instruction) et \INS{ADDIU} ajoute les 16 bits de poids faible
de l'adresse.

\INS{ADDIU} suit \INS{JALR} (vous n'avez pas déjà oublié le \emph{slot de
délai de branchement} ?).
Le registre \$4 est aussi appelé \$A0, qui est utilisé pour passer le premier
argument d'une fonction \footnote{La table des registres MIPS est disponible en
appendice \myref{MIPS_registers_ref}}.

\myindex{MIPS!\Instructions!JALR}

\INS{JALR} (\q{Jump and Link Register}) saute à l'adresse stockée dans le registre
\$25 (adresse de \puts) en sauvant l'adresse de la prochaine instruction (LW)
dans \ac{RA}.
C'est très similaire à ARM.
Oh, encore une chose importante, l'adresse sauvée dans \ac{RA} n'est pas
l'adresse de l'instruction suivante (car c'est celle du \emph{slot de délai} et
elle est exécutée avant l'instruction de saut), mais l'adresse de l'instruction
après la suivante (après le \emph{slot de délai}).
Par conséquent, $PC + 8$ est écrit dans \ac{RA} pendant l'exécution de \TT{JALR},
dans notre cas, c'est l'adresse de l'instruction \INS{LW} après \INS{ADDIU}.

\INS{LW} (\q{Load Word}) à la ligne 20 restaure \ac{RA} depuis la pile locale
(cette instruction fait partie de l'épilogue de la fonction).

\myindex{MIPS!\Pseudoinstructions!MOVE}

\INS{MOVE} à la ligne 22 copie la valeur du registre \$0 (\$ZERO) dans \$2 (\$V0).
\label{MIPS_zero_register}

MIPS a un registre \emph{constant}, qui contient toujours zéro.
Apparemment, les développeurs de MIPS avaient à l'esprit que zéro est la constante
la plus utilisée en programmation, utilisons donc le registre \$0 à chaque fois
que zéro est requis.

Un autre fait intéressant est qu'il manque en MIPS une instruction qui transfère
des données entre des registres.
En fait,  \TT{MOVE DST, SRC} est \TT{ADD DST, SRC, \$ZERO} ($DST=SRC+0$), qui
fait la même chose.
Manifestement, les développeurs de MIPS voulaient une table des opcodes compacte.
Cela ne signifie pas qu'il y a une addition à chaque instruction \INS{MOVE}.
Très probablement, le \ac{CPU} optimise ces pseudo-instructions et l'\ac{ALU} n'est
jamais utilisé.

\myindex{MIPS!\Instructions!J}

\INS{J} à la ligne 24 saute à l'adresse dans \ac{RA}, qui effectue effectivement
un retour de la fonction.
\INS{ADDIU} après \INS{J} est en fait exécutée avant \INS{J} (vous vous rappeler
du \emph{slot de délai de branchement}?) et fait partie de l'épilogue de la fonction.
Voici un listing généré par \IDA. Chaque registre a son propre pseudo nom:

\lstinputlisting[caption=GCC 4.4.5 \Optimizing (\IDA),numbers=left,style=customasmMIPS]{patterns/01_helloworld/MIPS/hw_O3_IDA_FR.lst}

L'instruction à la ligne 15 sauve la valeur de GP sur la pile locale, et cette
instruction manque mystérieusement dans le listing de sortie de GCC, peut-être
une erreur de GCC
\footnote{Apparemment, les fonctions générant les listings ne sont pas si critique
pour les utilisateurs de GCC, donc des erreurs peuvent toujours subsister.}.
La valeur de GP doit effectivement être sauvée, car chaque fonction utilise sa
propre fenêtre de 64KiB.
Le registre contenant l'adresse de \puts est appelé \$T9, car les registres
préfixés avec T- sont appelés \q{temporaires} et leur contenu ne doit pas être
préservé. 

\subsubsection{GCC \NonOptimizing}

GCC \NonOptimizing est plus verbeux.

\lstinputlisting[caption=GCC 4.4.5 \NonOptimizing (\assemblyOutput),numbers=left,style=customasmMIPS]{patterns/01_helloworld/MIPS/hw_O0_FR.s}

Nous voyons ici que le registre FP est utilisé comme un pointeur sur la pile.
Nous voyons aussi 3 \ac{NOP}s.
Le second et le troisième suivent une instruction de branchement.
Peut-être que le compilateur GCC ajoute toujours des \ac{NOP}s (à cause du
\emph{slot de retard de branchement}) après les instructions de branchement, et,
si l'optimisation est demandée, il essaye alors de les éliminer.
Donc, dans ce cas, ils sont laissés en place.

Voici le listing \IDA:

\lstinputlisting[caption=GCC 4.4.5 \NonOptimizing (\IDA),numbers=left,style=customasmMIPS]{patterns/01_helloworld/MIPS/hw_O0_IDA_FR.lst}

\myindex{MIPS!\Pseudoinstructions!LA}

Intéressant, \IDA a reconnu les instructions \INS{LUI}/\INS{ADDIU} et les a agrégées
en une pseudo instruction \INS{LA} (\q{Load Address}) à la ligne 15.
Nous pouvons voir que cette pseudo instruction a une taille de 8 octets!
C'est une pseudo instruction (ou \emph{macro}) car ce n'est pas une instruction MIPS
réelle, mais plutôt un nom pratique pour une paire d'instructions.

\myindex{MIPS!\Pseudoinstructions!NOP}
\myindex{MIPS!\Instructions!OR}

Une autre chose est qu'\IDA ne reconnaît pas les instructions \ac{NOP}, donc ici
elles se trouvent aux lignes 22, 26 et 41.
C'est \TT{OR \$AT, \$ZERO}.
Essentiellement, cette instruction applique l'opération OR au contenu du registre
\$AT avec zéro, ce qui, bien sûr, est une instruction sans effet.
MIPS, comme beaucoup d'autres \ac{ISA}s, n'a pas une instruction \ac{NOP}.

\subsubsection{Rôle de la pile dans cet exemple}

L'adresse de la chaîne de texte est passée dans le registre.
Pourquoi définir une pile locale quand même?
La raison de cela est que la valeur des registres \ac{RA} et GP doit être sauvée
quelque part (car \printf est appelée), et que la pile locale est utilisée pour cela.
Si cela avait été une \glslink{leaf function}{fonction leaf}, il aurait été
possible de se passer du prologue et de l'épilogue de la fonction, par
exemple: \myref{MIPS_leaf_function_ex1}.

\subsubsection{GCC \Optimizing: chargeons-le dans GDB}

\myindex{GDB}
\lstinputlisting[caption=extrait d'une session GDB]{patterns/01_helloworld/MIPS/O3_GDB.txt}

