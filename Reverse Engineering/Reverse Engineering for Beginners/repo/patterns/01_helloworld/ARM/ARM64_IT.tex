\subsubsection{ARM64}

\myparagraph{GCC}

Compiliamo l'esempio con GCC 4.8.1 per ARM64:

\lstinputlisting[numbers=left,label=hw_ARM64_GCC,caption=\NonOptimizing GCC 4.8.1 + objdump,style=customasmARM]{patterns/01_helloworld/ARM/hw.lst}

In ARM64 non ci sono le modalità Thumb e Thumb-2, ma solo ARM, quindi esistono soltanto istruzioni a 32-bit.
Il numero di registri è raddoppiato: \myref{ARM64_GPRs}.
I registri a 64-bit hanno il prefisso \TT{X-} prefixes, mentre le loro parti a 32-bit hanno il prefisso --- \TT{W-}.

\myindex{ARM!\Instructions!STP}
L'istruzione \TT{STP} (\emph{Store Pair})
salva simultaneamente due registri nello stack: \RegX{29} e \RegX{30}.

Questa istruzione può ovviamente salvare la coppia di valori in una posizione arbitraria in memoria, tuttavia in questo caso è
specificato il registro \ac{SP}, e di conseguenza la coppia viene salvata nello stack.

I registri ARM64 sono a 64-bit, ognuno di essi ha dimensione pari a 8 byte, quindi sono necessari 16 byte per salvare i due registri.

Il punto esclamativo (``!'') dopo l'operando sta a significare che 16 deve esse prima sottratto da \ac{SP}, e solo successivamente
i valori devono essere scritti nello stack.

Questo è anche detto \emph{pre-index}.
Per le differenze tra \emph{post-index} e \emph{pre-index}
leggere qui: \myref{ARM_postindex_vs_preindex}.

Quindi, in termini del più familiare x86, la prima istruzione è semplicamente l'analogo della coppia
\TT{PUSH X29} e \TT{PUSH X30}.
\RegX{29} in ARM64 è usato come \ac{FP}, e \RegX{30}
come \ac{LR}, e questo spiega perchè sono salvati nel prologo della funzione e ripristinati nell'epilogo.

La seconda istruzione copia \ac{SP} in \RegX{29} (o \ac{FP}).
Ciò viene fatto per impostare lo stack frame della funzione.

\label{pointers_ADRP_and_ADD}
\myindex{ARM!\Instructions!ADRP/ADD pair}
Le istruzioni \TT{ADRP} e \ADD sono usate per inserire
l'indirizzo della stringa \q{Hello!} nel registro \RegX{0},
poichè il primo argomento della funzione viene passato in questo registro.

Non esiste alcun tipo di istruzione in ARM in grado di salvare un numero molto grande in un registro (perchè la lunghezza delle
istruzioni è limitata a 4 byte, maggiori informazioni qui: \myref{ARM_big_constants_loading}).
Perciò devono essere utilizzate più istruzioni. La prima (\TT{ADRP}), scrive l'indirizzo della pagina di 4KiB (4KiB page)
in cui si trova la stringa, nel registro \RegX{0}, e la seconda (\ADD) aggiunge semplicemente il resto dell'indirizzo.
Maggiori informazioni su questo tema: \myref{ARM64_relocs}.

\TT{0x400000 + 0x648 = 0x400648}, e vediamo la nostra C-string \q{Hello!} nel \TT{.rodata} data segment a questo indirizzo.

\myindex{ARM!\Instructions!BL}

\puts viene chiamata subito dopo usando l'istruzione \TT{BL}. Questo è già stato discusso: \myref{puts}.

\MOV scrive 0 in \RegW{0}.
\RegW{0} è la parte bassa a 32 bits del registro a 64-bit \RegX{0}:

\begin{center}
\begin{tabular}{ | l | l | }
\hline
	Parte alta dei 32 bit & Parte bassa dei 32 bit \\
\hline
\multicolumn{2}{ | c | }{X0} \\
\hline
\multicolumn{1}{ | c | }{} & \multicolumn{1}{ c | }{W0} \\
\hline
\end{tabular}
\end{center}


Il risultato della funzione viene restituito tramite \RegX{0} e \main restituisce 0, quindi è in questo modo che viene preparato
il valore da restituire.
Ma perchè usare la parte a 32-bit?

Perchè il tipo \Tint in ARM64, esattamente come in x86-64, è ancora a 32 bit, per maggiore compatibilità.
Quindi se una funzione restituisce un \Tint a 32 bit, solo la parte più bassa a 32 bit del registro \RegX{0} verrà utilizzata.

Per verificare quanto detto, cambiamo leggermente l'esempio e ricompiliamolo.
Adesso \main restituisce un valore a 64-bit:

\begin{lstlisting}[caption=\main che ritorna un valore di tipo \TT{uint64\_t},style=customc]
#include <stdio.h>
#include <stdint.h>

uint64_t main()
{
        printf ("Hello!\n");
        return 0;
}
\end{lstlisting}

Il risultato è lo stesso, ma quell'istruzione MOV adesso appare così:

\begin{lstlisting}[caption=\NonOptimizing GCC 4.8.1 + objdump]
  4005a4:       d2800000        mov     x0, #0x0      // #0
\end{lstlisting}

\myindex{ARM!\Instructions!LDP}

\INS{LDP} (\emph{Load Pair}) infine riprisina i registri \RegX{29} e \RegX{30}.

Non c'è il punto esclamativo dopo l'istruzione: ciò implica che il valore viene prima caricato dallo stack, e solo successivamente
\ac{SP} è incrementato di 16.
Questo viene detto \emph{post-index}.

\myindex{ARM!\Instructions!RET}
Una nuova istruzione è apparsa in ARM64: \RET.
Funziona esattamente come \TT{BX LR}, con l'aggiunta di uno speciale \emph{hint} bit, che informa la \ac{CPU}
del fatto che si tratta di un ritorno da una funzione, e non soltanto una normale istruzione jump, in questo modo
può venire eseguita in modo più ottimale.

A causa della semplicità della funzione, GCC con le opzioni di ottimizzazione genera esattamente lo stesso codice.
