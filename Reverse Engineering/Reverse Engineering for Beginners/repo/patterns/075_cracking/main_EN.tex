\mysection{Software cracking}
\myindex{\SoftwareCracking}

The vast majority of software can be cracked like that --- by searching the very place where protection is checked, a dongle
(\myref{dongles}), license key, serial number, etc.

Often, it looks like:

\begin{lstlisting}[style=customasmx86]
...
call check_protection
jz   all_OK
call message_box_protection_missing
call exit
all_OK:
; proceed
...
\end{lstlisting}

So if you see a patch (or ``crack''), that cracks a software, and that patch replaces 0x74/0x75 (JZ/JNZ) byte(s) by 0xEB (JMP),
this is it.

The process of software cracking comes down to a search of that \INS{JMP}.

\myhrule{}

There are also a cases, when a software checks protection from time to time, this can be a dongle, or a license server
can be queried through the Internet.
Then you have to look for a function that checks protection.
Then to patch it, to put there \TT{xor eax, eax / retn}, or \TT{mov eax, 1 / retn}.

It's important to understand that after patching of function beginning, usually, a garbage follows these two instructions.
The garbage consists of part of one instruction and the several next instructions.

This is a real case.
The beginning of a function which we want to \emph{replace} by \TT{return 1;}

\begin{lstlisting}[style=customasmx86,caption=Before]
8BFF                           mov         edi,edi
55                             push        ebp
8BEC                           mov         ebp,esp
81EC68080000                   sub         esp,000000868
A110C00001                     mov         eax,[00100C010]
33C5                           xor         eax,ebp
8945FC                         mov         [ebp][-4],eax
53                             push        ebx
8B5D08                         mov         ebx,[ebp][8]
...
\end{lstlisting}

\begin{lstlisting}[style=customasmx86,caption=After]
B801000000                     mov         eax,1
C3                             retn
EC                             in          al,dx
68080000A1                     push        0A1000008
10C0                           adc         al,al
0001                           add         [ecx],al
33C5                           xor         eax,ebp
8945FC                         mov         [ebp][-4],eax
53                             push        ebx
8B5D08                         mov         ebx,[ebp][8]
...
\end{lstlisting}

Several incorrect instructions appears --- \INS{IN}, \INS{PUSH}, \INS{ADC}, \INS{ADD},
after which, Hiew disassembler (which I just used) synchronized and continued to disassemble all the rest.

This is not important --- all these instructions followed \INS{RETN} will never be executed,
unless a direct jump would occur from some place, and that wouldn't be possible in general case.

\myhrule{}

Also, a global boolean variable can be present, having a flag, was the software registered or not.

\begin{lstlisting}[style=customasmx86]
init_etc proc
...
call check_protection_or_license_file
mov  is_demo, eax
...
retn
init_etc endp

...

save_file proc
...
mov  eax, is_demo
cmp  eax, 1
jz   all_OK1

call message_box_it_is_a_demo_no_saving_allowed
retn

:all_OK1
; continue saving file

...

save_proc endp

somewhere_else proc

mov  eax, is_demo
cmp  eax, 1
jz   all_OK

; check if we run for 15 minutes
; exit if it is so
; or show nagging screen

:all_OK2
; continue

somewhere_else endp
\end{lstlisting}

A beginning of the \verb|check_protection_or_license_file()| function could be patched, so that it will always return 1,
or, if this is better by some reason, all \INS{JZ}/\INS{JNZ} instructions can be patched as well.

More about patching: \ref{patching}.

