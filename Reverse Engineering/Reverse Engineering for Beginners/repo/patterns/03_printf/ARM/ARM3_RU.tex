\subsubsection{ARM: 3 целочисленных аргумента}

В ARM традиционно принята такая схема передачи аргументов в функцию: 
4 первых аргумента через регистры \Reg{0}-\Reg{3}; а остальные~--- через стек.
Это немного похоже на то, как аргументы передаются в 
fastcall~(\myref{fastcall}) или win64~(\myref{sec:callingconventions_win64}).

\myparagraph{32-битный ARM}

\mysubparagraph{\NonOptimizingKeilVI (\ARMMode)}

\begin{lstlisting}[caption=\NonOptimizingKeilVI (\ARMMode),style=customasmARM]
.text:00000000 main
.text:00000000 10 40 2D E9   STMFD   SP!, {R4,LR}
.text:00000004 03 30 A0 E3   MOV     R3, #3
.text:00000008 02 20 A0 E3   MOV     R2, #2
.text:0000000C 01 10 A0 E3   MOV     R1, #1
.text:00000010 08 00 8F E2   ADR     R0, aADBDCD     ; "a=\%d; b=\%d; c=\%d"
.text:00000014 06 00 00 EB   BL      __2printf
.text:00000018 00 00 A0 E3   MOV     R0, #0          ; return 0
.text:0000001C 10 80 BD E8   LDMFD   SP!, {R4,PC}
\end{lstlisting}

Итак, первые 4 аргумента передаются через регистры \Reg{0}-\Reg{3}, по порядку: 
указатель на формат-строку для \printf
в \Reg{0}, затем 1 в \Reg{1}, 2 в \Reg{2} и 3 в \Reg{3}.

Инструкция на \GTT{0x18} записывает 0 в \Reg{0} --- это выражение в Си \emph{return 0}.
Пока что здесь нет ничего необычного.
\OptimizingKeilVI генерирует точно такой же код.

\mysubparagraph{\OptimizingKeilVI (\ThumbMode)}

\begin{lstlisting}[caption=\OptimizingKeilVI (\ThumbMode),style=customasmARM]
.text:00000000 main
.text:00000000 10 B5        PUSH    {R4,LR}
.text:00000002 03 23        MOVS    R3, #3
.text:00000004 02 22        MOVS    R2, #2
.text:00000006 01 21        MOVS    R1, #1
.text:00000008 02 A0        ADR     R0, aADBDCD     ; "a=\%d; b=\%d; c=\%d"
.text:0000000A 00 F0 0D F8  BL      __2printf
.text:0000000E 00 20        MOVS    R0, #0
.text:00000010 10 BD        POP     {R4,PC}
\end{lstlisting}

Здесь нет особых отличий от неоптимизированного варианта для режима ARM.
\mysubparagraph{\OptimizingKeilVI (\ARMMode) + убираем return}
\label{ARM_B_to_printf}

Немного переделаем пример, убрав \emph{return 0}:

\begin{lstlisting}[style=customc]
#include <stdio.h>

void main()
{
	printf("a=%d; b=%d; c=%d", 1, 2, 3);
};
\end{lstlisting}

Результат получится необычным:

\lstinputlisting[caption=\OptimizingKeilVI (\ARMMode),style=customasmARM]{patterns/03_printf/ARM/tmp3.asm}

\myindex{ARM!\Registers!Link Register}
\myindex{ARM!\Instructions!B}
\myindex{Function epilogue}
Это оптимизированная версия (\Othree) для режима ARM, и здесь мы видим последнюю инструкцию 
\INS{B} вместо привычной нам \INS{BL}.
Отличия между этой оптимизированной версией и предыдущей, скомпилированной без оптимизации, 
ещё и в том, что здесь нет пролога и эпилога функции (инструкций, сохраняющих состояние регистров \Reg{0} и \ac{LR}).
\myindex{x86!\Instructions!JMP}
Инструкция \INS{B} просто переходит на другой адрес, без манипуляций с регистром \ac{LR}, то есть это аналог \JMP в x86.
Почему это работает нормально? Потому что этот код эквивалентен предыдущему.

Основных причин две: 1) стек не модифицируется, как и \glslink{stack pointer}{указатель стека} \ac{SP}; 2) вызов функции \printf последний, после него ничего не происходит.
Функция \printf, отработав, просто возвращает управление по адресу, записанному в \ac{LR}.

Но в \ac{LR} находится адрес места, откуда была вызвана наша функция!
А следовательно, управление из \printf вернется сразу туда.

Значит нет нужды сохранять \ac{LR}, потому что нет нужды модифицировать \ac{LR}.
А нет нужды модифицировать \ac{LR}, потому что нет иных вызовов функций, кроме \printf, к тому же, после этого вызова не нужно ничего здесь больше делать!
Поэтому такая оптимизация возможна.

Эта оптимизация часто используется в функциях, где последнее выражение~--- это вызов другой функции.

Ещё один похожий пример описан здесь:
\myref{jump_to_last_printf}.

% TBT
% A somewhat simpler case was described above: \myref{Boolector}.

\myparagraph{ARM64}

\mysubparagraph{\NonOptimizing GCC (Linaro) 4.9}

\lstinputlisting[caption=\NonOptimizing GCC (Linaro) 4.9,style=customasmARM]{patterns/03_printf/ARM/ARM3_O0_RU.lst}

\myindex{ARM!\Instructions!STP}
Итак, первая инструкция \INS{STP} (\emph{Store Pair}) сохраняет \ac{FP} (X29) и \ac{LR} (X30) в стеке.
Вторая инструкция \INS{ADD X29, SP, 0} формирует стековый фрейм.
Это просто запись значения \ac{SP} в X29.

\myindex{ARM!\Instructions!ADRP/ADD pair}
Далее уже знакомая пара инструкций \INS{ADRP}/\ADD формирует указатель на строку.

\myindex{ARM64!lo12}
\emph{lo12} означает младшие 12 бит, т.е., линкер запишет младшие 12 бит адреса метки LC1 в опкод инструкции \ADD.
1, 2 и 3 это 32-битные значения типа \Tint{}, так что они загружаются в 32-битные части регистров
\footnote{Изменение 1 на 1L изменит эту константу на 64-битную и она будет загружаться в 64-битный регистр.
См. о целочисленных константах/литералах:
\href{https://en.cppreference.com/w/c/language/integer_constant}{1},
\href{https://en.cppreference.com/w/cpp/language/integer_literal}{2}.}

\Optimizing GCC (Linaro) 4.9 генерирует почти такой же код.

