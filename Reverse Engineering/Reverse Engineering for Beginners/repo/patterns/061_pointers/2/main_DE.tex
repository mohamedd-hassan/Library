\subsection{Werte zurückgeben}
Pointer werden oft verwendet um Funktionsergebnisse zurückzuliefern (siehe der Fall \scanf ~(\myref{label_scanf})).

Zum Beispiel dann, wenn eine Funktion zwei Werte zurückgeben soll.

\subsubsection{Beispiel mit globalen Variablen}

\lstinputlisting[style=customc]{patterns/061_pointers/2/global.c}

Dies kompiliert zu:

\lstinputlisting[caption=\Optimizing MSVC 2010 (/Ob0),style=customasmx86]{patterns/061_pointers/2/global.asm}

\myindex{\olly}
\clearpage
Schauen wir es uns in \olly an:

\begin{figure}[H]
\centering
\myincludegraphics{patterns/061_pointers/2/olly_global1.png}
\caption{\olly: Adressen der globalen Variablen werden an \ttfone übergeben}
\label{fig:pointers_olly_global_1}
\end{figure}
Zuerst werden die Adressen der globalen Variablen an \ttfone übergeben.
Wir klicken auf \q{Follow in dump} beim Stackelement und wir sehen den Platz, der im Datensegment für die beiden
Variablen angelegt wird.

\clearpage
Diese Variablen werden auf Null gesetzt, denn nicht initialisierte Daten (aus \ac{BSS}) werden gelöscht, bevor die
Ausführung beinnt, [siehe \CNineNineStd{} 6.7.8p10].

Sie bleiben im Datensegment, was wir durch Drücken von Alt-M und untersuchen der Speicherzuordnung verifizieren können:

\begin{figure}[H]
\centering
\myincludegraphics{patterns/061_pointers/2/olly_global5.png}
\caption{\olly: Speicherzuordnung}
\label{fig:pointers_olly_global_5}
\end{figure}

\clearpage
Verfolgen wir den Ablauf (F7) bis zum Start von \ttfone: 

\begin{figure}[H]
\centering
\myincludegraphics{patterns/061_pointers/2/olly_global2.png}
\caption{\olly: \ttfone beginnt}
\label{fig:pointers_olly_global_2}
\end{figure}
Zwei Werte sind auf dem Stack sichtbar: 456 (\TT{0x1C8}) und
123 (\TT{0x7B}) und außerdem die Adressen der beiden globalen Variablen.

\clearpage
Verfolgen wir den Ablauf bis zum Ende von \ttfone.
Im linken unteren Fenster sehen wir wie die Ergebnisse der Berechnung in den globalen Variablen erscheinen:

\begin{figure}[H]
\centering
\myincludegraphics{patterns/061_pointers/2/olly_global3.png}
\caption{\olly: Ausführung von \ttfone beendet}
\label{fig:pointers_olly_global_3}
\end{figure}

\clearpage
Jetzt werden die Werte der globalen Variablen in Register geladen, um dann an \printf übergeben zu werden (über den
Stack):

\begin{figure}[H]
\centering
\myincludegraphics{patterns/061_pointers/2/olly_global4.png}
\caption{\olly: Adressen der globalen Variablen werden an \printf übergeben}
\label{fig:pointers_olly_global_4}
\end{figure}

\subsubsection{Beispiel mit lokalen Variablen}

Verändern wir unser Beispiel ein wenig:

\lstinputlisting[caption=\TT{sum} und \TT{product} sind jetzt lokale Variablen,style=customc]{patterns/061_pointers/2/local_DE.c}

Der Code von \ttfone wird sich nicht verändern.
Nur den Code von \main wird sich verändern:

\lstinputlisting[caption=\Optimizing MSVC 2010 (/Ob0),style=customasmx86]{patterns/061_pointers/2/local.asm}

\newcommand{\PtrsAddresses}{\TT{0x2EF854} und \TT{0x2EF858}\xspace}

\clearpage
Schauen wir es uns erneut mit \olly an. 
Die Adressen der lokalen Variablen auf dem Stack sind \PtrsAddresses.
Wir erkennen wie diese auf dem Stack abgelegt werden:

\begin{figure}[H]
\centering
\myincludegraphics{patterns/061_pointers/2/olly_stk1.png}
\caption{\olly: Adressen der lokalen Variablen werden auf dem Stack abgelegt}
\label{fig:pointers_olly_stk_1}
\end{figure}

\clearpage
\ttfone beginnt.
Bis hierher befinden sich nur zufällige Werte auf dem Stack an den Adressen \PtrsAddresses:

\begin{figure}[H]
\centering
\myincludegraphics{patterns/061_pointers/2/olly_stk2.png}
\caption{\olly: \ttfone beginnt}
\label{fig:pointers_olly_stk_2}
\end{figure}

\clearpage
\ttfone endet:

\begin{figure}[H]
\centering
\myincludegraphics{patterns/061_pointers/2/olly_stk3.png}
\caption{\olly: Ausführung von \ttfone endet}
\label{fig:pointers_olly_stk_3}
\end{figure}
Wir finden nun \TT{0xDB18} und \TT{0x243} an den Adressen \PtrsAddresses.
Diese Werte sind die Ergebnisse von \ttfone.

% \Conclusion{} is a macro, don't translate it:
\subsubsection{\Conclusion{}}
\ttfone kann Pointer auf jede beliebige Speicherstelle zurückgeben.
Dies ist die Quintessenz der Nützlichkeit von Pointern. 
 
\Cpp \emph{references} funktionieren übrigens genau auf die gleiche Weise. Lesen Sie mehr dazu: (\myref{cpp_references}).
