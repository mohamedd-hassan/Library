\subsection{Защита от переполнения буфера}
\label{subsec:BO_protection}

В наше время пытаются бороться с переполнением буфера невзирая на халатность программистов на \CCpp. 
В MSVC есть опции вроде\footnote{описания защит, которые компилятор может вставлять в код:
\href{http://en.wikipedia.org/wiki/Buffer_overflow_protection}{wikipedia.org/wiki/Buffer\_overflow\_protection}}:

\begin{lstlisting}
 /RTCs Stack Frame runtime checking
 /GZ Enable stack checks (/RTCs)
\end{lstlisting}

\myindex{x86!\Instructions!RET}
\myindex{Function prologue}
\myindex{Security cookie}
Одним из методов является вставка в прологе функции некоего случайного значения в область локальных переменных 
и проверка этого значения в эпилоге функции перед выходом. 
Если проверка не прошла, то не выполнять инструкцию \RET, а остановиться (или зависнуть). 
Процесс зависнет, но это лучше, чем удаленная атака на ваш компьютер.

\newcommand{\CANARYURL}{\href{http://miningwiki.ru/wiki/Канарейка_в_шахте}{miningwiki.ru/wiki/Канарейка\_в\_шахте}}

\myindex{Canary}
Это случайное значение иногда называют \q{канарейкой}
\footnote{\q{canary} в англоязычной литературе}, 
по аналогии с шахтной канарейкой\footnote{\CANARYURL}.
Раньше использовали шахтеры, чтобы определять, есть ли в шахте опасный газ.

Канарейки очень к нему чувствительны и либо проявляли сильное беспокойство, либо гибли от газа.

Если скомпилировать наш простейший пример работы с массивом ~(\myref{arrays_simple}) в \ac{MSVC}
с опцией RTC1 или RTCs,
в конце нашей функции будет вызов функции \\
\TT{@\_RTC\_CheckStackVars@8}, проверяющей корректность \q{канарейки}.

Посмотрим, как дела обстоят в GCC. 
Возьмем пример из секции про \TT{alloca()}~(\myref{alloca}):

\lstinputlisting[style=customc]{patterns/02_stack/04_alloca/2_1.c}

По умолчанию, без дополнительных ключей, GCC 4.7.3 вставит в код проверку \q{канарейки}:

\lstinputlisting[caption=GCC 4.7.3,style=customasmx86]{patterns/13_arrays/3_BO_protection/gcc_canary_RU.asm}

\myindex{x86!\Registers!GS}
Случайное значение находится в \TT{gs:20}. 
Оно записывается в стек, затем, в конце функции, значение в стеке
сравнивается с корректной \q{канарейкой} в \TT{gs:20}. 
Если значения не равны, будет вызвана функция 
\TT{\_\_stack\_chk\_fail} и в консоли мы увидим что-то вроде такого
 (Ubuntu 13.04 x86):

\begin{lstlisting}
*** buffer overflow detected ***: ./2_1 terminated
======= Backtrace: =========
/lib/i386-linux-gnu/libc.so.6(__fortify_fail+0x63)[0xb7699bc3]
/lib/i386-linux-gnu/libc.so.6(+0x10593a)[0xb769893a]
/lib/i386-linux-gnu/libc.so.6(+0x105008)[0xb7698008]
/lib/i386-linux-gnu/libc.so.6(_IO_default_xsputn+0x8c)[0xb7606e5c]
/lib/i386-linux-gnu/libc.so.6(_IO_vfprintf+0x165)[0xb75d7a45]
/lib/i386-linux-gnu/libc.so.6(__vsprintf_chk+0xc9)[0xb76980d9]
/lib/i386-linux-gnu/libc.so.6(__sprintf_chk+0x2f)[0xb7697fef]
./2_1[0x8048404]
/lib/i386-linux-gnu/libc.so.6(__libc_start_main+0xf5)[0xb75ac935]
======= Memory map: ========
08048000-08049000 r-xp 00000000 08:01 2097586    /home/dennis/2_1
08049000-0804a000 r--p 00000000 08:01 2097586    /home/dennis/2_1
0804a000-0804b000 rw-p 00001000 08:01 2097586    /home/dennis/2_1
094d1000-094f2000 rw-p 00000000 00:00 0          [heap]
b7560000-b757b000 r-xp 00000000 08:01 1048602    /lib/i386-linux-gnu/libgcc_s.so.1
b757b000-b757c000 r--p 0001a000 08:01 1048602    /lib/i386-linux-gnu/libgcc_s.so.1
b757c000-b757d000 rw-p 0001b000 08:01 1048602    /lib/i386-linux-gnu/libgcc_s.so.1
b7592000-b7593000 rw-p 00000000 00:00 0
b7593000-b7740000 r-xp 00000000 08:01 1050781    /lib/i386-linux-gnu/libc-2.17.so
b7740000-b7742000 r--p 001ad000 08:01 1050781    /lib/i386-linux-gnu/libc-2.17.so
b7742000-b7743000 rw-p 001af000 08:01 1050781    /lib/i386-linux-gnu/libc-2.17.so
b7743000-b7746000 rw-p 00000000 00:00 0
b775a000-b775d000 rw-p 00000000 00:00 0
b775d000-b775e000 r-xp 00000000 00:00 0          [vdso]
b775e000-b777e000 r-xp 00000000 08:01 1050794    /lib/i386-linux-gnu/ld-2.17.so
b777e000-b777f000 r--p 0001f000 08:01 1050794    /lib/i386-linux-gnu/ld-2.17.so
b777f000-b7780000 rw-p 00020000 08:01 1050794    /lib/i386-linux-gnu/ld-2.17.so
bff35000-bff56000 rw-p 00000000 00:00 0          [stack]
Aborted (core dumped)
\end{lstlisting}

\myindex{MS-DOS}
gs это так называемый сегментный регистр. Эти регистры широко использовались во времена MS-DOS 
и DOS-экстендеров.
Сейчас их функция немного изменилась.
\myindex{TLS}
\myindex{Windows!TIB}
Если говорить кратко, в Linux \TT{gs} всегда указывает на \ac{TLS}~(\myref{TLS})~--- там находится различная 
информация, специфичная для выполняющегося потока.

Кстати, в win32 эту же роль играет сегментный регистр \TT{fs},
он всегда указывает на
\ac{TIB} \footnote{\href{https://en.wikipedia.org/wiki/Win32_Thread_Information_Block}{wikipedia.org/wiki/Win32\_Thread\_Information\_Block}}. 

Больше информации можно почерпнуть из исходных кодов Linux (по крайней мере, в версии 3.11):\\
в файле \emph{arch/x86/include/asm/stackprotector.h} в комментариях описывается эта переменная.

\subsubsection{ARM + \OptimizingXcodeIV (\ARMMode)}

\lstinputlisting[caption=\OptimizingXcodeIV (\ARMMode),label=ARM_leaf_example4,style=customasmARM]{patterns/14_bitfields/4_popcnt/ARM_Xcode_O3_RU.lst}

\myindex{ARM!\Instructions!TST}
\TST это то же что и \TEST в x86.

\myindex{ARM!Optional operators!LSL}
\myindex{ARM!Optional operators!LSR}
\myindex{ARM!Optional operators!ASR}
\myindex{ARM!Optional operators!ROR}
\myindex{ARM!Optional operators!RRX}
\myindex{ARM!\Instructions!MOV}
\myindex{ARM!\Instructions!TST}
\myindex{ARM!\Instructions!CMP}
\myindex{ARM!\Instructions!ADD}
\myindex{ARM!\Instructions!SUB}
\myindex{ARM!\Instructions!RSB}
Как уже было указано~(\myref{shifts_in_ARM_mode}),
в режиме ARM нет отдельной инструкции для сдвигов.

Однако, модификаторами 
LSL (\emph{Logical Shift Left}), 
LSR (\emph{Logical Shift Right}), 
ASR (\emph{Arithmetic Shift Right}), 
ROR (\emph{Rotate Right}) и
RRX (\emph{Rotate Right with Extend}) можно дополнять некоторые инструкции, такие как \MOV, \TST,
\CMP, \ADD, \SUB, \RSB\footnote{\DataProcessingInstructionsFootNote}.

Эти модификаторы указывают, как сдвигать второй операнд, и на сколько.

\myindex{ARM!\Instructions!TST}
\myindex{ARM!Optional operators!LSL}
Таким образом, инструкция  \TT{\q{TST R1, R2,LSL R3}} здесь работает как $R1 \land (R2 \ll R3)$.

\subsubsection{ARM + \OptimizingXcodeIV (\ThumbTwoMode)}

\myindex{ARM!\Instructions!LSL.W}
\myindex{ARM!\Instructions!LSL}
Почти такое же, только здесь применяется пара инструкций \INS{LSL.W}/\TST вместо одной \TST,
ведь в режиме Thumb нельзя добавлять модификатор \LSL прямо в \TST.

\begin{lstlisting}[label=ARM_leaf_example5,style=customasmARM]
                MOV             R1, R0
                MOVS            R0, #0
                MOV.W           R9, #1
                MOVS            R3, #0
loc_2F7A
                LSL.W           R2, R9, R3
                TST             R2, R1
                ADD.W           R3, R3, #1
                IT NE
                ADDNE           R0, #1
                CMP             R3, #32
                BNE             loc_2F7A
                BX              LR
\end{lstlisting}

\subsubsection{ARM64 + \Optimizing GCC 4.9}

Возьмем 64-битный пример, который уже был здесь использован: \myref{popcnt_x64_example}.

\lstinputlisting[caption=\Optimizing GCC (Linaro) 4.8,style=customasmARM]{patterns/14_bitfields/4_popcnt/ARM64_GCC_O3_RU.s}
Результат очень похож на тот, что GCC сгенерировал для x64: \myref{shifts64_GCC_O3}.

\myindex{ARM!\Instructions!CSEL}
Инструкция \CSEL это \q{Conditional SELect} (выбор при условии). 
Она просто выбирает одну из переменных, в зависимости от флагов выставленных \TST и копирует значение в регистр \RegW{2}, содержащий переменную \q{rt}.

\subsubsection{ARM64 + \NonOptimizing GCC 4.9}

И снова будем использовать 64-битный пример, который мы использовали ранее: \myref{popcnt_x64_example}.
Код более многословный, как обычно.

\lstinputlisting[caption=\NonOptimizing GCC (Linaro) 4.8,style=customasmARM]{patterns/14_bitfields/4_popcnt/ARM64_GCC_O0_RU.s}


