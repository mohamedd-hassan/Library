\subsection{Méthodes de protection contre les débordements de tampon}
\label{subsec:BO_protection}

Il existe quelques méthodes pour protéger contre ce fléau, indépendamment de la négligence
des programmeurs \CCpp.
MSVC possède des options comme\footnote{méthode de protection contre les débordements
de tampons côté compilateur:\href{http://en.wikipedia.org/wiki/Buffer_overflow_protection}{wikipedia.org/wiki/Buffer\_overflow\_protection}}:

\begin{lstlisting}
 /RTCs Stack Frame runtime checking
 /GZ Enable stack checks (/RTCs)
\end{lstlisting}

\myindex{x86!\Instructions!RET}
\myindex{Function prologue}
\myindex{Security cookie}

Une des méthodes est d'écrire une valeur aléatoire entre les variables locales sur
la pile dans le prologue de la fonction et de la vérifier dans l'épilogue, avant de
sortir de la fonction.
Si la valeur n'est pas la même, ne pas exécuter la dernière instruction \RET, mais
stopper (ou bloquer).
Le processus va s'arrêter, mais c'est mieux qu'une attaque distante sur votre ordinateur.
    
\newcommand{\CANARYURL}{\href{http://en.wikipedia.org/wiki/Domestic_Canary\#Miner.27s_canary}{wikipedia.org/wiki/Domestic\_canary\#Miner.27s\_canary}}

\myindex{Canary}

Cette valeur aléatoire est parfois appelé un \q{canari}, c'est lié au canari\footnote{\CANARYURL}
que les mineurs utilisaient dans le passé afin de détecter rapidement les gaz toxiques.

Les canaris sont très sensibles aux gaz, ils deviennent très agités en cas de danger,
et même meurent.

Si nous compilons notre exemple de tableau très simple~(\myref{arrays_simple}) dans
\ac{MSVC} avec les options RTC1 et RTCs, nous voyons un appel à \TT{@\_RTC\_CheckStackVars@8}
une fonction à la fin de la fonction qui vérifie si le \q{canari} est correct.

Voyons comment GCC gère ceci.
Prenons un exemple \TT{alloca()}~(\myref{alloca}):

\lstinputlisting[style=customc]{patterns/02_stack/04_alloca/2_1.c}

Par défaut, sans option supplémentaire, GCC 4.7.3 insère un test de  \q{canari} dans
le code:

\lstinputlisting[caption=GCC 4.7.3,style=customasmx86]{patterns/13_arrays/3_BO_protection/gcc_canary_FR.asm}

\myindex{x86!\Registers!GS}
La valeur aléatoire se trouve en \TT{gs:20}.
Elle est écrite sur la pile et à la fin de la fonction, la valeur sur la pile est
comparée avec le \q{canari} correct dans \TT{gs:20}.
Si les valeurs ne sont pas égales, la fonction \TT{\_\_stack\_chk\_fail} est appelée
et nous voyons dans la console quelque chose comme ça (Ubuntu 13.04 x86):

\begin{lstlisting}
*** buffer overflow detected ***: ./2_1 terminated
======= Backtrace: =========
/lib/i386-linux-gnu/libc.so.6(__fortify_fail+0x63)[0xb7699bc3]
/lib/i386-linux-gnu/libc.so.6(+0x10593a)[0xb769893a]
/lib/i386-linux-gnu/libc.so.6(+0x105008)[0xb7698008]
/lib/i386-linux-gnu/libc.so.6(_IO_default_xsputn+0x8c)[0xb7606e5c]
/lib/i386-linux-gnu/libc.so.6(_IO_vfprintf+0x165)[0xb75d7a45]
/lib/i386-linux-gnu/libc.so.6(__vsprintf_chk+0xc9)[0xb76980d9]
/lib/i386-linux-gnu/libc.so.6(__sprintf_chk+0x2f)[0xb7697fef]
./2_1[0x8048404]
/lib/i386-linux-gnu/libc.so.6(__libc_start_main+0xf5)[0xb75ac935]
======= Memory map: ========
08048000-08049000 r-xp 00000000 08:01 2097586    /home/dennis/2_1
08049000-0804a000 r--p 00000000 08:01 2097586    /home/dennis/2_1
0804a000-0804b000 rw-p 00001000 08:01 2097586    /home/dennis/2_1
094d1000-094f2000 rw-p 00000000 00:00 0          [heap]
b7560000-b757b000 r-xp 00000000 08:01 1048602    /lib/i386-linux-gnu/libgcc_s.so.1
b757b000-b757c000 r--p 0001a000 08:01 1048602    /lib/i386-linux-gnu/libgcc_s.so.1
b757c000-b757d000 rw-p 0001b000 08:01 1048602    /lib/i386-linux-gnu/libgcc_s.so.1
b7592000-b7593000 rw-p 00000000 00:00 0
b7593000-b7740000 r-xp 00000000 08:01 1050781    /lib/i386-linux-gnu/libc-2.17.so
b7740000-b7742000 r--p 001ad000 08:01 1050781    /lib/i386-linux-gnu/libc-2.17.so
b7742000-b7743000 rw-p 001af000 08:01 1050781    /lib/i386-linux-gnu/libc-2.17.so
b7743000-b7746000 rw-p 00000000 00:00 0
b775a000-b775d000 rw-p 00000000 00:00 0
b775d000-b775e000 r-xp 00000000 00:00 0          [vdso]
b775e000-b777e000 r-xp 00000000 08:01 1050794    /lib/i386-linux-gnu/ld-2.17.so
b777e000-b777f000 r--p 0001f000 08:01 1050794    /lib/i386-linux-gnu/ld-2.17.so
b777f000-b7780000 rw-p 00020000 08:01 1050794    /lib/i386-linux-gnu/ld-2.17.so
bff35000-bff56000 rw-p 00000000 00:00 0          [stack]
Aborted (core dumped)
\end{lstlisting}

\myindex{MS-DOS}
gs est ainsi appelé registre de segment. Ces registres étaient beaucoup utilisés
du temps de MS-DOS et des extensions de DOS.
Aujourd'hui, sa fonction est différente.
\myindex{TLS}
\myindex{Windows!TIB}

Dit brièvement, le registre \TT{gs} dans Linux pointe toujours sur le
\ac{TLS}~(\myref{TLS})---des informations spécifiques au thread sont stockées là.
À propos, en win32 le registre \TT{fs} joue le même rôle, pointant sur \ac{TIB}
\footnote{\href{https://en.wikipedia.org/wiki/Win32_Thread_Information_Block}{wikipedia.org/wiki/Win32\_Thread\_Information\_Block}}.

Il y a plus d'information dans le code source du noyau Linux (au moins dans la version 3.11),
dans\\
\emph{arch/x86/include/asm/stackprotector.h} cette variable est décrite dans les commentaires.

\subsubsection{ARM: \OptimizingKeilVI (\ARMMode)}
\myindex{\CLanguageElements!switch}

\lstinputlisting[style=customasmARM]{patterns/08_switch/1_few/few_ARM_ARM_O3.asm}

A nouveau, en investiguant ce code, nous ne pouvons pas dire si il y avait un switch()
dans le code source d'origine ou juste un ensemble de déclarations if().

\myindex{ARM!\Instructions!ADRcc}

En tout cas, nous voyons ici des instructions conditionnelles (comme \ADREQ (\emph{Equal}))
qui ne sont exécutées que si $R0=0$, et qui chargent ensuite l'adresse de la chaîne
\emph{<<zero\textbackslash{}n>>} dans \Reg{0}.
\myindex{ARM!\Instructions!BEQ}
L'instruction suivante \ac{BEQ} redirige le flux d'exécution en \TT{loc\_170}, si $R0=0$.

Le lecteur attentif peut se demander si \ac{BEQ} s'exécute correctement puisque \ADREQ
a déjà mis une autre valeur dans le registre \Reg{0}.

Oui, elle s'exécutera correctement, car \ac{BEQ} vérifie les flags mis par l'instruction
\CMP et \ADREQ ne modifie aucun flag.

Les instructions restantes nous sont déjà familières.
Il y a seulement un appel à \printf, à la fin, et nous avons déjà examiné cette
astuce ici~(\myref{ARM_B_to_printf}).
A la fin, il y a trois chemins vers \printf{}.

\myindex{ARM!\Instructions!ADRcc}
\myindex{ARM!\Instructions!CMP}
La dernière instruction, \TT{CMP R0, \#2}, est nécessaire pour vérifier si $a=2$.

Si ce n'est pas vrai, alors \ADRNE charge un pointeur sur la chaîne \emph{<<something unknown \textbackslash{}n>>}
dans \Reg{0}, puisque $a$ a déjà été comparée pour savoir s'elle est égale
à 0 ou 1, et nous sommes sûrs que la variable $a$ n'est pas égale à l'un de
ces nombres, à ce point.
Et si $R0=2$, un pointeur sur la chaîne \emph{<<two\textbackslash{}n>>} sera chargé
par \ADREQ dans \Reg{0}.

\subsubsection{ARM: \OptimizingKeilVI (\ThumbMode)}

\lstinputlisting[style=customasmARM]{patterns/08_switch/1_few/few_ARM_thumb_O3.asm}

% FIXME а каким можно? к каким нельзя? \myref{} ->

Comme il y déjà été dit, il n'est pas possible d'ajouter un prédicat conditionnel
à la plupart des instructions en mode Thumb, donc ce dernier est quelque peu similaire
au code \ac{CISC}-style x86, facilement compréhensible.

\subsubsection{ARM64: GCC (Linaro) 4.9 \NonOptimizing}

\lstinputlisting[style=customasmARM]{patterns/08_switch/1_few/ARM64_GCC_O0_FR.lst}

Le type de la valeur d'entrée est \Tint, par conséquent le registre \RegW{0} est
utilisé pour garder la valeur au lieu du registre complet \RegX{0}.

Les pointeurs de chaîne sont passés à \puts en utilisant la paire d'instructions
\INS{ADRP}/\INS{ADD} comme expliqué dans l'exemple \q{\HelloWorldSectionName}:~\myref{pointers_ADRP_and_ADD}.

\subsubsection{ARM64: GCC (Linaro) 4.9 \Optimizing}

\lstinputlisting[style=customasmARM]{patterns/08_switch/1_few/ARM64_GCC_O3_FR.lst}

Ce morceau de code est mieux optimisé.
L'instruction \TT{CBZ} (\emph{Compare and Branch on Zero} comparer et sauter si zéro)
effectue un saut si \RegW{0} vaut zéro.
Il y a alors un saut direct à \puts au lieu de l'appeler, comme cela a été expliqué
avant:~\myref{JMP_instead_of_RET}.


