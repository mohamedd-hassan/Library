\subsection{Multiplikation}

\subsubsection{Multiplikation durch Addition}

Hier ist ein einfaches Beispiel:

\begin{lstlisting}[style=customc]
unsigned int f(unsigned int a)
{
	return a*8;
};
\end{lstlisting}

Die Multiplikation mit 8 wird durch 3 Additionsbefehle ersetzt, welche das
gleiche Ergebnis erzielen. Offenbar hat der MSVC Optimierer entschieden, dass
der Code so schneller sein kann.

\begin{lstlisting}[caption=\Optimizing MSVC 2010,style=customasmx86]
_TEXT	SEGMENT
_a$ = 8		; size = 4
_f	PROC
	mov	eax, DWORD PTR _a$[esp-4]
	add	eax, eax
	add	eax, eax
	add	eax, eax
	ret	0
_f	ENDP
_TEXT	ENDS
END
\end{lstlisting}

\subsubsection{Multiplikation durch Verschieben}
\label{subsec:mult_using_shifts}
Multiplikation mit und Divisionen durch Zahlen, die Potenzen von 2 sind, werden
oft durch Schiebebefehle (oft auch Shifting genannt) ersetzt.

\begin{lstlisting}[style=customc]
unsigned int f(unsigned int a)
{
	return a*4;
};
\end{lstlisting}

\begin{lstlisting}[caption=\NonOptimizing MSVC 2010,style=customasmx86]
_a$ = 8		; size = 4
_f	PROC
	push	ebp
	mov	ebp, esp
	mov	eax, DWORD PTR _a$[ebp]
	shl	eax, 2
	pop	ebp
	ret	0
_f	ENDP
\end{lstlisting}

Die Multiplikation mit 4 entspricht einer Linksverschiebung der Zahl um 2 Bit
und Einfügen zweier Nullen an der rechten Seite (an den niederwertigsten beiden
Bits). Das Prinzip ist das gleiche wie bei der dezimalen Multiplikation von 3
mit 100~--wir schreiben einfach zwei Nullen rechts an die Zahl.

Der Befehl für Linksverschiebung funktioniert wie folgt:

\myindex{x86!\Instructions!SHL}
\begin{center}
	\begin{tikzpicture}[scale=0.7, every node/.style={scale=0.7}]
	\edef\bitsize{1cm}
	\tikzstyle{byte}=[draw,minimum size=\bitsize]	
	\tikzstyle{every path}=[thick]

	\node [draw,rectangle,minimum size=\bitsize] (a1) {7};
	\node [draw,rectangle,minimum size=\bitsize] (a2) [right of=a1] {6};
	\node [draw,rectangle,minimum size=\bitsize] (a3) [right of=a2] {5};
	\node [draw,rectangle,minimum size=\bitsize] (a4) [right of=a3] {4};
	\node [draw,rectangle,minimum size=\bitsize] (a5) [right of=a4] {3};
	\node [draw,rectangle,minimum size=\bitsize] (a6) [right of=a5] {2};
	\node [draw,rectangle,minimum size=\bitsize] (a7) [right of=a6] {1};
	\node [draw,rectangle,minimum size=\bitsize] (a8) [right of=a7] {0};

	\node (empty) [below of=a1] {};

	\node [draw,rectangle,minimum size=\bitsize] (b1) [below of=empty] {7};
	\node [draw,rectangle,minimum size=\bitsize] (b2) [right of=b1] {6};
	\node [draw,rectangle,minimum size=\bitsize] (b3) [right of=b2] {5};
	\node [draw,rectangle,minimum size=\bitsize] (b4) [right of=b3] {4};
	\node [draw,rectangle,minimum size=\bitsize] (b5) [right of=b4] {3};
	\node [draw,rectangle,minimum size=\bitsize] (b6) [right of=b5] {2};
	\node [draw,rectangle,minimum size=\bitsize] (b7) [right of=b6] {1};
	\node [draw,rectangle,minimum size=\bitsize] (b8) [right of=b7] {0};
	
	\node [shape=rectangle,draw,minimum size=\bitsize] (d) [left=of b1] {CF};
	\node [shape=rectangle,draw,minimum size=\bitsize] (c) [right=of b8] {0};
	
	\draw [->] (c.west) -- (b8.east);

	\draw [->] (a2.south) -- (b1.north);
	\draw [->] (a3.south) -- (b2.north);
	\draw [->] (a4.south) -- (b3.north);
	\draw [->] (a5.south) -- (b4.north);
	\draw [->] (a6.south) -- (b5.north);
	\draw [->] (a7.south) -- (b6.north);
	\draw [->] (a8.south) -- (b7.north);
	
	\draw [->] (a1.south) -- (d.north);

	\end{tikzpicture}
\end{center}


Die beiden rechts angefügten Bits sind stets Nullen.

Multiplikation mit 4 in ARM:

\begin{lstlisting}[caption=\NonOptimizingKeilVI (\ARMMode),style=customasmARM]
f PROC
        LSL      r0,r0,#2
        BX       lr
        ENDP
\end{lstlisting}

Multiplikation mit 4 in MIPS:

\lstinputlisting[caption=\Optimizing GCC 4.4.5 (IDA),style=customasmMIPS]{patterns/11_arith_optimizations/MIPS_SLL.lst}

\myindex{MIPS!\Instructions!SLL}
\INS{SLL} bedeutet \q{Shift Left Logical}.

\subsubsection{Multiplikation durch Verschieben, Subtrahieren und Addieren}
\label{multiplication_using_shifts_adds_subs}
Es ist auch möglich die Multiplikation zu ersetzen, wenn man mit Zahlen wie 7
oder 17 multipliziert, wenn Verschiebung verwendet wird.
Die zugrundeliegende Mathematik ist relativ einfach.

\myparagraph{32-bit}

\lstinputlisting[style=customc]{patterns/11_arith_optimizations/mult_shifts.c}

\mysubparagraph{x86}

\lstinputlisting[caption=\Optimizing MSVC 2012,style=customasmx86]{patterns/11_arith_optimizations/mult_shifts_MSVC_2012_Ox.asm}

\mysubparagraph{ARM}

Keil im ARM mode benutzt den Umwandler zur Verschiebung im zweiten Operanden:

\lstinputlisting[caption=\OptimizingKeilVI (\ARMMode),style=customasmARM]{patterns/11_arith_optimizations/mult_shifts_Keil_ARM_O3.s}

Da es im Thumb mode keine solchen Umwandler gibt, kann folglich \TT{f2()} nicht
optimiert werden:

\lstinputlisting[caption=\OptimizingKeilVI (\ThumbMode),style=customasmARM]{patterns/11_arith_optimizations/mult_shifts_Keil_thumb_O3.s}

\mysubparagraph{MIPS}

\lstinputlisting[caption=\Optimizing GCC 4.4.5 (IDA),style=customasmMIPS]{patterns/11_arith_optimizations/mult_shifts_MIPS_O3_IDA.lst}

\myparagraph{64-bit}

\lstinputlisting[style=customc]{patterns/11_arith_optimizations/mult_shifts_64.c}

\mysubparagraph{x64}

\lstinputlisting[caption=\Optimizing MSVC 2012,style=customasmx86]{patterns/11_arith_optimizations/mult_shifts_64_GCC49_x64_O3.s}

\mysubparagraph{ARM64}

GCC 4.9 für ARM64 fasst sich dank der Verschiebe-Umwandler ebenfalls kurz:

\lstinputlisting[caption=\Optimizing GCC (Linaro) 4.9 ARM64,style=customasmARM]{patterns/11_arith_optimizations/mult_shifts_64_GCC49_ARM64.s}

\myparagraph{Booths Multiplikationsalgorithmus}

\myindex{Data general Nova}
\myindex{Booth's multiplication algorithm}
Es gab Zeiten, in denen Computer groß und so teuer waren, dass einige von ihnen
keinen Hardwaresupport für die Multiplikation in der \ac{CPU} besaßen, so zum
Beispiel der Data General Nova. 
Wenn dort eine Multiplikation benötigt wurde, musste diese softwareseitig
abgebildet werden, zum Beispiel durch Booths Multiplikationsalgorithmus. 
Dabei handelt es sich um einen Algorithmus zur Multiplikation, welcher lediglich
Addtionen und Verschiebeoperationen verwendet. 

Zwar gehen moderne optimierende Compiler hier anders vor, aber das Ziel (die
Multiplikation) und die Ressourcenfrage (schnellere Operationen) sind gleich.

