\subsection{除算}

\subsubsection{ビットシフトによる除算}
\label{division_by_shifting}

4で除算する例:

\begin{lstlisting}[style=customc]
unsigned int f(unsigned int a)
{
	return a/4;
};
\end{lstlisting}

MSVC 2010の結果

\begin{lstlisting}[caption=MSVC 2010,style=customasmx86]
_a$ = 8		; size = 4
_f	PROC
	mov	eax, DWORD PTR _a$[esp-4]
	shr	eax, 2
	ret	0
_f	ENDP
\end{lstlisting}

\label{SHR}
\myindex{x86!\Instructions!SHR}

この例の \SHR (\emph{SHift Right})命令は、右に2ビット分シフトしています。 
左の2つの解放されたビット(例えば、2つの最上位ビット)はゼロに設定されます。
2つの最下位ビットは破棄されます。 
実際には、これらの2つのドロップビットは除算演算の余りです。
SHR命令はSHLのように動作しますが、他の方向に動作します。

\myindex{x86!\Instructions!SHR}

\SHR 命令は \SHL のように動作しますが、別の方向に動作します。

\begin{center}
	\begin{tikzpicture}[scale=0.7, every node/.style={scale=0.7}]
	\edef\bitsize{1cm}
	\tikzstyle{byte}=[draw,minimum size=\bitsize]	
	\tikzstyle{every path}=[thick]

	\node [draw,rectangle,minimum size=\bitsize] (a1) {7};
	\node [draw,rectangle,minimum size=\bitsize] (a2) [right of=a1] {6};
	\node [draw,rectangle,minimum size=\bitsize] (a3) [right of=a2] {5};
	\node [draw,rectangle,minimum size=\bitsize] (a4) [right of=a3] {4};
	\node [draw,rectangle,minimum size=\bitsize] (a5) [right of=a4] {3};
	\node [draw,rectangle,minimum size=\bitsize] (a6) [right of=a5] {2};
	\node [draw,rectangle,minimum size=\bitsize] (a7) [right of=a6] {1};
	\node [draw,rectangle,minimum size=\bitsize] (a8) [right of=a7] {0};

	\node (empty) [below of=a1] {};

	\node [draw,rectangle,minimum size=\bitsize] (b1) [below of=empty] {7};
	\node [draw,rectangle,minimum size=\bitsize] (b2) [right of=b1] {6};
	\node [draw,rectangle,minimum size=\bitsize] (b3) [right of=b2] {5};
	\node [draw,rectangle,minimum size=\bitsize] (b4) [right of=b3] {4};
	\node [draw,rectangle,minimum size=\bitsize] (b5) [right of=b4] {3};
	\node [draw,rectangle,minimum size=\bitsize] (b6) [right of=b5] {2};
	\node [draw,rectangle,minimum size=\bitsize] (b7) [right of=b6] {1};
	\node [draw,rectangle,minimum size=\bitsize] (b8) [right of=b7] {0};
	
	\node [shape=rectangle,draw,minimum size=\bitsize] (c) [left=of b1] {0};
	\node [shape=rectangle,draw,minimum size=\bitsize] (d) [right=of b8] {CF};
	
	\draw [->] (c.east) -- (b1.west);

	\draw [->] (a1.south) -- (b2.north);
	\draw [->] (a2.south) -- (b3.north);
	\draw [->] (a3.south) -- (b4.north);
	\draw [->] (a4.south) -- (b5.north);
	\draw [->] (a5.south) -- (b6.north);
	\draw [->] (a6.south) -- (b7.north);
	\draw [->] (a7.south) -- (b8.north);
	
	\draw [->] (a8.south) -- (d.north);

	\end{tikzpicture}
\end{center}



10進数の数字で23を想像すると理解しやすいでしょう。
最後の桁(3:除算した結果の余り)を落とすだけで、23を10で簡単に除算することができます。
2は、\gls{quotient}として動作の後に残されます。

残りの部分は削除されますが、それは問題ありません。整数値で作業しますが、
これらは\glslink{real number}{real numbers}ではありません!

ARMでの4の除算

\begin{lstlisting}[caption=\NonOptimizingKeilVI (\ARMMode),style=customasmARM]
f PROC
        LSR      r0,r0,#2
        BX       lr
        ENDP
\end{lstlisting}

MIPSでの4の除算

\begin{lstlisting}[caption=\Optimizing GCC 4.4.5 (IDA),style=customasmMIPS]
        jr      $ra
        srl     $v0, $a0, 2 ; branch delay slot
\end{lstlisting}

\myindex{MIPS!\Instructions!SRL}
SRL命令は\q{Shift Right Logical}の略です。
