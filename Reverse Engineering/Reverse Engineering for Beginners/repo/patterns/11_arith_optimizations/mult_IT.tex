\subsection{Moltiplicazioni}

\subsubsection{Moltiplicare usando addizioni}

Ecco un semplice esempio:

\begin{lstlisting}[style=customc]
unsigned int f(unsigned int a)
{
	return a*8;
};
\end{lstlisting}

La moltiplicazione per 8 è stata rimpiazzata con 3 istruzioni di addizione, che fanno la stessa cosa.
Apparentemente, l' ottimizzatore di MSVC ha deciso che questo codice potrebbe essere più veloce.

\begin{lstlisting}[caption=\Optimizing MSVC 2010,style=customasmx86]
_TEXT	SEGMENT
_a$ = 8		; size = 4
_f	PROC
	mov	eax, DWORD PTR _a$[esp-4]
	add	eax, eax
	add	eax, eax
	add	eax, eax
	ret	0
_f	ENDP
_TEXT	ENDS
END
\end{lstlisting}

\subsubsection{Moltiplicare usando "shift"}
\label{subsec:mult_using_shifts}

Le istruzioni di moltiplicazione e divisione con numeri che sono potenze di 2 sono spesso rimpiazzate con istruzioni "shift".

\begin{lstlisting}[style=customc]
unsigned int f(unsigned int a)
{
	return a*4;
};
\end{lstlisting}

\begin{lstlisting}[caption=\NonOptimizing MSVC 2010,style=customasmx86]
_a$ = 8		; size = 4
_f	PROC
	push	ebp
	mov	ebp, esp
	mov	eax, DWORD PTR _a$[ebp]
	shl	eax, 2
	pop	ebp
	ret	0
_f	ENDP
\end{lstlisting}


Moltiplicare per 4, significa semplicemente traslare il numero a sinistra di 2 bit
e inserire due bit zero a destra (ultimi 2 bit).
Esattamente come moltiplicare 3 per 100~---dobbiamo solo aggiungere due zeri a destra.

Questo è il funzionamento dell' istruzione "shift" a sinistra:

\myindex{x86!\Instructions!SHL}
\begin{center}
	\begin{tikzpicture}[scale=0.7, every node/.style={scale=0.7}]
	\edef\bitsize{1cm}
	\tikzstyle{byte}=[draw,minimum size=\bitsize]	
	\tikzstyle{every path}=[thick]

	\node [draw,rectangle,minimum size=\bitsize] (a1) {7};
	\node [draw,rectangle,minimum size=\bitsize] (a2) [right of=a1] {6};
	\node [draw,rectangle,minimum size=\bitsize] (a3) [right of=a2] {5};
	\node [draw,rectangle,minimum size=\bitsize] (a4) [right of=a3] {4};
	\node [draw,rectangle,minimum size=\bitsize] (a5) [right of=a4] {3};
	\node [draw,rectangle,minimum size=\bitsize] (a6) [right of=a5] {2};
	\node [draw,rectangle,minimum size=\bitsize] (a7) [right of=a6] {1};
	\node [draw,rectangle,minimum size=\bitsize] (a8) [right of=a7] {0};

	\node (empty) [below of=a1] {};

	\node [draw,rectangle,minimum size=\bitsize] (b1) [below of=empty] {7};
	\node [draw,rectangle,minimum size=\bitsize] (b2) [right of=b1] {6};
	\node [draw,rectangle,minimum size=\bitsize] (b3) [right of=b2] {5};
	\node [draw,rectangle,minimum size=\bitsize] (b4) [right of=b3] {4};
	\node [draw,rectangle,minimum size=\bitsize] (b5) [right of=b4] {3};
	\node [draw,rectangle,minimum size=\bitsize] (b6) [right of=b5] {2};
	\node [draw,rectangle,minimum size=\bitsize] (b7) [right of=b6] {1};
	\node [draw,rectangle,minimum size=\bitsize] (b8) [right of=b7] {0};
	
	\node [shape=rectangle,draw,minimum size=\bitsize] (d) [left=of b1] {CF};
	\node [shape=rectangle,draw,minimum size=\bitsize] (c) [right=of b8] {0};
	
	\draw [->] (c.west) -- (b8.east);

	\draw [->] (a2.south) -- (b1.north);
	\draw [->] (a3.south) -- (b2.north);
	\draw [->] (a4.south) -- (b3.north);
	\draw [->] (a5.south) -- (b4.north);
	\draw [->] (a6.south) -- (b5.north);
	\draw [->] (a7.south) -- (b6.north);
	\draw [->] (a8.south) -- (b7.north);
	
	\draw [->] (a1.south) -- (d.north);

	\end{tikzpicture}
\end{center}


I bit aggiunti a destra sono sempre zeri.

Moltiplicazione per 4 in ARM:

\begin{lstlisting}[caption=\NonOptimizingKeilVI (\ARMMode),style=customasmARM]
f PROC
        LSL      r0,r0,#2
        BX       lr
        ENDP
\end{lstlisting}

Moltiplicazione per 4 in MIPS:

\lstinputlisting[caption=\Optimizing GCC 4.4.5 (IDA),style=customasmMIPS]{patterns/11_arith_optimizations/MIPS_SLL.lst}

\myindex{MIPS!\Instructions!SLL}
\INS{SLL} corrisponde a \q{Shift Left Logical}.

\subsubsection{Moltiplicare usando shift, sottrazioni e addizioni}
\label{multiplication_using_shifts_adds_subs}

È anche  possibile eliminare l'operazione di moltiplicazione quando si moltiplica  per numeri come 
7 o 17 utilizzando nuovamente lo shift.
La matematica utilizzata è relativamente semplice.

\myparagraph{32-bit}

\lstinputlisting[style=customc]{patterns/11_arith_optimizations/mult_shifts.c}

\mysubparagraph{x86}

\lstinputlisting[caption=\Optimizing MSVC 2012,style=customasmx86]{patterns/11_arith_optimizations/mult_shifts_MSVC_2012_Ox.asm}

\mysubparagraph{ARM}

Keil in modalità ARM sfrutta i modificatori dello shift del secondo operando:

\lstinputlisting[caption=\OptimizingKeilVI (\ARMMode),style=customasmARM]{patterns/11_arith_optimizations/mult_shifts_Keil_ARM_O3.s}

Tali modificatori non sono presenti in modalità Thumb.
Non si può nemmeno ottimizzare \TT{f2()}:

\lstinputlisting[caption=\OptimizingKeilVI (\ThumbMode),style=customasmARM]{patterns/11_arith_optimizations/mult_shifts_Keil_thumb_O3.s}

\mysubparagraph{MIPS}

\lstinputlisting[caption=\Optimizing GCC 4.4.5 (IDA),style=customasmMIPS]{patterns/11_arith_optimizations/mult_shifts_MIPS_O3_IDA.lst}

\myparagraph{64-bit}

\lstinputlisting[style=customc]{patterns/11_arith_optimizations/mult_shifts_64.c}

\mysubparagraph{x64}

\lstinputlisting[caption=\Optimizing MSVC 2012,style=customasmx86]{patterns/11_arith_optimizations/mult_shifts_64_GCC49_x64_O3.s}

\mysubparagraph{ARM64}

Anche GCC 4.9 per ARM64 è conciso, grazie ai modificatori dello "shift":

\lstinputlisting[caption=\Optimizing GCC (Linaro) 4.9 ARM64,style=customasmARM]{patterns/11_arith_optimizations/mult_shifts_64_GCC49_ARM64.s}

\myparagraph{Algoritmo di moltiplicazione di Booth}

\myindex{Data general Nova}
\myindex{Booth's multiplication algorithm}
Ci fu un tempo in cui i computer erano grossi e costosi, per molti di essi mancava il supporto hardware per l' operazione di moltiplicazione
nella \ac{CPU}, come Data General Nova.
E quando serviva l' operazione di moltiplicazione, veniva fornita a livello software, per esempio, utilizzando l' algoritmo di moltiplicazione di Booth.
Esso è un algoritmi di moltiplicazione che usa solamente addizioni e shift.

Ciò che i moderni compilatori ottimizzati fanno, è diverso,
ma lo scopo (moltiplicare) e le risorse (operazioni veloci) sono le stesse.
