\newcommand{\AcronymsUsed}{Acronimi utilizzati}

\newcommand{\TitleRE}{Reverse Engineering per Principianti}

\newcommand{\TitleUAL}{Capire il linguaggio Assembly}

\newcommand{\AUTHOR}{Dennis Yurichev}

\newcommand{\figname}{fig.\xspace}
\newcommand{\listingname}{listato.\xspace}
\newcommand{\Sourcecode}{Codice sorgente\xspace}
\newcommand{\Seealso}{Vedi anche\xspace}
\newcommand{\tableheader}{\headercolor{} offset & \headercolor{} descrizione}

% instructions descriptions
\newcommand{\ASRdesc}{scorrimento aritmetico a destra}

% x86 registers tables
\newcommand{\RegHeaderTop}{ \multicolumn{8}{ | c | }{ Numero byte } }
% TODO: non-overlapping color!
\newcommand{\RegHeader}{ 7°  & 6°  & 5°  & 4°  & 3°  & 2°  & 1°  & 0 }
\newcommand{\ReturnAddress}{Indirizzo di Ritorno}

\newcommand{\localVariable}{variabile locale}

\newcommand{\savedValueOf}{valore memorizzato di}
% for index
\newcommand{\GrepUsage}{uso di grep}
\newcommand{\SyntacticSugar}{Syntactic Sugar}
\newcommand{\CompilerAnomaly}{Anomalie del compilatore}
\newcommand{\CLanguageElements}{Elementi del linguaggio C}
\newcommand{\CStandardLibrary}{Libreria C standard}
\newcommand{\Instructions}{Istruzioni}
\newcommand{\Pseudoinstructions}{Pseudo-istruzioni}
\newcommand{\Prefixes}{Prefissi}

\newcommand{\Flags}{Flag}
\newcommand{\Registers}{Registri}
\newcommand{\registers}{registri}
\newcommand{\Stack}{Stack}
\newcommand{\Recursion}{Ricorsione}
\newcommand{\RAM}{RAM}
\newcommand{\ROM}{ROM}
\newcommand{\Pointers}{Puntatori}
\newcommand{\BufferOverflow}{Buffer Overflow}

\newcommand{\Conclusion}{Conclusione}

\newcommand{\Exercise}{Esercizio}
\newcommand{\Exercises}{Esercizi\xspace}
\newcommand{\Arrays}{Array}
\newcommand{\Cpp}{C++\xspace}
\newcommand{\CCpp}{C/C++\xspace}
\newcommand{\NonOptimizing}{Senza ottimizzazione\xspace}
\newcommand{\Optimizing}{Con ottimizzazione\xspace}
\newcommand{\ARMMode}{Modalità ARM\xspace}
\newcommand{\ThumbMode}{Modalità Thumb\xspace}
\newcommand{\ThumbTwoMode}{Modalità Thumb-2\xspace}

\newcommand{\DataProcessingInstructionsFootNote}{\ITph{}}

% for .bib files
\newcommand{\AlsoAvailableAs}{\ITph{}}

% section names
\newcommand{\ShiftsSectionName}{\ITph{}}
\newcommand{\HelloWorldSectionName}{Hello, world!}
\newcommand{\SwitchCaseDefaultSectionName}{switch()/case/default}
\newcommand{\PrintfSeveralArgumentsSectionName}{printf() con più argomenti}
\newcommand{\BitfieldsChapter}{Manipolando dei bit speifici}
\newcommand{\ArithOptimizations}{Sostituzione di istruzioni aritmetiche con altre}
\newcommand{\FPUChapterName}{Floating-point unit}
\newcommand{\MoreAboutStrings}{Maggiori informazioni sulle stringhe}
\newcommand{\DivisionByMultSectionName}{Divisione utilizzando la moltiplicazione}
\newcommand{\Answer}{Risposta}
\newcommand{\WhatThisCodeDoes}{Cosa fa qusto codice}
\newcommand{\WorkingWithFloatAsWithStructSubSubSectionName}{Utilizzare i tipi di dato float come una struttura}

\newcommand{\MinesweeperWinXPExampleChapterName}{\ITph{} (Windows XP)}
\newcommand{\StructurePackingSectionName}{Organizzazione dei campi in una struttura}
\newcommand{\StructuresChapterName}{Strutture}
\newcommand{\PICcode}{codice indipendente dalla posizione}
\newcommand{\CapitalPICcode}{Codice indipendente dalla posizione}
\newcommand{\Loops}{Cicli}

% C
\newcommand{\PostIncrement}{Post-incremento}
\newcommand{\PostDecrement}{Post-decremento}
\newcommand{\PreIncrement}{Pre-incremento}
\newcommand{\PreDecrement}{Pre-decremento}

% MIPS
\newcommand{\GlobalPointer}{Puntatore Globale}

\newcommand{\garbage}{garbage}
\newcommand{\IntelSyntax}{Sintassi Intel}
\newcommand{\ATTSyntax}{Sintassi AT\&T}
\newcommand{\randomNoise}{rumore casuale}
\newcommand{\Example}{Esempio}
\newcommand{\argument}{argomento}
\newcommand{\MarkedInIDAAs}{marcato in IDA come}
\newcommand{\stepover}{step over}
\newcommand{\ShortHotKeyCheatsheet}{Elenco delle scorciatoie da tastiera}

\newcommand{\assemblyOutput}{risultato dell'assembly}

% ML prefix is for multi-lingual words and sentences:
\newcommand{\MLHeap}{Heap}
\newcommand{\MLStack}{Stack}
\newcommand{\MLStackOverflow}{Stack overflow}
\newcommand{\MLStartOfHeap}{Inizio dell'heap}
\newcommand{\MLStartOfStack}{Inizio dello stack}
\newcommand{\MLinputA}{input A}
\newcommand{\MLinputB}{input A}
\newcommand{\MLoutput}{output}
\newcommand{\SoftwareCracking}{\ITph{}}

\newcommand{\EMAILPRI}{\href{https://yurichev.com/contact.html}{my emails}}
\newcommand{\EMAILS}{\href{https://yurichev.com/contact.html}{my emails}}

