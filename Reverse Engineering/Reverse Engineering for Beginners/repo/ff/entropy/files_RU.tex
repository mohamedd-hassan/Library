\subsection{Энтропия различных файлов}

Энтропия случайной информации близка к 8:

\begin{lstlisting}
% dd bs=1M count=1 if=/dev/urandom | ent
Entropy = 7.999803 bits per byte.
\end{lstlisting}

Это означает, что почти всё доступное место внутри байта заполнено информацией.

256 байта в пределах 0..255 дает точное значение 8:

\begin{lstlisting}[style=custompy]
#!/usr/bin/env python
import sys

for i in range(256):
    sys.stdout.write(chr(i))
\end{lstlisting}

\begin{lstlisting}
% python 1.py | ent
Entropy = 8.000000 bits per byte.
\end{lstlisting}

Порядок не важен.
Это означает, что всё доступное место внутри байта заполнено.

Энтропия любого блока заполненного нулевыми байтами это 0:

\begin{lstlisting}
% dd bs=1M count=1 if=/dev/zero | ent
Entropy = 0.000000 bits per byte.
\end{lstlisting}

Энтропия строки, состоящей из одного (любого) байта это 0:

\begin{lstlisting}
% echo -n "aaaaaaaaaaaaaaaaaaa" | ent
Entropy = 0.000000 bits per byte.
\end{lstlisting}

\myindex{base64}
Энтропия base64-строки такая же, как и энтропия исходных данных, но умножена на $\frac{3}{4}$.
Это потому что кодирование в base64 использует 64 символа вместо 256.

\begin{lstlisting}
% dd bs=1M count=1 if=/dev/urandom | base64 | ent
Entropy = 6.022068 bits per byte.
\end{lstlisting}

Вероятно, 6.02 чуть больше 6 из-за того, что выравнивающий символ (\TT{=}) немного портит статистику.

\myindex{Uuencode}
Uuencode также использует 64 символа:

\begin{lstlisting}
% dd bs=1M count=1 if=/dev/urandom | uuencode - | ent
Entropy = 6.013162 bits per byte.
\end{lstlisting}

Это означает, что любая строка в base64 и Uuencode может быть передана используя 6-битные байты или символы.

Любая случайная информация в шестнадцатеричном виде имеет энтропию в 4 бита на байт:

\begin{lstlisting}
% openssl rand -hex $\$$(( 2**16 )) | ent
Entropy = 4.000013 bits per byte.
\end{lstlisting}

Энтропия случайно выбранного текста на английском из библиотеки Гутенберга имеет энтропию $\approx 4.5$.
Это потому что английские тексты используют, в основном, 26 латинских символов, и $log_2(26)=\approx 4.7$, т.е., вам
нужны 5-битные байты для передачи несжатых английских текстов, это будет достаточно (так это и было в эпоху телетайпов).

Случайно выбранный текст на русском из библиотеки \url{http://lib.ru}
это Ф.М.Достоевский ``Идиот''\footnote{\url{http://az.lib.ru/d/dostoewskij_f_m/text_0070.shtml}},
закодированный в CP1251.

И этот файл имеет энтропию в $\approx 4.98$.
В кириллице 33 буквы, и $log_2(33)=\approx 5.04$.
Но в русской письменности есть малопопулярная и редкая буква ``ё''.
% FIXME YO letter isn't rendered in Eng version
И $log_2(32)=5$ (кирилличный алфавит без этой редкой буквы) --- теперь это близко к тому, что мы получили.

Впрочем, этот текст использует букву ``ё'', но, наверное, и там она встречается не часто.

\myindex{UTF-8}
Тот же файл перекодированный из CP1251 в UTF-8 дает энтропию в $\approx 4.23$.
Каждый символ из кириллицы кодируется в UTF-8 при помощи пары, и первый байт всегда один из этих двух: 0xD0 or 0xD1.
Видимо, это причина перекоса.

Будем генерировать случайные биты и выводить их как символы ``T'' и ``F'':

\begin{lstlisting}[style=custompy]
#!/usr/bin/env python
import random, sys

rt=""
for i in range(102400):
    if random.randint(0,1)==1:
        rt=rt+"T"
    else:
        rt=rt+"F"
print rt
\end{lstlisting}

Пример: \TT{...TTTFTFTTTFFFTTTFTTTTTTFTTFFTTTFTFTTFTTFFFFFF...}.

Энтропия очень близка к 1 (т.е., 1 бит на байт).

Будем генерировать случайные десятичные цифры:

\begin{lstlisting}[style=custompy]
#!/usr/bin/env python
import random, sys

rt=""
for i in range(102400):
    rt=rt+"%d" % random.randint(0,9)
print rt
\end{lstlisting}

Например: \TT{...52203466119390328807552582367031963888032...}.

Энтропия близка к 3.32, действительно, это $log_2(10)$.

